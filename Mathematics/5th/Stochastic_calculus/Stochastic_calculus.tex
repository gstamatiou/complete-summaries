\documentclass[../../../main_math.tex]{subfiles}

\begin{document}
\changecolor{SC}
\begin{multicols}{2}[\section{Stochastic calculus}]
  Along the document we assume that we work in a probability space $(\Omega,\mathcal{F},\Prob)$ and that all the random variables are defined on this space.
  \subsection{Preliminaries}
  \subsubsection{Stochastic processes}
  \begin{proposition}
    A stochastic process $X={(X_t)}_{t\in \TT}$ is Gaussian if and only if $\forall n\in\NN$, $\forall t_1,\ldots,t_n\in\TT$, $\forall \lambda_1,\ldots,\lambda_n\in\RR$,
    $$
      Z:=\lambda_1 X_{t_1}+\cdots+\lambda_n X_{t_n}
    $$
    is a Gaussian random variable. In particular, we have:
    $$
      \Exp(\exp{\ii Z})=\exp{\ii \Exp(Z)-\frac{1}{2}\Var(Z)}
    $$
  \end{proposition}
  \begin{remark}
    A stochastic process $X={(X_t)}_{t\in \TT}$ can also be viewed as a single random variable taking values in $\RR^{\TT}$, equipped with the product $\sigma$-algebra $\displaystyle \bigotimes_{t\in\TT}\mathcal{B}(\RR)$.
  \end{remark}
  \begin{proposition}
    Let $m:\TT\to\RR$ be a measurable function and $\gamma:\TT^2\to\RR$ be a symmetric positive-definite function. Then, there exists a Gaussian process ${(X_t)}_{t\in \TT}$ such that $\Exp(X_t)=m(t)$ and $\cov(X_s,X_t)=\gamma(s,t)$.
  \end{proposition}
  \begin{definition}
    Let ${(X_t)}_{t\in \TT}$, ${(Y_s)}_{s\in \SS}$ be two stochastic processes. We say that they are \emph{jointly Gaussian} if the concatenated process $({(X_t)}_{t\in \TT},{(Y_s)}_{s\in \SS})$ is Gaussian.
  \end{definition}
  \begin{lemma}\label{SC:indep_joint_gauss}
    Two jointly Gaussian stochastic processes ${(X_t)}_{t\in \TT}$, ${(Y_s)}_{s\in \SS}$ are independent if and only if $\forall t\in\TT$, $\forall s\in\SS$, $\cov(X_t,Y_s)=0$.
  \end{lemma}
  \begin{proposition}
    Two stochastic processes ${(X_t)}_{t\in \TT}$, ${(Y_s)}_{s\in \SS}$ are independent if and only if $\forall n\in\NN$, $\forall t_1,\ldots,t_n\in\TT$, $\forall s_1,\ldots,s_n\in\SS$ and $\forall f,g:\RR^n\to\RR$ bounded and measurable functions, we have:
    \begin{multline*}
      \Exp(f(X_{t_1},\ldots,X_{t_n})g(Y_{s_1},\ldots,Y_{s_n}))=\\=
      \Exp(f(X_{t_1},\ldots,X_{t_n}))\Exp(g(Y_{s_1},\ldots,Y_{s_n}))
    \end{multline*}
  \end{proposition}
  \subsubsection{Brownian motion}
  \begin{definition}
    A \emph{Brownian motion} is a stochastic process ${(B_t)}_{t\geq 0}$ such that:
    \begin{enumerate}
      \item $B$ is Gaussian with $\Exp(B_t)=0$ and $\cov(B_s,B_t)=s\wedge t$.
      \item $B$ has continuous paths.
    \end{enumerate}
  \end{definition}
  \begin{proposition}
    Let $B$ be a Brownian motion. Then:
    \begin{enumerate}
      \item $B_0=0$ a.s.
      \item $B$ has independent increments.
      \item $B$ has stationary increments.
    \end{enumerate}
    Conversely, any stochastic process with these properties has the law of a Brownian motion.
  \end{proposition}
  \begin{theorem}[Strong law of large numbers for Brownian motion]
    Let ${(B_t)}_{t\geq 0}$ be a Brownian motion. Then:
    $$
      \frac{B_t}{t}\underset{t\to\infty}{\overset{\text{a.s.}}{\longrightarrow}}0
    $$
  \end{theorem}
  \begin{proof}
    We already now that the process $s\to sB_{1/s} \indi{s>0}$ is a Brownian motion. In particular, we must have continuity at $0=B_0$.
  \end{proof}
  \begin{theorem}[Markov property for Brownian motion]
    Let $B={(B_t)}_{t\geq 0}$ be a Brownian motion and $a\geq 0$ fixed. Then, the Brownian motion ${(B_{t+a}-B_a)}_{t\geq 0}$ is independent of ${(B_s)}_{s\in [0,a]}$.
  \end{theorem}
  \begin{proof}
    The processes ${(B_s)}_{s\in[0,a]}$ and ${(B_{t+a}-B_a)}_{t\geq 0}$ are jointly Gaussian, because their coordinates are linear combinations of coordinates of the same Gaussian process $B$. Thus, by \mcref{SC:indep_joint_gauss} it reduces to compute the following correlation:
    $$
      \cov(B_s,B_{t+a}-B_a)=s\wedge(t+a)-s\wedge a=0
    $$
  \end{proof}
  \begin{remark}
    Recall that $s\wedge t:=\min(s,t)$ and $s\vee t:=\max(s,t)$.
  \end{remark}
  \subsubsection{Martingales}
  \begin{definition}
    Let ${(X_t)}_{t\geq 0}$ be a stochastic process. We define the \emph{natural filtration} of $X$ as $\mathcal{F}^X:={(\mathcal{F}_t^X)}_{t\geq 0}$, where $\mathcal{F}_t^X:=\sigma(X_s:s\leq t)$.
  \end{definition}
  From now on, we will assume that we work in a filtered probability space $(\Omega,\mathcal{F},\Prob,{(\mathcal{F}_t)}_{t\geq 0})$.
  \begin{definition}[Martingale]
    A stochastic process ${(X_t)}_{t\geq 0}$ is a \emph{martingale} if:
    \begin{enumerate}
      \item it is \emph{adapted}, i.e.\ $X_t$ is $\mathcal{F}_t$-measurable for all $t\geq 0$.
      \item $\Exp(\abs{X_t})<\infty$ for all $t\geq 0$.
      \item $\Exp(X_t\mid \mathcal{F}_s)=X_s$ for all $0\leq s\leq t$.
    \end{enumerate}
  \end{definition}
  \begin{proposition}
    Let $B={(B_t)}_{t\geq 0}$ be a Brownian motion. Then, the following processes are martingales ${(M_t)}_{t\geq 0}$ with respect to the natural filtration induced by $B$:
    \begin{itemize}
      \item $M_t=B_t$
      \item $M_t=B_t^2-t$
      \item $M_t=\exp{\theta B_t-\frac{1}{2}\theta^2t}$, for any fixed $\theta\in\RR$.
    \end{itemize}
  \end{proposition}
  \begin{proposition}
    Let $A\subseteq \RR$ be a closed set and $X={(X_t)}_{t\geq 0}$ be an adapted continuous process. Then, the \emph{hitting time} of $A$ by $X$, defined as:
    $$
      T_A:=\inf\{t\geq 0:X_t\in A\}
    $$
    is a stopping time.
  \end{proposition}
  \begin{proof}
    Using the continuity of $X$ and the fact that $A$ is closed, one can easily check that:
    $$
      \{T_A \leq t\}=\bigcap_{k\in\NN}\bigcup_{s\in[0,t]\cap\QQ}\left\{d(X_s,A)\leq\frac{1}{k}\right\}
    $$
    Now, $\left\{d(X_s,A)\leq\frac{1}{k}\right\}\in \mathcal{F}_s\subseteq \mathcal{F}_t$ because $X$ is adapted and $z\mapsto d(z,A)$ is measurable.
    Thus, $\{T_A \leq t\}\in \mathcal{F}_t$ because it is a countable union and intersection of events in $\mathcal{F}_t$.
  \end{proof}
  \begin{theorem}[Doob's optional sampling theorem]\label{SC:doob_sampling}
    Let ${(M_t)}_{t\geq 0}$ be a continuous martingale and $T$ be a stopping time. Then, the \emph{stopped process} $M^T:={(M_{t\wedge T})}_{t\geq 0}$ is a continuous martingale. In particular, $\forall t\geq 0$, $\Exp(M_{t\wedge T})=\Exp(M_0)$. If $M^T$ is uniformly integrable and $T\overset{\text{a.s.}}{\leq}\infty$, then taking $t\to\infty$ we have $\Exp(M_T)=\Exp(M_0)$.
  \end{theorem}
  \begin{lemma}[Orthogonality of martingales]\label{SC:orthogonality_martingales}
    Let ${(M_t)}_{t\geq 0}$ be a continuous martingale and let $0\leq s\leq t$. Then:
    $$
      \Exp({(M_t-M_s)}^2\mid \mathcal{F}_s)=\Exp({M_t}^2-{M_s}^2\mid \mathcal{F}_s)
    $$
  \end{lemma}
  \begin{proof}
    We have that:
    \begin{multline*}
      \Exp({(M_t-M_s)}^2\mid \mathcal{F}_s) =\Exp({M_t}^2-2M_tM_s+{M_s}^2\mid \mathcal{F}_s) =\\=\Exp({M_t}^2+{M_s}^2\!\mid \!\mathcal{F}_s)-2M_s\Exp(M_t\!\mid\! \mathcal{F}_s) =\Exp({M_t}^2-{M_s}^2\!\mid\! \mathcal{F}_s)
    \end{multline*}
  \end{proof}
  \begin{theorem}[Doob's maximal inequality]\label{SC:doob_maximal}
    If $M$ is a continuous square-integrable martingale, then $\forall a,t\geq 0$ we have:
    $$
      \Prob\left(\sup_{0\leq s\leq t}\abs{M_s}\geq a\right)\leq \frac{\Exp({M_t}^2)}{a^2}
    $$
  \end{theorem}
  \begin{proposition}\label{SC:limit_of_martingales}
    Let $(M^n)$ be a sequence of continuous square-integrable martingales and suppose that for each $t\geq 0$, the limit $\displaystyle M_t:=\overset{L^2}{=}\lim_{n\to\infty}M_t^n$ exists. Then, $M={(M_t)}_{t\geq 0}$ is a continuous square-integrable martingale.
  \end{proposition}
  \begin{proof}
    By \mnameref{SC:doob_maximal} applied to $M^n-M^m$ we have that for fixed $t\geq 0$ and $k\in\NN$:
    $$
      \Prob\left(\sup_{0\leq s\leq t}\abs{M_s^n-M_s^m}\geq \frac{1}{kmn }\right)\leq k^2\Exp({(M_t^n-M_t^m)}^2)\leq \frac{1}{k^2}
    $$
    where in the last inequality we have used that ${(M^n)}$ converges in $L^2$ and so we have chosen $n,m$ large enough so that the inequality holds. Thus, there is an increasing sequence $(n_k)$ such that:
    $$
      \Prob\left(\sup_{0\leq s\leq t}\abs{M_s^{n_{k+1}}-M_s^{n_k}}\geq \frac{1}{k}\right)\leq \frac{1}{k^2}
    $$
    By \mnameref{P:borel-cantelli1}, we deduce that
    $$
      \sum_{k=1}^{\infty}\sup_{0\leq s\leq t}\abs{M_s^{n_{k+1}}-M_s^{n_k}}<\infty
    $$
    which ensures that $(M^{n_k})$ is continuous in the space of continuous functions equipped with the topology of uniform convergence on every compact set. But the limit is necessarily a version of $M$, because for each $t\geq 0$ we have $M_t^n\to M_t$ in $L^2$.
  \end{proof}
  \subsubsection{Quadratic variation}
  \begin{definition}
    Let $f:\RR_{\geq 0}\to\RR$ be a function. We define the \emph{absoulte variation} of $f$ on the interval $[s,t]$ as:
    $$
      V(f,s,t):=\sup_{{(t_k)}_{0\leq k\leq n}\in \mathrm{P}([s,t])}\sum_{k=1}^{n}\abs{f(t_{k+1})-f(t_k)}
    $$
    where $\mathrm{P}([s,t])$ is the set of all partitions of $[s,t]$. A function has \emph{finite variation} if $V(f,s,t)<\infty$ for all $0\leq s\leq t$.
  \end{definition}
  \begin{lemma}\label{SC:properties_variation}
    Let $f,g:\RR_{\geq 0}\to\RR$ be a function and $0\leq s\leq t$. Then:
    \begin{itemize}
      \item $V(f,s,t)=V(f,s,u)+V(f,u,t)$, for all $s\leq u\leq t$.
      \item If $f\in C^1$, then $V(f,s,t)=\int_s^t\abs{f'(u)}\dd{u}$.
      \item If $f$ is monotone, then $V(f,s,t)=\abs{f(t)-f(s)}$.
      \item $\displaystyle V(f+g,s,t)\leq V(f,s,t)+V(g,s,t)$.
      \item Finite variation functions form a vector space.
    \end{itemize}
  \end{lemma}
  \begin{proposition}\label{SC:difference_of_increasing}
    Let $f:\RR_{\geq 0}\to\RR$. Then, $f$ has finite variation if and only if it can be written as the difference of two non-decreasing functions.
  \end{proposition}
  \begin{sproof}
    \mcref{SC:properties_variation} gives us the implication to the left. For the other one, note that the functions $f_1(t):=V(f,0,t)$ and $f_2(t):=V(f,0,t)-f(t)$ are non-decreasing.
  \end{sproof}
  \begin{theorem}[Quadratic variation]
    Let $M={(M_t)}_{t\geq 0}$ be a continuous square-integrable martingale. Then, for each $t\geq 0$ the limit
    $$
      {\langle M\rangle}_t:=\lim_{n\to\infty}\sum_{k=1}^{n}\abs{M_{t_k^n}-M_{t_{k-1}^n}}^2
    $$
    exists in $L^1$ and does not depend on the partition ${(t_k^n)}_{0\leq k\leq n}\in \mathrm{P}([0,t])$ chosen as long as the \emph{mesh} $\Delta_n:= \max_{1\leq k\leq n}(t_k^n-t_{k-1}^n)$ goes to $0$ as $n\to\infty$. Moreover, $\langle M\rangle=({\langle M\rangle}_t)_{t\geq 0}$ has the following properties:
    \begin{enumerate}
      \item\label{SC:quad_var1} ${\langle M\rangle}_0=0$
      \item\label{SC:quad_var2} $\langle M\rangle$ is non-decreasing.
      \item\label{SC:quad_var3} The function $t\mapsto {\langle M\rangle}_t$ is continuous.
      \item\label{SC:quad_var4} ${({M_t}^2-{\langle M\rangle}_t)}_{t\geq 0}$ is a martingale.
    \end{enumerate}
  \end{theorem}
  \begin{proof}
    We omit the proof of the existence and continuity. We will only prove the last property. Let $0\leq s\leq t$ and ${(t_k^n)}_{0\leq k\leq n}\in \mathrm{P}([s,t])$ be such that $\Delta_n\to 0$. Then:
    \begin{align*}
      \Exp({M_t}^2-{M_s}^2\mid \mathcal{F}_s) & =\sum_{k=1}^n \Exp({M_{t_k^n}}^2-{M_{t_{k-1}^n}}^2\mid \mathcal{F}_s)          \\
                                              & =\sum_{k=1}^n \Exp\left({(M_{t_k}^n-M_{t_{k-1}}^n)}^2\mid \mathcal{F}_s\right) \\
    \end{align*}
    by the \mnameref{SC:orthogonality_martingales}. Now since we have convergence of $\sum_{k=1}^{n}{(M_{t_k}^n-M_{t_{k-1}}^n)}^2$ to ${\langle M\rangle}_t-{\langle M\rangle}_s$ in $L^1$, we get the result:
    $$
      \Exp({M_t}^2-{M_s}^2\mid \mathcal{F}_s)=\Exp({\langle M\rangle}_t-{\langle M\rangle}_s\mid \mathcal{F}_s)
    $$
  \end{proof}
  \begin{proposition}
    Let $B$ be a Brownian motion. Then:
    $$
      \Prob(\forall s,t\geq 0, V(B,s,t)=\infty)=1
    $$
    But, ${\langle B\rangle}_t=t$ for all $t\geq 0$.
  \end{proposition}
  \begin{proof}
    Let $B={(B_t)}_{t\geq 0}$. Then:
    $$
      V(B,s,t)\!\geq\! \sum_{k=1}^{n}\abs{B_{s+k\frac{t-s}{n}}-B_{s+(k-1)\frac{t-s}{n}}}\!=\!\sqrt{\frac{t-s}{n}}\sum_{k=1}^{n}\abs{\xi_k}
    $$
    where $\xi_k$ are \iid $N(0,1)$. By the \mnameref{P:weaklaw} we get the result. The second part is similar, but we get convergence instead.
  \end{proof}
  \begin{proposition}\label{SC:prop_variation_fg}
    If a function $f$ has finite variation and $g$ is continuous, then:
    $$
      \sum_{k=1}^n(f(t_k)-f(t_{k-1}))(g(t_k)-g(t_{k-1}))\overset{n\to\infty}{\longrightarrow}0
    $$
  \end{proposition}
  \begin{proof}
    Note that:
    \begin{multline*}
      \abs{\sum_{k=1}^n(f(t_k)-f(t_{k-1}))(g(t_k)-g(t_{k-1}))}\leq\\\leq V(f,0,t)\max_{\substack{0\leq u\leq v\leq t\\\abs{u-v}\leq \Delta_n}}\abs{g(u)-g(v)}
    \end{multline*}
    which goes to zero by uniform continuity of $g$ at $[0,t]$.
  \end{proof}
  \begin{corollary}\label{SC:corollary_finite_variation}
    Let $M={(M_t)}_{t\geq 0}$ be a continuous square-integrable martingale with finite variation a.s. Fix $t\geq 0$. By \mcref{SC:prop_variation_fg} we have that ${\langle M\rangle}_t=0$
    Then:
    $$
      \Prob(\forall t\geq 0,\ M_t=M_0)=1
    $$
  \end{corollary}
  \begin{proof}
    By \mnameref{SC:orthogonality_martingales}, we have:
    $$
      \Exp({(M_t-M_0)}^2)=\Exp({M_t}^2) - \Exp({M_0}^2) = \Exp({\langle M\rangle}_t)=0
    $$
    where the penultimate equality follows from the fact that ${M_t}^2-{\langle M\rangle}_t$ is a martingale and so it has constant expectation. This shows that $\Prob (\forall t\geq 0,\ M_t=M_0)=1$. Now we can use the fact that $M$ is continuous to conclude using $t\in\QQ$.
  \end{proof}
  \begin{proposition}
    The quadratic variation is the unique process that satisfies \mcref{SC:quad_var1,SC:quad_var2,SC:quad_var3,SC:quad_var4}.
  \end{proposition}
  \begin{proof}
    Let $A$ be another process satisfying such properties. Then, ${M}^2-{\langle M\rangle}$ and ${M}^2-A$ are both martingales. Thus, $A-{\langle M\rangle}$ is also a martingale. But it is also continuous and has finite variation (by \mcref{SC:difference_of_increasing}). So by \mcref{SC:corollary_finite_variation}, $A={\langle M\rangle}$.
  \end{proof}
  \subsubsection{Local martingales}
  \begin{definition}
    A stochastic process ${(M_t)}_{t\geq 0}$ is a \emph{continuous local martingale} if there exists a sequence of stopping times ${(T_n)}_{n\in\NN}$ (called \emph{localizing sequence}) such that:
    \begin{enumerate}
      \item $T_n\nearrow \infty$ a.s.
      \item $M^{T_n}:={(M_{t\wedge T_n})}_{t\geq 0}$ is a martingale for all $n\in\NN$.
    \end{enumerate}
  \end{definition}
  \begin{remark}
    If $M$ is a martingale, then $M$ is a local martingale by taking $T_n=+\infty$ for all $n\in\NN$.
  \end{remark}
  \begin{remark}
    Any local martingale is adapted because it is the pointwise limit of $M^{T_n}$, which are adapted by definition.
  \end{remark}
  \begin{proposition}
    Let $M={(M_t)}_{t\geq 0}$ be a continuous local martingale. Then, if $\forall t\geq 0$ we have
    $$
      \Exp\left(\sup_{0\leq s\leq t}\abs{M_s}\right)<\infty
    $$
    then $M$ is a martingale.
  \end{proposition}
  \begin{proof}
    We've argued that local martingales are automatically adapted. Moreover:
    $$
      \Exp(\abs{M_t})\leq \Exp\left(\sup_{0\leq s\leq t}\abs{M_s}\right)<\infty
    $$
    Finally, fix $0\leq s\leq t$. For all $n\in\NN$ we have:
    $$
      \Exp(M_{t\wedge T_n}\mid \mathcal{F}_s)=M_{s\wedge T_n}
    $$
    And using the \mnameref{P:dominated} with $M_{t\wedge T_n}\leq \sup_{0\leq s\leq t}\abs{M_s}$ we conclude the result.
  \end{proof}
  \begin{remark}
    Note that if $M$ is a continuous local martingale with $M_0=0$, then we can always take $T_n=\inf\{t\geq 0:\abs{M_t}\geq n\}$ as a localizing sequence.
  \end{remark}
  \begin{theorem}[Doob's optional sampling theorem for local martingales]
    Let $M={(M_t)}_{t\geq 0}$ be a continuous local martingale and $T$ be a stopping time. Then, the stopped process $M^T:={(M_{t\wedge T})}_{t\geq 0}$ is a continuous local martingale.
  \end{theorem}
  \begin{proof}
    Let ${(T_n)}_{n\in\NN}$ be a localizing sequence for $M$. Since $M^{T_n}$ is a continuous martingale, by \mnameref{SC:doob_sampling} we have that $M^{T_n\wedge T}$ is a continuous martingale. Thus, $M^T$ is a local martingale with localizing sequence ${(T_n)}_{n\in\NN}$.
  \end{proof}
  \begin{proposition}
    Continuous local martingales form a vector space.
  \end{proposition}
  \begin{proof}
    Let $M$, $\tilde{M}$ be continuous local martingales with localizing sequences ${(T_n)}_{n\in\NN}$ and ${(\tilde{T}_n)}_{n\in\NN}$ respectively and $\lambda,\tilde{\lambda}\in\RR$. Then, ${(T_n\wedge \tilde{T}_n)}_{n\in\NN}$ is a localizing sequence for both $M$ and $\tilde{M}$ and so $\lambda M^{T_n\wedge \tilde{T}_n}+\tilde{\lambda}\tilde{M}^{T_n\wedge \tilde{T}_n}$ is a martingale.
  \end{proof}
  \begin{proposition}
    If $M$ is a continuous local martingale which has finite variation a.s., then:
    $$
      \Prob(\forall t\geq 0,\ M_t=M_0)=1
    $$
  \end{proposition}
  \begin{proof}
    Let ${(T_n)}_{n\in\NN}$ be a localizing sequence for $M$. Then, $M^{T_n}$ is a martingale and $V(M^{T_n},0,t)=V(M,0,t\wedge T_n)<\infty$. Thus, by \mcref{SC:corollary_finite_variation} we have that $M^{T_n}_t=M^{T_n}_0$ $\forall t\geq 0$ and $n\in\NN$. Taking $n\to\infty$ we get the result.
  \end{proof}
  \begin{proposition}
    Let $M$ be a continuous local martingale. Then, the limit
    $$
      {\langle M\rangle}_t:=\lim_{n\to\infty}\sum_{k=1}^{n}\abs{M_{t_k^n}-M_{t_{k-1}^n}}^2
    $$
    exists in probability for any $t\geq 0$ and does not depend on the partition ${(t_k^n)}_{0\leq k\leq n}\in \mathrm{P}([0,t])$ chosen as long as $\Delta_n\to 0$. Moreover, $\langle M\rangle=({\langle M\rangle}_t)_{t\geq 0}$ is the unique process (up to modification) such that:
    \begin{enumerate}
      \item ${\langle M\rangle}_0=0$
      \item $t\mapsto {\langle M\rangle}_t$ is a.s.\ continuous.
      \item $\langle M\rangle$ is a.s.\ non-decreasing.
      \item ${({M_t}^2-{\langle M\rangle}_t)}_{t\geq 0}$ is a continuous local martingale.
    \end{enumerate}
  \end{proposition}
  \begin{theorem}[Levy's characterization of Brownian motion]
    Let $M={(M_t)}_{t\geq 0}$ be a stochastic process. Then, the following are equivalent:
    \begin{enumerate}
      \item $M$ is a continuous local square-integrable martingale with $M_0=0$ and ${\langle M\rangle}_t=t$.
      \item $M$ is a ${(\mathcal{F}_t)}_{t\geq 0}$-Brownian motion.
    \end{enumerate}
  \end{theorem}
  \subsection{Stochastic integration}
  \subsubsection{Wiener isometry}
  \begin{definition}
    Let $H$, $H'$ be Hilbert. A map $I:H\to H'$ is called \emph{isometry} if it is linear and $\forall x\in H$ we have:
    $$
      \norm{I(x)}_{H'}=\norm{x}_H
    $$
    We speak of \emph{partial isometry} when $I$ is only defined on a subspace of $H$.
  \end{definition}
  \begin{theorem}
    Let $H$, $H'$ be Hilbert, $V\subseteq H$ be a dense subspace and $I:V\to H'$ be a partial isometry. Then, there exists a unique continuous isometry extension of $I$ to $H$.
  \end{theorem}
  \begin{proof}
    Let $x\in H\setminus V$. Then, $\exists(x_n)\in V$ such that $x_n\to x$. Clearly, any continuous extension must satisfy $I(x):=\lim_{n\to\infty}I(x_n)$, so we take it as a definition. Note that, first, the limit exists because $(I(x_n))$ is Cauchy and moreover this definition does not depend on the sequence $(x_n)$. From this definition, the extension is automatically linear and norm-preserving (because of the continuity).
  \end{proof}
  \begin{definition}
    Let ${(B_t)}_{t\geq 0}$ be a Brownian motion and $f\in\mathcal{S}(\RR_{\geq 0})$ be a simple function such that $f=\sum_{k=1}^na_k\indi{(t_{k-1},t_k]}$ with $0=t_0\leq t_1\leq \cdots\leq t_n$. We define the \emph{Wiener integral} of $f$ as:
    $$
      I(f):=\sum_{k=1}^na_k(B_{t_k}-B_{t_{k-1}})
    $$
  \end{definition}
  \begin{remark}
    Recall that simple functions are dense in $L^p$ (\mcref{RFA:continuousdenseLp}).
  \end{remark}
  \begin{theorem}
    Let ${(B_t)}_{t\geq 0}$ be a Brownian motion on $(\Omega, \mathcal{F}, \Prob)$. Then, there exists a unique linear and continuous map $I:L^2(\RR_{\geq 0})\to L^2((\Omega, \mathcal{F}, \Prob))$ such that for all $0\leq s\leq t$:
    $$
      I(\indi{(s,t]})=B_t-B_s
    $$
    Moreover, $I$ is an isometry. The map $I$ is called \emph{Wiener isometry} (or \emph{Wiener integral}) and denoted by $I(f)=\int_0^\infty f(u)\dd{B_u}$.
  \end{theorem}
  \begin{remark}
    Recall that the limit of Gaussian variables is Gaussian.
  \end{remark}
  \begin{proposition}
    Let ${(B_t)}_{t\geq 0}$ be a Brownian motion. Then, the following are satisfied:
    \begin{itemize}
      \item For any $f\in L^2(\RR_{\geq 0})$ we have:
            $$
              \int_0^\infty f(u)\dd{B_u}\overset{L^2}{=}\lim_{n\to\infty}\sum_{k=1}^{n^2}a_{n,k}(f)(B_{\frac{k+1}{n}}-B_{\frac{k}{n}})
            $$
            where $a_{n,k}(f):=n\int_{\frac{k}{n}}^{\frac{k+1}{n}}f(u)\dd{u}$ is an approximation of $f$ in the interval $[\frac{k}{n},\frac{k+1}{n}]$.
      \item The Wiener integral is a Gaussian variable with zero mean and variance $\int_0^\infty{f(u)}^2\dd{u}$.
      \item For any $f,g\in L^2(\RR_{\geq 0})$ we have:
            $$
              \cov\left(\int_0^\infty f(u)\dd{B_u},\int_0^\infty g(u)\dd{B_u}\right)=\int_0^\infty f(u)g(u)\dd{u}
            $$
    \end{itemize}
  \end{proposition}
  \subsubsection{The Wiener integral as a process}
  \begin{definition}
    Let $f\in L_{\text{loc}}^2(\RR_{\geq 0})$ and $0\leq s\leq t$. We define the \emph{Wiener integral} of $f$ as:
    $$
      \int_s^t f(u)\dd{B_u}:=\int_0^\infty f(u)\indi{(s,t]}(u)\dd{B_u}
    $$
  \end{definition}
  \begin{lemma}[Chasles relation]
    Let $f\in L_{\text{loc}}^2(\RR_{\geq 0})$ and $0\leq r\leq s\leq t$. Then:
    $$
      \int_r^t f(u)\dd{B_u}=\int_r^s f(u)\dd{B_u}+\int_s^t f(u)\dd{B_u}
    $$
  \end{lemma}
  \begin{proposition}
    Let ${(B_t)}_{t\geq 0}$ be a Brownian motion and $f\in L_{\text{loc}}^2(\RR_{\geq 0})$. Then, the associate process $M^f={(M_t^f)}_{t\geq 0}$ defined as:
    $$
      M_t^f:=\int_0^t f(u)\dd{B_u}
    $$
    is a centered Gaussian process with covariance function:
    $$
      \cov(M_s^f,M_t^f)=\int_0^{s\wedge t}{f(u)}^2\dd{u}
    $$
  \end{proposition}
  \begin{proof}
    We'll only proof that $M^f$ is Gaussian (the computation of the mean and covariance functions is easy). Let $n\in\NN$, $(t_1,\ldots,t_n)\in\RR^n$ and $(\lambda_1,\ldots,\lambda_n)\in\RR^n$. Then:
    $$
      \sum_{k=1}^n\lambda_k M_{t_k}^f=\int_0^\infty g(u)\dd{B_u}
    $$
    with $g(u)=\sum_{k=1}^n\lambda_k f(u)\indi{(0,t_k]}(u)\in L^2(\RR_{\geq 0})$, and the right-hand side is Gaussian because it is a Wiener integral.
  \end{proof}
  \begin{theorem}
    Let $f\in L_{\text{loc}}^2(\RR_{\geq 0})$. Then, $M^f$ is a continuous square-integrable martingale with:
    $$
      {\langle M^f\rangle}_t=\int_0^t{f(u)}^2\dd{u}
    $$
  \end{theorem}
  \begin{proof}
    The integrability and square-integrability is clear because $M^f$ is Gaussian. Note that $t\mapsto M_t^f$ is continuous when $f=\indi{(0,a]}$, because the Brownian motion is continuous. Now using \mcref{SC:limit_of_martingales} we get the result true for any $f\in L_{\text{loc}}^2(\RR_{\geq 0})$. Now let's prove that $M^f$ is a martingale. We have:
    \begin{align*}
      M_t^f & =\lim_{n\to\infty}\sum_{k=1}^{n^2}a_{n,k}(f\indi{(0,t]})(B_{\frac{k+1}{n}}-B_{\frac{k}{n}})                 \\
            & =\lim_{n\to\infty}\sum_{k=1}^{n^2}a_{n,k}(f\indi{(0,t]})(B_{\frac{k+1}{n}\wedge t}-B_{\frac{k}{n}}\wedge t)
    \end{align*}
    and the last expression is $\mathcal{F}_t$-measurable. Finally, if $0\leq s\leq t$ we have that since $M_t^f-M_s^f$ is independent of $\mathcal{F}_s$:
    $$
      \Exp(M_t^f-M_s^f\mid \mathcal{F}_s)=\Exp(M_t^f-M_s^f)=0
    $$
    Moreover, $({(M_t^f)}^2)_{t\geq 0}$ is clearly adapted and:
    \begin{multline*}
      \Exp\left({(M_t^f)}^2-{(M_s^f)}^2\mid \mathcal{F}_s\right)=\Exp\left({(M_t^f-M_s^f)}^2\mid\mathcal{F_s}\right)=\\=\Exp\left({(M_t^f-M_s^f)}^2\right)=\norm{I(f\indi{(s,t]})}_{L^2(\Omega)}^2=\\=\norm{f\indi{(s,t]}}_{L^2(\RR_{\geq 0})}^2=\int_s^t {f(u)}^2\dd{u}
    \end{multline*}
    where the first equality is due to \mnameref{SC:orthogonality_martingales} and the we used the isometry property of $I$. This implies that ${{(M_t^f)}^2}_{t\geq 0}-\int_0^t{f(u)}^2\dd{u}$ is a martingale and by the uniqueness of the quadratic variation we get the result.
  \end{proof}
  \begin{proposition}
    Let ${(B_t)}_{t\geq 0}$ be a Brownian motion. For any $f\in L_{\text{loc}}^2(\RR_{\geq 0})$, the process $Z^f={(Z_t^f)}_{t\geq 0}$ defined as:
    $$
      Z_t^f:=\exp{\int_0^t f(u)\dd{B_u}-\frac{1}{2}\int_0^t{f(u)}^2\dd{u}}
    $$
    is a continuous square-integrable martingale.
  \end{proposition}
  \begin{proof}
    The integrability and adaptedness poses no problem. Now fix $0\leq s\leq t$. We previously saw that $\int_s^t f(u)\dd{B_u}$ is independent of $\mathcal{F}_s$ and so:
    $$
      \Exp\left(Z_t^f\mid \mathcal{F}_s\right)=Z_s^f\Exp\left(\exp{\int_s^t f(u)\dd{B_u}-\frac{1}{2}\int_s^t{f(u)}^2\dd{u}}\right)=Z_s^f
    $$
  \end{proof}
\end{multicols}
\end{document}