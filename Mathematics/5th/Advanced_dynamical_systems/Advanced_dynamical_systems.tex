\documentclass[../../../main_math.tex]{subfiles}

\begin{document}
\changecolor{ADS}
\begin{multicols}{2}[\section{Advanced dynamical sytems}]
  \subsection{Discrete maps}
  \subsubsection{Maps in \texorpdfstring{$\TT^1$}{S1}}
  \begin{proposition}
    Let $\alpha=\frac{p}{q}\in\QQ$ and let $R_\alpha:\TT^1\to \TT^1$ be the rotation of angle $\alpha$. Then, all the points of $\TT^1$ are periodic for $R_\alpha$ with period $q$.
  \end{proposition}
  \begin{proof}
    We identify the elements of $\TT^1$ as $\quot{\RR}{\ZZ}$. Let $x\in \TT^1$. Then, ${R_\alpha}^q x=x+\alpha q=x+p=x$. And $q$ is the smallest integer such that ${R_\alpha}^q x=x$ because we assume that $p$ and $q$ are coprime.
  \end{proof}
  \begin{proposition}
    Let $\alpha\in\RR\setminus\QQ$ and let $R_\alpha:\TT^1\to \TT^1$ be the rotation of angle $\alpha$. Then, all the points of $\TT^1$ are dense in $\TT^1$.
  \end{proposition}
  \begin{proof}
    Let $\varepsilon>0$, $x,y\in \TT^1$. Discretize $\TT^1$ in intervals of length at most $\frac{1}{\varepsilon}$. Then, $\exists m,n\in \NN$ with $m< n\leq \frac{1}{\varepsilon}+1$ such that ${R_\alpha}^m x$ and ${R_\alpha}^nx$ are in the same interval. Thus, $\abs{{R_\alpha}^{n-m}x-x}<\varepsilon$. Now, concatenating ${R_\alpha}^{n-m}x$ repeatedly, we will eventually have $\abs{{R_\alpha}^{k(n-m)}x - y}<\varepsilon$ for some $k\in \NN$.
  \end{proof}
  \begin{corollary}
    Let $\alpha\in\RR\setminus\QQ$ and $A\subset \TT^1$ be a non-empty closed invariant set for $R_\alpha$. Then, $A=\TT^1$.
  \end{corollary}
  \begin{proof}
    Let $x\in \TT^1$ and $y\in A$. Then, $\forall k\in\NN$ $\exists n_k\in\NN$ such that $R_\alpha^{n_k}y\in(x-\frac{1}{k},x+\frac{1}{k})$. Thus, $R_\alpha^{n_k}y\to x$ and $x\in A$ because $A$ is closed and $R_\alpha^{n_k}y\in A$ $\forall k\in\NN$.
  \end{proof}
  \begin{definition}
    Consider the set $$\Sigma_m
      := \{(x_1,x_2,\ldots):x_i\in\{0,1,\ldots,m-1\}\}$$
    We define the \emph{shift map} as:
    $$
      \function{\sigma_m}{\Sigma_m}{\Sigma_m}{(x_1,x_2,\ldots)}{(x_2,x_3,\ldots)}
    $$
  \end{definition}
  \begin{remark}
    Note that some elements in $[0,1]$ have two different representations in base-$m$ identified as elements of $\Sigma_m$. So we can think of $\Sigma_m$ a the quotient space $\quot{\Sigma_m}{\sim}$ where $(x_1,x_2,\ldots)\sim (y_1,y_2,\ldots)$ if and only if $\sum_{i=1}^\infty \frac{x_i}{m^i}=\sum_{i=1}^\infty \frac{y_i}{m^i}$.
  \end{remark}
  \begin{proposition}
    Let $m\in\NN$. Consider the \emph{expansion map}
    $$
      \function{E_m}{\TT^1}{\TT^1}{x}{mx}
    $$
    Then, if $\phi:\Sigma_m\to \TT^1$ is the map $\phi(x_1,x_2,\ldots)=\sum_{i=1}^\infty \frac{x_i}{m^i}$, we have that $E_m\circ \phi=\phi\circ \sigma_m$. In particular, $\phi$ is a bijection, and thus it is a conjugacy between $E_m$ and $\sigma_m$.
  \end{proposition}
  \begin{proof}
    Let $x=(x_1,x_2,\ldots)\in \Sigma_m$. Then, $\phi\circ \sigma_m(x)=\sum_{i=1}^\infty \frac{x_{i+1}}{m^i}$. Moreover:
    \begin{multline*}
      E_m\circ \phi(x)=E\left(\sum_{i=1}^\infty \frac{x_i}{m^i}\right)=\sum_{i=1}^\infty \frac{x_i}{m^{i-1}}=\\=x_i+\sum_{i=1}\frac{x_{i+1}}{m^i}\equiv\sum_{i=1}\frac{x_{i+1}}{m^i}
    \end{multline*}
  \end{proof}
  \begin{remark}
    Note that $E$ preserves the Lebesgue measure \textit{backwards}: $\abs{{E_m}^{-1}(A)}=\abs{A}$ for all $A\subseteq \TT^1$, but $\abs{E_m(A)}\ne \abs{A}$ in general.
  \end{remark}
  \begin{definition}
    We define the following distance in $\Sigma_m$. For all $x,x'\in\Sigma_m$:
    $$
      d(x,x'):=\frac{1}{2^\ell}\quad\text{with }\ell:=\min\{i:x_i\ne x_i'\}
    $$
  \end{definition}
  \begin{proposition}
    Periodic points of $E_m$ are dense in $\TT^1$.
  \end{proposition}
  \begin{proof}
    By conjugacy it suffices to show that periodic points of $\sigma_m$ are dense in $\Sigma_m$. Let $x\in \Sigma_m$ and $\varepsilon>0$. Then, $\varepsilon>\frac{1}{2^\ell}$ for some $\ell$. And so the orbit of
    $$
      y=(x_1,\ldots,x_\ell,x_1,\ldots,x_\ell,x_1,\ldots,x_\ell,\ldots)
    $$
    is periodic and $d(x,y)<\varepsilon$. So periodic points of $\sigma_m$ are dense in $\Sigma_m$.
  \end{proof}
  \begin{proposition}
    Le $x\in \TT^1$. Then, the positive orbit of $x$ for $E_m$ is dense in $\TT^1$.
  \end{proposition}
  \begin{proof}
    By conjugacy, we only prove it for $\sigma_m$. But this is clear by taking:
    $$
      x=(0,1,\ldots,m-1,10,\ldots,1(m-1),20,\ldots,2(m-1),\ldots)
    $$
  \end{proof}
  \subsubsection{A hyperbolic automorphism of \texorpdfstring{$T^2$}{T2}}
  \begin{proposition}
    Consider $\vf{A}=\begin{pmatrix}
        2 & 1 \\
        1 & 1
      \end{pmatrix}\in \GL_2(\RR)$. Then, $\vf{A}(\ZZ^2)=\ZZ^2$ and this induces an automorphism $\vf{\tilde{A}}$ of $T^2=\quot{\RR^2}{\ZZ^2}$.
  \end{proposition}
  \begin{definition}
    We define the set of periodic points of $\vf{\tilde{A}}$ as $\Per\vf{\tilde{A}}$.
  \end{definition}
  \begin{lemma}
    $\Per\vf{\tilde{A}}=\quot{\QQ^2}{\ZZ^2}$. Thus, $\Per\vf{\tilde{A}}$ is dense in $T^2$.
  \end{lemma}
  \begin{proof}
    Let $\vf{x}\in \Per\vf{\tilde{A}}$. Then, $\exists k\in\NN$ and $\vf{n}\in\ZZ^2$ such that $\vf{A}^k\vf{x}=\vf{x}+\vf{n}$. One can easily check that $\sigma(\vf{\tilde{A}})=\left\{\frac{3}{2}\pm \frac{\sqrt{5}}{2}\right\}=:\{\lambda_{\pm}\}$ with $\lambda_-<1<\lambda_+$. Thus,
    $$
      \det(\vf{A}^k-\vf{I})=({\lambda_+}^k-1)({\lambda_-}^k-1)\ne 0
    $$
    and so the equation $\vf{A}^k\vf{x}=\vf{x}+\vf{n}$ has a unique (rational) solution. Now let $(\frac{p_1}{q_1},\frac{p_2}{q_2})\in \quot{\QQ^2}{\ZZ^2}$ and $N\geq 1$ left to be chosen. We define the set $Q_N:=\frac{\ZZ^2}{N} \mod{\ZZ^2}$, which is a subset finite set of $T^2$. But observe that $Q_N$ is invariant under $\vf{\tilde{A}}$, and thus, all of its points are periodic. For the above rational numbers, just choose $N=q_1q_2$.
  \end{proof}
  \begin{remark}
    The \emph{hyperbolicity} comes from the fact that there is one eigenvector with eigenvalue greater than $1$ and another with eigenvalue less than $1$, both eigenvalues being positive.
  \end{remark}
  \begin{theorem}
    The iterates of $\vf{\tilde{A}}$ smear every domain $F\subseteq T^2$ uniformly over $T^2$, that is, for every domain $G\subseteq T^2$, we have that the following limit exists:
    $$
      \abs{(\vf{\tilde{A}}^{-n} F)\cap G}\overset{n\to\infty}{\longrightarrow} \abs{F}\abs{G}
    $$
    This property of $\vf{\tilde{A}}$ is called \emph{mixing}.
  \end{theorem}
  \begin{proof}
    We can prove a more general property in terms of functions in the torus (and then apply it to $f=\indi{F}$ and $g=\indi{G}$):
    $$
      \lim_{n\to\infty}\int_{T^2} f(\vf{\tilde{A}}^n \vf{x}) g(\vf{x})\dd{\vf{x}}=\int_{T^2} f(\vf{x})\dd{\vf{x}}\int_{T^2} g(\vf{x})\dd{\vf{x}}
    $$
    We will prove this for the orthonormal basis of Fourier series $\{\exp{2\pi i \vf{p}\cdot \vf{x}}\}_{\vf{p}\in\ZZ^2}$. Note that:
    $$
      \int_{T^2} \exp{2\pi i (\transpose{(\vf{\tilde{A}}^n)}\vf{p})\cdot \vf{x}}\dd{\vf{x}}=\begin{cases}
        1 & \text{if }\vf{p}=\vf{0}    \\
        0 & \text{if }\vf{p}\ne \vf{0}
      \end{cases}
    $$
    Therefore, since $\transpose{(\vf{\tilde{A}}^n)}\vf{p}$ takes infinitely many values for $\vf{p}\ne \vf{0}$, we have that if $g=\exp{2\pi i \vf{q} \cdot \vf{x}}$ then:
    $$
      \lim_{n\to\infty}\int_{T^2} \exp{2\pi i(\transpose{(\vf{\tilde{A}}^n)}\vf{p}+\vf{q})\cdot \vf{x}}\dd{\vf{x}}=0
    $$
    So for any $\vf{p}, \vf{q}\in\ZZ^2$ we have the equality. Then, we use that any function nice enough can be approximated with its Fourier series.
  \end{proof}
  \begin{theorem}
    On the torus $T^2$ there exist two direction fields invariant with respect to the automorphism $\vf{\tilde{A}}$. The integral curves of each of these directions fields are everywhere dense on the torus. The automorphism $\vf{\tilde{A}}$ converts the integral curves of each field into integral curves of the same field, expanding by $\lambda_+$ for the first field and contracting by $\lambda_-$ for the second.
  \end{theorem}
  \begin{proof}
    Let $\vf{e}_+$ and $\vf{e}_-$ be the eigenvectors of $\vf{A}$ with eigenvalues $\lambda_+$ and $\lambda_-$ respectively. Let $\vf{x}\in T^2$ and
    $$
      \function{\vf\gamma_+}{\RR}{T^2}{t}{\vf{x}+t \vf{e}_+}\quad
      \function{\vf\gamma_-}{\RR}{T^2}{t}{\vf{x}+t \vf{e}_-}
    $$
    be the expanding and contracting curves and let $\vf{\xi}_{\vf{x}}=\im(\vf\gamma_+)$, $\vf{\eta}_{\vf{x}}=\im(\vf\gamma_-)$.
  \end{proof}
  \begin{definition}
    Let $\vf{A},\vf{B}:T^2\rightarrow T^2$ be $\mathcal{C}^1$ functions. We say that $B$ is \emph{$\mathcal{C}^0$-close} to $\vf{A}$ if for all $\varepsilon>0$:
    $$
      \sup_{\vf{x}\in T^2}\norm{\vf{A}(\vf{x})-\vf{B}(\vf{x})}<\varepsilon
    $$
    We say that $\vf{B}$ is \emph{$\mathcal{C}^1$-close} to $\vf{A}$ if for all $\varepsilon>0$, $\vf{B}$ is $\mathcal{C}^0$-close to $\vf{A}$ and:
    $$
      \sup_{\vf{x}\in T^2}\norm{\vf{D}\vf{A}(\vf{x})-\vf{D}\vf{B}(\vf{x})}<\varepsilon
    $$
  \end{definition}
  \begin{theorem}[Structal stability]
    Let $\vf{B}$ be a diffeomorphism on $T^2$ $\mathcal{C}^1$-close to $\vf{\tilde{A}}$. Then, $\vf{B}$ is $\mathcal{C}^0$-conjugate to $\vf{\tilde{A}}$.
  \end{theorem}
  \begin{proof}
    We need to find a $\mathcal{C}^0$-conjugacy $\vf{H}$ between $\vf{B}$ and $\vf{\tilde{A}}$. Since, $\vf{B}$ is $\mathcal{C}^1$-close to $\vf{\tilde{A}}$, we may expect that both $\vf{H}$ and $\vf{B}$ are small perturbations of the identity and $\vf{\tilde{A}}$ respectively. So set $\vf{H}=\vf{I}+\vf{h}$ and $\vf{B}=\vf{\tilde{A}}+\vf{b}$. Then, we want to find $\vf{h}$ and $\vf{b}$ such that:
    $$
      \vf{H}\circ \vf{\tilde{A}}=\vf{B}\circ \vf{H}\iff
      \vf{h}(\vf{\tilde{A}x})-\vf{\tilde{A}} \vf{h}(\vf{x})=\vf{b}(\vf{x}+\vf{h}(\vf{x}))
    $$
    This equation is called \emph{conjugacy equation}. Consider the operators
    \begin{gather*}
      \function{\vf{S}_{\vf{\tilde{A}}}}{\mathcal{C}^0(T^2,\RR^2)}{\mathcal{C}^0(T^2,\RR^2)}{\vf{h}}{\vf{h}(\vf{\tilde{A}}(\vf{x}))}\\
      \function{\vf{L}_{\vf{\tilde{A}}}}{\mathcal{C}^0(T^2,\RR^2)}{\mathcal{C}^0(T^2,\RR^2)}{\vf{h}}{\vf{S}_{\vf{\tilde{A}}}\vf{h}-\vf{\tilde{A}}\vf{h}}
    \end{gather*}
    Observe that:
    $$
      \sup_{\vf{x}\in T^2}\norm{\vf{S}_{\vf{\tilde{A}}}\vf{h}(\vf{x})}=\sup_{\vf{x}\in T^2}\norm{\vf{S}_{\vf{\tilde{A}}}\vf{h}(\vf{\tilde{A}}^{-1}\vf{x})}= \sup_{\vf{x}\in T^2}\norm{\vf{h}(\vf{x})}
    $$
    Hence, $\norm{\vf{S}_{\vf{\tilde{A}}}}=1$ and similarly $\norm{\vf{S}_{\vf{\tilde{A}}}^{-1}}=1$, where $\vf{S}_{\vf{\tilde{A}}}^{-1}:\vf{h}\mapsto \vf{h}(\vf{\tilde{A}}^{-1}(\vf{x}))$. We'll now prove that $\vf{L}_{\vf{\tilde{A}}}$ is invertible. Note that $\RR^2=\langle \vf{e}_+\rangle \oplus \langle \vf{e}_-\rangle$ because $\vf{\tilde{A}}$ is invertible. Thus:
    $$
      \vf{L}_{\vf{\tilde{A}}}\vf{h}=\vf{c}\iff \begin{cases}
        \vf{L}_{\vf{\tilde{A}}}\vf{h}_+=\vf{S}_{\vf{\tilde{A}}}\vf{h}_+-\lambda_+\vf{h}_+=\vf{c}_+ \\
        \vf{L}_{\vf{\tilde{A}}}\vf{h}_-=\vf{S}_{\vf{\tilde{A}}}\vf{h}_--\lambda_-\vf{h}_-=\vf{c}_-
      \end{cases}
    $$
    where $\vf{h}=\vf{h}_++\vf{h}_-$, $\vf{c}=\vf{c}_++\vf{c}_-$ and $\vf{h}_\pm,\vf{c}_\pm\in \langle \vf{e}_\pm\rangle$. Now, note that $\norm{\frac{\vf{S}_{\vf{\tilde{A}}}}{\lambda_+}}<1$ and so
    $$
      (\vf{S}_{\vf{\tilde{A}}}-\lambda_+\vf{I})=\lambda_+\left(\frac{\vf{S}_{\vf{\tilde{A}}}}{\lambda_+}-\vf{I}\right)
    $$
    is invertible. Similarly, we have $\norm{\vf{S}_{\vf{\tilde{A}}}^{-1}\lambda_-}<1$ and so
    $$
      (\vf{S}_{\vf{\tilde{A}}}^{-1}-\lambda_-\vf{I})=\vf{S}_{\vf{\tilde{A}}}^{-1}\left(\vf{I}-\lambda_-\vf{S}_{\vf{\tilde{A}}}^{-1}\right)
    $$
    is invertible because it is a product of invertible operators. Thus, $\vf{L}_{\vf{\tilde{A}}}$ is invertible. Now, we return to our initial problem. Find $\vf{h}$ such that $\vf{h}= {\vf{L}_{\vf{\tilde{A}}}}^{-1}(\vf{b}(\vf{x}+\vf{h}(\vf{x})))=:\vf\Psi(\vf{h})$, which is a fixed-point problem. Note that $\vf\Psi$ is a contraction. Indeed:
    \begin{align*}
      \norm{\vf\Psi(\vf{h})-\vf\Psi(\vf{h}')} & \leq \norm{{\vf{L}_{\vf{\tilde{A}}}}^{-1}}\!\norm{\vf{b}(\vf{x}+\vf{h}(\vf{x}))\!-\!\vf{b}(\vf{x}+\vf{h}'(\vf{x}))} \\
                                              & \leq \norm{{\vf{L}_{\vf{\tilde{A}}}}^{-1}}\norm{\vf{Db}}\norm{\vf{h}-\vf{h}'}
    \end{align*}
    which is arbitrarily small ($\norm{\vf{Db}}$ is arbitrarily small) because $\vf{B}$ is $\mathcal{C}^1$-close to $\vf{\tilde{A}}$. Thus, $\vf{h}$ exists and it's unique.
  \end{proof}
  \begin{definition}
    A dynamical system $f : X\rightarrow X$ has \emph{sensitive dependence on initial conditions} on $X$ if $\exists\varepsilon >0$ such that for each $x\in X$ and any neighborhood $N_x$ of $x$, exists $y \in N_x$ and $n \geq  0$ such that $d(f^n(x),f^n(y)) > \varepsilon$.
  \end{definition}
  \begin{definition}
    Let $U\subseteq \RR^n$, $\vf{f}:U\to U$ be a dynamical system and $\vf{x}\in U$ and $\vf{v}\in\RR^n$. We define the \emph{Lyapunov exponent} as:
    $$
      \chi(x,\vf{v}):=\limsup_{n\to\infty}\frac{1}{n}\log\norm{\vf{D}(\vf{f}^n)(x)\vf{v}}
    $$
  \end{definition}
  \begin{remark}
    The Lyapunov exponent measures the exponential growth rate of tangent vectors along orbits. It can rarely be computed explicitly, but if we can show that $\chi(x,\vf{v})>0$ for some $\vf{v}$, then we know that the system is \emph{chaotic}.
  \end{remark}
  \subsubsection{Hamiltonian systems}
  \begin{definition}
    Let $U\subseteq \RR^n\times \RR^n$ be open and $H:U\rightarrow \RR$ be a $\mathcal{C}^1$ function. We define the \emph{Hamiltonian vector field} associated to $H$ as:
    \begin{equation}\label{ADS:ham_system}
      \begin{cases}
        \displaystyle\dot{\vf{x}}=\pdv{H}{\vf{p}}\vspace{0.1cm} \\
        \displaystyle\dot{\vf{p}}=-\pdv{H}{\vf{x}}
      \end{cases}
    \end{equation}
  \end{definition}
  \begin{remark}
    Recall that $H$ is a first integral of the system \mcref{ADS:ham_system}.
  \end{remark}
  \begin{lemma}
    Let $H:U\rightarrow \RR$ be a $\mathcal{C}^1$ function, $W\subseteq U$. Then, the volume of $W$ under the field of \mcref{ADS:ham_system} is preserved.
  \end{lemma}
  \begin{proof}
    Let $W_t:=\vf{\phi}_t(W)$, where $\phi_t$ is the flow of \mcref{ADS:ham_system}. Then:
    \begin{multline*}
      \dv{}{t}\vol(W_t)=\dv{}{t}\int_{\phi_t(W)}\dd{\vf{x}}=\int_W\dv{}{t}\det \vf{D\phi}_t=\\
      =\int_W\trace\left(\dv{}{t}\vf{D\phi}_t\right)=\int_W\div \vf{X}_H
    \end{multline*}
    where $\vf{X}_H$ is the vector field of \mcref{ADS:ham_system}, and  we used that the derivative of the determinant map is the trace. But an easy computation shows that $\div \vf{X}_H=0$.
  \end{proof}
  \subsection{Circle dynamics}
  \subsubsection{Generalities}
  \begin{definition}
    Let $x,x'\in\RR$. We say that $x\sim x'$ if and only if $x-x'\in\ZZ$. We define the \emph{circle} as $\TT^1:=\quot{\RR}{\sim}$. We define the following distance in $\TT^1$:
    $$
      d(\overline{x},\overline{y})=\min_{x'\in\overline{x},y'\in\overline{y}}\abs{x'-y'}
    $$
  \end{definition}
  \begin{proposition}[Existence of a lift]\hfill
    \begin{enumerate}
      \item For any continuous map $F:\TT^1\to \TT^1$ there exists a \emph{lift} $f$, i.e.\ a continuous map $f:\RR\to \RR$ such that $F\circ \pi=\pi\circ f$, where $\pi:\RR\to \TT^1$ is the canonical projection.
      \item If $g$ is another lift of $F$, then $g-f=k\in\ZZ$.
    \end{enumerate}
  \end{proposition}
  \begin{proof}
    We only prove the second property. Since, $f$, $g$ are both lifts of $F$, they belong to the same equivalence class. Thus, $f-g\in\ZZ$. And now use the continuity of $f-g$.
  \end{proof}
  \begin{remark}
    Recall that $f:\RR\to\RR$ is a homeomorphism if and only if $f$ is monotone.
  \end{remark}
  \begin{definition}
    We say that a homeomorphism $F$ \emph{preserves orientation} if and only if $f$ is strictly increasing. We define the set of $\Homeo(\TT^1)$ as the set of homeomorphisms of $\TT^1$ that preserve orientation.
  \end{definition}
  \begin{proposition}
    Let $F\in\Homeo(\TT^1)$. Then, $F$ admits a lift $f$ such that $f(x)=x+\varphi(x)$, where $\varphi:\RR\to\RR$ is a 1-periodic function.
  \end{proposition}
  \begin{proof}
    We already now that $F$ admits a lift $f$. A straightforward calculation shows that $f_1:\RR\to\RR$ defined by $f_1(x)=f(x+1)$ is also a lift of $F$. Thus, $f_1-f=k\in \ZZ$. Now, since $f$ must be strictly increasing, we need $k\in \NN$. Moreover, since $F$ is injective, $f|_{[0,1)}$ is injective and its image cannot contain 2 points whose difference is an integer. Thus, $k=1$. Now, define $\varphi(x)=f(x)-x$, which is 1-periodic:
    $$
      \varphi(x+1)=f(x+1)-(x+1)=f(x)-x=\varphi(x)
    $$
  \end{proof}
  \begin{definition}
    We define the set:
    \begin{multline*}
      \mathcal{D}^0(\TT^1):\{f:\RR\to\RR:f\text{ increasing and}\\
      \text{ homeomorphism}, f(x+1)=f(x)+1\}
    \end{multline*}
    Note that we have the projection:
    $$
      \function{}{\mathcal{D}^0(\TT^1)}{\Homeo(\TT^1)}{f}{F}
    $$
    We can define a distance in $\mathcal{D}^0(\TT^1)$ as:
    $$
      d(f,g)=\max\{ \sup_{x\in\RR}\abs{f(x)-g(x)},\sup_{x\in\RR}\abs{f^{-1}(x)-g^{-1}(x)}\}
    $$
  \end{definition}
  \begin{lemma}
    $\mathcal{D}^0(\TT^1)$ is a complete metric space. Moreover:
    \begin{enumerate}
      \item $f\to f^{-1}$ is continuous, $f\in \mathcal{D}^0(\TT^1)$.
      \item $(f,g)\to f\circ g$ is continuous, $(f,g)\in \mathcal{D}^0(\TT^1)\times \mathcal{D}^0(\TT^1)$.
    \end{enumerate}
    Thus, $\mathcal{D}^0(\TT^1)$ is a topological group with the composition.
  \end{lemma}
  \begin{definition}
    Let $\varepsilon\geq 0$ and $\alpha\in\RR$. We define the \emph{Arnold family} as:
    $$
      \function{f_{\alpha,\varepsilon}}{\RR}{\RR}{x}{x+\alpha+\varepsilon\sin(2\pi x)}
    $$
  \end{definition}
  \begin{lemma}
    If $0\leq \varepsilon<\frac{1}{2\pi}$, then $f_{\alpha,\varepsilon}\in \mathcal{D}^0(\TT^1)$.
  \end{lemma}
  \begin{proof}
    Note that ${f_{\alpha,\varepsilon}}'>0\iff \varepsilon<\frac{1}{2\pi}$. Thus, $f_{\alpha,\varepsilon}$ is strictly increasing, and therefore it is a homeomorphism.
  \end{proof}
  \subsubsection{Rotation number}\label{ADS:rotation_number_section}
  \begin{remark}
    Recall that $f=\id+\varphi$ with $\varphi$ 1-periodic. And thus:
    $$
      f^n=\id + \sum_{i=0}^{n-1} \varphi\circ f^i=: \id + \varphi_n
    $$
    with $\varphi_n$ 1-periodic.
  \end{remark}
  \begin{lemma}\label{ADS:lema1}
    Let $f\in \mathcal{D}^0(\TT^1)$ be such that $f=\id +\varphi$, with $\varphi$ 1-periodic. Let $m:=\min_{x\in\RR}\varphi$ and $M:=\max_{x\in\RR}\varphi$. Then, we have $m\leq M< m+1$.
  \end{lemma}
  \begin{proof}
    Let $0\leq x\leq y<1\leq x+1$. Then, $f(y)<f(x+1)=f(x)+1$. Thus, $f(y)+x< f(x)+1+y$, and so $\varphi(y) < \varphi(x)+1$. Now take supremum in $y$ and infimum in $x$.
  \end{proof}
  \begin{definition}
    Let ${(u_n)}\in\RR$ be a sequence. We say that $(u_n)$ is \emph{subadditive} if $u_{n+m}\leq u_n+u_m$ for all $n,m\in\NN$. We say that $(u_n)$ is \emph{superadditive} if $u_{n+m}\geq u_n+u_m$ for all $n,m\in\NN$, that is, if $(-u_n)$ is subadditive.
  \end{definition}
  \begin{lemma}\label{ADS:lema2}
    Let $f\in \mathcal{D}^0(\TT^1)$ be such that $f=\id +\varphi$. We can write $f^n=\id +\varphi_n$ and let $m_n:=\min_{x\in\RR}\varphi_n$ and $M_n:=\max_{x\in\RR}\varphi_n$. Then, $(M_n)$ is subadditive and $(m_n)$ is superadditive.
  \end{lemma}
  \begin{proof}
    We have that:
    \begin{equation*}
      f^{n+m}(x)-x=(f^m-\id)(f^n(x))+f^n(x)-x\leq M_m+M_n
    \end{equation*}
    Now take the supremum in $x$. The other inequality is analogous.
  \end{proof}
  \begin{lemma}\label{ADS:lema3}
    Let $(u_n)\in\RR$ be a subadditive sequence. Then, $\displaystyle\lim_{n\to\infty}\frac{u_n}{n}$ exists and it is equal to $\inf_{n\in\NN}\frac{u_n}{n}$. Analogously, if $(u_n)$ is superadditive, then $\displaystyle\lim_{n\to\infty}\frac{u_n}{n}$ exists and it is equal to $\sup_{n\in\NN}\frac{u_n}{n}$.
  \end{lemma}
  \begin{proof}
    Assume $(u_n)$ is positive and subadditive and fix $p\in\NN$. Let $n\geq p$ be such that $n=k_np+r_n$ with $r<p$. Then:
    $$
      \frac{u_n}{n}\leq \frac{u_{k_np}+u_{r_n}}{n}\leq \frac{k_nu_p}{n}+\frac{u_{r_n}}{n}=\frac{u_p}{p+\frac{r_n}{k_n}}+\frac{u_{r_n}}{n}
    $$
    where in the first and second inequalities we used that $(u_n)$ is subadditive. Now to show that the limit exists and that the value is the one of above, take first $\limsup$ in $n$ and then $\inf$ in $p$:
    $$
      \limsup_{n\to\infty}\frac{u_n}{n}\leq \inf_{p\in\NN}\frac{u_p}{p}\leq \liminf_{p\to\infty}\frac{u_p}{p}
    $$
  \end{proof}
  \begin{theorem}[Existence of the rotation number]
    For all $f\in \mathcal{D}^0(\TT^1)$, we have that the sequence of functions $\frac{1}{n}(f^n-\id)$ convergence uniformly to constant function $\rho(f)\in\RR$. This number is called the \emph{rotation number} of $f$.
  \end{theorem}
  \begin{proof}
    By \mcref{ADS:lema1} we have that $\frac{m_n}{n}\leq \frac{M_n}{n}< \frac{m_n}{n}+\frac{1}{n}$, where $m_n:= \min_{x\in\RR}\varphi_n$ and $M_n:= \max_{x\in\RR}\varphi_n$. By \mcref{ADS:lema2,ADS:lema3}, we have that $\frac{m_n}{n}$ and $\frac{M_n}{n}$ have the same limit and moreover:
    $$
      \frac{m_n}{n}\leq \frac{1}{n}(f^n(x)-x)\leq \frac{M_n}{n}
    $$
    So we have the result, and in fact the convergence is uniform by domination.
  \end{proof}
  \begin{proposition}
    The following properties are satisfied:
    \begin{enumerate}
      \item $\rho(R_\alpha)=\alpha$ $\forall\alpha\in\RR$.
      \item $\rho(f^n)=n\rho(f)$ $\forall f\in \mathcal{D}^0(\TT^1)$, $n\in\NN$.
      \item $f\leq g\implies \rho(f)\leq \rho(g)$ $\forall f,g\in \mathcal{D}^0(\TT^1)$.
      \item $\rho(f+k)=:\rho(R_k\circ f)=\rho(f) + k$ $\forall f\in \mathcal{D}^0(\TT^1)$, $k\in\ZZ$.
    \end{enumerate}
  \end{proposition}
  \begin{definition}
    Let $F\in \Homeo(\TT^1)$ with lift $f$. We define the \emph{rotation number} of $F$ as $\rho(F):=[\rho(f)]\in \TT^1$.
  \end{definition}
  \begin{definition}
    Let $F,G\in\Homeo(\TT^1)$. We say that $G$ is \emph{semi-conjugate} to $F$ if there exists a continuous surjective map $H:\TT^1\to \TT^1$ such that $H\circ F=G\circ H$. We say that $G$ is \emph{conjugate} to $F$ if $H$ is a homeomorphism.
  \end{definition}
  \begin{lemma}
    Let $F,G\in\Homeo(\TT^1)$ be such that $G$ is semi-conjugate to $F$. Then, if $G$ has a periodic point, then $F$ has a periodic point.
  \end{lemma}
  \begin{theorem}
    Let $F,G\in\Homeo(\TT^1)$ be conjugate by $H\in \Homeo(\TT^1)$. Then, $\rho(F)=\rho(G)$.
  \end{theorem}
  \begin{proof}
    Let $h$ and $f$ be lifts of $H$ and $F$ respectively. Then, an easy check shows that $g:=h\circ f\circ h^{-1}$ is a lift of $G$. It suffices to prove that $\rho(g)=\rho(f)$. Note that, by induction we have $h\circ f^n=g^n\circ h$ for all $n\in\NN$. Now write $h=\id + \varphi$ with $\varphi\in \mathcal{C}(\TT^1)$. Then:
    \begin{equation*}
      \frac{f^n(x)-x+\varphi(f^n(x))}{n}= \frac{g^n(h(x))-h(x)}{n}+\frac{h(x)-x}{n}
    \end{equation*}
    Taking limits, we have that $\rho(f)=\rho(g)$, as $\varphi$ is bounded.
  \end{proof}
  \subsubsection{Rotation number and invariant measure}
  \begin{definition}
    We say that $\mu:\mathcal{C}(\TT^1)\to\RR$ is a \emph{measure on $\mathcal{C}(\TT^1)$} if:
    \begin{enumerate}
      \item $\mu$ is linear.
      \item $\mu$ is continuous.
      \item $\mu(\varphi)\geq 0$, whenever $\varphi\geq 0$.
    \end{enumerate}
    We say that $\mu$ is a \emph{probability measure} if $\mu(1)=1$. We denote by $\mathcal{M}(\TT^1)$ the set of all probability measures on $\mathcal{C}(\TT^1)$.
  \end{definition}
  \begin{remark}
    Usually we will denote $\mu(\varphi)$ by $\int_{\TT^1}\varphi\dd{\mu}$ or $\int_{\TT^1}\varphi(x)\dd{\mu(x)}$.
  \end{remark}
  \begin{remark}
    Note that we then have $\mu(\varphi)>0$ whenever $\varphi>0$, because $\varphi$ attains its minimum at some point $x_0$ (by the compactness of $\TT^1$). Similarly, $\mu(\varphi)\leq 0$ whenever $\varphi\leq 0$, and $\mu(\varphi)<0$ whenever $\varphi<0$.
  \end{remark}
  \begin{remark}
    Examples of such measures are the Dirac measures
    $$\delta_x(\varphi)=\varphi(x)\qquad x\in \TT^1$$
    or the Lebesgue measure:
    $$\text{Leb}(\varphi):=\int_{0}^1\varphi(x)\dd{x}$$
  \end{remark}
  \begin{definition}
    Let $F\in \Homeo(\TT^1)$ and $\mu\in \mathcal{M}(\TT^1)$. We define the \emph{pushforward measure} as $F_*\mu(\varphi):=\mu(\varphi\circ F)$.
  \end{definition}
  \begin{definition}
    We say that a measure $\mu\in\mathcal{M}(\TT^1)$ is \emph{invariant} by $F\in\Homeo(\TT^1)$ (or \emph{$F$-invariant}) if $F_*\mu=\mu$. We will denote by $\mathcal{M}_F(\TT^1)$ the set of $F$-invariant probability measures.
  \end{definition}
  \begin{proposition}
    Let $F\in \Homeo(\TT^1)$, $x\in\TT^1$ and $n\in\NN$.
    \begin{itemize}
      \item Note that $\text{Leb}$ is invariant under $R_\alpha$ $\forall \alpha\in\RR$.
      \item $\delta_x$ is $F$-invariant $\iff F(x)=x$
      \item $\displaystyle\frac{\delta_x+\cdots+\delta_{F^{n-1}(x)}}{n}$ is $F$-invariant $\iff F^n(x)=x$
    \end{itemize}
  \end{proposition}
  \begin{theorem}
    Let $F\in\Homeo(\TT^1)$. Then, $\mathcal{M}_F(\TT^1)\ne\varnothing$.
  \end{theorem}
  \begin{proposition}
    Let $F\in\Homeo(\TT^1)$ and $f=\id+\varphi$ be a lift of $F$, with $\varphi\in\mathcal{C}(\TT^1)$. Then, $\forall\mu\in\mathcal{M}_F(\TT^1)$, $\rho(f)=\mu(\varphi)$. Moreover:
    \begin{enumerate}
      \item $\norm{f^n-\id -n\rho(f)}_{\mathcal{C}(\TT^1)}<1$
      \item $\forall n\in\NN$, $\exists x_n\in\RR$ such that $f^n(x_n)-x_n=n\rho(f)$.
    \end{enumerate}
  \end{proposition}
  \begin{proof}
    Let $\psi_n:= f^n-\id -n\mu(\varphi)$. We have that:
    \begin{multline*}
      \mu(\psi_n)=\sum_{i=0}^{n-1}\mu(\varphi\circ f^i)-n\varphi(n)=\sum_{i=0}^{n-1}\mu(\varphi\circ F^i)-n\varphi(n)=0
    \end{multline*}
    where we have used the first remark in the previous section. Now we must have that $\psi_n$ change their sign in $[0,1]$ because otherwise that would contradict $\mu(\psi_n)=0$. So $\exists x_n\in[0,1]$ such that $\psi_n(x_n)=0$. So:
    $$
      f^n(x_n)-x_n=n\mu(\varphi)
    $$
    Dividing by $n$ and taking limits, we have that $\rho(f)=\mu(\varphi)$. This also shows the second point. To prove the first one, note that $\min\psi_n\leq 0$ and so by \mcref{ADS:lema1} we have $\max\psi_n <1$. Moreover, $\min\psi_n =-\max(-\psi_n) >-1$ (using the same argument as before) and so $\norm{\psi_n}_{\mathcal{C}(\TT^1)}<1$.
  \end{proof}
  \subsubsection{Rational rotation number}
  \begin{proposition}\label{ADS:characterisation_rot_number}
    Let $f\in \mathcal{D}^0(\TT^1)$, $p\in\ZZ$ and $q\in\NN$. Then:
    \begin{align*}
      \rho(f)=\frac{p}{q} & \iff \exists x\in\RR\text{ such that }f^q(x)=x+p \\
      \rho(f)>\frac{p}{q} & \iff \forall x\in\RR\text{ we have }f^q(x)>x+p   \\
      \rho(f)<\frac{p}{q} & \iff \forall x\in\RR\text{ we have }f^q(x)<x+p
    \end{align*}
  \end{proposition}
  \begin{proof}
    Since $\rho(f^q)= q\rho(f)$ and $\rho(f+p)=\rho(f)+p$, we have that if $g=f^q-p$, $\rho(g)=q \rho(f)-p$. Thus, an easy check shows that we can assume that $p=0$ and $q=1$. We will only prove the equivalences to the left, as it is sufficient.
    \begin{enumerate}
      \item Assume $f(x)=x$ for some $x\in\RR$. Then, from the definition of $\rho(f)$ applied to the point $x$, we have that $\rho(f)=0$.
      \item Assume $f(x)>x$ and write $f=\id+\varphi$ with $\varphi\in\mathcal{C}(\TT^1)$ and $\varphi>0$. Since, $\TT^1$ is compact, we have in fact that $\varphi\geq \min\varphi=:\varepsilon>0$. Now:
            $$
              f^n-\id = \sum_{i=0}^{n-1}\varphi\circ f^i\geq n\varepsilon
            $$
            And so $\rho(f)\geq \varepsilon>0$.
      \item Proceed as in the previous case.
    \end{enumerate}
  \end{proof}
  \begin{definition}
    Let $F\in\Homeo(\TT^1)$ and $x\in \TT^1$. We define the \emph{orbit} of $x$ as:
    $$
      \mathcal{O}_F(x):=\{F^n(x):n\in\ZZ\}
    $$
    We also define the \emph{positive orbit} of $x$  and the \emph{negative orbit} of $x$ as:
    \begin{align*}
      \mathcal{O}_F^+(x) & :=\{F^n(x):n\in\ZZ_{\geq 0}\} \\
      \mathcal{O}_F^-(x) & :=\{F^n(x):n\in\ZZ_{\leq 0}\}
    \end{align*}
    If the homeomorphism is not specified, we will omit the subscript.
  \end{definition}
  \begin{definition}
    Let $F\in\Homeo(\TT^1)$ and $X\subset \TT^1$. We say that $X$ is \emph{positively invariant} if $F(X)\subseteq X$ and \emph{negatively invariant} if $F^{-1}(X)\subseteq X$. We say that $X$ is \emph{invariant} if $F(X)=X$.
  \end{definition}
  \begin{proposition}
    Let $X\subset \TT^1$ and $x\in \TT^1$. Then:
    \begin{enumerate}
      \item $X$ is invariant $\iff \forall x\in X$, $\mathcal{O}(x)\subseteq X\iff X$ is a union of orbits.
      \item $\mathcal{O}(x)$ is finite $\iff x$ is periodic.
      \item The omega limit $\omega(x)$ and the alpha limit $\alpha(x)$ are non-empty compact invariant sets.
    \end{enumerate}
  \end{proposition}
  \begin{definition}
    Let $F\in \Homeo(\TT^1)$. We define the \emph{positively recurrent points} and \emph{negatively recurrent points} as:
    \begin{align*}
      R^+(F):=\{x\in\TT^1:x\in\omega(x)\} \\
      R^-(F):=\{x\in\TT^1:x\in\alpha(x)\}
    \end{align*}
  \end{definition}
  \begin{proposition}
    Let $F\in \Homeo(\TT^1)$. Then, $R^\pm(F)$ are invariant non-closed sets.
  \end{proposition}
  \begin{definition}
    Let $F\in \Homeo(\TT^1)$ and $x\in \TT^1$. We say that $x$ is a \emph{wandering point} if there exists a neighborhood $U$ of $x$ such that $\forall n\geq 1$ we have $F^n(U)\cap U=\varnothing$. The orbit $U$ is called a \emph{wandering domain}. We define the set:
    $$
      \Omega(F):=\{ x\in\TT^1: x\text{ is not wandering}\}
    $$
  \end{definition}
  \begin{remark}
    A point $x\in\TT^1$ is \emph{non-wandering} if it is not wandering, i.e.\ if $\forall U$ neighborhood of $x$ $\exists n\geq 1$ such that $F^n(U)\cap U\ne\varnothing$.
  \end{remark}
  \begin{proposition}
    Let $F\in \Homeo(\TT^1)$. Then, $\Omega(F)$ is an invariant closed set.
  \end{proposition}
  \begin{remark}
    Note that:
    $$
      \Fix(F)\subseteq \Per(F)\subseteq R^\pm(F)\subseteq \Omega(F)\subseteq \TT^1
    $$
  \end{remark}
  \begin{definition}
    Let $F\in \Homeo(\TT^1)$ and $X\subseteq \TT^1$ be closed and invariant. We say that $X$ is \emph{minimal} if $\forall x\in X$, $\overline{\mathcal{O}(x)}=X$. If $X=\TT^1$, we say that $F$ is \emph{minimal}.
  \end{definition}
  \begin{proposition}
    Let $F\in \Homeo(\TT^1)$ and $X\subseteq \TT^1$ be a closed and invariant. Then, $X$ is minimal $\iff$ $\forall Y\subseteq X$ closed, invariant and non-empty, $Y=X$.
  \end{proposition}
  \begin{theorem}
    Let $F\in \Homeo(\TT^1)$ with $\rho(F)=\frac{p}{q}\in \quot{\QQ}{\ZZ}$. Then:
    \begin{enumerate}
      \item $F$ has periodic points of period $q$, and any periodic point of $F$ has minimal period $q$.
      \item For any $x\in \TT^1$, $\omega(x)$ and $\alpha(x)$ are periodic orbits.
    \end{enumerate}
  \end{theorem}
  \begin{proof}
    First we assume $q=1$ and $p=0$. Let $f\in \mathcal{D}^0(\TT^1)$ be a lift of $F$. By \mcref{ADS:characterisation_rot_number}, we have that $\exists x\in \RR$ with $f(x)=x$. So $\Fix(f)\ne \varnothing$, and it is closed and invariant by translations. Now we write $\RR\setminus\Fix(f)$ as union of open intervals. Let $(a,b)$ be one connected component. Inside it, we must have either $f(x)>x$ or $f(x)<x$. In the first case we have that $(f^n(x))$ is strictly increasing $\forall x\in (a,b)$ and so $\omega(x)=\{b\}$ and $\alpha(x)=\{a\}$ $\forall x\in(a,b)$. The second case is exactly the opposite.

    Now we do the general case. Assume $\rho(f)=\frac{p}{q}$. Then, again by \mcref{ADS:characterisation_rot_number}, we have that $\exists x\in \RR$ with $f^q(x)=x+p$. Assume we have $x'\in\RR$ and $p',q'\in\ZZ$ with $q'\geq 1$ such that $f^{q'}(x')=x'+p'$. By \mcref{ADS:characterisation_rot_number}, we have that $\frac{p}{q}=\frac{p'}{q'}$ and so $\exists k\in\NN$ such that $q'=kq$ and $p'=kp'$ because $\frac{p}{q}$ is irreducible. Now let $g=f^q-p$. Then, an easy calculation shows that $g^k(x')=x'$. But $\rho(g)=0$ and in the previous case we have seen that the periodic points are only fixed points, so $k=1$. For the second part, we proceed as in the previous case with the function $g=f^q-p$.
  \end{proof}
\end{multicols}
\end{document}