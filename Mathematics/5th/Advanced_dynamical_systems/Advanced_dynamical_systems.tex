\documentclass[../../../main_math.tex]{subfiles}

\begin{document}
\changecolor{ADS}
\begin{multicols}{2}[\section{Advanced dynamical sytems}]
  \subsection{Introduction}
  \subsubsection{Maps in \texorpdfstring{$\S^1$}{S1}}
  \begin{proposition}
    Let $\alpha=\frac{p}{q}\in\QQ$ and let $R_\alpha:\S^1\to \S^1$ be the rotation of angle $\alpha$. Then, all the points of $\S^1$ are periodic for $R_\alpha$ with period $q$.
  \end{proposition}
  \begin{proof}
    We identify the elements of $\S^1$ as $\quot{\RR}{\ZZ}$. Let $x\in \S^1$. Then, ${R_\alpha}^q x=x+\alpha q=x+p=x$. And $q$ is the smallest integer such that ${R_\alpha}^q x=x$ because we assume that $p$ and $q$ are coprime.
  \end{proof}
  \begin{proposition}
    Let $\alpha\in\RR\setminus\QQ$ and let $R_\alpha:\S^1\to \S^1$ be the rotation of angle $\alpha$. Then, all the points of $\S^1$ are dense in $\S^1$.
  \end{proposition}
  \begin{proof}
    Let $\varepsilon>0$, $x,y\in \S^1$. Discretize $\S^1$ in intervals of length at most $\frac{1}{\varepsilon}$. Then, $\exists m,n\in \NN$ with $m< n\leq \frac{1}{\varepsilon}+1$ such that ${R_\alpha}^m x$ and ${R_\alpha}^nx$ are in the same interval. Thus, $\abs{{R_\alpha}^{n-m}x-x}<\varepsilon$. Now, concatenating ${R_\alpha}^{n-m}x$ repeatedly, we will eventually have $\abs{{R_\alpha}^{k(n-m)}x - y}<\varepsilon$ for some $k\in \NN$.
  \end{proof}
  \begin{corollary}
    Let $\alpha\in\RR\setminus\QQ$ and $A\subset \S^1$ be a non-empty closed invariant set for $R_\alpha$. Then, $A=\S^1$.
  \end{corollary}
  \begin{proof}
    Let $x\in \S^1$ and $y\in A$. Then, $\forall k\in\NN$ $\exists n_k\in\NN$ such that $R_\alpha^{n_k}y\in(x-\frac{1}{k},x+\frac{1}{k})$. Thus, $R_\alpha^{n_k}y\to x$ and $x\in A$ because $A$ is closed and $R_\alpha^{n_k}y\in A$ $\forall k\in\NN$.
  \end{proof}
  \begin{definition}
    Consider the set $$\Sigma_m
      := \{(x_1,x_2,\ldots):x_i\in\{0,1,\ldots,m-1\}\}$$
    We define the \emph{shift map} as:
    $$
      \function{\sigma_m}{\Sigma_m}{\Sigma_m}{(x_1,x_2,\ldots)}{(x_2,x_3,\ldots)}
    $$
  \end{definition}
  \begin{remark}
    Note that some elements in $[0,1]$ have two different representations in base-$m$ identified as elements of $\Sigma_m$. So we can think of $\Sigma_m$ a the quotient space $\quot{\Sigma_m}{\sim}$ where $(x_1,x_2,\ldots)\sim (y_1,y_2,\ldots)$ if and only if $\sum_{i=1}^\infty \frac{x_i}{m^i}=\sum_{i=1}^\infty \frac{y_i}{m^i}$.
  \end{remark}
  \begin{proposition}
    Let $m\in\NN$. Consider the \emph{expansion map}
    $$
      \function{E_m}{\S^1}{\S^1}{x}{mx}
    $$
    Then, if $\phi:\Sigma_m\to \S^1$ is the map $\phi(x_1,x_2,\ldots)=\sum_{i=1}^\infty \frac{x_i}{m^i}$, we have that $E_m\circ \phi=\phi\circ \sigma_m$. In particular, $\phi$ is a bijection, and thus it is a conjugacy between $E_m$ and $\sigma_m$.
  \end{proposition}
  \begin{proof}
    Let $x=(x_1,x_2,\ldots)\in \Sigma_m$. Then, $\phi\circ \sigma_m(x)=\sum_{i=1}^\infty \frac{x_{i+1}}{m^i}$. Moreover:
    \begin{multline*}
      E_m\circ \phi(x)=E\left(\sum_{i=1}^\infty \frac{x_i}{m^i}\right)=\sum_{i=1}^\infty \frac{x_i}{m^{i-1}}=\\=x_i+\sum_{i=1}\frac{x_{i+1}}{m^i}\equiv\sum_{i=1}\frac{x_{i+1}}{m^i}
    \end{multline*}
  \end{proof}
  \begin{remark}
    Note that $E$ preserves the Lebesgue measure \textit{backwards}: $\abs{{E_m}^{-1}(A)}=\abs{A}$ for all $A\subseteq \S^1$, but $\abs{E_m(A)}\ne \abs{A}$ in general.
  \end{remark}
  \begin{definition}
    We define the following distance in $\Sigma_m$. For all $x,x'\in\Sigma_m$:
    $$
      d(x,x'):=\frac{1}{2^\ell}\quad\text{with }\ell:=\min\{i:x_i\ne x_i'\}
    $$
  \end{definition}
  \begin{proposition}
    Periodic points of $E_m$ are dense in $\S^1$.
  \end{proposition}
  \begin{proof}
    By conjugacy it suffices to show that periodic points of $\sigma_m$ are dense in $\Sigma_m$. Let $x\in \Sigma_m$ and $\varepsilon>0$. Then, $\varepsilon>\frac{1}{2^\ell}$ for some $\ell$. And so the orbit of
    $$
      y=(x_1,\ldots,x_\ell,x_1,\ldots,x_\ell,x_1,\ldots,x_\ell,\ldots)
    $$
    is periodic and $d(x,y)<\varepsilon$. So periodic points of $\sigma_m$ are dense in $\Sigma_m$.
  \end{proof}
  \begin{proposition}
    Le $x\in \S^1$. Then, the positive orbit of $x$ for $E_m$ is dense in $\S^1$.
  \end{proposition}
  \begin{proof}
    By conjugacy, we only prove it for $\sigma_m$. But this is clear by taking:
    $$
      x=(0,1,\ldots,m-1,10,\ldots,1(m-1),20,\ldots,2(m-1),\ldots)
    $$
  \end{proof}
  \subsubsection{A hyperbolic automorphism of \texorpdfstring{$T^2$}{T2}}
  \begin{proposition}
    Consider $\vf{A}=\begin{pmatrix}
        2 & 1 \\
        1 & 1
      \end{pmatrix}\in \GL_2(\RR)$. Then, $\vf{A}(\ZZ^2)=\ZZ^2$ and this induces an automorphism $\vf{\tilde{A}}$ of $T^2=\quot{\RR^2}{\ZZ^2}$.
  \end{proposition}
  \begin{definition}
    We define the set of periodic points of $\vf{\tilde{A}}$ as $\Per\vf{\tilde{A}}$.
  \end{definition}
  \begin{lemma}
    $\Per\vf{\tilde{A}}=\quot{\QQ^2}{\ZZ^2}$. Thus, $\Per\vf{\tilde{A}}$ is dense in $T^2$.
  \end{lemma}
  \begin{proof}
    Let $\vf{x}\in \Per\vf{\tilde{A}}$. Then, $\exists k\in\NN$ and $\vf{n}\in\ZZ^2$ such that $\vf{A}^k\vf{x}=\vf{x}+\vf{n}$. One can easily check that $\sigma(\vf{\tilde{A}})=\left\{\frac{3}{2}\pm \frac{\sqrt{5}}{2}\right\}=:\{\lambda_{\pm}\}$ with $\lambda_-<1<\lambda_+$. Thus,
    $$
      \det(\vf{A}^k-\vf{I})=({\lambda_+}^k-1)({\lambda_-}^k-1)\ne 0
    $$
    and so the equation $\vf{A}^k\vf{x}=\vf{x}+\vf{n}$ has a unique (rational) solution.
  \end{proof}
  \begin{remark}
    The \emph{hyperbolicity} comes from the fact that there is one eigenvector with eigenvalue greater than $1$ and another with eigenvalue less than $1$.
  \end{remark}
  \begin{theorem}
    The iterates of $\vf{\tilde{A}}$ smear every domain $F\subseteq T^2$ uniformly over $T^2$, that is, for every domain $G\subseteq T^2$, we have that the following limit exists:
    $$
      \abs{(\vf{\tilde{A}}^{-n} F)\cap G}\overset{n\to\infty}{\longrightarrow} \abs{F}\abs{G}
    $$
    This property of $\vf{\tilde{A}}$ is called \emph{mixing}.
  \end{theorem}
  \begin{proof}
    We can prove a more general property in terms of functions in the torus (and then apply it to $f=\indi{F}$ and $g=\indi{G}$):
    $$
      \lim_{n\to\infty}\int_{T^2} f(\vf{\tilde{A}}^n \vf{x}) g(\vf{x})\dd{\vf{x}}=\int_{T^2} f(\vf{x})\dd{\vf{x}}\int_{T^2} g(\vf{x})\dd{\vf{x}}
    $$
    We will prove this for the orthonormal basis of Fourier series $\{\exp{2\pi i \vf{p}\cdot \vf{x}}\}_{\vf{p}\in\ZZ^2}$. Note that:
    $$
      \int_{T^2} \exp{2\pi i (\transpose{(\vf{\tilde{A}}^n)}\vf{p})\cdot \vf{x}}\dd{\vf{x}}=\begin{cases}
        1 & \text{if }\vf{p}=\vf{0}    \\
        0 & \text{if }\vf{p}\ne \vf{0}
      \end{cases}
    $$
    Therefore, since $\transpose{(\vf{\tilde{A}}^n)}\vf{p}$ takes infinitely many values for $\vf{p}\ne \vf{0}$, we have that if $g=\exp{2\pi i \vf{q} \cdot \vf{x}}$ then:
    $$
      \lim_{n\to\infty}\int_{T^2} \exp{2\pi i(\transpose{(\vf{\tilde{A}}^n)}\vf{p}+\vf{q})\cdot \vf{x}}\dd{\vf{x}}=0
    $$
    So for any $\vf{p}, \vf{q}\in\ZZ^2$ we have the equality. Then, we use that any function nice enough can be approximated with its Fourier series.
  \end{proof}
  \begin{theorem}
    On the torus $T^2$ there exist two direction fields invariant with respect to the automorphism $\vf{\tilde{A}}$. The integral curves of each of these directions fields are everywhere dense on the torus. The automorphism $\vf{\tilde{A}}$ converts the integral curves of each field into integral curves of the same field, expanding by $\lambda_+$ for the first field and contracting by $\lambda_-$ for the second.
  \end{theorem}
  \begin{proof}
    Let $\vf{e}_+$ and $\vf{e}_-$ be the eigenvectors of $\vf{A}$ with eigenvalues $\lambda_+$ and $\lambda_-$ respectively. Let $\vf{x}\in T^2$ and
    $$
      \function{\vf\gamma_+}{\RR}{T^2}{t}{\vf{x}+t \vf{e}_+}\quad
      \function{\vf\gamma_-}{\RR}{T^2}{t}{\vf{x}+t \vf{e}_-}
    $$
    be the expanding and contracting curves and let $\vf{\xi}_{\vf{x}}=\im(\vf\gamma_+)$, $\vf{\eta}_{\vf{x}}=\im(\vf\gamma_-)$.
  \end{proof}
  \begin{definition}
    Let $\vf{A},\vf{B}:T^2\rightarrow T^2$ be $\mathcal{C}^1$ functions. We say that $B$ is \emph{$\mathcal{C}^0$-close} to $\vf{A}$ if for all $\varepsilon>0$:
    $$
      \sup_{\vf{x}\in T^2}\norm{\vf{A}(\vf{x})-\vf{B}(\vf{x})}<\varepsilon
    $$
    We say that $\vf{B}$ is \emph{$\mathcal{C}^1$-close} to $\vf{A}$ if for all $\varepsilon>0$, $\vf{B}$ is $\mathcal{C}^0$-close to $\vf{A}$ and:
    $$
      \sup_{\vf{x}\in T^2}\norm{\vf{D}\vf{A}(\vf{x})-\vf{D}\vf{B}(\vf{x})}<\varepsilon
    $$
  \end{definition}
  \begin{theorem}[Structal stability]
    Let $\vf{B}$ be a diffeomorphism on $T^2$ $\mathcal{C}^1$-close to $\vf{\tilde{A}}$. Then, $\vf{B}$ is conjugate to $\vf{\tilde{A}}$.
  \end{theorem}
\end{multicols}
\end{document}