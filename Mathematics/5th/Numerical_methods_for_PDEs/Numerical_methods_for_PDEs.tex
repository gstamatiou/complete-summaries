\documentclass[../../../main_math.tex]{subfiles}

\begin{document}
\changecolor{NMPDE}
\begin{multicols}{2}[\section{Numerical methods for PDEs}]
  \subsection{Finite element method}
  \subsubsection{Variational formulation of elliptic PDEs}
  \begin{definition}
    Consider a elliptic pde of the form:
    \begin{equation}\label{NMPDE:elliptic_pde}
      -\sum_{i,j=1}^n\partial_j(a_{ij}\partial_i u)+\sum_{i=1}^n b_i\partial_i u+cu=f
    \end{equation}
    in a bounded open subset $\Omega\subset\mathbb{R}^n$ with $a_{ij}$, $b_i$, $c$ and $f$ sufficiently regular functions. The variational formulation of this problem is:
    \begin{multline*}
      \sum_{i,j=1}^n{\langle a_{ij}\partial_i u,\partial_j v\rangle}_{L^2(\Omega)}+\sum_{i=1}^n{\langle b_i\partial_i u,v\rangle}_{L^2(\Omega)}+{\langle cu,v\rangle}_{L^2(\Omega)}=\\={\langle f,v\rangle}_{L^2(\Omega)}+\sum_{i,j=1}^n{\langle a_{ij}\partial_i u\nu_j,v\rangle}_{L^2(\Fr{\Omega})}
    \end{multline*}
    where $v$ is a test function (sufficiently regular) and $\vf{\nu}$ is the outward unit normal vector on $\Fr\Omega\in\mathcal{C}^1$. The functions in the last inner product on $\Fr\Omega$ are meant to be in the sense of traces. From now on, the left hand side of the variational formulation will be denoted by $a(u,v)$ and $\ell(v):={\langle f,v\rangle}_{L^2(\Omega)}$.
  \end{definition}
  \begin{remark}
    This formulation makes sense for $a_{ij},b_i,c\in L^\infty(\Omega)$ and $f\in L^2(\Omega)$. We will look for a solution $u\in V$, where $V$ is a suitable space of functions, such that the variational formulation holds for all $v\in V$.
  \end{remark}
  \begin{definition}[Dirichlet boundary conditions]
    Consider \mcref{NMPDE:elliptic_pde} with Dirichlet boundary conditions $u=g$ on $\Fr\Omega$. If $g=0$, the boundary term disappears and we can choose $V=H^1_0(\Omega)$. If $g\neq 0$, and both $g$ and $\Fr\Omega$ are smooth (i.e.\ $g\in H^1(\Omega)$), by \mcref{ATFAPDE:trace_thm} we can find $u_g\in H^1(\Omega)$ such that $\Tr(u_g)=g$. Then, we set $\tilde{u}:=u-u_g\in H^1_0(\Omega)$ and satisfies:
    $$
      a(\tilde{u},v)=\ell(v)-a(u_g,v)\quad\forall v\in H^1_0(\Omega)
    $$
  \end{definition}
  \begin{definition}[Neumann boundary conditions]
    Consider \mcref{NMPDE:elliptic_pde} with Neumann boundary conditions $\sum_{i,j=1}^n a_{ij}\partial_i u\nu_j=g$ on $\Fr\Omega$, for $g\in L^2(\Fr\Omega)$. In this case, we take $V=H^1(\Omega)$ and look for a solution $u\in V$ such that:
    $$
      a(u,v)=\ell(v)+{\langle g,v\rangle}_{L^2(\Fr\Omega)}\quad\forall v\in V
    $$
  \end{definition}
  \begin{definition}[Robin boundary conditions]
    Consider \mcref{NMPDE:elliptic_pde} with Robin boundary conditions $\gamma u+\sum_{i,j=1}^n a_{ij}\partial_i u\nu_j=g$ on $\Fr\Omega$, for $g\in L^2(\Fr\Omega)$ and $\gamma\in L^\infty(\Fr\Omega)$. In this case, we take $V=H^1(\Omega)$ and look for a solution $u\in V$ such that:
    $$
      a(u,v)+{\langle \gamma u,v\rangle}_{L^2(\Fr\Omega)}=\ell(v)+{\langle g,v\rangle}_{L^2(\Fr\Omega)}\quad\forall v\in V
    $$
  \end{definition}
  \begin{remark}
    Recall that for these problems to have a unique solution, we need to impose the coercivity and continuity in \mnameref{RFA:laxmilgram}.
  \end{remark}
  \begin{proposition}
    Consider the homogeneous Dirichlet problem from \mcref{NMPDE:elliptic_pde} and set $\beta=\alpha^{-1}\sum_{i=1}^n{\norm{b_i}_{L^\infty(\Omega)}}^2$, where $\alpha$ is the ellipticity constant of the PDE. Then, the homogeneous Dirichlet problem has a unique solution $u$ in $H_0^1(\Omega)$ if $\forall x\in\Omega$ we have $c-\frac{\beta}{2}\geq 0$. In this case, $\exists C>0$ such that:
    $$
      \norm{u}_{H^1(\Omega)}\leq C\norm{f}_{L^2(\Omega)}
    $$
    Consequently, the non-homogeneous Dirichlet problem for $g\in H^1(\Fr\Omega)$ has a unique solution $u$ in $H^1(\Omega)$ such that:
    $$
      \norm{u}_{H^1(\Omega)}\leq \tilde{C}(\norm{f}_{L^2(\Omega)}+\norm{g}_{H^1(\Fr\Omega)})
    $$
  \end{proposition}
  \begin{proposition}
    Consider the Neumann problem from \mcref{NMPDE:elliptic_pde} for $g\in L^2(\Fr\Omega)$ and set $\beta=\alpha^{-1}\sum_{i=1}^n{\norm{b_i}_{L^\infty(\Omega)}}^2$, where $\alpha$ is the ellipticity constant of the PDE. Then, the Neumann problem has a unique solution $u$ in $H^1(\Omega)$ if $\forall x\in\Omega$ we have $c-\frac{\beta}{2}\geq\delta> 0$. In this case, $\exists C>0$ such that:
    $$
      \norm{u}_{H^1(\Omega)}\leq C(\norm{f}_{L^2(\Omega)}+\norm{g}_{L^2(\Fr\Omega)})
    $$
  \end{proposition}
  \begin{proposition}
    Consider the Robin problem from \mcref{NMPDE:elliptic_pde} for $g\in L^2(\Fr\Omega)$ and $\gamma\in L^\infty(\Fr\Omega)$ and set $\beta=\alpha^{-1}\sum_{i=1}^n{\norm{b_i}_{L^\infty(\Omega)}}^2$, where $\alpha$ is the ellipticity constant of the PDE. Then, the Robin problem has a unique solution $u$ in $H^1(\Omega)$ if $\forall x\in\Omega$ we have $c-\frac{\beta}{2}\geq\delta\geq 0$ and $\gamma\geq \eta\geq 0$ with either $\delta>0$ or $\eta>0$. In this case, $\exists C>0$ such that:
    $$
      \norm{u}_{H^1(\Omega)}\leq C(\norm{f}_{L^2(\Omega)}+\norm{g}_{L^2(\Fr\Omega)})
    $$
  \end{proposition}
  \subsubsection{Galerkin method}
  \begin{definition}
    The \emph{conforming Galerkin method} consists in looking for a solution $u_h\in V_h\subset V$ such that:
    $$
      a_h(u_h,v_h)=\ell_h(v_h)\quad\forall v_h\in V_h
    $$
    where $V_h$ is a closed finite-dimensional subspace of $V$ and $a_h$ and $\ell_h$ are the approximations of $a$ and $\ell$ in $V_h$.
  \end{definition}
  \begin{remark}
    Here $h$ is meant to be a discretization parameter.
  \end{remark}
  \begin{theorem}[Céa's lemma]
    Let $u\in V$ be the solution of the variational formulation of \mcref{NMPDE:elliptic_pde} and $u_h\in V_h$ be the solution of the Galerkin method. Then, $\exists C>0$ such that:
    $$
      \norm{u-u_h}_V\leq C\inf_{v_h\in V_h}\norm{u-v_h}_V
    $$

  \end{theorem}
  \subsubsection{Finite element spaces}
  \begin{remark}
    Finite element methods are a special case of Galerkin methods in which the finite-dimensional subspace $V_h$ consists of piecewise polynomial functions over a mesh.
  \end{remark}
  \begin{definition}
    A \emph{finite element} is a triplet $(K,\mathcal{P},\mathcal{N})$ where:
    \begin{itemize}
      \item $K\subseteq\RR^n$ is a simply connected bounded open set with piecewise smooth boundary (the \textit{geometric} element).
      \item $\mathcal{P}$ is a finite-dimensional space of functions defined on $K$, whose elements are called \emph{basis functions}.
      \item $\mathcal{N}=\{N_1,\dots,N_d\}$ is a basis of $\mathcal{P}^*$.
    \end{itemize}
  \end{definition}
  \begin{lemma}
    Let $\mathcal{P}$ be $d$-dimensional vector space and let $\{N_1,\dots,N_d\}\subset \mathcal{P}^*$. The following statements are equivalent:
    \begin{enumerate}
      \item $\{N_1,\dots,N_d\}$ is a basis of $\mathcal{P}^*$.
      \item For all $v\in \mathcal{P}$ such that $N_i(v)=0$ for all $i=1,\dots,d$, then $v=0$.
    \end{enumerate}
  \end{lemma}
  \begin{remark}
    To construct a finite element, we usually proceed as follows:
    \begin{enumerate}
      \item We choose a geometric element $K$.
      \item We choose a polynomial space $\mathcal{P}$ up to a given degree $k$.
      \item We choose the degrees of freedom $\mathcal{N}=\{N_1,\dots,N_d\}$, where $d=\dim \mathcal{P}$, such that the corresponding interpolation problem is well-posed.
      \item We compute the dual basis of $\mathcal{N}$, which gives a basis of $\mathcal{P}$.
    \end{enumerate}
  \end{remark}
  \begin{definition}
    Let $(K,\mathcal{P},\mathcal{N})$ be a finite element and $\{\psi_1,\ldots,\psi_d\}$ be the corresponding basis of $\mathcal{P}$. For a given function $v$ such that $N_i(v)$ is defined $\forall i\in\{1,\ldots,d\}$, we define the \emph{local interpolant} of $v$ as:
    $$
      I_Kv:=\sum_{i=1}^d N_i(v)\psi_i
    $$
  \end{definition}
  \begin{lemma}
    Let $(K,\mathcal{P},\mathcal{N})$ be a finite element and $I_K$ be the local interpolant operator associated to it. Then, the following properties hold:
    \begin{enumerate}
      \item $I_K$ is linear.
      \item $N_i(I_Kv)=N_i(v)$ $\forall i\in\{1,\ldots,d\}$.
      \item $I_Kv=v$ $\forall v\in \mathcal{P}$, i.e.\ $I_K$ is a projection.
    \end{enumerate}
  \end{lemma}
  \begin{definition}
    A \emph{subdivision} of a bounded open set $\Omega\subset \RR^n$ is a collection $\mathcal{T}$ of open sets $K_i$ such that:
    \begin{enumerate}
      \item $K_i\cap K_j=\varnothing$ $\forall i\neq j$.
      \item $\overline{\Omega}= \bigcup_{K\in\mathcal{T}}\overline{K}$.
    \end{enumerate}
  \end{definition}
  \begin{definition}
    Let $\mathcal{T}$ be a subdivision of $\Omega$ such that for each $K\in\mathcal{T}$ there exists a finite element $(K,\mathcal{P},\mathcal{N})$ with local interpolant $I_K$. Let $m$ be the order of the highest partial derivative appearing in any of the degrees of freedom of $\mathcal{N}$. We define the \emph{global interpolant} $I_\mathcal{T}v$ of $\mathcal{T}$, for $v\in \mathcal{C}^m(\overline{\Omega})$, as:
    $$
      I_\mathcal{T}v|_K:=I_Kv\quad \forall K\in\mathcal{T}
    $$
  \end{definition}
  \begin{definition}
    A \emph{triangulation} of a bounded open set $\Omega\subset \RR^2$ is a subdivision $\mathcal{T}$ of $\Omega$ such that:
    \begin{enumerate}
      \item Each $K\in\mathcal{T}$ is a triangle.
      \item The intersection of two triangles is either empty or a common vertex or a common edge.
    \end{enumerate}
  \end{definition}
  \begin{definition}
    Let $(\widehat{K}, \widehat{\mathcal{P}}, \widehat{\mathcal{N}})$, $(K,\mathcal{P},\mathcal{N})$ be finite elements and $T:\RR^n\to\RR^n$ be an affine transformation. We say that these finite elements are \emph{affinely equivalent} by $T$ if:
    \begin{enumerate}
      \item $K=T(\widehat{K})$.
      \item $\mathcal{P}=\{\widehat{p}\circ T^{-1}:\widehat{p}\in\widehat{\mathcal{P}}\}$.
      \item $\mathcal{N}=\{N_i\}$, where $N_i(p)=\widehat{N}_i(p\circ T)$ $\forall p\in\mathcal{P}$.
    \end{enumerate}
  \end{definition}
  \begin{lemma}
    Let $(\widehat{K}, \widehat{\mathcal{P}}, \widehat{\mathcal{N}})$, $(K,\mathcal{P},\mathcal{N})$ be two affine equivalent finite elements by the affine transformation $\vf{T}_K$. Then:
    $$
      I_{\widehat{K}}(v\circ \vf{T}_K)=I_Kv\circ T
    $$
  \end{lemma}
  \subsubsection{Polygonal interpolation in Sobolev spaces}
  \begin{lemma}[Bramble-Hilbert lemma]
    Let $F:W^{k,p}(\Omega)\to\RR$ be such that:
    \begin{enumerate}
      \item $\abs{F(v)}\leq c_1\abs{v}_{W^{k,p}(\Omega)}$ $\forall v\in W^{k,p}(\Omega)$, where $$
              \abs{v}_{W^{k,p}(\Omega)}:=
              \begin{cases}
                {\left({\sum_{\abs{\alpha}= k}\norm{\partial^\alpha v}_{L^p(\Omega)}}\right)}^{1/p} & \text{if }p<\infty \\
                \max_{\abs{\alpha}= k}\norm{\partial^\alpha v}_{L^\infty(\Omega)}                   & \text{if }p=\infty
              \end{cases}
            $$
      \item $\abs{F(u+v)}\leq c_2(\abs{F(u)}+\abs{F(v)})$ $\forall u,v\in W^{k,p}(\Omega)$.
      \item $\abs{F(q)}=0$ $\forall q\in\mathcal{P}_{k-1}(\Omega)$, where $\mathcal{P}_{\ell}(\Omega)$ is the space of polynomials of degree less than $\ell$.
    \end{enumerate}
    Then, $\exists C>0$ such that $\forall v\in W^{k,p}(\Omega)$:
    $$
      \abs{F(v)}\leq C\abs{v}_{W^{k,p}(\Omega)}
    $$
  \end{lemma}
  \begin{theorem}
    Let $(K,\mathcal{P},\mathcal{N})$ be a finite element such that $\mathcal{P}_{k-1}\subseteq \mathcal{P}$ for some $k\in\NN$ and all $N\in\mathcal{N}$ be bounded in $W^{k,p}(K)$ for some $p\in[1,\infty]$. Then, $\exists C>0$ such that $\forall v\in W^{k,p}(K)$:
    $$
      \abs{v-I_Kv}_{W^{\ell,p}(K)}\leq C\abs{v}_{W^{k,p}(K)} \quad\forall \ell\in\{0,\ldots,k\}
    $$
  \end{theorem}
  \begin{remark}
    Let $(K,\mathcal{P},\mathcal{N})$ be a finite element and $(\widehat{K}, \widehat{\mathcal{P}}, \widehat{\mathcal{N}})$ be the reference element. From now on, if they are affine equivalent by $\vf{T}_K:\widehat{K}\to K$, we will assume that $\vf{T}_K\widehat{x}=\vf{A}_K \widehat{x}+\vf{b}_K$, with $\vf{A}_K$ invertible.
  \end{remark}
  \begin{lemma}
    Let $k\in \NN$ and $p\in[1,\infty]$. Then, $\exists C>0$ such that $\forall K\subset\Omega$ and $\forall v\in W^{k,p}(\widehat{K})$:
    \begin{align*}
      \abs{v}_{W^{k,p}(\widehat{K})} & \leq C\norm{\vf{A}_K}^k\abs{\det \vf{A}_K}^{-1/p}\abs{v}_{W^{k,p}(K)}                 \\
      \abs{v}_{W^{k,p}(K)}           & \leq C\norm{{\vf{A}_K}^{-1}}^k\abs{\det \vf{A}_K}^{1/p}\abs{v}_{W^{k,p}(\widehat{K})}
    \end{align*}
  \end{lemma}
  \begin{definition}
    Let $(K,\mathcal{P},\mathcal{N})$ be a finite element. We define the \emph{diameter} of $K$ as:
    $$
      h_K:=\max_{x,y\in K}\norm{x-y}
    $$
    We define the \emph{insphere diameter} of $K$ as:
    $$
      \rho_K:=2\max\{\rho>0:B(x,\rho)\subset K\text{ for some }x\in K\}
    $$
    We define the \emph{condition number} of $K$ as $\sigma_K:=\frac{h_K}{\rho_K}$.
  \end{definition}
  \begin{lemma}
    Let $(K,\mathcal{P},\mathcal{N})$, $(\widehat{K}, \widehat{\mathcal{P}}, \widehat{\mathcal{N}})$ be affine equivalent finite elements by $\vf{T}_K:\widehat{K}\to K$. Then, $\abs{\det \vf{A}_K}=\frac{\vol(K)}{\vol(\widehat{K})}$, $\norm{\vf{A}_K}\leq \frac{h_K}{\rho_{\widehat{K}}}$ and $\norm{{\vf{A}_K}^{-1}}\leq \frac{h_{\widehat{K}}}{\rho_K}$.
  \end{lemma}
  \begin{theorem}[Local interpolation error]
    Let $(\widehat{K}, \widehat{\mathcal{P}}, \widehat{\mathcal{N}})$ be a finite element with $\mathcal{P}_{k-1}\subseteq \mathcal{P}$ for some $k\in\NN$ and all $N\in\mathcal{N}$ be bounded in $W^{k,p}(\widehat{K})$ for some $p\in[1,\infty]$. Then, for all finite element $(K,\mathcal{P},\mathcal{N})$ affine equivalent to $(\widehat{K}, \widehat{\mathcal{P}}, \widehat{\mathcal{N}})$ by $\vf{T}_K:\widehat{K}\to K$, $\exists C>0$ (independent of $K$) such that $\forall v\in W^{k,p}(K)$:
    $$
      \abs{v-I_Kv}_{W^{\ell,p}(K)}\leq C{h_K}^k{\sigma_K}^{\ell}\abs{v}_{W^{k,p}(K)} \quad\forall \ell\in\{0,\ldots,k\}
    $$
  \end{theorem}
  \begin{definition}
    A subdivision $\mathcal{T}$ of $\Omega\in \RR^n$ is called \emph{regular} if $\exists C>0$ such that $\forall K\in\mathcal{T}$ we have $\sigma_K\leq C$.
  \end{definition}
  \begin{theorem}[Global interpolation error]
    Let $\mathcal{T}$ be a regular subdivision of $\Omega\in \RR^n$ and $(\widehat{K}, \widehat{\mathcal{P}}, \widehat{\mathcal{N}})$ be a reference finite element with $\mathcal{P}_{k-1}\subseteq \mathcal{P}$ for some $k\in\NN$ and all $N\in\mathcal{N}$ be bounded in $W^{k,p}(\widehat{K})$ for some $p\in[1,\infty]$. Let $h:= \max_{K\in\mathcal{T}}h_K$. Then, $\exists C>0$ (independent of $h$) such that $\forall v\in W^{k,p}(\Omega)$:
    \begin{multline*}
      \abs{v-I_\mathcal{T}v}_{W^{\ell,p}(\Omega)}+\sum_{\ell=1}^k\left(h^\ell\sum_{K\in\mathcal{T}} \abs{v-I_Kv}_{W^{\ell,p}(K)}^p\right)^{1/p}\leq \\\leq C h^k\abs{v}_{W^{k,p}(\Omega)}
    \end{multline*}
    if $p<\infty$ and:
    \begin{multline*}
      \abs{v-I_\mathcal{T}v}_{W^{\ell,\infty}(\Omega)}+\sum_{\ell=1}^kh^\ell \max_{K\in\mathcal{T}}\abs{v-I_Kv}_{W^{\ell,\infty}(K)}\leq\\\leq C h^k\abs{v}_{W^{k,\infty}(\Omega)}
    \end{multline*}
    if $p=\infty$.
  \end{theorem}
  \subsubsection{Error estimates for finite element approximation}
  \begin{theorem}
    Let $\Omega\subset\RR^n$ be open and bounded, $u\in H^1(\Omega)$ be the solution of the boundary value problem and $\mathcal{T}$ be a regular triangulation of $\Omega$ with reference element $(\widehat{K}, \widehat{\mathcal{P}}, \widehat{\mathcal{N}})$ such that $\mathcal{P}_{k-1}\subseteq \mathcal{P}$ for some $k\in\NN$. Let $u_h\in V_h$ be the solution of the Galerkin method. Then, if $u\in H^m$, with $\frac{n}{2}< m<k$, then $\exists C>0$ (independent of $h$ and $u$) such that:
    $$
      \norm{u-u_h}_{H^1(\Omega)}\leq C h^{m-1}\norm{u}_{H^m(\Omega)}
    $$
  \end{theorem}
  \subsection{Spectral methods}
  \begin{definition}\label{NMPDE:spectral_method}
    The idea of \emph{spectral methods} is to approximate the solution $u$ of a boundary value problem by an expression in terms of the so-called \emph{trial functions}:
    $$
      u(x)\approx \sum_{i=1}^N \tilde{u}_i\phi_i(x)
    $$
    We will impose the following requirements on the trial functions:
    \begin{enumerate}
      \item The approximation should converge rapidly as $N\to\infty$.
      \item The computation of the coefficients $\tilde{u}_i$ and the reconstruction of $u$ should be efficient.
      \item Given the coefficients of some function $u$, it should be easy to determine the coefficients of the derivative of $u$.
    \end{enumerate}
  \end{definition}
  \begin{remark}
    When using spectral methods, it is generally assumed that the solution of the problem of interest is very smooth, and thus, the trial functions are \emph{globally smooth}, i.e.\ algebraic or trigonometric polynomials.
  \end{remark}
  \begin{definition}
    The choice of the test functions distinguishes between three types of spectral methods:
    \begin{enumerate}
      \item \emph{Galerkin methods}: the test functions are the same as the trial functions. These test functions usually satisfy some or all the boundary conditions. The PDE is enforced by requiring that the residual is orthogonal to the test functions.
      \item \emph{Collocation methods}: the test functions are Dirac delta distributions centered at the so-called \emph{collocation points}. This approach requires the PDE to be satisfied exactly at the collocation points. A supplementary set of equations may be imposed to satisfy the boundary conditions.
      \item \emph{$\tau$ methods}: these are similar to the Galerkin methods in the way the PDE is enforced, but the test functions don't need to satisfy the boundary conditions.
    \end{enumerate}
  \end{definition}
  \begin{remark}
    From what follows, we will focus collocation methods.
  \end{remark}
  \subsubsection{Periodic problem}
  When considering a problem with periodic boundary conditions, we can use trigonometric polynomials as trial functions and Fourier analysis
  \begin{remark}
    It can be seen that the Fourier basis functions $\{ \exp{\ii k x}:k\in\ZZ\}$ and it's coefficients satisfy the requirements of \mcref{NMPDE:spectral_method}, using the FFT for the computation of the coefficients.
  \end{remark}

\end{multicols}
\end{document}
