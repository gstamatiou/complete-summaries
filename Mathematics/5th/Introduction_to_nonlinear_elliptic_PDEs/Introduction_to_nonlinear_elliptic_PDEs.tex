\documentclass[../../../main_math.tex]{subfiles}

\begin{document}
\changecolor{INLEPDE}
\begin{multicols}{2}[\section{Introduction to nonlinear elliptic PDEs}]
  \subsection{Introduction}
  \begin{definition}
    Let $a_{ij}$, $b_j$, $c$, $f$ be known scalar functions defined on $\Omega\subseteq \RR^d$. Usually we will denote $\vf{A}=(a_{ij})$ and $\vf{b}=(b_j)$. A \emph{linear second-order PDE} is an equation of the form:
    \begin{equation*}
      -\sum_{i,j=1}^da_{ij}(\vf{x})\partial_{ij}^2u(\vf{x})+\sum_{j=1}^db_j(\vf{x})\partial_ju(\vf{x})+c(\vf{x})u(\vf{x})=f(\vf{x})
    \end{equation*}
    where $u:\Omega\to \RR$ is the unknown function. This form is called \emph{non-divergence form}. If we write the equation in the form:
    \begin{multline*}
      -\sum_{i=1}^d\pdv{}{x_i}\left(\sum_{j=1}^da_{ij}(\vf{x})\partial_ju(\vf{x})\right)+\sum_{j=1}^db_j(\vf{x})\partial_ju(\vf{x})+\\+c(\vf{x})u(\vf{x})=f(\vf{x})
    \end{multline*}
    then we say that the equation is in \emph{divergence form}. Together with the PDE we usually impose boundary conditions on $\partial\Omega$. The \emph{Dirichlet boundary condition} is:
    $$
      u|_{\partial\Omega}=g
    $$
    and it is called \emph{homogeneous} if $g=0$. The \emph{Neumann boundary condition} is:
    $$
      \langle \vf{n},\vf{A} \grad u\rangle|_{\partial\Omega}=g
    $$
    where we have assumed that the boundary of $\Omega$ is smooth enough to define the normal vector $\vf{n}$. The condition is called \emph{homogeneous} if $g=0$. Note that if $\vf{A}=\vf{I}_d$, then the Neumann boundary condition is just $\partial_{\vf{n}} u=g$.
  \end{definition}
  \begin{definition}
    Let $a_{ij},b_j,c$ be known functions on $\Omega\subseteq \RR^d$. We say that the operator $$L=-\sum_{i,j=1}^da_{ij}\partial_{ij}^2 + \sum_{j=1}^d b_j\partial_j+c$$ is \emph{uniformly elliptic} if there exists $\theta>0$ such that for all $x\in \Omega$ and all $p\in \RR^d$ we have:
    \begin{equation}
      Q_x(\vf{p})=\sum_{i,j=1}^da_{ij}(\vf{x})p_ip_j\geq \theta \sum_{i=1}^{d} {p_i}^2
    \end{equation}
  \end{definition}
  \begin{remark}
    Geometrically speaking, this implies that the sets
    $$
      \xi_{x,h}=\{ \vf{p}\in \RR^d: Q_x(\vf{p})=h\}
    $$
    are ellipsoids.
  \end{remark}
  \begin{proposition}
    Let $H$ be Hilbert and $K:H\to H$ be a continuous linear operator. Then, the following are equivalent:
    \begin{enumerate}
      \item $K$ is compact.
      \item For any bounded sequence $(u_n)\in H$, the sequence $(Ku_n)$ has a convergent subsequence.
      \item For any sequence $(u_n)\in H$ such that $u_n\rightharpoonup u$, we have $Ku_n\to Ku$.
    \end{enumerate}
  \end{proposition}
  \subsection{Hilbert space methods for divergence form linear PDEs}
  In this section, we will assume that $\Omega\subset\RR^d$ is an open, bounded subset, $a_{ij}=a_{ji}$ and $a_{ij},b_j,c\in L^\infty(\Omega)$.
  \subsubsection{Fredholm alternative}
  \begin{theorem}[Abstract Fredholm alternative]
    Let $H$ be Hilbert and $K:H\to H$ be a compact linear operator. Then:
    \begin{enumerate}
      \item $\ker(\id-K)$ and $\ker(\id-K^*)$ are both finite dimensional and they have the same dimension.
      \item $\im(\id-K)={(\ker(\id-K^*))}^\perp$. In particular, $\im(\id-K)$ is closed.
      \item Either $\ker(\id-K)\ne\{0\}$ or $\id -K$ is and isomorphism.
    \end{enumerate}
  \end{theorem}
\end{multicols}
\end{document}