\documentclass[../../../main_math.tex]{subfiles}

\begin{document}
\changecolor{INLEPDE}
\begin{multicols}{2}[\section{Introduction to nonlinear elliptic PDEs}]
  \subsection{Introduction}
  \begin{definition}
    Let $a_{ij}$, $b_j$, $c$, $f$ be known scalar functions defined on $\Omega\subseteq \RR^d$. Usually we will denote $\vf{A}=(a_{ij})$ and $\vf{b}=(b_j)$. A \emph{linear second-order PDE} is an equation of the form:
    \begin{equation*}
      -\sum_{i,j=1}^da_{ij}(\vf{x})\partial_{ij}^2u(\vf{x})+\sum_{j=1}^db_j(\vf{x})\partial_ju(\vf{x})+c(\vf{x})u(\vf{x})=f(\vf{x})
    \end{equation*}
    where $u:\Omega\to \RR$ is the unknown function. This form is called \emph{non-divergence form}. If we write the equation in the form:
    \begin{multline*}
      -\sum_{i=1}^d\pdv{}{x_i}\left(\sum_{j=1}^da_{ij}(\vf{x})\partial_ju(\vf{x})\right)+\sum_{j=1}^db_j(\vf{x})\partial_ju(\vf{x})+\\+c(\vf{x})u(\vf{x})=f(\vf{x})
    \end{multline*}
    then we say that the equation is in \emph{divergence form}. Together with the PDE we usually impose boundary conditions on $\partial\Omega$. The \emph{Dirichlet boundary condition} is:
    $$
      u|_{\partial\Omega}=g
    $$
    and it is called \emph{homogeneous} if $g=0$. The \emph{Neumann boundary condition} is:
    $$
      \langle \vf{n},\vf{A} \grad u\rangle|_{\partial\Omega}=g
    $$
    where we have assumed that the boundary of $\Omega$ is smooth enough to define the normal vector $\vf{n}$. The condition is called \emph{homogeneous} if $g=0$. Note that if $\vf{A}=\vf{I}_d$, then the Neumann boundary condition is just $\partial_{\vf{n}} u=g$.
  \end{definition}
  \begin{remark}
    If the coefficients $a_{ij}\in\mathcal{C}^1$, then we convert the equation from non-divergence form to divergence form and viceversa.
  \end{remark}
  \begin{definition}
    Let $a_{ij},b_j,c$ be known functions on $\Omega\subseteq \RR^d$. We say that the operator
    \begin{equation}\label{INEPDE:operator}
      L=-\sum_{i,j=1}^da_{ij}\partial_{ij}^2 + \sum_{j=1}^d b_j\partial_j+c
    \end{equation}
    is \emph{uniformly elliptic} if there exists $\theta>0$ such that for all $x\in \Omega$ and all $\vf{p}\in \RR^d$ we have:
    \begin{equation*}
      Q_x(\vf{p}):=\transpose{\vf{p}}\vf{A}(\vf{x})\vf{p}=\!\sum_{i,j=1}^da_{ij}(\vf{x})p_ip_j\geq \theta \sum_{i=1}^{d} {p_i}^2=\theta \norm{\vf{p}}^2
    \end{equation*}
  \end{definition}
  \begin{remark}
    Geometrically speaking, this implies that the sets
    $$
      \xi_{x,h}=\{ \vf{p}\in \RR^d: Q_x(\vf{p})=h\}
    $$
    are ellipsoids.
  \end{remark}
  \begin{definition}
    Consider the problem
    $$
      \mathcal{D}_f:=\begin{cases}
        Lu=f & \text{in }\Omega         \\
        u=0  & \text{on }\partial\Omega
      \end{cases}
    $$
    where $L$ is as in \mcref{INEPDE:operator}. The \emph{weak formulation} (or \emph{variational formulation}) of the problem is:
    \begin{equation*}
      {\langle \grad u,\grad v\rangle}_2+{\langle \vf{b}\cdot \grad u,v\rangle}_2 + {\langle cu,v\rangle}_2={\langle f,v\rangle}_2\quad \forall v\in H_0^1(\Omega)
    \end{equation*}
    A solution of such problem is called a \emph{weak solution} of $\mathcal{D}_f$.
  \end{definition}
  \begin{definition}
    If the weak solution $u_f$ of the problem $\mathcal{D}_f$ is in $H^1_0(\Omega)\cap W^{2,p}(\Omega)$ for some $p\in [1,\infty)$, then $u_f$ is called a \emph{strong solution} of $\mathcal{D}_f$. If $u_f\in \mathcal{C}^2(\Omega)\cap H^1_0(\Omega)$, then we say that $u_f$ is a \emph{classical solution} of $\mathcal{D}_f$.
  \end{definition}
  \begin{proposition}
    Let $H$ be Hilbert and $K:H\to H$ be a continuous linear operator. Then, the following are equivalent:
    \begin{enumerate}
      \item $K$ is compact.
      \item For any bounded sequence $(u_n)\in H$, the sequence $(Ku_n)$ has a convergent subsequence.
      \item For any sequence $(u_n)\in H$ such that $u_n\rightharpoonup u$, we have $Ku_n\to Ku$.
    \end{enumerate}
  \end{proposition}
  \subsection{Hilbert space methods for divergence form linear PDEs}
  In this section, we will assume that $\Omega\subset\RR^d$ is an open, bounded subset, $a_{ij}=a_{ji}$ and $a_{ij},b_j,c\in L^\infty(\Omega)$.
  \subsubsection{Lax-Milgram theorem}
  \begin{remark}
    Instead of the usual norm for $H_0^1(\Omega)$, here we will use the following one:
    $$
      \norm{u}_{H_0^1(\Omega)}^2=\norm{\grad u}_{L^2(\Omega)}^2
    $$
  \end{remark}
  \begin{definition}
    Let $H$ be a Hilbert space and $a:H\times H\rightarrow\RR$ be a bilinear map. We say that $a$ is \emph{continuous} if $\exists C>0$ such that $\forall u,v\in H$ we have: $$\abs{a(u,v)}\leq C\norm{u}\norm{v}$$
  \end{definition}
  \begin{definition}
    Let $H$ be a Hilbert space and $a:H\times H\rightarrow\RR$ be a bilinear map. We say that $a$ is \emph{coercive} if $\exists\alpha>0$ such that $\forall u\in H$ we have: $$a(u,u)\geq\alpha\norm{u}^2$$
  \end{definition}
  \begin{definition}
    Let $H$ be a Hilbert space and $a:H\times H\rightarrow\CC$ be a bilinear map. We say that $a$ is \emph{symmetric} if $\forall u,v\in H$ we have: $$a(u,v)=\overline{a(v,u)}$$
  \end{definition}
  \begin{theorem}[Lax-Milgram theorem]\label{INEPDE:laxmilgram}
    Let $H$ be a Hilbert space and $a:H\times H\rightarrow\RR$ be a continuous and coercive bilinear map. Then, $\forall f\in H^*$ $\exists! u_f\in H$ such that: $$a(u_f,v)=f(v)\quad \forall v\in H$$
    In addition, if ${H}$ is a real Hilbert space and $a$ is symmetric, then $u$ is the unique minimizer of:
    $$\min_{v\in H}\left\{\frac{1}{2}a(v,v)-f(v)\right\}$$
  \end{theorem}
  \begin{proposition}
    Consider the problem:
    $$
      \begin{cases}
        L u=f & \text{in }\Omega         \\
        u=0   & \text{on }\partial\Omega
      \end{cases}
    $$
    with $L=-\sum_{i,j=1}^d\partial_i(a_{ij}\partial_j)$ and $f\in L^2(\Omega)$. Then, the problem has a unique weak solution $u\in H_0^1(\Omega)$ and
    $$
      \norm{u}_{H_0^1(\Omega)}\leq C\norm{f}_{L^2(\Omega)}
    $$
  \end{proposition}
  \begin{proof}
    Consider the bilinear form $$
      a(u,v):=\int_\Omega\sum_{i,j=1}^da_{ij}\partial_iu\partial_jv
    $$
    We check the hypotheses of \mnameref{INEPDE:laxmilgram}:
    \begin{enumerate}
      \item $a$ is continuous:
            \begin{align*}
              \abs{a(u,v)} & \leq \sum_{i,j=1}^d\norm{a_{ij}}_{\infty}\norm{\grad u}_{2}\norm{\grad v}_{2} \\
                           & \leq C\norm{u}_{H_0^1(\Omega)}\norm{v}_{H_0^1(\Omega)}
            \end{align*}
      \item $a$ is coercive:
            \begin{align*}
              a(u,u) & =\int_\Omega\sum_{i,j=1}^da_{ij}\partial_iu\partial_ju   \\
                     & \geq \theta \int_\Omega\sum_{i=1}^d{\abs{\partial_iu}}^2 \\
                     & =\theta {\norm{u}_{H_0^1(\Omega)}}^2
            \end{align*}
            by the uniform ellipticity of $L$ and the \mnameref{ATFAPDE:poincare_ineq}.
    \end{enumerate}
    Moreover, since $ a(u,u)={\langle f,u\rangle}_2$ we have that:
    \begin{align*}
      \theta {\norm{u}_{H_0^1(\Omega)}}^2\leq {\langle f,u\rangle}_2\leq \norm{f}_2\norm{u}_{2}\leq C \norm{f}_2\norm{u}_{H_0^1(\Omega)}
    \end{align*}
    again by the \mnameref{ATFAPDE:poincare_ineq}.
  \end{proof}
  \subsubsection{Abstract Fredholm alternative}
  \begin{remark}
    One can check that if we try to apply \mnameref{INEPDE:laxmilgram} to the problem:
    $$
      \begin{cases}
        Lu=f & \text{in }\Omega         \\
        u=0  & \text{on }\partial\Omega
      \end{cases}
    $$
    with $L=-\sum_{i,j=1}^d\partial_i(a_{ij}\partial_j)+\sum_{j=1}^db_j\partial_j$, it fails due to the coercivity condition.
  \end{remark}
  \begin{proposition}
    Consider the problem:
    $$
      \begin{cases}
        L_\mu u=f & \text{in }\Omega         \\
        u=0       & \text{on }\partial\Omega
      \end{cases}
    $$
    with $L_\mu=-\sum_{i,j=1}^d\partial_i(a_{ij}\partial_j)+\sum_{j=1}^db_j\partial_j+\mu c$. Then, if $\mu>0$ is large enough, the problem has a unique weak solution in $H_0^1(\Omega)$
  \end{proposition}
  \begin{sproof}
    Taking the natural bilinear map $a$, the coercivity condition becomes:
    $$
      a_\mu(u,u)\geq \theta \norm{u}_{H_0^1(\Omega)}^2-C\norm{u}_{H_0^1(\Omega)}\norm{u}_2 + \mu \norm{u}_2^2
    $$
    which is for $\mu$ large enough it is bigger than $\delta \norm{u}_{H_0^1(\Omega)}^2$ for some $\delta>0$.
  \end{sproof}
  \begin{lemma}\label{INEPDE:lemma1_fredholm}
    Let $H$ be Hilbert and $K:H\to H$ be a compact linear operator. Then, $\dim \ker(\id-K)<\infty$.
  \end{lemma}
  \begin{proof}
    We first prove that $\dim\ker(\id-K)<\infty$. If $\dim\ker(\id-K)=\infty$, then $\exists (u_n)\in \ker(\id-K)$ orthonormal, and thus bounded. In particular, $u_n=Ku_n$ and since $K$ is compact, we have that $(Ku_n)$ has a convergent subsequence. But:
    \begin{align*}
      0 & =\lim_{k\to\infty}\norm{Ku_{n_k}-Ku_{n_{k+1}}}^2        \\
        & =\lim_{k\to\infty}\norm{u_{n_k}-u_{n_{k+1}}}^2          \\
        & =\lim_{k\to\infty}\norm{u_{n_k}}^2+\norm{u_{n_{k+1}}}^2 \\
        & =2
    \end{align*}
    by \mnameref{RFA:pythagorean}.
  \end{proof}
  \begin{lemma}\label{INEPDE:lemma2_fredholm}
    Let $H$ be Hilbert and $K:H\to H$ be a compact linear operator. Then, $\exists c>0$ such that $\forall u\in {\ker(\id-K)}^{\perp}$ we have $\norm{u-Ku}\geq c\norm{u}$.
  \end{lemma}
  \begin{proof}
    We proceed by contradiction. Suppose we have a sequence $(u_n)\in {\ker(\id-K)}^{\perp}$ with $\norm{u_n}=1$ such that $\norm{u_n-Ku_n}\to 0$. Since $(u_n)$ is bounded, we have that $(u_n)$ has a weakly convergent subsequence $(u_{n_k})$ to $u\in H$. Since $K$ is compact, we have that $Ku_{n_k}\to Ku$, and thus by continuity of the norm, $u=Ku$. Thus $u\in \ker(\id-K)$ and $u\in {\ker(\id-K)}^{\perp}$, which implies $u=0$, a contraction with $\norm{u}=1$.
  \end{proof}
  \begin{lemma}\label{INEPDE:lemma3_fredholm}
    Let $H$ be Hilbert and $K:H\to H$ be a compact linear operator. Then, $\im(\id-K)$ is closed.
  \end{lemma}
  \begin{proof}
    Let $(v_n)\in \im(\id-K)$ be such that $v_n\to v\in H$. Then, $\exists (u_n)\in H$ such that $v_n=(\id-K)u_n$. Thanks to \mnameref{RFA:projection}, we can write $u_n=u_n^{\text{ker}}+ u_n^{\text{ker}^\perp}$, where $u_n^{\text{ker}}\in \ker(\id-K)$ and $u_n^{\text{ker}^\perp}\in {\ker(\id-K)}^{\perp}$. Thus, $v_n=(\id-K)u_n^{\text{ker}^\perp}$ and by \mcref{INEPDE:lemma2_fredholm}, we have:
    $$
      \norm{v_n-v_m}\geq c\norm{u_n^{\text{ker}^\perp}-u_m^{\text{ker}^\perp}}
    $$
    Since $(v_n)$ is Cauchy, so it is $(u_n^{\text{ker}^\perp})$, and thus $(u_n^{\text{ker}^\perp})$ converges to some $u\in {\ker(\id-K)}^{\perp}$. Thus, $v=(\id-K)u\in \im(\id-K)$.
  \end{proof}
  % \begin{lemma}
  %   Let $H$ be Hilbert and $K:H\to H$ be a compact linear operator. Then, $\ker(\id-K)=\{0\}\iff \ker(\id-K^*)=\{0\}$.
  % \end{lemma}
  % \begin{proof}
  %   The argument is symmetric since $K^{**}=K$ and $K$ is compact $\iff K^*$ is compact. So suppose $\ker(\id-K)=\{0\}$. Then, $\id -K$ is injective.
  % \end{proof}
  \begin{theorem}[Abstract Fredholm alternative]
    Let $H$ be Hilbert and $K:H\to H$ be a compact linear operator. Then:
    \begin{enumerate}
      \item $\ker(\id-K)$ and $\ker(\id-K^*)$ are both finite dimensional, and they have the same dimension.
      \item $\im(\id-K)={\ker(\id-K^*)}^\perp$. In particular, $\im(\id-K)$ is closed.
      \item Either $\ker(\id-K)\ne\{0\}$ or $\id -K$ is an isomorphism.
    \end{enumerate}
  \end{theorem}
  \begin{proof}
    We organize the proof in several steps:
    \begin{enumerate}
      \setcounter{enumi}{1}
      \item From \mcref{RFA:adjoint_im_ker} we have that $\overline{\im A}={\ker A^*}^\perp$ for any general operator $A$ between Hilbert spaces. Thus, $\im(\id-K)={\ker(\id-K^*)}^\perp\iff \im(\id-K)$ is closed, which reduces to \mcref{INEPDE:lemma3_fredholm}.
      \item We first show that $\ker(\id-K)=\{0\}\iff \ker(\id-K^*)=\{0\}$. The argument is symmetric since $K^{**}=K$ and the fact that $K$ is compact $\iff K^*$ is compact. So suppose $\ker(\id-K)=\{0\}$. Then, $\id -K$ is injective. Assume $\ker(\id-K^*)\ne\{0\}$. Then, $\im (\id-K)=\ker(\id-K)^\perp\ne H$ and so $\im({(\id-K)}^2)\subsetneq \im(\id-K)$. Indeed, if we had equality, then for any $u\in H$, we would have ${(\id-K)u}\in \im({(\id-K)}^2)$, and thus $\exists v\in H$ such that ${(\id-K)u}={(\id-K)}^2v$, which implies $u={(\id-K)}v$ because $\ker (\id-K)=\{0\}$. Recursively, we have an infinite sequence $\im({(\id-K)}^{n+1})\subsetneq \im({(\id-K)}^n)$, which implies that $\forall n$ $\exists u_n\in \im({(\id-K)}^n)\cap\im({(\id-K)}^{n+1})^\perp$ with $\norm{u_n}=1$. Thus, $\langle u_n,u_m\rangle=\delta_{n,m}$. But $u_n-Ku_n\in \im({(\id-K)}^{n+1})$ so, $u_n-Ku_n\perp u_n$. This implies, by \mnameref{RFA:pythagorean}, that $\norm{Ku_n}=\norm{u_n-Ku_n}+\norm{u_n}\geq 1$, which is a contradiction with the compactness of $K$ because any orthonormal sequence always converges weakly to zero (and so $Ku_n\to 0$). So either $\ker(\id-K)\ne\{0\}$ or $\id-K$ is bijective.

            To finish this point, we need to prove that if $\ker(\id-K)$, then ${(\id-K)}^{-1}$ is a bounded linear operator. But this is a consequence of \mcref{INEPDE:lemma2_fredholm}: if $u\in H$, then $u\in \ker {(\id-K)}^\perp$ and thus $\norm{(\id-K)u}\geq c\norm{u}$, which implies that $\norm{v}\geq c \norm{{(\id-K)}^{-1}v}$ taking $v=(\id-K)u$.
            \setcounter{enumi}{0}
      \item Assume without loss of generality that $\dim\ker(\id-K)<\dim \ker(\id-K^*)$. Then, there exists a linear injective map $A:\ker(\id-K)\to \ker(\id-K^*)=\im(\id-K)^\perp$. Let $\tilde{K}$ be the operator defined by $\tilde{K}u=Ku+Au^{\text{ker}}$, where $u^{\text{ker}}$ is the projection of $u$ onto $\ker(\id-K)$. Then, $\tilde{K}$ is compact (because $K$ is compact and so is $A$, because it has finite range). Moreover, if $u\in \ker(\id-\tilde{K})$, then $(\id-K)u+Au^{\text{ker}}=0$, which since $(\id-K)u\in \im(\id-K)$ and $Au^{\text{ker}}\in \im(\id-K)^\perp$ implies that both terms are zero. So $u=u^{\text{ker}}\in \ker(\id-K)$ and since $A$ is injective, $u=u^{\text{ker}}=0$. Thus, $\ker(\id-\tilde{K})=\{0\}$ and by the previous point, $\id-\tilde{K}$ is an isomorphism from $H$ to itself. So, for every $w\in \ker(\id-K^*)$, $\exists u\in H$ such that $w=(\id-\tilde{K})u$. Projecting both side onto $\ker(\id-K^*)=\im(\id-K)^\perp$, we have $w=-Au^{\text{ker}}$, which implies that $A$ is onto, and so $\dim\ker(\id-K)=\dim\ker(\id-K^*)$. \mcref{INEPDE:lemma1_fredholm} finishes the proof.
    \end{enumerate}
  \end{proof}
  \begin{definition}
    Consider the operator $L$ as in \mcref{INEPDE:operator}. We define the \emph{formal adjoint} of $L$ as:
    \begin{align*}
      L^*v & :=-\sum_{i,j=1}^d\partial_i(a_{ij}\partial_jv)-\sum_{j=1}^d\partial_j(b_jv)+c v                                    \\
           & =\sum_{i,j=1}^d\partial_i(a_{ij}\partial_jv)-\sum_{j=1}^db_j\partial_jv+ \left(c-\sum_{j=1}^d\partial_jb_j\right)v
    \end{align*}
    It satisfies $\langle Lu,v\rangle=\langle u,L^*v\rangle$ for all $u,v\in H_0^1(\Omega)$.
  \end{definition}
  \begin{proposition}
    The \emph{homogeneous adjoint problem}
    $$
      \mathcal{D}_0^*:=\begin{cases}
        L^*v=0 & \text{in }\Omega         \\
        v=0    & \text{on }\partial\Omega
      \end{cases}
    $$
    whose weak formulation is
    \begin{equation*}
      {\langle \grad v,\grad w\rangle}_2+{\langle \vf{b}\cdot \grad v,w\rangle}_2=0\quad \forall w\in H_0^1(\Omega)
    \end{equation*}
    has a finite dimensional solution space $W_0$, as well as the space $V_0$ of solutions of $\mathcal{D}_0$, and $\dim W_0=\dim V_0$. Moreover, if $f\in L^2(\Omega)$, $\mathcal{D}_f$ is solvable if and only if $\langle f,v\rangle=0$ for all $v\in W_0$.
  \end{proposition}
  \begin{proposition}

  \end{proposition}
  \begin{definition}
    We define the following problem:
    $$
      \mathcal{N}_f:=\begin{cases}
        -\laplacian u=f   & \text{in }\Omega         \\
        \pdv{u}{\vf{n}}=0 & \text{on }\partial\Omega
      \end{cases}
    $$
    and $\mathcal{N}_f^*=\mathcal{N}_f$. The weak formulation of the problem is:
    \begin{equation*}
      \langle \grad u,\grad v\rangle=\langle f,v\rangle\quad \forall v\in H^1(\Omega)
    \end{equation*}
  \end{definition}
  \begin{proposition}
    $\mathcal{N}_f$ has at least one solution if and only if for any weak solution $v$ of $\mathcal{N}_0$ we have $\langle f,v\rangle=0$.
  \end{proposition}
  \subsubsection{Spectrum of compact operators}
  In this section $\KK$ will denote either $\RR$ or $\CC$.
  \begin{definition}
    Let $H$ be a $\KK$-Hilbert space and $K:H\to H$ be a compact operator. We define the \emph{resolvent set} of $K$ as:
    $$
      \rho(K)=\{\lambda\in \KK: \lambda\id-K \text{ is invertible}\}
    $$
    and the \emph{spectrum} of $K$ as:
    $$
      \sigma(K)=\KK\setminus \rho(K)
    $$
  \end{definition}
  \begin{theorem}
    Let $H$ be an infinite-dimensional Hilbert space and $K:H\to H$ be a compact operator. Then, $0\in \sigma(K)$ and $\sigma(K)$ is closed and at most countable. Moreover, if $\lambda\in \sigma(K)\setminus\{0\}$, then $\lambda$ is an eigenvalue of $K$ and:
    $$
      \dim\left(\bigcup_{p\geq 1}\ker{(\lambda\id-K)}^p\right)<\infty
    $$
    If $\sigma(K)\cap\RR^*$ is infinite, then it is of the form $\{\lambda_n\}_{n\in \NN}$ with $\lambda_n\to 0$.
  \end{theorem}
  \begin{proof}
    \begin{enumerate}
      \item Assume $0\notin \sigma(K)$. Then, $K$ is bijective and so $\id = K\circ K^{-1}$ is compact, as it is the composition of a compact operator and a bounded operator. But this is a contradiction with \mcref{INEPDE:lemma1_fredholm}.
    \end{enumerate}
  \end{proof}
  \begin{lemma}
    Let $H$ be a Hilbert space and $T:H\to H$ be a continuous self-adjoint operator. Then:
    $$
      \norm{T}=\sup_{\norm{x}=1}\abs{\langle x,Tx\rangle}
    $$
  \end{lemma}
  \begin{proof}
    Clearly $\alpha:=\sup_{\norm{x}=1}\abs{\langle x,Tx\rangle}\leq \norm{T}$. For the converse, it suffices to show that $\abs{\langle Tx,y\rangle}\leq \alpha$ for all $\norm{x}=\norm{y}=1$. We have:
    $$
      \langle Tx,y\rangle = \frac{1}{4}\left(\langle T(x+y),x+y\rangle-\langle T(x-y),x-y\rangle\right)
    $$
    And then, by \mnameref{RFA:parallelogram}:
    $$
      \abs{\langle Tx,y\rangle}\leq \frac{\alpha}{4}\left(\norm{x+y}^2+\norm{x-y}^2\right)= \alpha
    $$
  \end{proof}
  \begin{lemma}
    Let $H\ne\{ 0\}$ be Hilbert and $K:H\to H$ be a compact and self-adjoint operator. Then:
    $$
      \sup_{\norm{x}=1}\langle x,Kx\rangle=\lambda
    $$
    where $\lambda$ is the largest eigenvalue of $K$.
  \end{lemma}
  \subsubsection{Regularity theorems for weak solutions of divergence-form elliptic PDEs}
  \begin{theorem}[Inner regularity]
    Assume, in addition to the usual assumptions, that $a_{ij}\in \mathcal{C}^1(\Omega)$. Let $f\in L^2(\Omega)$ and $u\in H^1(\Omega)$ be a weak solution of $Lu=f$. Then, $u\in H^2_{\text{loc}}(\Omega)$ and for any compact embedding $\omega\subset\subset \Omega$, meaning that $\overline{\omega}\subset\Omega$ compact, we have $u\in H^2(\omega)$ and:
    $$
      \norm{u}_{H^2(\omega)}\leq C\left(\norm{f}_{L^2(\Omega)}+\norm{u}_{L^2(\Omega)}\right)
    $$
  \end{theorem}
  \begin{corollary}
    Assume that $a_{ij}\in\mathcal{C}^{m+1}(\Omega)$ for some $m\in\NN$, and $b_j,c\in \mathcal{C}^{m}(\Omega)$. Let $f\in H^{m}(\Omega)$ and $u\in H^1$ be a weak solution of $Lu=f$. Then, $u\in H^{m+2}_{\text{loc}}(\Omega)$ and for any $\omega\subset\subset \Omega$ we have $u\in H^{m+2}(\omega)$ and:
    $$
      \norm{u}_{H^{m+2}(\omega)}\leq C\left(\norm{f}_{H^{m}(\Omega)}+\norm{u}_{L^2(\Omega)}\right)
    $$
  \end{corollary}
  \begin{corollary}
    Assume $a_{ij},b_j,c,f\in\mathcal{C}^\infty(\Omega)$. Let $u\in H^1(\Omega)$ be a weak solution of $Lu=f$. Then, $u\in \mathcal{C}^\infty(\Omega)$.
  \end{corollary}
  \begin{theorem}[Regularity up to the boundary]
    Assume that $\Fr{\Omega}$ is $\mathcal{C}^2$ and that $a_{ij}\in \mathcal{C}^1(\overline{\Omega})$, $b_j,c\in L^\infty(\Omega)$. Let $f\in L^2(\Omega)$ and $u\in H^1_0(\Omega)$ be a weak solution of $\mathcal{D}_f$. Then, $u\in H^2(\Omega)$ and:
    $$
      \norm{u}_{H^2(\Omega)}\leq C\left(\norm{f}_{L^2(\Omega)}+\norm{u}_{L^2(\Omega)}\right)
    $$
  \end{theorem}
  \begin{corollary}
    Assume that $\Fr{\Omega}$ is $\mathcal{C}^{m}$, $m\in\NN$, and that $a_{ij}\in \mathcal{C}^{m+1}(\overline{\Omega})$, $b_j,c\in \mathcal{C}^{m}(\overline{\Omega})$. Let $f\in H^{m}(\Omega)$ and $u\in H^1_0(\Omega)$ be a weak solution of $\mathcal{D}_f$. Then, $u\in H^{m+2}(\Omega)$ and:
    $$
      \norm{u}_{H^{m+2}(\Omega)}\leq C\left(\norm{f}_{H^{m}(\Omega)}+\norm{u}_{L^2(\Omega)}\right)
    $$
  \end{corollary}
  \begin{corollary}
    Assume that $\Fr{\Omega}$ is $\mathcal{C}^\infty$ and that $a_{ij},b_j,c,f\in \mathcal{C}^\infty(\overline{\Omega})$. Let $u\in H^1_0(\Omega)$ be a weak solution of $\mathcal{D}_f$. Then, $u\in \mathcal{C}^\infty(\Omega)$ and $\forall m\in \NN$:
    $$
      \norm{u}_{H^{m}(\Omega)}\leq C\left(\norm{f}_{H^{m}(\Omega)}+\norm{u}_{L^2(\Omega)}\right)
    $$
  \end{corollary}
  \subsubsection{Weak maximum principle for weak solutions of divergence-form elliptic PDEs}
  \begin{lemma}\label{INEPDE:lemma1_weak_max}
    Let $\Omega\subseteq\RR^d$ open and $u\in H^1(\Omega)$. Then:
    $$
      u^{+}:=\begin{cases}
        u & \text{if }u> 0    \\
        0 & \text{if }u\leq 0
      \end{cases}\qquad
      u^{-}:=\begin{cases}
        -u & \text{if }u< 0    \\
        0  & \text{if }u\leq 0
      \end{cases}
    $$
    are also in $H^1(\Omega)$ and:
    $$
      \grad(u^+)\almoste{=}\begin{cases}
        \grad u & \text{if }u>0     \\
        0       & \text{if }u\leq 0
      \end{cases}\;\;
      \grad(u^-)\almoste{=}\begin{cases}
        -\grad u & \text{if }u<0     \\
        0        & \text{if }u\geq 0
      \end{cases}
    $$
  \end{lemma}
  \begin{corollary}
    Let $\Omega\subseteq\RR^d$ open and $u\in H^1(\Omega)$. Then, $\abs{u}\in H^1(\Omega)$ and $\grad{\abs{u}}=\sign\grad{u}$.
  \end{corollary}
  \begin{lemma}
    Let $(u_n)\in H^1(\Omega)$ be such that $u_n\overset{H^1(\Omega)}{\longrightarrow} u$. Then, ${u_n}^\pm\overset{H^1(\Omega)}{\longrightarrow} u^\pm$.
  \end{lemma}
  \begin{corollary}
    Let $u\in H^1(\Omega)$. Then, $\Tr_{\partial\Omega}(u^\pm)={(\Tr_{\partial\Omega}u)}^\pm$.
  \end{corollary}
  \begin{lemma}
    Let $\Omega\subseteq\RR^d$ open with $\mathcal{C}^1$ boundary, $u\in H^1(\Omega)$ and $\Tr_{\partial\Omega}u\almoste{\leq}0$. Then, $u^+\in H^1_0(\Omega)$.
  \end{lemma}
  \begin{theorem}[Weak maximum principle]
    Let $\Omega\subseteq\RR^d$ open and bounded with $\mathcal{C}^1$ boundary, $a_{ij}=a_{ji},c\in L^\infty(\Omega)$, $c\almoste{\geq}0$, $L=-\sum_{i,j=1}^d\partial_i(a_{ij}\partial_j)+c$ be elliptic and $f\in L^2(\Omega)$ with $f\almoste{\leq} 0$. Let $u\in H^1(\Omega)$ be such that:
    \begin{itemize}
      \item $\displaystyle \int_\Omega\left[\sum_{i,j=1}^da_{ij}\partial_iu\partial_jv+cuv\right]=\int_\Omega fv$ $\forall v\in H^1_0(\Omega)$
      \item $\Tr_{\partial\Omega}u\almoste{\leq}0$
    \end{itemize}
    Then, $u\almoste{\leq}0$.
  \end{theorem}
  \begin{proof}
    Take $v=u^+\in H^1_0(\Omega)$ by \mcref{INEPDE:lemma1_weak_max}. Then, we have:
    $$
      0\leq \theta {\norm{\grad u^+}_{L^2}}^2\leq\!\! \int_{\{u>0\}}\!\!\sum_{i,j=1}^d a_{ij}\partial_iu\partial_ju+cu^2=\!\!\int_{\{u>0\}} \!\!fu\leq 0
    $$
    where in the second inequality we used the ellipticity of $L$. Thus, we must have $\grad u^+=0$ a.e. in $\Omega$, which implies $u^+=0$ a.e. in $\Omega$, because $u^+|_{\Fr{\Omega}}=0$.
  \end{proof}
  \begin{theorem}[Weak maximum principle]
    Let $\Omega\subseteq\RR^d$ open and bounded with $\mathcal{C}^1$ boundary, $a_{ij}=a_{ji},b_j,c\in L^\infty(\Omega)$, $c \almoste{\geq}0$, $L=-\sum_{i,j=1}^d\partial_i(a_{ij}\partial_j)+\sum_{j=1}^db_j\partial_j+c$ be elliptic and $f\in L^2(\Omega)$ with $f\almoste{\leq} 0$. Let $u\in H^1(\Omega)$ be such that:
    \begin{itemize}
      \item $\displaystyle \int_\Omega\left[\sum_{i,j=1}^da_{ij}\partial_iu\partial_jv+ \sum_{j=1}^db_jv\partial_ju+cuv\right]=\int_\Omega fv$ $\forall v\in H^1_0(\Omega)$
      \item $\Tr_{\partial\Omega}u\almoste{\leq}0$
    \end{itemize}
    Then, $u\almoste{\leq}0$.
  \end{theorem}
  \begin{proof}
    TODO
  \end{proof}
  \begin{corollary}
    For each $f\in L^2(\Omega)$, the problem $\mathcal{D}_f$ has a unique weak solution $u_f$. Moreover, if $\Fr{\Omega}\in\mathcal{C}^1$, then $u_f\in H^2(\Omega)$ and $f\mapsto u_f$ is a bounded linear operator from $L^2(\Omega)$ to $H^2(\Omega)$. If $\Fr{\Omega}\in\mathcal{C}^{m+1}$, $b_j\in\mathcal{C}^{m-1}$ and $f\in H^{m-1}(\Omega)$, then $u_f\in H^{m+1}(\Omega)$ and $f\mapsto u_f$ is a bounded linear operator from $H^{m-1}(\Omega)$ to $H^{m+1}(\Omega)$.
  \end{corollary}
  \begin{theorem}
    Let $1<p<\infty$ and $\Omega\subset\RR^d$ be open and bounded with $\mathcal{C}^{m+1}$ boundary, $m\geq 1$. Let $a_{ij}\in \mathcal{C}^m(\overline{\Omega})$, $b_j,c\in \mathcal{C}^{m-1}(\overline{\Omega})$ and $Lu=-\sum_{i,j=1}^d \partial_i(a_{ij}\partial_j u)+\sum_{j=1}^d b_j\partial_j u+cu$ be an elliptic operator. Then, for any $f\in W^{m-1,p}(\Omega)$, if $u\in H^1_0(\Omega)$ is a weak solution of $\mathcal{D}_f$, then $u\in W^{m+1,p}(\Omega)$ and:
    $$
      \norm{u}_{W^{m+1,p}(\Omega)}\leq C\left(\norm{f}_{W^{m-1,p}(\Omega)}+\norm{u}_{L^2(\Omega)}\right)
    $$
    If in addition the weak solution of $\mathcal{D}_0$ is $u=0$, then $L:W^{m+1,p}(\Omega)\cap W_0^{1,p}(\Omega)\to W^{m-1,p}(\Omega)$ is an isomorphism, where $W_0^{1,p}(\Omega)$ is the closure of $C_0^\infty(\Omega)$ in $W^{1,p}(\Omega)$.
  \end{theorem}
  \subsection{Regularity in \texorpdfstring{$\mathcal{C}^{k,\alpha}$}{Ckalpha} for non-divergence form elliptic PDEs}

  In this section we will still always work in $\Omega\subset\RR^d$ open and bounded and the elliptic operator $L$ (with ellipticity constant $\theta$) will be in its non-divergence form:
  $$
    L=-\sum_{i,j=1}^d a_{ij}\partial_{ij}^2+\sum_{j=1}^d b_j\partial_j+c
  $$
  with $a_{ij}=a_{ji}$.  Moreover we will not use the usual Hölder norm
  $$
    \norm{u}_{\mathcal{C}^{k,\alpha}(\Omega)}=\sup_{\substack{x\ne y\\ \abs{\beta}=k}}\frac{\abs{\partial^\beta u(x)-\partial^\beta u(y)}}{\abs{x-y}^\alpha}
  $$
  but the following one:
  $$
    \norm{u}_{\mathcal{C}^{k,\alpha}(\overline\Omega)}=\sup_{\substack{x\in \overline\Omega\\ \abs{\beta}\leq k}}\abs{\partial^\beta u(x)}+\sup_{\substack{x\ne y\\ \abs{\beta}=k}}\frac{\abs{\partial^\beta u(x)-\partial^\beta u(y)}}{\abs{x-y}^\alpha}
  $$

  \begin{remark}
    Recall that $(\mathcal{C}^{k,\alpha}(\overline\Omega),\norm{\cdot}_{\mathcal{C}^{k,\alpha}(\overline\Omega)})$ is a Banach space and that if $0<\alpha_1\leq\alpha_2<1$, then $
      \mathcal{C}^{k,\alpha_2}(\overline{\Omega})\subseteq \mathcal{C}^{k,\alpha_1}(\overline{\Omega})$
  \end{remark}

  \subsubsection{Schauder estimates}
  \begin{theorem}
    Let $\Omega\subset \RR^d$ be open and bounded with $\Fr{\Omega}\in\mathcal{C}^{2,\alpha}$ for some $0<\alpha<1$. In the elliptic operator $L$ assume that $a_{ij},b_j,c\in\mathcal{C}^{0,\alpha}(\overline{\Omega})$. Then, $\exists c>0$ such that if $u\in\mathcal{C}^2(\Omega)\cap \mathcal{C}^0(\overline{\Omega})$ solves $Lu=f$, with $f\in\mathcal{C}^{0,\alpha}(\overline{\Omega})$, then $u\in \mathcal{C}^{2,\alpha}(\overline{\Omega})$ and:
    $$
      \norm{u}_{\mathcal{C}^{2,\alpha}(\overline{\Omega})}\leq C\left(\norm{f}_{\mathcal{C}^{0,\alpha}(\overline{\Omega})}+\norm{u}_{\mathcal{C}^{1,\alpha}(\overline{\Omega})}\right)
    $$
    Moreover we have:
    $$
      \norm{u}_{\mathcal{C}^{2,\alpha}(\overline{\Omega})}\leq \tilde{C}\left(\norm{f}_{\mathcal{C}^{0,\alpha}(\overline{\Omega})}+\norm{u}_{\mathcal{C}^{0}(\overline{\Omega})}\right)
    $$
  \end{theorem}
  \begin{corollary}
    Let $\Omega\subset \RR^d$ be open and bounded with $\Fr{\Omega}\in\mathcal{C}^{k+2,\alpha}$ for some $0<\alpha<1$ and $k\geq 0$. In the elliptic operator $L$ assume that $a_{ij},b_j,c\in\mathcal{C}^{k,\alpha}(\overline{\Omega})$. Then, $\exists c>0$ such that if $u\in\mathcal{C}^{k+2}(\Omega)\cap \mathcal{C}^k(\overline{\Omega})$ solves $Lu=f$, with $f\in\mathcal{C}^{k,\alpha}(\overline{\Omega})$, then $u\in \mathcal{C}^{k+2,\alpha}(\overline{\Omega})$ and:
    $$
      \norm{u}_{\mathcal{C}^{k+2,\alpha}(\overline{\Omega})}\leq C\left(\norm{f}_{\mathcal{C}^{k,\alpha}(\overline{\Omega})}+\norm{u}_{\mathcal{C}^{k+1,\alpha}(\overline{\Omega})}\right)
    $$
  \end{corollary}
  \subsubsection{Maximum and comparison principles}
  \begin{theorem}[Weak maximum principle]
    Let $u\in \mathcal{C}^2(\Omega)$ be such that $Lu\leq 0$. Then:
    \begin{itemize}
      \item If $c=0$, $\displaystyle \max_{\overline{\Omega}}u=\max_{\partial\Omega}u$.
      \item If $c\geq 0$, $\displaystyle \max_{\overline{\Omega}}u^+\leq \max_{\partial\Omega}u^+$.
    \end{itemize}
  \end{theorem}
  \begin{lemma}[Hopf's lemma]\label{INLEPDE:Hopf}
    Let $u\in \mathcal{C}^2(\Omega)$ be such that $Lu\leq 0$ and suppose the region $\Omega$ is connected and that satisfies the \emph{interior ball condition}: for any $x\in \partial\Omega$ there exists $r>0$ and $y\in \Omega$ such that $B(y,r)\subset \Omega$ and $\overline{B(y,r)}\cap \Fr{\Omega}=\{x\}$. Suppose in addition that $c=0$ and $x_0\in\Fr\Omega$ be such that $u(x_0)=\max_{\overline{\Omega}}u$. Then, either $u$ is constant in $\Omega$ or
    $$
      \liminf_{t\to 0^+}\frac{u(x_0) - u(x_0+t\vf{n})}{t}>0
    $$
    for any vector $\vf{n}$ of the form $\vf{n}=\frac{x_0-y_0}{\norm{x_0-y_0}}$ with $B(y_0,r)\subset \Omega$ and $\overline{B(y_0,r)}\cap \Fr{\Omega}=\{x_0\}$.
  \end{lemma}
  \begin{remark}
    In particular, if $\Fr\Omega\in\mathcal{C}^1$ and $u\in \mathcal{C}^1(\overline{\Omega})$, then \mnameref{INLEPDE:Hopf} implies that either $u$ is constant in $\Omega$ or $\partial_{\vf{n}}u(x_0)=\grad u(x_0)\cdot \vf{n}>0$.
  \end{remark}
  \begin{theorem}[Strong maximum principle]
    Let $\Omega\subset\RR^d$ be open, bounded and connected, and $u\in \mathcal{C}^2(\Omega)$ be such that $Lu\leq 0$ with $c=0$. If $\exists x_0\in\Omega$ such that $u(x_0)\geq u(x)$ $\forall x\in\Omega$, then $u$ is constant in $\Omega$. Similarly, if $c\geq 0$ and $\exists x_0\in\Omega$ such that $u(x_0)\geq 0$ and $u(x_0)\geq u(x)$ $\forall x\in\Omega$, then $u$ is constant in $\Omega$.
  \end{theorem}
  \begin{theorem}
    Suppose that $c\geq 0$ and $u\in \mathcal{C}^2(\Omega)$ is a solution to
    $$
      \begin{cases}
        Lu=f                  & \text{in }\Omega         \\
        u|_{\partial\Omega}=h & \text{on }\partial\Omega
      \end{cases}
    $$
    with $f\in \mathcal{C}^0(\overline{\Omega})$ and $h\in \mathcal{C}^0(\partial\Omega)$. Then, $\forall x\in\overline{\Omega}$:
    $$
      u(x)\leq \max_{\partial\Omega}h^++C\max_{\overline{\Omega}}f^+
    $$
    with $C$ independent of $u$, $f$ and $h$.
  \end{theorem}
  \subsubsection{Continuation method}
  \begin{theorem}[Continuation method]
    Let $\Omega\subset\RR^d$ be open and bounded with $\Fr{\Omega}\in\mathcal{C}^{2,\alpha}$ for some $0<\alpha<1$. Consider the problem:
    $$
      \begin{cases}
        Lu=f                  & \text{in }\Omega         \\
        u|_{\partial\Omega}=h & \text{on }\partial\Omega
      \end{cases}
    $$
    with $f,a_{ij},b_j,c\in\mathcal{C}^{0,\alpha}( \overline{\Omega})$ and $h\in\mathcal{C}^{0,\alpha}(\partial\Omega)$. Then, there exists a solution to this problem in $\mathcal{C}^{2,\alpha}(\overline{\Omega})$.
  \end{theorem}
  \subsection{Existence theorems for nonlinear elliptic PDEs by fixed point methods}
  In this section we will mostly consider almost linear elliptic PDEs of the form:
  \begin{equation}\label{INLEPDE:AlmostLinear}
    \begin{cases}
      Lu=f(x,u) \\
      u|_{\partial\Omega}=0
    \end{cases}
  \end{equation}
  with $L$ either $-\sum_{i,j=1}^d\partial_i(a_{ij}\partial_j)+\sum_{j=1}^db_j\partial_j$ or $-\sum_{i,j=1}^d a_{ij} \partial_{ij}^2+\sum_{j=1}^db_j\partial_j$, and $f:\Omega\times \RR\to \RR$.
  \subsubsection{Method of subsoltions and supersolutions}
  \begin{theorem}
    Suppose that an operator $L$ is uniformly elliptic on an open bounded set $\Omega\subset\RR^d$ with $\Fr{\Omega}\in \mathcal{C}^2$, with $c=0$ and either in divergence form (with $a_{ij}\in\mathcal{C}^1$) or non-divergence form (with $a_{ij},b_j\in\mathcal{C}^{0,\alpha}$). Suppose that $f\in\mathcal{C}^1(\overline{\Omega}\times \RR)$ and assume that the problem of \mcref{INLEPDE:AlmostLinear} has a bounded subsolution $\underline{u}$ and a bounded supersolution $\overline{u}$ such that $\underline{u}\leq \overline{u}$. Then, there exists a solution $u$ to \mcref{INLEPDE:AlmostLinear} such that $\underline{u}\leq u\leq \overline{u}$, which is in $H_0^1(\Omega)\cap H_0^2(\Omega)$ if $L$ is in divergence form and in $\mathcal{C}^{2,\alpha}(\overline{\Omega})$ if $L$ is in non-divergence form.
  \end{theorem}
  \subsubsection{Topological fixed point theorems}
  \begin{theorem}[Brower fixed point]
    Let $C\subset \RR^n$ be a closed convex bounded set and $f:C\to C$ be a continuous function. Then, $f$ has at least a fixed point.
  \end{theorem}
  \begin{theorem}[Schauder fixed point]
    Let $C$ be a convex set in a Banach space $(E,\norm{\cdot})$ and $f:C\to C$ be a continuous function. Assume one of the following two assumptions:
    \begin{itemize}
      \item $C$ is compact for $\norm{\cdot}$.
      \item $C$ is closed and bounded and $f$ is compact.
    \end{itemize}
    Then, $f$ has at least a fixed point.
  \end{theorem}
  \begin{theorem}[Schaefer fixed point]
    Let $(E, \norm{\cdot})$ be Banach and $f:E\to E$ be a continuous and compact. Suppose that $\exists M>0$ such that $\forall (\lambda,u)\in [0,1]\times E$ with $u=\lambda f(u)$ we have $\norm{u}<M$. Then, $f$ has at least a fixed point, that lies in $\overline{B(0,M)}$.
  \end{theorem}
  \subsection{Variational methods for nonlinear elliptic PDEs}
  In this section we will solve a PDE $Lu=f(x,u,\grad u)$ by minimizing a certain functional under some constraints.
  \subsection{Linear case}
  \begin{proposition}
    Consider the problem:
    $$
      \begin{cases}
        Lu:=-\sum_{i,j=1}^d \partial_i(a_{ij}\partial_j u)+cu=f \\
        u|_{\partial\Omega}=0
      \end{cases}
    $$
    with $L$ elliptic, $a_{ij},c\in L^\infty(\Omega)$ with $a_{ij}= a_{ji}$, $c\geq 0$, and $f\in L^2(\Omega)$. Then, the problem has a unique weak solution $u\in H_0^1(\Omega)$, and it minimizes the functional:
    $$
      I(u)=\frac{1}{2}\int_\Omega\sum_{i,j=1}^d a_{ij}\partial_iu\partial_ju+cu^2-\int_\Omega fu
    $$
  \end{proposition}

  \begin{lemma}\label{INEPDE:optimization}
    Let $X$ be a Banach space and $\Phi:X\to \RR$ be continuous and convex, then it is weakly sequentially lower semicontinuous, that is, if $u_n\rightharpoonup u$ in $X$, then $\displaystyle\Phi(u)\leq \liminf_{n\to\infty}\Phi(u_n)$.
  \end{lemma}
  \begin{theorem}
    Let $(X,\norm{\cdot})$ be a reflexive Banach space and $\Phi:X\to \RR$ be continuous, convex and such that $\displaystyle \lim_{\norm{u}\to\infty}\Phi(u)=+\infty$. Then, $\Phi$ has a minimizer. This minimizer is unique if $\Phi$ is strictly convex.
  \end{theorem}
  \begin{proof}
    Let $\{u_n\}_{n\in\NN}\subset X$ be a minimizing sequence. Then, $\sup_{n\in\NN}\Phi(u_n)<\infty$, so by the coercivity property of $\Phi$ we have that $\{u_n\}_{n\in\NN}$ is bounded, and so $\{u_n\}_{n\in\NN}$ has a weakly convergent subsequence $\{u_{n_k}\}_{k\in\NN}$ with limit $u\in X$. By \mcref{INEPDE:optimization} we have:
    $$
      \Phi(u)\leq \lim_{k\to\infty}\Phi(u_{n_k})=\inf_{u\in X}\Phi(u)
    $$
    But $\Phi(u)\geq \inf_{u\in X}\Phi(u)$, so $u$ is a minimizer.
  \end{proof}
\end{multicols}
\end{document}