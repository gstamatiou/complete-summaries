\documentclass[class=article,10pt,crop=false]{standalone}
\usepackage{standalone}
\usepackage{preamble}

\begin{document}
\begin{multicols}{2}[\section{Algebraic structures}]
\subsection{Groups}
\subsubsection*{Groups, subgroups and homomorphisms}
\begin{definition}[Group]
A group is a non-empty set $G$ together with a binary operation \begin{align*}
    \cdot:G\times G&\rightarrow G\\
    (g_1,g_2)&\mapsto g_1\cdot g_2
\end{align*}
satisfying the following properties:
\begin{enumerate}
    \item Associativity: $$(g_1\cdot g_2)\cdot g_3=g_1\cdot(g_2\cdot g_3),\quad\forall g_1,g_2,g_3\in G.$$
    \item Identity element: $$\exists e\in G:e\cdot g=g\cdot e=g,\quad\forall g\in G.\footnote{From now on, we will denote $e$ or $e_G$ the identity element of the group $(G,\cdot)$.}$$
    \item Inverse element: $$\forall g\in G, \exists h\in G:g\cdot h=h\cdot g=e.$$ We denote $h$ by $g^{-1}$.
\end{enumerate}
If, moreover, we have $g_1\cdot g_2=g_2\cdot g_1$ $\forall g_1,g_2\in G$, we say the group $(G,\cdot)$ is \textit{commutative} or \textit{abelian}.\footnote{Sometimes to simplify the notation and if the context is clear, we will refer to $G$ directly as the group as well as the set.}
\end{definition}
\begin{lemma}
Let $(G,\cdot)$ be a group. Then,
\begin{enumerate}
    \item The identity element is unique.
    \item Given an element $g\in G$, $\exists! h\in G$ such that $g\cdot h=h\cdot g=e$.
    \item Given $g,h\in G$ such that $g\cdot h=e$, we have $h=g^{-1}$.
\end{enumerate}
\end{lemma}
\begin{definition}[Subgroup]
Let $(G,\cdot)$ be a group and $H$ a subset of $G$. $(H,\cdot)$ is called a \textit{subgroup} of $(G,\cdot)$\footnote{Sometimes we will denote that $(H,\cdot)$ is a subgroup of $(G,\cdot)$ as $H\leq G$.} if satisfies:
\begin{enumerate}
    \item If $h_1,h_2\in H$, then $h_1\cdot h_2\in H$.
    \item $e\in H$.
    \item If $h\in H$, then $h^{-1}\in H$.
\end{enumerate}
\end{definition}
\begin{prop}
Let $(G,\cdot)$ be a group and $H\ne\emptyset$ a subset of $G$. Then $$(H,\cdot)\text{ is a subgroup }\iff h_1\cdot h_2^{-1}\in H\quad\forall h_1,h_2\in H.$$
\end{prop}
\begin{prop}
If $(H,+)$ is a subgroup of $(\mathbb{Z},+)$, then $\exists n\in\mathbb{Z}$ such that $H=n\mathbb{Z}=\{nk:k\in\mathbb{Z}\}$.
\end{prop}
\begin{prop}
Let $(G_i,*_i)$ for $i=1,\ldots, n$ be groups. Then the product $$(G_1,*_1)\times\cdots\times(G_n,*_n)$$ is also a group with the operation $\cdot$ defined as $$(g_1,\ldots,g_n)\cdot(g_1',\ldots,g_n')=(g_1*_1g_1',\ldots,g_n*_ng_n'),$$ where $g_i,g_i'\in G_i$.
\end{prop}
\begin{definition}
The \textit{order of a group} is the number of elements in its set.
\end{definition}
\begin{lemma}
Let $(G,\cdot)$ be a group and $\{(H_i,\cdot):i\in I\}$ a set of subgroups of $(G,\cdot)$. Then if $$H=\displaystyle\bigcap_{i\in I}H_i,$$ we have that $(H,\cdot)$ is also a subgroup of $(G,\cdot)$.
\end{lemma}
\begin{definition}
Let $(G,\cdot)$ be a group and $X\subseteq G$ a subset of $G$. The \textit{subgroup of $(G,\cdot)$ generated by $X$}, $\langle X\rangle$, is the smallest subgroup of $(G,\cdot)$ containing $X$, that is, $$\langle X\rangle=\bigcap_{X\subseteq H\leq G}H.$$
\end{definition}
\begin{definition}
Let $(G,\cdot)$ be a group, $g\in G$ and $n\in\mathbb{Z}$. We define $g^n$ as: $$g^n=\left\{\begin{array}{ll}
    g\cdot\overset{(n)}{\cdots}\cdot g & \text{si }n>0  \\
    1 & \text{si }n=0  \\
    (-g)\cdot\overset{(|n|)}{\cdots}\cdot(-g) & \text{si }n<0 
\end{array}\right.$$
\end{definition}
\begin{lemma}
Let $(G,\cdot)$ be a group and $g\in G$. Then for all $n,m\in\mathbb{Z}$ we have
\begin{enumerate}
    \item $g^n\cdot g^m=g^{n+m}=g^m\cdot g^n$.
    \item $(g^n)^m=g^{nm}=(g^m)^n$.
\end{enumerate}
\end{lemma}
\begin{definition}
Let $(G,\cdot)$ be a group and $g\in G$. The subgroup $(\langle g\rangle,\cdot)$ of $(G,\cdot)$ generated by a single element $g$ is called a \textit{cyclic group}.
\end{definition}
\begin{prop}
Let $(G,\cdot)$ be a group and $g\in G$. Then, $$\langle g\rangle=\bigcup_{i\in\mathbb{Z}}\{g^i\}.$$
\end{prop}
\begin{definition}
Let $(G,\cdot)$ be a group and $g\in G$. The \textit{order of $g$} is $o(g):=|\langle g\rangle|$.
\end{definition}
\begin{prop}
Let $(G,\cdot)$ be a group and $g\in G$. Then, $$o(g)=\min\{i>0:g^i=e\}.$$ If no such $i$ exists we say $o(g)=\infty$.
\end{prop}
\begin{corollary}
Let $n>1$ be an integer and $\Bar{a}\in\mathbb{Z}/n\mathbb{Z}$. Then $$o(\Bar{a})=\frac{n}{\gcd(a,n)}.$$
\end{corollary}
\begin{corollary}
Let $(G_i,*_i)$ for $i=1,\ldots, n$ be groups. For $i=1,\ldots,n$, let $g_i\in G_i$ and consider the element $g=(g_1,\ldots,g_n)\in(G_1,*_1)\times\cdots\times(G_n,*_n)$. Then $$o(g)=\lcm(o(g_1),\ldots,o(g_n)).$$
\end{corollary}
\begin{prop}
Let $(G,\cdot)$ be a group and $X\subseteq G$ a subset of $G$. Then $$\langle X\rangle=\{e\}\cup\{g_1^{\alpha_1}\cdot\cdots\cdot g_n^{\alpha_n}:n\in\mathbb{N},\alpha_i\in\mathbb{Z},g_i\in X\}.$$
\end{prop}
\begin{definition}[Group morphism]
Let $(G,*)$, $(H,\cdot)$ be two groups. A \textit{morphism from $(G,*)$ to $(H,\cdot)$} is a function $\phi:G\rightarrow H$ such that $$\phi(g_1*g_2)=\phi(g_1)\cdot\phi(g_2),\quad\forall g_1,g_2\in G.$$
\end{definition}
\begin{lemma}
Let $\phi:G_1\rightarrow G_1$ be a morphism between $(G_1,*)$ and $(G_2,\cdot)$. Then,
\begin{enumerate}
    \item $\phi(e_1)=e_2$.
    \item $\phi(g^{-1})=\phi(g)^{-1},\quad\forall g\in G_1$.
    \item $\phi(g^n)=\phi(g)^n,\quad\forall g\in G_1$ and $\forall n\in\mathbb{Z}$.
\end{enumerate}
\end{lemma}
\begin{definition}
Let $(G_1,*)$, $(G_2,\cdot)$ be two groups and $\phi:G_1\rightarrow G_2$ be a group morphism. The \textit{kernel of $\phi$} is $$\ker(\phi)=\{g\in G_1:\phi(g)=e_2\}.$$ The \textit{image of $\phi$} is $$\im(\phi)=\{h\in G_2:\phi(g)=h\text{ for some }g\in G_1\}.$$
\end{definition}
\begin{prop}
Let $(G_1,*)$, $(G_2,\cdot)$ be two groups and $\phi:G_1\rightarrow G_2$ be a group morphism. Then:
\begin{enumerate}
    \item $\ker(\phi)$ is a subgroup of $G_1$ and $\im(\phi)$ is a subgroup of $G_2$.
    \item The following are equivalent:
    \begin{enumerate}
        \item $\phi(g_1)=\phi(g_2)$.
        \item $g_1*g_2^{-1}\in\ker(\phi)$.
        \item $g_2^{-1}*g_1\in\ker(\phi)$.
    \end{enumerate}
    \item $\phi$ is injective if and only if $\ker(\phi)=\{e_1\}$.
\end{enumerate}
\end{prop}
\begin{definition}
We say a subgroup $(H,\cdot)$ of a group $(G,\cdot)$ is \textit{normal} ($H\lhd G$) if and only if for all $h\in H$ and $g\in G$, we have $g\cdot h\cdot g^{-1}\in H$.
\end{definition}
\begin{prop}
Let $(G_1,\cdot_1)$, $(G_2,\cdot_2)$ and $(G_3,\cdot_3)$ be three groups and $\phi:(G_1,\cdot_1)\rightarrow(G_2,\cdot_2)$, $\psi:(G_2,\cdot_2)\rightarrow(G_3,\cdot_3)$ be two group morphism. Then the composition $\psi\circ\phi$ is also a group morphism.
\end{prop}
\begin{prop}
Let $(G_1,*)$, $(G_2,\cdot)$ be groups and consider the isomorphism $\phi:(G_1,*)\rightarrow(G_2,\cdot)$. Then $\phi:(G_1,*)\rightarrow(G_2,\cdot)$ is a group morphism.
\end{prop}
\begin{theorem}[Classification of cyclic groups]
Let $(G,\cdot)$ be a group and let $g\in G$ be an element such that $\langle g\rangle=G$. Then:
\begin{itemize}
    \item If $|G|=\infty$, then $G\cong\mathbb{Z}$. We can define the isomorphism as follows: 
    \begin{align*}
        \phi:\mathbb{Z}&\rightarrow G\\
        k&\mapsto g^k
    \end{align*}
    \item If $|G|=n$, then $G\cong\mathbb{Z}/n\mathbb{Z}$. We can define the isomorphism as follows: 
    \begin{align*}
        \phi:\mathbb{Z}/n\mathbb{Z}&\rightarrow G\\
        \Bar{k}&\mapsto g^k
    \end{align*}
\end{itemize}
\end{theorem}
\begin{corollary}
Let $(G,\cdot)$ be a group and let $g\in G$ be an element such that $\langle g\rangle=G$. Then all subgroups of $G$ are cyclic and moreover:
\begin{itemize}
    \item If $|G|=\infty$, subgroups of $(G,\cdot)$ are of the form $\langle g^n\rangle$, $n\in\mathbb{N}\cup\{0\}$. 
    \item If $|G|=n$, then there is a unique subgroup $(H,\cdot)$ of $(G,\cdot)$ for every divisor $d>0$ of $n$. In fact, if $n=dq$, then $H=\langle g^q\rangle$ and $|H|=d$.
\end{itemize}
\end{corollary}
\begin{definition}
Let $X$ be a set. We define the \textit{symmetric group $S(X)$} as $$S(X)=\{f:X\rightarrow X:f\text{ is bijective}\}\footnote{Observe that if $X=\{1,\ldots,n\}$, then $S(X)=S_n$.}.$$ 
\end{definition}
\begin{definition}
Let $(G,\cdot)$ be a group. We define the functions
\begin{align*}
    \ell_g:G&\rightarrow G\qquad &r_g:G&\rightarrow G\\
    x&\mapsto g\cdot x &x&\mapsto x\cdot g
\end{align*}
\end{definition}
\begin{lemma}
Let $(G,\cdot)$ be a group. The functions $\ell_g$, $r_g$ are bijective and its inverses are $\ell_{g^{-1}}$, $r_{g^{-1}}$, respectively.
\end{lemma}
\begin{prop}
Let $(G,\cdot)$ be a group. We define the functions:
\begin{align*}
    \phi:G&\rightarrow S(G)\qquad &\psi:G&\rightarrow G\\
    g&\mapsto \ell_g &g&\mapsto r_{g^{-1}}
\end{align*}
Then, $\phi$ and $\psi$ are injective group morphisms.
\end{prop}
\begin{theorem}[Cayley's theorem]
Let $(G,\cdot)$ be a group. Then there exists an injective morphism $$\phi:(G,\cdot)\rightarrow S(G).$$
\end{theorem}
\begin{corollary}
If $(G,\cdot)$ is a group with $|G|=n\in\mathbb{N}$, then we can think $(G,\cdot)$ as as subgroup of $S_n$.
\end{corollary}
\begin{definition}
Let $(G,\cdot)$ be a finite group and $(H,\cdot)$ a subgroup of $(G,\cdot)$. Let $g_1,g_2\in G$. We say $g_1\sim g_2\iff g_1g_2^{-1}\in H$.
\end{definition}
\begin{lemma}
Let $(G,\cdot)$ be a finite group and $(H,\cdot)$ a subgroup of $(G,\cdot)$. Then:
\begin{enumerate}
    \item $\sim$ is an equivalence relation.
    \item If $g\in G$, $[g]=Hg=\{h\cdot g:h\in H\}$.
\end{enumerate}
\end{lemma}
\begin{theorem}[Lagrange's theorem]
Let $(G,\cdot)$ be a finite group and $(H,\cdot)$ a subgroup of $(G,\cdot)$. Then $|H|\mid|G|$.
\end{theorem}
\end{multicols}
\end{document}