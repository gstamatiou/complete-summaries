\documentclass[../../../main.tex]{subfiles}

\begin{document}
\begin{multicols}{2}[\section{Algebraic structures}]
\subsection{Groups}\label{AS-G}
\subsubsection{Groups and subgroups}
\begin{definition}[Group]\label{AS-group}
    A \textit{group} is a non-empty set $G$ together with a binary operation 
    \begin{align*}
        \cdot:G\times G&\longrightarrow G\\
        (g_1,g_2)&\longmapsto g_1\cdot g_2
    \end{align*}
    satisfying the following properties:
    \begin{enumerate}
        \item Associativity: $$(g_1\cdot g_2)\cdot g_3=g_1\cdot(g_2\cdot g_3)\quad\forall g_1,g_2,g_3\in G.$$
        \item Identity element: $$\exists e\in G:e\cdot g=g\cdot e=g\quad\forall g\in G\footnote{From now on, we will denote $e$ or $e_G$ the identity element of the group $(G,\cdot)$.}.$$
        \item Inverse element: $$\forall g\in G, \exists h\in G:g\cdot h=h\cdot g=e.$$ We denote $h$ by $g^{-1}$.
    \end{enumerate}
    In this context we say $(G\cdot)$ is a group. If, moreover, we have $g_1\cdot g_2=g_2\cdot g_1$ $\forall g_1,g_2\in G$, we say that the group $(G,\cdot)$ is \textit{commutative} or \textit{abelian}\footnote{Sometimes to simplify the notation and if the context is clear, we will refer to $G$ directly as the group as well as the set.}.
\end{definition}
\begin{lemma}
    Let $(G,\cdot)$ be a group. Then:
    \begin{enumerate}
        \item The identity element is unique.
        \item Given an element $g\in G$, $\exists! h\in G$ such that $g\cdot h=h\cdot g=e$.
        \item Given $g,h\in G$ such that $g\cdot h=e$, we have $h=g^{-1}$.
    \end{enumerate}
\end{lemma}
\begin{definition}[Subgroup]
    Let $(G,\cdot)$ be a group and $H$ be a subset of $G$. $(H,\cdot)$ is called a \textit{subgroup} of $(G,\cdot)$\footnote{Sometimes we will denote that $(H,\cdot)$ is a subgroup of $(G,\cdot)$ by $H\leq G$.} if satisfies:
    \begin{enumerate}
        \item If $h_1,h_2\in H$, then $h_1\cdot h_2\in H$.
        \item $e\in H$.
        \item If $h\in H$, then $h^{-1}\in H$.
    \end{enumerate}
\end{definition}
\begin{prop}
    Let $(G,\cdot)$ be a group and $H\ne\emptyset$ be a subset of $G$. Then: $$(H,\cdot)\text{ is a subgroup}\iff h_1\cdot h_2^{-1}\in H\quad\forall h_1,h_2\in H.$$
\end{prop}
\begin{prop}
    If $(H,+)$ is a subgroup of $(\ZZ ,+)$, then $\exists n\in\ZZ $ such that $H=n\ZZ =\{nk:k\in\ZZ \}$.
\end{prop}
\begin{prop}
    Let $(G_i,*_i)$, $i=1,\ldots, n$, be groups. Then the product $$(G_1,*_1)\times\cdots\times(G_n,*_n)$$ induces a group with the operation $\cdot$ defined as $$(g_1,\ldots,g_n)\cdot(g_1',\ldots,g_n')=(g_1*_1g_1',\ldots,g_n*_ng_n'),$$ where $g_i,g_i'\in G_i$.
\end{prop}
\begin{definition}
    The \textit{order of a group $(G,\cdot)$} is the number of elements in its set, that is, $|G|$.
\end{definition}
\begin{lemma}
    Let $(G,\cdot)$ be a group and $\{(H_i,\cdot):i\in I\}$ be a set of subgroups of $(G,\cdot)$. Then if $$H=\displaystyle\bigcap_{i\in I}H_i,$$ we have that $(H,\cdot)$ is also a subgroup of $(G,\cdot)$.
\end{lemma}
\begin{definition}
    Let $(G,\cdot)$ be a group and $X\subseteq G$ be a subset of $G$. The \textit{subgroup of $(G,\cdot)$ generated by $X$}, $(\langle X\rangle,\cdot)$, is the smallest subgroup of $(G,\cdot)$ containing $X$, that is, $$\langle X\rangle=\bigcap_{X\subseteq H\leq G}H.$$
\end{definition}
\begin{definition}
    Let $(G,*)$ be a group, $g\in G$ and $n\in\ZZ $. We define $g^n$ as: 
    $$g^n=\left\{
    \begin{array}{lll}
        g*\overset{(n)}{\cdots}* g & \text{if} & n>0  \\
        1 & \text{if} & n=0  \\
        (g^{-1})*\overset{(|n|)}{\cdots}*(g^{-1}) & \text{if} & n<0 
    \end{array}\right.$$
\end{definition}
\begin{lemma}
    Let $(G,\cdot)$ be a group and $g\in G$. Then for all $n,m\in\ZZ $ we have:
    \begin{enumerate}
        \item $g^n\cdot g^m=g^{n+m}=g^m\cdot g^n$.
        \item $(g^n)^m=g^{nm}=(g^m)^n$.
    \end{enumerate}
\end{lemma}
\begin{prop}
    Let $(G,*)$ be a group and $X\subseteq G$ be a subset of $G$. Then: $$\langle X\rangle=\{e\}\cup\{g_1^{\alpha_1}*\cdots* g_n^{\alpha_n}:n\in\NN ,\alpha_i\in\ZZ ,g_i\in X\}.$$
\end{prop}
\begin{corollary}
    Let $(G,\cdot)$ be a group and $g\in G$. Then: $$\langle g\rangle=\{g^i:i\in\ZZ \}.$$
\end{corollary}
\begin{definition}
    Let $(G,\cdot)$ be a group and $g\in G$. A subgroup $(\langle g\rangle,\cdot)$ of $(G,\cdot)$ generated by a single element $g$ is called a \textit{cyclic group}.
\end{definition}
\begin{definition}
    Let $(G,\cdot)$ be a group and $g\in G$. The \textit{order of $g$} is $\text{ord}(g):=|\langle g\rangle|$.
\end{definition}
\begin{prop}
    Let $(G,\cdot)$ be a group and $g\in G$. Then: $$\text{ord}(g)=\min\{i\in\NN :g^i=e\}.$$ If no such $i$ exists, we say $\text{ord}(g)=\infty$.
\end{prop}
\begin{corollary}
    Let $n\in\NN $ such that $n>1$ and $\Bar{a}\in\ZZ /n\ZZ $. Then: $$\text{ord}(\Bar{a})=\frac{n}{\gcd(a,n)}.$$
\end{corollary}
\begin{lemma}
    Let $(G,\cdot)$ be a group and $g\in G$ such that $\text{ord}(g)=n$. Then:
    \begin{enumerate}
        \item $g^m=e\iff n\mid m$.
        \item $g^m=g^{m'}\iff m=m'\mod{n}$.
        \item If $0\leq i\leq n$, then $g^{-i}=(g^i)^{-1}=g^{n-i}$.
    \end{enumerate}
\end{lemma}
\begin{corollary}
    Let $(G_i,*_i)$, $i=1,\ldots, n$, be groups. For $i=1,\ldots,n$, let $g_i\in G_i$ and consider the element $g=(g_1,\ldots,g_n)\in(G_1,*_1)\times\cdots\times(G_n,*_n)$. Then: $$\text{ord}(g)=\lcm(\text{ord}(g_1),\ldots,\text{ord}(g_n)).$$
\end{corollary}
\subsubsection{Group morphisms}
\begin{definition}[Group morphism]\label{AS-groupmorphism}
    Let $(G,*)$, $(H,\cdot)$ be two groups. A \textit{group morphism from $(G,*)$ to $(H,\cdot)$} is a function $\phi:G\rightarrow H$ such that: $$\phi(g_1*g_2)=\phi(g_1)\cdot\phi(g_2)\quad\forall g_1,g_2\in G.$$
\end{definition}
\begin{lemma}
    Let $\phi:G_1\rightarrow G_2$ be a morphism between $(G_1,*)$ and $(G_2,\cdot)$. Then,
    \begin{enumerate}
        \item $\phi(e_1)=e_2$.
        \item $\phi(g^{-1})=\phi(g)^{-1}\quad\forall g\in G_1$.
        \item $\phi(g^n)=\phi(g)^n\quad\forall g\in G_1$ and $\forall n\in\ZZ $.
    \end{enumerate}
\end{lemma}
\begin{definition}
    We say a subgroup $(H,\cdot)$ of a group $(G,\cdot)$ is \textit{normal}, $H\lhd G$, if and only if $\forall h\in H$ and $\forall g\in G$, we have $g\cdot h\cdot g^{-1}\in H$.
\end{definition}
\begin{definition}
    Let $(G_1,*)$, $(G_2,\cdot)$ be two groups and $\phi:G_1\rightarrow G_2$ be a group morphism. The \textit{kernel of $\phi$} is: $$\ker\phi=\{g\in G_1:\phi(g)=e_2\}.$$ The \textit{image of $\phi$} is: $$\im\phi=\{h\in G_2:\phi(g)=h\text{ for some }g\in G_1\}.$$
\end{definition}
\begin{prop}
    Let $(G_1,*)$, $(G_2,\cdot)$ be two groups and $\phi:G_1\rightarrow G_2$ be a group morphism. Then:
    \begin{enumerate}
        \item $(\ker\phi,*)$ is a normal subgroup of $(G_1,*)$ and $(\im\phi,\cdot)$ is a subgroup of $(G_2,\cdot)$.
        \item Let $g,g'\in G_1$. The following statements are equivalent:
        \begin{enumerate}
            \item $\phi(g)=\phi(g')$.
            \item $g*g'^{-1}\in\ker\phi$.
            \item $g'^{-1}*g\in\ker\phi$.
        \end{enumerate}
        \item $\phi$ is injective if and only if $\ker\phi=\{e_1\}$.
        \item $\phi$ is surjective if and only if $\im\phi=G_2$.
    \end{enumerate}
\end{prop}
\begin{definition}
    Let $(G,*)$, $(H,\cdot)$ be two groups. An \textit{isomorphism between $(G,*)$ and $(H,\cdot)$} is a bijective morphism between these groups. In this case, we say that $(G,*)$, $(H,\cdot)$ are \textit{isomorphic}: $G\cong H$.
\end{definition}
\begin{prop}
    Let $(G_1,\cdot_1)$, $(G_2,\cdot_2)$, $(G_3,\cdot_3)$ be three groups and $\phi:G_1\rightarrow G_2$, $\psi:G_2\rightarrow G_3$ be two group morphisms. Then the composition $\psi\circ\phi$ is also a group morphism.
\end{prop}
\begin{prop}
    Let $(G_1,*)$, $(G_2,\cdot)$ be groups and let $\phi: G_1\rightarrow G_2$ be an isomorphism. Then $\phi^{-1}: G_2\rightarrow G_1$ is also an isomorphism.
\end{prop}
\begin{theorem}[Classification of cyclic groups]
    Let $(G,\cdot)$ be a group and $g\in G$ be an element such that $\langle g\rangle=G$.
    \begin{itemize}
        \item If $|G|=\infty$, then $G\cong\ZZ $. We can define the isomorphism as follows: 
        \begin{align*}
            \phi:\ZZ &\longrightarrow G\\
            k&\longmapsto g^k
        \end{align*}
        \item If $|G|=n$, then $G\cong\ZZ /n\ZZ $. We can define the isomorphism as follows: 
        \begin{align*}
            \phi:\ZZ /n\ZZ &\longrightarrow G\\
            \Bar{k}&\longmapsto g^k
        \end{align*}
    \end{itemize}
\end{theorem}
\begin{corollary}
    Let $(G,\cdot)$ be a group and $g\in G$ be such that $\langle g\rangle=G$. Then all subgroups of $G$ are cyclic. Moreover:
    \begin{itemize}
        \item If $|G|=\infty$, subgroups of $(G,\cdot)$ are of the form $\langle g^n\rangle$, $n\in\NN \cup\{0\}$. 
        \item If $|G|=n$, then there is a unique subgroup $(H,\cdot)$ of $(G,\cdot)$ for every divisor $d>0$ of $n$. In fact, if $n=dq$, then $H=\langle g^q\rangle$ and $|H|=d$.
    \end{itemize}
\end{corollary}
\begin{definition}
    Let $X$ be a set. We define the \textit{symmetric group $(S(X),\circ)$} as: $$S(X)=\{f:X\rightarrow X:f\text{ is bijective}\}\footnote{Observe that if $X=\{1,\ldots,n\}$, then $S(X)=S_n$.}.$$ 
\end{definition}
\begin{definition}
    Let $(G,\cdot)$ be a group. We define the functions:
    \begin{align*}
        \ell_g:G&\longrightarrow G&r_g:G&\longrightarrow G\\
        x&\longmapsto g\cdot x &x&\longmapsto x\cdot g
    \end{align*}
\end{definition}
\begin{lemma}
    Let $(G,\cdot)$ be a group. The functions $\ell_g$, $r_g$ are bijective and its inverses are $\ell_{g^{-1}}$, $r_{g^{-1}}$, respectively.
\end{lemma}
\begin{prop}
    Let $(G,\cdot)$ be a group. We define the functions:
    \begin{align*}
        \phi:G&\longrightarrow S(G)&\psi:G&\longrightarrow S(G)\\
        g&\longmapsto \ell_g &g&\longmapsto r_{g^{-1}}
    \end{align*}
    Then, $\phi$ and $\psi$ are injective group morphisms.
\end{prop}
\begin{theorem}[Cayley's theorem]
    Let $(G,\cdot)$ be a group. Then, there is an injective morphism: $$\phi:G\longrightarrow S(G)$$
\end{theorem}
\begin{corollary}
    If $(G,\cdot)$ is a group with $|G|=n$, then $(G,\cdot)$ is isomorphic to a subgroup of $(S_n,\circ)$.
\end{corollary}
\subsubsection{Cosets}
\begin{definition}\label{AS_equiv} 
    Let $(G,\cdot)$ be a finite group, $(H,\cdot)$ be a subgroup of $(G,\cdot)$ and $g_1,g_2\in G$. 
    \begin{itemize}
        \item We say $g_1\sim g_2\iff g_1\cdot g_2^{-1}\in H$.
        \item We say $g_1\approx g_2\iff g_2^{-1}\cdot g_1\in H$.
    \end{itemize}
\end{definition}
\begin{lemma}
    Let $(G,\cdot)$ be a finite group and $(H,\cdot)$ be a subgroup of $(G,\cdot)$. Then:
    \begin{enumerate}
        \item $\sim$ and $\approx$ are equivalence relations.
        \item If $g\in G$, then 
        \begin{gather*}
            [g]_\sim=H\cdot g=\{h\cdot g:h\in H\},\\ [g]_\approx=g\cdot H=\{g\cdot h':h'\in H\}.
        \end{gather*}
        Usually we say that $H\cdot g$ are the \textit{right cosets in $G$} and $g\cdot H$, the \textit{left cosets in $G$}. 
    \end{enumerate}
\end{lemma}
\begin{definition}
    Let $(G,\cdot)$ be a finite group and $(H,\cdot)$ be a subgroup of $(G,\cdot)$. We define the \textit{set of left cosets} and the \textit{set of right cosets} as follows:
    $$\quot{G}{\sim}=\{H\cdot g:g\in G\},\quad\quot{G}{\approx}=\{g\cdot H:g\in G\}.$$
\end{definition}
\begin{prop}
    Let $(G,\cdot)$ be a group and $(H,\cdot)$ be a subgroup of $(G,\cdot)$. The following statements are equivalent:
    \begin{enumerate}
        \item $H\lhd G$.
        \item $g\cdot H=H\cdot g\quad\forall g\in G$.
    \end{enumerate}
\end{prop}
\begin{theorem}[Lagrange's theorem]
    Let $(G,\cdot)$ be a finite group and $(H,\cdot)$ be a subgroup of $(G,\cdot)$. Then: $$|H|\mid|G|$$
\end{theorem}
\begin{definition}
    Let $(G,\cdot)$ be a finite group and $(H,\cdot)$ be a subgroup of $(G,\cdot)$. We define the \textit{index of $(H,\cdot)$ in $(G,\cdot)$} as: $$[G:H]:=\frac{|G|}{|H|}$$
\end{definition}
\begin{corollary}
    Let $(G,\cdot)$ be a finite group and $(H,\cdot)$ be a subgroup of $(G,\cdot)$. Then: $$[G:H]=\left|\quot{G}{\sim}\right|=\left|\quot{G}{\approx}\right|$$
\end{corollary}
\begin{corollary}
    Let $(G,\cdot)$ be a finite group. 
    \begin{enumerate}
        \item If $g\in G$, then $\text{ord}(g)\mid |G|$.
        \item If $|G|$ is prime, then $(G,\cdot)$ is cyclic.
        \item If $(H,\cdot)$ and $(K,\cdot)$ are subgroups of $(G,\cdot)$ and $\gcd(|H|,|K|)=1$, then $H\cap K=\{e\}$.
    \end{enumerate}
\end{corollary}
\begin{definition}[Quotient group]
    Let $(G,\cdot)$ be a finite group and $(H,\cdot)$ be a subgroup of $(G,\cdot)$ such that $H\lhd G$. We define the \textit{quotient group $\left(G/H,*\right)$} as $$\quot{G}{H}=\quot{G}{\sim}=\quot{G}{\approx}$$ and 
\begin{align*}
    *:\quot{G}{H}\times\quot{G}{H}&\longrightarrow\quot{G}{H}\\
    (g_1\cdot H,g_2\cdot H)&\longmapsto (g_1\cdot g_2)\cdot H
\end{align*}
\end{definition}
\begin{lemma}
Let $(G,\cdot)$ be a finite group and $(H,\cdot)$ be a subgroup of $(G,\cdot)$ such that $H\lhd G$. The projection 
\begin{align*}
    \pi:G&\longrightarrow\quot{G}{H}\\
    g&\longmapsto[g]=g\cdot H
\end{align*}
is a group morphism.
\end{lemma}
\subsubsection{Isomorphism theorems}
\begin{theorem}[First isomorphism theorem]
    Let $(G_1,*)$, $(G_2,\cdot)$ be groups, $\phi:G_1\rightarrow G_2$ be a group morphism and $(H,*)$ be a subgroup of $(G_1,*)$ such that $H\lhd G_1$. If $(H,*)$ is a subgroup of $(\ker\phi,*)$, then there exists a unique group morphism $\psi:G_1/H\rightarrow G_2$ such that the diagram of figure \ref{theorem1} is commutative, that is, $\phi=\psi\circ\pi$.
    \illustration{0.35}{Images/theorem1}{}{theorem1}
    In particular, if $H=\ker\phi$, then $\psi$ is injective and therefore there is an isomorphism $\psi:G_1/\ker\phi\rightarrow\im\phi$.
\end{theorem}
\begin{theorem}
    Let
    \begin{equation*}
        \begin{array}{r@{\hspace{0.5\tabcolsep}}c@{\hspace{0.5\tabcolsep}}c@{\hspace{0.5\tabcolsep}}c}
            \phi:&\ZZ &\longrightarrow&\displaystyle\quot{\ZZ }{n\ZZ }\times\quot{\ZZ }{m\ZZ }\\
            &1&\longmapsto&(\Bar{1},\Bar{1})
        \end{array}
    \end{equation*}
    be a group morphism. Then, $\phi$ induces a morphism $\psi:\ZZ /nm\ZZ \rightarrow\ZZ /n\ZZ \times\ZZ /m\ZZ $. Moreover, $\psi$ is injective if and only if $\gcd(n,m)=1$ and in this case $\psi$ is an isomorphism. 
\end{theorem}
\begin{corollary}
    Let $n,m\in\ZZ $ be two coprime integers and $a,b\in\ZZ $. The system of congruences $$\left\{\begin{array}{l}
        x\equiv a\mod{n}  \\
        x\equiv b\mod{m}  \\
    \end{array}\right.$$ has solutions and these are of the form $x\equiv c\mod{nm}$, where $c\equiv a\mod{n}$ and $c\equiv b\mod{m}$.
\end{corollary}
\begin{definition}
    Let $(G,\cdot)$ be a group and $(H,\cdot)$, $(K,\cdot)$ be subgroups of $(G,\cdot)$. We define the \textit{products of group subsets $K$, $H$} as the sets 
    \begin{gather*}
        H\cdot K=\{h\cdot k:h\in H,k\in K\},\\
        K\cdot H=\{k\cdot h:k\in K,h\in H\}.
    \end{gather*}
\end{definition}
\begin{prop}
    Let $(G,\cdot)$ be a group and $(H,\cdot)$, $(K,\cdot)$ be subgroups of $(G,\cdot)$ such that $H\lhd G$. Then, $(H\cdot K,\cdot)$ is a subgroup of $(G,\cdot)$ and $H\cdot K=K\cdot H$.
\end{prop}
\begin{prop}
    Let $(G,\cdot)$ be a group and $(H,\cdot)$, $(K,\cdot)$ be subgroups of $(G,\cdot)$ such that $H\cap K=\{e\}$. If $H,K\lhd G$, then the function 
    \begin{align*}
        \phi:H\times K&\longrightarrow H\cdot K\\
        (h,k)&\longmapsto h\cdot k
    \end{align*}
    is an isomorphism. In particular, $\forall h\in H$ and $\forall k\in K$, $h\cdot k=k\cdot h$.
\end{prop}
\begin{theorem}[Second isomorphism theorem]
    Let $(G,\cdot)$ be a group and $(H,\cdot)$, $(K,\cdot)$ be subgroups of $(G,\cdot)$ such that $H\lhd G$. Then $H\cap K\lhd K$ and $$\quot{K}{H\cap K}\cong\quot{H\cdot K}{H}$$
\end{theorem}
\begin{corollary}
    Let $(G,\cdot)$ be a group and $(H,\cdot)$, $(K,\cdot)$ be subgroups of $(G,\cdot)$. Then: $$|H||K|=|H\cap K||H\cdot K|$$
\end{corollary}
\begin{lemma}
    Let $(G,\cdot)$ be a group and $(H,\cdot)$, $(K,\cdot)$ be subgroups of $(G,\cdot)$ such that $H\lhd G$ and $H\subseteq K$. Then $H\lhd K$, $(K/H,*)$ is a subgroup of $(G/H,*)$ and moreover $$\quot{K}{H}\lhd\quot{G}{H}\iff K\lhd G$$
\end{lemma}
\begin{theorem}[Correspondence theorem]
    Let $(G,\cdot)$ be a group and $(H,\cdot)$ be a subgroup of $(G,\cdot)$ such that $H\lhd G$. Then there is a bijection $\phi$ from the set $\mathcal{G}$ of all subgroups $(K,\cdot)$ of $(G,\cdot)$ such that $H\subseteq K$ onto the set $\mathcal{H}$ of all subgroups $\left(K/H,*\right)$ of $\left(G/H,*\right)$. More precisely, the bijection is:
    \begin{align*}
        \phi:\mathcal{G}&\longrightarrow\mathcal{H}\\
        K&\longmapsto K/H
    \end{align*}
\end{theorem}
\begin{theorem}[Third isomorphism theorem]
    Let $(G,\cdot)$ be a group and $(H,\cdot)$, $(K,\cdot)$ be subgroups of $(G,\cdot)$ such that $H,K\lhd G$ and $H\subseteq K$. Then $K/H\lhd G/H$ and $$\quot{\left(\quot{G}{H}\right)}{\left(\quot{K}{H}\right)}\cong\quot{G}{K}$$
\end{theorem}
\subsubsection{Group actions}
\begin{definition}
    Let $X$ be a set and $(G,\cdot)$ be a group. A \textit{(left) group action of $(G,\cdot)$ on $X$} is a function 
    \begin{align*}
        *:G\times X&\longrightarrow X\\
        (g,x)&\longmapsto g*x
    \end{align*}
    satisfying the following properties:
    \begin{enumerate}
        \item $e*x=x$, $\forall x\in X$.
        \item $(g_1\cdot g_2)*x=g_1*(g_2*x)$, $\forall x\in X$ and $\forall g_1,g_2\in G$.
    \end{enumerate}
    A set $X$ together with an action $*$ of $(G,\cdot)$ is usually called a \textit{(left) $G$-set}.
\end{definition}
\begin{lemma}
    Let $(G,\cdot)$ be a group and $X$ be a $G$-set. For all $g\in G$ the function
    \begin{align*}
        \ell_g:X&\longrightarrow X\\
        x&\longmapsto g*x
    \end{align*} is bijective an its inverse is $\ell_{g^{-1}}$.
\end{lemma}
\begin{definition}
    Let $(G,\cdot)$ be a group and $X$ be a $G$-set. For all $x,y\in X$, we say $x\backsim y\iff\exists g\in G:y=g*x$.
\end{definition}
\begin{lemma}
    The relation $\backsim$ is an equivalence relation.
\end{lemma}
\begin{definition}
    Let $(G,\cdot)$ be a group and $X$ be a $G$-set. If $x\in X$, we define the \textit{orbit of $x$} as: $$\mathcal{O}_x=[x]_\backsim=\{g*x:g\in G\}$$
\end{definition}
\begin{definition}
    Let $(G,\cdot)$ be a group and $X$ be a $G$-set. For $x\in X$, we define the \textit{stabilizer of $(G,\cdot)$ with respect to $x$} as the set: $$G_x=\{g\in G:g*x=x\}$$
\end{definition}
\begin{prop}
    Let $(G,\cdot)$ be a group and $X$ be a $G$-set. For all $x\in X$, $(G_x,\cdot)$ is a subgroup of $(G,\cdot)$.
\end{prop}
\begin{theorem}[Orbit-stabilizer theorem]
    Let $(G,\cdot)$ be a group, $X$ be a $G$-set and $x\in X$. The surjective function
    \begin{align*}
        \phi:G&\longrightarrow \mathcal{O}_x\\
        g&\longmapsto g*x
    \end{align*}
    induces a bijective function $\psi:\quot{G}{\!\!\approx}\;\rightarrow\mathcal{O}_x$, where $\approx$ is the equivalence relation $g_1\approx g_2\iff g_2^{-1}\cdot g_1\in G_x$ $\forall g_1,g_2\in G$\footnote{Note that the notation $\approx$ for the equivalence relation correspond with the one defined in definition \ref{AS_equiv}.}. In particular, if $G$ is finite: $$|\mathcal{O}_x|=|[G:G_x]|$$
\end{theorem}
\begin{corollary}[Orbits formula]
    Let $(G,\cdot)$ be a finite group and $X$ be a finite $G$-set. If $x_1,\ldots,x_m$ are the elements of $X$ and $|\mathcal{O}_{x_i}|=1$ for $i=1,\ldots,r$, then:
    \begin{equation}
        |X|=r+\sum_{i=r+1}^m|\mathcal{O}_{x_i}|=r+\sum_{i=r+1}^m|[G:G_{x_i}]|
        \label{EA-obritsformula}
    \end{equation}
\end{corollary}
\subsubsection{Applications of orbits formula}
\begin{theorem}[Cauchy's theorem]
    Let $(G,\cdot)$ be a finite group of order $n$ and $p\in\PP$. If $p\mid n$, then $(G,\cdot)$ has an element of order $p$.
\end{theorem}
\begin{corollary}
    Let $p$ be an odd prime number. Then groups of order $2p$ are isomorphic to $(\ZZ /2p\ZZ ,+)$ or $(\text{D}_{2p},\circ)$\footnote{See the end of section \ref{AS-examples}.}.
\end{corollary}
\begin{prop}
    Let $(G,\cdot)$ be a group. The function 
    \begin{align*}
        G\times G&\longrightarrow G\\
        (g,x)&\longmapsto g\cdot x\cdot g^{-1}
    \end{align*} is an action of $(G,\cdot)$ over itself. It is called the \textit{conjugation action}.
\end{prop}
\begin{definition}[Center of a group]
    Let $(G,\cdot)$ be a group. We define the \textit{center of $(G,\cdot)$} as $$Z(G)=\{z\in G:z\cdot g=g\cdot z\;\forall g\in G\}\footnote{Note that, by orbits formula \eqref{EA-obritsformula}, if we consider the conjugation action we have: $$|G|=|Z(G)|+\sum_{|\mathcal{O}_x|>1}|\mathcal{O}_x|$$}.$$
\end{definition}
\begin{prop}
    Let $p\in\PP$ and $(G,\cdot)$ be a finite group of order $p^n$ for some $n\geq 1$. Then, $|Z(G)|>1$.
\end{prop}
\begin{lemma}
    Let $(G,\cdot)$ be a group and $(H,\cdot)$ be a subgroup of $(G,\cdot)$. Consider the application 
    \begin{align*}
        H\times\quot{G}{\approx}&\longrightarrow \quot{G}{\approx}\\
        (h,g\cdot H)&\longmapsto(h\cdot g)\cdot H
    \end{align*}
    This application defines an action of the subgroup $(H,\cdot)$ over the set $\quot{G}{\!\!\approx}$.
    \label{AS_action1}
\end{lemma}
\begin{definition}
    Let $(G,\cdot)$ be a group and $(H,\cdot)$ be a subgroup of $(G,\cdot)$. The \textit{normalizer of $(H,\cdot)$ in $(G,\cdot)$} is $$N_G(H)=\{g\in G:g\cdot h\cdot g^{-1}\in H\;\forall h\in H\}.$$
\end{definition}
\begin{lemma}
    Let $(G,\cdot)$ be a group and $(H,\cdot)$ be a subgroup of $(G,\cdot)$. Then, $(N_G(H),\cdot)$ is a subgroup of $(G,\cdot)$ containing $H$ and, moreover, $H\lhd N_G(H)$.
\end{lemma}
\begin{corollary}
    Let $(G,\cdot)$ be a finite group and $(H,\cdot)$ be a subgroup of $(G,\cdot)$. Then, by orbits formula applied to action defined on lemma \ref{AS_action1}, we have: $$[G:H]=[N_G(H):H]+\sum_{|\mathcal{O}_x|>1}|\mathcal{O}_x|.$$
\end{corollary}
\begin{prop}
    Let $(G,\cdot)$ be a group of order $n\in\NN $, $p\in\PP$ such that $p\mid n$ and $(H,\cdot)$ be a subgroup of $(G,\cdot)$ of order $p^i$, $i\geq 1$. Suppose $p\mid[G:H]$. Then, $p\mid[N_G(H):H]$.
\end{prop}
\subsubsection{Sylow's theorems}
\begin{corollary}
    Let $(G,\cdot)$ be a group of order $n\in\NN $, $p\in\PP$ and $(H,\cdot)$ be a subgroup of $(G,\cdot)$ such that $|H|=p^i$, $i\geq 0$. Suppose $p\mid[G:H]$. Then, there is a subgroup $(H',\cdot)$ of $(G,\cdot)$ such that $H\subset H'$ and $|H'|=p^{i+1}$. Moreover, $H\lhd H'$ and $H'/H\cong\ZZ /p\ZZ $.
\end{corollary}
\begin{theorem}[First Sylow theorem]
    Let $(G,\cdot)$ be a finite group and $p\in\PP$. Suppose $|G|=p^r m$, where $r\geq 0$ and $\gcd(p,m)=1$. Then, there is a subgroup $(K,\cdot)$ of $(G,\cdot)$ of order $p^r$. Moreover there is a chain of subgroups $(H_i,\cdot)$ satisfying: $$\{e\}=H_0\lhd H_1\lhd\cdots\lhd H_r=K,$$ such that $H_{i+1}/H_i\cong\ZZ /p\ZZ $ for $0\leq i<r$.
\end{theorem}
\begin{definition}
    Let $p\in\PP$. A group $(G,\cdot)$ is a \textit{$p$-group} if $|G|=p^r$, for some $r\in\NN $.
\end{definition}
\begin{definition}
    Let $p\in\PP$ and $(G,\cdot)$ be a group. A \textit{Sylow $p$-subgroup} is a $p$-subgroup of $(G,\cdot)$ of maximum order.
\end{definition}
\begin{definition}
    Let $(G,\cdot)$ be a finite group. We say $(G,\cdot)$ is \textit{soluble} if there is a chain of subgroups $(H_i,\cdot)$ of $(G,\cdot)$ satisfying: $$\{e\}=H_0\lhd H_1\lhd\cdots\lhd H_r=K,$$ and such that the subgroups $(H_{i+1}/H_i,*)$, $0\leq i<r$, are cyclic. 
\end{definition}
\begin{theorem}[Second Sylow theorem]
    Let $(G,\cdot)$ be a finite group and $p\in\PP$. Suppose $|G|=p^r m$, where $r\geq 0$ and $\gcd(p,m)=1$. Let $(K,\cdot)$ be a Sylow $p$-subgroup of $(G,\cdot)$. Then, if $(H,\cdot)$ is a subgroup of $(G,\cdot)$ of order $p^i$, $\exists g\in G$ such that $g\cdot H\cdot g^{-1}\subseteq K$. In particular two different Sylow $p$-subgroups $(K_1,\cdot)$ and $(K_2,\cdot)$ are conjugate, that is, there exists an element $g\in G$ such that $g\cdot K_1\cdot g^{-1}=K_2$.
\end{theorem}
\begin{theorem}[Third Sylow theorem]
    Let $(G,\cdot)$ be a finite group and $p\in\PP$. Suppose $|G|=p^r m$, where $r\geq 0$ and $\gcd(p,m)=1$. Let $(K,\cdot)$ be a Sylow $p$-subgroup of $(G,\cdot)$ and $n_p$ be the number of different Sylow $p$-subgroups of $(G,\cdot)$. Then, $n_p=[G:N_G(K)]$. Therefore, $n_p\mid m$ and $n_p\equiv1\mod p$. 
\end{theorem}
\begin{corollary}
    Let $p,q\in\PP$ be such that $p<q$ and $q\not\equiv 1\mod p$. If $(G,\cdot)$ is a group of order $pq$, then $G\cong\quot{\ZZ}{pq\ZZ}$.
\end{corollary}
\subsubsection{Examples of groups}\label{AS-examples}
Let $n\in\NN$ and $p\in\PP$.
\begin{itemize}
    \item $(\ZZ,+)$, $(\ZZ/n\ZZ,+)$, $(\QQ,+)$, $(\RR,+)$, $(\CC,+)$
    \item $((\ZZ/p\ZZ)^*,\cdot)$, $(\QQ^*,\cdot)$, $(\RR^*,\cdot)$, $(\CC^*,\cdot)$
    \item $(S_n,\circ)$
    \item $(A_n,\circ)$, where $A_n=\{\sigma\in S_n:\varepsilon(\sigma)=1\}$. Note that $|A_n|=\frac{S_n}{2}=\frac{n!}{2}$.
    \item $(\text{GL}_n(\mathbb{A}),\cdot)$, where $\text{GL}_n(\mathbb{A})=\{\mathblack{M}\in\mathcal{M}_n(\mathbb{A}):\mathblack{M}\text{ is invertible}\}$ and $\mathbb{A}=\ZZ,\;\QQ,\;\RR\text{ or }\CC$.
    \item $(\text{SL}_n(\mathbb{A}),\cdot)$, where $\text{SL}_n(\mathbb{A})=\{\mathblack{M}\in\text{GL}_n(\mathbb{A}):\det\mathblack{M}=1\}$ and $\mathbb{A}=\ZZ,\;\QQ,\;\RR\text{ or }\CC$.
    \item $(\text{D}_{2n},\circ)$, where $\text{D}_{2n}$ is the set of rotations and reflections that leave invariant the regular polygon of $n$ vertices centered at origin. It can be seen that $\text{D}_{2n}=\langle r,s:\text{ord}(r)=n,\text{ord}(s)=2,r\circ s=s\circ r^{-1}\rangle$. This group is called the \textit{dihedral group}.
    \item $(\text{Q}_8,\cdot)$, where $\text{Q}_8=\langle a,b:\text{ord}(a)=\text{ord}(b)=4,b\cdot a=a^{-1}\cdot b\rangle$. This group is called the \textit{quaternion group}.
\end{itemize}
\subsection{Rings and fields}\label{AS-R}
\subsubsection{Rings, subrings and ring morphisms}
\begin{definition}[Ring]
    A \textit{ring} is a set $R$ equipped with two binary operations (called addition and multiplication):
    \begin{align*}
        +:R\times R&\longrightarrow R&\cdot:R\times R&\longrightarrow R\\
        (r_1,r_2)&\longmapsto r_1+ r_2&(r_1,r_2)&\longmapsto r_1\cdot r_2
    \end{align*}
    satisfying the following properties:
    \begin{enumerate}
        \item $(R,+)$ is an abelian group.
        \item $(R,\cdot)$ satisfies\footnote{Some definitions state that the commutative property is not necessary to define a ring. However, in these notes we will take the definition given.}:
        \begin{enumerate}
            \item Associativity: $$(r_1\cdot r_2)\cdot r_3=r_1\cdot(r_2\cdot r_3)\quad\forall r_1,r_2,r_3\in R.$$
            \item Identity element\footnote{It is common to denote the additive identity element as 0 and the multiplicative identity element as 1.}: $$\exists 1\in R:1\cdot r=r\cdot 1=r\quad\forall r\in R.$$
            \item Commutativity: $$r_1\cdot r_2=r_2\cdot r_1\quad\forall r_1,r_2\in R.$$
        \end{enumerate}
        \item Multiplication is distributive with respect to addition: $$(r_1+r_2)\cdot r_3=r_1\cdot r_3+r_2\cdot r_3\quad\forall r_1,r_2,r_3\in R.$$
    \end{enumerate}
    In this context we say $(R,+,\cdot)$ is a ring.
\end{definition}
\begin{definition}
    A \textit{noncommutative ring} is a ring whose multiplication is not commutative.
\end{definition}
\begin{definition}[Field]
    Let $(R,+,\cdot)$ be a ring. If every nonzero element of $R$ has a multiplicative inverse (that is, $(R,\cdot)$ is an abelian group), we say that $R$ is a \textit{field}.
\end{definition}
\begin{prop}
    Let $(R_i,+_i,\cdot_i)$, $i=1,\ldots, n$, be rings. Then the product $$(R_1,+_1,\cdot_1)\times\cdots\times(R_n,+_n,\cdot_n)$$ induces a ring with operations $+$ and $\cdot$ defined as
    \begin{gather*}
        (r_1,\ldots,r_n)+(r_1',\ldots,r_n')=(r_1+_1r_1',\ldots,r_n+_nr_n'),\\
        (r_1,\ldots,r_n)\cdot(r_1',\ldots,r_n')=(r_1\cdot_1r_1',\ldots,r_n\cdot_nr_n'),
    \end{gather*}
    where $r_i,r_i'\in R_i$.
\end{prop}
\begin{prop}
    Let $(R,+,\cdot)$ be a ring. We define the  \textit{set of polynomials over the ring $(R,+,\cdot)$} as: $$R[x]:=\{r_0+r_1\cdot x+\cdots+r_n\cdot x^n:r_i\in R\ \forall i\text{ and }n\geq 0\}.$$ Moreover, $(R[x],+,\cdot)$ is a ring.
\end{prop}
\begin{definition}
    A ring $(R,+,\cdot)$ is a \textit{Boolean ring} if $r^2=r$ $\forall r\in R$.
\end{definition}
\begin{lemma}
    Let $(R,+,\cdot)$ be a ring. Then:
    \begin{enumerate}
        \item The multiplicative identity element is unique.
        \item $\forall r\in R$, $0\cdot r=0$.
        \item $\forall r\in R$, $(-1)\cdot r=-r$, where $-1$ is the additive inverse of 1.
        \item $\forall r,s\in R$, $(-r)\cdot s=-(r\cdot s)$ and $(-r)\cdot (-s)=r\cdot s$.
    \end{enumerate}
\end{lemma}
\begin{definition}[Subring]
    Let $(R,+,\cdot)$ be a ring and $S\subseteq R$ be a subset of $R$. $(S,+,\cdot)$ is called a \textit{subring} of $(R,+,\cdot)$ if satisfies:
    \begin{enumerate}
        \item $(S,+)$ is a subgroup of $(R,+)$.
        \item $\forall s_1,s_2\in S$, $s_1\cdot s_2\in S$.
        \item $1\in S$.
    \end{enumerate}
\end{definition}
\begin{definition}[Ring morphism]
    Let $(R,+,\cdot)$, $(S,\oplus,\odot)$ be two rings. A \textit{ring morphism from $(R,+,\cdot)$ to $(S,\oplus,\odot)$} is a function $\phi:R\rightarrow S$ such that:
    \begin{enumerate}
        \item $\phi(r_1+ r_2)=\phi(r_1)\oplus\phi(r_2)\quad\forall r_1,r_2\in R$\footnote{That is, $\phi$ is a group morphism between groups $(R,+)$ and $(S,\oplus)$.}.
        \item $\phi(r_1\cdot r_2)=\phi(r_1)\odot\phi(r_2)\quad\forall r_1,r_2\in R$.
        \item $\phi(1_R)=1_S$.
    \end{enumerate}
\end{definition}
\begin{lemma}
    Let $(R,+,\cdot)$, $(S,\oplus,\odot)$ be two rings and $\phi:R\rightarrow S$ be a ring morphism. Then, knowing that $\ker\phi=\{r\in R:f(r)=0\}$, then:
    \begin{enumerate}
        \item $(\ker\phi,+)$ is a subgroup of $(R,+)$.
        \item $\forall k\in\ker\phi$ and $\forall r\in R$, $k\cdot r\in\ker\phi$.
    \end{enumerate}
\end{lemma}
\begin{prop}
    Let $(R,+,\cdot)$, $(S,\oplus,\odot)$ be two rings and $\phi:R\rightarrow S$ be a ring morphism. Then:
    \begin{enumerate}
        \item $f(0)=0$.
        \item $\forall r\in R$, $f(-r)=-f(r)$.
        \item If $r\in R$ has a multiplicative inverse, then $f(r)$ so it has and, moreover, $f(r^{-1})=f(r)^{-1}$.
    \end{enumerate}
\end{prop}
\begin{prop}
    Let $(R_1,+_1,\cdot_1)$, $(R_2,+_2,\cdot_2)$ and $(R_3,+_3,\cdot_3)$ be rings and $\phi:R_1\rightarrow R_2$, $\psi:R_2\rightarrow R_3$ be two ring morphisms. Then, the composition $\psi\circ\phi$ is also a ring morphism.
\end{prop}
\begin{prop}
    Let $(R,+,\cdot)$, $(S,\oplus,\odot)$ be rings and let $\phi: R\rightarrow S$ be a bijective ring morphism. Then $\phi^{-1}: S\rightarrow R$ is also a bijective ring morphism.
\end{prop}
\subsubsection{Ideals}
\begin{definition}[Ideal]
    Let $(R,+,\cdot)$ be a ring. A subgroup $(I,+)$ of $(R,+)$ is an \textit{ideal} if $\forall x\in I$ and $\forall r\in R$, $x\cdot r\in I$.
\end{definition}
\begin{lemma}[Principal ideal]
    Let $(R,+,\cdot)$ be a ring and $a\in R$. The set $$(a):=a\cdot R=\{a\cdot r:r\in R\}$$ is an ideal of $(R,+,\cdot)$ and it is called \textit{principal ideal generated by $a$}.
\end{lemma}
\begin{prop}
    Let $(R,+,\cdot)$ be a nonzero ring. $R$ is a field if and only if $(R,+,\cdot)$ has only two ideals: $\{0\}$ and $R$.
\end{prop}
\begin{definition}
    Let $(R,+,\cdot)$ be a ring. An element $r\in R$ is a \textit{unit} if it has a multiplicative inverse. The set of units in $(R,+,\cdot)$ is denoted by $R^*$ (or $U(R)$) and $(R^*,\cdot)$ is a group called \textit{multiplicative group of $(R,+,\cdot)$}.
\end{definition}
\begin{lemma}
    Let $(R,+,\cdot)$, $(S,\oplus,\odot)$ be rings and $u\in R^*$. Then,
    \begin{enumerate}
        \item If $r\in R$, then $r\cdot R=r\cdot u\cdot R$.
        \item If $f:R\rightarrow S$ is a ring morphism, then $f:R^*\rightarrow S^*$ is a group morphism.
    \end{enumerate}
\end{lemma}
\begin{prop}
    Let $K$ be a field. Then, all ideals of $K[x]$ are principal. Moreover if $I\ne\{0\}$ is an ideal of $K[x]$, exists a monic polynomial $p(x)\in K[x]$ such that $I=p(x)\cdot K[x]$.
\end{prop}
\begin{prop}
    Let $(R,+,\cdot)$ be a ring and $I$, $J$ be a ideals of $(R,+,\cdot)$. Then the sets
    \begin{gather*}
        I\cap J:=\{x:x\in I,\;x\in J\},\\
        I+J:=\{x+y:x\in I,\;y\in J\},\\
        I\cdot J:=\left\{\sum_{i=1}^nx_iy_i:n\geq 0,\;x_i\in I,\;y_i\in J\right\},
    \end{gather*}
    are all ideals. In particular $I\cap J$ is the largest ideal contained in $I$ and $J$, and $I+J$ is the smallest ideal containing $I$ and $J$.
\end{prop}
\begin{definition}
    Let $(R,+,\cdot)$ be a ring and $I$, $J$ be a ideals of $(R,+,\cdot)$. If $I=(a)$ and $J=(b)$ for some $a,b\in R$, then we define $(a,b)$ as: $$(a,b)=(a)+(b)$$
\end{definition}
\begin{definition}
    A ring is \textit{Noetherian} if all its ideals are finitely generated.
\end{definition}
\begin{theorem}[Hilbert's basis theorem]
    If $(R,+,\cdot)$ is a Noetherian ring, then $(R[x_1,\ldots,x_n],+,\cdot)$ is a Noetherian ring.
\end{theorem}
\begin{lemma}
    Let $(R,+,\cdot)$, $(S,\oplus,\odot)$ be two rings and $\phi:R\rightarrow S$ be a ring morphism. Then:
    \begin{enumerate}
        \item $\ker\phi$ is an ideal of $(R,+,\cdot)$.
        \item $\im\phi$ is a subring of $(S,\oplus,\odot)$.
    \end{enumerate}
\end{lemma}
\subsubsection{Ideal quotient}
\begin{definition}
    Let $(R,+,\cdot)$ be a ring and $I$ be an ideal of $(R,+,\cdot)$. For all $r_1,r_2\in R$, we say $r_1\sim r_2\iff r_1-r_2\in I$. Since $(I,+)$ is a subgroup of $(R,+)$, $\sim$ is an equivalence relation and we denote by $$\quot{R}{I}:=\{x+I:x\in R\}$$ the set of equivalence classes.
\end{definition}
\begin{prop}
    Let $(R,+,\cdot)$ be a ring and $I$ be an ideal of $(R,+,\cdot)$. Then $\quot{R}{I}$ is a ring with operations defined as:
    \begin{itemize}
        \item $\forall r_1,r_2\in R,\ \overline{r_1}+\overline{r_2}=\overline{r_1+r_2}$. $\overline{0}$ is the identity element with respect to this operation.
        \item $\forall r_1,r_2\in R,\ \overline{r_1}\cdot\overline{r_2}=\overline{r_1\cdot r_2}$. $\overline{1}$ is the identity element with respect to this operation.
    \end{itemize}
    Moreover the projection:
    \begin{align*}
        \pi:R&\longrightarrow\quot{R}{I}\\
        r&\longmapsto\overline{r}
    \end{align*}
    is a surjective ring morphism with $\ker\pi=I$.
\end{prop}
\begin{corollary}
    Let $(R,+,\cdot)$ be a ring and $I$ be an ideal of $(R,+,\cdot)$. Ideals of $\quot{R}{I}$ are of the form $\quot{J}{I}$ where $J$ is an ideal of $(R,+,\cdot)$ containing $I$.
\end{corollary}
\subsubsection{Isomorphism theorems}
\begin{theorem}[First isomorphism theorem]
    Let $(R,+,\cdot)$, $(S,\oplus,\odot)$ be two rings, $\phi:R\rightarrow S$ be a ring morphism and $I$ be an ideal such that $I$ is a subgroup of $(\ker\phi,+)$. Then there exists a unique ring morphism $\psi:R/I\rightarrow S$ such that the diagram of figure \ref{theorem2} is commutative, that is, $\phi=\psi\circ\pi$.
    \illustration{0.35}{Images/theorem2}{}{theorem2}
    In particular, if $I=\ker\phi$, then $\psi$ is injective and therefore there is an isomorphism $\psi:R/\ker\phi\rightarrow\im\phi$.
\end{theorem}
\begin{theorem}[Second isomorphism theorem]
    Let $(R,+,\cdot)$ be a ring and $I$, $J$ be ideals of $(R,+,\cdot)$. Then, $\quot{(I+J)}{I}$ is an ideal of $\quot{R}{I}$ and there is a group isomorphism 
    $$\phi:\quot{(I+J)}{I}\longrightarrow\quot{J}{(I\cap J)},$$ such that $\phi(a\cdot b)=\phi(a)\cdot\phi(b)$ $\forall a,b\in J$.
\end{theorem}
\begin{theorem}[Third isomorphism theorem]
    Let $(R,+,\cdot)$ be a ring and $I$, $J$ be ideals of $(R,+,\cdot)$ such that $I\subseteq J$. Then, there is a ring isomorphism 
    $$\quot{\left(\quot{R}{I}\right)}{\left(\quot{J}{I}\right)}\cong\quot{R}{J}$$
\end{theorem}
\begin{theorem}[Correspondence theorem]
    Let $(R,+,\cdot)$ be ring and $I$ be an ideal of $(R,+,\cdot)$. Then there is a bijection $\phi$ from the set $\mathcal{R}$ of all ideals $J$ of $(R,+,\cdot)$ such that $I\subseteq J$ onto the set $\mathcal{I}$ of all ideals of $\quot{R}{I}$. More precisely, the bijection is:
    \begin{align*}
        \phi:\mathcal{R}&\longrightarrow\mathcal{I}\\
        J&\longmapsto\quot{J}{I}
    \end{align*}
\end{theorem}
\subsubsection{Special rings and ideals}
\begin{definition}
    A ring $R\ne\{0\}$\footnote{From now on, for simplicity, we will denote the ring $(R,+,\cdot)$ as $R$.} is an \textit{integral domain} if the product of any two nonzero elements is nonzero.
\end{definition}
\begin{definition}
    Let $R$ be a ring. We say $r\in R$ is a \textit{zero divisor} if $\exists s\in R\setminus\{0\}$ such that $r\cdot s=0$. We say $r\in R$ is \textit{not} a \textit{zero divisor} if $r\cdot s=0\implies s=0$. 
\end{definition}
\begin{definition}
    Let $R$ be an integral domain. We say $R$ is a \textit{principal integral domain} (\textit{PID}) if every ideal of $R$ is principal.
\end{definition}
\begin{definition}
    Let $R$ be a ring and $P\ne R$ be an ideal of $R$. We say $P$ is \textit{prime} if $\forall a,b\in R$, we have $a\cdot b\in P\iff a\in P\text{ or }b\in P$.
\end{definition}
\begin{definition}
    Let $R$ be a ring and $M\ne R$ be an ideal of $R$. We say $M$ is \textit{maximal} if for any ideal $I$ of $R$ with $M\subseteq I$, either $I=R$ or $I=M$.
\end{definition}
\begin{prop}
    Let $R$ be a ring. Then:
    \begin{enumerate}
        \item An ideal $P$ of $R$ is prime if and only if $\quot{R}{P}$ is an integral domain.
        \item An ideal $M$ of $R$ is maximal if and only if $\quot{R}{M}$ is a field.
    \end{enumerate}
    In particular, all maximal ideals are prime.
\end{prop}
\begin{definition}
    Let $R$ be an integral domain and $a\in R\setminus\{0\}$ be a non-unit element. We say $a$ is \textit{irreducible} if every factorization of $a$ contains at least one unit. 
\end{definition}
\begin{definition}
    Let $R$ be an integral domain and $a\in R\setminus\{0\}$ be a non-unit element. We say $a$ is \textit{prime} if and only if $(a)$ is a prime ideal or, equivalently, if $b,c\in R$ are such that $a\mid b\cdot c$, then $a\mid b$ or $a\mid c$. 
\end{definition}
\begin{prop}
    Let $R$ be an integral domain and $a\in R\setminus\{0\}$ be a non-unit element.
    \begin{enumerate}
        \item If $a$ is prime, then $a$ is irreducible.
        \item If $R$ is a PID, the following statements are equivalent:
        \begin{enumerate}
            \item $a$ is irreducible.
            \item $(a)$ is maximal.
            \item $a$ is prime.
        \end{enumerate}
    \end{enumerate}
\end{prop}
\begin{theorem}
    Let $R$ be a ring. All ideals $I\ne R$ are contained in a maximal ideal.
\end{theorem}
\subsubsection{Polynomial ring}
\begin{definition}
    Let $R$ be a ring and $p(x)\in R[x]$. If $p(x)=a_0+a_1x+\cdots+a_nx^n$ with $a_n\ne 0$, we define the \textit{degree of $p(x)$} as:
    \begin{equation*}
        \deg p(x)=\left\{
        \begin{array}{ccc}
            n & \text{if} & p(x)\ne 0 \\
            -\infty & \text{if} & p(x)= 0
        \end{array}\right.
    \end{equation*}
\end{definition}
\begin{prop}\label{AG-deg}
    Let $R$ be a ring and $p(x),q(x)\in R[x]$ be polynomials with leading coefficients $p_n$ and $q_n$ respectively. Then:
    \begin{enumerate}
        \item $\deg(p(x)+q(x))\leq\max\{\deg p(x),\deg q(x)\}$ and the equality holds when $\deg p(x)\ne\deg q(x)$.
        \item $\deg(p(x)\cdot q(x))\leq\deg p(x)+\deg q(x)$ and the equality holds when either $p_n$ or $q_n$ is not a zero divisor.
    \end{enumerate}
\end{prop}
\begin{prop}
    Let $R$ be a ring and $b(x),a(x)\in R[x]$ such that the leading coefficient of $b(x)$ is a unit. Then, $\exists! q(x),r(x)\in R[x]$ such that $a(x)=b(x)q(x)+r(x)$ with $\deg r(x)<\deg b(x)$.
\end{prop}
\begin{prop}[Universal property of polynomials]
    Let $R$, $S$ be two rings, $\phi:R\rightarrow S$ be a ring morphism and $s\in S$. Then $\exists!\psi:R[x]\rightarrow S$ such that $\psi$ is a ring morphism, $\psi(r)=\phi(r)$ $\forall r\in R$ and $\psi(x)=s$. That is, the diagram of figure \ref{theorem3} is commutative and $\psi(x)=s$.
    \illustration{0.35}{Images/theorem3}{}{theorem3}
\end{prop}
\begin{prop}[Universal property of polynomials in several variables]
    Let $R$, $S$ be two rings, $\phi:R\rightarrow S$ be a ring morphism and $s_1,\ldots,s_n\in S$ be not necessarily distinct elements of $S$. Then $\exists!\psi:R[x_1,\ldots,x_n]\rightarrow S$ such that $\psi$ is a ring morphism, $\psi(r)=\phi(r)$ $\forall r\in R$ and $\psi(x_i)=s_i$ for $i=1,\ldots,n$.
\end{prop}
\begin{corollary}
    Let $R$ be a ring and $r\in R$. Then, the function 
    \begin{align*}
        \phi_r:R[x]&\longrightarrow R\\
        p(x)&\longmapsto p(r)        
    \end{align*}
    is a ring morphism. Moreover $\ker\phi_r=(x-r)\cdot R[x]$ and for all $p(x)\in R[x]$ $\exists q(x)\in R[x]$ such that: $$p(x)=(x-r)\cdot q(x)+p(r)$$
\end{corollary}
\begin{corollary}
    Let $R$ be a ring and $r_1,\ldots,r_n\in R$. Then, the function 
    \begin{align*}
        \phi:R[x_1,\ldots,x_n]&\longrightarrow R\\
        p(x_1,\ldots,x_n)&\longmapsto p(r_1,\ldots,r_n)        
    \end{align*}
    is a ring morphism. Moreover for all $p(x_1,\ldots,x_n)\in R[x_1,\ldots,x_n]$ $\exists q_i(x_1,\ldots,x_n)\in R[x]$ for $i=1,\ldots,n$ such that: $$p(x_1,\ldots,x_n)=p(r_1,\ldots,r_n)+\sum_{i=1}^n(x_i-r_i)\cdot q_i(x_1,\ldots,x_n)$$ Therefore, $\ker\phi=(x_1-r_1,\ldots,x_n-r_n)$ and consequently: $$\quot{R[x_1,\ldots,x_n]}{(x_1-r_1,\ldots,x_n-r_n)}\cong R$$
\end{corollary}
\begin{corollary}
    Let $K$ be a field and $r_1,\ldots,r_n\in K$. Then, the ideal $(x_1-r_1,\ldots,x_n-r_n)$ is maximal in $K[x_1,\ldots,x_n]$ and $$\quot{K[x_1,\ldots,x_n]}{(x_1-r_1,\ldots,x_n-r_n)}\cong K$$
\end{corollary}
\begin{theorem}[Fundamental theorem of algebra]
    Ideals of $\CC[x]$ are of the form $(x-z)$, where $z\in\CC$. That is, irreducible polynomials in $\CC[x]$ have degree 1.
\end{theorem}
\begin{theorem}[Hilbert's Nullstellensatz]
    Maximal ideals of $\CC[x_1,\ldots,x_n]$ are of the form $(x_1-z_1,\ldots,x_n-z_n)$, where $z_1,\ldots,z_n\in\CC$.
\end{theorem}
\begin{theorem}[Eisenstein's criterion]
    Let $a(x)\in\ZZ[x]\setminus\{0\}$ be such that $a(x)=\sum_{i=0}^na_ix^i$ with $\gcd(a_0,\ldots,a_n)=1$. If $\exists p\in\PP$ such that:
    \begin{itemize}
        \item $p\mid a_i$, $i=0,1,\ldots,n-1$,
        \item $p\nmid a_n$,
        \item $p^2\nmid a_0$,
    \end{itemize}
    then $a(x)$ is irreducible in $\ZZ[x]$ and in $\QQ[x]$.
\end{theorem}
\begin{theorem}[General Eisenstein's criterion]
    Let $R$ be an integral domain, $a(x)=\sum_{i=0}^na_ix^i\in R[x]\setminus\{0\}$ and $p$ be a prime element in $R$ such that:
    \begin{itemize}
        \item $p\mid a_i$, $i=0,1,\ldots,n-1$,
        \item $p\nmid a_n$,
        \item $p^2\nmid a_0$.
    \end{itemize}
    Then, if $a(x)=b(x)\cdot c(x)$, either $\deg b(x)=0$ or $\deg c(x)=0$.
\end{theorem}
\subsubsection{Unique factorization domains}
\begin{definition}
    Let $R$ be an integral domain. We say that two elements $a,b\in R\setminus\{0\}$ are \textit{associated} if $\exists u\in R^*$ such that $a=b\cdot u$.
\end{definition}
\begin{definition}
    Let $R$ be an integral domain. We say that $R$ is a \textit{unique factorization domain} or \textit{UFD} if $\forall a\in R\setminus\{0\}$:
    \begin{enumerate}
        \item $$a=up_1^{\alpha_1}\cdots p_r^{\alpha_r},$$ where $u\in R^*$, $p_i$ are irreducible elements of $R$ and $\alpha_i\in\NN$ $\forall i$. 
        \item Such representation is unique in the sense that if $a=vq_1^{\beta_1}\cdots q_s^{\beta_s}$, where $v\in R^*$, $q_i$ are irreducible elements of $R$ and $\beta_i\in\NN$ $\forall i$, then $r=s$ and $\exists\sigma\in S_n$ such that $p_i$ and $q_{\sigma(i)}$ are associated and $\alpha_i=\beta_{\sigma(i)}$ for $i=1,\ldots,r$\footnote{Equivalently, such representation is unique in the sense that if $a=up_1\cdots p_r=vq_1\cdots q_s$, where $u,v\in R^*$ and $p_i,q_i$ are irreducible elements of $R$ $\forall i$, then $r=s$ and $\exists\sigma\in S_n$ such that $p_i$ and $q_{\sigma(i)}$ are associated for $i=1,\ldots,r$.}.
    \end{enumerate}
\end{definition}
\begin{definition}
    Let $R$ be an integral domain and $a,b\in R$ be such that at least one of them is nonzero. A \textit{greatest common divisor of $a$ and $b$} is an element $d\in R$ such that:
    \begin{enumerate}
        \item $d\mid a$ and $d\mid b$.
        \item If $d'$ is a common divisor of $a$ and $b$, then $d'\mid d$.
    \end{enumerate}
\end{definition}
\begin{prop}
    Let $R$ be a UFD. Then, $\forall a,b\in R\setminus\{0\}$ there exists the greatest common divisor of $a$ and $b$.
\end{prop}
\begin{prop}
    Let $R$ be an integral domain. Then:
    \begin{enumerate}
        \item If $R$ is a UFD, all irreducible elements are prime.
        \item If $$up_1\cdots p_r=vq_1\cdots q_s,$$ where $u,v\in R^*$ and both $p_i$ and $q_i$ are prime elements $\forall i$, then $r=s$ and $\exists\sigma\in S_r$ such that $p_i$ is associated with $q_{\sigma(i)}$ for $i=1,\ldots,r$.
    \end{enumerate}
\end{prop}
\begin{prop}
    Let $R$ be an integral domain.
    \begin{enumerate}
        \item If $R$ is a UFD, then $R$ satisfies the ascending chain condition on principal ideals (ACCP):\par If $$a_1\cdot R\subseteq\cdots\subseteq a_n\cdot R $$ is an ascending chain of principal ideals, then $\exists n_0\in\NN$ such that $a_{n_0}\cdot R=a_i\cdot R$ for $i\geq n_0$.
        \item If $R$ satisfies the ACCP, then all elements in $R$ are product of irreducible factors.
    \end{enumerate}
\end{prop}
\begin{theorem}
    Let $R$ be an integral domain. Then, R is UFD if and only if:
    \begin{enumerate}
        \item All irreducible elements in $R$ are prime.
        \item ACCP is satisfied.
    \end{enumerate}
\end{theorem}
\begin{lemma}
    Let $R$ be an integral domain. Let $$I_1\subseteq I_2\subseteq\cdots\subseteq I_n\subseteq\cdots$$ be a chain of ideals of $R$. Then, $$\bigcup_{n\in\NN}I_n$$ is an ideal of $R$.
\end{lemma}
\begin{theorem}
    Let $R$ be a PID. Then, $R$ is a UFD.
\end{theorem}
\begin{corollary}
    Let $d\in\ZZ\setminus\{0\}$ such that $d$ is square-free. Then, $\ZZ[\sqrt{d}]$ satisfies the ACCP.
\end{corollary}
\begin{prop}
    Let $R$ be a integral domain. If $R$ satisfies the ACCP, then $R[x]$ also satisfies the ACCP.
\end{prop}
\begin{corollary}
    Let $R$ be a UFD. Then, $\forall n\geq 0$, all nonzero elements of $R[x_1,\ldots,x_n]$ are product of irreducible elements.
\end{corollary}
\subsubsection{Field of fractions}
\begin{definition}
    Let $R$ be an integral domain. Consider the set: $$R\times(R\setminus\{0\})=\{(a,b):a,b\in R,b\ne 0\}.$$ We define the relation $\sim$ in the following way: $$(a_1,b_1)\sim(a_2,b_2)\iff a_1b_2=a_2b_1,$$ for all $(a_1,b_1),(a_2,b_2)\in R\times(R\setminus\{0\})$.
\end{definition}
\begin{lemma}
    The relation $\sim$ is an equivalence relation. We denote by $\mathrm{Frac}(R)$ the set of equivalence classes $\quot{R\times(R\setminus\{0\})}{\sim}$ and by $\frac{a}{b}$ the equivalence class $\overline{(a,b)}\in\mathrm{Frac}(R)$.
\end{lemma}
\begin{definition}
    Let $R$ be an integral domain. We define the sum and multiplication in $\mathrm{Frac}(R)$ as follows:
    \begin{enumerate}
        \item $\frac{a_1}{b_1}+\frac{a_2}{b_2}=\frac{a_1b_2+a_2b_1}{b_1b_2}$, $\forall\frac{a_1}{b_1},\frac{a_2}{b_2}\in\mathrm{Frac}(R)$
        \item $\frac{a_1}{b_1}\cdot\frac{a_2}{b_2}=\frac{a_1a_2}{b_1b_2}$, $\forall\frac{a_1}{b_1},\frac{a_2}{b_2}\in\mathrm{Frac}(R)$
    \end{enumerate}
\end{definition}
\begin{theorem}
    Let $R$ be an integral domain and consider $(\mathrm{Frac}(R),+,\cdot)$ with the operations $+$ and $\cdot$ defined above. Then:
    \begin{enumerate}
        \item $(\mathrm{Frac}(R),+,\cdot)$ is a field.
        \item The function
        \begin{align*}
            i:R&\longrightarrow \mathrm{Frac}(R)\\
            r&\longmapsto \frac{r}{1}
        \end{align*}
        is an injective ring morphism and satisfies the following property: If $K$ is a field and $\phi:R\rightarrow K$ is an injective ring morphism, then $\exists!\psi:\mathrm{Frac}(R)\rightarrow K$ such that $\psi$ is a ring morphism, and the diagram of figure \ref{theorem4} is commutative.
    \end{enumerate}
    \illustration{0.35}{Images/theorem4}{}{theorem4}
\end{theorem}
\subsubsection{Irreducible and prime elements in $R[X]$}
\begin{prop}
    Let $R$ be a UFD and $p\in R$. The following statements are equivalent:
    \begin{enumerate}
        \item $p$ is irreducible in $R$.
        \item $p$ is irreducible in $R[x]$.
        \item $p$ is prime in $R$.
        \item $p$ is prime in $R[x]$.
    \end{enumerate}
\end{prop}
\begin{definition}
    Let $R$ be a UFD and $a(x)=\sum_{i=0}^na_ix^i\in R[x]\setminus\{0\}$. We say $p(x)$ is a \textit{primitive polynomial} if 1 is a greatest common divisor of $a_0,\ldots,a_n)$.
\end{definition}
\begin{lemma}[Gau\ss' lemma]
    Let $R$ be a UFD and $a(x),b(x)\in R[x]\setminus\{0\}$ be primitive polynomials. Then, $a(x)\cdot b(x)$ is primitive.
\end{lemma}
\begin{lemma}
    Let $R$ be a UFD. Then:
    \begin{enumerate}
        \item If $c_1\cdot a(x)=c_2\cdot b(x)$, where $c_1,c_2\in R$, $a(x),b(x)\in R[x]$ and $b(x)$ is primitive, then $c_1\mid c_2$.
        \item If moreover $a(x)$ is also primitive, then $\exists u\in R^*$ such that $c_1=u\cdot c_2$.
    \end{enumerate}
\end{lemma}
\begin{prop}
    Let $R$ be a UFD and $p(x)\in R[x]$. The following statements are equivalent:
    \begin{enumerate}
        \item $p(x)$ is irreducible in $R[x]$.
        \item $p(x)$ is irreducible in $\mathrm{Frac}(R[x])$.
        \item $p(x)$ is prime in $R[x]$.
        \item $p(x)$ is prime in $\mathrm{Frac}(R[x])$.
    \end{enumerate}
\end{prop}
\begin{theorem}
    Let $R$ be a UFD. Then, $R[x]$ is a UFD.
\end{theorem}
\begin{corollary}
    $\ZZ[x_1,\ldots,x_n]$ and $K[x_1,\ldots,x_n]$, where $K$ is a field, are both UFD.
\end{corollary}
%\subsection{Finite fields}
\subsubsection{Examples of rings}\label{AS-examples2}
Let $n\in\NN$ and $d\in\ZZ$ such that $d$ is square-free.
\begin{itemize}
    \item $\ZZ$, $\ZZ/n\ZZ$, $\QQ$, $\RR$, $\CC$
    \item $R[x]$, where $R$ is a ring\footnote{Note that if $R=R[y]$, then $R[x]=(R[y])[x]=R[x,y]$. So the set of polynomials with several variables over a ring $R$ is also a ring with the same operations as $R$.}.
    \item $\mathcal{M}_n(K)$, where $K$ is a field. Note that this is a noncommutative ring.
    \item $\ZZ[\sqrt{d}]$, where $Z[\sqrt{d}]=\{a+b\sqrt{d}:a,b\in\ZZ\}$. In particular, the set $\ZZ[\sqrt{-1}]=\ZZ[\ii]$ is called the set of \textit{Gau\ss ian integers}.
    \item $\QQ(\sqrt{d})$, where $\QQ(\sqrt{d})=\{a+b\sqrt{d}:a,b\in\QQ\}$.
\end{itemize}
\illustration{1}{Images/circles}{Inclusions of algebraic structures. Here $R$ and $S$ are nonzero rings, $T$ is a UFD, $K$ is a field, $d\in\ZZ$ such that $d$ is square-free, $n\in\NN$ and $p\in\PP$.}{}
\end{multicols}
\end{document}