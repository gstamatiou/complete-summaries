\documentclass[../../../main.tex]{subfiles}
%%% Note there is a page-break between the last theorem and the last corollary.

\begin{document}
\begin{multicols}{2}[\section{Functions of several variables}]
  \subsection{Topology of \texorpdfstring{$\RR^n$}{Rn}}
  \begin{definition}
    Let $M$ be a set. A \textit{distance in $M$} is a function $d:M\times M\rightarrow\RR $ such that $\forall x,y,z\in M$ the following properties are satisfied:
    \begin{enumerate}
      \item $d(x,y)\geq 0$.
      \item $d(x,y)=0\iff x=y$.
      \item $d(x,y)=d(y,x)$.
      \item $d(x,y)\leq d(x,z)+d(z,y)\quad$\textit{(triangular inequality)}.
    \end{enumerate}
    We define a \textit{metric space} as a pair $(M,d)$ that satisfy the previous properties.
    \label{FOSV_metric}
  \end{definition}
  \begin{definition}\label{FOSV_norm}
    Let $V$ be a real vector space. A \textit{norm} on $V$ is a function $\|\cdot\|:V\rightarrow\RR $ such that $\forall \vf{u},\vf{v}\in V$ and $\forall\lambda\in\RR $ the following properties are satisfied:
    \begin{enumerate}
      \item $\|\vf{u}\|\geq 0$.
      \item $\|\vf{u}\|=0\iff \vf{u}=0$.
      \item $\|\lambda \vf{u}\|=|\lambda|\|\vf{u}\|$.
      \item $\|\vf{u}+\vf{v}\|\leq \|\vf{u}\|+\|\vf{v}\|\quad$\textit{(triangular inequality)}.
    \end{enumerate}
    We define a \textit{normed vector space} as a pair $(V,\|\cdot\|)$ that satisfy the previous properties.
  \end{definition}
  \begin{prop}
    Let $(V,\|\cdot\|)$ be a normed vector space. Then $(V,d)$ is a metric space with associated distance $d(\vf{u},\vf{v}):=\|\vf{u}-\vf{v}\|$, $\forall \vf{u},\vf{v}\in V$.
  \end{prop}
  \begin{definition}
    Let $V$ be a real vector space. A \textit{dot product} on $V$ is a function $\langle\cdot,\cdot\rangle:V\times V\rightarrow\RR $ such that $\forall \vf{u},\vf{v},\vf{w}\in V$ and $\forall\alpha,\beta\in\RR $ the following properties are satisfied:
    \begin{enumerate}
      \item $\langle \alpha \vf{u}+\beta \vf{w},\vf{v}\rangle=\alpha\langle \vf{u},\vf{v}\rangle+\beta\langle \vf{u},\vf{v}\rangle,\\\langle \vf{u},\alpha \vf{v}+\beta \vf{w}\rangle=\alpha\langle \vf{u},\vf{v}\rangle+\beta\langle \vf{u},\vf{w}\rangle$.
      \item $\langle \vf{u},\vf{v}\rangle=\langle \vf{v},\vf{u}\rangle$.
      \item $\langle \vf{u},\vf{u}\rangle\geq 0$.
      \item $\langle \vf{u},\vf{u}\rangle=0\iff \vf{u}=0$.
    \end{enumerate}
    We define an \textit{Euclidean space} as a pair $(V,\langle\cdot,\cdot\rangle)$ that satisfy the previous properties\footnote{Sometimes the notation $\vf{u}\cdot \vf{v}$ is used, instead of $\langle \vf{u},\vf{v}\rangle$, to denote the dot product between $\vf{u}$ and $\vf{v}$.}.
  \end{definition}
  \begin{prop}
    Let $(V,\langle\cdot,\cdot\rangle)$ be an Euclidean space. Then $(V,\|\cdot\|)$ is a normed space with associated norm $\|\vf{u}\|:=\sqrt{\langle \vf{u},\vf{u}\rangle}$, $\forall \vf{u}\in V$. .
  \end{prop}
  \begin{prop}
    Let $\langle\cdot,\cdot\rangle_2:\RR^n\times\RR^n\rightarrow\RR $ be a map defined by $$\langle \vf{u},\vf{v}\rangle_2=\sum_{i=1}^nu_iv_i$$ $\forall \vf{u},\vf{v}\in \RR^n$, being $\vf{u}=(u_1,\ldots,u_n)$ and $\vf{v}=(v_1,\ldots,v_n)$. Then, the pair $(\RR^n,\langle\cdot,\cdot\rangle_2)$ is an Euclidean space.
  \end{prop}
  \begin{corollary}
    Consider the norm $\|\cdot\|_2$ and distance $d_2$ in $\RR^n$ defined as follows:
    \begin{gather*}
      \|\vf{u}\|_2=\sqrt{\langle \vf{u},\vf{u}\rangle_2}=\sqrt{\sum_{i=1}^nu_i^2},\\
      d_2(\vf{u},\vf{v})=\|\vf{u}-\vf{v}\|=\sqrt{\sum_{i=1}^n(u_i-v_i)^2}.
    \end{gather*}
    Then, $(\RR^n,\|\cdot\|_2)$ is a normed space and $(\RR^n,d_2)$ is a metric space.
  \end{corollary}
  \begin{prop}
    Let $(V,\langle\cdot,\cdot\rangle)$ be an Euclidean space with the norm defined as $\|\vf{u}\|:=\sqrt{\langle \vf{u},\vf{u}\rangle}$. Then for all $\vf{u},\vf{v}\in V$ the following properties are satisfied:
    \begin{enumerate}
      \item $\langle \vf{u},\vf{v}\rangle\leq\|\vf{u}\|\|\vf{v}\|\quad$\textit{(Cauchy-Schwarz inequality)}.
      \item $\|\vf{u}-\vf{v}\|\geq\|\vf{u}\|-\|\vf{v}\|$.
      \item $\|\vf{u}+\vf{v}\|^2+\|\vf{u}-\vf{v}\|^2=2(\|\vf{u}\|^2+\|\vf{v}\|^2)\quad$ \textit{(Parallelogram law)}.
      \item $\|\vf{u}+\vf{v}\|^2-\|\vf{u}-\vf{v}\|^2=4\langle \vf{u},\vf{v}\rangle$.
      \item On $(\RR^n,\langle\cdot,\cdot\rangle_2)$, if $\vf{u}=(u_1,\ldots,u_n)$, then: $$|u_i|\leq\|\vf{u}\|\leq\sum_{i=1}^n|u_i|$$
    \end{enumerate}
  \end{prop}
  \begin{definition}
    Let $\vf{L}:\RR^n\rightarrow\RR^m$ be a linear map. We define the \textit{norm of $\vf{L}$} as $$\|\vf{L}\|=\sup\{\|\vf{L}(x)\|:\|x\|=1\}$$
  \end{definition}
  \begin{lemma}
    Let $\Phi:\mathcal{L}(\RR^n,\RR^m)\rightarrow\RR $ be a map defined as $\Phi(\vf{L})=\|\vf{L}\|$. Then, $\Phi$ is a norm on the vector space $\mathcal{L}(\RR^n,\RR^m)$.
  \end{lemma}
  \begin{prop}
    Let $\vf{L}\in\mathcal{L}(\RR^n,\RR^m)$. Then: $$\|\vf{L}\|=\inf\{C:\|\vf{L}(x)\|\leq C\|x\|\}$$
  \end{prop}
  \begin{corollary}
    Let $\vf{L},\vf{M}\in\mathcal{L}(\RR^n,\RR^m)$ be linear maps with associated matrices $\vf{L}=(a_{ij})$, $\vf{M}=(b_{ij})$ respectively. The following properties are satisfied:
    \begin{enumerate}
      \item $\|\vf{L}(x)\|\leq\|\vf{L}\|\|x\|$.
      \item $\displaystyle\|\vf{L}\|\leq\left(\sum_{i=1}^m\sum_{j=1}^na_{ij}^2\right)^{1/2}$.
      \item $|a_{ij}-b_{ij}|<\varepsilon,\forall i,j\iff\|\vf{L}-\vf{M}\|<\varepsilon'$.
    \end{enumerate}
  \end{corollary}
  \begin{definition}
    Let $(M,d)$ be a metric space. The \textit{sphere with center $p$ and radius $r\in\RR_{\geq 0}$} is the set $S(p,r)=\{x\in M:d(x,p)=r\}$.
  \end{definition}
  \begin{definition}
    Let $(M,d)$ be a metric space. The \textit{open ball with center $p$ and radius $r\in\RR_{>0}$} is the set $B(p,r)=\{x\in M:d(x,p)<r\}$.
  \end{definition}
  \begin{definition}
    Let $(M,d)$ be a metric space. The \textit{closed ball with center $p$ and radius $r\in\RR_{\geq 0}$} is the set $\overline{B}(p,r)=\{x\in M:d(x,p)\leq r\}$.
  \end{definition}
  \begin{definition}
    Let $(M,d)$ be a metric space and $A\subseteq M$ be a subset of $M$. $A$ is a \textit{bounded set} if exists a ball containing it.
  \end{definition}
  \begin{definition}
    Let $(M,d)$ be a metric space. A \textit{neighborhood of $p$} is a bounded set $E(p)\subset M$ such that $\exists r\in\RR_{>0}$ satisfying $B(p,r)\subset E(p)$.
  \end{definition}
  \begin{definition}
    Let $(M,d)$ be a metric space and $A\subseteq M$ be a subset of $M$. $p$ is an \textit{interior point of $A$} if $\exists r\in\RR_{>0}$ such that $B(p,r)\subset A$. The \textit{interior of $A$} is the set $\mathring A$\footnote{Sometimes the set $\mathring A$ is denoted as $\text{Int }A$.} containing all interior points of $A$.
  \end{definition}
  \begin{definition}
    Let $(M,d)$ be a metric space and $A\subseteq M$ be a subset of $M$. $p$ is an \textit{exterior point of $A$} if $\exists r\in\RR_{>0}$ such that $B(p,r)\cap A=\varnothing$. The \textit{exterior of $A$} is the set $\text{Ext } A$ containing all exterior points of $A$.
  \end{definition}
  \begin{definition}
    Let $(M,d)$ be a metric space and $A\subseteq M$ be a subset of $M$. $p$ is an \textit{adherent point} of $A$ if $\forall r\in\RR_{>0}$, $B(p,r)\cap A\ne\varnothing$. The \textit{adherence} of $A$ is the set $\overline{A}$ containing all adherent points of $A$.
  \end{definition}
  \begin{definition}
    Let $(M,d)$ be a metric space and $A\subseteq M$ be a subset of $M$. $p$ is a \textit{accumulation point} of $A$ if $\forall r\in\RR_{>0}$, $B(p,r)\setminus\{p\}\cap A\ne\varnothing$. The \textit{limit set} of $A$ is the set $A'$ containing all accumulation points of $A$.
  \end{definition}
  \begin{definition}
    Let $(M,d)$ be a metric space and $A\subseteq M$ be a subset of $M$. $p$ is an \textit{isolated point} of $A$ if it is an adherent but not accumulation point, that is, if $p\in A$ and $\exists r\in\RR_{>0}$ such that $B(p,r)\setminus\{p\}\cap A=\varnothing$.
  \end{definition}
  \begin{definition}
    Let $(M,d)$ be a metric space and $A\subseteq M$ be a subset of $M$. $p$ is a \textit{boundary point} of $A$ if $\forall r\in\RR_{>0}$, $B(p,r)\cap A\ne\varnothing$ and $B(p,r)\cap A^c\ne\varnothing$. The \textit{boundary} of $A$ is the set $\partial A$ containing all boundary points of $A$.
  \end{definition}
  \begin{prop}
    Let $(M,d)$ be a metric space and $A\subseteq M$ be a subset of $M$. If $p$ is a accumulation point of $A$, then $\forall r\in\RR_{>0}$, $B(p,r)$ has infinity many points of $A$.
  \end{prop}
  \begin{theorem}[Bolzano-Weierstra\ss\space theorem]
    Let $B\subset\RR^n$ be a set. If $B$ has infinity many points and it is bounded, then it has at least a accumulation point.
  \end{theorem}
  \begin{definition}
    Let $(M,d)$ be a metric space and $A\subseteq M$ be a subset of $M$. $A$ is \textit{open} if $\forall p\in A$, $\exists r\in\RR_{>0}$ such that $B(p,r)\subset A$.
  \end{definition}
  \begin{definition}
    Let $(M,d)$ be a metric space and $A\subseteq M$ be a subset of $M$. $A$ is \textit{closed} if its complementary $A^c$ is open.
  \end{definition}
  \begin{prop}
    Let $(M,d)$ be a metric space and $A\subseteq M$ be a subset of $M$. $A$ is closed $\iff A=\overline{A}\iff\partial A\subset A\iff A'\subset A$.
  \end{prop}
  \begin{prop}
    Let $(M,d)$ be a metric space and $A\subseteq M$ be a subset of $M$. $A$ is open $\iff A=\mathring A$.
  \end{prop}
  \begin{prop}
    Let $(M,d)$ be a metric space and $A\subseteq M$ be a subset of $M$.
    \begin{itemize}
      \item $\mathring A$ is the biggest open set contained in $A$. That is, if $B\subset A$ is open, $B\subset\mathring A$.
      \item $\overline{A}$ is the smallest set containing $A$. That is, if $B\supset A$ is closed, $B\supset\overline{A}$.
    \end{itemize}
  \end{prop}
  \begin{prop}
    \hfill
    \begin{itemize}
      \item The union of open sets is open.
      \item The intersection of a finite number of open sets is open.
      \item The union of a finite number of closed sets is closed.
      \item The intersection of closed sets is closed.
    \end{itemize}
  \end{prop}
  \begin{definition}
    We say that a set $A$ is \textit{connected} if there are no open sets $U,V\ne\varnothing$ such that:
    \begin{gather*}
      A\subseteq U\cup V\quad A\cap U\cap V=\varnothing\quad A\cap U\ne\varnothing\quad A\cap V\ne\varnothing
    \end{gather*}
  \end{definition}
  \subsubsection{Sequences}
  \begin{definition}
    Let $(M,d)$ be a metric space. A \textit{sequence} in $M$ is a map
    \begin{align*}
      x:\NN & \longrightarrow M \\
      n     & \longmapsto x(n)
    \end{align*}
    The sequence $x(n)$ is usually represented as $(x_n)$.
  \end{definition}
  \begin{definition}
    Let $(M,d)$ be a metric space. We say a sequence $(x_n)\subset M$ is \textit{convergent} to $p\in M$ if $$\forall\varepsilon\in\RR_{>0},\;\exists n_0\in\NN :d(x_n,p)<\varepsilon\text{ if }n>n_0$$
  \end{definition}
  \begin{definition}
    Let $(M,d)$ be a metric space. We say a sequence $(x_n)$ is a \textit{Cauchy sequence} if $\forall\varepsilon>0$ $\exists n_0$ such that $d(x_n,x_m)<\varepsilon$, for all $m,n\geq n_0$.
  \end{definition}
  \begin{definition}
    A metric space $(M,d)$ is \textit{complete} if every Cauchy sequence in $M$ converges in $M$.
    \label{FOSV_complete}
  \end{definition}
  \begin{definition}
    A subset $K\subset\RR^n$ is \textit{compact} if it is closed and bounded.
  \end{definition}
  \begin{theorem}
    Let $K\subset\RR^n$ be an arbitrary set and $(x_m)\in K$ be a sequence. Then $K$ is compact if and only if there exists a partial sequence $(x_{m_k})$ and $x\in K$ such that $\displaystyle\lim_{k\to\infty}x_{m_k}=x$.
  \end{theorem}
  \subsection{Continuity}
  \begin{definition}[Graph of a function]
    Let $f:U\subseteq\RR^n\rightarrow\RR $. We define the \textit{graph of $f$} as the following subset of $\RR^{n+1}$: $$\text{graph}(f)=\{(x,f(x))\in\RR^{n+1}:x\in U\}$$
  \end{definition}
  \begin{definition}
    Given a function $f:U\subseteq\RR^n\rightarrow\RR $, we define the \textit{level set $C_k(f)$} as $C_k(f)=\{x\in\RR^n:f(x)=k\}$.
  \end{definition}
  \begin{definition}
    Let $\vf{f}:U\subseteq\RR^n\rightarrow\RR^m$ and $p\in U'$. We say $\displaystyle\lim_{x\to p}\vf{f}(x)=L$ if $\forall\varepsilon>0$, $\exists\delta>0$ such that $\|\vf{f}(x)-L\|<\varepsilon$ if $\|x-p\|<\delta$.
  \end{definition}
  \begin{prop}
    Let $\vf{f}:U\subseteq\RR^n\rightarrow\RR^m$, $\vf{f}=(f_1,\ldots,f_m)$, and $p\in U'$.
    \begin{enumerate}
      \item The limit of $\vf{f}$ at point $p$, if exists, is unique.
      \item Suppose $L=(L_1,\ldots,L_m)$. Then, $\displaystyle\lim_{x\to p}\vf{f}(x)=L\iff\lim_{x\to p}f_j(x)=L_j\quad\forall j=1,\ldots,m$.
    \end{enumerate}
  \end{prop}
  \begin{lemma}
    Let $\vf{f}:U\subseteq\RR^n\rightarrow\RR^m$ and $p\in U'$. $\displaystyle\exists\lim_{x\to p}\vf{f}(x)=L\iff\forall(x_n)\in\RR^n:\lim_{n\to\infty}x_n=p$ and $x_n\ne p$ for all $n$ we have $\displaystyle\lim_{n\to \infty}\vf{f}(x_n)=L$.
  \end{lemma}
  \begin{definition}
    We say that $\vf{f}:U\subseteq\RR^n\rightarrow\RR^m$ is \textit{continuous} at $p\in U'$ if $\displaystyle\lim_{x\to p}\vf{f}(x)=\vf{f}(p)$. We say that $\vf{f}$ is continuous on $U$, if so it is at each point $p\in U$.
  \end{definition}
  \begin{definition}
    We say that $\vf{f}:U\subseteq\RR^n\rightarrow\RR^m$ is \textit{uniformly continuous} on $U$ if $\forall\varepsilon>0$, $\exists\delta>0:\|\vf{f}(x)-\vf{f}(y)\|<\varepsilon$, $\forall x,y\in U:\|x-y\|<\delta$.
  \end{definition}
  \begin{corollary}
    A uniformly continuous function is continuous.
  \end{corollary}
  \begin{theorem}[Heine's theorem]
    Let $\vf{f}:K\subset\RR^n\rightarrow\RR^m$ be continuous function and $K$ be a compact set. Then, $\vf{f}$ is uniformly continuous on $K$.
  \end{theorem}
  \begin{theorem}
    Let $\vf{f}:U\subseteq\RR^n\rightarrow\RR^m$ be an uniformly continuous function and $(x_n)\in U$ be a Cauchy sequence. Then $(\vf{f}(x_n))\in\RR^m$ is a Cauchy sequence.
  \end{theorem}
  \begin{theorem}
    Let $\vf{f}:K\subset\RR^n\rightarrow\RR^m$ be a continuous function and $K$ be a compact set. Then $\vf{f}(K)$ is a compact set.
  \end{theorem}
  \begin{theorem}[Weierstra\ss' theorem]
    Let $f:K\subset\RR^n\rightarrow\RR $ be a continuous function and $K$ a compact set. Then $f$ attains a maximum and a minimum on $K$.
  \end{theorem}
  \begin{theorem}[Intermediate value theorem]
    Let $f:U\subseteq\RR^n\rightarrow\RR $ be a continuous function and $U$ be a connected set. Then $\forall x,y\in U$ and $\forall c\in[f(x),f(y)]$, $\exists z\in U$ such that $f(z)=c$.
  \end{theorem}
  \begin{definition}
    A function $\vf{f}:U\subseteq\RR^n\rightarrow\RR^m$ is called \textit{Lipschitz continuous} if $\exists k>0$ such that $$\|\vf{f}(x)-\vf{f}(y)\|\leq k\|x-y\|$$ $\forall x,y\in U$. If $0\leq k<1$, we say that $\vf{f}$ is a \textit{contraction}.
    \label{FOSV_contr}
  \end{definition}
  \begin{prop}
    Let $\vf{f}:U\subseteq\RR^n\rightarrow\RR^m$ be a locally Lipschitz continuous function at $p\in U$. Then $\vf{f}$ is continuous at $p$.
  \end{prop}
  \begin{definition}
    Let $(M,d)$ be a metric space and $f:M\rightarrow\RR $ a function. We define the \textit{modulus of continuity of $f$} as the function $\omega_f:(0,\infty)\rightarrow[0,\infty]$ defined as: $$\omega_f(\delta):=\sup\{|f(x)-f(y)|:d(x,y)<\delta, x,y\in M\}$$
  \end{definition}
  \subsection{Differential calculus}
  \subsubsection{Differential of a function}
  \begin{definition}
    Let $\vf{f}:U\subseteq\RR^n\rightarrow\RR^m$ and $a\in U$. The function $\vf{f}$ is \textit{differentiable at $a$} if there exists a linear map $\vf{Df}(a):\RR^n\rightarrow\RR^m$ such that:
    \begin{multline*}
      \lim_{x\to a}\frac{\|\vf{f}(x)-\vf{f}(a)-\vf{Df}(a)(x-a)\|}{\|x-a\|}=\\=\lim_{h\to 0}\frac{\|\vf{f}(a+h)-\vf{f}(a)-\vf{Df}(a)(h)\|}{\|h\|}=0
    \end{multline*}
    $\vf{Df}(a)$ is called the \textit{differential of $\vf{f}$ at point $a$}. Furthermore, we say $\vf{f}$ is differentiable on $B\subseteq U$ if it is differentiable at each point of $B$.
  \end{definition}
  \begin{prop}
    Let $\vf{f}:U\subseteq\RR^n\rightarrow\RR^m$ and $a\in U$. $\vf{f}=(f_1,\ldots,f_m)$ is differentiable at $a$ if and only if every component function $f_j:U\subseteq\RR^n\rightarrow\RR $ is differentiable at $a$.
  \end{prop}
  \begin{definition}
    Let $f:U\subseteq\RR^n\rightarrow\RR $, $a\in U$ and $\vf{v}\in\RR^n$ such that $\|\vf{v}\|=1$. The \textit{directional derivative of $f$ at $a$} in the direction of $\vf{v}$ is: $$D_{\vf{v}}f(a)=\lim_{t\to 0}\frac{f(a+t\vf{v})-f(a)}{t}$$
  \end{definition}
  \begin{definition}
    Let $U\subseteq\RR^n$ be an open set, $f:U\rightarrow\RR $ and $a\in U$. If the following limit exists, we define the \textit{partial derivative with respect to $x_j$ of $f$ at $a$} as: $$\frac{\partial f}{\partial x_j}(a)=\lim_{h\to 0}\frac{f(a+h\vf{e}_j)-f(a)}{h}\footnote{Here $\vf{e}_j$ is the $j$-th vector of the canonical basis of $\RR^n$, that is, $\vf{e}_j=(0,\ldots,0,\overset{(j)}{1},0,\ldots,0)$.}$$
  \end{definition}
  \begin{definition}
    Let $\vf{f}:U\subseteq\RR^n\rightarrow\RR^m$ and $a\in U$. If all partial derivatives of $\vf{f}$ at $a$ exist, we call \textit{Jacobian matrix of $\vf{f}$ at $a$} the matrix associated with $\vf{Df}(a)$ (with respect to the canonical basis of $\RR^n$ and $\RR^m$):
    $$\vf{D}\vf{f}(a)=
      \begin{pmatrix}
        \displaystyle \frac{\partial f_1}{\partial x_1}(a) & \cdots & \displaystyle \frac{\partial f_1}{\partial x_n}(a) \\
        \vdots                                             & \ddots & \vdots                                             \\
        \displaystyle \frac{\partial f_m}{\partial x_1}(a) & \cdots & \displaystyle \frac{\partial f_m}{\partial x_n}(a)
      \end{pmatrix}$$ If $n=m$, we define the \textit{Jacobian determinant} as $J\vf{f}(a)=\det\vf{D}\vf{f}(a)$.
  \end{definition}
  \begin{definition}
    Let $U\subseteq\RR^n$ be an open set, $f:U\rightarrow\RR $ and $a\in U$ such that $f$ is differentiable at $a\in U$. The \textit{gradient of $f$ at $a$} is: $$\grad f(a):=\vf{D}f(a)=\left(\frac{\partial f}{\partial x_1}(a),\ldots,\frac{\partial f}{\partial x_n}(a)\right)$$
  \end{definition}
  \begin{prop}
    Let $U\subseteq\RR^n$ be an open set and $f:U\rightarrow\RR $ be a differentiable function at $a\in U$. Then there exists the tangent hyperplane to the graph of $f$ at $a$ and has the equation: $$x_{n+1}=f(a)+\grad f(a)\cdot (x-a)\footnote{In general (not only the case of the graph of a function) the tangent hyperplane to function $f$ at a point $a$ is given by the equation $$\grad f(a)\cdot (x-a)=0$$}$$
  \end{prop}
  \begin{theorem}
    Let $U\subseteq\RR^n$ be an open set, $f:U\rightarrow\RR $, $a\in U$ and $\vf{v}\in\RR^n$ such that $\|\vf{v}\|=1$. If $f$ is differentiable at $a$, the $D_{\vf{v}}f(a)$ exists and: $$D_{\vf{v}}f(a)=\grad f(a)\cdot \vf{v}$$
  \end{theorem}
  \begin{prop}
    Let $U\subseteq\RR^n$ be an open set, $f:U\rightarrow\RR $ be a differentiable function on $U$ and $C_k$ be the level set of value $k\in\RR $. Then $\grad f(a)\perp C_k$ at $a\in C_k$.
  \end{prop}
  \begin{prop}
    Let $U\subseteq\RR^n$ be an open set and $f:U\rightarrow\RR $ a differentiable function at $a\in U$ and $\vf{v}\in\RR^n$. Then:
    \begin{itemize}
      \item $\displaystyle\max\{D_{\vf{v}}f(a):\|\vf{v}\|=1\}=\|\grad f(a)\|$ and it is attained when $\displaystyle \vf{v}=\frac{\grad f(a)}{\|\grad f(a)\|}$.
      \item $\displaystyle\min\{D_{\vf{v}}f(a):\|\vf{v}\|=1\}=-\|\grad f(a)\|$ and it is attained when $\displaystyle \vf{v}=-\frac{\grad f(a)}{\|\grad f(a)\|}$.
    \end{itemize}
  \end{prop}
  \begin{theorem}
    Let $\vf{f}:U\subseteq\RR^n\rightarrow\RR^m$ be a differentiable function at $a\in U$. Then $\vf{f}$ is locally Lipschitz continuous at $a$.
  \end{theorem}
  \begin{theorem}
    Let $\vf{f},\vf{g}:U\subseteq\RR^n\to\RR^m$ be two differentiable functions at a point $a\in U$ and let $c\in\RR $. Then:
    \begin{enumerate}
      \item $\vf{f}+\vf{g}$ is differentiable at $a$ and: $$\vf{D}(\vf{f}+\vf{g})(a)=\vf{Df}(a)+\vf{Dg}(a)$$
      \item $c\vf{f}$ is differentiable at $a$ and:
            $$\vf{D}(c\vf{f})(a)=c\vf{Df}(a)$$
      \item If $m=1$, then $(fg)(x)=f(x)g(x)$ is differentiable at $a$ and: $$D(fg)(a)=g(a)Df(a)+f(a)Dg(a)$$
      \item If $m=1$ and $g(a)\ne0$, then $\displaystyle\left(\frac{f}{g}\right)(x)=\frac{f(x)}{g(x)}$ is differentiable at $a$ and: $$D\left(\frac{f}{g}\right)(a)=\frac{g(a)Df(a)-f(a)Dg(a)}{{g(a)}^2}$$
    \end{enumerate}
  \end{theorem}
  \begin{theorem}[Chain rule]
    Let $U\subseteq\RR^n$ and $V\subseteq\RR^m$ be open sets. Let $\vf{f}:U\rightarrow\RR^m$ and $\vf{g}:V\rightarrow\RR^p$. Suppose that $\vf{f}(U)\subset V$, $\vf{f}$ is differentiable at $a\in U$ and $\vf{g}$ is differentiable at $\vf{f}(a)$. Then $\vf{g}\circ \vf{f}$ is differentiable at $a$ and: $$\vf{D}(\vf{g}\circ \vf{f})(a)=\vf{Dg}(\vf{f}(a))\circ \vf{Df}(a)$$
  \end{theorem}
  \begin{definition}
    Let $U\subseteq\RR^n$ be an open set and $\vf{f}:U\rightarrow\RR^m$. We say that $\vf{f}$ is a \textit{function of class $\mathcal{C}^k(U)$}, $k\in\NN $, if all partial derivatives of order $k$ exists and are continuous on $U$. We say that $\vf{f}$ is \textit{function of class $\mathcal{C}^\infty(U)$} if it is of class $\mathcal{C}^k(U)$, $\forall k\in\NN $.
  \end{definition}
  \begin{theorem}[Differentiability criterion]
    Let $\vf{f}:U\subseteq\RR^n\rightarrow\RR^m$, $\vf{f}(x)=(f_1(x),\ldots,f_m(x))$. If all partial derivatives $\displaystyle \frac{\partial f_i(x)}{\partial x_j}$ exists in a neighborhood of $a\in U$ and are continuous at $a$, then $\vf{f}$ is differentiable at $a\in U$.
  \end{theorem}
  \begin{prop}
    Let $\vf{f}:U\subseteq\RR^n\rightarrow\RR^m$ and $A\subseteq U$. If all partial derivatives of $\vf{f}$ exist on $A$ and are bounded functions on $A$, then $\vf{f}$ is uniformly continuous on $A$.
  \end{prop}
  \begin{theorem}[Mean value theorem]
    Let $f:B\rightarrow\RR $ be a function of class $\mathcal{C}^1$ in an open connected set $B\subseteq\RR^n$ and $x,y\in B$. Then: $$f(x)-f(y)=\grad f(z)\cdot(x-y)$$ for some $z\in[x,y]$.
  \end{theorem}
  \begin{theorem}[Mean value theorem for vector-valued functions]
    Let $\vf{f}:B\rightarrow\RR^m$ be a function of class $\mathcal{C}^1$ in an open connected set $B\subseteq\RR^n$ and $x,y\in B$. Then: $$\|\vf{f}(x)-\vf{f}(y)\|\leq\|\vf{Df}(z)\|\|x-y\|$$ for some $z\in[x,y]$.
  \end{theorem}
  \subsubsection{Higher order derivatives}
  \begin{definition}
    Let $U\subseteq\RR^n$ be an open set and $f:U\rightarrow\RR $. We denote the \textit{partial derivative of $f$ of order $k$ with respect to the variables $x_{i_1},\ldots,x_{i_k}$} at a point $a\in U$ as: $$\frac{\partial^kf}{\partial x_{i_k}\cdots\partial x_{i_1}}(a)$$
  \end{definition}
  \begin{definition}
    Let $U\subseteq\RR^n$ be an open set. If $f:U\rightarrow\RR $ has second order partial derivatives at $a\in U$, we define the \textit{hessian matrix of $f$ at a point $a$} as:
    $$\vf{H}f(a)=
      \begin{pmatrix}
        \displaystyle \frac{\partial^2 f}{{\partial x_1}^2}(a)         & \cdots & \displaystyle \frac{\partial^2 f}{\partial x_n\partial x_1}(a) \\
        \vdots                                                         & \ddots & \vdots                                                         \\
        \displaystyle \frac{\partial^2 f}{\partial x_1\partial x_n}(a) & \cdots & \displaystyle \frac{\partial^2 f}{{\partial x_n}^2}(a)
      \end{pmatrix}\footnote{Note that we can think $\vf{H}f(a)$ to be the associated matrix of a bilinear form $\vf{H}f(a)$.}$$
  \end{definition}
  \begin{theorem}[Schwarz's theorem]
    Let $U\subseteq\RR^n$ be an open set and $f:U\rightarrow\RR $. If $f$ has mixed partial derivatives of order $k$ and are continuous functions on $A\subseteq U$, then for any permutation $\sigma\in S_k$ we have: $$\frac{\partial^kf}{\partial x_{i_k}\cdots\partial x_{i_1}}(a)=\frac{\partial^kf}{\partial x_{\sigma(i_k)}\cdots\partial x_{\sigma(i_1)}}(a)\qquad\forall a\in A$$
  \end{theorem}
  \subsubsection{Inverse and implicit function theorems}
  \begin{lemma}
    Let $U\subseteq\RR^n$ be an open set and $\vf{f}:U\rightarrow\RR^m$ with $\vf{f}\in \mathcal{C}^1(U)$. Given $a\in U$ and $\varepsilon>0$, $\exists B(a,r)\subset U$ such that: $$\|\vf{f}(x)-\vf{f}(y)\|\leq(\|\vf{Df}(a)\|+\varepsilon)\|x-y\|\quad\forall x,y\in B(a,r)$$
  \end{lemma}
  \begin{lemma}
    Let $U\subseteq\RR^n$ be an open set and $\vf{f}:U\rightarrow\RR^n$ with $\vf{f}\in \mathcal{C}^1(U)$. Suppose that for some $a\in U$, $J\vf{f}(a)\ne 0$. Then $\exists B(a,r)\subset U$ and $c>0$ such that: $$\|\vf{f}(y)-\vf{f}(x)\|\geq c\|x-y\|,\quad\forall x,y\in B(a,r)$$ In particular, $\vf{f}$ is injective on $B(a,r)$.
  \end{lemma}
  \begin{theorem}[Inverse function theorem]
    Let $U\subseteq\RR^n$ be an open set, $\vf{f}:U\rightarrow\RR^n$ with $\vf{f}\in \mathcal{C}^1(U)$ and $a\in U$ such that $J\vf{f}(a)\ne0$. Then $\exists B:=B(a,r)\subset U$ such that:
    \begin{enumerate}
      \item $\vf{f}$ is injective on $B$.
      \item $\vf{f}(B)=V$ is an open set of $\RR^n$.
      \item $\vf{f}^{-1}:V\rightarrow B$ is of class $\mathcal{C}^1$ on $V$.
    \end{enumerate} Moreover, it is satisfied that $\vf{D}{\vf{f}}^{-1}(\vf{f}(a))={\vf{Df}(a)}^{-1}$
  \end{theorem}
  \begin{definition}
    A function $\vf{f}:U\subseteq\RR^n\rightarrow\RR^n$ is a \textit{diffeomorphism of class $\mathcal{C}^k$} if it is bijective and both $\vf{f}$ and $\vf{f}^{-1}$ are of class $\mathcal{C}^k$\footnote{By default, if we omit to say the class of the diffeomorphism, we will refer to a diffeomorphism of class $\mathcal{C}^1$.}.
  \end{definition}
  \begin{theorem}[Implicit function theorem]
    Let $U\subseteq\RR^{n+m}$ be an open set, $\vf{f}:U\rightarrow\RR^m$ with $\vf{f}\in \mathcal{C}^1(U)$ and $(a,b)=(a_1,\ldots,a_n,b_1,\ldots,b_m)\in U$ such that $\vf{f}(a,b)=0$. If $\vf{D}\vf{f}(x)=(\vf{D}\vf{f}_1(x)|\vf{D}\vf{f}_2(x))$ with $\vf{D}\vf{f}_1(x)\in\mathcal{M}_{m\times n}(\RR )$, $\vf{D}\vf{f}_2(x)\in\mathcal{M}_m(\RR )$ and $\det \vf{Df}_2(x)\ne 0$ (i.e. $\text{rang }\vf{Df}(a,b)=m$), then there exists an open set $W\subseteq\RR^n$ and a function $\vf{g}:W\rightarrow\RR^m$ such that $a\in W$, $\vf{g}\in\mathcal{C}^1(W)$ and: $$\vf{g}(a)=b\quad\text{and}\quad \vf{f}(x,\vf{g}(x))=0\quad\forall x\in W$$ Moreover, is is satisfied that: $$\vf{Dg}(a)=-{\vf{Df}_2(a,\vf{g}(a))}^{-1}\circ \vf{Df}_1(a,\vf{g}(a))$$
  \end{theorem}
  \subsubsection{Taylor's polynomial and maxima and minima}
  \begin{theorem}[Taylor's theorem]
    Let $U\subseteq\RR^n$ be an open set, $f:U\rightarrow\RR $, $a\in U$ and $f\in \mathcal{C}^{k+1}(U)$. Then:
    \begin{multline*}
      f(x)=f(a)+\\+\sum_{m=1}^k\frac{1}{m!}\left(\sum_{i_m,\ldots,i_1=1}^n\frac{\partial^mf}{\partial x_{i_m}\cdots\partial x_{i_1}}(a)\prod_{j=1}^m(x_{i_j}-a_{i_j})\right)+\\+R_k(f,a),
    \end{multline*} where
    \begin{multline*}
      R_k(f,a)=\\=\frac{1}{(k+1)!}\sum_{i_{k+1},\ldots,i_1=1}^n\frac{\partial^{k+1}f}{\partial x_{i_{k+1}}\cdots\partial x_{i_1}}(\xi)\prod_{j=1}^{k+1}(x_{i_j}-a_{i_j})=\\=o(\|x-a\|^k)
    \end{multline*} for some $\xi\in[a,x]$. In particular, for $k=2$ we have:
    \begin{multline*}
      f(x)=f(a)+Df(a)(x-a)+\frac{1}{2}Hf(a)(x-a,x-a)+\\+R_2(f,a),
    \end{multline*} where $R_2(f,a)=o(\|x-a\|^2)$.
  \end{theorem}
  \begin{definition}
    Let $U\subseteq\RR^n$ be an open set and $f:U\rightarrow\RR $. We say that $f$ has a \textit{local maximum at $a\in U$} if $\exists B(a,r)\subset U$ such that $f(x)\leq f(a)$, $\forall x\in B(a,r)$. Analogously, we say that $f$ has a \textit{local minimum at $a\in U$} if $\exists B(a,r)\subset U$ such that $f(x)\geq f(a)$, $\forall x\in B(a,r)$. A \textit{local extremum} is either a local maximum or a local minimum. Moreover, if $f(x)\leq f(a)$ $\forall x\in U$, we say that $f$ has a \textit{global maximum at $a\in U$}. Similarly if $f(x)\geq f(a)$ $\forall x\in U$, we say that $f$ has a \textit{global minimum at $a\in U$}.
  \end{definition}
  \begin{prop}
    Let $U\subseteq\RR^n$ be an open set and $f:U\rightarrow\RR $ be a differentiable function at $a\in U$. If $f$ has a local extremum at $a$, then $\grad f(a)=0$.
  \end{prop}
  \begin{definition}
    Let $U\subseteq\RR^n$ be an open set and $f:U\rightarrow\RR $. We say that $a\in U$ is a \textit{critical point of $f$} if $\grad f(a)=0$. We say that $a\in U$ is a \textit{saddle point} if $a$ is a critical point but not a local extremum.
  \end{definition}
  \begin{theorem}
    Let $\mathcal{Q}$ be a quadratic form. Then for all $x\ne 0$ we have:
    \begin{gather*}
      \mathcal{Q}\text{ is defined positive}\iff\exists\lambda\in\RR_{>0}:\mathcal{Q}(x)\geq\lambda\|x\|^2.\\
      \mathcal{Q}\text{ is defined negative}\iff\exists\lambda\in\RR^-:\mathcal{Q}(x)\leq\lambda\|x\|^2.
    \end{gather*}
  \end{theorem}
  \begin{prop}[Sylvester's criterion]
    Let $\vf{A}=(a_{ij})\in\mathcal{M}_n(\RR )$ be a symmetric matrix. $\vf{A}$ is defined positive if and only if all its principal minors are positive, that is:
    $$a_{11}>0,
      \begin{vmatrix}
        a_{11} & a_{12} \\
        a_{21} & a_{22}\end{vmatrix}>0,\ldots,
      \begin{vmatrix}
        a_{11} & \cdots & a_{1n} \\
        \vdots & \ddots & \vdots \\
        a_{n1} & \cdots & a_{nn}
      \end{vmatrix}>0$$
    $\vf{A}$ is defined negative if and only if its principal minor of order $k$ have sign $(-1)^k$, that is:
    $$a_{11}<0,
      \begin{vmatrix}
        a_{11} & a_{12} \\
        a_{21} & a_{22}
      \end{vmatrix}>0,\ldots,
      (-1)^n\begin{vmatrix}
        a_{11} & \cdots & a_{1n} \\
        \vdots & \ddots & \vdots \\
        a_{n1} & \cdots & a_{nn}
      \end{vmatrix}>0$$
  \end{prop}
  \begin{theorem}
    Let $U\subseteq\RR^2$ be an open set, $f:U\rightarrow\RR $ a function of class $\mathcal{C}^2(U)$ and $a\in U:\grad f(a)=0$. Let $\vf{H}f(a)$ be the hessian matrix of $f$ at $a$ and $Hf(a)$ be its associated quadratic form. Then:
    \begin{enumerate}
      \item If $Hf(a)$ is defined positive $\implies f$ has a local minimum at $a$.
      \item If $Hf(a)$ is defined negative $\implies f$ has a local maximum at $a$.
      \item If $Hf(a)$ is undefined $\implies f$ has a saddle point at $a$.
    \end{enumerate}
  \end{theorem}
  \begin{theorem}[Lagrange multipliers theorem]
    Let $f,g_i:U\subseteq\RR^n\rightarrow\RR $ be functions of class $\mathcal{C}^1(U)$ for $i=1,\ldots,k$ and $1\leq k<n$. Let $S=\{x\in U:g_i(x)=0,\;\forall i\}$ and $a\in S$ such that $f|_S(a)$ is a local extremum. If the vectors $\grad g_1(a),\ldots,\grad g_k(a)$ are linearly independent, then $\exists\lambda_1,\ldots,\lambda_k\in\RR $ such that: $$\grad f(a)=\sum_{i=1}^k\lambda_i\grad g_i(a)$$
  \end{theorem}
  \subsection{Integral calculus}
  \subsubsection{Integration over compact rectangles}
  \begin{definition}
    A \textit{rectangle} $R$ of $\RR^n$ is a product $R=I_1\times\cdots\times I_n$ where $I_j\in\RR $ are bounded and non-degenerate\footnote{That is, non-empty intervals with more than one point.} intervals.
  \end{definition}
  \begin{definition}
    The \textit{$n$-dimensional volume} (\textit{length} if $n=1$ and \textit{surface} if $n=2$) of a bounded rectangle $R=I_1\times\cdots\times I_n$, $I_i=[a_i,b_i]$ is: $$\text{vol}(R)=\prod_{i=1}^n(b_i-a_i)$$
  \end{definition}
  \begin{definition}
    Given a rectangle $R=I_1\times\cdots\times I_n$, a \textit{partition of $R$} is the product $\mathcal{P}=\mathcal{P}_1\times\cdots\times\mathcal{P}_n$ where $\mathcal{P}_j$ is a partition of the interval $I_j$. A partition $\mathcal{P}$ is \textit{regular} if for all $j$, $\mathcal{P}_j$ is regular, that is, all subintervals in $\mathcal{P}_j$ have the same size. We denote by $\textbf{P}(R)$ the set of all partitions of $R$.
  \end{definition}
  \begin{definition}
    Given two partitions $\mathcal{P}=\mathcal{P}_1\times\cdots\times \mathcal{P}_n$ and $\mathcal{P}'=\mathcal{P}_1'\times\cdots\times \mathcal{P}_n'$ of a rectangle $R$, we say that $\mathcal{P}'$ is \textit{finer than} $\mathcal{P}$ if each $\mathcal{P}_j'$ is finer than $\mathcal{P}_j$.
  \end{definition}
  \begin{definition}
    Let $R\subset\RR^n$ be a compact rectangle, $f:R\rightarrow\RR $ be a bounded function and $\mathcal{P}\in\textbf{P}(R)$. For each subrectangle $R_j$, $j=1,\ldots,m$, determined by $\mathcal{P}$ let $$m_j:=\inf\{f(x):x\in R_j\}\;\;\text{and}\;\; M_j:=\sup\{f(x):x\in R_j\}$$ We define the \textit{lower sum} and the \textit{upper sum of $f$ with respect to $\mathcal{P}$} as: $$L(f,\mathcal{P})=\sum_{j=1}^mm_j\text{vol}(R)\qquad U(f,\mathcal{P})=\sum_{j=1}^mM_j\text{vol}(R)\footnote{We will omit the results related to these definitions because of they are a natural extension of results of single-variable functions course and can be deduced easily. That's why we only expose the most important ones here.}$$
  \end{definition}
  \begin{definition}
    Let $R\subset\RR^n$ be a compact rectangle and $f:R\rightarrow\RR $ be a bounded function. We define the \textit{lower integral} and \textit{upper integral of $f$ on $R$} as
    \begin{gather*}
      \lowint{R}{}f=\sup\{L(f,\mathcal{P}):\mathcal{P}\in\textbf{P}\}\\
      \upint{R}{}f=\inf\{U(f,\mathcal{P}):\mathcal{P}\in\textbf{P}\}
    \end{gather*} We say that $f$ is \textit{Riemann-integrable on $R$} if $\displaystyle\lowint{R}{}f=\upint{R}{}f$.
  \end{definition}
  \begin{prop}
    Let $R\subset\RR^n$ be a compact rectangle and $f:R\rightarrow\RR $ be a bounded function. $f$ is Riemann-integrable if and only if $\forall\varepsilon$ $\exists\mathcal{P}\in\textbf{P}(R)$ such that $U(f,\mathcal{P})-L(f,\mathcal{P})<\varepsilon$.
  \end{prop}
  \begin{definition}
    Let $R\subset\RR^n$ be a compact rectangle, $f:R\rightarrow\RR $ be a bounded function, $\mathcal{P}\in\textbf{P}(R)$ and $\xi_j$ be an arbitrary point of the subrectangle $R_j$ for $j=1,\ldots,m$. Then, we define the \textit{Riemann sum of $f$ associated to $\mathcal{P}$} as: $$S(f,\mathcal{P})=\sum_{j=1}^mf(\xi_j)\text{vol}(R_j)$$
  \end{definition}
  \begin{theorem}
    Let $R\subset\RR^n$ be a compact rectangle and $f:R\rightarrow\RR $ be a bounded function. $f$ is Riemann-integrable over $R$ if and only if $\forall\varepsilon>0$ $\exists\mathcal{P}_\varepsilon\in\textbf{P}(R)$ such that: $$\left|S(f,\mathcal{P})-\int_Rf\right|=\left|\sum_{j=1}^mf(\xi_j)\text{vol}(R_j)-\int_Rf\right|<\varepsilon$$ for any $\mathcal{P}\in\textbf{P}(R)$ finer than $\mathcal{P}_\varepsilon$ and for any $\xi_j\in R_j$.
  \end{theorem}
  \subsubsection{Fubini's theorem}
  \begin{theorem}[Fubini's theorem]
    Let $R_1\subset\RR^n$ and $R_2\subset\RR^m$ be closed rectangles and $f:R_1\times R_2\rightarrow\RR $ be an integrable\footnote{As we have only defined Riemann-integration, it goes without saying that an \textit{integrable function} means a \textit{Riemann-integrable function}.} function. Suppose for every $x_0\in R_1$, $f(x_0,y)$ is integrable over $R_2$. Then, the function $\displaystyle g(x)=\int_{R_2}f(x,y)\dd y$ is integrable over $R_1$ and $$\int_{R_1\times R_2}f(x,y)=\int_{R_1}\dd x\int_{R_2}f(x,y)\dd y$$ Similarly if for every $y_0\in R_2$, $f(x,y_0)$ is integrable over $R_1$, then the function $\displaystyle h(y)=\int_{R_1}f(x,y)\dd x$ is integrable over $R_2$ and $$\int_{R_1\times R_2}f(x,y)=\int_{R_2}\dd y\int_{R_1}f(x,y)\dd x$$
    \begin{corollary}
      Let $R_1\subset\RR^n$ and $R_2\subset\RR^m$ be closed rectangles and let $f:R_1\times R_2\rightarrow\RR $ be a continuous function on $R_1\times R_2$. Then: $$\int_{R_1\times R_2}f=\int_{R_1}\dd x\int_{R_2}f(x,y)\dd y=\int_{R_2}\dd y\int_{R_1}f(x,y)\dd x$$
    \end{corollary}
    \begin{corollary}
      Let $R=[a_1,b_1]\times\cdots\times[a_n,b_n]\subset\RR^n$ be a rectangle. If $f:R\rightarrow\RR $ is a continuous function, then $$\int_Rf=\int_{a_n}^{b_n}\dd x_n\int_{a_{n-1}}^{b_{n-1}}\dd x_{n-1}\cdots\int_{a_1}^{b_1}f(x_1,\ldots,x_n)\dd x_1$$
    \end{corollary}
    \begin{definition}
      Let $D\subset\RR^{n-1}$ be a compact set and $\varphi_1,\varphi_2:D\rightarrow\RR $ be continuous functions such that $\varphi_1(x)\leq\varphi_2(x)$ $\forall x\in D$. The set $$S=\{(x,y)\subset\RR^n:x\in D, \varphi_1(x)\leq y\leq\varphi_2(x)\}$$ is called an \textit{elementary region in $\RR^n$}. In particular, if $n=2$, we say $S$ is \textit{$x$-simple}. An elementary region in $V\subset\RR^3$ is called \textit{$xy$-simple} if it is of the form: $$V=\{(x,y,z)\in\RR^3:(x,y)\in U, \phi_1(x,y)\leq z\leq\phi_2(x,y)\}$$ where $U$ is an elementary region in $\RR^2$ and $\phi_1,\phi_2$ are continuous functions on $U$\footnote{Analogously we define \textit{$y$-simple} regions in $\RR^2$ and \textit{$yz$-simple} or \textit{$xz$-simple} regions in $\RR^3$.}.
    \end{definition}
    \begin{theorem}[Fubini's theorem for elementary regions]
      Let $D\subset\RR^{n-1}$ be a compact set, $\varphi_1,\varphi_2:D\rightarrow\RR $ be continuous functions such that $\varphi_1(x)\leq\varphi_2(x)$ $\forall x\in D$, $S=\{(x,y)\subset\RR^n:x\in D, \varphi_1(x)\leq y\leq\varphi_2(x)\}$ be an elementary region in $\RR^n$ and $f:S\rightarrow\RR $. If $f$ is integrable over $S$ and for all $x_0\in D$ the function $f(x_0,y)$ is integrable over $[-M,M]$, $M\in\RR $, then: $$\int_Sf=\int_D\dd x\int_{\varphi_1(x)}^{\varphi_2(x)}f(x,y)\dd y$$
    \end{theorem}
    \begin{definition}
      Let $D\subset\RR^{n-1}$ be a compact set, $\varphi_1,\varphi_2:D\rightarrow\RR $ be continuous functions such that $\varphi_1(x)\leq\varphi_2(x)$ $\forall x\in D$ and $S=\{(x,y)\subset\RR^n:x\in D, \varphi_1(x)\leq y\leq\varphi_2(x)\}$ an elementary region. We define the \textit{$n$-dimensional volume of $S$} as $$\text{vol}(S):=\int_S\dd x=\int_D\dd x\int_{\varphi_1(x)}^{\varphi_2(x)}\dd y\footnote{In particular, we define the area of a region $S\subset\RR^2$ as $\displaystyle\text{area}(S)=\iint_S\dd x\dd y$ and the volume of a region $\Omega\subset\RR^3$ as $\displaystyle\text{vol}(\Omega)=\iiint_\Omega \dd x\dd y\dd z$.}$$
    \end{definition}
    \begin{corollary}[Cavalieri's principle]
      Let $\Omega\subset R\times[a,b]$ be a set in $\RR^n$ where $R\subset\RR^{n-1}$ is a rectangle. For every $t\in[a,b]$ let $$\Omega_t=\{(x,y)\in\Omega:y=t\}\subset\RR^n$$ be the section of $\Omega$ corresponding to the hyperplane $y=t$. If $\nu(\Omega_t)$ is the $(n-1)$-dimensional volume (length if $n=2$ and area if $n=3$) of $\Omega_t$, then: $$\text{vol}(\Omega)=\int_a^b\nu(\Omega_t)\dd t$$
    \end{corollary}
  \end{theorem}
  \begin{definition}[Center of mass]
    The \textit{center of mass of an object with mass density $\rho(x,y,z)$} occupying a region $\Omega\subset\RR^3$ is the point $(\overline{x},\overline{y},\overline{z})\in\RR^3$ whose coordinates are:
    \begin{gather*}
      \overline{x}=\frac{1}{m}\iiint_\Omega x\rho(x,y,z)\dd x\dd y\dd z,\\
      \overline{y}=\frac{1}{m}\iiint_\Omega y\rho(x,y,z)\dd x\dd y\dd z,\\
      \overline{z}=\frac{1}{m}\iiint_\Omega z\rho(x,y,z)\dd x\dd y\dd z,
    \end{gather*}
    where $\displaystyle m=\iiint_\Omega\rho(x,y,z)\dd x\dd y\dd z$ is the total mass of the object.
  \end{definition}
  \begin{definition}[Moment of inertia]
    Given a body with mass density $\rho(x,y,z)$ occupying a region $\Omega\subset\RR^3$ and a line $L\subset\RR^3$, the \textit{moment of inertia of the body about the line $L$} is: $$I_L=\iiint_\Omega d(x,y,z)^2\rho(x,y,z)\dd x\dd y\dd z$$ where $d(x,y,z)$ denotes the distance from $(x,y,z)$ to the line $L$. In particular, when $L$ is the $z$-axis, then: $$I_z=\iiint_\Omega (x^2+y^2)\rho(x,y,z)\dd x\dd y\dd z$$ and similarly for $I_x$ and $I_y$. The moment of inertia of the body about the $xy$-plane is defined by: $$I_{xy}=\iiint_\Omega z^2\rho(x,y,z)\dd x\dd y\dd z$$ and similarly for $I_{yz}$ and $I_{zx}.$
  \end{definition}
  \subsubsection{Change of variable}
  \begin{theorem}[Change of variable theorem]
    Let $U\subseteq\RR^n$ be an open set and let $\vf{\varphi}:U\rightarrow\RR^n$ be a diffeomorphism. If $f:\vf{\varphi}(U)\rightarrow\RR $ is integrable on $\vf{\varphi}(U)$, then $$\int_{\vf{\varphi}(U)} f=\int_U(f\circ\vf{\varphi})|J\vf{\varphi}|$$
  \end{theorem}
  \begin{corollary}[Integral in polar coordinates]
    Let $\varphi:U\subseteq[0,\infty)\times[0,2\pi)\rightarrow\RR^2 $ be such that:
    $$\varphi(r,\theta)\longmapsto(r\cos\theta,r\sin\theta)$$
    Then, we have $|J\varphi|=r$ and therefore: $$\int_{\varphi(U)}f(x,y)\dd x\dd y=\int_Uf(r\cos\theta,r\sin\theta)r\dd r \dd\theta$$
  \end{corollary}
  \begin{corollary}[Integral in cylindrical coordinates]
    Let $\varphi:U\subseteq[0,\infty)\times[0,2\pi)\times\RR \rightarrow\RR^3 $ be such that: $$\varphi(r,\theta,z)\longmapsto(r\cos\theta,r\sin\theta,z)$$
    Then, we have $|J\varphi|=r$ and therefore: $$\int_{\varphi(U)}f(x,y,z)\dd x\dd y\dd z=\int_Uf(r\cos\theta,r\sin\theta,z)r\dd r \dd\theta \dd z$$
  \end{corollary}
  \begin{corollary}[Integral in spherical coordinates]
    Let $\varphi:U\subseteq[0,\infty)\times[0,2\pi)\times[0,\pi]\rightarrow\RR^3 $ be such that: $$\varphi(\rho,\theta,\phi)\longmapsto(\rho\sin\phi\cos\theta,\rho\sin\phi\sin\theta,\rho\cos\phi)$$
    Then, we have $|J\varphi|=\rho^2\sin\phi$ and therefore:
    \begin{multline*}
      \int_{\varphi(U)}f(x,y,z)\dd x\dd y\dd z=\\=\int_Uf(\rho\sin\phi\cos\theta,\rho\sin\phi\sin\theta,\rho\cos\phi)\rho^2\sin\phi \dd \rho \dd \theta \dd \phi
    \end{multline*}
  \end{corollary}
  \subsection{Vector calculus}
  \subsubsection{Arc-length and line integrals}
  \begin{definition}
    Let $\vf{\gamma}:[a,b]\rightarrow\RR^n$ be a parametrization of a curve and $\mathcal{P}=\{t_0,\ldots,t_n\}$ be a partition of $[a,b]$. Then, the \textit{length of the polygonal} created from the vertices $\vf{\gamma}(t_i)$, $i=1,\ldots,n$, is: $$L(\vf{\gamma},\mathcal{P})=\sum_{i=1}^n\|\vf{\gamma}(t_i)-\vf{\gamma}(t_{i-1})\|$$
  \end{definition}
  \begin{definition}
    Let $\vf{\gamma}:[a,b]\rightarrow\RR^n$ be a parametrization of a curve $C$. The \textit{arc length of $C$} is $$L(C)=\sup\{L(\vf{\gamma},\mathcal{P}):\mathcal{P}\in\textbf{P}([a,b])\}\in[0,\infty]$$
  \end{definition}
  \begin{definition}
    We say that a curve $C$ is \textit{rectifiable} if it has a finite arc length, that is, if $L(C)<\infty$.
  \end{definition}
  \begin{prop}
    Let $\vf{\gamma}:[a,b]\rightarrow\RR^n$ be a parametrization of class $\mathcal{C}^1$ of a curve $C$. Then $C$ is rectifiable and $$L(C)=\int_a^b\|\vf{\gamma}'(t)\|\dd t\footnote{It can be seen that the arc length of a curve does not depend on its parametrization.}$$
  \end{prop}
  \begin{definition}
    Let $\vf{F}:U\subset\RR^m\rightarrow\RR^n$ be a vector field\footnote{A \textit{vector field} is nothing more than a vector-valued function.}. If all its component functions $F_i$ are integrable, we define: $$\int_U\vf{F}:=\left(\int_UF_1,\ldots,\int_UF_n\right)\in\RR^n$$
  \end{definition}
  \begin{definition}
    Let $C$ be a curve in $\RR^2$ parametrized by $\vf{\gamma}=(x(t),y(t))$. The unit tangent vector to the curve at time $t$ is: $$\vf{t}=\frac{\vf{\gamma}'(t)}{\|\vf{\gamma}'(t)\|}$$ The normal vector to the curve is $N(t)=(y'(t),-x'(t))$ and the unit normal vector to the curve is: $$\vf{n}=\frac{N(t)}{\|N(t)\|}\footnote{Observe that $-N(t)$ is also a normal vector to the curve but, by agreement, we take the one pointing to the right of the curve or, if the curve is closed, the one pointing outwards from the curve.}$$
  \end{definition}
  \begin{definition}
    Let $C$ be a curve parametrized by $\vf{\gamma}:[a,b]\rightarrow\RR^n$ and $\varphi:[c,d]\rightarrow[a,b]$ be a diffeomorphism. The composition $\vf{\gamma}\circ\varphi:[c,d]\rightarrow\RR^n$ is called a \textit{reparametrization of $C$}.
  \end{definition}
  \begin{definition}
    Let $C$ be a curve of class $\mathcal{C}^1$ parametrized by $\vf{\gamma}:[a,b]\rightarrow\RR^n$ an $L$ be its arc length. We define the \textit{arc length parameter} as: $$s(t)=\int_a^t\|\vf{\gamma}'(t)\|\dd t$$ We reparametrize $C$ by $\rho (s)=\vf{\gamma}(t(s))$, $0\leq s\leq L$. Then $\rho'(s)$ is a unit tangent vector to $C$ and $\rho''(s)$ is perpendicular to $C$ at the point $\rho(s)$.
  \end{definition}
  \begin{definition}
    Let $C$ be a curve of class $\mathcal{C}^2$ and $s$ be its arc length parameter. We define the \textit{curvature} of $C$ at the point $\rho(s)$ as $$\kappa(\rho(s))=\|\rho''(s)\|$$
  \end{definition}
  \begin{definition}
    Let $C=\{\vf{\gamma}(t):t\in[a,b]\}\subset\RR^n$ be a curve of class $\mathcal{C}^1$ and $f:\RR^n\rightarrow\RR $ be continuous function. We define the \textit{line integral of $f$ along $C$} as: $$\int_Cf\dd s=\int_a^bf(\vf{\gamma}(t))\|\vf{\gamma}'(t)\|\dd t\footnote{It can be seen that this integral is independent of the parametrization of $C$.}$$
  \end{definition}
  \begin{definition}
    Let $C=\{\vf{\gamma}(t):t\in[a,b]\}\subset\RR^n$ be a curve of class $\mathcal{C}^1$ and $\vf{F}:\RR^n\rightarrow\RR^n$ be a continuous vector field. We define the \textit{line integral of $\vf{F}$ along $C$} as $$\int_C\vf{F}\cdot \dd \vf{s}=\int_C\vf{F}\cdot \vf{t} \dd s=\int_a^b\vf{F}(\vf{\gamma}(t))\cdot\vf{\gamma}'(t) \dd t$$ where $\vf{t}$ is the unit tangent vector to $C$\footnote{It can be seen that the latter integral is independent of the parametrization of $C$ except for a factor of $-1$ that depends on the orientation of the parametrization.}. If $C$ is closed, this integral is called the \textit{circulation of $\vf{F}$ around $C$}.
  \end{definition}
  \begin{definition}
    A \textit{Jordan arc} is the image of an injective continuous map $\vf{\gamma}:[a,b]\rightarrow\RR^n$. A \textit{Jordan closed curve} is the image of an injective continuous map $\vf{\gamma}:[a,b]\rightarrow\RR^n$ such that $\vf{\gamma}(a)=\vf{\gamma}(b)$.
  \end{definition}
  \subsubsection{Conservative vector fields}
  \begin{definition}
    Let $U\subseteq\RR^n$ be a domain and $f:U\rightarrow\RR $ be a function of class $\mathcal{C}^1$. We say that $\vf{F}:U\rightarrow\RR^n$ is a \textit{conservative} or a \textit{gradient vector field} if $$\vf{F}(x)=\grad f(x)\qquad \forall x\in U$$ The function $f$ is called the \textit{potential} of $\vf{F}$.
  \end{definition}
  \begin{theorem}
    Let $\vf{F}=\grad f$ be a conservative vector field on $U\subseteq\RR^n$ and $C$ be a closed curve that admits a parametrization $\vf{\gamma}(t):[a,b]\rightarrow\RR^n$ of class $\mathcal{C}^1(U)$. Then: $$\int_C\vf{F}\cdot \dd \vf{s}=f(\vf{\gamma}(b))-f(\vf{\gamma}(a))$$
  \end{theorem}
  \begin{corollary}
    Let $\vf{F}$ be a conservative vector field on $U$ and $C$ be a closed curve that admits a parametrization of class $\mathcal{C}^1(U)$. Then $\displaystyle\int_C\vf{F}\cdot \dd \vf{s}=0$.
  \end{corollary}
  \subsubsection{Divergence, curl and Laplacian}
  \begin{definition}
    Let $\vf{F}=(F_1,\ldots, F_n)$ be a vector field of class $\mathcal{C}^1(U)$, $U\subseteq\RR^n$. The \textit{divergence of $\vf{F}$} is: $$\div\vf{F}=\divp\vf{F}=\sum_{i=1}^n\frac{\partial F_j}{\partial x_j}$$
  \end{definition}
  \begin{definition}
    Let $\vf{F}=(F_1,F_2,F_3)$ be a vector field of class $\mathcal{C}^1(U)$, $U\subseteq\RR^3$. The \textit{curl of $\vf{F}$} is:
    \begin{multline*}
      \rot\vf{F}=\rotp\vf{F}=
      \begin{vmatrix}
        \vf{i}                      & \vf{j}                      & \vf{k}                      \\
        \frac{\partial}{\partial x} & \frac{\partial}{\partial y} & \frac{\partial}{\partial z} \\
        F_1                         & F_2                         & F_3                         \\
      \end{vmatrix}=\\=\left(\frac{\partial F_3}{\partial y}-\frac{\partial F_2}{\partial z},\frac{\partial F_1}{\partial z}-\frac{\partial F_3}{\partial x},\frac{\partial F_2}{\partial x}-\frac{\partial F_1}{\partial y}\right)
    \end{multline*}
  \end{definition}
  \begin{definition}
    Let $f:U\subseteq\RR^n\rightarrow\RR $ be a function of class $\mathcal{C}^2(U)$, $U\subseteq\RR^3$. The \textit{Laplacian of $f$} is $$\laplacian f=\Delta f=\sum_{i=1}^n\frac{\partial^2 f}{\partial x_j^2}$$
  \end{definition}
  \begin{prop}
    Let $U$ be an open set of $\RR^3$ and $f:U\rightarrow\RR $, $\vf{g}:U\rightarrow\RR^3$ be functions of class $\mathcal{C}^2(U)$. Then for all $x\in U$ we have: $$\rot(\grad f)=0\qquad\div(\rot\vf{g})=0\;\;\;\text{and}\;\;\;\div(\grad f)=\laplacian f$$
  \end{prop}
  \subsubsection{Surface area and surface integrals}
  \begin{prop}
    Let $S$ be the graph of a function $z=f(x,y)$ of class $\mathcal{C}^1(U)$, $U\subseteq\RR^2$. Then $$\text{area}\,(S)=\iint_U\sqrt{1+{\left(\frac{\partial f}{\partial x}\right)}^2+{\left(\frac{\partial f}{\partial y}\right)}^2}\dd x\dd y$$
  \end{prop}
  \begin{definition}
    A \textit{parametrized surface $S\subset\RR^3$} is the image of a map $\vf{\Phi}:U\subseteq\RR^2\rightarrow\RR^3$ of class $\mathcal{C}^1(U)$ defined by $\vf{\Phi}(u,v)=(x(u,v),y(u,v),z(u,v))$.
  \end{definition}
  \begin{prop}
    Let $S=\vf{\Phi}(U)$ be a surface in $\RR^3$ parametrized by $\vf{\Phi}\in\mathcal{C}^1(U)$. Then the unit normal vector to $S$ at the point $\vf{\Phi}(u,v)$ is $$\vf{n}(u,v)=\frac{\frac{\partial\vf{\Phi}}{\partial u}\crossprod\frac{\partial\vf{\Phi}}{\partial v}}{\left\|\frac{\partial\vf{\Phi}}{\partial u}\crossprod\frac{\partial\vf{\Phi}}{\partial v}\right\|}$$
  \end{prop}
  \begin{prop}
    Let $S=\vf{\Phi}(U)$ be a surface in $\RR^3$ parametrized by $\vf{\Phi}\in\mathcal{C}^1(U)$. Then: $$\text{area}(S)=\iint_U\left\|\frac{\partial \vf{\Phi}}{\partial u}\crossprod\frac{\partial \vf{\Phi}}{\partial v}\right\|\dd u\dd v$$
  \end{prop}
  \begin{definition}
    Let $S=\vf{\Phi}(U)$ be a surface in $\RR^3$ parametrized by $\vf{\Phi}\in\mathcal{C}^1(U)$ and $f:\RR^3\rightarrow\RR $ be a continuous function whose domain contain $S$. We define the \textit{surface integral $f$ over $S$} as: $$\iint_Sf\dd S=\iint_U f(\vf{\Phi}(u,v))\left\|\frac{\partial \vf{\Phi}}{\partial u}\crossprod\frac{\partial \vf{\Phi}}{\partial v}\right\|\dd u\dd v\footnote{It can be seen that this integral is independent of the parametrization of $S$.}$$
  \end{definition}
  \begin{definition}
    Let $S=\vf{\Phi}(U)$ be a surface in $\RR^3$ parametrized by $\vf{\Phi}\in\mathcal{C}^1(U)$ and $\vf{f}:\RR^3\rightarrow\RR^3$ be a continuous vector field  whose domain contain $S$. We define the \textit{surface integral $\vf{f}$ over $S$} or the \textit{flux of $\vf{f}$ across $S$} as:
    \begin{multline*}
      \iint_S\vf{f}\cdot \dd \vf{S}=\iint_S\vf{f}\cdot\vf{n} \dd S=\\=\iint_U \vf{f}(\vf{\Phi}(u,v))\cdot\left(\frac{\partial \vf{\Phi}}{\partial u}\crossprod\frac{\partial \vf{\Phi}}{\partial v}\right) \dd u\dd v
    \end{multline*} where $\vf{n}$ is the unit normal vector to $S$\footnote{It can be seen that the latter integral is independent of the parametrization of $S$ except for a factor of $-1$ that depends on the orientation of the normal vector $\vf{n}$.}.
  \end{definition}
  \subsubsection{Theorems of vector calculus on \texorpdfstring{$\RR^2$}{R2}}
  \begin{definition}
    Let $U\subseteq\RR^3$ be an open set. A \textit{differential 1-form on $U$} is an expression of the form $$\omega=f_1\dd x+f_2\dd y+f_3\dd z$$ where $f_1,f_2,f_3$ are scalar functions defined on $U$\footnote{Extending this notion, we can define 2-forms and 3-forms as:
      $$\begin{array}{cl}
          \omega=f_1\dd x\dd y+f_2\dd y\dd z+f_3\dd z\dd y & \text{2-form} \\
          \omega=f\dd x\dd y\dd z                          & \text{3-form}
        \end{array}$$}.
  \end{definition}
  \begin{theorem}[Green's theorem]
    Let $\vf{F}=(F_1,F_2)$ be a vector field of class $\mathcal{C}^1(U)$, $U\subseteq\RR^2$, and $c=\partial U$ be the curve formed from the boundary of $U$\footnote{It goes without saying that the orientation is chosen positive, that is counterclockwise.}. Then: $$\int_{\partial U}\vf{F}\cdot \dd \vf{s}=\iint_U\rot\vf{F}\dd x\dd y\footnote{Alternatively, using differential forms, we get $$\int_{\partial U}(F_1\dd x+F_2\dd y)=\iint_U\left(\frac{\partial F_2}{\partial x}-\frac{\partial F_1}{\partial y}\right)\dd x\dd y$$}$$
  \end{theorem}
  \begin{corollary}
    Let $U$ be a region in $\RR^2$ and $\partial U$ be its boundary. Then: $$\text{area}(U)=\int_{\partial U}x\dd y=-\int_{\partial U}y\dd x=\frac{1}{2}\int_{\partial U}(x\dd y-y\dd x)$$
  \end{corollary}
  \begin{theorem}[Divergence theorem on $\RR^2$]
    Let $\vf{F}=(F_1,F_2)$ be a vector field of class $\mathcal{C}^1(U)$, $U\subseteq\RR^2$ with boundary $\partial U$. Then: $$\int_{\partial U}\vf{F}\cdot\vf{n} \dd s=\iint_U\div\vf{F}\dd x\dd y\footnote{The first integral represents the flux of $\vf{F}$ across the curve $\partial U$.}$$
  \end{theorem}
  \subsubsection{Theorems of vector calculus on \texorpdfstring{$\RR^3$}{R3}}
  \begin{theorem}[Stokes' theorem]
    Let $S$ be a parametrized surface of class $\mathcal{C}^1$ and $\partial S$ be its boundary. Let $\vf{F}=(F_1,F_2,F_3)$ be a vector field of class $\mathcal{C}^1$ in a domain containing $S\cup\partial S$. Then: $$\int_{\partial S}\vf{F}\cdot \dd \vf{s}=\iint_S\rot\vf{F}\cdot\vf{n} \dd S$$
  \end{theorem}
  \begin{corollary}
    Let $a\in\RR^3$ and $\vf{n}$ be a unit vector. Suppose $D_r=D(a,r)$ is a disk of radius $r$ centered at $a$ and perpendicular to $\vf{n}$. Let $\vf{F}$ be a vector field of class $\mathcal{C}^1(D_r)$. Then: $$\rot\vf{F}(a)\cdot\vf{n}=\lim_{r\to 0}\frac{1}{\text{area}(D_r)}\int_{\partial D_r}\vf{F}\cdot \dd \vf{s}$$ Therefore, the $\vf{n}$-th component of $\rot\vf{F}(a)$ is the circulation of $\vf{F}$ in a small circular surface perpendicular to $\vf{n}$, per unit of area.
  \end{corollary}
  \begin{definition}
    A region of $\RR^3$ is \textit{symmetric} if is $xy$-simple, $yz$-simple and $xz$-simple.
  \end{definition}
  \begin{theorem}[Gau\ss' or divergence theorem on $\RR^3$]
    Let $\vf{F}=(F_1,F_2,F_3)$ be a vector field of class $\mathcal{C}^1$ on a symmetric region $V\subset\RR^3$ with boundary $\partial V$. Then: $$\iint_{\partial V}\vf{F}\cdot\vf{n} \dd S=\iiint_V\div\vf{F}\dd x\dd y\dd z$$
  \end{theorem}
  \begin{corollary}
    Let $B_r=B(a,r)$ be a ball of radius $r$ centered at $a\in\RR^3$ and $\vf{F}$ be a vector field of class $\mathcal{C}^1(B_r)$. Then: $$\div\vf{F}(a)=\lim_{r\to 0}\frac{1}{\text{vol}(B_r)}\iint_{\partial B_r}\vf{F}\cdot\vf{n} \dd S$$ Therefore, $\div\vf{F}(a)$ is the flux of $\vf{F}$ outward form $a$, in the normal direction across the surface of a small ball centered on $a$, per unit of volume.
  \end{corollary}
\end{multicols}
\end{document}
