\documentclass[../../../main_math.tex]{subfiles}
% break in Fejér's and Dini's theorem 

\begin{document}
\changecolor{MA}
\begin{multicols}{2}[\section{Mathematical analysis}]
  \subsection{Numeric series}
  \subsubsection{Series convergence}
  \begin{definition}
    Let $(a_n)$ be a sequence of real numbers. A \emph{numeric series} is an expression of the form $$\sum_{n=1}^\infty a_n$$ We call $a_n$ \emph{general term of the series} and $\displaystyle S_N=\sum_{n=1}^N a_n$, for all $N\in\NN $, \emph{$N$-th partial sum of the series}\footnote{From now on we will write $\sum a_n$ to refer $\displaystyle\sum_{n=1}^\infty a_n$.}.
  \end{definition}
  \begin{definition}
    We say the series $\sum a_n$ is \emph{convergent} if the sequence of partial sums is convergent, that is, if $\displaystyle S=\lim_{N\to\infty}S_N$ exists and it is finite. In that case, $S$ is called the \emph{sum of the series}. If the previous limit doesn't exists or it is infinite, we say the series is \emph{divergent}\footnote{We will use the notation $\sum a_n<\infty$ or $\sum a_n=+\infty$ to express that the series converges or diverges, respectively.}.
  \end{definition}
  \begin{proposition}
    Let $(a_n)$ be a sequence such that $\sum a_n<\infty$. Then, $\forall\varepsilon>0$ $\exists n_0\in\NN $ such that $$\left|\sum_{n=1}^N a_n-\sum_{n=1}^\infty a_n\right|<\varepsilon$$ if $N\geq n_0$.
  \end{proposition}
  \begin{theorem}[Cauchy's test]
    Let $(a_n)$ be a sequence. $\sum a_n<\infty$ if and only if $\forall\varepsilon>0$ $\exists n_0\in\NN $ such that $$\left|\sum_{n=N}^Ma_n\right|<\varepsilon$$ if $M\geq N\geq n_0$.
  \end{theorem}
  \begin{corollary}
    Changing a finite number of terms in a series has no effect on the convergence or divergence of the series.
  \end{corollary}
  \begin{corollary}
    If $\sum a_n<\infty$, then $\displaystyle\lim_{n\to \infty} a_n=0$.
  \end{corollary}
  \begin{theorem}[Linearity]
    Let $\sum a_n,\sum b_n$ be two convergent series with sums $A$ and $B$ respectively and let $\lambda$ be a real number. The series $$\sum_{n=1}^\infty (a_n+\lambda b_n)$$ is convergent and has sum $A+\lambda B$.
  \end{theorem}
  \begin{theorem}[Associative property]
    Let $\sum a_n$ be a convergent series with sum $A$. Suppose $(n_k)$ is a strictly increasing sequence of natural numbers. The series $\sum b_n$, with $b_k=a_{n_{k-1}+1}+\cdots+a_{n_k}$ for all $i\in\NN $, is convergent and its sum is $A$.
  \end{theorem}
  \subsubsection{Non-negative terms series}
  \begin{theorem}
    Let $\sum a_n$ be a series of non-negative terms $a_n\geq 0$\footnote{Obviously the following results are also valid if the series is of non-positive terms or has a finite number of negative or positive terms.}. The series converges if and only if the sequence $(S_N)$ of partial sums is bounded.
  \end{theorem}
  \begin{theorem}[Comparison test]
    Let $(a_n),(b_n)\geq 0$ be two sequences of real numbers. Suppose that exists a constant $C>0$ and a number $n_0\in\NN $ such that $a_n\leq Cb_n$ for all $n\geq n_0$.
    \begin{enumerate}
      \item If $\sum b_n<\infty\implies\sum a_n<\infty$
      \item If $\sum a_n=+\infty\implies\sum b_n=+\infty$
    \end{enumerate}
  \end{theorem}
  \begin{theorem}[Limit comparison test]
    Let $(a_n)$, $(b_n)\geq 0$ be two sequences of real numbers. Suppose that the limit $\ell=\displaystyle\lim_{n\to\infty}\frac{a_n}{b_n}$ exists.
    \begin{enumerate}
      \item If $0<\ell<\infty\implies\sum a_n<\infty\iff\sum b_n<\infty$
      \item If $\ell=0$ and $\sum b_n<\infty\implies\sum a_n<\infty$
      \item If $\ell=\infty$ and $\sum a_n<\infty\implies\sum b_n<\infty$
    \end{enumerate}
  \end{theorem}
  \begin{theorem}[Root test]
    Let $(a_n)\geq 0$. Suppose that the limit $\displaystyle \ell=\lim_{n\to\infty}\sqrt[n]{a_n}$ exists.
    \begin{enumerate}
      \item If $\ell<1\implies\sum a_n<\infty$
      \item If $\ell>1\implies\sum a_n=+\infty$
    \end{enumerate}
  \end{theorem}
  \begin{theorem}[Ratio test]
    Let $(a_n)\geq 0$. Suppose that the limit $\displaystyle \ell=\lim_{n\to\infty}\frac{a_{n+1}}{a_n}$ exists.
    \begin{enumerate}
      \item If $\ell<1\implies\sum a_n<\infty$
      \item If $\ell>1\implies\sum a_n=+\infty$
    \end{enumerate}
  \end{theorem}
  \begin{theorem}[Raabe's test]
    Let $(a_n)\geq 0$. Suppose that the limit $\displaystyle \ell=\lim_{n\to\infty}n\left(1-\frac{a_{n+1}}{a_n}\right)$ exists.
    \begin{enumerate}
      \item If $\ell>1\implies\sum a_n<\infty$
      \item If $\ell<1\implies\sum a_n=+\infty$
    \end{enumerate}
  \end{theorem}
  \begin{theorem}[Condensation test]
    Let $(a_n)\geq 0$ be a decreasing sequence. Then: $$\textstyle\sum a_n<\infty\iff\sum 2^na_{2^n}<\infty$$
  \end{theorem}
  \begin{theorem}[Logarithmic test]
    Let $(a_n)\geq 0$. Suppose that the limit $\displaystyle \ell=\lim_{n\to\infty}\frac{\log\frac{1}{a_n}}{\log n}$ exists.
    \begin{enumerate}
      \item If $\ell>1\implies\sum a_n<\infty$
      \item If $\ell<1\implies\sum a_n=+\infty$
    \end{enumerate}
  \end{theorem}
  \begin{theorem}[Integral test]\label{MA:inttest}
    Let $f:[1,\infty)\rightarrow(0,\infty)$ be a decreasing function. Then:
    \begin{multline*}
      \textstyle\sum f(n)<\infty\iff\\ \displaystyle\iff\exists C>0\text{ such that }\int_1^nf(x)\dd{x}\leq C\;\forall n
    \end{multline*}
  \end{theorem}
  \subsubsection{Alternating series}
  \begin{definition}
    An \emph{alternating series} is a series of the form $\sum (-1)^na_n$, with $(a_n)\geq 0$.
  \end{definition}
  \begin{theorem}[Leibnitz's test]
    Let $(a_n)\geq 0$ be a decreasing sequence such that $\displaystyle\lim_{n\to\infty}a_n=0$. Then, $\sum (-1)^na_n$ is convergent.
  \end{theorem}
  \begin{theorem}[Abel's summation formula]
    Let $(a_n),(b_n)$ be two sequences of real numbers. Then:
    \begin{multline*}
      \sum_{n=N}^M a_n(b_{n+1}-b_n)=a_{M+1}b_{M+1}-a_Nb_N-\\
      -\sum_{n=N}^Mb_{n+1}(a_{n+1}-a_n)
    \end{multline*}
  \end{theorem}
  \begin{theorem}[Dirichlet's test]
    Let $(a_n),(b_n)$ be two sequences of real numbers such that:
    \begin{enumerate}
      \item $\exists C>0$ such that $\displaystyle\left|\sum_{n=1}^Na_n\right|\leq C$ for all $N\in\NN $.
      \item $(b_n)$ is monotone and $\displaystyle\lim_{n\to\infty}b_n=0$.
    \end{enumerate}
    Then, $\sum a_nb_n$ is convergent.
  \end{theorem}
  \begin{theorem}[Abel's test]
    Let $(a_n),(b_n)$ be two sequences of real numbers such that:
    \begin{enumerate}
      \item The series $\sum a_n$ is convergent.
      \item $(b_n)$ is monotone and bounded.
    \end{enumerate}
    Then, $\sum a_nb_n$ is convergent.
  \end{theorem}
  \subsubsection{Absolute convergence and rearrangement of series}
  \begin{definition}
    We say a series $\sum a_n$ is \emph{absolutely convergent} if $\sum |a_n|$ is convergent.
  \end{definition}
  \begin{theorem}
    If a series converges absolutely, it converges.
  \end{theorem}
  \begin{definition}
    We say a sequence $(b_n)$ is a \emph{rearrangement of the sequence} $(a_n)$ if exists a bijective map $\sigma:\NN \rightarrow\NN $ such that $b_n=a_{\sigma(n)}$. A \emph{rearrangement of the series} $\sum a_n$ is the series $\sum a_{\sigma(n)}$ for some bijection $\sigma:\NN \rightarrow\NN $.
  \end{definition}
  \begin{definition}
    Let $x\in\RR$. We define the \emph{positive part} of $x$ as $$x^+=
      \begin{cases}
        x & \text{if }x\geq 0 \\
        0 & \text{if }x<0
      \end{cases}$$ Analogously, we define the \emph{negative part} of $x$ as $$x^-=
      \begin{cases}
        0  & \text{if }x\geq 0 \\
        -x & \text{if }x<0
      \end{cases}$$ Note that we can write $x=x^+-x^-$ and $|x|=x^++x^-$.
  \end{definition}
  \begin{theorem}
    A series $\sum a_n$ is absolutely convergent if and only if positive and negative terms series, $\sum {a_n}^+$ and $\sum {a_n}^-$, converge. In this case, $$\sum_{n=1}^\infty a_n=\sum_{n=1}^\infty {a_n}^+-\sum_{n=1}^\infty {a_n}^-$$
  \end{theorem}
  \begin{theorem}
    Let $\sum a_n$ be an absolutely convergent series. Then, for all bijection $\sigma:\NN \rightarrow\NN $, the rearranged series $\sum a_{\sigma(n)}$ is absolutely convergent and $\sum a_n=\sum a_{\sigma(n)}$.
  \end{theorem}
  \begin{theorem}[Riemann's theorem]
    Let $\sum a_n$ be a convergent series but not absolutely convergent. Then, $\forall\alpha\in\RR \cup\{\infty\}$, there exists a bijective map $\sigma:\NN \rightarrow\NN $ such that $\sum a_{\sigma(n)}$ converges and $\sum a_{\sigma(n)}=\alpha$.
  \end{theorem}
  \begin{theorem}
    A series $\sum a_n$ is absolutely convergent if and only if any rearranged series converges to the same value of $\sum a_n$.
  \end{theorem}
  \subsection{Sequences and series of functions}
  \subsubsection{Sequences of functions}
  \begin{definition}
    Let $D\subseteq\RR $. A set $$(f_n(x))=\{f_1(x),f_2(x),\ldots,f_n(x),\ldots\}$$ is a \emph{sequence of real functions} if $f_i:D\rightarrow\RR $ is a real-valued function. In this case we say the sequence $(f_n(x))$, or simply $(f_n)$, is well-defined on $D$.
  \end{definition}
  \begin{definition}
    Let $(f_n)$ be a sequence of functions defined on $D\subseteq\RR $ and $f:D\rightarrow\RR $. We say $(f_n)$ \emph{converges pointwise} to $f$ on $D$ if $\forall x\in D$, $\displaystyle\lim_{n\to\infty}f_n(x)=f(x)$
  \end{definition}
  \begin{definition}
    Let $(f_n)$ be a sequence of functions defined on $D\subseteq\RR $ and $f:D\rightarrow\RR $. We say $(f_n)$ \emph{converges uniformly} to $f$ on $D$ if $\forall\varepsilon>0$, $\exists n_0:|f_n(x)-f(x)|<\varepsilon$ $\forall n\geq n_0$ and $\forall x\in D$.
  \end{definition}
  \begin{lemma}
    Let $(f_n)$ be an uniform convergent sequence of functions defined on $D\subseteq\RR $ and let $f$ be a function such that $(f_n)$ converges pointwise to $f$. Then, $(f_n)$ converges uniformly $f$ on $D$.
  \end{lemma}
  \begin{lemma}\label{MA:sup-equivalence}
    Let $(f_n)$ be a sequence of functions defined on $D\subseteq\RR $. $(f_n)$ converges uniformly to $f$ on $D$ if and only if $\displaystyle \lim_{n\to\infty}\sup\left\{|f_n(x)-f(x)|:x\in D\right\}=0$.
  \end{lemma}
  \begin{corollary}
    A sequence of functions $(f_n)$ converges uniformly to $f$ on $D\subseteq\RR $ if and only if there is a sequence $(a_n)$, with $a_n\geq 0$ and $\displaystyle \lim_{n\to\infty} a_n=0$, and a number $\displaystyle n_0\in\NN $ such that $\sup\left\{|f_n(x)-f(x)|: x\in D\right\}\leq a_n$, $\forall n\geq n_0$.
  \end{corollary}
  \begin{theorem}[Cauchy's test]
    A sequence of functions $(f_n)$ converges uniformly to $f$ on $D\subseteq\RR $ if and only if $\forall\varepsilon>0$ $\displaystyle\exists n_0:\sup\left\{|f_n(x)-f_m(x)|:\\ x\in D\right\}< \varepsilon$ if $n,m\geq n_0$.
  \end{theorem}
  \begin{theorem}
    Let $(f_n)$ be a sequence of continuous functions defined on $D\subseteq\RR $. If $(f_n)$ converges uniformly to $f$ on $D$, then $f$ is continuous on $D$, that is, for any $x_0\in D$, it satisfies: $$\lim_{n\to\infty}\left(\lim_{x\to x_0} f_n(x)\right)=\lim_{x\to x_0} f(x)$$
  \end{theorem}
  \begin{theorem}
    Let $(f_n)$ be a sequence of functions defined on  $I=[a,b]\subseteq\RR $. If $(f_n)$ are Riemann-integrable on $I$ and $(f_n)$ converges uniformly to $f$ on $I$, then $f$ is Riemann-integrable on $I$ and $$\int_a^b\lim_{n\to\infty} f_n(x) \dd{x}=\lim_{n\to\infty} \int_a^bf_n(x) \dd{x}$$
  \end{theorem}
  \begin{theorem}
    Let $(f_n)$ be a sequence of functions defined on $I=(a,b)\subset\RR $. If $(f_n)$ are derivable on $I$, $(f_n'(x))$ converges uniformly on $I$ and $\exists x_0\in I$ such that $\displaystyle\lim_{n\to\infty}f_n(x_0)\in\RR $, then there is a function $f$ such that $(f_n)$ converges uniformly to $f$ on $I$, $f$ is derivable on $I$ and $(f_n'(x))$ converges uniformly to $f'$ on $I$.
  \end{theorem}
  \subsubsection{Series of functions}
  \begin{definition}
    Let $(f_n)$ be a sequence of functions defined on $D\subseteq\RR $. The expression $$\sum_{n=1}^\infty f_n(x)$$ is the \emph{series of functions} associated with $(f_n)$.
  \end{definition}
  \begin{definition}
    A series of functions $\sum f_n(x)$ defined on $D\subseteq\RR $ \emph{converges pointwise} on $D$ if the sequence of partials sums $$F_N(x)=\sum_{n=1}^Nf_n(x)$$ converges pointwise. If the pointwise limit of $(F_N)$ is $F(x)$, we say $F$ is the \emph{sum of the series in a pointwise sense}.
  \end{definition}
  \begin{definition}
    A series of functions $\sum f_n(x)$ defined on $D\subseteq\RR $ \emph{converges uniformly} on $D$ if the sequence of partials sums $$F_N(x)=\sum_{n=1}^Nf_n(x)$$ converges uniformly. If the uniform limit of $(F_N)$ is $F(x)$, we say $F$ is the \emph{sum of the series in an uniform sense}.
  \end{definition}
  \begin{theorem}[Cauchy's test]
    A series of functions $\sum f_n(x)$ defined on $D\subseteq\RR $ converges uniformly if and only if $\forall\varepsilon>0$ $\exists n_0$ such that $$\sup\left\{\left|\sum_{n=N}^Mf_n(x)\right|:x\in D\right\}< \varepsilon$$ if $M\geq N\geq n_0$.
  \end{theorem}
  \begin{corollary}
    If $\sum f_n(x)$ is an uniformly convergent series of functions on $D\subseteq\RR $, then $(f_n)$ converges uniformly to zero on $D$.
  \end{corollary}
  \begin{theorem}\label{MA:seriessumcontinuous}
    If $\sum f_n(x)$ is an uniformly convergent series of continuous functions on $D\subseteq\RR $, then its sum function is also continuous on $D$.
  \end{theorem}
  \begin{theorem}\label{MA:seriesuniformintegral}
    Let $(f_n)$ be a sequence of functions defined on $I=[a,b]\subseteq\RR $. If $(f_n)$ are Riemann-integrable on $I$ and $\sum f_n(x)$ converges uniformly on $I$, then $\sum f_n(x)$ is Riemann-integrable on $I$ and $$\int_a^b\sum_{n=1}^\infty f_n(x) \dd{x}=\sum_{n=1}^\infty \int_a^bf_n(x) \dd{x}$$
  \end{theorem}
  \begin{theorem}
    Let $(f_n)$ be a sequence of functions defined on $I=(a,b)\subset\RR $. If $(f_n)$ are derivable on $I$, $\sum f_n'(x)$ converges uniformly on $I$ and $\exists c\in I:\sum f_n(c)<\infty$, then $\sum f_n(x)$ converges uniformly on $I$, $\sum f_n(x)$ is derivable on $I$ and $$\left(\sum_{n=1}^\infty f_n(x)\right)'=\sum_{n=1}^\infty f_n'(x)$$
  \end{theorem}
  \begin{theorem}[Weierstra\ss\  M-test]\label{MA:Mweierstrass}
    Let $(f_n)$ be a sequence of functions defined on $D\subseteq\RR $ such that $|f_n(x)|\leq M_n$ $\forall x\in D$ and suppose that $\sum M_n$ is a convergent numeric series. Then, $\sum f_n(x)$ converges uniformly on $D$.
  \end{theorem}
  \begin{theorem}[Dirichlet's test]
    Let $(f_n),(g_n)$ be two sequences of functions defined on $D\subseteq\RR $. Suppose:
    \begin{enumerate}
      \item $\displaystyle\exists C>0: \sup\left\{\left|\sum_{n=1}^Nf_n(x)\right|:x\in D\right\}\leq C$, $\forall N$.
      \item $(g_n(x))$ is a monotone sequence for all $x\in D$ and $\displaystyle\lim_{n\to\infty}\sup\{|g_n(x)|:x\in D\}=0$.
    \end{enumerate}
    Then, $\sum f_n(x)g_n(x)$ converges uniformly on $D$.
  \end{theorem}
  \begin{theorem}[Abel's test]
    Let $(f_n),(g_n)$ be two sequences of functions defined on $D\subseteq\RR $. Suppose:
    \begin{enumerate}
      \item The series $\sum f_n(x)$ converges uniformly on $D$.
      \item $(g_n(x))$ is a monotone and bounded sequence for all $x\in D$.
    \end{enumerate}
    Then, $\sum f_n(x)g_n(x)$ converges uniformly on $D$.
  \end{theorem}
  \subsubsection{Power series}
  \begin{definition}
    Let $(a_n)$ be a sequence of real numbers and $x_0\in\RR $. A \emph{power series} centred on $x_0$ is a series of functions of the form $$\sum_{n=0}^\infty a_n{(x-x_0)}^n$$
  \end{definition}
  \begin{proposition}
    Let $\sum a_n{(x-x_0)}^n$ be a power series. Suppose there exists an $x_1\in\RR $ such that $\sum a_n(x_1-x_0)^n<\infty$. Then, $\sum a_n{(x-x_0)}^n$ converges uniformly on any closed interval $I\subset A=\{x\in\RR :|x-x_0|<|x_1-x_0|\}$.
  \end{proposition}
  \begin{theorem}\label{MA:radius}
    Let $\sum a_n{(x-x_0)}^n$ be a power series and consider $$R=\left(\limsup_{n\to\infty}\sqrt[n]{|a_n|}\right)^{-1}\in[0,\infty]$$
    Then:
    \begin{enumerate}
      \item If $|x-x_0|<R\implies\sum a_n{(x-x_0)}^n$ converges absolutely.
      \item If $0\leq r<R\implies\sum a_n{(x-x_0)}^n$ converges uniformly on $[x_0-r,x_0+r]$.
      \item If $|x-x_0|>R\implies\sum a_n{(x-x_0)}^n$ diverges.
    \end{enumerate}
    The number $R$ is called \emph{radius of convergence of the power series}.
  \end{theorem}
  \begin{theorem}[Abel's theorem]\label{MA:abelthm}
    Let $\sum a_nx^n$ be a power series\footnote{From now on we will suppose, for simplicity, $x_0=0$.} with radius of convergence $R$ satisfying $\sum a_nR^n<\infty$. Then, the series $\sum a_nx^n$ converges uniformly on $[0,R]$. In particular, if $f(x)=\sum a_nx^n$, $$\lim_{x\to R^-}f(x)=\sum_{n=0}^\infty a_nR^n$$
  \end{theorem}
  \begin{corollary}
    Let $f$ be the sum function of a power series $\sum a_nx^n$. Then, $f$ is continuous on the domain of convergence of the series.
  \end{corollary}
  \begin{corollary}
    If the series $\sum a_nx^n$ has radius of convergence $R\ne 0$ and $f$ is its sum function, then $f$ is Riemann-integrable on any closed subinterval on the domain of convergence of the series. In particular, for $|x|<R$, $$\int_0^xf(t)\dd{t}=\sum_{n=0}^\infty a_n\frac{x^{n+1}}{n+1}\footnote{The formula is also valid for $|x|=R$ if the series $\sum a_nR^n$ (or $\sum a_n(-R)^n$) is convergent.}$$
  \end{corollary}
  \begin{corollary}
    Let $f$ be the sum function of the power series $\sum a_nx^n$. Then, $f$ is derivable within the domain of convergence of the series and $$f'(x)=\sum_{n=0}^\infty na_nx^{n-1}$$
  \end{corollary}
  \begin{corollary}
    Any function $f$ defined as a sum of a power series $\sum a_nx^n$ is indefinitely derivable within the domain of convergence of the series and $$f^{(k)}(x)=\sum_{n=k}^\infty n(n-1)\cdots(n-k+1)a_nx^{n-k}$$
    for all $k\in\NN \cup\{0\}$. In particular $f^{(k)}(0)=k!a_k$.
  \end{corollary}
  \begin{definition}
    A function is \emph{analytic} if it can be expressed locally as a power series.
  \end{definition}
  \subsubsection{Stone-Weierstra\ss\  approximation theorem}
  \begin{definition}
    Let $f$ be a real-valued function. We say $f$ has \emph{compact support}\footnote{In general, the support of a function is the adherence of the set of points which are not mapped to zero.} if exists $M>0$ such that $f(x)=0$ for all $x\in\RR \setminus[-M,M]$.
  \end{definition}
  \begin{definition}
    Let $f,g$ be real-valued functions with compact support. We define the \emph{convolution} of $f$ with $g$ as $$(f*g)(x)=\int_\RR f(t)g(x-t)\dd{t}=\int_{\RR}f(x-t)g(t)\dd{t}$$
  \end{definition}
  \begin{remark}
    The idea behind the convolution is to ``blend'' one function with the other one. In Fourier Analysis, $g$ represents an input signal and $f$ a \emph{kernel function} for our purpose. This results in a new function that averages both functions.
  \end{remark}
  \begin{definition}
    We say that a sequence of functions $(\phi_\varepsilon)$ with compact support is an \emph{approximation of unity} if
    \begin{enumerate}
      \item $\phi_\varepsilon\geq 0$.
      \item $\displaystyle\int_\RR \phi_\varepsilon=1$.
      \item For all $\delta>0$, $\phi_\varepsilon(t)$ converges uniformly to zero when $\varepsilon\rightarrow 0$ if $|t|>\delta$.
    \end{enumerate}
  \end{definition}
  \begin{lemma}
    Let $f:\RR \rightarrow\RR $ be a continuous function with compact support. Let $(\phi_\varepsilon)$ be an approximation of unity. Then, $(f*\phi_\varepsilon)$ converges uniformly to $f$ on $\RR $ when $\varepsilon\rightarrow 0$.
  \end{lemma}
  \begin{theorem}[Weierstra\ss\  approximation theorem]\label{MA:weierstrasspolynomials}
    Let $f:[a,b]\rightarrow\RR $ be a continuous function. Then, there exists polynomials $p_n\in\RR [x]$ such that the sequence $(p_n)$ converge uniformly to $f$ on $[a,b]$.
  \end{theorem}
  \subsection{Improper integrals}
  \subsubsection{Locally integrable functions}
  \begin{definition}
    Let $f:[a,b)\rightarrow\RR $, with $b\in\RR \cup\{\infty\}$. We say $f$ is \emph{locally integrable} on $[a,b)$ if $f$ is Riemann-integrable on $[a,x]$ for all $a\leq x<b$.
  \end{definition}
  \begin{definition}
    Let $f:[a,b)\rightarrow\RR $ be a locally integrable function. If there exists $$\lim_{x\to b^-}\int_a^x f$$ and it's finite, we say that the \emph{improper integral} of $f$ on $[a,b)$, $\displaystyle\int_a^b f$, is \emph{convergent}.
  \end{definition}
  \begin{theorem}[Cauchy's test]
    Let $f:[a,b)\rightarrow\RR $ be a locally integrable function. The improper integral $\displaystyle\int_a^b f$ is convergent if and only if $\forall\varepsilon>0$ $\exists b_0$, $a<b_0<b$, such that $$\left|\int_{x_1}^{x_2} f\right|<\varepsilon$$ if $b_0<x_1<x_2<b$.
  \end{theorem}
  \subsubsection{Improper integrals of non-negative functions}
  \begin{theorem}
    Let $f:[a,b)\rightarrow\RR $ be a locally integrable non-negative function. A necessary and sufficient condition for $\displaystyle\int_a^b f$ to be convergent is that the function $$F(x)=\displaystyle\int_a^x f(t)\dd{t}$$ must be bounded for all $x<b$.
  \end{theorem}
  \begin{theorem}[Comparison test]
    Let $f,g:[a,b)\rightarrow[0,+\infty)$ be two locally integrable non-negative functions. Then:
    \begin{enumerate}
      \item If $\exists C>0$ such that $f(x)\leq Cg(x)$ $\forall x$ on a neighbourhood of $b$ and $\displaystyle\int_a^b g<\infty\implies\int_a^b f<\infty$.
      \item Suppose the limit $\displaystyle\ell=\lim_{x\to b}\frac{f(x)}{g(x)}$ exists.
            \begin{enumerate}
              \item If $\displaystyle\ell\in(0,\infty)\implies\int_a^b f<\infty\iff\int_a^b g<\infty$.
              \item If $\ell=0$ and $\displaystyle\int_a^b g<\infty\implies\int_a^b f<\infty$.
              \item If $\ell=\infty$ and $\displaystyle\int_a^b f<\infty\implies\int_a^b g<\infty$.
            \end{enumerate}
    \end{enumerate}
  \end{theorem}
  \begin{theorem}[Integral test]
    Let $f:[1,\infty)\rightarrow(0,\infty)$ be a locally integrable decreasing function. Then:
    $$\sum f(n)<\infty\iff\int_1^\infty f(x)\dd{x}<\infty\footnote{This is another way of formulating \mcref{MA:inttest}.}$$
  \end{theorem}
  \subsubsection{Absolute convergence of improper integrals}
  \begin{definition}
    Let $f:[a,b)\rightarrow(0,\infty)$ be a locally integrable function. We say $\displaystyle\int_a^b f$ \emph{converges absolutely} if  $\displaystyle\int_a^b |f|$ is convergent.
  \end{definition}
  \begin{theorem}[Dirichlet's test]
    Let $f,g:[a,b)\rightarrow\RR $ be two locally integrable functions Suppose:
    \begin{enumerate}
      \item $\displaystyle\exists C>0$ such that $\left|\int_a^xf(t)\dd{t}\right|\leq C$ for all $x\in[a,b)$.
      \item $g$ is monotone and $\displaystyle\lim_{x\to b}g(x)=0$.
    \end{enumerate}
    Then, $\displaystyle\int_a^b fg$ is convergent.
  \end{theorem}
  \begin{theorem}[Abel's test]
    Let $f,g:[a,b)\rightarrow\RR $ be two locally integrable functions. Suppose:
    \begin{enumerate}
      \item $\displaystyle\int_a^b f$ is convergent.
      \item $g$ is monotone and bounded.
    \end{enumerate}
    Then, $\displaystyle\int_a^b fg$ is convergent.
  \end{theorem}
  \subsubsection{Differentiation under integral sign}
  \begin{theorem}
    Let $f:[a,b]\times[c,d]\rightarrow\RR $ be a continuous function on $[a,b]\times[c,d]$. Consider the function $\displaystyle F(y)=\int_a^bf(x,y)\dd{x}$ defined on $[c,d]$. Then, $F$ is continuous, that is, if $c<y_0<d$,
    \begin{multline*}
      \lim_{y\to y_0}F(y)=\lim_{y\to y_0}\int_a^bf(x,y)\dd{x}=\int_a^b\lim_{y\to y_0}f(x,y)\dd{x}=\\=\int_a^bf(x,y_0)\dd{x}=F(y_0)
    \end{multline*}
  \end{theorem}
  \begin{theorem}
    Let $f:[a,b]\times[c,d]\rightarrow\RR $ be a Riemann-integrable function and let $\displaystyle F(y)=\int_a^bf(x,y)\dd{x}$. If $f$ is differentiable with respect to $y$ and $\partial f/\partial y$ is continuous on $[a,b]\times[c,d]$, then $F(y)$ is derivable on $(c,d)$ and its derivative is $$F'(y)=\int_a^b\frac{\partial f}{\partial y}(x,y)\dd{x}$$ for all $y\in(c,d)$.
  \end{theorem}
  \begin{theorem}
    Let $f:[a,b]\times[c,d]\rightarrow\RR $ be a continuous function on $[a,b]\times[c,d]$. Let $a,b:[c,d]\rightarrow\RR $ be to differentiable functions satisfying $a\leq a(y)\leq b(y)\leq b$ for every $y\in[c,d]$. Suppose that $\partial f/\partial y$ is continuous on $\{(x,y)\in\RR^2:a(y)\leq x\leq b(y),\; c\leq y\leq d\}$. Then, $\displaystyle F(y)=\int_{a(y)}^{b(y)}f(x,y)\dd{x}$ is derivable on $(c,d)$ and its derivative is
    \begin{multline*}
      F'(y)=b'(y)f(b(y),y)-a'(y)f(a(y),y)+\\+\int_{a(y)}^{b(y)}\frac{\partial f}{\partial y}(x,y)\dd{x}
    \end{multline*} for all $y\in(c,d)$.
  \end{theorem}
  \begin{theorem}
    Let $f:[a,b)\times[c,d]\rightarrow\RR $ be a continuous function on $[a,b)\times[c,d]$. We consider $\displaystyle F(y)=\int_a^bf(x,y)\dd{x}$. Suppose that:
    \begin{enumerate}
      \item $\displaystyle\frac{\partial f}{\partial y}$ is continuous on $[a,b)\times[c,d]$.
      \item Given $y_0\in[c,d]$, $\exists\delta>0$ such that the integral $$\int_a^b\sup\left\{\left|\frac{\partial f}{\partial y}(x,y)\right|:y\in(y_0-\delta,y_0+\delta)\right\}\dd{x}$$ exists and it's finite on $[a,b)$.
    \end{enumerate}
    Then, $F(y)$ is derivable at $y_0$ and $$F'(y_0)=\int_a^b\frac{\partial f}{\partial y}(x,y_0)\dd{x}$$
  \end{theorem}
  \begin{theorem}
    Let $f:[a,b)\times[c,d]\rightarrow\RR $ be a continuous function on $[a,b)\times[c,d]$. Let $a,b:[c,d]\rightarrow\RR $ be two differentiable functions satisfying $a\leq a(y)\leq b(y)\leq b$ for every $y\in[c,d]$. We consider $\displaystyle F(y)=\int_{a(y)}^{b(y)}f(x,y)\dd{x}$. Suppose that:
    \begin{enumerate}
      \item $\displaystyle\frac{\partial f}{\partial y}$ is continuous on $\{(x,y)\in\RR^2:a(y)\leq x\leq b(y),\; c\leq y\leq d\}$.
      \item Given $y_0\in[c,d]$, $\exists\delta>0$ such that the integral $$\int_{a(y)}^{b(y)}\sup\left\{\left|\frac{\partial f}{\partial y}(x,y)\right|:y\in(y_0-\delta,y_0+\delta)\right\}\dd{x}$$ exists and it's finite on $[a,b)$.
    \end{enumerate}
    Then, $F(y)$ is derivable at $y_0$ and
    \begin{multline*}
      F'(y_0)=b'(y_0)f(b(y_0),y_0)-a'(y_0)f(a(y_0),y_0)+\\+\int_{a(y_0)}^{b(y_0)}\frac{\partial f}{\partial y}(x,y_0)\dd{x}
    \end{multline*}
  \end{theorem}
  \subsubsection{Gamma function}
  \begin{definition}
    For $x>0$, \emph{Gamma function} is defined as $$\Gamma(x)=\int_0^\infty t^{x-1}\exp{-t}\dd{t}$$
  \end{definition}
  \begin{theorem}
    Gamma function is a generalization of the factorial. In fact, for $x>0$ we have $$\Gamma(x+1)=x\Gamma(x)$$ In particular, $\Gamma(n+1)=n!$ for all $n\in\NN $.
  \end{theorem}
  \begin{theorem}
    Gamma function satisfies: $$\lim_{x\to\infty}\frac{\Gamma(x+1)}{(x/e)^x\sqrt{2\pi x}}=1$$
  \end{theorem}
  \begin{corollary}[Stirling's formula]
    $$\lim_{n\to\infty}\frac{n!}{n^n\exp{-n}\sqrt{2\pi n}}=1$$
  \end{corollary}
  \subsection{Fourier series}\label{MA:fouriersection}
  \subsubsection{Periodic functions}
  \begin{definition}
    Let $f:\RR \rightarrow\CC $ be a function. We say that $f$ is \emph{$T$-periodic}, or is \emph{periodic with period $T$}, being $T>0$, if $f(x+T)=f(x)$ for all $x\in\RR $.
  \end{definition}
  \begin{remark}
    In general we take $T$ to be the least positive constant satisfying that property.
  \end{remark}
  \begin{lemma}
    Let $f:\RR \rightarrow\CC $ be a $T$-periodic function. Then, $f(x+T')=f(x)$ for all $x\in\RR $ if and only if $T'=kT$ for some $k\in\ZZ $.
  \end{lemma}
  \begin{sproof}
    \begin{itemizeiff}
      $$f(x+kT)=f(x+(k-1)T)=\cdots=f(x)$$
      \item Assume $T'=kT+\alpha$, $\alpha\in[0,T)$. Then:
      $$f(x)=f(x+T')=f(x+\alpha)\quad\forall x\in\RR$$
      which implies $\alpha=0$ because otherwise $f$ would be $\alpha$-periodic with $\alpha<T$.
    \end{itemizeiff}
  \end{sproof}
  \begin{proposition}\label{MA:invarianceperiodicint}
    Let $f:\RR \rightarrow\CC $ be a $T$-periodic function. Then: $$\int_a^{a+T}f(x)\dd{x}=\int_0^Tf(x)\dd{x}$$ where $a\in\RR $. In particular, $$\int_a^{a+kT}f(x)\dd{x}=k\int_0^Tf(x)\dd{x}$$
  \end{proposition}
  \begin{lemma}\label{MA:periodicbounded}
    Let $f:\RR \rightarrow\CC $ be a $T$-periodic continuous function. Then, $|f|$ is bounded.
  \end{lemma}
  \begin{sproof}
    Use \mnameref{RVF:weierstrass} on the interval $[0,T]$ and the periodicity of $f$.
  \end{sproof}
  \begin{proposition}
    Given a $T$-periodic function $f$, there are no power series uniformly convergent to $f$ on $\RR$.
  \end{proposition}
  \begin{proof}
    Suppose $\sum a_nx^n$ converges uniformly to $f$. By \mcref{MA:seriessumcontinuous}, $f$ is continuous and by \mcref{MA:periodicbounded}, $\abs{f}$ is bounded. Therefore $$\sup_{x\in\RR}\left|\sum_{n=1}^{N}a_nx^n-f(x)\right|$$
    cannot be arbitrarily small as $N\to\infty$ because $\sum_{n=1}^{N}a_nx^n$ is a polynomial, and therefore, unbounded.
  \end{proof}
  \subsubsection{Orthogonal systems}
  \begin{definition}
    Let $f:\RR \rightarrow\CC $ be a function. We say that $f\in L^p(I)$, $p\geq1$, if: $$\norm{f}_p:=\left(\int_I{|f(t)|}^p\dd{t}\right)^{1/p}<\infty$$
  \end{definition}
  \begin{definition}
    Let $f,g:[a,b]\rightarrow\CC $ be Riemann-integrable functions. We define the \emph{inner product} of $f$ and $g$ as $$\langle f,g\rangle:=\int_a^bf(x)\overline{g(x)}\dd{x}$$ where $\overline{g}$ denotes the complex conjugate of $g$. We define the \emph{norm} of $f$ as: $$\norm{f}:={\langle f,f\rangle}^{1/2}=\left(\int_a^b{|f(x)|}^2\dd{x}\right)^{1/2}=\norm{f}_2$$ And the \emph{distance} between $f$ and $g$ as: $$d(f,g):=\|f-g\|$$
  \end{definition}
  \begin{proposition}
    Let $f,g:[a,b]\rightarrow\CC $ be Riemann-integrable functions and let $\alpha\in\CC $. Then, we have:
    \begin{enumerate}
      \item $\langle f,f\rangle\geq 0$.
      \item $\langle f+h,g\rangle=\langle f,g\rangle+\langle h,g\rangle$ and $\langle f,g+h\rangle=\langle f,g\rangle+\langle f,h\rangle$.
            \item\label{MA:orto3} $\langle f,g\rangle=\overline{\langle g,f\rangle}$.
      \item $\langle \alpha f,g\rangle=\alpha\langle f,g\rangle$ and $\langle f,\alpha g\rangle=\overline{\alpha}\langle f,g\rangle$.
    \end{enumerate}
  \end{proposition}
  \begin{sproof}
    They follow from the linearity of the integral. For \mcref{MA:orto3}, write $f=\Re f+\ii \Im f$ and $g=\Re g+\ii \Im g$ and expand the products of both sides of the equation.
  \end{sproof}
  \begin{theorem}[Cauchy-Schwarz inequality]\label{MA:cauchyschwarz}
    Let $f,g:[a,b]\rightarrow\CC $ be Riemann-integrable functions. Then: $$\abs{\langle f,g\rangle}\leq\norm{f}\cdot\|g\|$$ which can be written as: $$\int_a^bf\overline{g}\leq\left(\int_a^b{|f|}^2\right)^{1/2}\left(\int_a^b{|g|}^2\right)^{1/2}$$
  \end{theorem}
  \begin{proof}
    First suppose that $\norm{f}=\norm{g}=1$. Then:
    $$\abs{\langle f,g\rangle}\leq\int_a^b\abs{fg}\leq\int_a^b\frac{\abs{f}^2+\abs{g}^2}{2}=1$$
    because $\abs{ab}\leq \frac{a^2+b^2}{2}$ $\forall a,b\in\RR$.

    For the general case, note that $\frac{f}{\norm{f}}$ and $\frac{g}{\norm{g}}$ have norm 1 and so: $$\abs{\left\langle \frac{f}{\norm{f}},\frac{g}{\norm{g}}\right\rangle}\leq 1\implies\abs{\langle f,g\rangle}\leq\norm{f}\norm{g}$$
  \end{proof}
  \begin{theorem}[Minkowski inequality]
    Let $f,g:[a,b]\rightarrow\CC $ be Riemann-integrable functions. Then: $$\| f+g\|\leq\norm{f}+\|g\|$$
  \end{theorem}
  \begin{proof}
    Using \mnameref{MA:cauchyschwarz} we have:
    \begin{align*}
      \norm{f+g}^2 & =\norm{f}^2+\norm{g}^2+2\langle f,g\rangle   \\
                   & \leq\norm{f}^2+\norm{g}^2+\norm{f}\cdot\|g\| \\
                   & ={\left(\norm{f}+\norm{g}\right)}^2
    \end{align*}
  \end{proof}
  \begin{definition}
    Let $f,g:[a,b]\rightarrow\CC $ be Riemann-integrable functions with $f\ne g$. We say $f$ and $g$ are \emph{orthogonal} if $\langle f,g\rangle=0$. We say $f$ and $g$ are \emph{orthonormal} if they are orthogonal and $\norm{f}=\|g\|=1$.
  \end{definition}
  \begin{definition}
    Let $S=\{\phi_0,\phi_1,\ldots\}$ be a collection of Riemann-integrable functions on $[a,b]$. We say $S$ is an \emph{orthonormal system} if $\|\phi_n\|=1$ $\forall n$ and $\langle\phi_n,\phi_m\rangle=0$ $\forall n\ne m$.
  \end{definition}
  \begin{proposition}
    Let $T>0$ and:
    \begin{gather*}
      S_1=\left\{\frac{1}{\sqrt{T}}\exp{\frac{2\pi\ii nx}{T}}:n\in\ZZ \right\}\\ S_2=\left\{\frac{1}{\sqrt{T}},\frac{\cos\left(\frac{2\pi nx}{T}\right)}{\sqrt{T/2}},\frac{\sin\left(\frac{2\pi mx}{T}\right)}{\sqrt{T/2}}:n,m\in\NN \right\}
    \end{gather*} Then, $S_1$ and $S_2$ orthonormal systems on $[-T/2,T/2]$.
  \end{proposition}
  \begin{definition}
    A collection of functions $S=\{\phi_0,\phi_1,\\\ldots,\phi_n\}$ is \emph{linearly dependent} on $[a,b]$ if there exist $c_0,c_1,\ldots,c_n\in\RR $ not all zero, such that $$c_0\phi_0+c_1\phi_1+\cdots+c_n\phi_n=0,\quad\forall x\in[a,b]$$ Otherwise we say $S$ is \emph{linearly independent}. If the collection $S$ has an infinity number of functions, we say $S$ is linearly independent on $[a,b]$ if any finite subset of $S$ is linearly independent on $[a,b]$.
  \end{definition}
  \begin{theorem}
    Let $S=\{\phi_0,\phi_1,\ldots\}$ be an orthonormal system on $[a,b]$. Suppose that $\sum c_n\phi_n(x)$ converges uniformly to a function $f$ on $[a,b]$. Then, $f$ is Riemann-integrable on $[a,b]$ and, moreover: $$c_n=\langle f,\phi_n\rangle=\int_a^bf(x)\overline{\phi_n(x)}\dd{x},\quad\forall n\geq 0$$
  \end{theorem}
  \begin{proof}
    Using \mcref{MA:seriesuniformintegral} we have that $f$ is Riemann-integrable and that $\forall m\in\NN$:
    $$\langle f,\phi_m\rangle=\sum_{n=0}^\infty c_n\langle \phi_n,\phi_m\rangle = c_n$$
    by the orthonormality of $S$.
  \end{proof}
  \subsubsection{Fourier coefficients and Fourier series}
  \begin{definition}
    Let $\displaystyle S=\left\{\frac{1}{\sqrt{T}}\exp{\frac{2\pi\ii nx}{T}},n\in\ZZ \right\}$ be an orthonormal system on $[-T/2,T/2]$ and let $f\in L^1([-T/2,T/2])$\footnote{Saying that $f\in L^1([-T/2,T/2])$ is equivalent to say that $f$ is integrable on $[-T/2,T/2]$.} be a $T$-periodic function\footnote{From now on, we will work only with functions defined on $[-T/2,T/2]$ and extended periodically on $\RR $.}. We define the \emph{$n$-th Fourier coefficient} of $f$ as $$\widehat{f}(n)=\frac{1}{T}\left\langle f,\exp{\frac{2\pi\ii nx}{T}}\right\rangle=\frac{1}{T}\int_{-T/2}^{T/2}f(x)\exp{-\frac{2\pi\ii nx}{T}}\dd{x}$$ for all $n\in\ZZ $.
  \end{definition}
  \begin{proposition}
    Let $f,g\in L^1([-T/2,T/2])$. The following properties are satisfied:
    \begin{enumerate}
      \item For all $\lambda,\mu\in\CC $:
            $$\widehat{(\lambda f+\mu g)}(n)=\lambda\widehat{f}(n)+\mu\widehat{g}(n)$$
            \item\label{MA:fouriercoeffs2} Let $\tau\in\RR $. We define $f_\tau(x)=f(x-\tau)$. Then: $$\widehat{f_\tau}(n)=\exp{-\frac{2\pi\ii n\tau}{T}}\widehat{f}(n)$$
      \item If $f$ is even, then $\widehat{f}(n)=\widehat{f}(-n)$, $\forall n\in\ZZ $.\newline If $f$ is odd, then $\widehat{f}(n)=-\widehat{f}(-n)$, $\forall n\in\ZZ $.
      \item If $f\in \mathcal{C}^k$ such that $f^{(r)}(-T/2)=f^{(r)}(T/2)$ $\forall r=0,\ldots,k-1$, then $$\widehat{f^{(k)}}(n)=\left(\frac{2\pi\ii n}{T}\right)^k\widehat{f}(n)$$
      \item $\widehat{(f*g)}(n)=\widehat{f}(n)\widehat{g}(n)$.
    \end{enumerate}
  \end{proposition}
  \begin{proof}
    \begin{enumerate}
      \item It follows from the linearity of the integral.
      \item \begin{align*}
              \widehat{f_\tau}(n) & =\frac{1}{T}\int_{-T/2}^{T/2}f(x-\tau)\exp{-\frac{2\pi\ii nx}{T}}\dd{x}             \\
                                  & =\frac{1}{T}\int_{-T/2-\tau}^{T/2-\tau}f(u)\exp{-\frac{2\pi\ii n(u+\tau)}{T}}\dd{x} \\
                                  & =\exp{-\frac{2\pi\ii n\tau}{T}}\widehat{f}(n)
            \end{align*}
            where we have done the change of variable $u=x-\tau$ and we have used \mcref{MA:invarianceperiodicint}.
      \item Make the change of variable $u=-x$.
      \item We will use induction. The case $k=0$ is clear. For the other ones:
            \begin{align*}
              \widehat{f^{(k)}}(n) & =\frac{1}{T}\int_{-T/2}^{T/2} f^{(k)}(x)\exp{-\frac{2\pi\ii nx}{T}}\dd{x}           \\
                                   & =\frac{2\pi\ii n}{T}\int_{-T/2}^{T/2} f^{(k-1)}(x)\exp{-\frac{2\pi\ii nx}{T}}\dd{x} \\
                                   & =\left(\frac{2\pi\ii n}{T}\right)\widehat{f^{(k-1)}}(n)                             \\
                                   & ={\left(\frac{2\pi\ii n}{T}\right)}^k\widehat{f}(n)
            \end{align*}
            where we have used integration by parts.
      \item Using \mnameref{FSV:fubini} we have that:
            \begin{align*}
              \widehat{(f*g)}(n) & =\int_{-T/2}^{T/2}\int_{-T/2}^{T/2} f(t)g(x-t)\exp{-\frac{2\pi\ii n x}{T}}\dd{t}\dd{x}              \\
                                 & =\int_{-T/2}^{T/2}f(t)\left(\int_{-T/2}^{T/2} g(x-t)\exp{-\frac{2\pi\ii n x}{T}}\dd{x}\right)\dd{t} \\
                                 & =\int_{-T/2}^{T/2}f(t)\exp{-\frac{2\pi\ii nt}{T}}\widehat{g}(n)\dd{t}                               \\
                                 & =\widehat{f}(n)\widehat{g}(n)
            \end{align*}
            where we have used \mcref{MA:fouriercoeffs2}.
    \end{enumerate}
  \end{proof}
  \begin{definition}
    Let $f\in L^1([-T/2,T/2])$. We define the \emph{Fourier series} of $f$ as: $$\displaystyle Sf(x)=\sum_{n\in\ZZ }\widehat{f}(n)\exp{\frac{2\pi\ii nx}{T}}$$
  \end{definition}
  \begin{definition}
    Let $f\in L^1([-T/2,T/2])$ and $Sf$ be the Fourier series of $f$. We define \emph{$N$-th partial sum of $Sf$} as: $$S_Nf(x)=\sum_{n=-N}^N\widehat{f}(n)\exp{\frac{2\pi\ii nx}{T}}$$
  \end{definition}
  \begin{proposition}
    Let $f\in L^1([-T/2,T/2])$. Then: $$Sf(x)=\frac{a_0}{2}+\sum_{n=1}^\infty a_n\cos\left(\frac{2\pi nx}{T}\right)+b_n\sin\left(\frac{2\pi nx}{T}\right)$$ where \begin{gather*}
      a_n=\frac{2}{T}\int_{-T/2}^{T/2}f(x)\cos\left(\frac{2\pi nx}{T}\right)\dd{x},\\ b_n=\frac{2}{T}\int_{-T/2}^{T/2}f(x)\sin\left(\frac{2\pi nx}{T}\right)\dd{x},
    \end{gather*} for $n\geq 0$\footnote{The relation between $a_n,b_n$ and $\widehat{f}(n)$ is given by: $$a_n=\widehat{f}(n)+\widehat{f}(-n)\quad\text{and}\quad b_n=\ii\left[\widehat{f}(n)-\widehat{f}(-n)\right],\quad\forall n\in\NN \cup\{0\}$$}. In particular, if $f$ is even we have: $$Sf(x)=\frac{a_0}{2}+\sum_{n=1}^\infty a_n\cos\left(\frac{2\pi nx}{T}\right)$$ and if $f$ is odd we have: $$Sf(x)=\sum_{n=1}^\infty b_n\sin\left(\frac{2\pi nx}{T}\right)$$
  \end{proposition}
  \begin{sproof}
    Remember that: $$\exp{\frac{2\pi\ii nx}{T}}=\cos\left(\frac{2\pi nx}{T}\right)+\ii\sin\left(\frac{2\pi nx}{T}\right)$$
  \end{sproof}
  \begin{definition}
    Let $f:(0,L)\rightarrow\CC $ be a function. We define the \emph{even extension} of $f$ as $$f_\mathrm{e}(x)=
      \begin{cases}
        f(x)  & \text{if }x\in(0,L)  \\
        f(-x) & \text{if }x\in(-L,0)
      \end{cases}$$ Analogously, we define the \emph{odd extension} of $f$ as $$f_\mathrm{o}(x)=
      \begin{cases}
        f(x)   & \text{if }x\in(0,L)  \\
        -f(-x) & \text{if }x\in(-L,0)
      \end{cases}$$
  \end{definition}
  \begin{proposition}
    Let $f\in L^1([0,T/2])$. If we make the even extension of $f$\footnote{For simplicity, when we have a function $f$ and make its even or odd extension, we will still call its even or odd extension $f$ instead of $\Tilde{f}$ or $\hat{f}$.}, then $$Sf(x)=\frac{a_0}{2}+\sum_{n=1}^\infty a_n\cos\left(\frac{2\pi nx}{T}\right)$$ where $\displaystyle a_n=\frac{4}{T}\int_0^{T/2}f(x)\cos\left(\frac{2\pi nx}{T}\right)\dd{x}$ for $n\geq 0$. If we make the odd extension of $f$, then $$Sf(x)=\sum_{n=1}^\infty b_n\sin\left(\frac{2\pi nx}{T}\right)$$ where $\displaystyle b_n=\frac{4}{T}\int_0^{T/2}f(x)\sin\left(\frac{2\pi nx}{T}\right)\dd{x}$ for $n\geq 1$.
  \end{proposition}
  \subsubsection{Pointwise convergence}
  \begin{definition}[Dirichlet kernel]
    We define the \emph{Dirichlet kernel} of order $N\in\NN$ as: $$D_N(t)=\sum_{n=-N}^N\exp{\frac{2\pi\ii nt}{T}}$$
  \end{definition}
  \begin{lemma}\label{MA:dirichletkernelchar}
    Let $N\in\NN$. Then, $\forall t\in(0,T)$ we have:
    $$D_N(t)=\frac{\sin\left(\frac{(2N+1)\pi t}{T}\right)}{\sin\left(\frac{\pi t}{T}\right)}$$
  \end{lemma}
  \begin{proof}
    Using the geometric sum formula we have:
    \begin{align*}
      D_N(t) & =\frac{\exp{-\frac{2\pi\ii Nt}{T}}-\exp{\frac{2\pi\ii (N+1)t}{T}}}{1-\exp{\frac{2\pi\ii t}{T}}}                            \\
             & =\frac{\exp{-\frac{\pi\ii (2N+1)t}{T}}-\exp{\frac{\pi\ii (2N+1)t}{T}}}{\exp{-\frac{\pi\ii t}{T}}-\exp{\frac{\pi\ii t}{T}}} \\
             & =\frac{\sin\left(\frac{(2N+1)\pi t}{T}\right)}{\sin\left(\frac{\pi t}{T}\right)}
    \end{align*}
  \end{proof}
  \begin{proposition}
    The Dirichlet kernel has the following properties:
    \begin{enumerate}
      \item $D_N$ is a $T$-periodic and even function.
      \item $\displaystyle\frac{1}{T}\int_{-T/2}^{T/2}D_N(t)\dd{t}=1,\;\forall N\in\NN$.
    \end{enumerate}
  \end{proposition}
  \begin{sproof}
    \begin{enumerate}
      \item Use the characterization of \mcref{MA:dirichletkernelchar}.
      \item Note that if $n\ne 0$, $\displaystyle\int_{-T/2}^{T/2} \exp{\frac{2\pi\ii nt}{T}}\dd{t}=0$
    \end{enumerate}
  \end{sproof}
  \begin{proposition}\label{MA:dirichletconvolution}
    Let $f\in L^1([-T/2,T/2])$. Then:
    \begin{align*}
      S_Nf(x) & =(f*D_N)(x)                              \\
              & =\int_{-T/2}^{T/2}f(x-t)D_N(t)\dd{t}     \\
              & =\int_0^{T/2}[f(x+t)+f(x-t)]D_N(t)\dd{t}
    \end{align*}
  \end{proposition}
  \begin{sproof}
    The first equality follows from expandind the Fourier coefficients inside $S_Nf$. The second one, making the change of variables $u=x-t$ and noting that both integrant functions are $T$-periodic. For the last one, make the change of variables $u=-t$ and use that $D_N(t)$ is even.
  \end{sproof}
  \begin{lemma}[Riemann-Lebesgue lemma]\label{MA:riemannlebesgue}
    Let $f\in L^1([-T/2,T/2])$ and $\lambda\in\RR $. Then: $$\lim_{\lambda\to\infty}\int_{-T/2}^{T/2}f(t)\sin(\lambda t)\dd{t}=\lim_{\lambda\to\infty}\int_{-T/2}^{T/2}f(t)\cos(\lambda t)\dd{t}=0$$ In particular, $\displaystyle\lim_{|n|\to\infty}\widehat{f}(n)=0$.
  \end{lemma}
  \begin{proof}
    We first proof the statement for indicators functions $f(x)=\indi{[a,b]}(x)$, $[a,b]\subseteq [-T/2,T/2]$. We have that:
    $$\lim_{\lambda\to\infty}\int_{-T/2}^{T/2}\indi{[a,b]}(t)\sin(\lambda t)\dd{t}=\lim_{\lambda\to\infty}\frac{\cos(\lambda a)-\cos(\lambda b)}{\lambda}=0$$
    From the linearity of the integral the statement remains true for $f$ being linear combination of indicator functions. Finally note that taking an upper sum $g_\varepsilon$ of $f$ (see \mcref{RVF:upperlower}) such that $\int_{-T/2}^{T/2}\abs{f-g_\varepsilon}<\varepsilon$, we have:
    \begin{multline*}
      \abs{\int_{-T/2}^{T/2}f(t)\sin(\lambda t)\dd{t}}\leq\int_{-T/2}^{T/2}\abs{f(t)-g_\varepsilon(t)}\dd{t}+\\+\abs{\int_{-T/2}^{T/2}g_\varepsilon(t)\sin(\lambda t)\dd{t}}\overset{\substack{\lambda\to\infty\\\varepsilon\to 0}}{\longrightarrow}0
    \end{multline*}
    The same proof applies for the $\cos(\lambda t)$.
  \end{proof}
  \begin{theorem}[Dini's theorem]\label{MA:dini}
    Let \\$f\in L^1([-T/2,T/2])$, $x_0\in (-T/2,T/2)$ and $\ell\in\RR $ such that $$\int_0^\delta\frac{\abs{f(x_0+t)+f(x_0-t)-2\ell}}{t}\dd{t}<\infty$$ for some $\delta>0$. Then, $\displaystyle\lim_{N\to\infty}S_Nf(x_0)=\ell$.
  \end{theorem}
  \begin{proof}
    Note that $-\ell=-2\ell\int_{0}^{T/2}D_N(t)\dd{t}$. So:
    \begin{align*}
      S_Nf(x_0)-\ell & =\int_0^{T/2}[f(x_0+t)+f(x_0-t) - 2\ell]D_N(t)\dd{t} \\
      \begin{split}
        &=\int_{0}^{T/2}\frac{f(x_0+t)+f(x_0-t) - 2\ell}{t}\frac{t}{\sin\left(\frac{\pi t}{T}\right)}\cdot\\
        &\hspace{3cm}\cdot\sin\left(\frac{(2N+1)\pi t}{T}\right)\dd{t}
      \end{split}
    \end{align*}
    Since the first terms form an integrable function, we can use now the \mnameref{MA:riemannlebesgue}.
  \end{proof}
  \begin{corollary}
    Let $f\in L^1([-T/2,T/2])$ be a function left and right differentiable at $x_0$, that is, there exists the following limits
    \begin{gather*}
      f'({x_0}^+)=\lim_{t\to0^+}\frac{f(x_0+t)-f({x_0}^+)}{t}\\
      f'({x_0}^-)=\lim_{t\to0^-}\frac{f(x_0+t)-f({x_0}^-)}{t}
    \end{gather*}(supposing the existence of left- and right-sided limits). Then: $$\lim_{N\to\infty}S_Nf(x_0)=\frac{f({x_0}^+)+f({x_0}^-)}{2}$$
  \end{corollary}
  \begin{sproof}
    Use \mnameref{MA:dini} with $\ell=\frac{f({x_0}^+)+f({x_0}^-)}{2}$.
  \end{sproof}
  \begin{theorem}[Lipschitz's theorem]
    Let $f\in L^1([-T/2,T/2])$ such that at a point $x_0\in (-T/2,T/2)$ it satisfies $$|f(x_0+t)-f(x_0)|\leq k|t|$$ for some constant $k\in\RR $ and for $|t|<\delta$. Then, $\displaystyle\lim_{N\to\infty}S_Nf(x_0)=f(x_0)$.
  \end{theorem}
  \begin{sproof}
    Note that
    $$S_Nf(x_0)-f(x_0) =\int_{-T/2}^{T/2}[f(x_0+t) - f(x_0)]D_N(t)\dd{t}$$
    and proceed as in the proof of \mnameref{MA:dini}.
  \end{sproof}
  \begin{remark}
    Note that only the continuity is not sufficient to ensure the pointwise convergence of $Sf$ to $f$.
  \end{remark}
  \subsubsection{Uniform convergence}
  \begin{definition}
    Let $\sum a_n$ be a series with partial sums $S_k$. The series $\sum a_n$ is called \emph{Cesàro summable} with sum $S$ if $$\lim_{N\to\infty}\frac{S_1+\cdots+S_N}{N}=S$$
  \end{definition}
  \begin{remark}
    Note that if $(a_n)\in\RR$ has limit $\ell$, by \mnameref{RVF:stolz} we have that $(a_n)$ is Cesàro summable and $\displaystyle\lim_{n\to\infty}\frac{a_1+\cdots+a_n}{n}=\ell$. The other inclusion is false though. For example by taking $a_n={(-1)}^n$.
  \end{remark}
  \begin{definition}[Fejér kernel]\label{MA:fejerdef}
    We define the \emph{Fejér kernel} of order $N$ as $$F_N(t)=\frac{1}{N+1}\sum_{k=0}^ND_k(t)$$ being $D_k(t)$ the Dirichlet kernel of order $k$, $0\leq k\leq N$.
  \end{definition}
  \begin{lemma}\label{MA:fejerkernelchar}
    Let $N\in\NN$. Then, $\forall t\in(0,T)$ we have:
    $$F_N(t)=\frac{1}{N+1}\frac{\sin^2\left(\frac{(N+1)\pi t}{T}\right)}{\sin^2\left(\frac{\pi t}{T}\right)}$$
  \end{lemma}
  \begin{proof}
    Multiplying the expression of \mcref{MA:dirichletkernelchar} by $\sin(\frac{\pi t}{T})$ and using the trigonometry identity $\sin(x)\sin(y)=\frac{\cos(x-y)-\cos(x+y)}{2}$ we have that $F_N$ is a telescopic sum that simplifies to:
    $$F_N(t)=\frac{1-\cos(\frac{2(N+1)\pi t}{T})}{2(N+1)\sin^2\left(\frac{\pi t}{T}\right)}=\frac{1}{N+1}\frac{\sin^2\left(\frac{(N+1)\pi t}{T}\right)}{\sin^2\left(\frac{\pi t}{T}\right)}$$
  \end{proof}
  \begin{proposition}\label{MA:fejerprop}
    The Fejér kernel has the following properties:
    \begin{enumerate}
      \item $F_N$ is a $T$-periodic, even and non-negative function.
      \item $\displaystyle\frac{1}{T}\int_{-T/2}^{T/2}F_N(t)\dd{t}=1\quad\forall N$.
      \item $\forall\delta>0$, $\displaystyle\lim_{N\to\infty}\sup\{\abs{F_N(t)}:\delta\leq\abs{t}\leq T/2\}=0$
    \end{enumerate}
  \end{proposition}
  \begin{proof}
    The first two properties are consequence of the definition of Fejér kernel and the reexpression of \mcref{MA:fejerkernelchar}. For the last one, note that:
    $$|F_N(t)|\leq \frac{1}{N+1}\frac{1}{\sin^2\left(\frac{\pi \delta}{T}\right)}\overset{N\to\infty}{\longrightarrow} 0$$
  \end{proof}
  \begin{definition}
    Let $f\in L^1([-T/2,T/2])$. We define the \emph{Fejér means} $\sigma_Nf$, for all $N\in\NN $, as:
    \begin{equation}\label{MA:fejermeans}
      \sigma_Nf(x)=\frac{S_0f(x)+\cdots+S_Nf(x)}{N+1}
    \end{equation}
  \end{definition}
  \begin{proposition}
    Let $f\in L^1([-T/2,T/2])$. Then:
    \begin{align*}
      \sigma_Nf(x) & =(f*F_N)(x)                              \\
                   & =\int_{-T/2}^{T/2}f(x-t)F_N(t)\dd{t}     \\
                   & =\int_0^{T/2}[f(x+t)+f(x-t)]F_N(t)\dd{t}
    \end{align*}
  \end{proposition}
  \begin{proof}
    Consequence of \mcref{MA:dirichletconvolution} and the linearity of the convolution.
  \end{proof}
  \begin{theorem}[Fejér's theorem]\label{MA:fejerthm0}
    Let \\$f\in L^1([-T/2,T/2])$ be a function having left- and right-sided limits at point $x_0$. Then: $$\lim_{N\to\infty}\sigma_Nf(x_0)=\frac{f({x_0}^+)+f({x_0}^-)}{2}$$ In particular, if $f$ is continuous at $x_0$, $\displaystyle\lim_{N\to\infty}\sigma_Nf(x_0)=f(x_0).$
  \end{theorem}
  \begin{sproof}
    Let $\delta>0$ be small enough. Then:
    \begin{multline*}
      \abs{\sigma_Nf(x_0)-\frac{f({x_0}^+)+f({x_0}^-)}{2}}=\\
      =\abs{\int_0^{T/2}[f(x+t)-f({x_0}^+)+f(x-t)-f({x_0}^-)]F_N(t)\dd{t}} \\
      \leq \int_0^{T/2}\abs{f(x+t)-f({x_0}^+)}F_N(t)\dd{t}+\\+\int_0^{T/2}\abs{f(x-t)-f({x_0}^-)}F_N(t)\dd{t}
    \end{multline*}
    In order to bound the to intervals, divide the interval $[0,T/2]=[0,\delta]\cup[\delta,T/2]$. The first part is bounded by the right- (or left-) sided limit at $x_0$, and the second one is due to the uniform convergence (see \mcref{MA:fejerprop}).
  \end{sproof}
  \begin{theorem}[Fejér's theorem]\label{MA:fejerthm}
    Let $f$ be a continuous function on $[-T/2,T/2]$. Then, $\sigma_Nf$ converges uniformly to $f$ on $[-T/2,T/2]$.
  \end{theorem}
  \begin{proof}
    Let $\delta>0$ be small enough. Then:
    \begin{align*}
      \abs{\sigma_Nf(x)-f(x)}= & \abs{\int_0^{T/2}[f(x-t)-f(x)]F_N(t)\dd{t}} \\
      \begin{split}
        & \leq \int_0^{\delta}\abs{f(x-t)-f(x)}F_N(t)\dd{t}+\\
        &\hspace{1cm}+\int_\delta^{T/2}\abs{f(x-t)-f(x)}F_N(t)\dd{t}
      \end{split}
    \end{align*}
    To bound the first integral use the uniform continuity of $f$ in $[0,\delta]$ and for the second one use \mcref{MA:fejerprop}.
  \end{proof}
  \begin{corollary}
    Let $f$ be a continuous function on $[-T/2,T/2]$. Then, there exists a sequence of trigonometric polynomials that converge uniformly to $f$ on $[-T/2,T/2]$. In fact: $$\sigma_Nf(x)=\sum_{k=-N}^N\left(1-\frac{|k|}{N+1}\right)\widehat{f}(k)\exp{\frac{2\pi\ii kx}{T}}$$
  \end{corollary}
  \begin{sproof}
    Observe that the term $\widehat{f}(k)\exp{\frac{2\pi \ii k x}{T}}$ appears $N+1 - \abs{k}$ times on the numerator of the fraction of \mcref{MA:fejermeans}. Hence
    $$\sigma_Nf(x)=\sum_{k=-N}^N\left(1-\frac{|k|}{N+1}\right)\widehat{f}(k)\exp{\frac{2\pi\ii kx}{T}}$$
    which is a trigonometric polynomial.
  \end{sproof}
  \begin{corollary}
    Let $f$ and $g$ be continuous functions on $[-T/2,T/2]$ such that $Sf(x)=Sg(x)$. Then, $f=g$.
  \end{corollary}
  \begin{proof}
    $h:=f-g$ satisfies that $\widehat{h}(n)=0$ $\forall n\in\ZZ$. So $\sigma_Nh=0$ $\forall N\in\NN$. \mnameref{MA:fejerthm} implies $h=0$.
  \end{proof}
  \subsubsection{Convergence in norm}
  \begin{definition}
    We say a sequence $(f_N)$ \emph{converge in norm $L^p$} to $f$ if $\displaystyle\lim_{N\to\infty}\|f_N-f\|_p=0$.
  \end{definition}
  \begin{theorem}
    Let $f\in L^2([-T/2,T/2])$. Then: $$\lim_{N\to\infty}\norm{\sigma_Nf-f}_2=0$$
  \end{theorem}
  \begin{sproof}
    Use \mnameref{FSV:fubini} and the scheme of the proof of \mnameref{MA:fejerthm}.
  \end{sproof}
  \begin{corollary}
    Let $f\in L^1([-T/2,T/2])$. Then: $$\lim_{N\to\infty}\norm{\sigma_Nf-f}_1=0$$
  \end{corollary}
  \begin{sproof}
    Note that $\norm{\sigma_Nf-f}_1\leq\norm{\sigma_Nf-f}_2$ by the \mnameref{MA:cauchyschwarz}.
  \end{sproof}
  \begin{corollary}
    Let $f,g\in L^1([-T/2,T/2])$ be functions such that $Sf(x)=Sg(x)$. Then, $\displaystyle\lim_{N\to\infty}\norm{g-f}_1=0$.
  \end{corollary}
  \begin{proof}
    $h:=f-g$ satisfies that $\widehat{h}(n)=0$ $\forall n\in\ZZ$. So $\sigma_Nh=0$ $\forall N\in\NN$. Thus:
    $$\lim_{N\to\infty}\norm{h}_1=\lim_{N\to\infty}\norm{\sigma_Nh-h}_1=0$$
  \end{proof}
  \begin{theorem}
    $S_Nf$ is the trigonometric polynomial of degree $N$ that best approximates $f$ in norm $L^2$.
  \end{theorem}
  \begin{proof}
    Let $P(x)=\sum_{n=-N}^{N}c_n\exp{\frac{2\pi\ii nx}{T}}$ be a trigonometric polynomial. Expanding the norm $\norm{f-P}^2$ we have:
    $$\norm{f-P}^2=\norm{f}^2+\norm{P}^2-2\Re\left(\int_{-T/2}^{T/2}f(x)\overline{P(x)}\dd{x}\right)$$
    One the one hand:
    \begin{align*}
      \norm{P}^2 & =\int_{-T/2}^{T/2}\left(\sum_{n=-N}^{N}c_n\exp{\frac{2\pi\ii nx}{T}}\right)\left(\sum_{m=-N}^{N}\overline{c_m}\exp{-\frac{2\pi\ii mx}{T}}\right)\dd{x} \\
                 & =T\sum_{n=-N}^{N}\abs{c_n}^2
    \end{align*} by the orthogonality of the system.
    On the other hand:
    $$\int_{-T/2}^{T/2}f(x)\overline{P(x)}\dd{x}=T\sum_{n=-N}^N\overline{c_n}\widehat{f}(n)$$
    Finally using that $\abs{z-w}^2 -\abs{z}^2=\abs{w}^2-2\Re(z\overline{w})$, $z,w\in\CC$, we have:
    \begin{equation}\label{MA:minimumL2}
      \norm{f-P}^2=\norm{f}^2+T\sum_{n=-N}^{N}\abs{c_n-\widehat{f}(n)}^2-T\sum_{n=-N}^{N}\abs{\widehat{f}(n)}^2
    \end{equation}
    which is minimum if $c_n=\widehat{f}(n)$ $\forall n=-N,\ldots,N$. That is, $P=S_Nf$.
  \end{proof}
  \begin{corollary}[Bessel's inequality]\label{MA:bessel}
    Let $f\in L^2([-T/2,T/2])$. Then:
    \begin{gather*}
      T\sum_{n=-N}^N\abs{\widehat{f}(n)}^2\leq{\norm{f}}^2\\
      \frac{T}{2}\left(\frac{\abs{a_0}^2}{2}+\sum_{n=1}^N\abs{a_n}^2+\abs{b_n}^2\right)\leq \norm{f}^2
    \end{gather*} for all $N\in\NN $.
  \end{corollary}
  \begin{sproof}
    If follows from \mcref{MA:minimumL2} with $P=S_Nf$.
  \end{sproof}
  \begin{corollary}\label{MA:corollaryPolyAprox}
    Let $f\in L^2([-T/2,T/2])$. Then, $\displaystyle\lim_{N\to\infty}\|S_Nf-f\|=0$.
  \end{corollary}
  \begin{theorem}[Parseval's identity]\label{MA:parseval}
    Let $f,g\in L^2([-T/2,T/2])$. Then: $$\langle f,g\rangle=T\sum_{n\in\ZZ }\widehat{f}(n)\overline{\widehat{g}(n)}$$
    In particular, if $f=g$:
    \begin{gather*}
      T\sum_{n\in\ZZ}\abs{\widehat{f}(n)}^2={\norm{f}}^2\\
      \frac{T}{2}\left(\frac{\abs{a_0}^2}{2}+\sum_{n=1}^\infty\abs{a_n}^2+\abs{b_n}^2\right)= \norm{f}^2
    \end{gather*}
  \end{theorem}
  \begin{proof}
    Note that $\displaystyle\dotp{f}{g}=\lim_{N\to\infty}\dotp{f}{S_Ng}$. Indeed:
    $$\abs{\dotp{f}{g}-\dotp{f}{S_Ng}}=\abs{\dotp{f}{g-S_Ng}}\leq \norm{f}\norm{g-S_Ng}$$
    where we have applied \mnameref{MA:cauchyschwarz}. Now use \mcref{MA:corollaryPolyAprox} to conclude that the right side of the equation tends to 0 as $N\to\infty$.
    Thus:
    \begin{align*}
      \dotp{f}{g} & =\lim_{N\to\infty}\dotp{f}{S_Ng}                                                              \\
                  & =\lim_{N\to\infty}\sum_{n=-N}^N\overline{\widehat{g}(n)}\dotp{f}{\exp{\frac{2\pi\ii n x}{T}}} \\
                  & =T\sum_{n\in\ZZ }\widehat{f}(n)\overline{\widehat{g}(n)}
    \end{align*}
  \end{proof}
  \subsubsection{Applications of Fourier series}
  \begin{theorem}[Wirtinger's inequality]\label{MA:wirtinger1}
    Let $f$ be a function such that $f(0)=f(T)$, $f'\in L^2([0,T])$ and $\displaystyle\int_0^Tf(t)\dd{t}=0$. Then: $$\int_0^T\abs{f(x)}^2\dd{x}\leq\frac{T^2}{4\pi^2}\int_0^T\abs{f'(x)}^2\dd{x}$$ And the inequality holds if and only if $$f(x)=A\cos\left(\frac{2\pi x}{T}\right)+B\sin\left(\frac{2\pi x}{T}\right)$$
  \end{theorem}
  \begin{proof}
    The continuity of $f$ and $f(0)=f(T)$, implies that $f\in L^2([0,T])$ and that $\widehat{f'}(n)=\frac{2\pi\ii n}{T}\widehat{f}(n)$. Moreover, note that $\widehat{f'}(0)=0$ and by hypothesis $\widehat{f}(0)=0$. By \mnameref{MA:parseval} we have:
    \begin{multline*}
      \int_0^T\abs{f(x)}^2\dd{x} = T\sum_{n\in\ZZ}\abs{\widehat{f}(n)}^2 \leq T\sum_{\substack{n\in\ZZ                  \\n\ne 0}}n^2\abs{\widehat{f}(n)}^2=\\
      = T\sum_{\substack{n\in\ZZ                     \\n\ne 0}}\frac{T^2}{4\pi^2}\abs{\widehat{f'}(n)}^2=\frac{T^2}{4\pi^2}\int_0^T\abs{f'(x)}^2\dd{x}
    \end{multline*}
    The equality holds if and only if $\abs{\widehat{f}(n)}^2(n^2 - 1)=0$ $\forall n\in\ZZ\setminus\{0\}$. That is, if and only if $$f(x)=c_{-1}\exp{\frac{-2\pi\ii x}{T}}+c_{1}\exp{\frac{2\pi\ii x}{T}}$$
  \end{proof}
  \begin{theorem}[Wirtinger's inequality]\label{MA:wirtinger2}
    Let $f\in \mathcal{C}^1([a,b])$ with $f(a)=f(b)=0$. Then: $$\int_a^b\abs{f(x)}^2\dd{x}\leq\frac{{(b-a)}^2}{\pi^2}\int_a^b\abs{f'(x)}^2\dd{x}$$
    with equality if and only if $$f(x)=A\sin\left(\frac{\pi}{b-a}(x-a)\right)$$
  \end{theorem}
  \begin{sproof}
    Let $\tilde{f}:[a,2b-a]\rightarrow\RR$ be the odd extension of $f$ centered at $b$:
    $$
      \tilde{f}(x)=\begin{cases}
        f(x)     & \text{if }x\in[a,b]    \\
        -f(2b-x) & \text{if }x\in[b,2b-a]
      \end{cases}
    $$
    Now use \mnameref{MA:wirtinger1} to the function $\tilde{f}$.
  \end{sproof}
  \begin{theorem}[Isoperimetric inequality]
    Let $c$ be a planar simple and closed curve of class $\mathcal{C}^1$ whose length is $\ell$. If $A_\mathrm{c}$ is the area enclosed by $c$, then $$A_\mathrm{c}\leq\frac{\ell^2}{4\pi}$$ and the equality holds if and only if $c$ is a circle.
  \end{theorem}
  \begin{proof}
    Let $\vf\gamma(s)=(x(s),y(s))$ be the arc-length parametrization of $c$ (see \mcref{DG:arclength}). Thus:
    $$s=\int_0^s\sqrt{{x'(t)}^2+{y'(t)}^2}\dd{t}$$
    which implies ${x'(s)}^2+{y'(s)}^2=1$, by the \mnameref{RVF:fundamentalthmCalculus}. Now, by \mcref{FSV:areaR2} we have that:
    \begin{multline*}
      A_\mathrm{c}=\int_{0}^{\ell}x(s)y'(s)\dd{s}\leq\frac{2\pi}{\ell}\int_{0}^{\ell}\frac{{x(s)}^2+\frac{\ell^2}{4\pi^2}{y'(s)}^2}{2}\dd{s}=\\
      =\frac{2\pi}{\ell}\int_{0}^{\ell}\left(\frac{\ell^2}{8\pi^2}+\frac{{x(s)}^2-\frac{\ell^2}{4\pi^2}{x'(s)}^2}{2}\right)\dd{s}\leq \frac{\ell^2}{4\pi}
    \end{multline*}
    by the \mnameref{MA:wirtinger1} (with a translation we can suppose $x(0)=x(\ell)=0$ and $\int_0^\ell x(s)\dd{s}=0$). Clearly if $c$ is a circle, the equality is hold. Moreover if we have equality, by \mnameref{MA:wirtinger1} we have that $x(s)=A\cos\left(\frac{2\pi s}{\ell}\right)+B\sin\left(\frac{2\pi s}{\ell}\right)$ and since $2ab=a^2+b^2$ implies $b=a$, we have that $\frac{\ell}{2\pi}y'(s)=x(s)$. So $y(s)=A\sin\left(\frac{2\pi s}{\ell}\right)-B\cos\left(\frac{2\pi s}{\ell}\right)+C$ and therefore $c$ is a circle.
  \end{proof}
\end{multicols}
\end{document}