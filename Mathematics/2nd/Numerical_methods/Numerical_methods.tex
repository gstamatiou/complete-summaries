\documentclass[class=article,10pt,crop=false]{standalone}
\usepackage{standalone}
\usepackage{preamble}

\begin{document}
\begin{multicols}{2}[\section{Numerical methods}]
\subsection{Errors}
\begin{theorem}
Let $b\in\mathbb{N}$, $b\geq 2$. Any real number $x\in\mathbb{R}$ can be represented of the form 
\begin{equation*}
    x=s\left(\sum_{i=1}^\infty\alpha_ib^{-i}\right)b^q,
\end{equation*} where $s\in\{-1,1\}$, $q\in\mathbb{Z}$ and $\alpha_i\in\{0,1,\ldots,b-1\}$. Moreover, this representation is unique if $\alpha_1\ne0$ and $\forall i_0\in\mathbb{N}$, $\exists i\geq i_0:\alpha_i\ne b-1$.
\end{theorem}
\begin{definition}[Floating-point representation]
Let $x$ be a real number. Then the floating-point representation of $x$ is $$x=s\left(\sum_{i=1}^\infty t\alpha_ib^{-i}\right)b^q.$$ Here $s$ is called the \textit{sign}; $\sum_{i=1}^\infty t\alpha_ib^{-i}$, the \textit{significant} or \textit{mantissa}, and $q$, the \textit{exponent}, limited to a prefixed range, that is, $q_\text{min}\leq q\leq q_\text{max}$. So, the floating-point representation of $x$ is $$x=smb^q,$$ where $m=(0,\delta_1\delta_2\cdots\delta_t)_b=\delta_1b^{-1}+\delta_2b^{-2}+\cdots+\delta_tb^{-t}$. Finally we say a floating-point number is \textit{normalized} if $\delta_1\ne0$.
\end{definition}
\begin{definition}
Given a value $x\in\mathbb{R}$ and its approximation $\Tilde{x}$, the \textit{absolute error} is $$\Delta x:=e_a(\Tilde{x},x)=|x-\Tilde{x}|.$$ If $x\ne 0$, the \textit{relative error} is $$\delta x:=e_r(\Tilde{x},x)=\frac{|x-\Tilde{x}|}{x}.$$ If $x$ is unknown, we will take $$e_r(\Tilde{x},x)\approx\frac{|x-\Tilde{x}|}{\Tilde{x}}.$$
\end{definition}
\begin{prop}[Propagation of absolute errors]
Let $f:\mathbb{R}^n\rightarrow\mathbb{R}$ be a function of class $\mathcal{C}^2$. If $\Delta x_j$ is the absolute error of the variable $x_j$ and $\Delta f(x)$ is the absolute error of the function $f$ evaluated at the point $x=(x_1,\ldots,x_n)$, we have $$|\Delta f(x)|\lesssim\sum_{j=1}^n\left|\frac{\partial f}{\partial x_j}(x)\right||\Delta x_j|.\footnote{The symbol $\lesssim$ means that we are omitting terms of order $\Delta x_j\Delta x_k$ and higher.}$$
\end{prop}
\begin{prop}[Propagation of relative errors]
Let $f:\mathbb{R}^n\rightarrow\mathbb{R}$ be a function of class $\mathcal{C}^2$. If $\delta x_j$ is the relative error of the variable $x_j$ and $\delta f(x)$ is the relative error of the function $f$ evaluated at the point $x=(x_1,\ldots,x_n)$, we have $$|\delta f(x)|\lesssim\sum_{j=1}^n\frac{\left|\frac{\partial f}{\partial x_j}(x)\right|\left|x\right|}{\left|f(x)\right|}|\delta x_j|.$$
\end{prop}
\end{multicols}
\end{document}
