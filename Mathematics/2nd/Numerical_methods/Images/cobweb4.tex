\documentclass{standalone}
\usepackage{../../../../preamble_tikz}

\begin{document}
\pgfkeys{/pgf/declare function={f(\x)=3.75*\x*(1-\x);}} % Here you can declare multivariable functions to like this:
% declare function={
% excitation(\t,\w) = sin(\t*\w);
% noise = rnd - 0.5;
% source(\t) = excitation(\t,20) + noise;
% filter(\t) = 1 - abs(sin(mod(\t, 90)));
% speech(\t) = 1 + source(\t)*filter(\t);
% }
\begin{tikzpicture}
  \begin{axis}[xmin=0,xmax=1.2,ymin=0,ymax=1.2,
      axis equal image]
    \addplot [thin,samples=200,smooth,blue] {f(x)};
    \addplot [thin,samples=200,smooth,black] {x};
    \pgfplotsextra{
      \def\x{0.125}
      \def\y{0}
      \draw [very thin,red] (axis cs:\x,\y) \foreach \i in {0,...,50}{
          \pgfextra{
            \pgfkeys{/pgf/fpu=true, /pgf/fpu/output format=fixed}
            \pgfmathparse{f(\x)}\xdef\y{\pgfmathresult}
          }
          -- (axis cs:\x,\y)
          \pgfextra{\xdef\x{\y}}
          -- (axis cs:{\x},{\y})
        };}
    \draw [thin,color_green2,dashed] (axis cs:0.733,0) -- node[left,pos=0.05]{$\alpha$} (axis cs:0.733,0.733);
    \legend{$g(x)$}
  \end{axis}
\end{tikzpicture}
\end{document}