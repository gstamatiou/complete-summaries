\documentclass{standalone}
\usepackage{../../../../preamble_tikz}

\begin{document}
\begin{tikzpicture}
  \begin{axis}[
      xmin = -1, xmax = 1, %
      ymin = -1.1, ymax = 1.2, %
      xtick distance = 0.25, %Is the distance between major ticks in the x-axis.
      ytick distance = 0.25, %Is the distance between major ticks in the y-axis.
      minor tick num = 1, %Is the number of ticks between major ticks.
      major grid style = {lightgray}, %Changes the color and stroke of the major grid.
      minor grid style = {lightgray!25}, %Changes the color and stroke of the minor grid.
      width = 10cm, %sets the width of the figure
      height = 7.5cm,  %sets the height of the figure
      xlabel = {}, %
      ylabel = {}, %
      legend cell align = {left}, %
      legend style={at={(axis cs:0,-0.5)},anchor=center} % position of the legend box and anchor is the point on the box to be fitted exactly at the point of cs:<>,<>. Options are anchor=center,south west,south east,north west,north east,north,south,west...
    ]
    \addplot[
      domain=-1:1, %Domain of the fucntion
      samples=200, %This parameter determines the number of point to be plotted for the function, while bigger the number better looks the function.
      smooth, %f we use this option, the compiler makes an interpolation between the point plotted to get a soft appearance for the function.
      very thick, %Stroke of the function. Options: ultra thin, very thin, thin, semithick, thick, very thick, ultra thick.
      red %Color of the function.
    ]{1/(1+25*x^2)};
    \addplot[
      domain=-1:1, %Domain of the fucntion
      samples=200, %This parameter determines the number of point to be plotted for the function, while bigger the number better looks the function.
      smooth, %f we use this option, the compiler makes an interpolation between the point plotted to get a soft appearance for the function.
      thick, %Stroke of the function. Options: ultra thin, very thin, thin, semithick, thick, very thick, ultra thick.
      color_green1 %Color of the function.
    ]{1.20192307692308*x^4 - 1.73076923076923*x^2 + 0.567307692307692};
    \addplot[
      domain=-1:1, %Domain of the fucntion
      samples=200, %This parameter determines the number of point to be plotted for the function, while bigger the number better looks the function.
      smooth, %f we use this option, the compiler makes an interpolation between the point plotted to get a soft appearance for the function.
      thick, %Stroke of the function. Options: ultra thin, very thin, thin, semithick, thick, very thick, ultra thick.
      color_green2 %Color of the function
    ]{21.6247747536392*x^8 - 44.9154580808920*x^6 + 30.7285300420963*x^4 - 8.26092332834775*x^2 + 0.861538151965819};
    \addplot[
      domain=-1:1, %Domain of the fucntion
      samples=200, %This parameter determines the number of point to be plotted for the function, while bigger the number better looks the function.
      smooth, %f we use this option, the compiler makes an interpolation between the point plotted to get a soft appearance for the function.
      thick, %Stroke of the function. Options: ultra thin, very thin, thin, semithick, thick, very thick, ultra thick.
      color_green3 %Color of the function
    ]{369.006629567662*x^12 - 5.68434188608080e-14*x^11 - 1008.23965248026*x^10 - 2.16004991671070e-12*x^9 + 1036.16040389924*x^8 - 1.42108547152020e-12*x^7 - 507.215504127014*x^6 + 1.42108547152020e-13*x^5 + 124.636816868829*x^4 - 2.66453525910038e-14*x^3 - 15.2674460668263*x^2 - 8.88178419700125e-16*x + 0.957213876827098};
    \legend{$f(x)$,$p_5(x)$,$p_9(x)$,$p_{13}(x)$}
  \end{axis}
\end{tikzpicture}
\end{document}