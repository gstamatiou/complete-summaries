\documentclass[../../../main.tex]{subfiles}


\begin{document}
\renewcommand{\col}{\alg}
\begin{multicols}{2}[\section{Discrete mathematics}]
  \subsection{Generating functions and recurrence relations}
  \subsubsection{Generating functions}
  \begin{definition}
    Let $(a_n)$ be a sequence of real numbers. We define its \emph{ordinary generating function} as the following formal power series: $$a_0+a_1x+a_2x^2+a_3x^3+\cdots=\sum_{n=0}^\infty a_nx^n$$
  \end{definition}
  \begin{proposition}
    Let $\displaystyle\sum_{n=0}^\infty a_nx^n,\sum_{n=0}^\infty b_nx^n$ be two formal power series. Then:
    \begin{itemize}
      \item $\displaystyle\sum_{n=0}^\infty a_nx^n+\sum_{n=0}^\infty b_nx^n=\sum_{n=0}^\infty (a_n+b_n)x^n$
      \item $\displaystyle\lambda\sum_{n=0}^\infty a_nx^n=\sum_{n=0}^\infty\lambda a_nx^n$
      \item $\displaystyle\left(\sum_{n=0}^\infty a_nx^n\right)\left(\sum_{n=0}^\infty b_nx^n\right)=\sum_{n=0}^\infty (a_0b_n+a_1b_{n-1}+\cdots +a_nb_0)x^n$
      \item $\displaystyle\left(\sum_{n=0}^\infty a_nx^n\right)'=\sum_{n=1}^\infty na_nx^{n-1}$
    \end{itemize}
  \end{proposition}
  \begin{proposition}[Closed forms]
    We can write the following ordinary generating functions with their corresponding closed forms:
    \begin{itemize}
      \item $\displaystyle\sum_{n=0}^Nx^n=\frac{1-x^{N+1}}{1-x}$
      \item $\displaystyle\sum_{n=0}^\infty x^n=\frac{1}{1-x}$
      \item $\displaystyle\sum_{n=0}^\infty\binom{n+k-1}{n}x^n=\left(\frac{1}{1-x}\right)^k$
    \end{itemize}
  \end{proposition}
  \begin{proposition}
    Suppose $A$ and $B$ are two finite disjoint sets. We set some restrictions for the non-ordered selection of elements of $A\cup B$. For every $n\geq 0$, let:
    \begin{itemize}
      \item $a_n$ be the number of non-ordered selection of $n$ elements of $A$ satisfying the restrictions.
      \item $b_n$ be the number of non-ordered selection of $n$ elements of $B$ satisfying the restrictions.
      \item $c_n$ be the number of non-ordered selection of $n$ elements of $A\cup B$ satisfying the restrictions.
    \end{itemize}
    And let $f(x)$, $g(x)$, $h(x)$ be the ordinary generating functions of $(a_n)$, $(b_n)$, $(c_n)$, respectively. Then we have: $$h(x)=f(x)g(x)$$
  \end{proposition}
  \begin{definition}
    Let $(a_n)$ be a sequence of real numbers. We define its \emph{exponential generating function} as the following formal power series: $$a_0+a_1x+a_2\frac{x^2}{2!}+a_3\frac{x^3}{3!}+\cdots=\sum_{n=0}^\infty a_n\frac{x^n}{n!}$$
  \end{definition}
  \begin{definition}
    Let $(a_n)$ be a sequence of real numbers such that $a_i=1$ $\forall i$. Then its exponential generating function associated is the so called \emph{exponential series}: $$\exp{x}=\sum_{n=0}^\infty \frac{x^n}{n!}$$
  \end{definition}
  \begin{proposition}
    The exponential series has the following properties:
    \begin{enumerate}
      \item $\exp{x+y}=\exp{x}\exp{y}$ $\forall x,y\in\RR $.
      \item $(\exp{x})^n=\exp{nx}$ $\forall x,n\in\RR $.
    \end{enumerate}
  \end{proposition}
  \begin{proposition}
    Suppose $A$ and $B$ are two finite disjoint sets. We set some restrictions for the ordered selection of elements of $A\cup B$. For every $n\geq 0$, let:
    \begin{itemize}
      \item $a_n$ be the number of ordered selection of $n$ elements of $A$ satisfying the restrictions.
      \item $b_n$ be the number of ordered selection of $n$ elements of $B$ satisfying the restrictions.
      \item $c_n$ be the number of ordered selection of $n$ elements of $A\cup B$ satisfying the restrictions.
    \end{itemize}
    And let $f(x)$, $g(x)$, $h(x)$ be the exponential generating functions of $(a_n)$, $(b_n)$, $(c_n)$, respectively. Then we have: $$h(x)=f(x)g(x)$$
  \end{proposition}
  \subsubsection{Recurrence relations}
  \begin{definition}
    Let $(a_n)$ be a sequence of real numbers. A \emph{recurrence relation of order $k$} for $(a_n)$ is an expression that express $a_n$ in terms of $k$ consecutive terms of the sequence, $a_{n-1},\ldots,a_{n-k}$, for $k\leq n$. We say a sequence is \emph{recurrent} if it satisfies a recurrence relation or, equivalently, if it's a solution of the recurrence relation.
  \end{definition}
  \begin{definition}
    The \emph{initial values} of a recurrence relation of order $k$ are the values of the first $k$ terms for which the recurrence relation is still not valid, that is, the values $a_0,a_1,\ldots,a_{k-1}$.
  \end{definition}
  \begin{lemma}
    The solution of a recurrence relation of order $k$ with $k$ initial conditions is unique.
  \end{lemma}
  \begin{definition}
    A \emph{linear recurrence relation of order $k$} is a recurrence relation that can be written as the form: $$a_n+c_1a_{n-1}+\cdots c_ka_{n-k}=g(n)$$ where $c_1,\ldots c_k\in\RR , c_k\ne 0$ and $g:\NN \rightarrow\NN $ is an arbitrary function.
  \end{definition}
  \begin{definition}
    We say a linear recurrence relation is \emph{homogeneous} if $g(n)=0$, that is, if it's of the form: $$a_n+c_1a_{n-1}+\cdots c_ka_{n-k}=0\quad\text{with }c_k\ne 0$$
  \end{definition}
  \begin{proposition}
    The general solution to a recurrence relation $$a_n+c_1a_{n-1}+\cdots c_ka_{n-k}=g(n)$$ can be expressed as: $$(a_n^\text{part})+(a_n^\text{hom})$$ where $(a_n^\text{part})$ is a particular solution of the recurrence relation and $(a_n^\text{hom})$ is the general solution of its associated homogeneous recurrence relation.
  \end{proposition}
  \begin{proposition}
    Given $c_1,\ldots,c_k\in\RR $, the set of sequences that are solution of the homogeneous linear recurrence relation $a_n+c_1a_{n-1}+\cdots+c_ka_{n-k}=0$ form a real vector space.
  \end{proposition}
  \begin{definition}
    Let $a_n+c_1a_{n-1}+\cdots+c_ka_{n-k}=0$ be a homogeneous linear recurrence relation of order $k$. The \emph{characteristic polynomial} of the recurrence is: $$x^k+c_1x^{k-1}+\cdots+c_k=0$$
  \end{definition}
  \begin{proposition}
    Consider a homogeneous linear recurrence relation with characteristic polynomial $$(x-r_1)(x-r_2)\cdots(x-r_k)=0$$ where $r_1,\ldots,r_k\in\CC $ are different complex numbers. Then the general term of the sequences that satisfy the recurrence relation is: $$a_n=\lambda_1r_1^n+\cdots+\lambda_kr_k^n$$ for arbitrary numbers $\lambda_1,\ldots,\lambda_k\in\CC $.
  \end{proposition}
  \subsection{Graph theory}
  \begin{definition}
    A \emph{graph} $G$ is an structure based on a set $V(G)$ of vertices and a set $E(G)$ of edges, which are non-ordered pairs of vertices.
  \end{definition}
  \begin{definition}
    Let $G$ be a graph. The \emph{order} of $G$ is $n=|V(G)|$ and the \emph{size} of $G$ is $m=|E(G)|$.
  \end{definition}
  \begin{definition}
    Let $G$ be a graph. Two vertices $a,b\in V(G)$ are said to be \emph{adjacent} to one another if exists an edge $e\in E(G)$ that connects them. In this case we say the edge $e$ is \emph{incident} on vertices $a$ and $b$.
  \end{definition}
  \begin{definition}
    An edge that connects a vertex with itself is called a \emph{loop}.
  \end{definition}
  \begin{definition}
    Two or more edges incidents with the same vertices are called \emph{multiple edges}.
  \end{definition}
  \begin{definition}
    A graph $G$ is \emph{finite} if $V(G)$ and $E(G)$ are finite.
  \end{definition}
  \begin{definition}
    A graph is \emph{simple} if it has neither multiples edges nor loops.
  \end{definition}
  \begin{definition}
    A \emph{complete graph} is a graph in which each pair of different vertices is joined by an edge. We denote by $K_n$ the complete graph of order $n$.
  \end{definition}
  \begin{definition}
    Let $G$ be a finite graph. The \emph{degree of a vertex} is the number of edges that are incident to it. If $v\in V(G)$ we denote the degree of $v$ by $\deg v$ or $\deg_Gv$\footnote{Observe that with this definition every loop counts as two edges.}.
  \end{definition}
  \begin{lemma}[Handshaking lemma]
    For every graph $G$ we have: $$\sum_{v\in V(G)}\deg v=2|E(G)|$$
  \end{lemma}
  \begin{corollary}
    In any graph, the number of odd-degree vertices is even.
  \end{corollary}
  \begin{definition}
    Let $G$ be a graph with $V(G)=\{v_1,\ldots,v_n\}$. The \emph{degree sequence} of $G$ is the decreasing sequence: $$(\deg v_{i_1},\ldots,\deg v_{i_n})$$
  \end{definition}
  \begin{definition}
    We say a graph $G$ is $k$-\emph{regular} if $\deg v=k$ $\forall v\in V(G)$.
  \end{definition}
  \begin{definition}
    Let $G$ be a graph. A graph $F$ is an \emph{induced subgraph} of $G$ if $V(F)\subseteq V(G)$ and $E(F)\subseteq E(G)$.
  \end{definition}
  \begin{definition}
    A \emph{walk} of length $k$ in a graph $G$ is a sequence of vertices $(u_1,\ldots,u_k)$ where $u_iu_{i+1}\in E(G)$ for $i=1,\ldots,k-1$.
  \end{definition}
  \begin{definition}
    A walk in a graph is \emph{closed} if it starts and ends in the same vertex.
  \end{definition}
  \begin{definition}
    A walk in a graph is a \emph{trail} if all the edges of the walk are distinct.
  \end{definition}
  \begin{definition}
    A walk in a graph is a \emph{path} if all the vertices (and therefore the edges) of the walk are distinct.
  \end{definition}
  \begin{definition}
    A closed walk in a graph is a \emph{closed trail} if all the edges of the closed walk are distinct.
  \end{definition}
  \begin{definition}
    A closed path is called a \emph{cycle}.
  \end{definition}
  \begin{proposition}
    Let $G$ be a graph. Given $u,v\in V(G)$, there exists a walk between $u$ and $v$ if and only if there exists a path between $u$ and $v$.
  \end{proposition}
  \begin{definition}
    Let $G$ be a graph. Given $u,v\in V(G)$, we say that $u$ and $v$ are connected if there is a path in $G$ between $u$ and $v$.
  \end{definition}
  \begin{proposition}
    The relation $u\sim v$ if and only if $u$ and $v$ are connected is an equivalence relation. The equivalent classes are the \emph{connected components} of $G$.
  \end{proposition}
  \begin{definition}
    A graph $G$ is \emph{connected} if $\forall u,v\in V(G)$, $u$ and $v$ are connected.
  \end{definition}
  \begin{definition}
    A graph $G$ is \emph{bipartite} if $V(G)=X\sqcup Y$ and $\forall e\in E(G)$ we have $e=xy$ with $x\in X$ and $y\in Y$.
  \end{definition}
  \begin{definition}
    Let $G$ be a graph such that $E(G)\ne\varnothing$. Take an edge $e\in E(G)$. We denote by $G-e$ the induced graph of $G$ such that: $$V(G-e)=V(G)\quad\text{and}\quad E(G-e)=E(G)\setminus\{e\}$$
  \end{definition}
  \begin{definition}
    Given a connected graph $G$, we say that $e\in E(G)$ is a \emph{bridge} of $G$ if $G-e$ is non-connected.
  \end{definition}
  \begin{proposition}
    Let $G$ be a connected graph. $e\in E(G)$ is a bridge if and only if $e$ doesn't belong to any cycle of $G$.
  \end{proposition}
  \begin{definition}
    Let $G$ be a connected graph. An \emph{Eulerian trail} in $G$ is a trail that contain all the edges of $G$. An \emph{Eulerian circuit} in $G$ is a closed Eulerian trail. $G$ is called \emph{Eulerian} if it admits an Eulerian circuit.
  \end{definition}
  \begin{theorem}[Euler theorem]
    Let $G$ be a connected graph. $G$ is Eulerian $\iff\deg v=2k$ $\forall v\in V(G)$, $k\in\NN $.
  \end{theorem}
  \begin{definition}
    Let $G$ be a graph of order $n$ with $V(G)=\{v_1,\ldots,v_n\}$. We define the \emph{adjacency matrix} of $G$, $\vf{A}(G)\in\mathcal{M}_n(\RR )$, as $a_{ij}$ to be the number of edges incident with $v_i$ and $v_j$.
  \end{definition}
  \begin{proposition}
    Let $G$ be a graph of order $n$ with $V(G)=\{v_1,\ldots,v_n\}$ and let $\vf{A}(G)=(a_{ij})$ be the adjacency matrix of $G$. Then:
    \begin{enumerate}
      \item $\vf{A}(G)$ is symmetric.
      \item $\displaystyle\sum_{j=1}^n a_{jk}=\sum_{j=1}^n a_{kj}=\deg v_k,\quad k=1,\ldots,n$.
      \item For $k\in\NN $, consider $\vf{A}(G)^k=(b_{ij}^k)$. Then $b_{ij}^k$ is equal to the number of walks of length $k$ between vertices $v_i$ and $v_j$.
    \end{enumerate}
  \end{proposition}
  \begin{definition}
    A \emph{tree} is an acyclic connected graph, that is, a connected graph that has no cycles.
  \end{definition}
  \begin{definition}
    Let $T$ be a tree. A \emph{leave} of $T$ is a vertex of degree 1.
  \end{definition}
  \begin{definition}
    Let $G$ be a graph. A \emph{generator tree} is an induced subgraph $T$ of $G$ such that $|V(G)|=|V(T)|$ and $T$ is a tree.
  \end{definition}
  \begin{proposition}
    Let $G$ be a graph such that $|V(G)|=n\geq 2$. The following are equivalent:
    \begin{enumerate}
      \item $G$ is a tree.
      \item $G$ is connected and every edge of $G$ is a bridge.
      \item $G$ is connected and $|E(G)|=n-1$.
      \item $G$ is acyclic and $|E(G)|=n-1$.
      \item For $v_i,v_j\in V(G)$, $i\ne j$, there exists a unique path between $v_i,v_j$.
      \item $G$ is acyclic but adding a new edge creates exactly one cycle.
    \end{enumerate}
  \end{proposition}
  \begin{definition}
    Let $G$ be a connected graph. $G$ is called \emph{traversable} if admits an Eulerian trail.
  \end{definition}
  \begin{theorem}
    Let $G$ be a connected graph. $G$ is traversable if and only if $G$ has exactly to odd-degree vertices.
  \end{theorem}
  \begin{definition}
    Two graphs $G$, $H$ are said to be \emph{isomorphic} if exists a bijective map $f:V(G)\rightarrow V(H)$ such that $vv'\in E(G)\iff f(v)f(v')\in E(H)$.
  \end{definition}
  \begin{proposition}
    Two finite isomorphic graphs have the same order, size and degree sequence.
  \end{proposition}
  \begin{theorem}
    Two graphs $G$, $H$ are isomorphic if and only if exists a permutation matrix $\vf{P}$ such that: $$\vf{P}\vf{A}(G)\transpose{\vf{P}}=\vf{A}(H)$$ where $\vf{A}(G)$, $\vf{A}(H)$ are adjacency matrices of $G$, $H$, respectively.
  \end{theorem}
  \subsection{Linear programming}
  \begin{definition}
    Given vectors $\vf{c},\vf{u},\vf{v}\in\RR ^n$, $\vf{b}\in\RR ^m$ and a matrix $\vf{A}\in\mathcal{M}_{m\times n}(\RR )$, we define the \emph{linear programming to maximize}\footnote{Analogously we can define a \emph{linear programming to minimize} changing the objective function to a minimize function.} as: $$\text{LP}=
      \begin{cases}
        \max:              & z=\transpose{\vf{c}}x\qquad    (\emph{objective function}) \\
        \text{subject to}: & \vf{A}x\lesseqgtr \vf{b}\qquad\, (\emph{restrictions})     \\
                           & \vf{u}\leq x\leq \vf{v}
      \end{cases}$$
  \end{definition}
  \begin{definition}
    Given vectors $\vf{c},\vf{u},\vf{v}\in\RR ^n$, $\vf{b}\in\RR ^m$ and a matrix $\vf{A}\in\mathcal{M}_{m\times n}(\RR )$, we define the \emph{canonical form of a linear programming to maximize} as: $$\text{LP}=
      \begin{cases}
        \max:              & z=\transpose{\vf{c}}x\qquad   (\emph{objective function}) \\
        \text{subject to}: & \vf{A}x\leq \vf{b}\qquad\,      (\emph{restrictions} )    \\
                           & \vf{u}\leq x\leq \vf{v}
      \end{cases}$$
    Analogously we define the \emph{canonical form of a linear programming to minimize} as: $$\text{LP}=
      \begin{cases}
        \min:              & z=\transpose{\vf{c}}x   \\
        \text{subject to}: & \vf{A}x\geq \vf{b}      \\
                           & \vf{u}\leq x\leq \vf{v}
      \end{cases}$$
  \end{definition}
  \begin{definition}
    Given a linear program, the \emph{feasible region} of the program is the set: $$\mathfrak{F}=\{x\in\RR ^n:\vf{A}x\lesseqgtr \vf{b},\vf{u}\leq x\leq \vf{v}\}$$ That is, the set of the points that satisfy the conditions of the problem.
  \end{definition}
  \begin{proposition}
    Given $x\in\RR ^n$, $x$ is a \emph{feasible solution} of the linear program if and only if $x\in\mathfrak{F}$.
  \end{proposition}
  \begin{definition}
    A \emph{polyhedron} $P$ is a set of $\RR ^n$ that can be expressed as an intersection of a finite collection of half-spaces, that is: $$P=\{x\in\RR ^n:\vf{A}x\geq \vf{b}, \vf{A}\in\mathcal{M}_{m\times n}(\RR ),\vf{b}\in\RR ^m\}$$ A \emph{polytope} is a non-empty and bounded polyhedron. The feasible region of any linear program is a polyhedron.
  \end{definition}
  \begin{definition}
    Let $P\subset\RR ^n$ be a polyhedron. A point $x\in\RR ^n$ is an extreme point of $P$ if there is neither a pair of points $y,z\in P$, nor a scalar $\lambda\in[0,1]$ such that $x=\lambda y+(1-\lambda)z$.
  \end{definition}
  \begin{definition}
    Let LP be a linear program. We define the \emph{standard form} of LP as:
    $$\text{LP}=
      \begin{cases}
        \min:              & z=\transpose{\vf{c}}x \\
        \text{subject to}: & \vf{A}x=\vf{b}        \\
                           & x\geq 0
      \end{cases}$$
  \end{definition}
  \begin{definition}
    Let $\displaystyle \text{LP}=\min_{x\in\RR ^n}\{\transpose{\vf{c}}x:\vf{A}x=\vf{b},x\geq 0\}$. Feasible solution in which free variables or non-basic variable equal zero with respect to basis of basic variables are called \emph{basic feasible solutions}.
  \end{definition}
  \begin{proposition}
    If a linear program admits feasible solutions, exists a basic feasible solution. If a linear program admits an optimal solution, exists an optimal basic feasible solution.
  \end{proposition}
  \begin{theorem}
    Let $P$ be a non-empty polyhedron of a linear program in standard form with maximum rank and let $x\in P$. Then $x$ is an extreme point of $P$ if and only if $x$ is a basic feasible solution.
  \end{theorem}
  \begin{definition}[Simplex method: Phase I]
    Given a linear program in standard form:
    $$\text{LP}=
      \begin{cases}
        \min:              & z=\transpose{\vf{c}}x \\
        \text{subject to}: & \vf{A}x=\vf{b}        \\
                           & x\geq 0
      \end{cases}$$ its associated problem in phase I ($\text{LP}_1$) is: $$\text{LP}_1=
      \begin{cases}
        \min:              & \displaystyle w=\sum_{i=1}^my_i \\
        \text{subject to}: & \vf{A}x+\vf{I}_my=\vf{b}        \\
                           & x,y\geq 0
      \end{cases}$$
    A condition necessary for LP having basic feasible solutions is that the optimal solution of $\text{LP}_1$ must be $w=0$. In fact, if $w\ne 0$, then the original linear program has no feasible solutions\footnote{This phase is useful to find, if there is, an initial basic feasible solution.}.
  \end{definition}
  \begin{proposition}[Simplex method: Phase II]
    Suppose in a simplex table with positive pivots and therefore independent-terms vector $\vf{d}\geq 0$, there is a coefficient $c_j<0$. $$\left(\begin{array}{c|c}
          *      & \transpose{\vf{d}} \\
          \hline
          \vf{c} & z-z_0
        \end{array}\right)$$
  \end{proposition}
  To find a basic feasible solution with lower cost, we make the following change of variable:
  \begin{enumerate}
    \item The variable in column $j$ becomes a basic variable.
    \item The variable in row $i$ such that: $$\frac{d_i}{a_{ij}}=\min\left\{\frac{d_k}{a_{kj}}:a_{kj}>0\right\}$$ becomes a non-basic variable. If this variable does not exists, that is, $a_{kj}\leq0$ $\forall k$ then the linear program is not bounded.
  \end{enumerate}
  \begin{definition}
    Let $\text{LP}=\displaystyle\min_{x\in\RR ^n}\{\transpose{\vf{c}}x:\vf{A}x\geq \vf{b},x\geq 0\}$. We define the \emph{dual program} of LP as: $$\text{LP}^*=
      \begin{cases}
        \max:              & z=\transpose{\vf{b}}y          \\
        \text{subject to}: & \transpose{\vf{A}}y\leq \vf{c} \\
                           & y\geq 0
      \end{cases}$$ The linear program LP is called \emph{primal}.
  \end{definition}
  \begin{theorem}[Weak duality theorem]
    Let $x$ be a feasible solution of the primal linear program and $y$ a feasible solution of the dual linear program. Then we have:
    \begin{itemize}
      \item $\transpose{\vf{c}}x\leq \transpose{\vf{d}}y$ if the primal linear program is in canonical form to maximize.
      \item $\transpose{\vf{c}}x\geq \transpose{\vf{d}}y$ if the primal linear program is in canonical form to minimize.
    \end{itemize}
  \end{theorem}
  \begin{corollary}
    Let $x$, $y$ be feasible solutions of the primal and dual linear programs respectively such that $\transpose{\vf{c}}x=\transpose{\vf{d}}y$. Then $x$ and $y$ are optimal solutions.
  \end{corollary}
  \begin{theorem}[Strong duality theorem]
    Any linear program has an optimal solution if and only if its dual linear program does, and in this case, the values coincide.
  \end{theorem}
  \begin{theorem}[Complementary property]
    Suppose that the optimal table of the primal linear program is of the form:
    $$\left(
      \begin{array}{@{\,} c|c @{\,}}
          *      & \transpose{\vf{d}} \\
          \hline
          \vf{c} & z-z_0              \\
        \end{array}\right)$$ where $\vf{c}=(c_1,\ldots,c_{n+m})$ and $\vf{d}=(d_1,\ldots,d_m)$ with $c_i\geq0$, $i=1,\ldots,n+m$. If $(y_1,\ldots y_m,t_1^*,\ldots,t_n^*)$ is the optimal solution of the dual linear program, expressed in standard form, then: $$c_1=t_1^*,\ldots,c_n=t_n^*, c_{n+1}=y_1,\ldots,c_{n+m}=y_m$$
  \end{theorem}
\end{multicols}
\end{document}