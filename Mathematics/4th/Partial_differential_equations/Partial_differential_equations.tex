\documentclass[../../../main_math.tex]{subfiles}


\begin{document}
\renewcommand{\col}{\apl}
\begin{multicols}{2}[\section{Partial differential equations}]
  \subsection{PDEs in Physics}
  \subsubsection{Wave and membrane dynamics}
  \begin{proposition}[Wave equation]
    Consider a one-dimensional string of length $L$ and constant $k(x)$, $\rho$ be its linear density and $u(x,t)$ be the displacement of the point $x$ at the time $t$ from its equilibrium point. Then, the dynamics of the string are given by: $$\rho u_{tt}={(ku_x)}_x$$ If both $k$ and $\rho$ are constant, this equation is sometimes written as:
    \begin{equation}\label{PDE_waveeq}
      u_{tt}=c^2u_{xx}
    \end{equation}
    These kind of equations are called \emph{hyperbolic equations}.
  \end{proposition}
  \begin{proposition}[Navier-Cauchy equation]
    Consider a solid of mass density $\rho$ and let $\mu$ and $\lambda$ be the so-called \emph{Lamé coefficients} that describe the material. If $\vf{u}(\vf{x},t)$ is the displacement vector at the point $\vf{x}$ and the instant $t$, the equation that describes the deformation of the solid (\emph{elastodynamics}) is:
    $$\rho\vf{u}_{tt}=\mu\laplacian\vf{u}+(\lambda+\mu)\grad(\divp{\vf{u}})$$
  \end{proposition}
  \subsubsection{Fluid dynamics}
  \begin{definition}
    Given a vector field $\vf{u}(\vf{x},t)$, we define the \emph{material derivative operator} as: $$\matdv{\vf{u}}{t}:=\vf{u}_t+(\vf{u}\cdot\grad)\vf{u}$$
  \end{definition}
  \begin{definition}
    An \emph{incompressible flow} is a flow in which the material density is constant.
  \end{definition}
  \begin{proposition}[Continuous equation]
    Consider a fluid of density $\rho$ moving at a velocity $\vf{u}(\vf{x},t)$. The conservation of mass implies that the following equation (called \emph{continuous equation}) must hold:
    \begin{equation}\label{PDE_continuous}
      \rho_t+\divp(\rho\vf{u})=0
    \end{equation}
    If the fluid is incompressible, the previous equation becomes: $$\divp{\vf{u}}=0$$
  \end{proposition}
  \begin{proposition}[Cauchy momentum equation]
    Consider an inviscid fluid of density $\rho$ moving at a velocity $\vf{u}(\vf{x},t)$ and undergoing a pressure of $p(\vf{x},t)$. The conservation of momentum implies that the following equation (called \emph{Cauchy momentum equation}) must hold:
    \begin{equation}\label{PDE_cauchy}
      \rho\matdv{\vf{u}}{t}+\grad p=0
    \end{equation}
  \end{proposition}
  \begin{theorem}[Inviscid flow]
    Consider an incompressible inviscid flow of density $\rho$ moving at a velocity $\vf{u}(\vf{x},t)$ and undergoing a pressure of $p(\vf{x},t)$. The equations describing the dynamics of the flow are:
    \begin{equation*}
      \left\{
      \begin{aligned}
        \rho\matdv{\vf{u}}{t}+\grad p & =0 \\
        \divp{\vf{u}}                 & =0
      \end{aligned}
      \right.
    \end{equation*}
    If however the flow is compressible, the equations become:
    \begin{equation*}
      \left\{
      \begin{aligned}
        \rho\matdv{\vf{u}}{t}+\grad p & =0 \\
        \rho_t+\divp(\rho\vf{u})      & =0
      \end{aligned}
      \right.
    \end{equation*}
  \end{theorem}
  \begin{theorem}[Viscid flow]
    Consider an incompressible viscid fluid of density $\rho$, viscosity $\eta$, moving at a velocity $\vf{u}(\vf{x},t)$ and undergoing a pressure of $p(\vf{x},t)$. The equations describing the dynamics of the flow are:
    \begin{equation*}
      \left\{
      \begin{aligned}
        \rho\matdv{\vf{u}}{t}+\grad p & =\eta\laplacian\vf{u} \\
        \divp{\vf{u}}                 & =0
      \end{aligned}
      \right.
    \end{equation*}
    If however the flow is compressible, the equations become:
    \begin{equation*}
      \left\{
      \begin{aligned}
        \rho\matdv{\vf{u}}{t}+\grad p & =\eta\left(\laplacian\vf{u}+\frac{1}{3}\grad(\divp{\vf{u}})\right) \\
        \rho_t+\divp(\rho\vf{u})      & =0
      \end{aligned}
      \right.
    \end{equation*}
  \end{theorem}
  \subsubsection{Potential theory}
  \begin{proposition}
    Consider a body $\Omega\subset\RR^3$ with a density of mass $\rho$. The gravitational force done by this body to a mass $m$ located at the position $\vf{x}\in\RR^3$ is given by:
    $$\vf{F}(\vf{x})=-Gm\int_{\Omega}\frac{\vf{x}-\vf{y}}{{\norm{\vf{x}-\vf{y}}}^3}\rho(\vf{y})\dd^3{\vf{y}}$$
  \end{proposition}
  \begin{proposition}
    Consider a body $\Omega\subset\RR^3$ with a density of mass $\rho$. Then, $\vf{F}(\vf{x})=m\grad{u(\vf{x})}$ where $$u(\vf{x})=G\int_{\Omega}\frac{1}{\norm{\vf{x}-\vf{y}}}\rho(\vf{y})\dd^3{\vf{y}}$$ is the \emph{potential} created by the body $\Omega$ at the point $\vf{x}\in\RR^3$. Furthermore, if $\rho$ is regular enough, we have $\divp\vf{F}(\vf{x})=-4\pi\rho(\vf{x})$\footnote{That is, $\divp\vf{F}(\vf{x})=0$ $\forall\vf{x}\in\RR^3\setminus\Omega$.}. Combining these two equation, we get: $$\laplacian u=-4\pi\rho$$ which is the \emph{Poisson equation} (and also it is a \emph{elliptic equation}).
  \end{proposition}
  \subsubsection{Diffusion and heat equations}
  \begin{proposition}[Fick's law of diffusion]
    Consider a material with \emph{diffusivity} (or \emph{diffusion coefficient}) $D$, $\vf{\phi}$ the \emph{diffusion flux} and $u$ be its concentration. Then, \emph{Fick's law} states that: $$\vf{\phi}=-D\grad{u}$$
  \end{proposition}
  \begin{proposition}[Diffusion equation]
    Consider a material with diffusivity $D$, $\vf{\phi}$ the diffusion flux and $u$ its concentration. Then, the concentration of the material satisfies: $$\pdv{u}{t}=\divp\left(D\grad{u}\right)$$
    In particular, if $D=\const$, then we get $\pdv{u}{t}=D\laplacian{u}$.
  \end{proposition}
  \begin{proposition}[Fourier's law]
    Consider a material with \emph{thermal conductivity} $k$, $\vf{q}$ be the \emph{heat flux} and $u(x,t)$ its temperature. Then, \emph{Fourier's law} states that: $$\vf{q}=-k\grad{u}$$
  \end{proposition}
  \begin{proposition}[Heat equation]
    Consider a material with thermal conductivity $k$ and $u$ be its temperature. Then, the temperature of the material satisfies: $$\pdv{u}{t}=\frac{1}{c\rho}\divp\left(k\grad{u}\right)$$
    where $c$ is the \emph{specific heat capacity} and $\rho$ is the \emph{density}. In particular, if $k=\const$, then we get $\pdv{u}{t}=\alpha\laplacian{u}$, where $\alpha:=\frac{k}{c\rho}$ is the \emph{thermal diffusivity}.
  \end{proposition}
  \subsubsection{Maxwell equations}
  \begin{proposition}[Gau\ss's law]
    \emph{Gau\ss's law} states that a static electric field points away from positive charges and towards negative charges, and the net outflow of the electric field through a closed surface $\Fr{\Omega}$ is proportional to the enclosed charge.
    \begin{align*}
      \divp\vf{E}                                 & =\frac{\rho}{\varepsilon_0}                     & \text{(Differential form)} \\
      \oiint_{\Fr{\Omega}}\vf{E}\cdot \dd{\vf{S}} & =\frac{1}{\varepsilon_0}\iiint_\Omega\rho\dd{V} & \text{(Integral form)}
    \end{align*}
  \end{proposition}
  \begin{proposition}[Gau\ss's law for magnetism]
    \emph{Gau\ss's law for magnetism} states that for each volume element $\Omega$ in space, there are exactly the same number of magnetic field lines entering and exiting the volume. No total magnetic charge can build up in any point in space.
    \begin{align*}
      \divp\vf{B}                                 & =0 \\
      \oiint_{\Fr{\Omega}}\vf{B}\cdot \dd{\vf{S}} & =0
    \end{align*}
  \end{proposition}
  \begin{proposition}[Maxwell-Faraday equation]
    \emph{Maxwell-Faraday equation} states that a time-varying magnetic field always accompanies a spatially varying (also possibly time-varying), non-conservative electric field, and vice versa
    \begin{align*}
      \rotp\vf{E}                                   & =\pdv{\vf{B}}{t}                              \\
      \oint_{\Fr{\Sigma}}\vf{E}\cdot \dd{\vf{\ell}} & =-\dv{}{t}\iint_\Sigma\vf{B}\cdot \dd{\vf{S}}
    \end{align*}
  \end{proposition}
  \begin{proposition}[Ampère-Maxwell circuital law]
    The original \emph{Ampère's law} ($\rotp\vf{B}=\mu_0\vf{J}$) stats a relation between the total amount of magnetic field around some closed path $\Fr{\Sigma}$ due to the current that passes through that enclosed path $\Sigma$. The second term on the right-hand-side (added later by Maxwell) is the \emph{displacement current} associated with the polarization of the individual molecules of the dielectric material.
    \begin{align*}
      \rotp\vf{B}                                   & =\mu_0\left(\vf{J}+\varepsilon_0\pdv{\vf{E}}{t}\right)                                                          \\
      \oint_{\Fr{\Sigma}}\vf{B}\cdot \dd{\vf{\ell}} & =\mu_0 \left(\iint_\Sigma\vf{J}\cdot\dd{\vf{S}}+\varepsilon_0\dv{}{t}\iint_\Sigma\vf{E}\cdot \dd{\vf{S}}\right)
    \end{align*}
  \end{proposition}
  \subsubsection{Mechanics and optics}
  \begin{definition}
    We define the \emph{refractive index} is defined as: $$n(\vf{x})=\frac{c}{v(\vf{x})}$$ where $c$ is the speed of the light in the vacuum and $v(\vf{x})$ the speed of the light at the position $\vf{x}$ (located in some medium).
  \end{definition}
  \begin{proposition}[Fermat's principle]
    \emph{Fermat's principle} states that the path taken by a ray between two given points $a$ and $b$ is the path that can be traveled in the least time. Mathematically, we want to minimize the functional: $$\mathcal{T}(\vf{x})=\int_a^b\frac{\abs{\dd{\vf{x}}}}{v(\vf{x})}$$
    So we shall solve the equation $\delta \mathcal{T}=0$, which is equivalent to solve: $$\delta\int_a^bn(\vf{x})\dd{s}=0$$ where $s$ is the arc-length parameter. From the Euler-Lagrange equations, we get the following ode: $$\dv{}{s}\left(n\dv{\vf{x}}{s}\right)=\grad{n}$$
  \end{proposition}
  \begin{proposition}[Eikonal equation]
    The time $T(x)$ taken by the light to travel from a fixed point $x_0$ to $x$ in a medium of refractive index $n$ is given by: $${\norm{\grad{T}}}^2=n^2$$
  \end{proposition}
  \begin{definition}
    The \emph{action} $\mathcal{S}$ of a physical system is defined as the integral of the Lagrangian $L:=T-V$ between two instants of time $t_1$ and $t_2$. That is: $$\mathcal{S}(\vf{x},t)=\int_{t_1}^{t_2}L(\vf{x}(t),\vf{\dot{x}}(t),t)\dd{t}=\int_{t_1}^{t_2}\left(\frac{1}{2}m{\norm{\vf{\dot{x}}}}^2-V(\vf{x})\right)\dd{t}$$
    where $m$ is the mass of the particle, $T$ is the kinetic energy of the particle and $V$ is its potential energy.
  \end{definition}
  \begin{proposition}[Principle of least action]
    The path taken by a physical system between times $t_1$ and $t_2$ and configurations $\vf{x}_1$ and $\vf{x}_2$ is the one for which the action is stationary (no change) to first order. Mathematically, $\delta \mathcal{S}=0$, where $\delta$ means a \emph{small change}. This value $S(\vf{x},t)$ of the action satisfies the \emph{Hamilton-Jacobi equation}: $$\pdv{S}{t}+\frac{1}{2m}{\norm{\grad S}}^2+V=0$$
  \end{proposition}
  \begin{proposition}[Schrödinger equation]
    The \emph{Schrödinger equation} is a pde that governs the \emph{wave function} $\Psi$, which describes the quantum state of an isolated quantum system, of a quantum-mechanical system. This is given by: $$\ii \hbar\pdv{\Psi}{t}=\left(-\frac{\hbar^2}{2m}\laplacian+V\right){\Psi}$$ where $m$ is the mass of the particle and $V$ is the potential in which the particle exists. Furthermore, $\abs{\Psi}^2$ is the probability density function of the position of the particle.
  \end{proposition}
  \begin{proposition}
    Substituting ${\Psi}=\sqrt{\rho}\exp{\ii \frac{S}{\hbar}}$ into the Schrödinger equation and taking the limit $\hbar\to 0$ in the resulting equation yield the Hamilton-Jacobi equation. Moreover, if we define $\vf{v}=\frac{\grad{S}}{m}$, from one real equation (from the original one complex equation) we get the continuous equation (\cref{PDE_continuous}) and from the imaginary equation taking the limit $\hbar\to 0$ we get the Cauchy momentum equation (\cref{PDE_cauchy}).
  \end{proposition}
  \subsection{First order partial differential equations}
  \subsubsection{Vector calculus}
  \begin{theorem}[Fundamental lemma of calculus of variations]
    Let $\Omega\subset\RR^n$ be a domain and $\vf{f}:\Omega\rightarrow\RR^m$ be a continuous function. If $$\dotsint_U \vf{f}(\vf{x})\dd{\vf{x}}=0$$ for any subset  $U\subseteq\Omega$, then $\vf{f}=\vf{0}$ in $\Omega$.
  \end{theorem}
  \begin{proposition}
    Let $\Omega\subseteq\RR^3$ be a closed region and $f,g,k:\Omega\rightarrow\RR$ be functions of class $\mathcal{C}^1(\Omega)$. Then:
    \begin{gather*}
      \int_\Omega f\div(k\grad{g})                                =\int_{\Fr{\Omega}}fk\grad{g}\cdot\dd{\vf{S}} -\int_\Omega k\grad{f}\cdot\grad{g} \\
      \int_\Omega \left(f\div(k\grad{g})-g\div(k\grad{f})\right)  =\int_{\Fr{\Omega}} k\left(f\grad{g}-g\grad{f}\right)\cdot\dd{\vf{S}}
    \end{gather*}
  \end{proposition}
  \begin{corollary}[Green identities]
    Let $\Omega\subseteq\RR^3$ be a closed region and $f,g,k:\Omega\rightarrow\RR$ be functions of class $\mathcal{C}^1(\Omega)$. Then:
    \begin{gather*}
      \int_\Omega f\laplacian{g}                              =\int_{\Fr{\Omega}}f\grad{g}\cdot\dd{\vf{S}} -\int_\Omega \grad{f}\cdot\grad{g} \\
      \int_\Omega \left(f\laplacian{g}-g\laplacian{f}\right)  =\int_{\Fr{\Omega}} \left(f\grad{g}-g\grad{f}\right)\cdot\dd{\vf{S}}
    \end{gather*}
  \end{corollary}
  \subsubsection{Method of characteristics}
  \begin{proposition}[Method of characteristics]
    Consider the following quasilinear PDE
    \begin{equation}\label{PDE_char}
      a(x,t,u)\pdv{u}{x}+b(x,t,u)\pdv{u}{t}=c(x,t,u)
    \end{equation}
    with initial condition $u(x_0(s),t_0(s))=u_0(s)$.
    Note that we can write this equation as: $$\begin{pmatrix}
        a(x,t,u) & b(x,t,u) & c(x,t,u)
      \end{pmatrix}\cdot\begin{pmatrix}
        \pdv{u}{x} \\
        \pdv{u}{t} \\
        -1
      \end{pmatrix}=0$$
    And so the solutions of \cref{PDE_char} are the integral curves (called \emph{characteristic curves}) that form the surface of the graph $u(x,t)$. These are given by:
    \begin{equation*}
      \left\{
      \begin{aligned}
        \dv{x}{\tau} & = a(x,t,u) \\
        \dv{t}{\tau} & = b(x,t,u) \\
        \dv{u}{\tau} & = c(x,t,u) \\
      \end{aligned}
      \right.
    \end{equation*}
    with initial conditions $x(0,s)=x_0(s)$, $t(0,s)=t_0(s)$ and $u(0,s)=u_0(s)$.
  \end{proposition}
  \subsubsection{Traffic flow equation}
  \begin{proposition}[Traffic flow equation]
    Consider a one lane motorway with one entry an one exit. Let $\rho(x,t)$ be the density of cars per unit of length, $u(\rho)$ the average speed of the cars and $q=\rho u$ be the flux of cars. Then, we can model the traffic in the motorway with the equation: $$\rho_t+{(\rho u)}_x=\rho_t+q'(\rho){\rho}_x=0$$
    The integral form of the latter equation is: $$\pdv{}{t}\int_{a}^b\rho(x,t)\dd{x}=q(a,t)-q(b,t)$$
  \end{proposition}
  \begin{proposition}
    In the hypothesis of the traffic equation, an observer situated at $x(t)$ will observe a constant density $\rho(x(t),t)$ if $x'(t)=q'(\rho(x(t),t))$ (that is, if the frame of reference situated at $x(t)$ is moving at a speed of $q'(\rho(x(t),t))$)\footnote{Note that this
      velocity may be different from the velocity at which an individual car moves.}. Therefore, $\rho$ is constant in each line of the form $x(t)=x_0+q'(\rho(x_0,0))t$ (\emph{characteristic line}) (see \cref{PDE_traffic-char}). This determines $\rho(x,t)$ provided that we already know the initial condition $\rho_0(x):=\rho(x,0)$, $x\in\RR$. In other words, $\rho(x,t)$ is the solution $\xi$ of the density at the appropriate $x$-intercept of the line passing through $(x,t)$: $$\xi=\rho_0(x-q'(\xi)t)$$
  \end{proposition}
  \begin{center}
    \begin{minipage}{\linewidth}
      \centering
      \includestandalone[mode=image|tex,width=\linewidth]{Images/traffic_char}
      \captionof{figure}{Characteristics of the traffic flow. In each line the density $\rho$ is constant.}
      \label{PDE_traffic-char}
    \end{minipage}
  \end{center}
  \begin{proposition}[Rankine-Hugoniot equation]
    Let $x_s(t)$ be the position at time $t$ of a (jump) discontinuity in the function $\rho$. Then: $$\dv{x_s}{t}=\frac{[q]}{[\rho]}=\frac{{(\rho u)}_+-{(\rho u)}_-}{\rho_+-\rho_-}$$ where the notation $[x(t)]$ refers to: $$[x(t_0)]:=x_+(t_0)-x_-(t_0):=\lim_{t\to{t_0}^+}x(t)-\lim_{t\to{t_0}^-}x(t)$$
  \end{proposition}
  \begin{lemma}[Entropy condition]
    We will have existence and unicity of solutions for the traffic flow equation if: $$q'(\rho_+)<\frac{[q]}{[\rho]}<q'(\rho_-)$$
  \end{lemma}
  \subsection{Wave equation}
  \begin{proposition}
    Let $u:\RR^2\rightarrow\RR$ be a two-times differentiable function such that:
    \begin{multline*}
      \left.\int_{x_1}^{x_2}\rho u_t\dd{x}\right|_{t=t_2}-\left.\int_{x_1}^{x_2}\rho u_t\dd{x}\right|_{t=t_1}=\left.\int_{t_1}^{t_2}ku_x\dd{t}\right|_{x=x_2}-\\-\left.\int_{t_1}^{t_2}ku_x\dd{t}\right|_{x=x_1}+\int_{t_1}^{t_2}\int_{x_1}^{x_2}f(x,t)\dd{x}\dd{t}
    \end{multline*}
    for certain functions $\rho(x,t)$, $k(x)$, $f(x,t)$. Also suppose that we can freely exchange the derivative and the integral sign\footnote{From now on, we will simply say that $u$ is a \emph{good enough} function.}. Then, $u(x,t)$ is a solution to the wave equation with driven force $f$:
    $$\rho u_{tt}-{(ku_x)}_x=f(x,t)$$
  \end{proposition}
  \subsubsection{Constant coefficients}
  \begin{proposition}[D'Alembert formula]
    Let $u_0,v_0:\RR\rightarrow\RR$ be functions. The solution $u(x,t)$ to the wave equation with initial conditions $u(x,0)=u_0(x)$ and $u_t(x,0)=v_0(x)$ is:
    \begin{equation}\label{PDE_dAlembert}
      u(x,t)=\frac{u_0(x-ct)+u_0(x+ct)}{2}+\frac{1}{2c}\int_{x-ct}^{x+ct}v_0(s)\dd{s}
    \end{equation}
  \end{proposition}
  \begin{theorem}
    Let $u_0,v_0:\RR\rightarrow\RR$ and $f:\RR^2\rightarrow\RR$ be functions. The solution to the wave equation with driven force $f$ and initial conditions $u(x,0)=u_0(x)$ and $u_t(x,0)=v_0(x)$ is:
    \begin{multline*}
      u(x,t)=\frac{u_0(x-ct)+u_0(x+ct)}{2}+\frac{1}{2c}\int_{x-ct}^{x+ct}v_0(s)\dd{s}+\\+\frac{1}{2c}\int_0^t\int_{x-c(t-\tau)}^{x+c(t-\tau)}f(s,\tau)\dd{s}\dd{\tau}
    \end{multline*}
    If we think $u(t):x\rightarrow u(x,t)$, then we can write the expression above more compactly as: $$u(t)=\mathcal{S}'(t)u_0+\mathcal{S}(t)v_0+\int_0^t\mathcal{S}(t-\tau) f(\tau)\dd{\tau}$$ where the operator $\mathcal{S}(t)$ is defined as: $$\left[\mathcal{S}(t)\varphi\right](x)=\frac{1}{2c}\int_{x-ct}^{x+ct}\varphi(s)\dd{s}$$
  \end{theorem}
  \begin{theorem}
    Let $U\subset\RR^2$ be an open set, $u:U\rightarrow\RR$ be a good enough function such that it satisfies the wave equation with density $\rho(x,t)$, constant $k(x)$ and driven force $f(x,t)$. Then:
    $$\oint_{\Fr{U}}\rho u_t\dd{x}+k u_x\dd{t}=\iint_Uf(x,t)\dd{x}\dd{t}$$
  \end{theorem}
  \begin{proposition}
    Let $U\subset\RR^2$ be an open set, $u:U\rightarrow\RR$ be a good enough function such that it satisfies the wave equation with constant $c^2=\frac{k}{\rho}$ and driven force $f(x,t)$. Consider the four points $A$, $B$, $C$ and $D$ as shown in \cref{PDE_waves-char}. Then:
    \begin{equation}\label{PDE_charwaveseq}
      u(A)-u(B)+u(C)-u(D)=0
    \end{equation}
    \begin{center}
      \begin{minipage}{\linewidth}
        \centering
        \includestandalone[mode=image|tex,width=0.7\linewidth]{Images/waves_char}
        \captionof{figure}{Characteristics of the waves equation.}
        \label{PDE_waves-char}
      \end{minipage}
    \end{center}
  \end{proposition}
  \begin{proposition}[Conservation of energy]
    Consider the wave equation $\rho u_{tt}-{(ku_x)}_x=0$ and assume the functions $u_0$, $v_0$ of the initial conditions have compact support. Then: $$\dv{}{t}\int_{-\infty}^\infty\left(\frac{1}{2}\rho{u_t}^2+\frac{1}{2}k{u_x}^2\right)\dd{x}=0$$ That is, the energy is conserved.
  \end{proposition}
  \begin{corollary}
    The wave equation with drive force has existence and unicity of solutions.
  \end{corollary}
  \begin{proposition}
    Consider the wave equation $u_{tt}=c^2u_{xx}$ with initial conditions:
    $$
      \left\{
      \begin{aligned}
        u(x,0)   & =u_0(x)   \\
        u_t(x,0) & =v_0(x)   \\
        u_x(0,t) & =\beta(t)
      \end{aligned}
      \right.
    $$
    Then, the d'Alembert solution is $$u(x,t)=\phi(x+ct)+\psi(x-ct)$$ where:
    \begin{align*}
      \phi(x) & =\frac{1}{2}u_0(x)+\frac{1}{2c}\int_0^xv_0(s)\dd{s}\quad\text{for }x\geq 0 \\
      \psi(x) & =
      \begin{cases}
        \displaystyle\frac{1}{2}u_0(x)-\frac{1}{2c}\int_0^xv_0(s)\dd{s} & \text{if } x\geq 0 \\
        \displaystyle\phi(-x)+\int_0^x\beta(-s/c)\dd{s}                 & \text{if } x< 0
      \end{cases}
    \end{align*}
    In particular, if $\beta(t)=0$ and we make the even extension of both $u_0$ and $v_0$, we have:
    \begin{equation*}
      \psi(x) =\frac{1}{2}u_0(x)-\frac{1}{2c}\int_0^xv_0(s)\dd{s} \quad\forall x\in\RR
    \end{equation*}
  \end{proposition}
  \subsubsection*{Fixed-endpoints boundary condition}
  \begin{definition}
    Consider a string of length $L$ with its two endpoints fixed. In this section we will discuss how to obtain the solutions of its movement solving following boundary problem:
    \begin{equation}\label{PDE_fixedendpoints}
      \left\{
      \begin{aligned}
        u_{tt}   & =c^2u_{xx} \\
        u(0,t)   & =0         \\
        u(L,t)   & =0         \\
        u(x,0)   & =u_0(x)    \\
        u_t(x,0) & =v_0(x)
      \end{aligned}
      \right.
    \end{equation}
  \end{definition}
  \begin{method}
    Consider the odd extensions for $u_0$ and $v_0$ of \cref{PDE_fixedendpoints}. Then, the solutions of that equation are given by the d'Alembert formula (\cref{PDE_dAlembert})
  \end{method}
  \begin{method}\label{PDE_methodchar}
    Suppose we want to know the displacement $u(x,t)$ of the string at the position $x\in[0,L]$ and time $t\geq 0$. We will use \cref{PDE_charwaveseq} to determine $u(x,t)$. Construct the characteristic lines $x\pm ct$ passing through the point $(x,t)$ (highest green dot) and then the corresponding ones passing through the intersecting points of the previous ones with lines $x=0$ and $x=L$ (as shown in \cref{PDE_waves-char-solve}). Then, by \cref{PDE_charwaveseq} we have that the $u(x,t)$ in the green dots is $-u(x,t)$ in the yellow dots (because $u(x,t) = 0$ in the purple dots). Since we are provided with the equation at $t=0$, we can determine $u(x,t)$ in the brown dots and get back, thus, to the original green point through the last yellow point.
  \end{method}
  \begin{center}
    \begin{minipage}{\linewidth}
      \centering
      \includestandalone[mode=image|tex,width=0.7\linewidth]{Images/waves_char_solve}
      \captionof{figure}{Scheme for \cref{PDE_methodchar} of solving the wave equation}
      \label{PDE_waves-char-solve}
    \end{minipage}
  \end{center}
  \begin{method}[Separation of variables]
    The solution $u(x,t)$ to \cref{PDE_fixedendpoints}, using \emph{separation of variables} (i.e. assuming $u(x,t)=f(x)g(t)$), is: $$u(x,t)=\sum_{n=0}^\infty \sin\left(\frac{\pi n x}{L}\right)\left[a_n\cos\left(\frac{\pi n c}{L}t\right)+ b_n\sin\left( \frac{\pi n c}{L}t\right)\right]$$ where:
    \begin{align*}
      a_n & =\frac{1}{L}\int_{-L}^Lf(x)\cos\left(\frac{\pi n x}{L}\right)\dd{x}       \\
      b_n & =\frac{1}{\pi n c}\int_{-L}^Lg(x)\sin\left(\frac{\pi n x}{L}\right)\dd{x}
    \end{align*}
  \end{method}
  \begin{proposition}
    Consider the simplified Schrödinger equation problem:
    \begin{equation}\label{PDE_schro-fixedendpoints}
      \left\{
      \begin{aligned}
        \ii u_{t} & =u_{xx} \\
        u(0,t)    & =0      \\
        u(L,t)    & =0      \\
        u(x,0)    & =u_0(x)
      \end{aligned}
      \right.
    \end{equation}
    The solution $u(x,t)$ to \cref{PDE_schro-fixedendpoints}, using separation of variables is: $$u(x,t)=\sum_{n=1}^\infty a_n\exp{-\ii\frac{\pi^2n^2}{L^2}t}\sin\left(\frac{\pi n x}{L}\right)$$ where:
    \begin{equation*}
      a_n =\frac{1}{L}\int_{-L}^Lf(x)\sin\left(\frac{\pi n x}{L}\right)\dd{x}
    \end{equation*}
  \end{proposition}
  \subsubsection{Variable coefficients}
  \begin{theorem}[Sturm-Picone comparison theorem]
    Let $p_i,q_i:\RR\rightarrow\RR$, $i=1,2$, be functions such that $0<p_2<p_1$ and $q_1<q_2$. Suppose we have that the functions $u(x)$ and $v(x)$ satisfy the following differential equations:
    \begin{align*}
      {(p_1(x)u')}'+q_1(x)u=0 \\
      {(p_2(x)u')}'+q_2(x)u=0
    \end{align*}
    If $\alpha_1$, $\alpha_2$ are two successive roots of $u$, then one of the following holds:
    \begin{enumerate}
      \item $\exists \beta\in(\alpha_1,\alpha_2)$ such that $v(\beta)=0$.
      \item $\exists\lambda\in\RR$ such that $v(x)=\lambda u(x)$ $\forall x\in\RR$.
    \end{enumerate}
  \end{theorem}
  \begin{proposition}
    Consider the following problem of the wave equation of non-constant coefficients:
    \begin{equation}
      \left\{
      \begin{aligned}
        \rho u_{tt} & ={(ku_x)}_x \\
        u(0,t)      & =0          \\
        u(L,t)      & =0          \\
        u(x,0)      & =u_0(x)     \\
        u_t(x,0)    & =v_0(x)
      \end{aligned}
      \right.
    \end{equation}
    Then, the general solution to this problem is:
    $$u(x,t)=\sum_{n=0}^\infty X_n(x)\left[a_n\cos\left(\sqrt{\lambda_n}t\right)+ b_n\sin\left( \sqrt{\lambda_n}t\right)\right]$$ where $X_n(x)$ is the solution to the problem $$ \left\{
      \begin{aligned}
         & {(k{X_n}')}'+\lambda_n\rho X_n =0  \\
         & X_n(0)                          =0 \\
         & X_n(L)                          =0
      \end{aligned}
      \right.$$
    and:
    \begin{equation*}
      a_n =\frac{\displaystyle\int_{0}^Lu_0(x)X_n(x)\rho(x)\dd{x}}{\displaystyle\int_{0}^LX_n(x)^2\rho(x)\dd{x}}\ \
      b_n =\frac{\displaystyle\int_{0}^Lv_0(x)X_n(x)\rho(x)\dd{x}}{\displaystyle\sqrt{\lambda_n}\int_{0}^LX_n(x)^2\rho(x)\dd{x}}
    \end{equation*}
  \end{proposition}
  \subsection{Heat equation}
  \begin{proposition}
    Consider the following boundary problem of the heat equation:
    \begin{equation}\label{PDE_heat-fixedendpoints}
      \left\{
      \begin{aligned}
        u_{t}  & =\alpha u_{xx} \\
        u(0,t) & =0             \\
        u(L,t) & =0             \\
        u(x,0) & =u_0(x)
      \end{aligned}
      \right.
    \end{equation}
    with $\alpha=\const$.
    The solution $u(x,t)$ to \cref{PDE_heat-fixedendpoints}, using separation of variables is: $$u(x,t)=\sum_{n=1}^\infty a_n\exp{-\frac{\alpha\pi^2n^2}{L^2}t}\sin\left(\frac{\pi n x}{L}\right)$$ where:
    \begin{equation*}
      a_n =\frac{1}{L}\int_{-L}^Lf(x)\sin\left(\frac{\pi n x}{L}\right)\dd{x}
    \end{equation*}
  \end{proposition}
  \begin{definition}
    A function $f(x,t)$ is said to be \emph{self-similar} if $\exists \alpha\in\RR$ such that $\frac{x}{t^\alpha}=\const$.
  \end{definition}
  \begin{proposition}
    Consider the heat equation of constant coefficients $ u_{t} =\alpha u_{xx}$ on the whole real line. Then, if we impose $u$ being self-similar satisfying $u(x,t)=u(\lambda x,\lambda^2t)$ $\forall\lambda\in\RR$, we obtain: $$u(x,t)=c_1\int_0^{\frac{x}{\sqrt{t}}}\exp{-\frac{z^2}{4\alpha}}\dd{z} +c_2$$
  \end{proposition}
  \begin{definition}
    The \emph{Dirac delta distribution} $\delta(x)$ is the pdf defined as $\delta(x)=0$ for $x\ne 0$ and such that: $$\int_{-\infty}^{+\infty}\delta(x)\dd{x}=1$$
  \end{definition}
  \begin{definition}
    A \emph{fundamental solution} (or \emph{heat kernel}) is a solution of the heat equation corresponding to the initial condition of an initial point source of heat at a known position. That is, it is the solution to the problem:
    \begin{equation}\label{PDE_fundamental}
      \left\{
      \begin{aligned}
        u_{t}  & =\alpha u_{xx} \\
        u(x,0) & =\delta(x)
      \end{aligned}
      \right.
    \end{equation}
    where $\delta(x)$ is the Dirac delta distribution.
  \end{definition}
  \begin{theorem}
    The heat kernel of \cref{PDE_fundamental} is: $$u(x,t)=\frac{1}{\sqrt{4\pi\alpha t}}\exp{-\frac{x^2}{4\alpha t}}$$
  \end{theorem}
\end{multicols}
\end{document}