\documentclass[../../../main.tex]{subfiles}


\begin{document}
\renewcommand{\col}{\apl}
\begin{multicols}{2}[\section{Dynamical systems}]
  \subsection{Dynamical systems}
  \begin{definition}
    A \emph{dynamical system} is a triplet $(X,G,\Pi)$, where $G$ is a topological abelian group\footnote{That is, $G$ is an abelian group with an inherited topological structure.}, $X$ is a topological space and $\Pi:X\times G\rightarrow X$ is a function such that:
    \begin{itemize}
      \item $\Pi(\cdot,t)$ is continuous $\forall t\in G$.
      \item $\Pi(x,0)=x$ $\forall x\in X$.
      \item $\Pi(\Pi(t,x),s)=\Pi(x,t+s)$ $\forall s,t\in G$ and $\forall x\in X$.
    \end{itemize}
    We say that a dynamical system $(X,G,\Pi)$ is \emph{discrete} if $G=\ZZ$ and we say that it is \emph{continuous} if $G=\RR$.
  \end{definition}
  \begin{definition}
    Let $(X,G,\Pi)$ be a dynamical system and $x\in X$. We define the \emph{orbit} of $x$ as: $$\gamma_x:=\{\Pi(x,t):t\in G\}$$
  \end{definition}
  \begin{definition}
    Let $(X,G,\Pi)$ be a dynamical system. In the continuous case, we say that $x\in X$ is an \emph{equilibrium point} if $\Pi(x,t)=x$ $\forall t\in G$. In the discrete case, a point $x\in X$ satisfying this property is called a \emph{fixed point}.
  \end{definition}
  \begin{definition}
    Let $(X,G,\Pi)$ be a dynamical system. A \emph{periodic orbit of period $T$} is an orbit of the system that satisfies $\Pi(x,t+T)=\Pi(x,t)$ $\forall t\in G$ and for some $x\in X$.
  \end{definition}
\end{multicols}
\end{document}