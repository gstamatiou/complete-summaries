\documentclass[../../../main.tex]{subfiles}


\begin{document}
\renewcommand{\col}{\ana}
\begin{multicols}{2}[\section{Real and functional analysis}]
  \subsection{Measure theorey and Lebesgue integral}
  \subsubsection{Measures}
  \begin{definition}[$\sigma$-algebra]
    Let $\Omega$ be a  set and $\Sigma\subseteq\mathcal{P}(\Omega)$. We say that $\Sigma$ is a \emph{$\sigma$-algebra} over $\Omega$ if:
    \begin{enumerate}
      \item $\Omega\in\Sigma$.
      \item If $A\in\Sigma$, then $A^c\in\Sigma$.
      \item If $A_1,A_2,\ldots\in\Sigma$, then: $$\bigcup_{n=1}^\infty A_n\in\Sigma$$
    \end{enumerate}
  \end{definition}
  \begin{proposition}
    Let $\Sigma$ be an $\sigma$-algebra over a set $\Omega$. Then:
    \begin{enumerate}
      \item $\varnothing\in\Sigma$.
      \item If $A,B\in\Sigma$, then $A\setminus B\in\Sigma$.
      \item If $A_1,A_2,\ldots\in\Sigma$, then: $$\bigcap_{n=1}^\infty A_n\in\Sigma$$
    \end{enumerate}
  \end{proposition}
  \begin{definition}[Measure]
    Let $\Sigma$ be a $\sigma$-algebra over a set $\Omega$. A \emph{measure} over $\Omega$ is any function $$\mu:\Sigma\longrightarrow[0,\infty]$$ satisfying the following properties:
    \begin{itemize}
      \item $\mu(\varnothing)=0$.
      \item \emph{$\sigma$-additivity}: If $\{A_n:n\geq1\}\subseteq\Sigma$ are pairwise disjoint, then: $$\mu\left(\bigsqcup_{n=1}^\infty A_n\right)=\sum_{n=1}^\infty \mu(A_n)$$
    \end{itemize}
  \end{definition}
  \begin{definition}
    Let $\Sigma$ be a set and $\{A_n:n\geq1\}\subseteq\Sigma$ be subsets. We say that $A_n\nearrow A$ if $A_n\subseteq A_{n+1}$ $\forall n\in\NN$ and $A=\bigcup_{n=1}^\infty A_n$. Analogously, we say that $A_n\searrow A$ if $A_n\supseteq A_{n+1}$ $\forall n\in\NN$ and $A=\bigcap_{n=1}^\infty A_n$.
  \end{definition}
  \begin{proposition}
    Let $\Sigma$ be a $\sigma$-algebra over a set $\Omega$, $\mu:\Sigma\longrightarrow[0,\infty]$ be a measure over $\Omega$ and $A_n,A,B\in\Sigma$, $n\in\NN$. Then:
    \begin{itemize}
      \item If $A\subseteq B$, then $\mu(A)\subseteq\mu(B)$.
      \item If $A_n\nearrow A$, then $\displaystyle\mu(A)=\lim_{n\to\infty} A_n$.
      \item If $A_n\searrow A$ and $\mu(A_1)<\infty$, then $\displaystyle\mu(A)=\lim_{n\to\infty} A_n$.
    \end{itemize}
  \end{proposition}
  \begin{definition}
    An \emph{interval} $I\subseteq\RR^n$ is a set of the form:
    $$I=\abs{a_1,b_1}\times\cdots\times\abs{a_n,b_n}$$
    where $a_i,b_i\in\RR_\infty$ and the notation $\abs{a,b}$ represents either $(a,b)$, $[a,b)$, $(a,b]$ or $[a,b]$.
  \end{definition}
  \begin{definition}
    Let $I=\prod_{i=1}^n\abs{a_i,b_i}\subseteq\RR^n$ be an interval. We define its \emph{volume} as:
    $$\vol(I):=\prod_{i=1}^n(b_i-a_i)$$
  \end{definition}
  \begin{definition}
    Let $m\in\NN\cup\{0\}$. We define the \emph{$m$-th dyadic cube} as the set: $$\mathcal{D}_m:=[a_1,a_1+2^{-m})\times\cdots\times [a_n,a_n+2^{-m})$$
    where $a_i\in 2^{-m}\ZZ$\footnote{Note that each $\mathcal{D}_m$ forms a partition of $\RR^n$}.
  \end{definition}
  \begin{lemma}
    Let $m\in\NN\cup\{0\}$. Then the sidelength of $\mathcal{D}_m$ is $2^{-m}$, $\vol(\mathcal{D}_m)= 2^{-mn}$ and its diameter is $\mathcal{D}_m= 2^{-m}\sqrt{n}$.
  \end{lemma}
  \begin{proposition}
    All nonempty open set $U\subseteq\RR^n$ can be written as a countable union of disjoint dyadic cubes whose closure is in $U$.
  \end{proposition}
  \begin{definition}
    Let $A\subseteq\RR^n$ be a set. We denote by $\mathcal{I}(A)$ the set of covers of $A$ that are intervals. Analogously, we denote by $\mathcal{I}_0(A)$ the set of open covers of $A$ that are intervals.
  \end{definition}
  \begin{definition}[Outer measure]
    Let $A\subseteq\RR^n$ be a set. We define its \emph{outer measure} as the measure $\om{}$ defined by:
    $$\om{A}:=\inf\left\{\sum_{k= 1}^\infty \vol(I_k):\{I_k:k\geq 1\}\in \mathcal{I}(A)\right\}$$
  \end{definition}
  \begin{proposition}
    Let $A\subseteq\RR^n$ be a set. Then:
    $$\om{A}=\inf\left\{\sum_{k= 1}^\infty \vol(I_k):\{I_k:k\geq 1\}\in \mathcal{I}_0(A)\right\}$$
  \end{proposition}
  \begin{lemma}
    Let $I,J_1,\ldots,J_N\subseteq \RR^n$ be intervals such that $I\subseteq \bigcup_{k=1}^N J_k$. Then, $\vol(I)\leq\sum_{k=1}^N \vol(J_k)$.
  \end{lemma}
  \begin{theorem}
    The outer measure has the following properties:
    \begin{enumerate}
      \item $\om{\varnothing}=0$.
      \item If $A\subseteq B\subseteq\RR^n$, then $\om{A}\leq \om{B}$.
      \item If $A_1, A_2,\ldots \subseteq \RR^n$, then: $$\om{\bigcup_{k=1}^\infty A_k}\leq \sum_{k=1}^\infty \om{A_k}$$
      \item If $I\subseteq \RR^n$ is an open interval and $I\subseteq A\subseteq \cl{I}$, then $\om{A}=\vol(I)$.
      \item If $I_1,\ldots,I_N\subseteq \RR^n$ are disjoint intervals, then: $$\displaystyle \om{\bigsqcup_{k=1}^N I_k}= \sum_{k=1}^N \vol(I_k)$$
      \item If $A,B\subseteq \RR^n$ and $d(A,B):=\inf\{d(a,b):a\in A,b\in B\}>0$, then $\om{A\cup B}=\om{A}+\om{B}$.
      \item If $A\subseteq\RR^n$ and $x\in\RR^n$, then $\om{A+x}=\om{-A}=\om{A}$\footnote{Here $A+x:=\{a+x:a\in A\}$ and $-A:=\{-a:a\in A\}$}.
    \end{enumerate}
  \end{theorem}
  \begin{definition}
    A set $N\subset\RR^n$ is called a \emph{null set} if $\om{N}=0$.
  \end{definition}
  \begin{definition}
    We say that a property holds \emph{almost everywhere} (\emph{a.e.}) if the set of points that doesn't hold it is null.
  \end{definition}
  \begin{lemma}
    The countable union of null sets is null.
  \end{lemma}
  \begin{lemma}
    A point is null. Therefore, all countable sets are null.
  \end{lemma}
  \subsubsection{Lebesgue measure}
  \begin{definition}[Lebesgue measure]
    We say that $A\subseteq\RR^n$ is \emph{Lebesgue measurable} (or simply \emph{measurable}) if $\forall \varepsilon>0$, there exists an open set $U\supseteq A$ such that $\om{U\setminus A}<\varepsilon$. We denote by $\mathcal{M}$ the set of all Lebesgue measurable sets of $\RR^n$ and by $\m{}$ the restriction of $\om{}$ to ${\mathcal{M}}$.
  \end{definition}
  \begin{theorem}
    In $\RR^n$, $\mathcal{M}$ is a $\sigma$-algebra and $m:\mathcal{M}\rightarrow[0,\infty]$ is a measure (called \emph{Lebesgue measure}) that satisfies:
    \begin{enumerate}
      \item The open sets, closed sets and null sets are in $\mathcal{M}$.
      \item Each interval $I\subseteq \RR^n$ is measurable and $\m{I}=\vol(I)$.
      \item If $A\subseteq\mathcal{M}$ and $x\in\RR^n$, then $A+x,-A\in\mathcal{M}$ and $\m{A+x}=\m{-A}=\m{A}$.
      \item If $A\subseteq\mathcal{M}$:
            \begin{align*}
              \m{A} & =\inf\{\m{U}:A\subseteq U \subseteq\RR^n, U\text{ open}\}  \\
                    & =\sup\{\m{C}:C\subseteq A\subseteq\RR^n, C\text{ closed}\}
            \end{align*}
    \end{enumerate}
  \end{theorem}
  \subsubsection{Lebsegue measurable functions}
  \begin{definition}
    A \emph{real function} is a function $f:\RR^n\rightarrow[-\infty,+\infty]$. We will say that $f$ is \emph{finite} if $\pm\infty\notin\im f$.
  \end{definition}
  \begin{definition}
    Let $f$ be a real function. We say that $f$ is \emph{Lebesgue measurable} if $\{x\in\RR^n:f(x)> r\}\in \mathcal{M}$ $\forall r\in\RR$.
  \end{definition}
  \begin{lemma}
    Let $a,b\in[-\infty,+\infty]$ and $f$ be a real function. The sets:
    \begin{itemize}
      \item $\{a<f(x)<b:x\in\RR^n\}$
      \item $\{a\leq f(x)<b:x\in\RR^n\}$
      \item $\{a<f(x)\leq b:x\in\RR^n\}$
      \item $\{a\leq f(x)\leq b:x\in\RR^n\}$
    \end{itemize}
    are all measurable.
  \end{lemma}
  \begin{proposition}
    A function $f:\RR^n\rightarrow\RR$ is measurable if and only if for all open set $U\subseteq \RR$, $f^{-1}(U)\in\mathcal{M}$.
  \end{proposition}
  \begin{proposition}
    Let $f$ be a finite measurable real function, $U\subseteq\RR$ be an open set such that $\im f\subseteq U$ and $\varphi:U\rightarrow\RR$ be a continuous function. Then, $\varphi\circ f$ is also measurable.
  \end{proposition}
  \begin{proposition}
    Let $u$, $v$ be two finite measurable real functions, $U\subseteq\RR^2$ be an open set such that $(u(x),v(x))\in U$ $\forall x\in\RR^n$ and $\varphi:U\rightarrow\RR$ be a continuous function. Then, $\varphi(u(x),v(x))$ is also measurable.
  \end{proposition}
  \begin{proposition}
    Let $f$, $g$ be two real functions such that $f$ is measurable and $g\almoste{=} f$. Then, $g$ is also measurable.
  \end{proposition}
  \begin{proposition}
    Let $f$, $g$ be two measurable real functions. Then, so are $f\pm g$, $fg$ and $f/g$ if $g(x)\ne 0$ $\forall x\in\RR^n$.
  \end{proposition}
  \begin{proposition}
    Let $(f_m)$ be a sequence of measurable real functions. Then, the following sets are measurable:
    \begin{itemize}
      \item $\displaystyle\sup\{f_n(x):n\in\NN,x\in\RR^n\}$
      \item $\displaystyle\inf\{f_n(x):n\in\NN,x\in\RR^n\}$
      \item $\displaystyle\limsup_{m\to\infty} f_m$
      \item $\displaystyle\limsup_{m\to\infty} f_m$
    \end{itemize}
    Furthermore, any function being pointwise limit a.e. of a sequence of measurable functions is measurable.
  \end{proposition}
  \begin{definition}
    Let $f$ be a measurable function. We define the following measurable functions:
    $$f^+:=\sup\{f,0\}\qquad f^-:=\sup\{-f,0\}$$
    Note that then, $f=f^+-f^-$ and $\abs{f}=f^++f^-$.
  \end{definition}
  \begin{definition}
    A \emph{simple function} is a linear combination $$s:=\sum_{k=1}^N\alpha_k\indi{A_k}$$ where $\alpha_k\in\RR$ and $A_k\in\mathcal{M}$ for $K=1,\ldots,k$\footnote{We may suppose that the sets $A_k$ are pairwise disjoint, the quantities $\alpha_k$ are all different and that $A_k=s^{-\alpha_k}$.}.
  \end{definition}
  \begin{theorem}
    Let $f:\RR^n\rightarrow[0,+\infty]$ be a measurable function and let $$E(k,m):=\left\{\frac{k-1}{2^m}\leq f<\frac{k}{2^m}\right\}\;\;\text{and}\;\; F(m):=\{f\geq m\}$$
    Then, the sequence of positive simple functions $$s_m=m+\indi{F(k)}+\sum_{k=1}^{m2^m}\frac{k-1}{2^m}\indi{E(k,m)}$$ is increasing and $\displaystyle\lim_{m\to\infty}s_m(x)=f(x)$ $\forall x\in\RR^n$
  \end{theorem}
  \begin{theorem}
    Let $f:\RR^n\rightarrow[-\infty,+\infty]$ be a measurable function. Then, there exists a sequence of simple functions $(s_m)$ such that $\displaystyle\lim_{m\to\infty}s_m(x)=f(x)$ $\forall x\in\RR^n$ and $\abs{s_m}\leq\abs{s_{m+1}}\leq \abs{f}$ $\forall m\in\NN$.
  \end{theorem}
  \subsubsection{Lebesgue integral}
  \begin{definition}
    Let $N\in\NN$, $E_1,\ldots,E_N$ be measurable sets and $s=\sum_{k=1}^N\alpha_k\indi{E_k}$ be a positive simple function such that $0<\alpha_1<\cdots<\alpha_N<\infty$. We define the \emph{integral of $s$ over $\RR^n$} as: $$\int s:=\sum_{k=1}^N\alpha_k\m{E_k}$$
    We define the \emph{integral of $s$ over a measurable set $E$} as: $$\int_E s:=\int s\indi{E}=\sum_{k=1}^N\alpha_k\m{E_k\cap E}$$
  \end{definition}
  \begin{proposition}
    Let $A_1,A_2,\ldots$ be measurable sets and $s$, $t$ be simple functions. Then:
    \begin{itemize}
      \item If $\displaystyle E=\bigcap_{k=1}^\infty A_k$, then $\displaystyle\int_E s=\sum_{k=1}^\infty \int_{A_k} s$.
      \item $\displaystyle\int(s+t)=\int s+\int t$.
      \item If $\alpha\in\RR_{\geq 0}$, then $\displaystyle\int \alpha s=\alpha\int s$.
      \item If $s\leq t$, then $\displaystyle\int s\leq\int t$.
    \end{itemize}
  \end{proposition}
  \begin{proposition}
    The function $$\function{\mu_s}{\mathcal{M}}{[0,\infty]}{E}{\displaystyle\int_E s}$$ is a measure.
  \end{proposition}
  \begin{definition}
    Let $f:\RR^n\rightarrow[0,+\infty]$ be a measurable function. We define: $$\mathcal{S}(f):=\{s:s\text{ is a simple function such that }0\leq s\leq f\}$$
  \end{definition}
  \begin{definition}
    Let $f:\RR^n\rightarrow[0,+\infty]$ be a measurable function. We define the \emph{integral of $f$ over $\RR^n$} as: $$\int_{\RR^n}f(x)\dd{x}:=\sup_{s\in\mathcal{S}(f)}\int s$$ We define the \emph{integral of $f$ over a measurable set $E\subseteq \RR^n$} as: $$\int_{E}f(x)\dd{x}:=\int_{\RR^n}f(x)\indi{E}(x)\dd{x}=\sup_{s\in\mathcal{S}(f)}\int s$$
  \end{definition}
  \begin{proposition}
    Let $E\subseteq\RR^n$ be a measurable set, $s$ be a simple function and $f$, $g$ be measurable function such that $f(x)\leq g(x)$ $\forall x\in E$. Then:
    \begin{enumerate}
      \item $\displaystyle\int_E s=\int_Es(x)\dd{x}$
      \item $\displaystyle\int_E f\leq \int_Eg$
    \end{enumerate}
  \end{proposition}
  \begin{theorem}[Monotone convergence theorem]
    Let $E\subseteq\RR^n$ be a measurable set, $f\geq 0$ be a non-negative measurable function such that $\exists (f_m)\geq 0$ with $f_m\nearrow f$. Then: $$\int_Ef(x)\dd{x}=\lim_{m\to\infty}\int_Ef_m(x)\dd{x}$$
  \end{theorem}
  \begin{proposition}
    Let $E\subseteq\RR^n$ be a measurable set, $f, g, (f_m)\geq 0$ be a non-negative measurable functions. Then:
    \begin{enumerate}
      \item $\displaystyle\int_E\sum_{m=1}^\infty f_m(x)\dd{x}=\sum_{m=1}^\infty\int_E f_m(x)\dd{x}$
      \item If $(E_k)$ is a sequence of measurable sets such that $E=\bigsqcup_{k=1}^\infty E_k$, then: $$\int_Ef(x)\dd{x}=\sum_{k=1}^\infty\int_{E_k}f(x)\dd{x}$$
      \item If $\alpha\in[0,\infty)$, then $\displaystyle\int_E\alpha f(x)\dd{x}=\alpha\int_Ef(x)\dd{x}$.
      \item If $f\leq g$ on $E$, then $\displaystyle\int_Ef(x)\dd{x}\leq\int_Eg(x)\dd{x}$.
      \item $\displaystyle\int_Ef(x)\dd{x}=0\iff f\almoste{=}0$ on $E$.
      \item If $N\subset E$ is a null set, then $\displaystyle\int_Ef(x)\dd{x}=\int_{E\setminus N}f(x)\dd{x}$.
      \item If $\displaystyle\int_Ef(x)\dd{x}<\infty$, then $f\almoste{<}\infty$ on $E$.
      \item If $h\in\RR^n$, then $\displaystyle\int_{E-h}f(x+h)\dd{x}=\int_{-E}f(-x)\dd{x}=\int_{E}f(x)\dd{x}$
    \end{enumerate}
  \end{proposition}
  \begin{corollary}
    Let $E\subseteq\RR^n$ be a measurable set, $f\geq 0$ be a non-negative measurable function such that $\exists (f_m)\geq 0$ with $f_m\almoste{\nearrow} f$. Then: $$\int_Ef(x)\dd{x}=\lim_{m\to\infty}\int_Ef_m(x)\dd{x}$$
  \end{corollary}
  \begin{lemma}[Fatou lemma]
    Let $E\subseteq\RR^n$ be a measurable set and $(f_m)\geq 0$ be a sequence of non-negative measurable functions over $E$. Then: $$\int_E\liminf_{m\to\infty}f_m(x)\dd{x}\leq \liminf_{m\to\infty}\int_Ef_m(x)\dd{x}$$
  \end{lemma}
  \begin{definition}
    Let $E\subseteq\RR^n$ be a measurable set and $f:E\rightarrow[-\infty,+\infty]$ be a measurable function such that either $\int_Ef^+(x)\dd{x}<\infty$ or $\int_Ef^-(x)\dd{x}<\infty$. Then, we define the \emph{integral of $f$ over $E$} as: $$\int_Ef(x)\dd{x}:=\int_Ef^+(x)\dd{x}-\int_Ef^-(x)\dd{x}$$
    We say that $f$ is an \emph{integrable function over $E$} if $${\norm{f}}_1:=\int_E\abs{f(x)}\dd{x}<\infty$$ The set of such functions is denoted by $\mathcal{L}^1(E)$.
  \end{definition}
  \begin{proposition}
    Let $E\subseteq\RR^n$ be a measurable set and $f:E\rightarrow[-\infty,+\infty]$ be a measurable function. Then, the function $g=f\indi{\abs{f}<\infty}$ is finite and $f\almoste{=}g$.
  \end{proposition}
  \begin{proposition}
    Let $E\subseteq\RR^n$ be a measurable set. Then:
    \begin{enumerate}
      \item $\mathcal{L}^1(E)$ is a vector spaceover $\RR$.
      \item The integral $$\function{\int_E}{\mathcal{L}^1(E)}{\RR}{f}{\int_Ef}$$ is a linear form.
      \item If $f,g\in\mathcal{L}^1(E)$are such that $f\almoste{leq} g$ on $E$, then $\int_E f\leq\int_E g$. Moreover: $$\abs{\int_Ef(x)\dd{x}}\leq \int_E\abs{f(x)}\dd{x}$$
      \item If $f\in\mathcal{L}^1(E)$ and $E=E_1\sqcup E_2$ with $E_1$, $E_2$ measurable, then $$\int_{E_1\sqcup E_2}f(x)\dd{x}=\int_{E_1}f(x)\dd{x}+\int_{E_2}f(x)\dd{x}$$
      \item If $h\in\RR^n$, $f\in\mathcal{L}^1(E)$, then: $$\int_{E-h}f(x+h)\dd{x}=\int_{-E}f(-x)\dd{x}=\int_{E}f(x)\dd{x}$$
    \end{enumerate}
  \end{proposition}
  \begin{theorem}[Dominated convergence theorem]
    Let $E\subseteq\RR^n$ be a measurable set, $f$ be a measurable function over $E$ such that $\exists (f_m)\geq 0$ with $f_m\almoste{\rightarrow} f$ and $\abs{f_m(x)}\almoste\leq g(x)$ on $E$ with $g\in\mathcal{L}^1(E)$. Then, $f, f_m\in\mathcal{L}^1(E)$ $\forall m\in\NN$ and: $$\int_Ef(x)\dd{x}=\lim_{m\to\infty}\int_Ef_m(x)\dd{x}$$
  \end{theorem}
  \begin{proposition}
    Let $E\subseteq\RR^n$ be a measurable set, $f,g\in\mathcal{L}^1(E)$ and $\lambda\in\RR$. Then:
    \begin{enumerate}
      \item ${\norm{f+g}}_1\leq{\norm{f}}_1+{\norm{g}}_1$
      \item ${\norm{\lambda f}}_1=\abs{\lambda}{\norm{f}}_1$
      \item ${\norm{f}}_1=0\iff f\almoste{=}0$.
    \end{enumerate}
  \end{proposition}
  \begin{definition}
    Let $E\subseteq\RR^n$ be a measurable set and $(f_m),f\in\mathcal{L}^1(E)$. We say that $(f_m)$ \emph{converge in mean} to $f$ if $\lim_{m\to\infty}{\norm{f_m-f}}_1=0$, or equivalently $\lim_{m\to\infty} f_m=f$ on $\mathcal{L}^1(E)$.
  \end{definition}
  \begin{theorem}
    Let $E\subseteq\RR^n$ be a measurable set and $(f_m)\in\mathcal{L}^1(E)$.
    \begin{enumerate}
      \item If $\sum_{m=1}^\infty {\norm{f_m}}_1<\infty$, $\exists f\in\mathcal{L}^1(E)$ such that $\sum_{m=1}^\infty f_m =f$ on $\mathcal{L}^1(E)$ and $\sum_{m=1}^\infty f_m(x)=f(x)$ converges absolutely $\forall x\in E\setminus N$, where $N$ is a null set.
      \item If $\lim_{m\to\infty} f_m=f$ on $\mathcal{L}^1(E)$, then there exists a subsequence $(f_{m_k})$ such that $\lim_{k\to\infty} f_{m_k}=f$ on $\mathcal{L}^1(E)$ and  $\lim_{k\to\infty} f_{m_k}(x)=f(x)$ $\forall x\in E\setminus N$, where $N$ is a null set.
    \end{enumerate}
  \end{theorem}
  \begin{proposition}
    Let $E\subseteq\RR^n$ be a measurable set and $f\in\mathcal{L}^1(E)$. Then, there exists a sequence of integrable simple functions $(s_m)$ such that $\lim_{m\to\infty} s_m=f$ on $\mathcal{L}^1(E)$ and $\lim_{m\to\infty} s_m(x)=f(x)$ $\forall x\in E$, with $\abs{s_1}\leq\abs{s_2}\leq \cdots\leq \abs{f}$.
  \end{proposition}
  \subsubsection{Integral calculus in one variable and Riemann integral}
  \begin{definition}
    Given a function $f:\RR\rightarrow\RR$, we say that $\int_a^bf(x)\dd{x}$ \emph{exists and it is finite} if$f$ is integrable on $(\min\{a,b\},\max\{a,b\})$\footnote{Note that if $f$ is measurable, the integral always exists but it may be $\pm\infty$.}.
  \end{definition}
  \begin{theorem}[Mean value theorem for integrals]
    Let $f:\RR\rightarrow\RR_{\geq 0}$ be an positive integrable function over $(a,b)$ and $g:(a,b)\rightarrow\RR$ be a measurable and bounded function such that $\alpha\leq g(x)\leq\beta$ almost everywhere on $(a,b)$. Then, $\exists\gamma\in[\alpha,\beta]$ such that: $$\int_a^bg(x)f(x)\dd{x}=\gamma\int_a^bf(x)\dd{x}$$
    Moreover if $g$ is continuous, $\exists\xi\in(a,b)$ such that: $$\int_a^bg(x)f(x)\dd{x}=g(\xi)\int_a^bf(x)\dd{x}$$
    In particular, taking $f=1$, we get:  $$\int_a^bg(x)\dd{x}=g(\xi)(b-a)$$
  \end{theorem}
  \begin{theorem}[Barrow's law]
    If $f:[a,b]\rightarrow\RR$ is a continuous function and derivable on $(a,b)$ with bounded derivative, then $f'\in\mathcal{L}^1((a,b))$ and $$\int_a^bf'(x)\dd{x}=f(b)-f(a)$$
  \end{theorem}
  \begin{theorem}[Fundamental theorem of calculus]
    Let $f:(a,b)\rightarrow\RR$ be an integrable function and continuous at $y\in (a,b)$. Then, $F(x)=\int_a^xf(x)\dd{x}$ is derivable at $y$ and $F'(y)=f(y)$.
  \end{theorem}
  \begin{theorem}
    Let $\varphi:[\alpha,\beta]\rightarrow[c,d]$ be a continuous function and derivable with bounded derivative on $(\alpha,\beta)$. Let $a=\varphi(\alpha)$ and $b=\varphi(\beta)$. If $f:[c,d]\rightarrow\RR$ is a continuous function, then $(f\circ\varphi)\varphi'$ is integrable on $(\alpha,\beta)$ and: $$\int_c^df(x)\dd{x}=\int_\alpha^\beta f(\varphi(t))\varphi'(t)\dd{t}$$
  \end{theorem}
  \begin{theorem}[Integration by parts]
    Let $F,G:[a,b]\rightarrow\RR$ be the primitives of the two bounded functions $f,g:(a,b)\rightarrow\RR$. Then, $Fg,fG\in\mathcal{L}^(a,b)$ and: $$\int_a^bF(x)g(x)\dd{x}=F(a)G(a)-F(b)G(b)-\int_a^bf(x)G(x)\dd{x}$$
  \end{theorem}
  \begin{theorem}
    Let $f:[a,b]\rightarrow\RR$ be a Riemann integrable function. Then, $f$ is also Lebesgue integrable and both integrals coincide.
  \end{theorem}
  \subsubsection{Functions defined by integrals}
  \begin{theorem}
    Let $E\subseteq\RR^n$ be a measurable set, $I\subseteq \RR$ be an interval, $g\in\mathcal{L}^1(E)$ be such that $g\geq 0$ and $f(\cdot, t)$ be an integrable function $\forall t\in I$. We denote: $$\Phi(t):=\int_Ef(x,t)\dd{x}$$
    \begin{enumerate}
      \item If $f(x,\cdot)$ is continuous on $t_0$ almost everywhere on $E$ and $\abs{f(x,t)}\almoste\leq g(x)$ $\forall t\in I$, then $\Phi$ is continuous at $t_0$.
      \item If $f(x,\cdot)$ is derivable on $t_0$ almost everywhere on $E$ and $$\abs{\pdv{f}{t}(x,t)}\almoste\leq g(x)$$ then the function $\pdv{f}{t}(x,\cdot)$ is integrable on $E$ and $$\Phi'(t_0)=\int_E\pdv{f}{t}(x,t_0)\dd{x}$$
    \end{enumerate}
  \end{theorem}
  \subsubsection{Fubini-Tonell theorem}
\end{multicols}
\end{document}