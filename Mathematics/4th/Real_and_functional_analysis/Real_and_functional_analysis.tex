\documentclass[../../../main_math.tex]{subfiles}


\begin{document}
\renewcommand{\col}{\ana}
\begin{multicols}{2}[\section{Real and functional analysis}]
  \subsection{Measure theorey and Lebesgue integral}
  \subsubsection{Measures}
  \begin{definition}[$\sigma$-algebra]
    Let $\Omega$ be a  set and $\Sigma\subseteq\mathcal{P}(\Omega)$. We say that $\Sigma$ is a \emph{$\sigma$-algebra} over $\Omega$ if:
    \begin{enumerate}
      \item $\Omega\in\Sigma$.
      \item If $A\in\Sigma$, then $A^c\in\Sigma$.
      \item If $A_1,A_2,\ldots\in\Sigma$, then: $$\bigcup_{n=1}^\infty A_n\in\Sigma$$
    \end{enumerate}
  \end{definition}
  \begin{proposition}
    Let $\Sigma$ be an $\sigma$-algebra over a set $\Omega$. Then:
    \begin{enumerate}
      \item $\varnothing\in\Sigma$.
      \item If $A,B\in\Sigma$, then $A\setminus B\in\Sigma$.
      \item If $A_1,A_2,\ldots\in\Sigma$, then: $$\bigcap_{n=1}^\infty A_n\in\Sigma$$
    \end{enumerate}
  \end{proposition}
  \begin{definition}[Measure]
    Let $\Sigma$ be a $\sigma$-algebra over a set $\Omega$. A \emph{measure} over $\Omega$ is any function $$\mu:\Sigma\longrightarrow[0,\infty]$$ satisfying the following properties:
    \begin{itemize}
      \item $\mu(\varnothing)=0$.
      \item \emph{$\sigma$-additivity}: If $\{A_n:n\geq1\}\subseteq\Sigma$ are pairwise disjoint, then: $$\mu\left(\bigsqcup_{n=1}^\infty A_n\right)=\sum_{n=1}^\infty \mu(A_n)$$
    \end{itemize}
  \end{definition}
  \begin{definition}
    Let $\Sigma$ be a set and $\{A_n:n\geq1\}\subseteq\Sigma$ be subsets. We say that $A_n\nearrow A$ if $A_n\subseteq A_{n+1}$ $\forall n\in\NN$ and $A=\bigcup_{n=1}^\infty A_n$. Analogously, we say that $A_n\searrow A$ if $A_n\supseteq A_{n+1}$ $\forall n\in\NN$ and $A=\bigcap_{n=1}^\infty A_n$.
  \end{definition}
  \begin{proposition}
    Let $\Sigma$ be a $\sigma$-algebra over a set $\Omega$, $\mu:\Sigma\longrightarrow[0,\infty]$ be a measure over $\Omega$ and $A_n,A,B\in\Sigma$, $n\in\NN$. Then:
    \begin{itemize}
      \item If $A\subseteq B$, then $\mu(B\setminus A)=\mu(B)-\mu(A)$.
      \item If $A\subseteq B$, then $\mu(A)\subseteq\mu(B)$.
      \item If $A_n\nearrow A$, then $\displaystyle\mu(A)=\lim_{n\to\infty} A_n$.
      \item If $A_n\searrow A$ and $\mu(A_1)<\infty$, then $\displaystyle\mu(A)=\lim_{n\to\infty} A_n$.
    \end{itemize}
  \end{proposition}
  \begin{definition}
    An \emph{interval} $I\subseteq\RR^n$ is a set of the form:
    $$I=\abs{a_1,b_1}\times\cdots\times\abs{a_n,b_n}$$
    where $a_i,b_i\in\RR_\infty$ and the notation $\abs{a,b}$ represents either $(a,b)$, $[a,b)$, $(a,b]$ or $[a,b]$.
  \end{definition}
  \begin{definition}
    Let $I=\prod_{i=1}^n\abs{a_i,b_i}\subseteq\RR^n$ be an interval. We define its \emph{volume} as:
    $$\vol(I):=\prod_{i=1}^n(b_i-a_i)$$
  \end{definition}
  \begin{definition}
    Let $m\in\NN\cup\{0\}$. We define the \emph{$m$-th dyadic cube} as the set: $$\mathcal{D}_m:=[a_1,a_1+2^{-m})\times\cdots\times [a_n,a_n+2^{-m})$$
    where $a_i\in 2^{-m}\ZZ$\footnote{Note that each $\mathcal{D}_m$ forms a partition of $\RR^n$.}.
  \end{definition}
  \begin{lemma}
    Let $m\in\NN\cup\{0\}$. Then the sidelength of $\mathcal{D}_m$ is $2^{-m}$, $\vol(\mathcal{D}_m)= 2^{-mn}$ and its diameter is $2^{-m}\sqrt{n}$.
  \end{lemma}
  \begin{proposition}
    Any nonempty open set $U\subseteq\RR^n$ can be written as a countable union of disjoint dyadic cubes whose closure is in $U$.
  \end{proposition}
  \begin{definition}
    Let $A\subseteq\RR^n$ be a set. We denote by $\mathcal{I}(A)$ the set of covers of $A$ that are intervals. Analogously, we denote by $\mathcal{I}_0(A)$ the set of open covers of $A$ that are intervals.
  \end{definition}
  \begin{definition}[Outer measure]
    Let $A\subseteq\RR^n$ be a set. We define its \emph{outer measure} as the function $\om{}$ defined by:
    $$\om{A}:=\inf\left\{\sum_{k= 1}^\infty \vol(I_k):\{I_k:k\geq 1\}\in \mathcal{I}(A)\right\}$$
  \end{definition}
  \begin{proposition}
    Let $A\subseteq\RR^n$ be a set. Then:
    $$\om{A}=\inf\left\{\sum_{k= 1}^\infty \vol(I_k):\{I_k:k\geq 1\}\in \mathcal{I}_0(A)\right\}$$
  \end{proposition}
  \begin{lemma}
    Let $I,J_1,\ldots,J_N\subseteq \RR^n$ be intervals such that $I\subseteq \bigcup_{k=1}^N J_k$. Then, $\vol(I)\leq\sum_{k=1}^N \vol(J_k)$.
  \end{lemma}
  \begin{theorem}
    The outer measure has the following properties:
    \begin{enumerate}
      \item $\om{\varnothing}=0$.
      \item If $A\subseteq B\subseteq\RR^n$, then $\om{A}\leq \om{B}$.
      \item If $A_1, A_2,\ldots \subseteq \RR^n$, then: $$\om{\bigcup_{k=1}^\infty A_k}\leq \sum_{k=1}^\infty \om{A_k}$$
      \item If $I\subseteq \RR^n$ is an open interval and $I\subseteq A\subseteq \cl{I}$, then $\om{A}=\vol(I)$.
      \item If $I_1,\ldots,I_N\subseteq \RR^n$ are disjoint intervals, then: $$\displaystyle \om{\bigsqcup_{k=1}^N I_k}= \sum_{k=1}^N \vol(I_k)$$
      \item If $A,B\subseteq \RR^n$ and $d(A,B):=\inf\{d(a,b):a\in A,b\in B\}>0$, then $\om{A\sqcup B}=\om{A}+\om{B}$.
      \item If $A\subseteq\RR^n$ and $x\in\RR^n$, then $\om{A+x}=\om{-A}=\om{A}$\footnote{Here $A+x:=\{a+x:a\in A\}$ and $-A:=\{-a:a\in A\}$}.
    \end{enumerate}
  \end{theorem}
  \begin{definition}
    A set $N\subset\RR^n$ is called a \emph{null set} if $\om{N}=0$.
  \end{definition}
  \begin{definition}
    We say that a property holds \emph{almost everywhere} (\emph{a.e.}) if the set of points that doesn't hold it is null.
  \end{definition}
  \begin{lemma}
    The countable union of null sets is null.
  \end{lemma}
  \begin{lemma}
    A point is null. Therefore, all countable sets are null.
  \end{lemma}
  \subsubsection{Lebesgue measure}
  \begin{definition}[Lebesgue measure]
    We say that $A\subseteq\RR^n$ is \emph{Lebesgue measurable} (or simply \emph{measurable}) if $\forall \varepsilon>0$, there exists an open set $U\supseteq A$ such that $\om{U\setminus A}<\varepsilon$. We denote by $\mathcal{M}(\RR^n)$ the set of all Lebesgue measurable sets of $\RR^n$ and by $\m{}$ the restriction of $\om{}$ to ${\mathcal{M}(\RR^n)}$.
  \end{definition}
  \begin{theorem}
    $\mathcal{M}(\RR^n)$ is a $\sigma$-algebra and $m:\mathcal{M}(\RR^n)\rightarrow[0,\infty]$ is a measure (called \emph{Lebesgue measure}) that satisfies:
    \begin{enumerate}
      \item The open sets, closed sets and null sets are in $\mathcal{M}(\RR^n)$.
      \item Each interval $I\subseteq \RR^n$ is measurable and $\m{I}=\vol(I)$.
      \item If $A\in\mathcal{M}(\RR^n)$ and $x\in\RR^n$, then $A+x,-A\in\mathcal{M}(\RR^n)$ and $\m{A+x}=\m{-A}=\m{A}$.
      \item If $A\in\mathcal{M}(\RR^n)$:
            \begin{align*}
              \m{A} & =\inf\{\m{U}:A\subseteq U \subseteq\RR^n, U\text{ open}\}  \\
                    & =\sup\{\m{C}:C\subseteq A\subseteq\RR^n, C\text{ closed}\}
            \end{align*}
    \end{enumerate}
  \end{theorem}
  \begin{definition}
    A \emph{real function} is a function $f:\RR^n\rightarrow[-\infty,+\infty]$. We will say that $f$ is \emph{finite} if $\pm\infty\notin\im f$.
  \end{definition}
  \begin{definition}
    Let $f$ be a real function. We say that $f$ is \emph{Lebesgue measurable} (or simply \emph{measurable}) if $\{x\in\RR^n:f(x)> r\}\in \mathcal{M}(\RR^n)$ $\forall r\in\RR$.
  \end{definition}
  \begin{lemma}
    Let $a,b\in[-\infty,+\infty]$ and $f$ be a real function. The sets:
    \begin{itemize}
      \item $\{a<f(x)<b:x\in\RR^n\}$
      \item $\{a\leq f(x)<b:x\in\RR^n\}$
      \item $\{a<f(x)\leq b:x\in\RR^n\}$
      \item $\{a\leq f(x)\leq b:x\in\RR^n\}$
    \end{itemize}
    are all measurable.
  \end{lemma}
  \begin{proposition}
    A function $f:\RR^n\rightarrow\RR$ is measurable if and only if for all open set $U\subseteq \RR$, $f^{-1}(U)\in\mathcal{M}(\RR^n)$.
  \end{proposition}
  \begin{proposition}
    Let $f$ be a finite measurable real function, $U\subseteq\RR$ be an open set such that $\im f\subseteq U$ and $\varphi:U\rightarrow\RR$ be a continuous function. Then, $\varphi\circ f$ is also measurable.
  \end{proposition}
  \begin{proposition}
    Let $u$, $v$ be two finite measurable real functions, $U\subseteq\RR^2$ be an open set such that $(u(x),v(x))\in U$ $\forall x\in\RR^n$ and $\varphi:U\rightarrow\RR$ be a continuous function. Then, $\varphi(u(x),v(x))$ is also measurable.
  \end{proposition}
  \begin{proposition}
    Let $f$, $g$ be two real functions such that $f$ is measurable and $g\almoste{=} f$. Then, $g$ is also measurable.
  \end{proposition}
  \begin{proposition}
    Let $f$, $g$ be two measurable real functions. Then, so are $f\pm g$, $fg$ and $f/g$ if $g(x)\ne 0$ $\forall x\in\RR^n$.
  \end{proposition}
  \begin{proposition}
    Let $(f_n)$ be a sequence of measurable real functions. Then, the following sets are measurable:
    \begin{itemize}
      \item $\displaystyle\sup\{f_n:n\in\NN\}$
      \item $\displaystyle\inf\{f_n:n\in\NN\}$
      \item $\displaystyle\limsup_{n\to\infty} f_n$
      \item $\displaystyle\liminf_{n\to\infty} f_n$
    \end{itemize}
    Furthermore, any function being pointwise limit a.e. of a sequence of measurable functions is measurable.
  \end{proposition}
  \begin{definition}
    The \emph{Borel $\sigma$-algebra} over $\RR^n$, $\mathcal{B}(\RR^n)$, is the smallest $\sigma$-algebra that contains the open sets of $\RR^n$.
  \end{definition}
  \begin{lemma}
    We have that $\mathcal{B}(\RR^n)\subset \mathcal{M}(\RR^n)$.
  \end{lemma}
  \begin{definition}
    A function $g:\RR\rightarrow\RR$ is \emph{Borel measurable} if $\{x\in\RR^n:g(x)> r\}\in \mathcal{B}(\RR^n)$ $\forall r\in\RR$.
  \end{definition}
  \begin{proposition}
    Let $f:\RR\rightarrow\RR$ be a Lebesgue measurable function and $f:\RR\rightarrow\RR$ be a Borel measurable function. Then, $g\circ f$ is Lebesgue measurable.
  \end{proposition}
  \begin{definition}
    Let $f$ be a measurable function. We define the following measurable functions:
    $$f^+:=\sup\{f,0\}\qquad f^-:=\sup\{-f,0\}$$
    Note that then, $f=f^+-f^-$ and $\abs{f}=f^++f^-$.
  \end{definition}
  \begin{definition}
    A \emph{simple function} is a linear combination $$s:=\sum_{k=1}^N\alpha_k\indi{A_k}$$ where $\alpha_k\in\RR$ and $A_k\in\mathcal{M}(\RR^n)$ for $k=1,\ldots,N$\footnote{We may suppose that the sets $A_k$ are pairwise disjoint, the quantities $\alpha_k$ are all different and that $A_k=s^{-\alpha_k}$.}.
  \end{definition}
  \begin{theorem}
    Let $f:\RR^n\rightarrow[0,+\infty]$ be a measurable function and let $$E(k,m):=\left\{\frac{k-1}{2^m}\leq f<\frac{k}{2^m}\right\}\;\;\text{and}\;\; F(m):=\{f\geq m\}$$
    Then, the sequence of positive simple functions $$s_m=m+\indi{F(k)}+\sum_{k=1}^{m2^m}\frac{k-1}{2^m}\indi{E(k,m)}$$ is increasing and $\displaystyle\lim_{m\to\infty}s_m(x)=f(x)$ $\forall x\in\RR^n$.
  \end{theorem}
  \begin{theorem}
    Let $f:\RR^n\rightarrow[-\infty,+\infty]$ be a measurable function. Then, there exists a sequence of simple functions $(s_m)$ such that $\displaystyle\lim_{m\to\infty}s_m(x)=f(x)$ $\forall x\in\RR^n$ and $\abs{s_m}\leq\abs{s_{m+1}}\leq \abs{f}$ $\forall m\in\NN$.
  \end{theorem}
  \subsubsection{Lebesgue integral}
  \begin{definition}
    Let $N\in\NN$, $E_1,\ldots,E_N$ be measurable sets and $s=\sum_{k=1}^N\alpha_k\indi{E_k}$ be a positive simple function such that $0<\alpha_1<\cdots<\alpha_N<\infty$. We define the \emph{integral of $s$ over $\RR^n$} as: $$\int s:=\sum_{k=1}^N\alpha_k\m{E_k}$$
    We define the \emph{integral of $s$ over a measurable set $E$} as: $$\int_E s:=\int s\indi{E}=\sum_{k=1}^N\alpha_k\m{E_k\cap E}$$
  \end{definition}
  \begin{proposition}
    Let $E_1,E_2,\ldots$ be measurable sets and $s$, $t$ be simple functions. Then:
    \begin{itemize}
      \item If $\displaystyle E=\bigsqcup_{k=1}^\infty E_k$, then $\displaystyle\int_E s=\sum_{k=1}^\infty \int_{E_k} s$.
      \item $\displaystyle\int(s+t)=\int s+\int t$.
      \item If $\alpha\in\RR_{\geq 0}$, then $\displaystyle\int \alpha s=\alpha\int s$.
      \item If $s\leq t$, then $\displaystyle\int s\leq\int t$.
    \end{itemize}
  \end{proposition}
  \begin{proposition}
    The function $$\function{\mu_s}{\mathcal{M}(\RR^n)}{[0,\infty]}{E}{\displaystyle\int_E s}$$ is a measure.
  \end{proposition}
  \begin{definition}
    Let $f:\RR^n\rightarrow[0,+\infty]$ be a measurable function. We define: $$\mathcal{S}(f):=\{s:s\text{ is a simple function such that }0\leq s\leq f\}$$
  \end{definition}
  \begin{definition}
    Let $f:\RR^n\rightarrow[0,+\infty]$ be a measurable function. We define the \emph{integral of $f$ over $\RR^n$} as: $$\int_{\RR^n}f(x)\dd{x}:=\sup_{s\in\mathcal{S}(f)}\int s$$ We define the \emph{integral of $f$ over a measurable set $E\subseteq \RR^n$} as: $$\int_{E}f(x)\dd{x}:=\int_{\RR^n}f(x)\indi{E}(x)\dd{x}=\sup_{s\in\mathcal{S}(f\indi{E})}\int s$$
  \end{definition}
  \begin{proposition}
    Let $E\subseteq\RR^n$ be a measurable set, $s$ be a simple function and $f$, $g$ be measurable functions such that $f(x)\leq g(x)$ $\forall x\in E$. Then:
    \begin{enumerate}
      \item $\displaystyle\int_E s=\int_Es(x)\dd{x}$
      \item $\displaystyle\int_E f(x)\dd{x}\leq \int_Eg(x)\dd{x}$
    \end{enumerate}
  \end{proposition}
  \begin{theorem}[Monotone convergence theorem]
    Let $E\subseteq\RR^n$ be a measurable set, $f\geq 0$ be a non-negative measurable function such that $\exists (f_m)\geq 0$ with $f_m\nearrow f$. Then: $$\int_Ef(x)\dd{x}=\lim_{m\to\infty}\int_Ef_m(x)\dd{x}$$
  \end{theorem}
  \begin{proposition}
    Let $E\subseteq\RR^n$ be a measurable set, $f, g, (f_m)\geq 0$ be non-negative measurable functions. Then:
    \begin{enumerate}
      \item $\displaystyle\int_E (f+g)(x)\dd{x}=\int_Ef(x)\dd{x}+\int_Eg(x)\dd{x}$
      \item $\displaystyle\int_E\sum_{m=1}^\infty f_m(x)\dd{x}=\sum_{m=1}^\infty\int_E f_m(x)\dd{x}$
      \item If $(E_k)$ is a sequence of measurable sets such that $E=\bigsqcup_{k=1}^\infty E_k$, then: $$\int_Ef(x)\dd{x}=\sum_{k=1}^\infty\int_{E_k}f(x)\dd{x}$$
      \item If $\alpha\in[0,\infty)$, then $\displaystyle\int_E\alpha f(x)\dd{x}=\alpha\int_Ef(x)\dd{x}$.
      \item If $f\leq g$ on $E$, then $\displaystyle\int_Ef(x)\dd{x}\leq\int_Eg(x)\dd{x}$.
      \item $\displaystyle\int_Ef(x)\dd{x}=0\iff f\almoste{=}0$ on $E$.
      \item If $N\subset E$ is a null set, then $\displaystyle\int_Ef(x)\dd{x}=\int_{E\setminus N}f(x)\dd{x}$.
      \item If $\displaystyle\int_Ef(x)\dd{x}<\infty$, then $f\almoste{<}\infty$ on $E$.
      \item If $h\in\RR^n$, then $\displaystyle\int_{E-h}f(x+h)\dd{x}=\int_{-E}f(-x)\dd{x}=\int_{E}f(x)\dd{x}$
    \end{enumerate}
  \end{proposition}
  \begin{corollary}
    Let $E\subseteq\RR^n$ be a measurable set, $f\geq 0$ be a non-negative measurable function such that $\exists (f_m)\geq 0$ with $f_m\almoste{\nearrow} f$. Then: $$\int_Ef(x)\dd{x}=\lim_{m\to\infty}\int_Ef_m(x)\dd{x}$$
  \end{corollary}
  \begin{theorem}[Chebyshev's inequality]
    Let $E\subseteq\RR^n$ be a measurable set, $f:E\rightarrow\RR$ be a measurable function and $\alpha\in\RR_{>0}$. Then: $$\abs{\{x\in E:\abs{f(x)}\geq\alpha\}}\leq\frac{1}{\alpha}\int_E \abs{f(x)}\dd{x}$$
  \end{theorem}
  \begin{lemma}[Fatou's lemma]
    Let $E\subseteq\RR^n$ be a measurable set and $(f_m)\geq 0$ be a sequence of non-negative measurable functions over $E$. Then: $$\int_E\liminf_{m\to\infty}f_m(x)\dd{x}\leq \liminf_{m\to\infty}\int_Ef_m(x)\dd{x}$$
  \end{lemma}
  \begin{definition}
    Let $E\subseteq\RR^n$ be a measurable set and $f:E\rightarrow[-\infty,+\infty]$ be a measurable function such that either $\int_Ef^+(x)\dd{x}<\infty$ or $\int_Ef^-(x)\dd{x}<\infty$. Then, we define the \emph{integral of $f$ over $E$} as: $$\int_Ef(x)\dd{x}:=\int_Ef^+(x)\dd{x}-\int_Ef^-(x)\dd{x}$$
    We say that $f$ is an \emph{integrable function over $E$} if $${\norm{f}}_1:=\int_E\abs{f(x)}\dd{x}<\infty$$ The set of such functions is denoted by $ \mathcal{L}^1(E)$.
  \end{definition}
  \begin{proposition}
    Let $E\subseteq\RR^n$ be a measurable set and $f:E\rightarrow[-\infty,+\infty]$ be a measurable function. Then, the function $g=f\indi{\abs{f}<\infty}$ is finite and $f\almoste{=}g$.
  \end{proposition}
  \begin{proposition}
    Let $E\subseteq\RR^n$ be a measurable set. Then:
    \begin{enumerate}
      \item $ \mathcal{L}^1(E)$ is a vector space over $\RR$.
      \item The integral $$\function{\int_E}{ \mathcal{L}^1(E)}{\RR}{f}{\int_Ef}$$ is a linear form.
      \item If $f,g\in \mathcal{L}^1(E)$ are such that $f\almoste{\leq} g$ on $E$, then $\int_E f\leq\int_E g$. Moreover: $$\abs{\int_Ef(x)\dd{x}}\leq \int_E\abs{f(x)}\dd{x}$$
      \item If $f\in \mathcal{L}^1(E)$ and $E=E_1\sqcup E_2$ with $E_1$, $E_2$ measurable, then $$\int_{E_1\sqcup E_2}f(x)\dd{x}=\int_{E_1}f(x)\dd{x}+\int_{E_2}f(x)\dd{x}$$
      \item If $h\in\RR^n$, $f\in \mathcal{L}^1(E)$, then: $$\int_{E-h}f(x+h)\dd{x}=\int_{-E}f(-x)\dd{x}=\int_{E}f(x)\dd{x}$$
    \end{enumerate}
  \end{proposition}
  \begin{theorem}[Dominated convergence theorem]
    Let $E\subseteq\RR^n$ be a measurable set, $f$ be a measurable function over $E$ such that $\exists (f_m)\geq 0$ with $f_m\almoste{\rightarrow} f$ and $\abs{f_m(x)}\almoste\leq g(x)$ on $E$ with $g\in \mathcal{L}^1(E)$ $\forall m\in\NN$. Then, $f, f_m\in \mathcal{L}^1(E)$ $\forall m\in\NN$ and: $$\int_Ef(x)\dd{x}=\lim_{m\to\infty}\int_Ef_m(x)\dd{x}$$
  \end{theorem}
  \begin{proposition}
    Let $E\subseteq\RR^n$ be a measurable set, $f,g\in \mathcal{L}^1(E)$ and $\lambda\in\RR$. Then:
    \begin{enumerate}
      \item ${\norm{f+g}}_1\leq{\norm{f}}_1+{\norm{g}}_1$
      \item ${\norm{\lambda f}}_1=\abs{\lambda}{\norm{f}}_1$
      \item ${\norm{f}}_1=0\iff f\almoste{=}0$.
    \end{enumerate}
  \end{proposition}
  \begin{definition}
    Let $E\subseteq\RR^n$ be a measurable set and $(f_m),f\in \mathcal{L}^1(E)$. We say that $(f_m)$ \emph{converge in mean} to $f$ if $\displaystyle\lim_{m\to\infty}{\norm{f_m-f}}_1=0$, or equivalently $\displaystyle\lim_{m\to\infty} f_m=f$ on $ \mathcal{L}^1(E)$.
  \end{definition}
  \begin{theorem}
    Let $E\subseteq\RR^n$ be a measurable set and $(f_m)\in \mathcal{L}^1(E)$.
    \begin{enumerate}
      \item If $\sum_{m=1}^\infty {\norm{f_m}}_1<\infty$, $\exists f\in \mathcal{L}^1(E)$ such that $\sum_{m=1}^\infty f_m =f$ on $ \mathcal{L}^1(E)$ and $\sum_{m=1}^\infty f_m(x)=f(x)$ converges absolutely $\forall x\in E\setminus N$, where $N$ is a null set.
      \item If $\displaystyle\lim_{m\to\infty} f_m=f$ on $ \mathcal{L}^1(E)$, then there exists a subsequence $(f_{m_k})$ such that $\displaystyle\lim_{k\to\infty} f_{m_k}=f$ on $ \mathcal{L}^1(E)$ and  $\displaystyle\lim_{k\to\infty} f_{m_k}(x)=f(x)$ $\forall x\in E\setminus N$, where $N$ is a null set.
    \end{enumerate}
  \end{theorem}
  \begin{proposition}
    Let $E\subseteq\RR^n$ be a measurable set and $f\in \mathcal{L}^1(E)$. Then, there exists a sequence of integrable simple functions $(s_m)$ such that $\displaystyle\lim_{m\to\infty} s_m=f$ on $ \mathcal{L}^1(E)$, $\displaystyle\lim_{m\to\infty} s_m(x)=f(x)$ $\forall x\in E$ and $\abs{s_m}\leq\abs{s_{m+1}}\leq \abs{f}$ $\forall m\in\NN$.
  \end{proposition}
  \subsubsection{Integral calculus in one variable and Riemann integral}
  \begin{definition}
    Given a function $f:\RR\rightarrow\RR$, we say that $\int_a^bf(x)\dd{x}$ \emph{exists and it is finite} if $f$ is integrable on $(\min\{a,b\},\max\{a,b\})$\footnote{Note that if $f$ is measurable, the integral always exists but it may be $\pm\infty$.}.
  \end{definition}
  \begin{theorem}[Mean value theorem for integrals]
    Let $f:\RR\rightarrow\RR_{\geq 0}$ be an positive integrable function over $(a,b)$ and $g:(a,b)\rightarrow\RR$ be a measurable and bounded function such that $\alpha\leq g(x)\leq\beta$ almost everywhere on $(a,b)$. Then, $\exists\gamma\in[\alpha,\beta]$ such that: $$\int_a^bg(x)f(x)\dd{x}=\gamma\int_a^bf(x)\dd{x}$$
    Moreover if $g$ is continuous, $\exists\xi\in(a,b)$ such that: $$\int_a^bg(x)f(x)\dd{x}=g(\xi)\int_a^bf(x)\dd{x}$$
    In particular, taking $f=1$, we get:  $$\int_a^bg(x)\dd{x}=g(\xi)(b-a)$$
  \end{theorem}
  \begin{theorem}[Barrow's law]
    If $f:[a,b]\rightarrow\RR$ is a continuous function and derivable on $(a,b)$ with bounded derivative, then $f'\in \mathcal{L}^1((a,b))$ and $$\int_a^bf'(x)\dd{x}=f(b)-f(a)$$
  \end{theorem}
  \begin{theorem}[Fundamental theorem of calculus]
    Let $f:(a,b)\rightarrow\RR$ be an integrable function and continuous at $y\in (a,b)$. Then, $F(x)=\int_a^xf(s)\dd{s}$ is derivable at $y$ and $F'(y)=f(y)$.
  \end{theorem}
  \begin{theorem}
    Let $\varphi:[\alpha,\beta]\rightarrow[c,d]$ be a continuous function and derivable with bounded derivative on $(\alpha,\beta)$. Let $a=\varphi(\alpha)$ and $b=\varphi(\beta)$. If $f:[c,d]\rightarrow\RR$ is a continuous function, then $(f\circ\varphi)\varphi'$ is integrable on $(\alpha,\beta)$ and: $$\int_c^df(x)\dd{x}=\int_\alpha^\beta f(\varphi(t))\varphi'(t)\dd{t}$$
  \end{theorem}
  \begin{theorem}[Integration by parts]
    Let $F,G:[a,b]\rightarrow\RR$ be the primitives of the two bounded functions $f,g:(a,b)\rightarrow\RR$. Then, $Fg,fG\in \mathcal{L}^1(a,b)$ and: $$\int_a^bF(x)g(x)\dd{x}=F(a)G(a)-F(b)G(b)-\int_a^bf(x)G(x)\dd{x}$$
  \end{theorem}
  \begin{theorem}
    Let $f:[a,b]\rightarrow\RR$ be a Riemann integrable function. Then, $f$ is also Lebesgue integrable and both integrals coincide.
  \end{theorem}
  \subsubsection{Functions defined by integrals}
  \begin{theorem}
    Let $E\subseteq\RR^n$ be a measurable set, $I\subseteq \RR$ be an interval, $g\in \mathcal{L}^1(E)$ be such that $g\geq 0$ and $f(\cdot, t)$ be an integrable function $\forall t\in I$. We denote: $$\Phi(t):=\int_Ef(x,t)\dd{x}$$
    \begin{enumerate}
      \item If $f(x,\cdot)$ is continuous on $t_0$ almost everywhere on $E$ and $\abs{f(x,t)}\almoste\leq g(x)$ $\forall t\in I$, then $\Phi$ is continuous at $t_0$.
      \item If $f(x,\cdot)$ is derivable on $t_0$ almost everywhere on $E$ and $$\abs{\pdv{f}{t}(x,t)}\almoste\leq g(x)$$ then the function $\pdv{f}{t}(x,\cdot)$ is integrable on $E$ and $$\Phi'(t_0)=\int_E\pdv{f}{t}(x,t_0)\dd{x}$$
    \end{enumerate}
  \end{theorem}
  \begin{definition}[Hardy-Littlewood maximal function]
    Let $f\in \mathcal{L}^1(\RR^n)$ and $B\subseteq \RR^n$ be a ball. We define the \emph{Hardy-Littlewood maximal function} as: $$Mf(x)=\sup_{x\in B}\frac{1}{\m{B}}\int_B\abs{f(y)}\dd{y}$$
  \end{definition}
  \begin{theorem}
    Let $f\in \mathcal{L}^1(\RR^n)$ and $B\subseteq \RR^n$ be a ball. Then:
    \begin{enumerate}
      \item $Mf$ is measurable.
      \item $Mf\almoste{<}\infty$.
      \item $\displaystyle\m{x\in\RR^n : (Mf)(x) > \alpha} \leq \frac{A}{\alpha}\int_{\mathbf{R}^n} |f(x)|\dd{x}$.
    \end{enumerate}
  \end{theorem}
  \begin{theorem}[Lebesgue differentiation theorem]
    Let $f\in \mathcal{L}^1(\RR^n)$ and $B\subseteq \RR^n$ be a ball. Then: $$\lim_{\m{B}\to 0}\frac{1}{\m{B}}\int_Bf(y)\dd{y} \almoste{=} f(x) \qquad x\in  B$$
  \end{theorem}
  \begin{proposition}
    Let $f\in \mathcal{L}^1(\RR^n)$ and suppose $Mf\in \mathcal{L}^1(\RR^n)$. Then, $f=0$.
  \end{proposition}
  \subsubsection{Fubini-Tonell theorem}
  \begin{definition}
    Let $E\subseteq\RR^{p+q}$ and $y\in\RR^q$. We define the \emph{section} of $E$ at $y$ as: $$E(y):=\{x\in\RR^p:(x,y\in E)\}$$
  \end{definition}
  \begin{proposition}
    Let $E,F,E_k\subseteq\RR^{p+q}$, $k\in\NN$, and $y\in\RR^q$. Then:
    \begin{enumerate}
      \item If $E= A\times B$, with $A\subseteq \RR^p$ and $B\subseteq \RR^q$, then $E(y)=A$ if $y\in B$ and $E(y)=\varepsilon$ if $y\notin B$.
      \item $E\cap F=\varnothing\implies E(y)\cap F(y)=\varnothing$.
      \item \hfill\begin{enumerate}
              \item $\displaystyle\left(\bigcap_{k=1}^\infty E_k\right)(y)=\bigcap_{k=1}^\infty E_k(y)$
              \item $\displaystyle\left(\bigcup_{k=1}^\infty E_k\right)(y)=\bigcup_{k=1}^\infty E_k(y)$
              \item $\displaystyle\left(E\setminus F\right)(y)=E(y)\setminus F(y)$
            \end{enumerate}
      \item If $E(y)$ is measurable, then: $$\m{E(y)}=\int_{\RR^p}\indi{E}(x,y)\dd{x}$$ In particular, if $E$ is an interval $E=I_p\times I_q$, then: $$\m{I(y)}=\m{I_p}\indi{I_q}(y)$$
    \end{enumerate}
  \end{proposition}
  \begin{lemma}
    Let $E\subseteq\RR^{p+q}$ be a measurable set. Then:
    \begin{enumerate}
      \item There exits a null set $N\subset \RR^q$ such that $E(y)$ is measurable $\forall y\in\RR^q\setminus N$ (that is $E(y)$ is measurable almost everywhere $\forall y\in\RR^q$).
      \item The function $$\Phi(y)=\begin{cases}
                \m{E(y)} & \text{if } y\in\RR^q\setminus N \\
                0        & \text{if } y\in N
              \end{cases}$$ is measurable and positive on $\RR^q$.
      \item $\displaystyle\m{E}=\int_{\RR^q}\m{E(y)}\dd{y}$
    \end{enumerate}
  \end{lemma}
  \begin{theorem}[Tonelli's theorem]
    Let $f:\RR^{p+q}\rightarrow[0,\infty]$ be a non-negative measurable function. Then:
    \begin{enumerate}
      \item $f(\cdot,y)$ and $f(x,\cdot)$ are measurable almost everywhere $x\in\RR^p$, $y\in\RR^q$.
      \item Let $N_p$ and $N_q$ be the respective null sets where the above functions aren't measurable. Then the functions
            \begin{align*}
              \Phi(y) & =\begin{cases}
                           \int_{\RR^p}f(x,y)\dd{x} & \text{if } y\in\RR^q\setminus N_q \\
                           0                        & \text{if } y\in N_q
                         \end{cases} \\ \Psi(x)&=\begin{cases}
                \int_{\RR^q}f(x,y)\dd{y} & \text{if } x\in\RR^p\setminus N_p \\
                0                        & \text{if } x\in N_p
              \end{cases}
            \end{align*}
            are measurable on $\RR^q$ and $\RR^x$, respectively.
      \item \hfill $$\int_{\RR^q}\Phi(y)\dd{y}=\int_{\RR^{p+q}}f(x,y)\dd{(x,y)}=\int_{\RR^p}\Psi(x)\dd{x}$$
    \end{enumerate}
  \end{theorem}
  \begin{corollary}
    Let $f:\RR^{p+q}\rightarrow[0,\infty]$ be a non-negative measurable function. Then:
    \begin{align*}
      \int_{\RR^{p+q}}f(x,y)\dd{(x,y)} & =\int_{\RR^q}\left(\int_{\RR^p}f(x,y)\dd{x}\right)\dd{y} \\
                                       & =\int_{\RR^p}\left(\int_{\RR^q}f(x,y)\dd{y}\right)\dd{x}
    \end{align*}
    These identities are sometimes written as:
    \begin{align*}
      \int_{\RR^{p+q}}f(x,y)\dd{x}\dd{y} & =\int_{\RR^q}\dd{y}\int_{\RR^p}f(x,y)\dd{x} \\
                                         & =\int_{\RR^p}\dd{x}\int_{\RR^q}f(x,y)\dd{y}
    \end{align*}
  \end{corollary}
  \begin{corollary}
    Let $f:\RR^{p+q}\rightarrow\RR$ be a measurable function. Then, $f$ is integrable if and only if: $$\int_{\RR^q}\dd{y}\int_{\RR^p}\abs{f(x,y)}\dd{x}<\infty$$
  \end{corollary}
  \begin{theorem}[Fubini's theorem]
    Let $f\in \mathcal{L}^1(\RR^{p+q})$. Then:
    \begin{enumerate}
      \item $f(\cdot,y)\overset{\text{a.e.}}{\in} \mathcal{L}^1(\RR^p)$ and $f(x,\cdot)\overset{\text{a.e.}}{\in}\mathcal{L}^1(\RR^q)$, $x\in\RR^p$, $y\in\RR^q$.
      \item Let $N_p$ and $N_q$ be the respective null sets where the above functions aren't integrable. Then the functions
            \begin{align*}
              \Phi(y) & =\begin{cases}
                           \int_{\RR^p}f(x,y)\dd{x} & \text{if } y\in\RR^q\setminus N_q \\
                           0                        & \text{if } y\in N_q
                         \end{cases} \\ \Psi(x)&=\begin{cases}
                \int_{\RR^q}f(x,y)\dd{y} & \text{if } x\in\RR^p\setminus N_p \\
                0                        & \text{if } x\in N_p
              \end{cases}
            \end{align*}
            are integrable on $\RR^q$ and $\RR^p$, respectively.
      \item \hfill $$\int_{\RR^q}\Phi(y)\dd{y}=\int_{\RR^{p+q}}f(x,y)\dd{(x,y)}=\int_{\RR^p}\Psi(x)\dd{x}$$
    \end{enumerate}
  \end{theorem}
  \subsubsection{Change of variables}
  \begin{definition}
    Let $U,V\subseteq\RR^n$ be open sets. A \emph{change of variables} is a diffeomorphism $\vf\varphi:U\rightarrow V$ of class $\mathcal{C}^1$.
  \end{definition}
  \begin{theorem}[Change of variables]
    Let $U,V\subseteq\RR^n$ be open sets and $\vf\varphi:U\rightarrow V$ be a change of variables. If $f:\RR^n\rightarrow[0,\infty]$ is measurable or integrable on $V$, then so is $(f\circ\vf\varphi)\abs{J\vf\varphi}$ and: $$\int_{V} f(x)\dd{x}=\int_Uf(\vf\varphi(t))\abs{J\vf\varphi(t)}\dd{t}$$
  \end{theorem}

  \subsection{Functional analysis}
  \subsubsection{Normed vector spaces}
  \begin{definition}
    Let $M$ be a set. A \emph{distance} in $M$ is a function $d:M\times M\rightarrow\RR $ such that $\forall x,y,z\in M$ the following properties are satisfied:
    \begin{enumerate}
      \item $d(x,y)\geq 0$
      \item $d(x,y)=0\iff x=y$
      \item $d(x,y)=d(y,x)$
      \item $d(x,y)\leq d(x,z)+d(z,y)\quad$(\emph{triangular inequality})
    \end{enumerate}
    We define a \emph{metric space} as a pair $(M,d)$ that satisfy the previous properties.
  \end{definition}
  \begin{proposition}
    Let $(M_1,d_1),\ldots,(M_n,d_n)$ be metric spaces. Then, $M_1\times\cdots\times M_n$ with the distance $$d(x,y)=\max\{d_i(x_i,y_i):i=1,\ldots,n\}$$ where $x=(x_1,\ldots,x_n)$, $y=(y_1,\ldots,y_n)$, is a metric spaces.
  \end{proposition}
  \begin{definition}
    A metric space $(M,d)$ is \emph{complete} if every Cauchy sequence in $M$ converges in $M$.
  \end{definition}
  \begin{definition}
    Let $E$ be a real (or complex) vector space. A \emph{norm} on $E$ is a function $\|\cdot\|:E\rightarrow\RR $ such that $\forall \vf{u},\vf{v}\in E$ and $\forall\lambda\in\RR $ the following properties are satisfied:
    \begin{enumerate}
      \item $\|\vf{u}\|\geq 0$
      \item $\|\vf{u}\|=0\iff \vf{u}=0$
      \item $\|\lambda \vf{u}\|=|\lambda|\|\vf{u}\|$
      \item $\|\vf{u}+\vf{v}\|\leq \|\vf{u}\|+\|\vf{v}\|\quad$(\emph{triangular inequality})
    \end{enumerate}
    We define a \emph{normed vector space} as a pair $(E,\|\cdot\|)$ that satisfy the previous properties.
  \end{definition}
  \begin{proposition}
    Let $(E_1,\norm{\cdot}_1),\ldots,(E_n,\norm{\cdot}_n)$ be normed vector spaces. Then, $E_1\times\cdots\times E_n$ with the norm $$\norm{(x_1,\ldots,x_n)}=\max\{{\norm{x_i}}_i:i=1,\ldots,n\}$$ is a normed vector spaces.
  \end{proposition}
  \begin{proposition}
    Let $(E,\|\cdot\|)$ be a normed vector space and consider the following functions:
    $$\function{S}{E\times E}{E}{(x,y)}{x+y}\quad\function{P}{\RR\times E}{E}{(\lambda,x)}{\lambda x}$$ Then:
    \begin{enumerate}
      \item $S$ is uniformly continuous.
      \item $P$ is continuous.
      \item $\norm{\cdot}$ is uniformly continuous and: $$\abs{\norm{x}-\norm{y}}\leq\norm{x-y}\qquad\forall x,y\in E$$
    \end{enumerate}
  \end{proposition}
  \begin{definition}
    Let $(E,\|\cdot\|)$ be a normed vector space and $(x_n)\in E$ be a sequence. We say that $\sum_{n=1}^\infty x_n$ is a \emph{convergent series} in $E$ that converges to $x\in E$ if: $$\lim_{n\to\infty}\norm{x-\sum_{n=1}^\infty x_n}=0 $$
    We say that $\sum_{n=1}^\infty x_n$ is \emph{absolutely convergent} if: $$\sum_{n=1}^\infty\norm{x_n}<\infty$$
  \end{definition}
  \begin{proposition}
    Let $(E,\|\cdot\|)$ be a normed vector space and $\sum_{n=1}^\infty x_n=x$ be a convergent series. Then: $$\norm{x}\leq \sum_{n=1}^\infty\norm{x_n}$$
  \end{proposition}
  \begin{definition}[Banach space]
    A \emph{Banach space} is normed vector space which is complete with the distance associated with the norm.
  \end{definition}
  \begin{theorem}
    Let $(E,\|\cdot\|)$ be a normed vector space. Then, $(E,\|\cdot\|)$ is a Banach space if and only if every series in $E$ that converges absolutely converges.
  \end{theorem}
  \begin{proposition}
    Let $(E,\|\cdot\|)$ be a normed vector space and $F\subseteq E$ be a vector subspace. Then, $\overline{F}$ is a vector subspace of $E$.
  \end{proposition}
  \begin{proposition}
    Let $(E,\|\cdot\|)$ be a normed vector space and $F\subseteq E$ be a vector subspace.
    \begin{enumerate}
      \item If $F$ is complete, it is closed.
      \item If $F$ is closed and $E$ is Banach, then $F$ is complete.
    \end{enumerate}
  \end{proposition}
  \begin{definition}
    Let $(E,\|\cdot\|)$ be a normed vector space and $F\subseteq E$ be a vector subspace. We say that $F$ is a \emph{total subspace} if $\langle F\rangle$ is dense in $E$.
  \end{definition}
  \begin{definition}
    A metric space is called \emph{separable} if it contains a finite or countable dense subset.
  \end{definition}
  \begin{proposition}
    A normed vector space is separable if and only if it contains a total countable subset.
  \end{proposition}
  \subsubsection{\texorpdfstring{$L^p$}{Lp} spaces}
  \begin{definition}
    Let $E\subseteq\RR^n$ be a measurable set and $1\leq p<\infty$. We define:
    \begin{align*}
      \mathcal{L}^p(E) & :=\left\{{f}:E\rightarrow \RR^n\text{ measurable}:\int_E\abs{{f}}^p<\infty\right\}                                                                                                          \\
      \begin{split}
        \mathcal{L}^\infty(E) &:=\left\{{f}:E\rightarrow \RR^n\text{ measurable}:\exists M>0\text{ with }\right.\\&\hspace{4.5cm}\left.\abs{{f}(x)}\almoste{\leq} M, x\in E\right\}
      \end{split} \\
      \mathcal{N}(E)   & :=\{{f}:E\rightarrow \RR^n\text{ measurable}:{f}\almoste{=} 0\}
    \end{align*}
  \end{definition}
  \begin{definition}
    Let $E\subseteq\RR^n$ be a measurable set and $1\leq p\leq\infty$. We define: $$L^p(E):=\quot{\mathcal{L}^p(E)}{\mathcal{N}(E)}$$
  \end{definition}
  \begin{lemma}[Young's inequality for products]
    Let $a,b\in\RR$ and $1\leq p,q\leq \infty$ be such that $\frac{1}{p}+\frac{1}{q}=1$. Then:
    $$ab\leq\frac{a^p}{p}+\frac{b^q}{q}$$
    And the equality holds if and only if $a^p=b^q$.
  \end{lemma}
  \begin{lemma}[Hölder's inequality]
    Let $E\subseteq\RR^n$ be a measurable set, $1\leq p,q\leq \infty$ be such that $\frac{1}{p}+\frac{1}{q}=1$ and ${f}\in L^p(E)$, ${g}\in L^q(E)$. Then:
    $$\norm{{fg}}_1\leq\norm{{f}}_p\norm{{g}}_q$$
    And the equality holds if and only if $\exists \alpha,\beta\in\RR_{\geq 0}$ such that $\alpha{\norm{{f}}}^p\almoste{=}\beta{\norm{{g}}}^q$.
  \end{lemma}
  \begin{proposition}
    Let $E\subseteq\RR^n$ be a measurable set and $1\leq p<\infty$. The set $L^p$ is a normed vector space with the norm: $$\norm{{f}}_p:={\left(\int_E\abs{{f}}^p\right)}^{1/p}\qquad\forall{f}\in L^p$$
    And the set $L^\infty$ is also a normed vector space with the norm $$\norm{{f}}_\infty=\inf \{M:\abs{{f}(x)}\almoste{\leq} M, x\in E\}\qquad\forall{f}\in L^\infty$$
  \end{proposition}
  \begin{definition}
    Let $E\subseteq\RR^n$ be a measurable set and $U\subseteq \RR^n$ be an open set. We define:
    \begin{align*}
      S(E)                        & =\{f:E\rightarrow\RR:f\text{ is simple}\}                             \\
      \mathcal{C}_{\textrm{c}}(U) & =\{f:U\rightarrow\RR:f\in\mathcal{C}^0(U),\supp f\text{ is compact}\}
    \end{align*}
  \end{definition}
  \begin{theorem}
    Let $1\leq p<\infty$, $E\subseteq\RR^n$ be a measurable set and $U\subseteq \RR^n$ be an open set. Then:
    \begin{enumerate}
      \item $S(E)$ is dense in $L^p(E)$.
      \item $\mathcal{C}_{\textrm{c}}(U)$ is dense in $L^p(U)$.
    \end{enumerate}
  \end{theorem}
  \begin{theorem}
    Let $E\subseteq\RR^n$ be a measurable space, $({f}_k)\in L^p(E)$ be a sequence of functions and $1\leq p<\infty$. Then:
    \begin{enumerate}
      \item If $\lim_{k\to\infty}f_k(x)\almoste{=}{f}(x)$, $\norm{{f}_k}\almoste\leq g(x)\in L^p(E)$. Then, ${f}\in L^p(E)$ and $\lim_{k\to\infty}\norm{{f}_k-{f}}_p=0$ and we will write ${f}_k\overset{L^p}{\rightarrow}{f}$.
      \item If $\sum_{k=1}^\infty\norm{{f}_k}_p<\infty$, then $\sum_{k=1}^\infty\abs{f_k(x)}<\infty$ and $\exists f\in L^p(E)$ such that $\sum_{k=1}^\infty f_k(x)\almoste{=}f(x)$ and $\sum_{k=1}^N{f}_k\overset{L^p}{\rightarrow}{f}$. In particular, $(L^p,\norm{\cdot}_p)$ is a Banach space.
      \item If ${f}_k\overset{L^p}{\rightarrow}{f}$, then $\exists(f_{k_j})$ such that $\lim_{j\to\infty}f_{k_j}(x)\almoste{=}{f}(x)$.
    \end{enumerate}
  \end{theorem}
  \begin{theorem}
    Let $E\subseteq\RR^n$ be a measurable space and $1\leq p<\infty$. Then, $L^p(E)$ is separable.
  \end{theorem}
  \begin{definition}
    Let $T$ be an operator. We say that $T$ is \emph{sublinear} is $$\abs{T(f+g)}\leq \abs{Tf}+\abs{Tg}\qquad\forall f,g\in L^p$$
  \end{definition}
  \begin{theorem}[Marcinkiewicz interpolation theorem]
    Let $T$ be a sublinear operator. Then:
    \begin{enumerate}
      \item $\displaystyle\m{x\in\RR^n:\abs{Tf(x)}>t}\leq \frac{A}{t}\norm{f}_1$
      \item $\norm{Tf}_\infty\leq A_\infty\norm{f}_\infty$
    \end{enumerate}
  \end{theorem}
  \begin{corollary}
    Let $T$ be a sublinear operator. Then, $\norm{Tf}_p\leq A_p\norm{f}_p$ $\forall 1<p<\infty$ and $\forall f\in L^p$.
  \end{corollary}
  \subsubsection{Space of continuous functions}
  \begin{definition}
    Let $X\ne \varnothing$ be a set and $\mathcal{B}(X)$ be the vector space over $\KK=\RR,\CC$ of the functions $f:X\rightarrow\KK$ that are bounded with the \emph{uniform norm} (or \emph{supremum norm}): $$\norm{f}:=\norm{f}_X:=\sum\{\abs{f(x)}:x\in X\}$$
  \end{definition}
  \begin{proposition}
    Let $X\ne \varnothing$ be a set and $(f_n),f\in\mathcal{B}(X)$ be functions. Then: $$\lim_{n\to\infty}\norm{f_n-f}=0\iff f_n\text{ converges uniformly to }f$$
  \end{proposition}
  \begin{definition}
    Let $K\subseteq \RR^n$ be a compact set. We define $\mathcal{C}(K)$ as the closed subspace of $\mathcal{B}(K)$ containing the continuous functions.
  \end{definition}
  \begin{proposition}
    Let $K\subseteq \RR^n$ be a compact set and $f,g\in\mathcal{C}(K)$. Then: $$\norm{fg}_k\leq\norm{f}_K\norm{g}_K$$
  \end{proposition}
  \begin{definition}
    Let $K\subseteq \RR^n$ be a compact set and $A\subseteq \mathcal{C}(K)$ be a subset. We say that $A$ is a \emph{subalgebra} if $A$ is stable under the product, that is if $\forall f,g\in A$ we have $fg\in A$.
  \end{definition}
  \begin{proposition}
    Let $K\subseteq \RR^n$ be a compact set and $A\subseteq \mathcal{C}(K)$ be a subalgebra. Then, $\overline{A}$ is also a subalgebra.
  \end{proposition}
  \begin{definition}
    Let $K\subseteq \RR^n$ be a compact set and $A\subseteq \mathcal{C}(K)$ be a subalgebra. We say that $A$ is a \emph{separating set} (or \emph{separate} the points of $K$) if $\forall x,y\in K$ $\exists f\in A$ such that $f(x)\ne f(y)$.
  \end{definition}
  \begin{definition}
    Let $K\subseteq \RR^n$ be a compact set and $A\subseteq \mathcal{C}(K)$ be a subalgebra. We say that $A$ \emph{does not vanish} if $\forall x\in K$ $\exists f_x\in A$ such that $f_x(x)\ne 0$.
  \end{definition}
  \begin{lemma}
    Let $K\subseteq \RR^n$ be a compact set and $A\subseteq \mathcal{C}(K)$ be a subalgebra. If $A$ contains the constant functions, then $A$ does not vanish.
  \end{lemma}
  \begin{definition}
    Let $K\subseteq \RR^n$ be a compact set and $A\subseteq \mathcal{C}(K)$ be a subalgebra. We say that $A$ is \emph{autoconjugate} if $\overline{f}\in A$ whenever $f\in A$.
  \end{definition}
  \begin{theorem}[Stone-Weierstra\ss\ theorem]
    Let $K\subseteq \RR^n$ be a compact set and $A\subseteq \mathcal{C}(K)$ be a separating autoconjugate subalgebra that does not vanish. Then, $A$ is dense in $\mathcal{C}(K)$.
  \end{theorem}
\end{multicols}
\end{document}