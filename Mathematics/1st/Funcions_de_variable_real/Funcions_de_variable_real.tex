\documentclass[class=article,10pt,crop=false]{standalone}
\usepackage{standalone}
\usepackage{preamble}

\begin{document}
\begin{multicols}{2}[\section{Funcions de variable real}]
\subsection{La recta real}
\begin{definition}
Sigui $(K,+,\cdot)$ un cos. Diem que $K$ amb una relació d'ordre total $(\leq)$ és un cos ordenat si es verifiquen les següents propietats:
\begin{enumerate}
    \item Si $x,y,z\in K$ i $x\leq y$, aleshores $x+z\leq y+z$.
    \item Si $x,y\in K$ i $x\geq0$ i $y\geq0$, aleshores $x\cdot y\geq 0$.
\end{enumerate}
\end{definition}
\begin{definition}
Sigui $K$ un cos ordenat i $A\subset K$. Diem que $A$ està acotat superiorment
(respectivament inferiorment) si existeix $M\in K$, que anomenarem cota superior (respectivament cota inferior) de $A$, tal que $x\leq M$ (respectivament $x\geq M$) per a tot $x\in A$.
\end{definition}
\begin{definition}
Sigui $K$ un cos ordenat i $A\subset K$ acotat superiorment (respectivament inferiorment). Diem que $\alpha$ cota superior (respectivament inferior) de $A$ és el suprem (respectivament ínfim) de $A$ si qualsevol cota superior $\beta$ (respectivament inferior) verifica $\beta\geq\alpha$ (respectivament $\beta\leq\alpha$). El suprem de $A$ i l'ínfim de $A$, quan existeixin, els designem per $\text{sup }A$ i $\text{inf }A$ respectivament.
\end{definition}
\begin{theorem}[Axioma del suprem]
Existeix un únic cos ordenat amb la propietat que tot conjunt acotat
superiorment té suprem.
\end{theorem}
\begin{lemma}
Si $\alpha=\text{sup }A$, aleshores per a tot $\varepsilon>0$ l'interval $(\alpha-\varepsilon,\alpha]$ conté punts de $A$.
\end{lemma}
\begin{prop}
Els nombres naturals no estan acotats superiorment a $\mathbb{R}$.
\end{prop}
\begin{prop}
Entre dos nombres reals sempre n'hi ha un de racional i un d'irracional.
\end{prop}
\begin{definition}
Sigui $A$ un conjunt. Diem que $A$ és numerable si existeix una aplicació bijectiva de $A$ en $\mathbb{N}$.
\end{definition}
\begin{prop}
Tot subconjunt infinit de $\mathbb{N}$ és numerable.
\end{prop}
\begin{corollary}
Tot subconjunt infinit d’un conjunt numerable és numerable.
\end{corollary}
\begin{corollary}
Sigui $A$ un conjunt amb infinits elements. Perquè $A$ sigui numerable,
n'hi ha prou que existeixi una aplicació injectiva de $A$ en $\mathbb{N}$.
\end{corollary}
\begin{prop}
Si $A$ i $B$ són numerables, aleshores $A\times B$ també ho és.
\end{prop}
\begin{theorem}
$\mathbb{Q}$ no és numerable.
\end{theorem}
\begin{theorem}
$\mathbb{R}$ no és numerable.
\end{theorem}
\subsection{Successions}
\begin{definition}
Diem que la successió $(a_n)$ està acotada superiorment (respectivament acotada inferiorment) si existeix un nombre real $K$ de manera que $a_n\leq K$ (respectivament $a_n\geq K$) per a tot $n\in\mathbb{N}$.
\end{definition}
\begin{definition}
Diem que $\lim a_n=l$ si $\forall\varepsilon>0$ $\exists n_0$ tal que $|a_n-l|<\varepsilon$ $\forall n>n_0$.
\end{definition}
\begin{lemma}
Siguin $(a_n)$ i $(b_n)$ successions convergents amb límits $a$ i $b$ respectivament. Aleshores els següents fets són certs:
\begin{enumerate}
    \item Les successions $(a_n+b_n)$ i $(a_nb_n)$ són convergents i $\lim (a_n+b_n)=a+b$ i $\lim (a_nb_n)=ab$.
    \item Si $a\ne 0$, aleshores $a_n\ne 0$ per a $n$ prou gran la successió $(\frac{1}{a_n})$ és convergent i $\lim \frac{1}{a_n}=\frac{1}{a}$.
\end{enumerate}
\end{lemma}
\begin{theorem}
Tota successió monòtona i acotada és convergent.
\end{theorem}
\begin{lemma}[\bfseries Teorema del sandvitx]
Siguin $(a_n)$, $(b_n)$ i $(c_n)$ tres successions verificant $a_n\leq b_n\leq c_n$ $\forall n\in\mathbb{N}$. Suposem, a més, que $\lim a_n=\lim c_n=l$. Aleshores $(b_n)$ és convergent i $\lim b_n=l$.
\end{lemma}
\begin{lemma}
$e=\lim S_n=\lim T_n$ on $S_n=\sum_{i=0}^n \frac{1}{i!}$ i $T_n=\left(1+\frac{1}{n}\right)^n$.
\end{lemma}
\begin{theorem}
El nombre $e$ és irracional.
\end{theorem}
\begin{lemma}
Si $\lim a_n=l$, llavors qualsevol successió parcial de $(a_n)$ té també límit $l$. 
\end{lemma}
\begin{prop}
$a$ és un punt d'acumulació de $(a_n)$ si i només si existeix $(a_{k_n})$ parcial de $(a_n)$ amb $\lim a_{k_n}=a$.
\end{prop}
\begin{prop}
Tota successió té una parcial mo\-nò\-to\-na.
\end{prop}
\begin{theorem}[\bfseries Teorema de Bolzano-Weierstra\ss]
Tota successió de punts d’un interval tancat té una parcial convergent a un punt de l'interval.
\end{theorem}
\begin{lemma}
Sigui $(a_n)$ acotada. Llavors $(a_n)$ és convergent si i només si $\liminf a_n\\=\limsup a_n$. En aquest cas es té que $\lim a_n=\limsup a_n=\liminf a_n$
\end{lemma}
\begin{definition}
Diem que la successió $(a_n)$ és una successió de Cauchy si $\forall \varepsilon>0$, $\exists n_0$ tal que $|a_n-a_m|<\varepsilon$ per a qualsevol $n,m>n_0$.
\end{definition}
\begin{theorem}
Una successió és convergent si i només si és de Cauchy.
\end{theorem}
\begin{theorem}[\bfseries Criteri de Stolz]
Sigui $(a_n)$ una successió estrictament monòtona i $(b_n)$ una successió qualsevol. Suposem a més que $\text{lim}\frac{b_n-b_{n-1}}{a_n-a_{n-1}}=l\in\mathbb{R}\cup\{\pm\infty\}$. Aleshores les següents afirmacions són certes:
\begin{enumerate}
    \item Si $\lim a_n=\pm\infty$, llavors $\lim\frac{b_n}{a_n}=l$.
    \item Si $\lim b_n=\text{lim }a_n=0$, llavors $\lim\frac{b_n}{a_n}=l$.
\end{enumerate}
\end{theorem}
\subsection{Continuïtat}
\begin{definition}
Sigui $f:[a,b]\rightarrow\mathbb{R}$ una funció i $x_0\in(a,b)$. Diem que $l$ és el límit de la funció $f$ en el punt $x_0$ i escrivim $\lim_{x\to x_0}f(x)=l$, si $\forall\varepsilon>0$, $\exists\delta>0$ de manera que $|f(x)-l|<\varepsilon$ sempre que $|x-x_0|<\delta$.
\end{definition}
\begin{lemma}
Sigui $f:[a,b]\rightarrow\mathbb{R}$ i $x_0\in(a,b)$. Aleshores $\lim_{x\to x_0}f(x)=l$ si i només si per a tota successió $a_n$ de punts de $(a,b)\setminus\{x_0\}$ amb $\lim a_n=x_0$ es compleix que $\lim f(a_n)=l$.
\end{lemma}
\begin{definition}
Sigui $I$ un interval, $f:I\rightarrow\mathbb{R}$ una funció i $x_0\in I$. Diem que $f$ és contínua en $x_0$ si existeix el límit de $f$ en $x_0$ i és igual a $f(x_0)$.
\end{definition}
\begin{lemma}
$f$ és contínua en $x_0\in I$ si i només si per a tota successió $x_n\in I$ amb $\lim x_n=x_0$ es té que $\lim f(x_n)=f(x_0)$.
\end{lemma}
\begin{prop}
Siguin $f,g:I\rightarrow \mathbb{R}$ contínues en $x_0\in I$. Llavors:
\begin{enumerate}
    \item $f+g$ i $fg$ són contínues en $x_0$.
    \item Si $f(x_0)>0$ (respectivament $f(x_0)<0$), aleshores $f(x)>0$ (respectivament $f(x)<0$) en un entorn de $x_0$. A més, en ambdós casos, $\frac{1}{f}$ és contínua en $x_0$.
\end{enumerate}
\end{prop}
\begin{prop}
Siguin $f:I\rightarrow\mathbb{R}$ i $g:J\rightarrow\mathbb{R}$. Sigui $x_0\in I$ amb $f(x_0)\in J$ i suposem que $f$ és contínua en $x_0$ i $g$ també ho és en $f(x_0)$. Aleshores $g\circ f$ és contínua en $x_0$.
\end{prop}
\begin{theorem}[\bfseries Teorema de Weierstra\ss]
Sigui $f:[a,b]\rightarrow\mathbb{R}$ contínua. Aleshores $f$ està acotada en $[a,b]$. A més, existeixen $y,z\in[a,b]$ tals que $f(y)\leq f(x)\leq f(z)$ $\forall x\in [a,b]$. 
\end{theorem}
\begin{theorem}[\bfseries Teorema de Bolzano]
Sigui $f$ contínua en $[a,b]$. Si $f(a)f(b)<0$, llavors existeix $c\in(a,b)$ amb $f(c)=0$. 
\end{theorem}
\begin{theorem}
Sigui $f:(c,d)\rightarrow\mathbb{R}$ contínua. Si $f$ és injectiva i contínua, aleshores $f$ és monòtona. A més, $f^{-1}$ és també contínua en $f((c,d))$.
\end{theorem}
\subsection{Derivació}
\begin{definition}
Si $f:(a,b)\rightarrow\mathbb{R}$. Diem que $f$ és derivable en $x_0\in(a,b)$ si existeix el límit $$\lim_{h\to 0}\frac{f(x_0+h)-f(x_0)}{h}$$
\end{definition}
\begin{prop}
Siguin $f,g$ definides en un entorn de $a$ i derivables en $a$. Aleshores $f+g$ i $fg$ són derivables en $a$ i \begin{gather*}
    (f+a)'(a)=f'(a)+g'(a),\\
    (fg)'(a)=f'(a)g(a)+f(a)g'(a).
\end{gather*} Si a més $f(a)\ne 0$, llavors $\frac{1}{f}$ està definida en un entorn de $a$, és derivable en $a$ i $$\left(\frac{1}{f}\right)'(a)=-\frac{f'(a)}{f^2(a)}$$
\end{prop}
\begin{prop}[\bfseries Regla de la cadena]
Siguin $g:(a,b)\rightarrow\mathbb{R}$ i $f:(c,d)\rightarrow\mathbb{R}$. Suposem que $g$ és derivable en $x\in(a,b)$ i $f$ és derivable en $g(x)\in(c,d)$. Aleshores $f\circ g$ és derivable en $x$ i $(f\circ g)'(x)=f'(g(x))g'(x)$.
\end{prop}
\begin{prop}
Sigui $f:(a,b)\rightarrow\mathbb{R}$ injectiva i contínua en $(a,b)$ i derivable en $c\in(a,b)$ amb $f'(c)\ne 0$. Aleshores $f^{-1}$ és derivable en $f(c)$ i $$(f^{-1})'(f(c))=\frac{1}{f'(c)}$$
\end{prop}
\begin{prop}
Sigui $f:I\rightarrow\mathbb{R}$ i sigui $c\in I$ extrem local de $f$. Si $f$ és derivable en $c$, $f'(c)=0$.
\end{prop}
\begin{theorem}[\bfseries Teorema de Rolle]
Sigui $f:[a,b]\rightarrow\mathbb{R}$ contínua i derivable en $(a,b)$. Suposem $f(a)=f(b)$. Aleshores existeix un punt $c\in (a,b)$ amb $f'(c)=0$.
\end{theorem}
\begin{theorem}[\bfseries Teorema del valor mitjà]
Sigui $f:[a,b]\rightarrow\mathbb{R}$ contínua en $[a,b]$ i derivable en $(a,b)$. Aleshores existeix un punt $c\in (a,b)$ verificant $$f'(c)=\frac{f(b)-f(a)}{b-a}$$
\end{theorem}
\begin{corollary}
Sigui $f$ derivable en $(a,b)$ verificant $f'(x)>0$ (respectivament $f'(x)<0$) $\forall x\in(a,b)$. Aleshores $f$ és creixent (respectivament decreixent) en $(a,b)$.
\end{corollary}
\begin{theorem}[\bfseries Teorema de Cauchy]
Siguin $f,g:[a,b]\rightarrow\mathbb{R}$ contínues en $[a,b]$ i derivables en $(a,b)$. Aleshores existeix un punt $c\in (a,b)$ verificant $$f'(c)(g(b)-g(a))=g'(c)(f(b)-f(a))$$
\end{theorem}
\begin{theorem}[\bfseries Regla de l'Hôpital]
Suposem que $f,g$ són dues funcions definides en un entron de $a$ i que o bé $\lim_{x\to a} f=\lim_{x\to a} g=0$, o bé $\lim_{x\to a} g=\infty$. Suposem també que existeix el límit $\lim_{x\to a} \frac{f'(x)}{g'(x)}$. Aleshores també existeix el $\lim_{x\to a} \frac{f(x)}{g(x)}$ i $$\lim_{x\to a} \frac{f(x)}{g(x)}=\lim_{x\to a} \frac{f'(x)}{g'(x)}$$
\end{theorem}
\begin{theorem}[\bfseries Teorema de Darboux]
Sigui $f:(a,b)\rightarrow\mathbb{R}$ derivable i suposem que existeixen $x,y\in (a,b)$, $x<y$ amb $f'(x)f'(y)<0$. Aleshores existeix $z\in(x,y)$ tal que $f'(z)=0$. 
\end{theorem}
\subsection{Convexitat i segona derivada}
\begin{definition}
Diem que $f:I\rightarrow\mathbb{R}$ és convexa si donats dos punts qualssevol $a,b\in I$, $a<b$ el segment que uneix els punts $(a,f(a))$ i $(b,f(b))$ queda per damunt de la gràfica en $(a,b)$. Diem que $f:I\rightarrow\mathbb{R}$ és còncava si donats dos punts qualssevol $a,b\in I$, $a<b$ el segment que uneix els punts $(a,f(a))$ i $(b,f(b))$ queda per sota de la gràfica en $(a,b)$.
\end{definition}
\begin{lemma}
$f$ és convexa en $I$ si i només si per a qualssevol $a,x,b\in I$ amb $a<x<b$ es té: $$\frac{f(x)-f(a)}{x-a}<\frac{f(b)-f(a)}{b-a},$$ o, equivalentment, $$\frac{f(b)-f(a)}{b-a}<\frac{f(b)-f(x)}{b-x}.$$
Si $f$ és còncava, les desigualtats s'inverteixen.
\end{lemma}
\begin{theorem}
Sigui $f$ derivable en $I$. Llavors $f$ és convexa (respectivament còncava) si i només si $f'$ és creixent (respectivament decreixent).
\end{theorem}
\begin{theorem}
Sigui $f$ derivable en $I$. Llavors $f$ és convexa (respectivament còncava) si i només si qualsevol tangent a la gràfica queda per sota (respectivament sobre) de la gràfica excepte en el punt de contacte.
\end{theorem}
\begin{theorem}
Sigui $f$ dues vegades derivable en $I$. Llavors les següents afirmacions són certes:
\begin{enumerate}
    \item Si $f$ és convexa (respectivament còncava) en $I$ aleshores $f''(x)\geq 0$ (respectivament $f''(x)\leq 0$) $\forall x\in I$.
    \item Si $f''(x)>0$ (respectivament $f''(x)<0$) $\forall x\in I$ aleshores $f$ és convexa (respectivament còncava) en $I$.
\end{enumerate}
\begin{prop}
Sigui $f$ dues vegades derivable en $I$. Aleshores les següents afirmacions són certes:
\begin{enumerate}
    \item Si $a$ és un punt d'inflexió aleshores $f''(a)=0$.
    \item Suposem a més que $f''$ és contínua en $a\in I$. Aleshores si $f''(a)>0$ (respectivament $f''(a)<0$) aleshores $f$ és convexa (respectivament còncava) en $a$.
\end{enumerate}
\end{prop}
\end{theorem}
\subsection{Aproximació polinòmica}
\begin{definition}
Diem que $f$ i $g$ tenen un contacte d'ordre $\geq n$ en $a$ si $$\lim_{x\to a}\frac{f(x)-g(x)}{(x-a)^n}=0$$
\end{definition}
\begin{definition}
Diem que $f$ és de classe $\mathcal{C}^n$ en $a$ si $f$ és $n$ vegades derivable en un entorn de $a$ i $f^{(n)}$ és contínua en aquest entorn. Diem que $f$ és de classe $\mathcal{C}^\infty$ en $a$ si $f$ és de classe $\mathcal{C}^n$ en $a$ per a tot $n\in\mathbb{N}$.
\end{definition}
\begin{theorem}
Sigui $f$ $n$ vegades derivable en $a$. Aleshores el polinomi
\begin{multline*}
    P_{n,f,a}=f(a)+f'(a)(x-a)+\frac{f''(a)}{2!}(x-a)^2+\\+\frac{f^{(3)}(a)}{3!}(x-a)^3+\cdots+\frac{f^{(n)}(a)}{n!}(x-a)^n
\end{multline*} té un contacte amb $f$ d'ordre $\geq n$ en a.
\end{theorem}
\begin{theorem}
Sigui $f$ derivable $n$ vegades en $a$. Si $f'(a)=f''(a)=\cdots=f^{(n-1)}(a)=0$ i $f^{(n)}(a)\ne 0$ llavors:
\begin{enumerate}
    \item Si $n$ és senar $a$ no és extrem relatiu.
    \item Si $n$ és parell i $f^{(n)}>0$, aleshores $a$ és un mínim relatiu.
    \item Si $n$ és parell i $f^{(n)}<0$, aleshores $a$ és un màxim relatiu.
\end{enumerate}
\end{theorem}
\begin{theorem}
Sigui $f$ derivable $n+1$ vegades en $I$ un entorn de $a$. Sigui $P=P_{n,f,a}$ i $R_n=f-P$. Sigui $x\in I$. Llavors,
\begin{enumerate}
    \item Fórmula de Cauchy: $$R_n(x)=\frac{f^{(n+1)}(\xi)}{n!}(x-\xi)^n(x-a),$$ per algun $\xi$ entre $a$ i $x$.
    \item Fórmula de Lagrange: $$R_n(x)=\frac{f^{(n+1)}(\eta)}{(n+1)!}(x-a)^{n+1},$$ per algun $\eta$ entre $a$ i $x$.
    \item Si, a més $f^{(n+1)}$, és integrable en $[a,x]$: $$R_n(x)=\int_a^x\frac{f^{(n+1)}(t)}{n!}(x-t)^ndt$$
\end{enumerate}
\end{theorem}
\begin{definition}
Diem que $f$ és analítica en $a$ si és de classe $\mathcal{C}^\infty$ en un entorn de $a$ i $\lim_{n\to\infty}R_n(x)=0$ per a tot $x$ en aquest entorn.
\end{definition}
\subsection{Integral de Riemann}
\begin{definition}
Definim la suma inferior i superior de $f$, respectivament, associada a la partició $P$ com:
$$L(f,P)=\sum_{i=1}^nm_i(t_i-t_{i-1})\qquad U(f,P)=\sum_{i=1}^nM_i(t_i-t_{i-1})$$
on $m_i=\text{inf}\{f(x_i):x_i\in[t_{i-1},t_i]\}$ i $M_i=\text{sup}\{f(x_i):x_i\in[t_{i-1},t_i]\}$.
\end{definition}
\begin{definition}
Sigui $f$ acotada en $I=[a,b]$. Diem que $f$ és integrable en $I$ si $\lowint{a}{b}f=\upint{a}{b}f$ on $\lowint{a}{b}=\text{inf}\{L(f,P):P \text{ partició de }[a,b]\}$ i $\upint{a}{b}=\text{sup}\{U(f,P):P \text{ partició de }[a,b]\}$
\end{definition}
\begin{lemma}
Sigui $f$ acotada en $I=[a,b]$. Aleshores $f$ és integrable en $I$ si i només si $\forall\varepsilon>0$ existeix una partició $P$ de $I$ amb $U(f,P)-L(f,P)<\varepsilon$.
\end{lemma}
\begin{theorem}
Sigui $f$ monòtona i fitada a $I=[a,b]$. Llavors $f$ és integrable en $I$.
\end{theorem}
\begin{theorem}
Sigui $I=[a,b]$ i $f$ contínua en $I$. Aleshores $f$ és uniformement contínua en $I$.
\end{theorem}
\begin{theorem}
Sigui $f$ contínua en $I=[a,b]$. Aleshores $f$ és integrable en $I$.
\end{theorem}
\begin{prop}
Siguin $f$ i $g$ integrables en $I=[a,b]$ i $c\in\mathbb{R}$. Llavors $f+g$ i $cf$  són integrables en $I$ i $$\int_a^b(f+g)=\int_a^bf+\int_a^bg\qquad \int_a^bcf=c\int_a^bf$$
\end{prop}
\begin{theorem}
Sigui $f$ integrable en $[a,b]$ amb $f([a,b])\subset[c,d]$ i $g$ contínua en $[c,d]$. Llavors $g\circ f$ és integrable en $[a,b]$.
\end{theorem}
\begin{corollary}
Siguin $f$ i $g$ integrables en $[a,b]$. Llavors $fg$ és integrable en $[a,b]$.
\end{corollary}
\begin{prop}
Siguin $f,g$ integrables en $[a,b]$ amb $f(x)\leq g(x)$ $\forall x\in [a,b]$. Aleshores $\int_a^bf\leq\int_a^bg$.
\end{prop}
\begin{corollary}
Si $f$ és integrable en $[a,b]$ amb $m\leq f(x)\leq M$ $\forall x\in [a,b]$. Aleshores $m(b-a)\leq\int_a^bf\leq M(b-a)$. Si a més $f$ és contínua, aleshores existeix $c\in[a,b]$ amb $\int_a^bf=f(c)(b-a)$.
\end{corollary}
\begin{prop}
Si $f$ és integrable en $[a,b]$, aleshores $|f|$ també ho és i $$\left|\int_a^bf\right|\leq\int_a^b|f|$$
\end{prop}
\begin{prop}
Sigui $f$ una funció integrable en $[a,b]$ i $g$ una funció definida en $[a,b]$ diferent de $f$ en un nombre finit de punts. Aleshores $g$ és integrable i $$\int_a^bg=\int_a^bf$$
\end{prop}
\begin{prop}
Si $f:[a,c]\rightarrow\mathbb{R}$ i $b\in(a,c)$ aleshores $f$ integrable en $[a,c]$ si i només si $f$ és integrable en $[a,b]$ i en $[b,c]$. A més, $$\int_a^cf=\int_a^bf+\int_b^cf$$
\end{prop}
\begin{theorem}[\bfseries Teorema fonamental del càlcul]
Si $f$ és integrable en [a,b], aleshores $$F(x)=\int_a^xf$$ és contínua en $[a,b]$. Si, a més, $f$ és contínua en $c\in[a,b]$, aleshores $F$ és derivable en $c$ i $F'(c)=f(c)$.
\end{theorem}
\begin{theorem}
Sigui $f$ integrable en $[a,b]$ i $G$ una primitiva de $f$. Llavors $\int_a^bf=G(b)-G(a)$.
\end{theorem}
\begin{corollary}[\bfseries Integració per parts]
Siguin $f,g$ integrables en $[a,b]$ amb primitives $F$ i $G$ respectivament. Aleshores es compleix: $$\int_a^bFg=F(b)G(b)-F(a)G(a)-\int_a^bfG$$
\end{corollary}
\begin{corollary}[\bfseries Canvi de variable]
Sigui $\varphi:[c,d]\rightarrow[a,b]$ de classe $\mathcal{C}^1$ i tal que $\varphi(c)=a$ i $\varphi(d)=b$. Sigui $f$ contínua en $[a,b]$. Aleshores es compleix: $$\int_a^bf=\int_c^d(f\circ\varphi)\varphi'$$
\end{corollary}
\begin{definition}
Una suma de Riemann de $f$, $S(f,P)$, associada a una partició $P$, és qualsevol nombre obtingut de la següent manera: $$S(f,P)=\sum_{i=1}^nf(x_i)(t_i-t_{i-1})$$ on $x_i\in[t_{i-1},t_i]$.
\end{definition}
\begin{theorem}
Sigui $f$ contínua en $[a,b]$. Aleshores $\forall\varepsilon>0$ $\exists\delta>0$ tal que si $P=\{t_0,\ldots,t_n\}$ és una partició de $[a,b]$ amb $t_i-t_{i-1}<\delta$, llavors $$\left|\int_a^bf-S(f,P)\right|<\varepsilon$$ per a tota suma de Riemann associada a $P$.
\end{theorem}
\begin{corollary}
Sigui $f$ contínua en $[a,b]$ i sigui $P_n$ una successió de particions de $[a,b]$ tal que $t_i-t_{i-1}<1/n$ sempre que $t_i$ i $t_{i-1}$ siguin punts consecutius de $P_n$. Aleshores per a tota elecció $S(f,P_n)$ de sumes de Riemann associades a les particions $P_n$ es té que $$\int_a^b=\lim_{n\to\infty}S(f,P_n)$$
\end{corollary}
\begin{definition}
Sigui $f:[a,b]\rightarrow\mathbb{R}$ i $P=\{t_0,\ldots,t_n\}$ una partició de $[a,b]$. Definim $$l(f,P):=\sum_{i=1}^n\sqrt{(t_i-t_{i-1})^2+(f(t_i)-f(t_{i-1}))^2}$$Si el conjunt $\mathcal{L}=\{l(f,P):P \text{ partició de }[a,b]\}$ està acotat superiorment, diem que la gràfica és rectificable i definim la seva longitud $l(f,[a,b]):=\text{sup }\mathcal{L}$.
\end{definition}
\begin{prop}
Sigui $f$ de classe $\mathcal{C}^1$ a $[a,b]$. Aleshores $f$ és rectificable a $[a,b]$ i $$l(f,[a,b])=\int_a^b\sqrt{1+(f')^2}$$
\end{prop}
\begin{prop}
Sigui $\varphi:[a,b]\rightarrow\mathbb{R}^2$ amb $\varphi(t)=(x(t),y(t))$. Suposem que les funcions $x(t)$, $y(t)$ són de classe $\mathcal{C}^1$ a $[a,b]$. Aleshores la corba $\varphi$ és rectificable a $[a,b]$ i $$l(f,[a,b])=\int_a^b\sqrt{(x')^2+(y')^2}$$
\end{prop}
\begin{prop}
Sigui $f:[a,b]\rightarrow\mathbb{R}$ acotada i integrable. Aleshores el volum de revolució de $f$ respecte l'eix horitzontal és $$V_x=\pi\int_a^bf^2$$
\end{prop}
\begin{prop}
Sigui $f:[a,b]\rightarrow\mathbb{R}$ contínua. Aleshores el volum de revolució de $f$ respecte l'eix vertical és $$V_y=2\pi\int_a^bxf(x)dx$$
\end{prop}
\begin{prop}
Sigui $f:[a,b]\rightarrow\mathbb{R_+}$ de classe $\mathcal{C}^1$. Aleshores la superfície de revolució del gràfic de $f$ al voltant de l'eix horitzontal és $$S_x=2\pi\int_a^bf(x)\sqrt{1+(f'(x))^2}dx$$
\end{prop}
\begin{prop}
Si $a>0$ i $f:[a,b]\rightarrow\mathbb{R}$. La superfície de revolució del gràfic de $f$ al voltant de l'eix vertical és $$S_y=2\pi\int_a^bx\sqrt{1+(f'(x))^2}dx$$
\end{prop}
\begin{prop}
El centre de masses $(x_0,y_0)$ d'un sòlid molt prim de secció i densitat uniforme és: $$x_0=\frac{\int_a^bx\sqrt{1+(f'(x))^2}dx}{\int_a^b\sqrt{1+(f'(x))^2}dx},y_0=\frac{\int_a^b f(x)\sqrt{1+(f'(x))^2}dx}{\int_a^b\sqrt{1+(f'(x))^2}dx}$$
El moment d'inèrcia d'un sòlid és: $$I_x=\rho\sigma\int_a^bx^2f(x)dx\qquad I_y=\rho\sigma\int_a^bx^2\sqrt{1+(f'(x))^2}dx$$ on $\rho$ és la densitat del sòlid i $\sigma$ el gruix, suposat constant, del sòlid.
\end{prop}
\end{multicols}
\end{document}