\documentclass[class=article,10pt,crop=false]{standalone}
\usepackage{standalone}
\usepackage{preamble}

\begin{document}
\begin{multicols}{2}[\section{Àlgebra lineal}]
\subsection{Matrius}
\begin{definition}
Direm que una matriu $A\in\mathcal{M}_n(\mathbb{R})$ és invertible si existeix una matriu $B\in\mathcal{M}_n(\mathbb{R})$ que compleix $AB=BA=I_n$.
\end{definition}
\begin{lemma}
El producte de matrius invertibles és invertible.
\end{lemma}
\begin{theorem}[\bfseries Teorema de Gau\ss]
Donada una matriu qualsevol $A\in\mathcal{M}_{m\times n}(\mathbb{R})$, existeix una matriu invertible $P\in\mathcal{M}_m(\mathbb{R})$ tal que $A'=PA$ és esglaonada i reduïda. A més, $A'$ està únicament determinada per $A$.
\end{theorem}
\begin{theorem}[\bfseries Teorema de la PAQ reducció]
Donada una matriu $A\in\mathcal{M}_{m\times n}(\mathbb{R})$, existeixen matrius invertibles $P\in\mathcal{M}_m(\mathbb{R})$ i $Q\in\mathcal{M}_n(\mathbb{R})$ tals que $$PAQ=\left(\begin{array}{@{\,} c|c @{\,}}
    I_r & 0\\
    \hline
    0 & 0
    \end{array}\right).$$ El nombre $r$ és el rang de les matrius $PAQ$ i $A$.
\end{theorem}
\begin{prop}
$\forall A\in\mathcal{M}_{m\times n}(\mathbb{R})$ tenim que $\text{rang }A=\text{rang }A^t$.
\end{prop}
\begin{theorem}[\bfseries Teorema de Rouché]
Donat un sistema d'equacions lineals $Ax=b$ amb $n$ incògnites, el sistema és:
\begin{itemize}
    \item Compatible determinat $\iff\text{rang }A=\text{rang}(A\mid b)=n$.
    \item Compatible indeterminat amb $s$ variables lliures $\iff\text{rang }A=\text{rang}(A\mid b)=n-s$.
    \item Incompatible $\iff\text{rang }A\ne\text{rang}(A\mid b)$.
\end{itemize}
\end{theorem}
\begin{definition}
Un determinant és una aplicació $\mathcal{M}_n(\mathbb{R})\rightarrow\mathbb{R}$ que compleixi:
\begin{enumerate}
    \item L'aplicació ha de ser lineal en cada fila i columna. És a dir, si $a_1,\ldots,a_n$ són les columnes d'una matriu $A\in\mathcal{M}_n(\mathbb{R})$ i $a_j=\lambda u+\mu v$, aleshores:
    \begin{multline*}
        \det A=\det([a_1|\cdots|a_j|\cdots| a_n])=\\=\det([a_1|\cdots|\lambda u+\mu v|\cdots| a_n])=\\=\lambda\det([a_1|\cdots| u|\cdots| a_n])+\\+\mu\det([a_1|\cdots| v|\cdots| a_n])
    \end{multline*}
    \item El determinant canvia de signe si s'intercanvien dues columnes.
    \item $\det I_n=1$.
\end{enumerate}
\end{definition}
\begin{prop}
Donada una matriu $A\in\mathcal{M}_n(\mathbb{R})$, $A$ no és invertible $\iff\text{rang }A<n\iff\det A=0$.
\end{prop}
\begin{theorem}
$\forall A,B\in\mathcal{M}_n(\mathbb{R})$ tenim que $\det (AB)=\det A\det B$.
\end{theorem}
\begin{prop}
Denotem $S_n$ el grup de permutacions de $\{1,\ldots,n\}$ i sigui $A\in\mathcal{M}_n(\mathbb{R})$. Llavors: $$\det A=\sum_{\sigma\in S_n}\varepsilon(\sigma)\prod_{i=1}^na_{i\sigma(i)}$$
\end{prop}
\begin{prop}
$\forall A\in\mathcal{M}_n(\mathbb{R})$ tenim que $\det A=\det A^t$.
\end{prop}
\begin{theorem}
$\forall A\in\mathcal{M}_n(\mathbb{R})$ es compleix $A(\text{adj }A)^t=(\det A)I_n$ i si $\det A\ne 0$ aleshores, $$A^{-1}=\frac{1}{\det A}(\text{adj }A)^t$$
\end{theorem}
\subsection{Espais vectorials}
\begin{definition}
Sigui $E$ un $K$-espai vectorial i $F$ un subconjunt de $E$, llavors $(F,+_E,\cdot_E)$ és un $K$-espai vectorial si es verifica $\lambda v_1+\mu v_2\in F$ $\forall v_1,v_2\in F$ i $\forall\lambda,\mu\in K$.
\end{definition}
\begin{lemma}
La intersecció de subespais vectorials és un subespai vectorial.
\end{lemma}
\begin{definition}
Donat un $K$-espai vectorial $E$, una base de $E$ és un conjunt ordenat $B$ de vectors de $E$ que és: 
\begin{enumerate}
    \item Sistema de generadors de $E$.
    \item Linealment independent.
\end{enumerate}
\end{definition}
\begin{theorem}[\bfseries Teorema de Steinitz]
Donat un $K$-espai vectorial $E$, $B$ una base de $E$ i $(v_1,\ldots,v_k)$ vectors linealment independents de $E$, aleshores podem substituir $k$ vectors apropiats de $B$ per $(v_1,\ldots,v_k)$ i definir una nova base.
\end{theorem}
\begin{definition}
La suma de dos subespais $F,G$ dins d'un $K$-espai vectorial $E$ és: $F+G=\langle F\cup G\rangle=\{u+v\mid u\in F,v\in G\}$.
\end{definition}
\begin{prop}[\bfseries Fórmula de Gra\ss mann]
$\dim (F+G)+\dim(F\cap G)=\dim F+\dim G$.
\end{prop}
\begin{definition}
Sigui $E$ un $K$-espai vectorial i siguin $F,G\subset E$ dos subespais vectorials. Llavors la suma $F+G$ és directe ($F\oplus G$) $\iff F\cap G=\{0\}$.
\end{definition}
\begin{definition}
Donada una matriu $A\in\mathcal{M}_n(\mathbb{R})$, un menor d'ordre $r$ de $A$ és una submatriu $A'\in\mathcal{M}_r(\mathbb{R})$ obtinguda seleccionant $r$ files i $r$ columnes de $A$.
\end{definition}
\begin{definition}
Sigui $E$ un $K$-espai vectorial i $F\subset E$ un subespai vectorial. Anomenarem subespai complementari de $F$ a tot subespai $G\subset E$ que compleixi $F\oplus G=E$.
\end{definition}
\begin{definition}
Direm que dos vectors $u,v\in E$ són equivalents mòdul $F$ (on $F\subset E$ un subespai vectorial) si $u-v\in F$ i escriurem $u\sim_Fv$. $\sim_F$ és una relació d'equivalència.
\end{definition}
\begin{definition}
L'espai quocient $E/F$ és el conjunt de les classes d'equivalència $u+F=[u]$ amb les operacions:
$$[u]+[v]=[u+v]\qquad a[u]=[au]$$ $E/F$ és un $K$-espai vectorial.
\end{definition}
\begin{prop}
Sigui $E$ un $K$-espai vectorial de dimensió $n<\infty$ i $F\subset E$ un subespai vectorial, llavors $\dim (E/F)=\dim E-\dim F$.
\end{prop}
\subsection{Aplicacions lineals}
\begin{definition}
Sigui $E,F$ dos $K$-espais vectorials. Una aplicació $f:E\rightarrow F$ és lineal si es compleix $f(\lambda v_1+\mu v_2)=\lambda f(v_1)+\mu f(v_2)$ $\forall v_1,v_2\in E$ i $\forall\lambda,\mu\in K$.
\end{definition}
\begin{prop}
Si $f:E\rightarrow F$ i $g:F\rightarrow G$ són aplicacions lineals, llavors la composició $g\circ f:E\rightarrow G$ és lineal.
\end{prop}
\begin{prop}
Si $f:E\rightarrow F$ és una aplicació lineal, llavors $f^{-1}:F\rightarrow E$ és lineal.
\end{prop}
\begin{prop}
Sigui $f:E\rightarrow F$ una aplicació lineal entre $K$-espais vectorials i siguin $G\subset E$ i $H\subset F$ subespais vectorials. Aleshores:
\begin{enumerate}
    \item $f(G)=\{f(u)\mid u\in G\}\subset F$ és un subespai vectorial.
    \item $f^{-1}(H)=\{u\in E\mid f(u)\in H\}\subset E$ és un subespai vectorial.
\end{enumerate}
Si $G=E$ i $H=\{0\}$, aleshores:
\begin{enumerate}
    \item $f(E)$ és la imatge de $f$ i es denota $\text{Im }f$.
    \item $f^{-1}(0)$ és el nucli de $f$ i es denota $\text{Ker }f$.
\end{enumerate}
\end{prop}
\begin{prop}
Si $E,F$ són $K$-espais vectorials de dimensió finita i $f:E\rightarrow F$, aleshores $f$ és injectiva $\iff \dim(\text{Ker }f)=0$ i $f$ és exhaustiva $\iff \dim(\text{Im }f)=\dim F$.
\end{prop}
\begin{definition}
\hfill
\begin{enumerate}
    \item Un monomorfisme és una aplicació lineal injectiva.
    \item Un epimorfisme és una aplicació lineal exhaustiva.
    \item Un isomorfisme és una aplicació lineal bijectiva.
    \item Un endomorfisme és una aplicació lineal d'un espai vectorial en ell mateix.
    \item Un automorfisme és un endomorfisme bijectiu.
\end{enumerate}
\end{definition}
\begin{lemma}
Donada $f:E\rightarrow F$ una aplicació lineal, on $E,F$ són $K$-espais vectorials, i $u_1,\ldots,u_k\in E$ es compleix $\langle f(u_1),\ldots,f(u_k)\rangle=f(\langle u_1,\ldots,u_k\rangle)$.
\end{lemma}
\begin{theorem}[\bfseries Isomorfisme de coordenació]
Sigui $E$ un $K$-espai vectorial  de dimensió $n$ i $B=(u_1,\ldots,u_n)$ una base de $E$. Llavors l'aplicació $f:K^n\rightarrow E$, $f(a_1,\ldots,a_n)=a_1u_1+\cdots a_nu_n$ és un isomorfisme.
\end{theorem}
\begin{theorem}[\bfseries Teorema de l'isomorfisme]
Sigui $f:E\rightarrow F$ una aplicació lineal. Existeix un isomorfisme $\Tilde{f}:E/\text{Ker }f\rightarrow \text{Im }f$ complint $f=i\circ\Tilde{f}\circ\pi$ on $\pi:E\rightarrow E/\text{Ker }f$ i $i:\text{Im }f\rightarrow F$.
\end{theorem}
\begin{corollary}
Sigui $f:E\rightarrow F$ una aplicació lineal i suposem que $\dim E=n<\infty$. Llavors $n=\dim(\text{Ker }f)+\dim(\text{Im }f)$.
\end{corollary}
\begin{corollary}
Siguin $E,F$ dos $K$-espais vectorials de dimensió $n<\infty$ i $f:E\rightarrow F$ una aplicació lineal. Llavors $f$ és injectiva $\iff f$ és exhaustiva $\iff f$ és bijectiva.
\end{corollary}
\begin{theorem}[\bfseries Teoremes d'Emmy Noether]
Sigui $E$ un $K$-espai vectorial, $F,G\subset E$ dos subespais:
\begin{enumerate}
    \item Existeix un isomorfisme $F/(F\cap G)\cong(F+G)/G$.
    \item Si $G\subset F\subset E$, existeix un isomorfisme $(E/G)/(F/G)\cong E/F$.
\end{enumerate}
\end{theorem}
\begin{theorem}
Siguin $E,F$ dos $K$-espais vectorials i siguin $B=(u_1,\ldots,u_n)$ una base de $E$ i $v_1,\ldots,v_n\in F$ vectors qualssevol. Llavors existeix una única aplicació lineal $f:E\rightarrow F$ tal que $f(u_i)=v_i$, $i=1,\ldots,n$. 
\end{theorem}
\begin{prop}
Siguin $f:E\rightarrow F$ i $g:F\rightarrow G$ dues aplicacions lineals entre $K$-espais vectorials i siguin $B,B',B''$ bases de $E,F$ i $G$ respectivament. Llavors $g\circ f:E\rightarrow G$ té matriu $[g\circ f]_{B,B''}=[g]_{B',B''}[f]_{B,B'}$ en les bases $B, B''$.
\end{prop}
\begin{corollary}
Les matrius de canvi de base són invertibles i $[id]_{B,B'}^{-1}=[id]_{B',B}$.
\end{corollary}
\begin{prop}
Donada $f:E\rightarrow F$ una aplicació lineal i bases $B_1,B_2$ de $E$ i $B_1',B_2'$ de $F$ es compleix: $[f]_{B_2,B_2'}=[id]_{B_1',B_2'}[f]_{B_1,B_1'}[id]_{B_2,B_1}$.
\end{prop}
\begin{theorem}
Donada qualsevol aplicació lineal $f:E\rightarrow F$ on $\dim E=n, \dim F=m$. Existeixen bases $B_0$ de $E$ i $B_0'$ de $F$ en les quals $[f]_{B_0,B_0'}=\left(\begin{array}{@{\,} c|c @{\,}}
    I_r & 0\\
    \hline
    0 & 0
    \end{array}\right)$ on $r=\dim\text{Im }f$ i $[f]_{B_0,B_0'}=[id]_{B',B_0'}[f]_{B,B'}[id]_{B_0,B}$ $\forall B,B'$ base de $E$ i $F$ respectivament.
\end{theorem}
\begin{lemma}
Donats dos $K$ espais vectorials $E,F$ el conjunt $\mathcal{L}(E,F)=\{f\mid f\text{ és una aplicació}\\ \text{lineal de $E$ a $F$}\}$ és un $K$-espai vectorial i es compleix que $(\lambda f+\mu g)(v)=\lambda f(v)+\mu g(v)$ $\forall f,g\in\mathcal{L}(E,F)$ i $\lambda,\mu\in K$.
\end{lemma}
\begin{prop}
Siguin $E,F$ dos $K$-espais vectorials amb $\dim E=n<\infty$, $\dim F=m<\infty$. Per a tota base $B$ de $E$ i $B'$ de $F$, l'aplicació $\varphi:\mathcal{L}(E,F)\rightarrow\mathcal{M}_{m\times n}(K)$ definida per $\varphi(f)=[f]_{B,B'}$ és un isomorfisme.
\end{prop}
\begin{corollary}
Si $\dim E=n$ i $\dim F=m$ aleshores $\dim \mathcal{L}(E,F)=mn$.
\end{corollary}
\begin{definition}
L'espai dual d'un $K$-espai vectorial és $\mathcal{L}(E,K)=E^*$. En el cas $\dim E=n<\infty$, escollir una base $B$ determina un isomorfisme $\varphi:E^*\rightarrow\mathcal{M}_{1\times n}(K)$, definit per $\varphi(\omega)=[\omega]_{B,B'}$. Deduïm, doncs, que $\dim E^*=\dim E$.
\end{definition}
\begin{definition}
Donats $E$ un $K$-espai vectorial de dimensió finita i $B=(u_1,\ldots,u_n)$ una base de $E$. La base dual de $B$ és la base de $E^*$ formada per $(\eta_1,\ldots,\eta_n)$ on $\eta_i(u_j)=\delta_{ij}$.
\end{definition}
\begin{lemma}
Sigui $E$ un $K$-espai vectorial de dimensió $n<\infty$ i sigui $B=(u_1,\ldots,u_n)$ una base de $E$. $\forall u\in E$, $(u_1^*(u),\ldots,u_n^*(u))=[u]_B\in K^n$ on $(u_1^*,\ldots,u_n^*)$ és la base dual de $B$.
\end{lemma}
\begin{lemma}
Sigui $E$ un $K$-espai vectorial de dimensió $n<\infty$ i sigui $B=(u_1,\ldots,u_n)$ una base de $E$. $\forall \omega\in E^*$, $(\omega(u_1),\ldots,\omega(u))=[\omega]_{B^*}$.
\end{lemma}
\begin{definition}
Si $f\in \mathcal{L}(E,F)$ l'aplicació $f^*:F^*\rightarrow E^*$ definida per $f^*(\omega)=\omega\circ f$ s'anomena aplicació dual de $f$. Aquesta aplicació és lineal.
\end{definition}
\begin{theorem}
Siguin $E,F$ dos $K$-espais vectorials de dimensió finita i siguin $B,B'$ bases de $E,F$ respectivament. Llavors $[f^*]_{B'^*,B^*}=([f]_{B,B'})^t$.
\end{theorem}
\begin{definition}
Donat un $K$-espai vectorial $E$, el bidual és el $K$-espai vectorial definit per $(E^*)^*=\mathcal{L}(E^*,K)$. A més, si $\dim E=n<\infty$, l'aplicació $f:E\rightarrow (E^*)^*$ definida per $f(v)=\phi_v$ $(\phi_v:E^*\rightarrow K$, $\phi_v(\omega)=\omega(v))$ és un isomorfisme natural. 
\end{definition}
\begin{definition}
Sigui $E$ un $K$-espai vectorial i $F$ un subespai vectorial de $E^*$. El subespai de $E$ incident de $F$ és $F^{inc}=\{v\in E\mid \omega(v)=0\;\forall\omega\in F\}$.
\end{definition}
\begin{theorem}
Sigui $E$ un $K$-espai vectorial de dimensió $n<\infty$ i $F\subset E^*$ un subespai amb $\dim F=m$. Llavors $\dim F^{inc}=n-m$.
\end{theorem}
\begin{definition}
Donat un subespai vectorial i $F\subset E$, el seu subespai incident és $F^{inc}=\{\omega\in E^*\mid \omega(v)=0\;\forall v\in F\}$.
\end{definition}
\begin{prop}
$(F^{inc})^{inc}=F$ tant si $F\subset E$ com si $F\subset E^*$.
\end{prop}
\subsection{Classificació d'endomorfismes}
\begin{definition}
Dues matrius $M,N\in\mathcal{M}_n(K)$ s'anomenen similars si existeix $P\in\mathcal{M}_n(K)$ invertible tal que $M=P^{-1}NP$.
\end{definition}
\begin{prop}
Donats $f,g\in\mathcal{L}(E)$ on $E$ és un $K$ espai vectorial de dimensió $n<\infty$:
\begin{enumerate}
    \item $f$ i $g$ són similars $\iff\forall B$ base de $E$ les matrius $[f]_B$ i $[g]_B$ són similars.
    \item $f$ i $g$ són similars $\iff\exists h$ automorfisme tal que $g=h^{-1}fh$.
\end{enumerate}
\end{prop}
\begin{definition}
Una matriu $A\in\mathcal{M}_n(K)$ és diagonalitzable si és similar a una matriu diagonal. Un endomorfisme és diagonalitzable si la seva matriu en alguna base és diagonalitzable.
\end{definition}
\begin{definition}
Donat $f\in\mathcal{L}(E)$ diem que un vector $u\in E$, $u\ne 0$ és vector propi de $f$ de valor propi $\lambda$ si $f(u)=\lambda u$.
\end{definition}
\begin{lemma}
Donats $f\in\mathcal{L}(E)$ i $\lambda\in K$, els vectors propis de $f$ de valor propi $\lambda$ són els vectors no nuls del subespai $Ker(f-\lambda id)$ (subespai propi de valor propi $\lambda$).
\end{lemma}
\begin{definition}
Donada una matriu $A\in\mathcal{M}_n(K)$, el polinomi $p_A(\lambda)=\det(A-\lambda I_n)$ s'anomena polinomi característic de $A$.
\end{definition}
\begin{prop}
Donat $f\in\mathcal{L}(E)$, vectors propis de $f$ de valors propis diferents són linealment independents.
\end{prop}
\begin{prop}
Sigui $E$ un $K$-espai vectorial de dimensió $n<\infty$ i sigui $\lambda$ una arrel del polinomi característic $p_f(x)$ de multiplicitat $m$. Llavors $1\leq \dim(\text{Ker}(f-\lambda id))\leq m$.
\end{prop}
\begin{theorem}[\bfseries Teorema de diagonalització]
Sigui $f\in\mathcal{L}(E)$, $f$ és diagonalitzable si i només si:
\begin{enumerate}
    \item $p_f(x)=(-1)^n(x-\lambda_1)^{m_1}\cdots(x-\lambda_k)^{m_k}$ amb $\lambda_1,\ldots,\lambda_k\in K$ diferents.
    \item $\dim(\text{Ker}(f-\lambda_i id))=m_i$.
\end{enumerate}
\end{theorem}
\begin{corollary}
Si $n=\dim E$ i $f$ té $n$ valors propis diferents, $f$ és diagonalitzable.
\end{corollary}
\begin{definition}
El polinomi mínim de $f$ és un polinomi $P\in K[x]$ tal que:
\begin{itemize}
    \item $P(f)=0$.
    \item $P$ és mònic.
    \item $P$ és de grau mínim.
\end{itemize}
\end{definition}
\begin{theorem}[\bfseries Teorema de Cayley-Hamilton]
Sigui $K$ un cos, $n\geq 1$ i $A\in\mathcal{M}_n(K)$. Llavors $m_A(x)\mid p_A(x)\mid m_A(x)^n$. Sigui $K$ un cos i $f\in\mathcal{L}(E)$, $\dim_K E=n$. Llavors $m_f(x)\mid p_f(x)\mid m_f(x)^n$. 
\end{theorem}
\begin{definition}
Un cos satisfent que tot polinomi de grau $\geq 1$ factoritzi completament en factors lineals s'anomena algebraicament tancat.
\end{definition}
\begin{definition}
Donat $f\in\mathcal{L}(E)$ diem que $W\subseteq E$ és un subespai invariant de $E$ per $f$ si $f(W)\subseteq W$.
\end{definition}
\begin{lemma}
Donat $f\in\mathcal{L}(E)$ si $E=W_1\oplus W_2$ amb $W_1,W_2$ subespais invariants, aleshores $p_f(x)=p_{f|W_1}(x)p_{f|W_2}(x)$ i $m_f(x)=\text{mcm}(m_{f|W_1}(x),m_{f|W_2}(x))$.
\end{lemma}
\begin{theorem}
Sigui $f\in\mathcal{L}(E)$ , $\dim E=n<\infty$. Si $p_f(x)=q_1(x)^{n_1}\cdots q_r(x)^{n_r}$ i $m_f(x)=q_1(x)^{m_1}\cdots q_r(x)^{m_r}$ amb $q_i(x)$ factors irreductibles diferents. Aleshores $E=\text{Ker}(q_1(f)^{m_1})\oplus\cdots\oplus\text{Ker}(q_r(f)^{m_r})$. A més $\text{Ker}(q_i(f)^{m_i})=n_i\text{deg}(q_i(x))$.
\end{theorem}
\begin{theorem}
Si $p_f(x)=(x-\lambda_1)^{n_1}\cdots(x-\lambda_k)^{n_k}$ i $g$ compleix:
\begin{enumerate}
    \item $p_f(x)=p_g(x)$
    \item $m_f(x)=m_g(x)$
    \item $\dim(\text{Ker}((f-\lambda id)^r))=\dim(\text{Ker}((g-\lambda id)^r))$ $\forall\lambda\in K$ $\forall r\geq 1$,
\end{enumerate}
llavors $f\sim g$.
\end{theorem}
\begin{prop}
Donats $E$ un $K$-espai vectorial de dimensió $n<\infty$, $A\in\mathcal{M}_n(K)$. Si $p_A(x)=\pm(x-\lambda_1)^{n_1}\cdots(x-\lambda_k)^{n_k}$, existeix una matriu invertible $P$ complint: $$P^{-1}AP=\begin{pmatrix}
J_1 & & & \\
& J_2 & & \\
& & \ddots & \\
& & & J_e \\
\end{pmatrix}$$
on $J_1,\ldots,J_e$ són blocs de Jordan de valors propis $\lambda_1,\ldots,\lambda_k$ complint:
\begin{enumerate}
    \item Per a cada $\lambda_i$ la suma de les mides dels blocs de Jordan de valor propi $\lambda_i$ és $n_i$.
    \item Les mides dels blocs de Jordan estan determinades per $\dim(\text{Ker}((f-\lambda id)^r))$.
\end{enumerate}
\end{prop}
\subsection{Formes bilineals simètriques}
\begin{definition}
Siguin $K$ un cos i $E,F,G$ tres $K$-espais vectorials. Diem que una aplicació $\varphi:E\times F\rightarrow G$ és bilineal si:
\begin{enumerate}
    \item $\varphi(\lambda u_1+\mu u_2,v)=\lambda \varphi(u_1,v)+\mu \varphi(u_2,v)$ $\forall u_1,u_2\in E$, $\forall v\in F$ i $\forall\lambda,\mu\in K$.
    \item $\varphi(u,\lambda v_1+\mu v_2)=\lambda \varphi(u,v_1)+\mu \varphi(u,v_2)$ $\forall v_1,v_2\in F$, $\forall u\in E$ i $\forall\lambda,\mu\in K$.
\end{enumerate}
\end{definition}
\begin{definition}
Una forma bilineal sobre el $K$-espai vectorial $E$ és una aplicació lineal $\varphi:E\times E\rightarrow K$.
\end{definition}
\begin{definition}
Una forma bilineal $\varphi:E\times E\rightarrow K$ és simètrica si $\varphi(u,v)=\varphi(v,u)$ $\forall u,v\in E$.
\end{definition}
\begin{definition}
Sigui $\varphi:E\times E\rightarrow K$ una forma bilineal i sigui $B=(v_1,\ldots,v_n)$ una base de $E$. La matriu de la forma bilineal $\varphi$ respecte de la base $B$ és la matriu: $$[\varphi]_B=\begin{pmatrix}
\varphi(v_1,v_1) & \varphi(v_1,v_2) & \cdots & \varphi(v_1,v_n) \\
\varphi(v_2,v_1) & \varphi(v_2,v_2) & \cdots & \varphi(v_2,v_n) \\
\vdots & \vdots & \ddots & \vdots \\
\varphi(v_n,v_1) & \varphi(v_n,v_2) & \cdots & \varphi(v_n,v_n) \\
\end{pmatrix}.$$ 
\end{definition}
\begin{prop}
Sigui $B=(u_1,\ldots,u_n)$ una base de $E$. Una forma bilineal $\varphi$ sobre $E$ és simètrica si, i només si, $[\varphi]_B$ és una matriu simètrica.
\end{prop}
\begin{prop}
Sigui $\varphi:E\times E\rightarrow K$ una forma bilineal. Siguin $B,B'$ bases de $E$. Aleshores es compleix $[\varphi]_{B'}=([id]_{B',B})^t[\varphi]_B[id]_{B',B}$.
\end{prop}
\begin{definition}
Sigui $\varphi:E\times E\rightarrow \mathbb{R}$ una forma bilineal. Es diu que $u,v\in E$ són ortogonals si $\varphi(u,v)=0$. Un vector no nul $v\in E$ és isòtrop si és ortogonal a ell mateix, és a dir, si $\varphi(v,v)=0$. Sigui $B=(v_1,\ldots,v_n)$ una base de $E$. Es diu que $B$ és una base ortogonal respecte de $\varphi$ si $\varphi(v_i,v_j)=0$ $\forall i\ne j$.
\end{definition}
\begin{theorem}
Sigui $\varphi:E\times E\rightarrow \mathbb{R}$ una forma bilineal simètrica. Llavors $E$ té una base ortogonal respecte de $\varphi$.
\end{theorem}
\begin{definition}
Sigui $\varphi$ una forma bilineal simètrica sobre $E$. Sigui $W$ un subespai vectorial de $E$. Definim l'ortogonal de $W$ respecte de $\varphi$ per $$W^\perp=\{u\in E:\varphi(u,\omega)=0\;\forall\omega\in W\}$$ Definim el radical de $\varphi$ per $$\text{rad }\varphi=E^\perp$$ Direm que $\varphi$ és no singular si $\text{rad}(\varphi)=\{0\}$.
\end{definition}
\begin{definition}
Sigui $\varphi$ una forma bilineal simètrica no singular sobre $E$. Donat un $u\in E$ definim $\varphi_u:E\rightarrow\mathbb{R}$, $\varphi_u(v)=\varphi(u,v)$. Llavors l'aplicació 
\begin{align*}
    \Phi: E&\rightarrow E^*\\
    u&\mapsto\varphi_u
\end{align*} és un isomorfisme. 
\end{definition}
\begin{definition}
Sigui $\varphi$ una forma bilineal simètrica no singular sobre $E$. Sigui $W$ un subespai vectorial de $E$. Llavors:
\begin{enumerate}
    \item $\dim E=\dim W+\dim W^\perp$.
    \item $(W^\perp)^\perp=W$.
    \item Si la restricció de $\varphi$ sobre $W$ és no singular, llavors $E=W\oplus W^\perp$.
\end{enumerate}
\end{definition}
\begin{definition}
Sigui $\varphi$ una forma bilineal simètrica sobre $E$. Direm que la suma $W_1+W_2$ de dos subespais vectorials $W_1$ i $W_2$ de $E$ és una suma ortogonal si és directa i $\varphi(u,v)=0$ $\forall u\in W_1$ i $v\in W_2$. Escriurem $W_1\perp W_2$ per denotar que la suma $W_1+W_2$ és ortogonal.
\end{definition}
\begin{definition}
Sigui $\varphi$ una forma bilineal simètrica sobre $E$. Siguin $W_1$ i $W_2$ dos subespais vectorials de $E$ tals que $E=W_1\perp W_2$. Llavors per a cada vector $v\in E$ existeixen $v_1\in W_1$ i $v_2\in W_2$ únics tals que $v=v_1+v_2$. L'aplicació $\pi:E\rightarrow W_i$ definida per $\pi(v)=v_i$ amb $v_i\in W_i$ es diu que és la projecció ortogonal de $E$ sobre $W_i$ segons la descomposició $E=W_1\perp W_2$.
\label{perpendicular}
\end{definition}
\begin{definition}
Una geometria ortogonal sobre $\mathbb{R}$ és un parell $(E,\varphi)$, on $E$ és un $\mathbb{R}$-espai vectorial i $\varphi$ és una forma bilineal simètrica sobre $E$.
\end{definition}
\begin{definition}
Siguin $(E_1,\varphi_1)$ i $(E_2,\varphi_2)$ dues geometries ortogonals sobre $\mathbb{R}$. Una isometria de $(E_1,\varphi_1)$ a $(E_2,\varphi_2)$ és un isomorfisme $f:E_1\rightarrow E_2$ tal que $$\varphi_2(f(u),f(v))=\varphi_1(u,v)$$ per a tot $u,v\in E_1$. Direm que $(E_1,\varphi_1)$ i $(E_2,\varphi_2)$ són isomètriques si existeix una isometria de $(E_1,\varphi_1)$ a $(E_2,\varphi_2)$.
\label{isometry}
\end{definition}
\begin{definition}
Sigui $E$ un $\mathbb{R}$-espai vectorial. Direm que dues formes bilineals simètriques $\varphi_1,\varphi_2$ sobre $E$ són equivalents si, i només si, $(E,\varphi_1)$ i $(E,\varphi_2)$ són isomètriques.
\end{definition}
\begin{definition}
Siguin $A,B\in\mathcal{M}_n(\mathbb{R})$. Direm que $A$ i $B$ són congruents si existeix una matriu $P\in\mathcal{M}_n(\mathbb{R})$ invertible tal que $A=P^tBP$.
\end{definition}
\begin{prop}
Sigui $E$ un $\mathbb{R}$-espai vectorial de dimensió $n<\infty$ i $\varphi_1,\varphi_2$ formes bilineals simètriques sobre $E$. Sigui $B_1$ una base de $V$. Llavors les condicions següents són equivalents:
\begin{enumerate}
    \item Les geometries ortogonals $(E,\varphi_1)$ i $(E,\varphi_2)$ $\varphi$ i $\psi$ són isomètriques.
    \item Existeix una bases $B_2$ de $E$ tal que $[\varphi_1]_{B_1}=[\varphi_2]_{B_2}$.
    \item Les matrius $[\varphi_1]_{B_1}$ i $[\varphi_2]_{B_2}$ són congruents.
\end{enumerate}
\end{prop}
\begin{theorem}[\bfseries Teorema de Sylvester o Llei d'inèrcia]
Sigui $E$ un $\mathbb{R}$-espai vectorial de dimensió $n<\infty$. Sigui $\varphi$ una forma bilineal simètrica sobre $E$. Llavors existeix una base $B$ de $E$ tal que $$[\varphi]_B=\begin{pmatrix}
0 &&&&&&&&&\\
& \ddots&&&&&&&\\
&& 0&&&&\bigzero&&\\
&&& 1 &&&&&\\
&&&& \ddots &&&&\\
&&&&& 1 &&&\\
&&\bigzero&&&& -1 &&\\
&&&&&&&\ddots &\\
&&&&&&&& -1
\end{pmatrix}$$
on a la diagonal hi ha $r_0\;0's$, $r_+\;1's$ i $r_-\;-1's$ i $r,r_+,r_-$ no depenen de la base $B$.
\end{theorem}
\begin{definition}
Sigui $\varphi$ una forma bilineal simètrica sobre un $\mathbb{R}$-espai vectorial $E$ de dimensió $n<\infty$. Sigui $B$ una base ortogonal de $E$ respecte de $\varphi$. Definim el rang de $\varphi$ com $\text{rang}(\varphi)=\text{rang}([\varphi]_B)$. Definim la signatura de $\varphi$ com $\text{sig}(\varphi)=(r_+,r_-)$, on $r_+$ és el nombre de reals positius que hi ha a la diagonal de $[\varphi]_B$ i $r_-$ és el nombre de reals negatius que hi ha a la diagonal de $[\varphi]_B$.
\end{definition}
\begin{theorem}
Siguin $(E_1,\varphi_1)$, $(E_2,\varphi_2)$ dues geometries ortogonals sobre $\mathbb{R}$ de dimensió finita. Llavors $(E_1,\varphi_1)$ i $(E_2,\varphi_2)$ són isomètriques si, i només si, $\dim E_1=\dim E_2$ i $\text{sig}(\varphi_1)=\text{sig}(\varphi_2)$.
\end{theorem}
\begin{definition}
Sigui $\varphi$ una forma bilineal simètrica sobre un $\mathbb{R}$-espai vectorial $E$. Es diu que $\varphi$ és definida positiva si $\forall v\in E$, $v\ne 0$, tenim $\varphi(v,v)>0$. Es diu que $\varphi$ és definida negativa si $\forall v\in E$, $v\ne 0$, tenim $\varphi(v,v)<0$. 
\end{definition}
\begin{definition}
Un producte escalar sobre un $\mathbb{R}$-espai vectorial $E$ és una forma bilineal simètrica definida positiva sobre $E$. Un espai vectorial euclidià és un parell $(E,\varphi)$, on $E$ és un $\mathbb{R}$-espai vectorial i $\varphi$ és un producte escalar sobre $E$.\label{espai_euclidia}
\end{definition}
\begin{theorem}[Desigualtat de Cauchy-Schwartz]
Sigui $\varphi$ un producte escalar sobre un $\mathbb{R}$-espai vectorial $E$, llavors: $$\varphi(u,v)^2\leq \varphi(u,u)\varphi(v,v)\quad\forall u,v\in E$$
\end{theorem}
\begin{definition}
Sigui $E$ un $\mathbb{R}$-espai vectorial. Una norma sobre $E$ és una aplicació \begin{align*}
    \|\;\|:E&\rightarrow\mathbb{R}\\
    u&\mapsto\|u\|
\end{align*}
tal que \begin{enumerate}
    \item $\|u\|=0\iff u=0$.
    \item $\|\lambda u\|=|\lambda|\|u\|$, $\forall u\in E$, $\lambda\in\mathbb{R}$.
    \item $\|u+v\|\leq\|u\|+\|v\|$, $\forall u,v\in E$.
\end{enumerate}
\end{definition}
\begin{prop}
Sigui $(E,\varphi)$ un espai euclidià. Llavors l'aplicació
\begin{align*}
    \|\;\|_\varphi:E&\rightarrow\mathbb{R}\\
    u&\mapsto\|u\|_\varphi=\sqrt{\varphi(u,u)}
\end{align*}
és una norma, que es diu norma associada al producte escalar $\varphi$.
\end{prop}
\begin{definition}
Sigui $(E,\varphi)$ un espai vectorial euclidià de dimensió finita. Diem que una base $B=(v_1,\ldots,v_n)$ és ortonormal respecte de $\varphi$ si és ortogonal respecte de $\varphi$ i $\|v_i\|_\varphi=1$ per a $i=1,\ldots,n$.
\end{definition}
\begin{corollary}
Sigui $(E,\varphi)$ un espai vectorial euclidià de dimensió finita. Llavors $E$ té una base ortonormal respecte $\varphi$.
\end{corollary}
\begin{definition}
Sigui $(E,\varphi)$ un espai vectorial euclidià. Siguin $u,v\in E\setminus\{0\}$. Definim l'angle respecte $\varphi$ entre $u$ i $v$ com l'únic $\alpha\in[0,\pi]$ tal que: $$\cos{\alpha}=\frac{\varphi(u,v)}{\|u\|_\varphi\|v\|_\varphi}$$
\end{definition}
\begin{definition}
Sigui $(E,\varphi)$ un espai vectorial euclidià. Sigui $f\in\mathcal{L}(E)$. Definim l'adjunt de $f$ respecte de $\varphi$ com l'únic endomorfisme $f'$ de $E$ tal que $\varphi(f(u),v)=\varphi(u,f'(v))$ $\forall u,v\in E$. Si $f=f'$ diem que $f$ és autoadjunt. 
\end{definition}
\begin{lemma}
Sigui $(E,\varphi)$ un espai vectorial euclidià de dimensió $n<\infty$. Sigui $f$ un endomorfisme autoadut de $E$. Llavors existeixen $\lambda_1,\ldots,\lambda_n\in\mathbb{R}$ tals que $$p_f(x)=(x-\lambda_1 )\cdots(x-\lambda_n)$$
\end{lemma}
\begin{definition}
Sigui $A\in\mathcal{M}_n(K)$. Aleshores $A$ és ortogonal si, i només si, $PP^t=P^tP=I_n$.
\end{definition}
\begin{theorem}[Teorema espectral]
Sigui $(E,\varphi)$ un espai vectorial euclidià de dimensió $n<\infty$. Sigui $f\in\mathcal{L}(E)$ autoadjunt de $E$. Llavors $E$ té una base ortonormal de vectors propis de $f$. En particular, l’endomorfisme $f$ diagonalitza.
\end{theorem}
\begin{corollary}
Tota matriu simètrica $A\in\mathcal{M}_n(\mathbb{R})$ és diagonalitzable.
\end{corollary}
\begin{definition}
Donada una matriu $A\in\mathcal{M}_{m\times n}(\mathbb{C})$, $A=(a_{ij})$ denotem per $\overline{A}=(\overline{a_{ij}})$ la conjugada de $A$. $\forall A_1,A_2,A,B$ matrius de mides adequades i $\forall\lambda\in\mathbb{C}$ es satisfan les següents propietats:
\begin{enumerate}
    \item $\overline{A_1+A_2}=\overline{
    A_1}+\overline{A_2}$
    \item $\overline{AB}=\overline{
    A}\overline{B}$
    \item $\overline{\lambda A}=\overline{\lambda}\overline{A}$
\end{enumerate}
\end{definition}
\begin{theorem}[Regla dels signes de Descartes]
Donat un polinomi $P(x)=a_dx^d+\cdots+a_0$:
\begin{enumerate}
    \item El nombre d'arrels positives de $P(x)$ és com a molt igual al nombre de canvis de signe en $[a_{\!_d},a_{\!_{d-1}},\ldots,a_{\!_1},a_{\!_0}]$.
    \item SI $P(x)=a_d(x-\alpha_1)^{n_1}\cdots(x-\alpha_r)^{n_r}$, aleshores el nombre d'arrels positives és igual al nombre de canvis de signe (comptant les arrels amb multiplicitat).
\end{enumerate}
\end{theorem}
\end{multicols}
\end{document}