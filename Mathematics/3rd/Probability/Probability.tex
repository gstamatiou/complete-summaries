\documentclass[../../../main.tex]{subfiles}

\begin{document}
\begin{multicols}{2}[\section{Probability}]
  \subsection{Probabilistic models}
  \subsubsection*{Probability}
  \begin{definition}[Sample space]
    The \textit{sample space} $\Omega$ of an experiment is the set of all possible outcomes of that experiment.
  \end{definition}
  \begin{definition}
    Let $\Omega$ be a set. An \textit{event} is a subset of $\mathcal{P}(\Omega)$ for which we want to calculate the probability.
  \end{definition}
  \begin{definition}
    Let $\Omega$ be a set and $\mathcal{A}\subset\mathcal{P}(\Omega)$. We say that $\mathcal{A}$ is an \textit{algebra over $\Omega$} if:
    \begin{enumerate}
      \item $\Omega\in\mathcal{A}$.
      \item If $A\in\mathcal{A}$, then $A^c\in\mathcal{A}$.
      \item If $A,B\in\mathcal{A}$, then $A\cup B\in\mathcal{A}$.
    \end{enumerate}
  \end{definition}
  \begin{prop}
    Let $\mathcal{A}$ be an algebra over a set $\Omega$. Then:
    \begin{enumerate}
      \item $\varnothing\in\mathcal{A}$.
      \item If $A,B\in\mathcal{A}$, then $A\cap B\in\mathcal{A}$.
      \item For all $n\in\NN$, if $A_1,\ldots,A_n\in\mathcal{A}$, then: $$\bigcup_{i=1}^nA_i\in\mathcal{A}\quad\text{and}\quad\bigcap_{i=1}^nA_i\in\mathcal{A}$$
    \end{enumerate}
  \end{prop}
  \begin{definition}
    Let $\Omega$ be a set and $\mathcal{A}\subset\mathcal{P}(\Omega)$. We say that $\mathcal{A}$ is an \textit{$\sigma$-algebra over $\Omega$} if:
    \begin{enumerate}
      \item $\Omega\in\mathcal{A}$.
      \item If $A\in\mathcal{A}$, then $A^c\in\mathcal{A}$.
      \item If $A_1,A_2,\ldots\in\mathcal{A}$, then: $$\bigcup_{n=1}^\infty A_n\in\mathcal{A}$$
    \end{enumerate}
  \end{definition}
  \begin{prop}
    Let $\mathcal{A}$ be an $\sigma$-algebra over a set $\Omega$. Then:
    \begin{enumerate}
      \item $\varnothing\in\mathcal{A}$.
      \item If $A_1,A_2,\ldots\in\mathcal{A}$, then: $$\bigcap_{n=1}^\infty A_n\in\mathcal{A}$$
      \item For all $n\in\NN$, if $A_1,\ldots,A_n\in\mathcal{A}$, then: $$\bigcup_{i=1}^nA_i\in\mathcal{A}\quad\text{and}\quad\bigcap_{i=1}^nA_i\in\mathcal{A}$$
    \end{enumerate}
  \end{prop}
  \begin{definition}[Kolmogorov axioms]
    Let $\Omega$ be a set and $\mathcal{A}$ be a $\sigma$-algebra over $\Omega$. A \textit{probability} is any function $$P:\mathcal{A}\longrightarrow[0,\infty)$$ satisfying the following properties:
    \begin{itemize}
      \item $P(\Omega)=1$.
      \item (\textit{$\sigma$-additive set function}) If $\{A_n,n\geq1\}\subset\mathcal{A}$ are pairwise disjoint, then: $$P\left(\bigcup_{n=1}^\infty A_n\right)=\sum_{n=1}^\infty P(A_n)$$
    \end{itemize}
  \end{definition}
  \begin{definition}
    A \textit{probability space} is a triplet $(\Omega,\mathcal{A},P)$ where $\Omega$ is any set, $\mathcal{A}$ is a $\sigma$-algebra over $\Omega$ and $P$ is a probability over $\mathcal{A}$.
  \end{definition}
  \begin{prop}
    Let $(\Omega,\mathcal{A},P)$ be a probability space and $A,B\in\mathcal{A}$. Then, we have the following properties:
    \begin{enumerate}
      \item $P(\varnothing)=0$.
      \item If $A_i\in\mathcal{A}$, $i=1,\ldots,n$, is a finite set of pairwise disjoint events, then: $$P\left(\bigsqcup_{n=1}^\infty A_n\right)=\sum_{n=1}^\infty P(A_n)$$
      \item $P(A\setminus B) =P(A)-P(A\cap B)$.
      \item If $B\subset A$, then $P(A\setminus B)=P(A)-P(B)$.
      \item If $B\subset A$, then $P(B)\leq P(A)$.
      \item $P(A)\leq 1$.
      \item $P(A^c)=1-P(A)$.
      \item $P(A\cup B) =P(A)+P(B)-P(A\cap B)$.
      \item If $A_i,\ldots,A_n\in\mathcal{A}$, then:
            \begin{multline*}
              P\left(\bigcup_{i=1}^n A_i\right)=\sum_{i=1}^n P(A_i)-\\-\sum_{\substack{i,j=1\\i<j}}^nP(A_i\cap A_j)+\sum_{\substack{i,j,k=1\\i<j<k}}^nP(A_i\cap A_j\cap A_k)-\cdots+\\+{(-1)}^{n+1}P(A_1\cap\cdots\cap A_n)
            \end{multline*}
      \item If $A_i,\ldots,A_n\in\mathcal{A}$, then: $$P\left(\bigcup_{i=1}^n A_i\right)\leq\sum_{i=1}^n P(A_i)$$
    \end{enumerate}
  \end{prop}
  \begin{prop}
    Let $(\Omega,\mathcal{A},P)$ be a probability space such that $\Omega$ is finite and all its elements are equiprobable. Let $A\in\mathcal{A}$ be an event. Then: $$P(A)=\frac{|A|}{|\Omega|}$$
  \end{prop}
  \begin{definition}
    Let $\Omega$ be a set. The \textit{trivial $\sigma$-algebra} is the smallest $\sigma$-algebra over $\Omega$, that is, $\{\varnothing,\Omega\}$.
  \end{definition}
  \begin{definition}
    Let $\Omega$ be a set and $A\subseteq\Omega$ be a subset. The \textit{$\sigma$-algebra generated by $A$, $\sigma(A)$,} is the smallest $\sigma$-algebra over $\Omega$ containing $A$, that is: $$\sigma(A)=\{\varnothing,\Omega,A,A^c\}$$
  \end{definition}
  \begin{definition}
    Let $\Omega$ be a set and $\mathcal{C}\subseteq\mathcal{P}(\Omega)$ be a subset. The \textit{$\sigma$-algebra generated by $\mathcal{C}$, $\sigma(\mathcal{C})$,} is the smallest $\sigma$-algebra over $\Omega$ containing all the elements of $\mathcal{C}$. Moreover, if $\{\mathcal{A}_n:\mathcal{C}\subseteq\mathcal{A}_n,1\leq n\leq N\}$, $N\in\NN\cup\{\infty\}$, are all the $\sigma$-algebras over $\Omega$ containing $\mathcal{C}$, then:
    $$\sigma(\mathcal{C})=\bigcap_{n=1}^N\mathcal{A}_n$$
  \end{definition}
  \begin{definition}
    The \textit{Borel $\sigma$-algebra over $\RR$, $\mathcal{B}(\RR)$,} is the $\sigma$-algebra generated by the open sets of $\RR$: $$\mathcal{B}(\RR):=\sigma(\{U\subseteq\RR:U\text{ is open}\})$$
  \end{definition}
  \begin{theorem}
    Let $(\Omega,\mathcal{A},P)$ be a probability space and $\{A_n:n\geq 1\}\subset\mathcal{A}$ be an increasing sequence of events, so that: $$A_1\subset A_2\subset\cdots\subset A_n\subset\cdots$$ Let $A:=\bigcup_{n=1}^\infty A_n$. Then: $$P(A):=\lim_{n\to\infty}P(A_n)$$
  \end{theorem}
  \begin{corollary}
    Let $(\Omega,\mathcal{A},P)$ be a probability space and $\{A_n:n\geq 1\}\subset\mathcal{A}$ be an decreasing sequence of events, so that: $$A_1\supset A_2\supset\cdots\supset A_n\supset\cdots$$ Let $A:=\bigcap_{n=1}^\infty A_n$. Then: $$P(A):=\lim_{n\to\infty}P(A_n)$$
  \end{corollary}
  \begin{prop}
    Let $(\Omega,\mathcal{A},P)$ be a probability space and $\{A_n:n\geq 1\}\subset\mathcal{A}$ be a sequence of events. Then: $$P\left(\bigcup_{n=1}^\infty A_n\right)\leq\sum_{n=1}^\infty P(A_n)$$
  \end{prop}
  \begin{corollary}
    Let $(\Omega,\mathcal{A},P)$ be a probability space and $\{A_n:n\geq 1\}\subset\mathcal{A}$ be a sequence of events with probability 0. Then: $$P\left(\bigcup_{n=1}^\infty A_n\right)=0$$
  \end{corollary}
  \begin{corollary}
    Let $(\Omega,\mathcal{A},P)$ be a probability space and $\{A_n:n\geq 1\}\subset\mathcal{A}$ be a sequence of events with probability 1. Then: $$P\left(\bigcap_{n=1}^\infty A_n\right)=1$$
  \end{corollary}
  \subsubsection*{Conditional probability}
  \begin{definition}
    Let $(\Omega,\mathcal{A},P)$ be a probability space and $A\in\mathcal{A}$ be an event such that $P(A)>0$. The \textit{conditional probability that $B\in\mathcal{A}$ occurs given that $A$ occurs} is defined as: $$P(B\mid A):=\frac{P(A\cap B)}{P(A)}$$
  \end{definition}
  \begin{prop}
    Let $(\Omega,\mathcal{A},P)$ be a probability space and $A\in\mathcal{A}$ be an event such that $P(A)>0$. Then, the function
    \begin{align*}
      P(\cdot\mid A):\mathcal{A} & \longrightarrow [0,\infty] \\
      B                          & \longmapsto P(B\mid A)
    \end{align*}
    is a probability.
  \end{prop}
  \begin{prop}[Compound probability formula]
    Let $(\Omega,\mathcal{A},P)$ be a probability space and $A\in\mathcal{A}$ be an event such that $P(A)>0$. Then, $\forall B\in\mathcal{A}$: $$P(A\cap B)=P(B\mid A)P(A)$$
  \end{prop}
  \begin{prop}[Generalized compound probability formula]
    Let $(\Omega,\mathcal{A},P)$ be a probability space and $A_1,\ldots,A_n\in\mathcal{A}$, $n\geq 2$ be events such that $P(A_1\cap\cdots\cap A_{n-1})>0$. Then:
    \begin{multline*}
      P(A_1\cap\cdots\cap A_n)=P(A_1)P(A_2\mid A_1)P(A_3\mid A_2\cap A_1)\cdots\\\cdots P(A_n\mid A_1\cap\cdots\cap A_{n-1})
    \end{multline*}
  \end{prop}
  \begin{definition}
    Let $(\Omega,\mathcal{A},P)$ be a probability space and $A=\{A_n:1\leq n\leq N\}\subset\mathcal{A}$, $N\in\NN\cup\{\infty\}$, be a collection of events. We say that $A$ is a \textit{partition} of $\Omega$ if: $$\Omega=\bigsqcup_{n=1}^NA_n$$
  \end{definition}
  \begin{prop}[Total probability formula]
    Let $(\Omega,\mathcal{A},P)$ be a probability space and $\{A_n:1\leq n\leq N\}\subset\mathcal{A}$, $N\in\NN\cup\{\infty\}$, be a partition of $\Omega$ such that $P(A_n)>0$ for all $1\leq n\leq N$. Then, $\forall A\in\mathcal{A}$: $$P(A)=\sum_{n=1}^NP(A_n)P(A\mid A_n)$$
  \end{prop}
  \begin{prop}[Bayes' formula]
    Let $(\Omega,\mathcal{A},P)$ be a probability space and $\{A_n:1\leq n\leq N\}\subset\mathcal{A}$, $N\in\NN\cup\{\infty\}$, be a partition of $\Omega$ such that $P(A_n)>0$ for all $1\leq n\leq N$. Let $A\in\mathcal{A}$ with $P(A)>0$. Then, $\forall k\leq N$: $$P(A_k\mid A)=\frac{P(A_k)P(A\mid A_k)}{\sum_{n=1}^NP(A_n)P(A\mid A_n)}$$
  \end{prop}
  \subsubsection*{Independence of events}
  \begin{definition}
    Let $(\Omega,\mathcal{A},P)$ be a probability space. Then, $A,B\in\mathcal{A}$ are \textit{independent events} if $$P(A\cap B)=P(A)P(B)$$
  \end{definition}
  \begin{prop}
    Let $(\Omega,\mathcal{A},P)$ be a probability space. Then:
    \begin{enumerate}
      \item $\varnothing$ and $\Omega$ are independent of any event.
      \item If $A\in\mathcal{A}$ satisfies either $P(A)=0$ or $P(A)=1$, then $A$ is independent of any other event $B\in\mathcal{A}$.
      \item If an event $A\in\mathcal{A}$ is independent of itself, then either $P(A)=0$ or $P(A)=1$.
    \end{enumerate}
  \end{prop}
  \begin{prop}
    Let $(\Omega,\mathcal{A},P)$ be a probability space and $A,B\in\mathcal{A}$ be two events. The following statements are equivalent:
    \begin{itemize}
      \item $A$ and $B$ are independent.
      \item $A^c$ and $B$ are independent.
      \item $A$ and $B^c$ are independent.
      \item $A^c$ and $B^c$ are independent.
    \end{itemize}
  \end{prop}
  \begin{definition}
    Let $(\Omega,\mathcal{A},P)$ be a probability space and $n\in\NN$. We say that $A_1,\ldots,A_n\in\mathcal{A}$ are \textit{independent events} if for any $i_1,\ldots,i_k\in\{1,\ldots,n\}$, we have: $$P\left(\bigcap_{r=1}^kA_{i_r}\right)=\prod_{r=1}^kP(A_{i_r})$$
  \end{definition}
  \begin{definition}
    Let $(\Omega,\mathcal{A},P)$ be a probability space and $I$ be an arbitrary index set. We say that $\{A_i:i\in I\}\subset\mathcal{A}$ are \textit{independent events} if for any finite subset $A_{i_1},\ldots A_{i_r}$ of different events, we have: $$P\left(\bigcap_{r=1}^kA_{i_r}\right)=\prod_{r=1}^kP(A_{i_r})$$
  \end{definition}
  \subsection{Random variables and random vectors}
\end{multicols}
\end{document}