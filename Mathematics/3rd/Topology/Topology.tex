\documentclass[../../../main_math.tex]{subfiles}


\begin{document}
\renewcommand{\col}{\geo}
\begin{multicols}{2}[\section{Topology}]
  \subsection{Topological spaces}
  \subsubsection{Metric spaces}
  \begin{definition}
    Let $X$ be a set. A \emph{distance} (or \emph{metric}) in $X$ is a function $d:X\times X\rightarrow\RR $ such that $\forall x,y,z\in X$ the following properties are satisfied:
    \begin{enumerate}
      \item $d(x,y)=0\iff x=y$.
      \item $d(x,y)=d(y,x)$.
      \item $d(x,y)\leq d(x,z)+d(z,y)\quad$(\emph{triangular inequality}).
    \end{enumerate}
  \end{definition}
  \begin{definition}
    A \emph{metric space} is a pair $(X,d)$, where $X$ is a set and $d$ is a distance in $X$.
  \end{definition}
  \begin{proposition}
    Let $x,y\in\RR^n$ such that $x=(x_1,\ldots,x_n)$ and $y=(y_1,\ldots,y_n)$. The following functions are metrics in $\RR^n$.
    \begin{enumerate}
      \item \emph{Euclidean metric}: $$d(x,y)=\sqrt{\sum_{i=1}^n{(x_i-y_i)}^2}$$
      \item \emph{Taxicab metric}: $$d(x,y)=\sum_{i=1}^n|x_i-y_i|$$
      \item \emph{Maximum metric}: $$d(x,y)=\max\{|x_i-y_i|:i\in\{1,\ldots,n\}\}$$
    \end{enumerate}
  \end{proposition}
  \begin{proposition}
    Let $X$ be a set. Then, $(X,d)$ is a metric space, where: $$d(x,y)=
      \begin{cases}
        1 & \text{if }x\ne y \\
        0 & \text{if }x= y
      \end{cases}
    $$
    This metric $d$ is called \emph{discrete metric}.
  \end{proposition}
  \begin{definition}
    Let $(X,d)$ be a metric space $a\in X$ and $r\in\RR$. We define the \emph{ball} $B_d(a,r)$ of center $a$ and radius $r$ with the metric $(X,d)$ as: $$B_d(a,r)=\{x\in X:d(x,a)<r\}$$
  \end{definition}
  \begin{definition}
    Let $(X,d_X)$ and $(Y,d_Y)$ be two metric spaces and $f:(X,d_X)\rightarrow(Y,d_Y)$ be a function. We say that $f$ is continuous if $\forall a\in X$ and $\forall\varepsilon>0$, $\exists\delta>0$ such that $d_Y(f(x),f(a))<\varepsilon$ whenever $d_X(x,a)<\delta$ or, equivalently: $$f(B_{d_X}(a,\delta))\subseteq B_{d_Y}(f(a),\varepsilon)$$ which is equivalent to $B_{d_X}(a,\delta)\subseteq f^{-1}\left(B_{d_Y}(f(a),\varepsilon)\right)$.
  \end{definition}
  \begin{definition}
    Let $(X,d)$ be a metric space. We say that a subset $A\subseteq X$ is \emph{open} if $\forall a\in A$, $\exists\varepsilon>0$ such that $B_d(a,\varepsilon)\subseteq A$.
  \end{definition}
  \begin{proposition}
    Let $(X,d)$ be a metric space. Then:
    \begin{itemize}
      \item $\varnothing$ and $X$ are open sets.
      \item If $I$ is an arbitrary index set and $\{U_i:U_i\subseteq X\;\forall i\in I\}$ is a collection of open sets, then $\bigcup_{i\in I}U_i$ is an open set.
      \item If $\{U_i:U_i\subseteq X\;\forall i\in \{1,\ldots,n\}\}$ is a finite collection of open sets, then $\bigcap_{i=1}^nU_i$ is an open set.
    \end{itemize}
  \end{proposition}
  \begin{proposition}
    Let $(X,d)$ be a metric space and $x\in X$. Then, the ball $B_d(x,r)$ is open $\forall r\in\RR$.
  \end{proposition}
  \begin{proposition}
    Let $(X,d)$ be a metric space and $A\subseteq X$ be a subset of $X$. Then, $A$ is open if and only if $A=\bigcup_{i\in I}B_d(a_i,\varepsilon_i)$, where $I$ is an index set, $a_i\in A$ and $\varepsilon_i>0$ for all $i\in I$.
  \end{proposition}
  \begin{theorem}
    Let $(X,d_X)$ and $(Y,d_Y)$ be two metric spaces and $f:(X,d_X)\rightarrow(Y,d_Y)$ be a function. The following statements are equivalent:
    \begin{enumerate}
      \item $f$ is continuous.
      \item If $A\subseteq Y$ is open, then $f^{-1}(A)\subseteq X$ is also open.
    \end{enumerate}
  \end{theorem}
  \begin{proposition}
    Let $(X,d)$ be a metric space with $|X|\geq 2$ and $x,y\in X$. Then, $\exists\delta>0$ such that $x\in B_d(x,\delta)$, $y\in B_d(y,\delta)$ and $B_d(x,\delta)\cap B_d(y,\delta)=\varnothing$.
  \end{proposition}
  \subsubsection{Topological spaces}
  \begin{definition}[Topological space]
    Let $X$ be a set. A \emph{topology} $\topo$ on a set $X$ is a collection of subsets of $X$ (that is, $\topo\subseteq\mathcal{P}(X)$) satisfying the following properties:
    \begin{enumerate}
      \item $\varnothing, X\in\topo$.
      \item The intersection of any finite subcollection of $\topo$ is in $\topo$.
      \item The union of any subcollection of $\topo$ is in $\topo$.
    \end{enumerate}
    The ordered pair $(X,\topo)$ is called a \emph{topological space}\footnote{Sometimes, in order to simplify the notation, we will write $X$ instead of $(X,\topo)$ to denote the topological space $(X,\topo)$ as well as the set $X$.}. The elements of $X$ are called \emph{points} and the elements of $\topo$, \emph{open sets}.
  \end{definition}
  \begin{definition}
    Let $(X,\topo)$ and $(X,\topo')$ be topological spaces. We say that $\topo$ is \emph{finer} than $\topo'$ if $\topo'\subseteq\topo$.
  \end{definition}
  \begin{proposition}
    Let $X$ be a set, $p\in X$ be a point of $X$ and $d$ be a metric defined on $X$. Then, we can construct some topologies on $X$ as follows:
    \begin{itemize}
      \item \emph{Topology induced from the metric}: $$\topo:=\{U\subseteq X:U\text{ is open with the metric }d\}$$
      \item \emph{Trivial topology}: $\topo_\text{t}:=\{\varnothing,X\}$
      \item \emph{Discrete topology}: $\topo_\text{d}:=\mathcal{P}(X)$
      \item \emph{Cofinite topology}: $$\topo_\text{f}:=\{U\subseteq X:U=\varnothing\lor X\setminus U\text{ is finite}\}$$
      \item \emph{Cocountable topology}:
            \begin{equation*}
              \topo:=\{U\subseteq X:U=\varnothing\lor X\setminus U\text{ is finite or countable}\}
            \end{equation*}
      \item \emph{Particular point topology}: $$\topo:=\{U\subseteq X:U=\varnothing\lor p\in U\}$$
      \item \emph{Excluded point topology}: $$\topo:=\{U\subseteq X:U=X\lor p\notin U\}$$
      \item \emph{Sierpiński topology}: If $X=\{0,1\}$, $$\topo:=\{\varnothing,\{1\},\{0,1\}\}$$
    \end{itemize}
  \end{proposition}
  \begin{definition}
    Let $(X,\topo)$ be a topological space and $C\subseteq X$. We say that $C$ is \emph{closed} if $X\setminus C\in\topo$, that is, if $X\setminus C$ is open.
  \end{definition}
  \begin{definition}
    Let $(X,\topo)$ be a topological space and $A\subseteq X$. We say that $A$ is \emph{clopen} if it is both open and closed.
  \end{definition}
  \begin{proposition}
    Let $(X,\topo)$ be a topological space. Then:
    \begin{enumerate}
      \item $\varnothing$ and $X$ are closed.
      \item The union of any finite subcollection of closed sets in $(X,\topo)$ is closed in $(X,\topo)$.
      \item The intersection of any subcollection of closed sets in $(X,\topo)$ is closed in $(X,\topo)$.
    \end{enumerate}
  \end{proposition}
  \subsubsection{Basis for a topology}
  \begin{definition}
    Let $(X,\topo)$ be a topological space and $\mathcal{B}\subseteq\topo$ be a subset of open sets. We say that $\mathcal{B}$ is a \emph{basis} of $\topo$ if $\forall U\in\topo$ and $\forall x\in U$, $\exists B\in\mathcal{B}$ such that $x\in B\subseteq U$.
  \end{definition}
  \begin{proposition}
    Let $(X,\topo)$ be a topological space and $\mathcal{B}$ be a basis of $\topo$. Then, for all $U\in\topo$ we have: $$U=\bigcup_{x\in U}B_x$$ where $x\in B_x\subseteq U$ and $B_x\in\mathcal{B}$ $\forall x\in U$.
  \end{proposition}
  \begin{lemma}
    Let $(X,\topo)$ be a topological space, $\mathcal{B}\subseteq\topo$ be a basis of $\topo$ and $\{B_i\in\mathcal{B}:i=1,\ldots,n\}$ be a collection of elements of $\mathcal{B}$. Then, $\forall x\in\bigcap_{i=1}^nB_i$, $\exists B'\in\mathcal{B}$ such that $x\in B'\subseteq\bigcap_{i=1}^nB_i$.
  \end{lemma}
  \begin{proposition}
    Let $X$ be a set and $\mathcal{B}\subseteq\mathcal{P}(X)$ be a collection of subsets of $X$ such that:
    \begin{enumerate}
      \renewcommand{\labelenumi}{\alph{enumi})}
      \item $\displaystyle X=\bigcup_{B\in\mathcal{B}} B$
      \item $\forall U,V\in\mathcal{B}$ and  $\forall x\in U\cap V$, $\exists B\in\mathcal{B}$ such that $x\in B\subseteq U\cap V$.
    \end{enumerate}
    Then, there exists a unique topology $\topo$ of $X$ such that:
    \begin{enumerate}
      \item $\mathcal{B}$ is a basis of $\topo$.
      \item $\topo$ is the least finer topology that contains $\mathcal{B}$.
    \end{enumerate}
  \end{proposition}
  \begin{definition}
    Let $X$ be a set and $\mathcal{B}\subseteq\mathcal{P}(X)$ be a collection of subsets of $X$. The \emph{topology generated} by $\mathcal{B}$ is: $$\topo=\left\{U\subseteq X:U=\bigcup_{i\in I}B_i, B_i\in \mathcal{B}\ \forall i\in I\right\}$$ Or equivalently: $$\topo=\left\{U\subseteq X:\forall x\in U\;\exists B\in\mathcal{B}\text{ such that }x\in B\subseteq U\right\}$$
  \end{definition}
  \begin{definition}
    Let $$\mathcal{B}=\{[a,b)\subset\RR:a,b\in\RR, a<b\}$$
    We define the \emph{lower limit topology} as the topology generated by $\mathcal{B}$.
  \end{definition}
  \begin{proposition}
    The lower limit topology is finer that the usual topology of $\RR$.
  \end{proposition}
  \begin{definition}
    Let
    $$U_n=
      \begin{cases}
        \{n\}         & \text{if $n$ is odd}  \\
        \{n-1,n,n+1\} & \text{if $n$ is even}
      \end{cases}$$
    and $\mathcal{B}=\{U_n\subset \ZZ:n\in\ZZ\}$. We define the \emph{digital topology} as the topology generated by $\mathcal{B}$.
  \end{definition}
  \begin{definition}
    Let $(X,\topo)$ be a topological space and $\mathcal{S}\subseteq\topo$ be a subset. We say that $\mathcal{S}$ is a \emph{subbasis} of $\topo$ if $\forall U\in\topo$, $U$ can be written as a union of finite intersections of elements of $\mathcal{S}$.
  \end{definition}
  \begin{proposition}
    Let $X$ be a set and $\mathcal{S}\subseteq\mathcal{P}(X)$ such that $X=\bigcup_{S\in\mathcal{S}} S$. Then, there exists a unique topology $\topo$ of $X$ such that:
    \begin{enumerate}
      \item $\mathcal{S}$ is a subbasis of the topology $\topo$.
      \item $\topo$ is the least finer topology that contains $\mathcal{S}$.
    \end{enumerate}
  \end{proposition}
  \subsubsection{Interior, closure and boundary of a set}
  \begin{definition}[Interior]
    Let $(X,\topo)$ be a topological space and $A\subseteq X$ be a subset. The \emph{interior} of $A$, $\Int_{(X,\topo)} A$ or simply $\Int A$, is the largest open subset of $X$ contained in $A$.
  \end{definition}
  \begin{definition}[Closure]
    Let $(X,\topo)$ be a topological space and $A\subseteq X$ be a subset. The \emph{closure} of $A$, $\Cl_{(X,\topo)} A$ or simply $\Cl A$, is the smallest closed subset of $X$ containing $A$.
  \end{definition}
  \begin{proposition}
    Let $(X,\topo)$ be a topological space and $A\subseteq X$ be a subset. Then: $$\Int A=\bigcup_{\substack{U\subseteq A\\U\text{ is open}}}U\qquad\Cl A=\bigcap_{\substack{C\supseteq A\\C\text{ is closed}}}C$$
    Hence, we have the inclusions: $$\Int A\subseteq A\subseteq \Cl A$$
    And, furthermore:
    \begin{itemize}
      \item $\Int A=A\iff A$ is open
      \item $\Cl A=A\iff A$ is closed
    \end{itemize}
  \end{proposition}
  \begin{definition}
    Let $(X,\topo)$ be a topological space and $A\subseteq X$ be a subset. $A$ is called \emph{dense} in $(X,\topo)$ if $\forall U\in\topo$ with $U\ne\varnothing$ we have $U\cap A\ne\varnothing$.
  \end{definition}
  \begin{proposition}
    Let $(X,\topo)$ be a topological space and $A\subseteq X$ be a subset. Then, $A$ is dense in $(X,\topo)$ if and only if $\Cl A=X$.
  \end{proposition}
  \begin{proposition}
    Let $(X,\topo)$ be a topological space and $A\subseteq X$ be a subset. Then:
    \begin{itemize}
      \item If $U\subseteq A$ is open, then $U\subseteq\Int A$.
      \item If $C\supseteq A$ is closed, then $\Cl A\subseteq C$.
    \end{itemize}
  \end{proposition}
  \begin{definition}[Boundary]
    Let $(X,\topo)$ be a topological space and $A\subseteq X$ be a subset. The \emph{boundary} of $A$, $\Fr_{(X,\topo)}A$ or simply $\Fr A$, is: $$\Fr A:=\Cl(A)\cap\Cl(X\setminus A)$$
  \end{definition}
  \begin{proposition}
    Let $(X,\topo)$ be a topological space and $A\subseteq X$ be a subset. Then: $$X=\Int A\sqcup\Fr A\sqcup\Int(X\setminus A)$$
  \end{proposition}
  \begin{definition}
    Let $(X,\topo)$ be a topological space and $x\in X$. We say that $N\subseteq X$ is a \emph{neighbourhood} of $x$ if $\exists U\in\topo$ such that $x\in U\subseteq N$. We denote by $\mathcal{N}_x$ the set of all neighbourhoods in $(X,\topo)$ of $x$.
  \end{definition}
  \begin{definition}
    Let $(X,\topo)$ be a topological space and $A\subseteq X$ be a subset. We say that $x\in X$ is an \emph{interior point} of $A$ if $A$ is a neighbourhood of $x$.
  \end{definition}
  \begin{definition}
    Let $(X,\topo)$ be a topological space and $A\subseteq X$ be a subset. We say that $x\in X$ is an \emph{adherent point} of $A$ if $\forall N\in\mathcal{N}_x$ we have that $N\cap A\ne\varnothing$.
  \end{definition}
  \begin{proposition}
    Let $(X,\topo)$ be a topological space and $A\subseteq X$ be a subset. Then:
    \begin{enumerate}
      \item $\Int A$ is the set containing all the interior points of $A$.
      \item $\Cl A$ is the set containing all the adherent points of $A$.
    \end{enumerate}
  \end{proposition}
  \begin{proposition}
    Let $(X,\topo)$ be a topological space and $A,B\subseteq X$ be subsets.

    Properties regarding the interior:
    \begin{enumerate}[leftmargin=1.15cm]\renewcommand{\labelenumi}{1.\arabic{enumi}.}
      \item $\Int(\Int (A))=         \Int A$
      \item $A\subseteq B\implies    \Int A\subseteq\Int B$
      \item $\Int(X\setminus A)=     X\setminus \Cl A$
      \item $\Int(B\setminus A)=     \Int B\setminus \Cl A$
      \item $\Int(A\cap B)=          \Int A\cap\Int B$
      \item $\Int(A\cup B)\supseteq  \Int A\cup\Int B$
    \end{enumerate}
    Properties regarding the closure:
    \begin{enumerate}[leftmargin=1.15cm]\renewcommand{\labelenumi}{2.\arabic{enumi}.}
      \item $\Cl(\Cl (A))=          \Cl A$
      \item $A\subseteq B\implies   \Cl A\subseteq\Cl B$
      \item $\Cl(X\setminus A)=     X\setminus \Int A$
      \item $\Cl(A\cap B)\subseteq  \Cl A\cap\Cl B$
      \item $\Cl(A\cup B)=          \Cl A\cup\Cl B$
    \end{enumerate}
    Properties regarding the boundary:
    \begin{enumerate}[leftmargin=1.15cm]\renewcommand{\labelenumi}{3.\arabic{enumi}.}
      \item $\Fr A\cap\Int A=         \varnothing$
      \item $\Fr A=\Cl A\setminus\Int A$
      \item $\Fr A\cup\Int A=             \Cl A$
      \item $\Fr (A\cup B)\subseteq \Fr A\cup \Fr B$
      \item $\Fr (\Fr A)\subseteq \Fr A$
      \item $\Fr A\subseteq A\iff  A  \text{ is closed}$
      \item $\Fr A\cap A=\varnothing\iff  A\text{ is open}$
      \item $\Fr A=\varnothing\iff\text{$A$ is clopen}$
    \end{enumerate}
  \end{proposition}
  \begin{proposition}[Kuratowski's problem]
    Let $(X,\topo)$ be a topological space and $A\subseteq X$ be a subset. Then:
    \begin{gather*}
      \Cl(\Int(\Cl(\Int A)))=\Cl(\Int A)\\
      \Int(\Cl(\Int (\Cl A)))=\Int(\Cl A)
    \end{gather*}
  \end{proposition}
  \begin{definition}
    Let $(X,\topo)$ be a topological space and $A,B\subseteq X$ be subsets. We say that $A$ and $B$ are \emph{separated} if $$\Cl A\cap B=A\cap\Cl B=\varnothing$$
  \end{definition}
  \begin{definition}
    Let $(X,\topo)$ be a topological space and $A,B\subseteq X$ be subsets. We say that $A$ and $B$ are \emph{separated by closed neighbourhoods} if there are closed neighbourhoods $C_A$ and $C_B$ of $A$ and $B$ respectively, such that $C_A\cap C_B=\varnothing$.
  \end{definition}
  \begin{proposition}
    Let $(X,\topo)$ be a topological space and $A,B\subseteq X$ be subsets. Then, $A$ and $B$ are separated by closed neighbourhoods if and only if $\Cl A\cap\Cl B=\varnothing$.
  \end{proposition}
  \subsection{Functions between topological spaces}
  \begin{definition}[Continuous function]
    Let $(X,\topo_X)$ and $(Y,\topo_Y)$ be topological spaces and $f:(X,\topo_X)\rightarrow(Y,\topo_Y)$ be a function. We say that $f$ is continuous if for all $U\in\topo_Y$, we have $f^{-1}(U)\in\topo_X$.
  \end{definition}
  \begin{proposition}
    Let $(X,\topo_X)$ and $(Y,\topo_Y)$ be topological spaces and $f:(X,\topo_X)\rightarrow(Y,\topo_Y)$ be a function. We say that $f$ is continuous if and only if for all closed sets $C\subseteq Y$, we have $f^{-1}(C)\subseteq X$ is closed.
  \end{proposition}
  \begin{theorem}
    Let $(X,\topo_X)$ and $(Y,\topo_Y)$ be topological spaces, $f:(X,\topo_X)\rightarrow (Y,\topo_Y)$ be a function and $\mathcal{B}_Y$ be a basis of $\topo_Y$. Then: $$f\text{ is continuous}\iff f^{-1}(B)\in\topo_X\ \forall B\in \mathcal{B}_Y$$
  \end{theorem}
  \begin{theorem}
    Let $(X,\topo_X)$ and $(Y,\topo_Y)$ be topological spaces and $f:(X,\topo_X)\rightarrow (Y,\topo_Y)$ be a function. Then, the following statements are equivalent:
    \begin{enumerate}
      \item $f$ is continuous.
      \item $f^{-1}(\Int(B))\subseteq \Int(f^{-1}(B))$ for all subsets $B\subseteq Y$.
      \item $f(\Cl(A))\subseteq \Cl(f(A))$ for all subsets $A\subseteq X$.
    \end{enumerate}
  \end{theorem}
  \begin{theorem}
    Let $(X,\topo_X)$ and $(Y,\topo_Y)$ be topological spaces and $f:(X,\topo_X)\rightarrow (Y,\topo_Y)$ be a function. Then, $f$ is continuous if and only if $\forall x\in X$ and $\forall U\in\topo_Y$ such that $f(x)\in U$, there exists a neighbourhood $N$ of $x$ with $f(N)\subseteq U$.
  \end{theorem}
  \begin{proposition}
    Let $(X,\topo_X)$, $(Y,\topo_Y)$ and $(Z,\topo_Z)$ be topological spaces and $f:(X,\topo_X)\rightarrow (Y,\topo_Y)$, $g:(Y,\topo_Y)\rightarrow (Z,\topo_Z)$ be continuous functions. Then, $g\circ f:(X,\topo_X)\rightarrow (Z,\topo_Z)$ is continuous.
  \end{proposition}
  \begin{definition}
    Let $(X,\topo_X)$ and $(Y,\topo_Y)$ be topological spaces. A \emph{homeomorphism} between $(X,\topo_X)$ and $(Y,\topo_Y)$ is a bijective function that is continuous and whose inverse is also continuous. We say that $(X,\topo_X)$ and $(Y,\topo_Y)$ are \emph{homeomorphic}, and we denote it by $(X,\topo_X)\cong(Y,\topo_Y)$, if there exists a homeomorphism between them.
  \end{definition}
  \begin{definition}[Open function]
    Let $(X,\topo_X)$ and $(Y,\topo_Y)$ be topological spaces and $f:(X,\topo_X)\rightarrow (Y,\topo_Y)$ be a function. We say that $f$ is \emph{open} if $\forall U\in\topo_X$, we have $f(U)\in \topo_Y$.
  \end{definition}
  \begin{definition}[Closed function]
    Let $(X,\topo_X)$ and $(Y,\topo_Y)$ be topological spaces and $f:(X,\topo_X)\rightarrow (Y,\topo_Y)$ be a function. We say that $f$ is \emph{closed} if for all closed subsets $C\subseteq X$, we have that $f(C)$ is closed.
  \end{definition}
  \begin{theorem}
    Let $(X,\topo_X)$ and $(Y,\topo_Y)$ be topological spaces, $f:(X,\topo_X)\rightarrow (Y,\topo_Y)$ be a function and $\mathcal{B}_X$ be a basis of $\topo_X$. Then: $$f\text{ is open}\iff f(B)\in\topo_Y\ \forall B\in \mathcal{B}_X$$
  \end{theorem}
  \begin{theorem}
    Let $(X,\topo_X)$ and $(Y,\topo_Y)$ be topological spaces, $f:(X,\topo_X)\rightarrow (Y,\topo_Y)$ be a bijective function. Then: $$f\text{ is open}\iff f\text{ is closed}$$
  \end{theorem}
  \begin{proposition}
    Let $(X,\topo_X)$ and $(Y,\topo_Y)$ be topological spaces and $f:(X,\topo_X)\rightarrow (Y,\topo_Y)$ be a function. Then, the following statements are equivalent:
    \begin{enumerate}
      \item $f$ is open.
      \item $f(\Int(A))\subseteq \Int(f(A))$ for all subsets $A\subseteq X$.
    \end{enumerate}
  \end{proposition}
  \begin{proposition}
    Let $(X,\topo_X)$ and $(Y,\topo_Y)$ be topological spaces and $f:(X,\topo_X)\rightarrow (Y,\topo_Y)$ be a continuous bijective function. Then, the following statements are equivalent:
    \begin{enumerate}
      \item $f$ is a homeomorphism.
      \item $f$ is open.
      \item $f$ is closed.
    \end{enumerate}
  \end{proposition}
  \begin{proposition}
    Being homeomorphic as topological spaces is an equivalence relation.
  \end{proposition}
  \subsection{Subspaces}
  \subsubsection{Subspace topology}
  \begin{definition}
    Let $(X,\topo)$ be a topological space and $A\subseteq X$ be a subset. We define the following set: $$\topo_A=\{U\subseteq A:\exists V\in\topo\text{ such that }V\cap A=U\}$$ Then, $(A,\topo_A)$ is a topological space (called \emph{topological subspace} of $(X,\topo)$) and $\topo_A$ is called the \emph{subspace topology} on $A$. We will write $(A,\topo_A)\subseteq (X,\topo)$ to denote that $(A,\topo_A)$ is a topological subspace.
  \end{definition}
  \begin{proposition}
    Let $(X,\topo)$ be a topological space and $(A,\topo_A)\subseteq (X,\topo)$ be a topological subspace. Then, $C\subseteq A$ is closed on $(A,\topo_A)$ if and only if $C=K\cap A$, where $K\subseteq X$ is a closed subset on $(X,\topo)$.
  \end{proposition}
  \begin{proposition}
    Let $(X,\topo)$ be a topological space and $(A,\topo_A)\subseteq (X,\topo)$ be a topological subspace. Then:
    \begin{enumerate}
      \item If $A$ is open and $U\subseteq A$, then: $$U\in\topo_A\iff U\in\topo$$
      \item If $A$ is closed and $C\subseteq A$, then: $$C\text{ is closed on }(A,\topo_A)\iff C\text{ is closed on }(X,\topo)$$
    \end{enumerate}
  \end{proposition}
  \begin{proposition}
    Let $(X,\topo)$ be a topological space and $(A,\topo_A)\subseteq (X,\topo)$ be a topological subspace. Then, the inclusion $\iota:(A,\topo_A)\hookrightarrow (X,\topo)$ is continuous and $\topo_A$ is the least finer topology where $\iota$ is continuous.
  \end{proposition}
  \begin{corollary}
    Let $(X,\topo_X)$ and $(Y,\topo_Y)$ be topological spaces, $f:(X,\topo_X)\rightarrow (Y,\topo_Y)$ be a continuous function and $(A,\topo_A)\subseteq (X,\topo_X)$ be a topological subspace. Then, $f|_A$ is also continuous.
  \end{corollary}
  \begin{proposition}
    Let $(X,\topo)$ be a topological space, $\mathcal{B}$ be a basis of $\topo$ and $(A,\topo_A)\subseteq (X,\topo)$ be a topological subspace. Then, $$\mathcal{B}_A=\{B\cap A:B\in\mathcal{B}\}$$ is basis of $\topo_A$.
  \end{proposition}
  \begin{proposition}
    Let $(X,\topo_X)$ and $(Y,\topo_Y)$ be topological spaces and $(B,\topo_B)\subseteq (Y,\topo_Y)$ be a topological subspace. Let $f:(X,\topo_X)\rightarrow (B,\topo_B)$ be a function. Then, $f$ is continuous if and only if $\iota\circ f:(X,\topo_X)\rightarrow (B,\topo_B)\hookrightarrow (Y,\topo_Y)$ is continuous.
  \end{proposition}
  \begin{corollary}
    Let $(X,\topo_X)$ and $(Y,\topo_Y)$ be topological spaces and $f:(X,\topo_X)\rightarrow (Y,\topo_Y)$ be a continuous function. Then, $g:(X,\topo_X)\rightarrow (f(X),\topo_{f(X)})$ is also continuous.
  \end{corollary}
  \begin{proposition}
    Let $(X,\topo_X)$ and $(Y,\topo_Y)$ be topological spaces such that $X=A\cup B$, for some sets $A$, $B$. Consider a function $f:(X,\topo_X)\rightarrow (Y,\topo_Y)$ satisfying that that $f|_A$ and $f|_B$ are continuous. Then:
    \begin{enumerate}
      \item If $A$, $B$ are open, then $f$ is continuous.
      \item If $A$, $B$ are closed, then $f$ is continuous.
    \end{enumerate}
  \end{proposition}
  \subsubsection{Cantor set}
  \begin{definition}
    Let $C_0=[0,1]$. Define $I_1:=\left(\frac{1}{3},\frac{2}{3}\right)$ and $C_1:=C_0\setminus I_1$. Then, define $I_2:=I_1\cup\left(\frac{1}{9},\frac{2}{9}\right)\cup\left(\frac{7}{9},\frac{8}{9}\right)$ and $C_2:=C_0\setminus I_2$. In general, define:
    \begin{gather*}
      I_{n+1}=I_n\cup\left[\bigcup_{k=0}^{3^n-1}\left(\frac{3k+1}{3^{n+1}},\frac{3k+2}{3^{n+1}}\right)\right]\\
      C_{n+1}=C_0\setminus I_{n+1}
    \end{gather*}
    We define the \emph{Cantor set} $\mathcal{C}$ as: $$\mathcal{C}:=\bigcap_{n=0}^\infty C_n$$
  \end{definition}
  \begin{proposition}
    The Cantor set $\mathcal{C}$ can be expressed as: $$\mathcal{C}=\{x\in[0,1]:x_3\footnote{Here, $x_3$ mean the expression of $x$ in base 3.}\text{ does not contain the digit 1}\}$$
  \end{proposition}
  \begin{proposition}
    The Cantor set $\mathcal{C}$ satisfies the following properties:
    \begin{enumerate}
      \item $\mathcal{C}\ne\varnothing$.
      \item $\mathcal{C}$ is closed in $\RR$.
      \item $\mathcal{C}$ does not contain any interval of $\RR$.
      \item $\Int\mathcal{C}=\varnothing$.
      \item $\mathcal{C}$ does not have the discrete topology.
      \item $\mathcal{C}$ is not countable.
    \end{enumerate}
  \end{proposition}
  \subsection{Product topology}
  \subsubsection{Finite product}
  \begin{definition}
    Let $(X,\topo_X)$, $(Y,\topo_Y)$ be topological spaces. We define the \emph{product topology} on $X\times Y$, denoted by $\topo_{X\times Y}$, as the topology generated by: $$\mathcal{B}=\{U\times V:U\in\topo_X,V\in\topo_Y\}$$
  \end{definition}
  \begin{proposition}
    Let $(X,\topo_X)$, $(Y,\topo_Y)$ be topological spaces. Then, the projections
    \begin{align*}
      \pi_X:(X\times Y,\topo_{X\times Y})\longrightarrow (X,\topo_X) \\
      \pi_Y:(X\times Y,\topo_{X\times Y})\longrightarrow (Y,\topo_Y)
    \end{align*}
    are continuous and open.
  \end{proposition}
  \begin{proposition}
    Let $(X,\topo_X)$, $(Y,\topo_Y)$ be topological spaces. Then, $A\subseteq X\times Y$ is open on $(X\times Y,\topo_{X\times Y})$ if and only if $\forall a\in A$ there exist $U\in\topo_X$ and $V\in\topo_Y$ such that $a\in U\times V\subseteq A$.
  \end{proposition}
  \begin{proposition}
    Let $(X,\topo_X)$, $(Y,\topo_Y)$ be topological spaces. If $\mathcal{B}_X$ is a basis for $\topo_X$ and $\mathcal{B}_Y$ is a basis for $\topo_Y$, then $$\mathcal{B}=\{U\times V:U\in\mathcal{B}_X,V\in\mathcal{B}_Y\}$$
    is a basis for $\topo_{X\times Y}$.
  \end{proposition}
  \begin{proposition}
    Let $(X,\topo_X)$, $(Y,\topo_Y)$, $(Z,\topo_Z)$ be topological spaces and $f:(Z,\topo_Z)\rightarrow(X\times Y,\topo_{X\times Y})$ be a function. Then, $f$ is continuous if and only if $\pi_X\circ f$ and $\pi_Y\circ f$ are continuous.
  \end{proposition}
  \begin{proposition}
    Let $(X_i,\topo_{X_i})$, $(Y_i,\topo_{Y_i})$ be topological spaces and $f_i:(X_i,\topo_{X_i})\rightarrow(Y_i,\topo_{Y_i})$ be functions for $i=1,2$. Then,
    $$
      \function{f_1\times f_2}{(X_1\times X_2,\topo_{X_1\times X_2})}{(Y_1\times Y_2,\topo_{Y_1\times Y_2})}{(x_1,x_2)}{(f_1(x_1),f_2(x_2))}
    $$ is continuous if and only if $f_1$ and $f_2$ are both continuous.
  \end{proposition}
  \begin{proposition}
    Let $(X,\topo_X)$, $(Y,\topo_Y)$ be topological spaces and $A\subseteq X$, $B\subseteq Y$ be closed subsets. Then, $A\times B$ is closed.
  \end{proposition}
  \begin{proposition}
    Let $(X,\topo_X)$, $(Y,\topo_Y)$ be topological spaces and $(A,\topo_A)\subseteq (X,\topo_X)$, $(B,\topo_B)\subseteq (Y,\topo_Y)$ be topological subspaces. Then:
    \begin{enumerate}
      \item $\Int(A\times B)=\Int A\times \Int B$
      \item $\Cl(A\times B)=\Cl A\times \Cl B$
      \item $\Fr(A\times B)=(\Fr A\times\Cl B)\cup(\Cl A\times \Fr B)$
    \end{enumerate}
  \end{proposition}
  \subsubsection{Arbitrary product}
  \begin{definition}
    Let $I$ be an index set, $\{(X_i,\topo_{X_i}):i\in I\}$ be a collection of topological spaces and $X:=\prod_{i\in I}X_i$. We define the \emph{box topology} on $X$ as the topology generated by $$\mathcal{B}=\left\{\prod_{i\in I}U_i:U_i\in\topo_{X_i}\ \forall i\in I\right\}$$
  \end{definition}
  \begin{definition}
    Let $I$ be an index set, $\{(X_i,\topo_{X_i}):i\in I\}$ be a collection of topological spaces and $X:=\prod_{i\in I}X_i$. We define the \emph{infinite product topology} on $X$, denoted by $\topo_X$, as the topology generated by
    \begin{multline*}
      \mathcal{B}=\Bigg\{\prod_{i\in I}U_i:U_i\in\topo_{X_i}\ \forall i\in I\land U_i=X_i\text{ except for}\\\text{a finite number of indices}\Bigg\}
    \end{multline*}
  \end{definition}
  \begin{proposition}
    Let $I$ be an index set, $\{(X_i,\topo_{X_i}):i\in I\}$ be a collection of topological spaces and $X:=\prod_{i\in I}X_i$. Then, the projection $$\pi_{X_i}:\left(X,\topo_X\right)\longrightarrow (X_i,\topo_{X_i})$$
    is continuous and open for all $i\in I$.
  \end{proposition}
  \begin{proposition}
    Let $I$ be an index set, $\{(X_i,\topo_{X_i}):i\in I\}$ be a collection of topological spaces and $X:=\prod_{i\in I}X_i$. If $\mathcal{B}_{X_i}$ is a basis of $\topo_{X_i}$ $\forall i\in I$, then $$\mathcal{B}=\left\{\prod_{i\in I}U_i:U_i\in\mathcal{B}_{X_i}\ \forall i\in I\right\}$$
    is a basis of $\topo_X$.
  \end{proposition}
  \begin{proposition}
    Let $I$ be an index set, $\{(Y_i,\topo_{Y_i}):i\in I\}$ and $(X,\topo_X)$ be topological spaces, $Y:=\prod_{i\in I}Y_i$ and $f:(X,\topo_X)\rightarrow(Y,\topo_Y)$ be a function. Then, $f$ is continuous if and only if $\pi_{Y_i}\circ f$ is continuous for all $i\in I$.
  \end{proposition}
  \begin{proposition}
    Let $I$ be an index set, $\{(X_i,\topo_{X_i}):i\in I\}$ and $\{(Y_i,\topo_{Y_i}):i\in I\}$ be two collections of topological spaces, $X:=\prod_{i\in I}X_i$, $Y:=\prod_{i\in I}Y_i$ and $f_i:(X_i,\topo_{X_i})\rightarrow(Y_i,\topo_{Y_i})$ be a function for all $i\in I$. Then,
    $$
      \function{\displaystyle\prod_{i\in I}f_i}{\left(X,\topo_X\right)}{\left(Y,\topo_Y\right)}{(x_i)_{i\in I}}{(f_i(x_i))_{i\in I}}
    $$ is continuous if and only if $f_i$ is continuous $\forall i\in I$.
  \end{proposition}
  \begin{proposition}
    Let $(X_i,\topo_{X_i})$ be topological spaces and $(A_i,\topo_{A_i})\subseteq (X_i,\topo_{X_i})$ be topological subspaces for $i=1,\ldots, n$. Consider the following topological spaces:
    \begin{enumerate}
      \item The topological space created from the product of subspaces $A_i$.
      \item The topological space created from the subspace $\prod_{i=1}^nA_i$ of the product $\prod_{i=1}^nX_i$.
    \end{enumerate}
    Then, these topological spaces are the same.
  \end{proposition}
  \begin{proposition}
    Let $I$ be and index set, $(X_i,\topo_{X_i})$ be topological spaces $\forall i\in I$ and $A_i\subseteq X_i$ be subsets $\forall i\in I$. Let $X:=\prod_{i\in I}X_i$. Then, $\prod_{i\in I}A_i$ is dense in $(X,\topo_X)$ if and only if $A_i$ is dense in $(X_i,\topo_{X_i})$ $\forall i\in I$.
  \end{proposition}
  \begin{theorem}
    The function
    $$
      \function{\varphi}{\displaystyle\prod_{i=1}^\infty\{0,2\}}{\mathcal{C}}{(a_i)}{\displaystyle\sum_{i=1}^\infty\frac{a_i}{3^i}}
    $$
    is a homeomorphism.
  \end{theorem}
  \begin{definition}
    We define the \emph{$n-1$-th sphere} $\S^{n-1}\subset \RR^n$ as: $$\S^{n-1}:=\{x\in\RR^n:\|x\|=1\}$$
    We define the \emph{$n$-th ball} $B^n\subset \RR^n$ as: $$B^n:=\{x\in\RR^n:\|x\|^2<1\}$$
  \end{definition}
  \begin{definition}[Torus]
    We define the \emph{torus} $T^2\subset \RR^3$ of major radius $R$ and minor radius $r$ (see \cref{TOP_torus}) as: $$T^2:=\left\{(x,y,z)\in\RR^3:{\left(\sqrt{x^2+y^2}-R\right)}^2+z^2=r^2\right\}$$
  \end{definition}
  \begin{center}
    \begin{minipage}{\linewidth}
      \centering
      \includestandalone[mode=image|tex,width=0.65\linewidth]{Images/torus}
      \captionof{figure}{Torus $T^2$}
      \label{TOP_torus}
    \end{minipage}
  \end{center}
  \begin{proposition}
    With the ordinary topology of $\RR^n$ we have:
    \begin{itemize}
      \item $\S^n\setminus(0,\overset{(n)}{\ldots},0,1)\cong\RR^n$
      \item $\S^1\times \S^1\cong T^2$
    \end{itemize}
  \end{proposition}
  \subsection{Quotient topology}
  \subsubsection{Quotient topology}
  \begin{definition}
    Let $(X,\topo_X)$ be a topological space, $Y$ be a set and $f:X\rightarrow Y$ be a function. We define the \emph{quotient topology} on $Y$ defined by $f$ as: $$\topo_f:=\{U\subseteq Y:f^{-1}(U)\in\topo_X\}$$
  \end{definition}
  \begin{proposition}
    Let $(X,\topo_X)$, $(Y,\topo_f)$ be topological spaces, where $f:X\rightarrow Y$ is a function. Then, $f$ (thought as a function between topological spaces) is continuous and $\topo_f$ is the finest topology for which $f$ is continuous.
  \end{proposition}
  \begin{proposition}
    Let $(X,\topo_X)$, $(Y,\topo_f)$ be topological spaces, where $f:X\rightarrow Y$ is a function. Then, $C\subseteq Y$ is closed on $(Y,\topo_f)$ if and only if $f^{-1}(C)\subseteq X$ is closed on $(X,\topo_X)$.
  \end{proposition}
  \begin{proposition}
    Let $(X,\topo_X)$, $(Y,\topo_f)$ and $(Z,\topo_Z)$ be topological spaces, where $f:X\rightarrow Y$ is a function, and $h:(Y,\topo_f)\rightarrow(Z,\topo_Z)$ be a function. Then, $h$ is continuous if and only if $h\circ f$ is continuous.
  \end{proposition}
  \begin{definition}
    Let $(X,\topo_X)$ be a topological space and $f:X\rightarrow Y$ be a function. We say that $f$ is a \emph{quotient map} if it is surjective and $Y$ is equipped with the topology $\topo_f$.
  \end{definition}
  \begin{definition}
    Let $(X,\topo_X)$ be a topological space and $\sim$ be an equivalence relation on $X$. Consider the canonical function $f:X\rrightarrow\quot{X}{\!\sim}$. We define the \emph{quotient space} as $(\quot{X}{\!\sim},\topo_f)$.
  \end{definition}
  \begin{definition}
    Let $(X,\topo_X)$ be a topological space and $A\subseteq X$ be a subset. Consider the partition of $X$: $$X=A\sqcup\bigsqcup_{x\in X\setminus A}\{x\}$$
    and define the equivalence relation $\sim_A$ as follows: $$x\sim_A y\iff x,y\in A\lor x=y\in X\setminus A$$
    We define the \emph{quotient space of collapsing a set to a point} as $\quot{X}{A}:=\quot{X}{\!\sim_A}$ together with the quotient topology. We will write $[A]:=[a]$ $\forall a\in A$, which is well-defined.
  \end{definition}
  \begin{proposition}
    Let $(X,\topo_X)$ be a topological space, $A\subseteq X$ be a subset and $\pi:X\rrightarrow\quot{X}{A}$ be the projection. Then, for all $U\subseteq\quot{X}{A}$ we have:
    $$
      \pi^{-1}(U)=\begin{cases}
        \displaystyle\bigcup_{[x]\in U}\{x\} & \text{if }[A]\notin U \\
        \displaystyle A\cup \bigcup_{
          \begin{subarray}{c}
            [x]\in U \\
            [x]\ne [A]
          \end{subarray}
        }\{x\}                               & \text{if }[A]\in U
      \end{cases}
    $$
  \end{proposition}
  \subsubsection{Group actions on topological spaces}
  \begin{definition}
    Let $(X,\topo)$ be a topological space and $(G,\cdot)$ be a group. An \emph{action} of $(G,\cdot)$ on $(X,\topo)$ is a function:
    $$\function{f}{(G,\cdot)\times(X,\topo)}{(X,\topo)}{(g,x)}{f_g(x)}$$ where $f_g:(X,\topo)\rightarrow(X,\topo)$ is a homeomorphism for all $g\in G$ such that:
    \begin{enumerate}
      \item $f_e=\id$.
      \item $f_{g\cdot h}=f_g\circ f_h,\ \forall g,h\in G$.
    \end{enumerate}
    The pair $((X,\topo),f)$ is called a \emph{$(G,\cdot)$-space}.
  \end{definition}
  \begin{proposition}
    Let $((X,\topo),f)$ be a $G$-space. Consider the equivalence relation: $x\sim y\iff\exists g\in G$ such that $y=f_g(x)$\footnote{Note that this relation creates a partition of $X$ in terms of the orbits under $f$.}. Then, the set of orbits under $f$, $\quot{X}{G}:=\quot{X}{\!\sim}$, is a topological space together with the quotient topology. Furthermore, the projection $\pi:X\rrightarrow\quot{X}{G}$ is open.
  \end{proposition}
  \subsubsection{Homeomorphisms of quotient spaces}
  \begin{proposition}
    With the ordinary topology of $\RR^n$ we have: $$\quot{[0,1]}{\{0,1\}}\cong \S^1$$
  \end{proposition}
  \begin{proposition}\label{TOP_square-sim}
    Consider $X=[0,1]^2$. Define an equivalence relation $\sim$ in $X$ in the following ways:
    \begin{enumerate}
      \item $(0,t)\sim(1,t)\ \forall t\in[0,1]$. Then, $X\cong \S^1\times [0,1]$, which is a cylinder.
      \item $(0,t)\sim(1,t)$ and $(s,0)\sim(s,1)$ $\forall s,t\in[0,1]$. Then, $X\cong T^2$.
      \item $(0,t)\sim(1,1-t)\ \forall t\in[0,1]$. In that case, $X\cong \mathcal{M}$, where $\mathcal{M}$ is the \emph{Möbius band} (see \cref{TOP_moebius}).
      \item $(0,t)\sim(1,t)$ and $(s,0)\sim(1-s,1)$ $\forall s,t\in[0,1]$. In that case, $X\cong \mathcal{K}$, where $\mathcal{K}$ is the \emph{Klein bottle} (see \cref{TOP_klein}).
    \end{enumerate}
    \begin{center}
      \begin{minipage}{\linewidth}
        \centering
        \includestandalone[mode=image|tex,width=\linewidth]{Images/square}
        \captionof{figure}{Representation of the quotient spaces $\quot{{[0,1]}^2}{\sim}$, where $\sim$ is the equivalence relation defined on \cref{TOP_square-sim}}
        \label{TOP_fig-MK}
      \end{minipage}
    \end{center}
  \end{proposition}
  \begin{center}
    \begin{minipage}{\linewidth}
      \centering
      \includestandalone[mode=image|tex,width=0.65\linewidth]{Images/moebiusband}
      \captionof{figure}{Möbius band}
      \label{TOP_moebius}
    \end{minipage}
  \end{center}
  \begin{center}
    \begin{minipage}{\linewidth}
      \centering
      \includestandalone[mode=image|tex,width=0.65\linewidth]{Images/kleinbottle}
      \captionof{figure}{Klein bottle}
      \label{TOP_klein}
    \end{minipage}
  \end{center}
  \begin{proposition}\label{TOP_proj-sim}
    Let $\mathcal{P}_n(\RR)$ be the projective space of $\RR^{n+1}$. Consider the relation $\sim$ on $\RR^{n+1}$ such that $\vf{v}\sim -\vf{v}$ $\forall \vf{v}\in\RR^{n+1}$. Then: $$\mathcal{P}_n(\RR)\cong\quot{\S^n}{\sim}$$
  \end{proposition}
  \begin{proposition}\label{TOP_circle-sim}
    Consider the ball $B^2$ and the equivalence relation $\sim$ in $\Fr B^2=\S^1$ such that for all $(x,y)\in \Fr B^2$, $(x,y)\sim(x,-y)$. Then: $$\quot{B^2}{\sim}\cong \S^2$$
    \begin{center}
      \begin{minipage}{\linewidth}
        \centering
        \includestandalone[mode=image|tex,width=0.75\linewidth]{Images/circle}
        \captionof{figure}{Representation of the quotient space $\quot{\S^2}{\sim}$ (left), where $\sim$ is the equivalence relation defined on \cref{TOP_proj-sim} and the quotient space $\quot{B^2}{\sim}$ (right), where $\sim$ is the equivalence relation defined on \cref{TOP_circle-sim}}
        \label{TOP_fig-PS}
      \end{minipage}
    \end{center}
  \end{proposition}
  \subsection{Separation axioms}
  \begin{definition}[$T_0$ space]
    Let $(X,\topo)$ be a topological space. We say that $(X,\topo)$ is $T_0$\footnote{The letter $T$ comes from the German word ``Trennungsaxiom'' which means ``separation axioms''.} (or \emph{Kolmogorov}) if for any two distinct points of $X$, there exists an open set that contains one of them but not the other.
  \end{definition}
  \begin{definition}[$T_1$ space]
    Let $(X,\topo)$ be a topological space. We say that $(X,\topo)$ is $T_1$ (or \emph{Fréchet}) if for any two distinct points $x,y\in X$, there exists an open set $U\in\topo$ such that $x\in U$ and $y\notin U$.
  \end{definition}
  \begin{theorem}
    Let $(X,\topo)$ be a topological space. The statements following are equivalent:
    \begin{enumerate}
      \item $(X,\topo)$ is $T_1$.
      \item For all $x\in X$, $\{x\}=\bigcap_{N\in \mathcal{N}_x}N$.
      \item For all $x\in X$, $\{x\}$ is closed.
    \end{enumerate}
  \end{theorem}
  \begin{definition}[$T_2$ space]
    Let $(X,\topo)$ be a topological space. We say that $(X,\topo)$ is $T_2$ (or \emph{Hausdorff}) if for any two distinct points $x,y\in X$, there exist $U,V\in\topo$ such that $x\in U$, $y\in V$ and $U\cap V=\varnothing$.
  \end{definition}
  \begin{theorem}
    Let $(X,\topo)$ be a topological space. The statements following are equivalent:
    \begin{enumerate}
      \item $(X,\topo)$ is $T_2$.
      \item For all $x\in X$, $\{x\}=\bigcap_{N\in \mathcal{N}_x}\Cl(N)$.
      \item The diagonal $\Delta(X):=\{(x,x)\in X\times X\}\subset X\times X$ is closed.
    \end{enumerate}
  \end{theorem}
  \begin{proposition}
    Let $(X,\topo_X)$, $(Y,\topo_Y)$ be Hausdorff topological spaces. Then, $(X\times Y,\topo_{X\times Y})$ is Hausdorff.
  \end{proposition}
  \begin{definition}[$T_{2\frac{1}{2}}$ space]
    Let $(X,\topo)$ be a topological space. We say that $(X,\topo)$ is $T_{2\frac{1}{2}}$ if for any two distinct points $x,y\in X$, there exist open sets $U,V\in\topo$ separated by closed neighbourhoods ($\Cl U\cap \Cl V=\varnothing$) such that $x\in U$ and $y\in V$.
  \end{definition}
  \begin{definition}
    Let $(X,\topo)$ be a topological space. We say that $(X,\topo)$ is \emph{regular} if for all $x\in X$ and for all a closed sets $C\subseteq X$ such that $x\notin C$, there exist $U,V\in\topo$ such that $x\in U$, $C\subseteq V$ and $U\cap V=\varnothing$.
  \end{definition}
  \begin{definition}[$T_3$ space]
    Let $(X,\topo)$ be a topological space. We say that $(X,\topo)$ is $T_3$ if it is $T_1$ and regular.
  \end{definition}
  \begin{theorem}
    Let $(X,\topo)$ be a topological space. The statements following are equivalent:
    \begin{enumerate}
      \item $(X,\topo)$ is $T_3$.
      \item For all $x\in X$ and for all $U\in\topo$ such that $x\in U$, $\exists V\in\topo$ such that $x\in V\subseteq\Cl(V)\subseteq U$.
    \end{enumerate}
  \end{theorem}
  \begin{theorem}
    A subspace of a topological space $T_3$ is $T_3$.
  \end{theorem}
  \begin{theorem}
    The product of topological spaces $T_3$ is $T_3$.
  \end{theorem}
  \begin{definition}
    Let $(X,\topo)$ be a topological space. We say that $(X,\topo)$ is \emph{normal} if for all closed sets $C,K\subseteq X$ such that $C\cap K=\varnothing$, there exist $U,V\in\topo$ such that $C\subseteq U$, $K\subseteq V$ and $U\cap V=\varnothing$.
  \end{definition}
  \begin{definition}[$T_4$ space]
    Let $(X,\topo)$ be a topological space. We say that $(X,\topo)$ is $T_4$ if it is $T_1$ and normal.
  \end{definition}
  \begin{theorem}
    Let $(X,\topo)$ be a topological space. The statements following are equivalent:
    \begin{enumerate}
      \item $(X,\topo)$ is $T_4$.
      \item For all closed set $C\subseteq X$ and for all $U\in\topo$ such that $C\subseteq U$, $\exists V\in\topo$ such that $C\subseteq V\subseteq\Cl(V)\subseteq U$.
    \end{enumerate}
  \end{theorem}
  \begin{theorem}
    A closed subspace of a topological space $T_4$ is $T_4$.
  \end{theorem}
  \begin{theorem}
    If we also denote by $T_i$ the set of all topological spaces which are $T_i$, for $i\in\{0,1,2,2\frac{1}{2},3,4\}$, we have that: $$T_4\subset T_3\subset T_{2\frac{1}{2}}\subset T_2\subset T_1\subset T_0$$
  \end{theorem}
  \begin{lemma}[Urysohn's lemma]
    Let $(X,\topo)$ be a topological space $T_4$ and $A,B\subseteq X$ be closed sets. Then, there exists a continuous function $f:(X,\topo)\rightarrow[0,1]$ such that $A\subseteq f^{-1}(0)$ and $B\subseteq f^{-1}(1)$.
  \end{lemma}
  \begin{theorem}[Tietze extension theorem]
    Let $(X,\topo)$ be a topological space $T_4$, $C\subseteq X$ be a closed set and $f:C\rightarrow[0,1]$ be a continuous function. Then, there exists a continuous function $F:(X,\topo)\rightarrow[0,1]$ such that $F(x)=f(x)$ $\forall x\in C$.
  \end{theorem}
  \begin{definition}
    A topological space $(X,\topo)$ is said to be \emph{metrizable} if there is a metric $d$ such that the topology induced by $d$ is $\topo$.
  \end{definition}
  \begin{theorem}[Urysohn's metrization theorem]
    Let $(X,\topo)$ be a topological space $T_3$ such that it admits a countable basis of open sets. Then, $(X,\topo)$ is metrizable.
  \end{theorem}
  \subsection{Compactness}
  \begin{definition}
    We say that a property $P$ of a topological space is a \emph{topological property} if it is preserved under homeomorphisms. That is, if $(X,\topo_X)$, $(Y,\topo_Y)$ are homeomorphic topological spaces such that $(X,\topo_X)$ has the property $P$, then $(Y,\topo_Y)$ has the property $P$ too.
  \end{definition}
  \begin{proposition}
    The properties $T_i$ are topological properties for $i\in\{0,1, 2, 2\frac{1}{2}, 3, 4\}$.
  \end{proposition}
  \subsubsection{Covers}
  \begin{definition}[Cover]
    Let $(X,\topo)$ be a topological space. A \emph{cover} of $X$ is a collection $\{U_i:i\in I\}$ with $U_i\subseteq X$ $\forall i\in I$ such that $X=\bigcup_{i\in I}U_i$.
  \end{definition}
  \begin{definition}
    Let $(X,\topo)$ be a topological space and $U=\{U_i:i\in I\}$ be a cover of $X$.
    \begin{itemize}
      \item We say that $U$ is \emph{finite} if $I$ is finite.
      \item We say that $U$ is \emph{countable} if $I$ is countable.
      \item We say that $U$ is an \emph{open cover} if $U_i\in\topo$ $\forall i\in I$.
    \end{itemize}
  \end{definition}
  \begin{definition}
    Let $(X,\topo)$ be a topological space, $A\subseteq X$ be a subset and $U=\{U_i:i\in I\}$ with $U_i\subseteq X$ $\forall i\in I$. We say that $U$ is a \emph{cover} of $A$ if $A\subseteq\bigcup_{i\in I}U_i$.
  \end{definition}
  \begin{definition}
    Let $(X,\topo)$ be a topological space and $U=\{U_i:i\in I\}$ be a cover of $X$. A \emph{subcover} of $U$ is a collection $\{U_j:j\in J\}$ with $J\subseteq I$.
  \end{definition}
  \subsubsection{Compactness}
  \begin{definition}[Compact space]
    Let $(X,\topo)$ be a topological space. We say that $(X,\topo)$ is \emph{compact} if each of its open covers has a finite subcover.
  \end{definition}
  \begin{proposition}
    Let $(X,\topo)$ be a topological space. Then, $(X,\topo)$ is compact if and only if for any collection $C=\{C_i:i\in I\}$ of closed sets such that $\bigcap_{i\in I} C_i=\varnothing$, there exists a finite subcollection $\{C_{i_1},\ldots,C_{i_n}\}$ of $C$ such that $\bigcap_{j=1}^n C_{i_j}=\varnothing$
  \end{proposition}
  \begin{proposition}
    The \emph{compactness} is a topological property.
  \end{proposition}
  \begin{proposition}
    Let $(X,\topo_X)$, $(Y,\topo_Y)$ be topological spaces and $f:(X,\topo_X)\rightarrow(Y,\topo_Y)$ be a surjective continuous function. If $(X,\topo_X)$ is compact, then $(Y,\topo_Y)$ is also compact.
  \end{proposition}
  \begin{corollary}
    The quotient space of a compact space is compact.
  \end{corollary}
  \begin{definition}
    Let $(X,\topo)$ be a topological space and $A\subseteq X$ be a subset. We say that $A$ is a \emph{compact subset} of $X$ if $(A,\topo_A)$ is a compact space.
  \end{definition}
  \begin{lemma}
    Let $(X,\topo)$ be a topological space and $A\subseteq X$ be a subset. Then, $A$ is compact if and only if each open cover of $A$ in $(X,\topo)$ admits a finite subcover.
  \end{lemma}
  \begin{proposition}
    Let $(X,\topo)$ be a topological space and $\{K_i\subseteq X:i=1,\ldots,n\}$ be a collection of compact sets. Then, $\bigcup_{i=1}^nK_i$ is also compact.
  \end{proposition}
  \begin{theorem}
    Let $(X,\topo_X)$, $(Y,\topo_Y)$ be topological spaces, $f:(X,\topo_X)\rightarrow(Y,\topo_Y)$ be a continuous function and $A\subseteq X$ be a subset. If $A$ is compact, then $f(A)$ is also compact.
  \end{theorem}
  \begin{theorem}
    Let $(X,\topo)$ be a compact topological space and $C\subseteq X$ be a closed subset. Then, $C$ is compact.
  \end{theorem}
  \begin{theorem}
    Let $(X,\topo)$ be a Hausdorff topological space and $K\subseteq X$ be a compact subset. Then, $K$ is closed.
  \end{theorem}
  \begin{corollary}
    Let $(X,\topo)$ be a Hausdorff topological space and $\{K_i:i=1,\ldots,n\}$ be a collection of compact sets $K_i\subseteq X$, $i=1,\ldots,n$. Then, $\bigcap_{i\in I}K_i$ is compact.
  \end{corollary}
  \begin{corollary}
    Let $(X,\topo_X)$ be a compact topological space, $(Y,\topo_Y)$ be Hausdorff topological space and $f:(X,\topo_X)\rightarrow(Y,\topo_Y)$ be a continuous function. Then, $f$ is closed. Furthermore, if $f$ is bijective, then $f$ is a homeomorphism.
  \end{corollary}
  \begin{corollary}
    Let $(X,\topo)$ be a compact Hausdorff topological space and $\topo_1\subseteq\topo\subseteq\topo_2$. Then:
    \begin{itemize}
      \item If $(X,\topo_1)$ is Hausdorff, then $\id:(X,\topo)\rightarrow(X,\topo_1)$ is a homeomorphism and $\topo_1=\topo$.
      \item If $(X,\topo_2)$ is compact, then $\id:(X,\topo_2)\rightarrow(X,\topo)$ is a homeomorphism and $\topo_2=\topo$.
    \end{itemize}
  \end{corollary}
  \begin{proposition}
    Let $(X,\topo)$ be a compact Hausdorff topological space and $C\subseteq X$ be a closed subset. Then, $\quot{X}{C}$, together with the quotient topology, is compact Hausdorff.
  \end{proposition}
  \begin{proposition}
    Let $(X,\topo)$ be a Hausdorff topological space and $C,K\subseteq X$ be compact subsets. Then, there exist $U,V\in\topo$ such that $C\subseteq U$, $K\subseteq V$ and $U\cap V=\varnothing$.
  \end{proposition}
  \begin{corollary}
    Let $(X,\topo)$ be a compact Hausdorff topological space. Then, $(X,\topo)$ is $T_4$.
  \end{corollary}
  \subsubsection{Compactness of the product}
  \begin{lemma}
    Let $(X,\topo_X)$, $(Y,\topo_Y)$ be topological spaces such that $(Y,\topo_Y)$ is compact and $U\in\topo_{X\times Y}$ be an open set such that $\{x\}\times Y\subseteq U\subseteq X\times Y$. Then, $\exists V\in\topo_X$ such that $x\in V$ and $\{x\}\times Y\subseteq V\times Y\subseteq U$.
  \end{lemma}
  \begin{corollary}
    Let $(X,\topo_X)$, $(Y,\topo_Y)$ be topological spaces such that $(Y,\topo_Y)$ is compact. Then, the projection $\pi_X:(X\times Y,\topo_{X\times Y})\rightarrow (X,\topo_X)$ is closed.
  \end{corollary}
  \begin{theorem}[Tychonoff's theorem]
    Let $I$ be an index set, $\{(X_i,\topo_{X_i}):i\in I\}$ be a collection of topological spaces and $X:=\prod_{i\in I}X_i$. Then, $(X,\topo_X)$ is compact if and only if $(X_i,\topo_{X_i})$ is compact $\forall i\in I$.
  \end{theorem}
  \begin{axiom}[Axiom of choice]
    The Cartesian product of a collection of non-empty sets is non-empty.
  \end{axiom}
  \begin{theorem}[Kelley's theorem]
    Tychonoff's theorem implies the axiom of choice.
  \end{theorem}
  \subsubsection{Alexandroff extension}
  \begin{definition}
    Let $(X,\topo_X)$, $(Y,\topo_Y)$ be topological spaces and $f:(X,\topo_X)\rightarrow(Y,\topo_Y)$ be a continuous function. We say that $f$ is a \emph{topological embedding} if $f$ yields a homeomorphism between $(X,\topo_X)$ and $f(X)$ together with the subspace topology inherited from $(Y,\topo_Y)$.
  \end{definition}
  \begin{definition}
    Let $(X,\topo)$ be topological space and $X^*:=X\cup\{\infty\}$. We define the following set:
    $$\topo^*:=\{U\subseteq X^*:U\in\topo\lor(\infty\in U\land X^*\setminus U\text{ is compact})\}$$
  \end{definition}
  \begin{theorem}[One-point compactification]\label{TOP_alex}
    Let $(X,\topo)$ be Hausdorff topological space. Then, $(X^*,\topo^*)$ is a compact topological space, called \emph{one-point compactification} of $(X,\topo)$.
  \end{theorem}
  \begin{proposition}
    Let $(X,\topo)$ be Hausdorff topological space. Then, the inclusion $\iota: (X,\topo)\rightarrow(X^*,\topo^*)$ is a topological embedding.
  \end{proposition}
  \begin{definition}
    Let $(X,\topo)$ be topological space and $P$ be a property. We say that $(X,\topo)$ satisfies \emph{locally} $P$ if $\forall x\in X$ and $\forall U\in\topo$ such that $x\in U$, there exists a neighbourhood $N\in\mathcal{N}_x$, with $x\in N\subseteq U$, that satisfies $P$.
  \end{definition}
  \begin{definition}
    Let $(X,\topo)$ be topological space. We say that $(X,\topo)$ is \emph{locally compact} if $\forall x\in X$ and $\forall U\in\topo$ such that $x\in U$, there exists a compact neighbourhood $N\in\mathcal{N}_x$ such that $x\in N\subseteq U$.
  \end{definition}
  \begin{proposition}
    The local compactness is a topological property.
  \end{proposition}
  \begin{definition}
    Let $(X,\topo)$ be topological space. $(X,\topo)$ is locally compact Hausdorff if and only if $(X^*,\topo^*)$ is compact Hausdorff.
  \end{definition}
  \begin{proposition}
    We the usual topology, $\QQ\subset \RR$ is not locally compact.
  \end{proposition}
  \begin{theorem}
    Let $(X,\topo)$ be a locally compact Hausdorff topological space. Then, $(X,\topo)$ is $T_3$.
  \end{theorem}
  \subsubsection{Compactness of \texorpdfstring{$\RR^n$}{Rn}}
  \begin{theorem}[Heine-Borel theorem]
    Let $a,b\in\RR$ with $a<b$. Then, $[a,b]\subset\RR$ is compact.
  \end{theorem}
  \begin{theorem}[Heine-Borel theorem]
    Consider $\RR^n$ with the usual topology and $A\subseteq \RR^n$. Then, $A$ is compact if and only if $A$ is closed and bounded.
  \end{theorem}
  \begin{lemma}
    Let $K\subset\RR$ be a compact subset. Then, $\exists m,M\in K$ such that $m\leq k\leq M$ $\forall k\in K$.
  \end{lemma}
  \begin{theorem}[Weierstra\ss' theorem]
    Let $(X,\topo)$ be a compact topological space and $f:X\rightarrow\RR$ be a continuous function. Then, $f$ attains a maximum and a minimum.
  \end{theorem}
  \begin{proposition}
    $\S^n$, $T^2$, $\mathcal{M}$, $\mathcal{K}$ and $\mathcal{P}_n(\RR)$ are compact.
  \end{proposition}
  \subsection{Connectedness}
  \subsubsection{Connectedness}
  \begin{definition}[Connected space]
    Let $(X,\topo)$ be a topological space. We say that $(X,\topo)$ is \emph{connected} if do not exist non-empty open sets $U,V\in\topo$ such that $X=U\sqcup V$. Otherwise, that is, if there are non-empty open sets $U,V\in\topo$ such that $X=U\sqcup V$, we say that $(X,\topo)$ is \emph{disconnected}.
  \end{definition}
  \begin{proposition}
    Let $(X,\topo)$ be a topological space. The following statements are equivalent:
    \begin{enumerate}
      \item $(X,\topo)$ is connected.
      \item There are no non-empty closed sets $C,D\subset X$ such that $X=C\sqcup D$.
      \item There isn't a non-empty clopen set $U\subset X$.
    \end{enumerate}
  \end{proposition}
  \begin{definition}
    Let $(X,\topo)$ be a topological space and $A\subseteq X$ be a subset. We say that $A$ is a \emph{connected subset} of $X$ if $(A,\topo_A)$ is a connected space.
  \end{definition}
  \begin{proposition}
    Let $(X,\topo)$ be a topological space and $A\subseteq X$ be a subset. $A$ is connected if and only if there are no open sets $U,V\in \topo$ such that $A\subseteq U\cup V$, $A\cap U\ne\varnothing$, $A\cap V\ne\varnothing$ and $A\cap U\cap V=\varnothing$.
  \end{proposition}
  \begin{proposition}
    The connectedness is a topological property.
  \end{proposition}
  \begin{theorem}
    Let $(X,\topo_X)$, $(Y,\topo_Y)$ be a topological spaces such that $(X,\topo_X)$ is connected and $f:(X,\topo_X)\rightarrow(Y,\topo_Y)$ be a continuous function. Then, $f(X)\subset Y$ is connected.
  \end{theorem}
  \begin{corollary}
    The quotient space of a connected space is connected.
  \end{corollary}
  \begin{lemma}
    Let $(X,\topo)$ be a topological space and $C,D\subseteq X$ be subsets such that $C\subseteq D$ and $C$ is connected. Suppose that $D$ is disconnected and so that there exist non-empty open sets $U,V\in\topo$ such that $D\subseteq U\cup V$, $D\cap U\ne\varnothing$, $D\cap V\ne\varnothing$ and $D\cap U\cap V=\varnothing$. Then, either $C\subseteq U$ or $C\subseteq V$.
  \end{lemma}
  \begin{proposition}
    Let $(X,\topo)$ be a topological space and $\{C_i:i\in I\}$ be a collection of connected subsets of $X$ such that $\bigcap_{i\in I} C_i\ne\varnothing$. Then, $\bigcup_{i\in I} C_i$ is connected.
  \end{proposition}
  \begin{theorem}
    Let $(X_i,\topo_{X_i})$ be connected topological spaces for $i=1,\ldots,n$. Then, $\prod_{i=1}^n(X_i,\topo_{X_i})$ is connected.
  \end{theorem}
  \begin{theorem}
    Let $(X,\topo)$ be a topological space and $C\subseteq X$ be a connected subset. If $A\subseteq X$ is such that $C\subseteq A\subseteq \Cl C$, then $A$ is connected.
  \end{theorem}
  \begin{proposition}
    Let $(X,\topo)$ be a connected topological space and $A\subset X$ be a non-empty subset. Then, $\Fr A\ne\varnothing$.
  \end{proposition}
  \begin{proposition}
    Let $(X,\topo)$ be a topological space and $C\subset X$ be a connected non-empty subset. If $A\subseteq X$ is such that $C\cap A\ne\varnothing$ and $C\cap(X\setminus A)\ne\varnothing$, then $C\cap\Fr A\ne\varnothing$.
  \end{proposition}
  \begin{definition}
    Let $(X,\topo)$ be a topological space with $|X|>1$. We say that $(X,\topo)$ is \emph{totally disconnected} if all subsets with cardinal greater than 1 are disconnected.
  \end{definition}
  \begin{proposition}
    Let $(X,\topo)$ be a topological space with $|X|>1$. Then, $(X,\topo)$ is totally disconnected if and only if $\forall x,y\in X$, $x\ne y$, $\exists U,V\in\topo$ such that $x\in U$, $y\in V$ and $X=U\sqcup V$.
  \end{proposition}
  \begin{proposition}
    Let $(X,\topo_\text{d})$ be a topological space with $|X|>1$. Then, $(X,\topo_\text{d})$ is totally disconnected.
  \end{proposition}
  \begin{proposition}
    $\QQ\subseteq \RR$ with the usual topology is totally disconnected.
  \end{proposition}
  \subsubsection{Connectedness of \texorpdfstring{$\RR^n$}{Rn}}
  \begin{theorem}
    Consider $\RR$ together with the usual topology and let $a,b\in\RR$. Then, $[a,b]$ is connected.
  \end{theorem}
  \begin{theorem}
    Consider $\RR$ together with the usual topology. Then, $\RR$ is connected.
  \end{theorem}
  \begin{theorem}
    Consider $\RR$ together with the usual topology and let $A\subseteq \RR$ be a subset. Then: $$A\text{ is connected}\iff A\text{ is an interval}$$
  \end{theorem}
  \begin{theorem}[Intermediate value theorem]
    Let $(X,\topo)$ be a connected topological space, $f:(X,\topo)\rightarrow\RR$ be a continuous function. Let $p,q\in \im f$ and $r\in \RR$ be such that $p\leq r\leq q$. Then, $r\in \im f$.
  \end{theorem}
  \begin{corollary}[Bolzano's theorem]
    Let $f:[a,b]\rightarrow\RR$ be a continuous function such that $f(a)f(b)\leq 0$. Then, $\exists r\in [a,b]$ such that $f(r)=0$.
  \end{corollary}
  \begin{theorem}[Brouwer's fixed-point theorem]
    Let $\overline{B}^n\subset\RR$ be a closed $n$-th ball and $\vf{f}:\overline{B}^n\rightarrow\overline{B}^n$ be a continuous function. Then, $\vf{f}$ has a fixed point.
  \end{theorem}
  \begin{theorem}[Borsuk-Ulam theorem]
    Let $\vf{f}:\S^n\rightarrow\RR^n$ be a continuous function. Then, $\exists x\in \S^n$ such that $\vf{f}(x)=\vf{f}(-x)$.
  \end{theorem}
  \subsubsection{Connected components}
  \begin{definition}
    Let $(X,\topo)$ be a topological space. We define the relation $\sim$ in $(X,\topo)$ as $\forall x,y\in X$, $x\sim y$ if and only if there exists a connected subset $C\subseteq X$ such that $x,y\in C$.
  \end{definition}
  \begin{proposition}
    Let $(X,\topo)$ be a topological space. The relation $\sim$ is an equivalence relation.
  \end{proposition}
  \begin{definition}
    Let $(X,\topo)$ be a topological space with the relation $\sim$. We define the \emph{connected components} of $(X,\topo)$ as the equivalence classes under $\sim$.
  \end{definition}
  \begin{proposition}
    Let $(X,\topo)$ be a topological space with the relation $\sim$. Then:
    \begin{enumerate}
      \item Each connected component $C\subseteq X$ is connected. Moreover, if $p\in C$, $C$ is the maximal connected subset that contains $p$.
      \item The connected components are pairwise disjoint.
      \item If $A\subseteq X$ is a connected subspace, then $A\subseteq C$ for some connected component $C$ of $(X,\topo)$.
      \item The connected components are closed.
      \item If there is a finite number of connected components, then they are open.
    \end{enumerate}
  \end{proposition}
  \begin{theorem}
    Let $(X,\topo_X)$, $(Y,\topo_Y)$ be topological spaces, $f:(X,\topo_X)\rightarrow(Y,\topo_Y)$ be a continuous function and $C\subseteq X$ be a connected component. Then, $f(C)\subseteq D$, where $D$ is a connected component of $(Y,\topo_Y)$. Furthermore, if $f$ is a homeomorphism, $f(C)$ is a connected component of $(Y,\topo_Y)$.
  \end{theorem}
  \begin{corollary}
    Let $(X,\topo_X)$, $(Y,\topo_Y)$ be homeomorphic topological spaces. Then, they have the same number of connected components.
  \end{corollary}
  \begin{definition}
    Let $(X,\topo)$ be a topological space. We say that $(X,\topo)$ is \emph{locally connected} if $\forall x\in X$ and $\forall U\in\topo$ such that $x\in U$, there exists a connected neighbourhood $N\in\mathcal{N}_x$ such that $x\in N\subseteq U$.
  \end{definition}
  \begin{proposition}
    Let $(X,\topo)$ be a locally connected topological space. Then, the connected components are open.
  \end{proposition}
  \subsubsection{Path connectedness}
  \begin{definition}
    Let $(X,\topo)$ be a topological space. A \emph{path} in $(X,\topo)$ is a continuous function $\gamma:[0,1]\rightarrow (X,\topo)$. $\gamma(0)$ is called \emph{initial point} of the path and $\gamma(1)$, \emph{terminal point}.
  \end{definition}
  \begin{definition}
    Let $(X,\topo)$ be a topological space and $x,y\in X$. A \emph{path from $x$ to $y$} is a path whose initial point is $x$ and whose terminal point is $y$. If $x=y$, we say that the path is a \emph{loop}.
  \end{definition}
  \begin{proposition}
    Let $(X,\topo)$ be a topological space and $\gamma$ be a path in $(X,\topo)$ and $x,y\in X$. Then:
    \begin{enumerate}
      \item $\im(\gamma)$ is a connected subspace of $(X,\topo)$. Therefore, the initial and terminal points of $\gamma$ are in the same connected component.
      \item If $A\subseteq X$ is a subset satisfying $\gamma(0)\in A$ and $\gamma(1)\notin A$, then $\exists r\in[0,1]$ such that $\gamma(r)\in\Fr A$.
      \item If $\gamma(t)$ is a path from $x$ to $y$, then  $\gamma(1-t)$ is a path from $y$ to $x$
    \end{enumerate}
  \end{proposition}
  \begin{proposition}
    Let $(X,\topo)$ be a topological space $x,y,z\in X$, $\gamma_1$ be a path from $x$ to $y$ and $\gamma_2$ be a path from $y$ to $z$. Then, $$(\gamma_1+\gamma_2)(t):=
      \begin{cases}
        \gamma_1(2t)   & \text{if }0\leq t\leq 1/2 \\
        \gamma_2(2t-1) & \text{if }1/2 <t\leq 1
      \end{cases}
    $$
    is a path from $x$ to $z$.
  \end{proposition}
  \begin{definition}
    Let $(X,\topo)$ be a topological space. We say that $(X,\topo)$ is \emph{path-connected} if for all $x,y\in X$, there exists a path in $(X,\topo)$ from $x$ to $y$.
  \end{definition}
  \begin{proposition}
    The path-connectedness is a topological property.
  \end{proposition}
  \begin{proposition}
    Let $(X,\topo_X)$, $(Y,\topo_Y)$ be topological space such that $(X,\topo_X)$ is path-connected, and $f:(X,\topo_X)\rightarrow(Y,\topo_Y)$ be a continuous function. Then $f(X)\subseteq Y$ is path-connected.
  \end{proposition}
  \begin{corollary}
    The quotient space of a path-connected space is path-connected.
  \end{corollary}
  \begin{definition}
    Let $(X,\topo)$ be a topological space and $A\subseteq X$. We say that $A$ is \emph{path-connected} if $A$, together with the subspace topology, is path-connected.
  \end{definition}
  \begin{proposition}
    Let $(X,\topo)$ be a topological space and $\{C_i:i\in I\}$ be a collection of path-connected subsets of $X$ such that $\bigcap_{i\in I} C_i\ne\varnothing$. Then, $\bigcup_{i\in I} C_i$ is path-connected.
  \end{proposition}
  \begin{theorem}
    Let $(X_i,\topo_{X_i})$ be path-connected topological spaces for $i=1,\ldots,n$. Then, $\prod_{i=1}^n(X_i,\topo_{X_i})$ is path-connected.
  \end{theorem}
  \begin{theorem}
    Let $(X,\topo)$ be a path-connected topological space. Then, $(X,\topo)$ is connected.
  \end{theorem}
  \begin{definition}
    Let $(X,\topo)$ be a topological space. We define the relation $\sim_\text{p}$ in $(X,\topo)$ as $\forall x,y\in X$, $x\sim_\text{p} y$ if and only if there exists a path from $x$ to $y$.
  \end{definition}
  \begin{proposition}
    Let $(X,\topo)$ be a topological space. The relation $\sim_\text{p}$ is an equivalence relation.
  \end{proposition}
  \begin{definition}
    Let $(X,\topo)$ be a topological space with the relation $\sim_\text{p}$. We define the \emph{path-connected components} of $(X,\topo)$ as the equivalence classes under $\sim_\text{p}$.
  \end{definition}
  \begin{proposition}
    Let $(X,\topo)$ be a topological space with the relation $\sim_\text{p}$. Then:
    \begin{enumerate}
      \item Each path-connected component $C\subseteq X$ is path-connected. Moreover, if $p\in C$, $C$ is the maximal path-connected subset that contains $p$.
      \item The path-connected components are pairwise disjoint.
    \end{enumerate}
  \end{proposition}
  \begin{definition}
    Let $(X,\topo)$ be a topological space. We say that $(X,\topo)$ is \emph{locally path-connected} if $\forall x\in X$ and $\forall U\in\topo$ such that $x\in U$, there exists a path-connected neighbourhood $N\in\mathcal{N}_x$ such that $x\in N\subseteq U$.
  \end{definition}
  \begin{theorem}
    Let $(X,\topo)$ be a locally path-connected topological space. Then, the connected components and the path-connected components of $(X,\topo)$ are the same.
  \end{theorem}
  \begin{proposition}
    $\RR^n$, $\S^n$, $T^2$, $\mathcal{M}$, $\mathcal{K}$ and $\mathcal{P}_n(\RR)$ are connected and path-connected.
  \end{proposition}
  \begin{definition}
    Let $(X,\topo)$ be a topological space. We say that $(X,\topo)$ is \emph{simply connected} if every path between two points can be continuously transformed into any other such path while preserving the two endpoints in question\footnote{Roughly speaking this definition says that a sumply connected topological space doesn't have \emph{holes}.}.
  \end{definition}
  \subsection{Topological manifolds}
  \subsubsection{Topological manifolds}
  \begin{definition}[Topological manifold]
    A \emph{topological manifold} of dimension $m$, abbreviated as $m$-manifold, is a topological space $(M,\topo)$ such that:
    \begin{enumerate}
      \item $(M,\topo)$ is Hausdorff.
      \item $\topo$ admits a countable basis.
      \item $\forall x\in M$ $\exists N\in\mathcal{N}_x$ such that $N\cong\RR^n$.
    \end{enumerate}
  \end{definition}
  \begin{proposition}
    Let $(M,\topo)$ be a $m$-manifold. Then:
    \begin{enumerate}
      \item $(M,\topo)$ is locally connected.
      \item $(M,\topo)$ is locally path-connected.
      \item $(M,\topo)$ is locally compact.
    \end{enumerate}
  \end{proposition}
  \begin{corollary}
    Let $(M,\topo)$ be a $m$-manifold, $I$ be a finite or countable set and $\{M_i:i\in I\}$ be the connected components of $(M,\topo)$. Then, $M_i\subseteq M$ are clopen and $M\cong\bigsqcup_{i\in I}M_i$. Moreover for all $i\in I$, $M_i$ is a manifold itself of dimension $m$. Finally, if $I$ is finite and $(M,\topo)$ is compact, $M_i$ are compact connected manifolds $\forall i\in I$.
  \end{corollary}
  \begin{proposition}
    Being a topological manifold is a topological property.
  \end{proposition}
  \begin{definition}
    Let $(M,\topo)$ be a $m$-manifold. A \emph{coordinate chart} is a pair $(U,\varphi)$, where $U\in\topo$ and $\varphi:U\rightarrow\RR^n$ is a homeomorphism. A collection $\{(U_i,\varphi_i):i\in I\}$ of coordinate charts is called an \emph{atlas} if $M=\bigcup_{i\in I}U_i$.
  \end{definition}
  \begin{definition}
    Let $(M,\topo)$ be a $m$-manifold and $\{(U_i,\varphi_i):i\in I\}$ be an atlas. For all $i,j\in I$ such that $U_i\cap U_j\ne \varnothing$, we define the following homeomorphism: $$
      \begin{array}{r@{\hspace{0.5\tabcolsep}}c@{\hspace{0.5\tabcolsep}}c@{\hspace{0.5\tabcolsep}}c@{\hspace{0.5\tabcolsep}}c@{\hspace{0.5\tabcolsep}}c}
        \phi_{ij}: & \varphi_j(U_i\cap U_j) & \longrightarrow & U_i\cap U_j         & \longrightarrow & \varphi_i(U_i\cap U_j)                    \\
                   & x                      & \longmapsto     & {\varphi_j}^{-1}(x) & \longmapsto     & \varphi_i\left({\varphi_j}^{-1}(x)\right)
      \end{array}$$
    That is, $\phi_{ij}=\varphi_i\circ{\varphi_j}^{-1}|_{U_i\cap U_j}$. These functions are called \emph{transition functions}.
    \begin{itemize}
      \item If $\phi_{ij}$ is a piecewise linear function $\forall i,j\in I$, we say that $M$ is a \emph{piecewise linear manifold}.
      \item If $\phi_{ij}$ is a differentiable function $\forall i,j\in I$, we say that $M$ is a \emph{differentiable manifold}.
    \end{itemize}
  \end{definition}
  \begin{proposition}
    $\RR^n$, $\S^n$ and $\mathcal{P}_n(\RR)$ are $n$-manifolds.
  \end{proposition}
  \begin{proposition}
    $T^2$, $\mathcal{M}$ and $\mathcal{K}$ are compact $2$-manifolds.
  \end{proposition}
  \begin{proposition}
    Let $(M,\topo_M)$ be a $m$-manifold and $(N,\topo_N)$ be a $n$-manifold. Then, $(M\times N,\topo_{M\times N})$ is a $m+n$-manifold.
  \end{proposition}
  \begin{definition}[Connected sum]
    Let $(M_1,\topo_{M_1})$ and $(M_2,\topo_{M_2})$ be two $m$-manifolds, $p_1\in M_1$, $p_2\in M_2$ and $(U_1,\varphi_1)$, $(U_2,\varphi_2)$ be coordinate charts such that $p_i\in U_i$ and $\varphi_i(p_i)=0$ for $i=1,2$. Let $\varepsilon>0$ be such that $B(0,2\varepsilon)\subseteq \varphi(U_1)\cap \varphi(U_2)$ and $R:=B(0,2\varepsilon)\setminus\Cl (B(0,\varepsilon))$. Now consider the following homeomorphism:
    $$\function{\psi}{R}{R}{(x_1,\ldots,x_n)}{\displaystyle\frac{2\varepsilon^2}{{x_1}^2+\cdots+{x_n}^2}(x_1,\ldots,x_n)}$$
    Then, note that for $i=1,2$, ${M_i}':=M_i\setminus\Cl(\psi^{-1}(B(0,\varepsilon)))$ is a $m$-manifold. We define the \emph{connected sum} of $(M_1,\topo_{M_1})$ and $(M_2,\topo_{M_2})$ as: $$M_1\conn M_2:=\quot{{M_1}'\sqcup{M_2}'}{\sim}$$
    where $x\sim (\psi^{-1}\circ\phi_{21}\circ\psi)(x)$ $\forall x\in \psi^{-1}(B(0,\varepsilon))$\footnote{Roughly speaking, a connected sum of two m-manifolds is a manifold formed by deleting a ball inside each manifold and gluing together the resulting boundary spheres.}.
  \end{definition}
  \subsubsection{Orientability}
  \begin{definition}
    Let $V$ be a vector space and $\mathcal{B}_1$ and $\mathcal{B}_2$ be two bases of $V$. The bases $\mathcal{B}_1$ and $\mathcal{B}_2$ have the \emph{same orientation} if $\det\left([\id]_{\mathcal{B}_1,\mathcal{B}_2}\right)>0$. Otherwise, we say that they have \emph{opposite orientations}. Note that the property of having the same orientation defines an equivalence relation on the set of all bases for $V$.
  \end{definition}
  \begin{definition}
    An \emph{orientation} on a vector space is an assignment of $+1$ to one equivalence class and $-1$ to the other. A vector space with an orientation selected is called an \emph{oriented vector space}, while one not having an orientation selected, is called an \emph{unoriented vector space}.
  \end{definition}
  \begin{definition}
    Let $V$, $W$ be oriented vector spaces and $f:V\rightarrow W$ be a linear isomorphism. We say that $f$ is \emph{orientation-preserving} if $\det \left([f]_{\mathcal{B}_1,\mathcal{B}_2}\right)>0$ for some bases $\mathcal{B}_1$ of $V$ and $\mathcal{B}_2$ of $W$ according to the orientation chosen. Analogously, if $\det \left([f]_{\mathcal{B}_1,\mathcal{B}_2}\right)<0$ we say that $f$ is \emph{not orientation-preserving}.
  \end{definition}
  \begin{definition}
    Let $\vf{f}:\RR^n\rightarrow\RR^n$ be a differentiable homeomorphism. We say that $\vf{f}$ is \emph{orientation-preserving} if $\det \left(\vf{Df}(x)\right)>0$ $\forall x\in\RR^n$. Otherwise, we say that $f$ is \emph{not orientation-preserving}.
  \end{definition}
  \begin{definition}
    We say that a manifold $(M,\topo)$ is \emph{orientable} if it admits an atlas such that all the transition functions are orientation-preserving.
  \end{definition}
  \begin{proposition}
    Let $(M,\topo_M)$ and $(N,\topo_N)$ be orientable manifolds. Then, $(M\times N,\topo_{M\times N})$ is orientable.
  \end{proposition}
  \begin{proposition}
    $\RR^n$, $\S^n$ and $T^2$ are orientable, but $\mathcal{P}_n(\RR)$, $\mathcal{M}$ and $\mathcal{K}$ are not.
  \end{proposition}
  \subsubsection{1-manifolds}
  \begin{theorem}[Classification of connected 1-manifolds]
    Let $(M,\topo)$ be a connected 1-manifold. Then, $M$ is homeomorphic to exactly one of the following manifolds:
    \begin{itemize}
      \item $\RR$
      \item $\S^1$
    \end{itemize}
  \end{theorem}
  \subsection{Compact surfaces}
  \subsubsection{Connected sum of surfaces}
  \begin{definition}
    Let $(M,\topo)$ be a $m$-manifold. We say that $(M,\topo)$ is a \emph{surface} if $m=2$.
  \end{definition}
  \begin{proposition}[Connected sum of surfaces]
    Let $(S_1,\topo_{S_1})$ and $(S_2,\topo_{S_2})$ be two surfaces, $p_1\in M_1$, $p_2\in M_2$ and $(U_1,\varphi_1)$, $(U_2,\varphi_2)$ be coordinate charts such that $p_i\in U_i$ and $\varphi_i(p_i)=0$ for $i=1,2$. Let $D_i:={\varphi_i}^{-1}(B(0,1))$ for $i=1,2$. Then, note that for $i=1,2$, ${S_i}':=S_i\setminus D_i$ is a surface and $\Fr {S_i}'={\varphi_i}^{-1}(\Fr B(0,1))\cong \S^1$. Then, the connected sum of $(S_1,\topo_{S_1})$ and $(S_2,\topo_{S_2})$ is: $$S_1\conn S_2=\quot{{S_1}'\sqcup{S_2}'}{\Fr {S_1}'\sim\Fr {S_2}'}$$
  \end{proposition}
  \begin{proposition}
    Let $(S_1,\topo_{S_1})$, $(S_2,\topo_{S_2})$, $(S_3,\topo_{S_3})$ be compact connected surfaces. Then:
    \begin{enumerate}
      \item $S_1\conn  S_2\cong S_2\conn  S_1$
      \item $(S_1\conn  S_2)\conn  S_3\cong S_1\conn  (S_2\conn  S_3)$
      \item $S_1\conn \S^2\cong S_1$
      \item $S_1\conn  S_2$ is orientable $\iff S_1$ and $S_2$ are both orientable.
    \end{enumerate}
  \end{proposition}
  \begin{proposition}
    Let $(M,\topo)$ be a compact connected surface. Then:
    \begin{enumerate}
      \item $M\conn  T^2$ is a handle attached to $M$.
      \item $M\conn \mathcal{P}_n(\RR)$ is attaching a Möbius band to $M$.
    \end{enumerate}
  \end{proposition}
  \begin{definition}\label{TOP_genus}
    Let $g\in\NN\cup\{0\}$. We define the \emph{genus $g$ orientable surface} as: $$S_g:=\S^2\conn  T^2\conn \overset{(g)}{\cdots}\conn  T^2$$
  \end{definition}
  \begin{definition}
    Let $h\in\NN$. We define the \emph{genus $h$ non-orientable surface} as: $$N_h:=\mathcal{P}_2(\RR)\conn \overset{(h)}{\cdots}\conn  \mathcal{P}_2(\RR)$$
  \end{definition}
  \begin{center}
    \begin{minipage}{\linewidth}
      \centering
      \includestandalone[mode=image|tex,width=\linewidth]{Images/orientable_genusG}
      \captionof{figure}{Genus $g$ orientable surfaces}
    \end{minipage}
  \end{center}
  \subsubsection{Triangularization}
  \begin{definition}
    The \emph{standard $n$-simplex} is the set $$\Delta^n:=\{(x_0,\ldots,x_n)\in{\RR_{\geq 0}}^{n+1}:x_0+\cdots+x_n=1\}$$
  \end{definition}
  \begin{definition}
    Let $(S,\topo)$ be a compact connected surface. A \emph{triangularization} of $S$ is a finite collection $\{T_1,\ldots,T_n\}$ such that $S=\bigcup_{i=1}^nT_i$, $T_i\cong\Delta^2$ for $i=1,\ldots,n$ and if $T_i\cap T_j\ne\varnothing$ for $i\ne j$, then $T_i\cap T_j$ is an edge of $D_i$ and $D_j$ or a vertex of $D_i$ and $D_j$.
  \end{definition}
  \begin{definition}
    A \emph{simple curve} is a non-self-intersecting continuous loop in the plane.
  \end{definition}
  \begin{theorem}[Jordan curve theorem]
    All simple curves divides the plane in 2 connected components. One of these components is bounded and the other is unbounded.
  \end{theorem}
  \begin{theorem}[Radó theorem]
    Each compact surfaces has a triangularization.
  \end{theorem}
  \begin{theorem}
    Every compact surface can be constructed from a polygon with an even number of sides, called a \emph{fundamental polygon of the surface}, by pairwise identification of its edges\footnote{Any fundamental polygon can be written symbolically as follows. Begin at any vertex, and proceed around the perimeter of the polygon in either direction until returning to the starting vertex. During this traversal, record the label on each edge in order, with an exponent of $-1$ if the edge points opposite to the direction of traversal.}. Reciprocally, every polygon whose edges are pairwise identified produces a surface.
  \end{theorem}
  \begin{proposition}
    We have the following representations\footnote{Note that these representations are not unique.} of the most common surfaces\footnote{See \cref{TOP_fig-MK,TOP_fig-PS} for a better understanding.}:
    \begin{itemize}
      \item $\S^2$: $aa^{-1}=:1$
      \item $T^2$: $aba^{-1}b^{-1}$
      \item $\mathcal{P}_2(\RR)$: $aa$
      \item $\mathcal{K}$: $aba^{-1}b$
    \end{itemize}
  \end{proposition}
  \begin{center}
    \begin{minipage}{\linewidth}
      \centering
      \includestandalone[mode=image|tex,width=\linewidth]{Images/triangularization}
      \captionof{figure}{A triangularization of the torus $T^2$, the Klein bottle $\mathcal{K}$ and the projective plane $\mathcal{P}_2(\RR)$}
    \end{minipage}
  \end{center}
  \begin{definition}
    We say that a representation of a surface is \emph{normalized} if it is of one of the following forms:
    \begin{gather*}
      a_1b_1{a_1}^{-1}{b_1}^{-1}\cdots a_nb_n{a_n}^{-1}{b_n}^{-1}\\
      a_1b_1{a_1}^{-1}{b_1}^{-1}\cdots a_rb_r{a_r}^{-1}{b_r}^{-1}c_1c_1\cdots c_sc_s
    \end{gather*}
  \end{definition}
  \begin{proposition}
    Let $(S_1,\topo_{S_1})$ and $(S_2,\topo_{S_2})$ be two surfaces whose respect representations are: $$a_1\cdots a_n\quad b_1\cdots b_m$$
    Then, $S_1\conn S_2$ is represented by $$a_1\cdots a_nb_1\cdots b_m$$
  \end{proposition}
  \begin{proposition}
    On the representations of polygons, we have:
    \begin{itemize}
      \item $aba^{-1}b^{-1}\equiv cdc^{-1}d^{-1}$
      \item $aba^{-1}b^{-1}\equiv a_1a_2b{a_2}^{-1}{a_1}^{-1}b^{-1}$
      \item $aba^{-1}b^{-1}\equiv ba^{-1}b^{-1}a\equiv a^{-1}b^{-1}ab\equiv b^{-1}aba^{-1}$
      \item $abab^{-1}\equiv abc, ab^{-1}e^{-1}$
      \item $abab^{-1}\equiv abcc^{-1}ab^{-1}$
      \item $abab^{-1}\equiv a^{-1}b^{-1}a^{-1}b$
    \end{itemize}
  \end{proposition}
  \begin{corollary}
    Let $g\in\NN\cup\{0\}$ and $h\in\NN$. Then, the representations of $S_g$ and $N_h$ are:
    \begin{itemize}
      \item $S_g$: $a_1b_1{a_1}^{-1}{b_1}^{-1}\cdots a_gb_g{a_g}^{-1}{b_g}^{-1}$ (a polygon with $4g$ sides)
      \item $N_h$: $a_1a_1\cdots a_ha_h$ (a polygon with $2n$ sides)
    \end{itemize}
  \end{corollary}
  \begin{proposition}
    \hfill
    \begin{itemize}
      \item $\mathcal{K}\cong\mathcal{P}_2(\RR)\conn\mathcal{P}_2(\RR)$
      \item $T^2\conn\mathcal{P}_2(\RR)\cong\mathcal{K}\conn\mathcal{P}_2(\RR)\cong\mathcal{P}_2(\RR)\conn\mathcal{P}_2(\RR)\conn\mathcal{P}_2(\RR)$
    \end{itemize}
  \end{proposition}
  \begin{corollary}
    Let $g\in\NN\cup\{0\}$ and $h\in\NN$. Then: $$S_g\conn N_h\cong N_{h+2g}$$
  \end{corollary}
  \begin{proposition}
    Let $(S,\topo)$ be a compact connected surface. Then, $S$ is non-orientable if and only if it contains a Möbius strip, which can be identified by a representation of $S$ of the form $$aWaW'$$ where $W$ and $W'$ may contain more than one edge.
  \end{proposition}
  \subsubsection{Euler characteristic}
  \begin{definition}
    Let $(S,\topo)$ be a compact connected surface and $T=\{T_i:i=1,\ldots,n\}$ be a triangularization of $S$. We define the \emph{Euler characteristic} of $S$ with the triangularization $T$ as: $$\chi_T(S):=V-E+F$$ where $V$, $E$, and $F$ are respectively the numbers of vertices, edges and faces in the polygonal decomposition obtained from $T$ taking into account the identifications between vertices and edges\footnote{That is, two identified vertices (or edges) count as one.}.
  \end{definition}
  \begin{proposition}
    Let $(S,\topo)$ be a compact connected surface and $T$, $T'$ be triangularizations of $S$. Then: $$\chi_T(S)=\chi_{T'}(S)$$ Therefore, from now on we will denote the Euler characteristic of $S$ as $\chi(S)$.
  \end{proposition}
  \begin{proposition}
    We have the following Euler characteristics of the most common surfaces:
    \begin{itemize}
      \item $\chi(\S^2)=2$
      \item $\chi(T^2)=0$
      \item $\chi(\mathcal{P}_2(\RR))=1$
      \item $\chi(\mathcal{K})=0$
    \end{itemize}
  \end{proposition}
  \begin{proposition}
    Let $(S_1,\topo_{S_1})$ and $(S_2,\topo_{S_2})$ be two surfaces. Then: $$\chi(S_1\conn S_2)=\chi(S_1)+\chi(S_2)-2$$
  \end{proposition}
  \begin{corollary}
    Let $g\in\NN\cup\{0\}$ and $h\in\NN$. Then:
    \begin{itemize}
      \item $\chi(S_g)=2-2g$
      \item $\chi(N_h)=2-h$
    \end{itemize}
  \end{corollary}
  \begin{theorem}
    The Euler characteristic is a topological property.
  \end{theorem}
  \begin{theorem}[Classification of compact connected surfaces]
    Every compact connected surface is homeomorphic to exactly one of the following surfaces:
    \begin{itemize}
      \item $S_g$ for some $g\in\NN\cup\{0\}$.
      \item $N_h$ for some $h\in\NN$.
    \end{itemize}
  \end{theorem}
  \begin{corollary}
    Two compact connected surfaces are homeomorphic if and only if they have the same orientability and the same Euler characteristic.
  \end{corollary}
\end{multicols}
\end{document}