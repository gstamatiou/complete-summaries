\documentclass[../../../main.tex]{subfiles}

\begin{document}
\begin{multicols}{2}[\section{Galois theory}]
  \subsection{Introduction}
  \subsubsection*{Rings, integral domains and fields}
  \begin{prop}
    \hfill
    \begin{enumerate}
      \item A subring of an integral domain is an integral domain.
      \item A field is an integral domain.
      \item A subring of a field is an integral domain.
    \end{enumerate}
  \end{prop}
  \begin{lemma}
    Let $K$ be a field and $R\ne\{0\}$ be a ring. Then, all ring morphisms $f:K\rightarrow R$ are injective.
  \end{lemma}
  \begin{definition}
    Let $K$, $L$ be fields. A \textit{field morphism between $K$ and $L$} is a ring morphism $K\rightarrow L$.
  \end{definition}
  \begin{lemma}
    Let $R$ be a ring. Then, there exists a unique ring morphism $f:\ZZ\rightarrow R$ satisfying:
    \begin{itemize}
      \item $f(1+\overset{(n)}{\cdots}+1)=1_R+\overset{(n)}{\cdots}+1_R$ if $n\geq 1$.
      \item $f(n)=-f(-n)$ if $n\leq -1$.
    \end{itemize}
  \end{lemma}
  \begin{definition}
    Let $R$ be a ring and $f:\ZZ\rightarrow R$ be the ring morphism from $\ZZ$ to $R$. The \textit{characteristic of $R$}, $\ch (R)$, is defined to be the value of $n$ such that $\ker f=\quot{\ZZ}{n\ZZ}$.
  \end{definition}
  \begin{prop}
    Let $K$ be a field. Then, either $\ch(K)\in\PP$ or $\ch(K)=0$.
  \end{prop}
  \begin{definition}
    Let $R$ be a ring. We define the  \textit{polynomial ring $R[x]$} as: $$R[x]:=\{r_0+r_1\cdot x+\cdots+r_n\cdot x^n:r_i\in R\ \forall i\text{ and }n\geq 0\}$$ Moreover, we can iterate this definition to define the polynomial ring in $m$-unknowns: $$R[x_1,\ldots,x_m]=\left(R[x_1,\ldots,x_{m-1}]\right)[x_m]$$
  \end{definition}
  \begin{prop}[Universal property of polynomials in several variables]
    Let $R$, $S$ be two rings, $f:R\rightarrow S$ be a ring morphism and $s_1,\ldots,s_n\in S$ be not necessarily distinct elements of $S$. Then, the function $\text{eva}:R[x_1,\ldots,x_n]\rightarrow S$ defined by: $$\text{eva}(p):=\sum_{i_1,\ldots,i_n\geq 0}f(r_{i_1,\ldots,i_n}){s_1}^{i_1}\cdots {s_n}^{i_n}$$ is the unique ring morphism such that $\text{eva}(r)=f(r)$ $\forall r\in R$ and $\text{eva}(x_i)=s_i$ for $i=1,\ldots,n$. This function is called \textit{evaluation of $s_1,\ldots,s_n$ through $f$}\footnote{For the case $n=1$, we will denote by $\text{eva}_s$ the unique ring morphism $\text{eva}_s:R[x]\rightarrow S$ such that $\text{eva}_s(r)=f(r)$ $\forall r\in R$ and $\text{eva}_s(x)=s$.}.
  \end{prop}
  \subsubsection*{Field of fractions}
  \begin{theorem}
    All integral domains are a subring of a field. More explicitly, if $R$ is an integral domain and $K$ is a field, there exists another field $Q(R)$ and an injective ring morphism $\iota:R\hookrightarrow Q(R)$ so that for all injective ring morphism $f:R\hookrightarrow K$, there exists a unique field morphism $Q(f):Q(R)\rightarrow K$ such that $f=Q(f)\circ\iota$.
  \end{theorem}
  \begin{corollary}
    Let $R$ be an integral domain. The field $Q(R)$ with the injection $\iota$ is unique up to isomorphism, that is if there is a field $Q'(R)$ and an injective ring morphism $\iota':R\hookrightarrow Q(R)$ satisfying the property of above, then there is a unique isomorphism $Q(f):Q(R)\cong Q'(R)$ such that $\iota'=Q(\iota')\circ\iota$. This field $Q(R)$ is called \textit{field of fractions of $R$}.
  \end{corollary}
  \begin{definition}
    Let $K$ be a field. The field of fractions of $K[x]$ is defined as $K(x):=Q(K[x])$ and it is called \textit{field of rational functions}. More generally, the field of fractions of $K[x_1,\ldots,x_n]$ is defined as: $$K(x_1,\ldots,x_n):=Q(K[x_1,\ldots,x_n])$$
  \end{definition}
  \begin{lemma}
    Let $R$ be an integral domain. Then, $R[x]$ is also an integral domain and: $$Q(R[x])\cong Q(R)(x)$$
  \end{lemma}
  \begin{corollary}
    Let $K$ be a field. For all $n\geq 2$ we have: $$K(x_1,\ldots,x_n)\cong K(x_1,\ldots,x_{n-1})(x_n)$$
  \end{corollary}
  \subsubsection*{Subring and subfield generated by a set}
  \begin{definition}
    Let $(R,+,\cdot)$ be a ring and $X\subseteq R$ be a subset of $R$. Let $$P:=\{S\subseteq R: X\subseteq S,(S,+,\cdot)\text{ is a subring of }(R,+,\cdot)\}$$ Then, the smallest subring of $(R,+,\cdot)$ containing $X$ is: $$\langle X\rangle_\text{ring}=\bigcap_{S\in P}S$$
  \end{definition}
  \begin{definition}
    Let $R$ be a ring, $S\subseteq R$ be a subring of $R$ and $A\subseteq R$ be a subset of $R$. Then, the smallest subring of $R$ containing $S$ and $A$ is denoted by $S[A]$.
  \end{definition}
  \begin{lemma}
    Let $A$ be a finite set, $R$ and $S$ be rings and $\text{eva}:R[x_a:a\in A]\rightarrow S$ be the evaluation morphism such that $\text{eva}(r)=r$ $\forall r\in R$ and $\text{eva}(x_a)=a$ $\forall a\in A$. Then, $S[A]=\im(\text{eva})$.
  \end{lemma}
  \begin{definition}
    Let $(K,+,\cdot)$ be a field and $X\subseteq K$ be a subset of $K$. Let $$P:=\{L\subseteq K: X\subseteq L,(L,+,\cdot)\text{ is a subfield of }(R,+,\cdot)\}$$ Then, the smallest subfield of $(K,+,\cdot)$ containing $X$ is: $$\langle X\rangle_\text{field}=\bigcap_{L\in P}L$$
  \end{definition}
  \begin{definition}
    Let $L$ be a field, $K\subseteq L$ be a subfield of $L$ and $A\subseteq L$ be a subset of $L$. Then, the smallest subfield of $L$ containing $K$ and $A$ is denoted by $K(A)$.
  \end{definition}
  \subsubsection*{Algebraic and transcendental numbers}
  \begin{prop}
    Let $\alpha\in\CC$. Then, the subring generated by $\alpha$ is equal to the image of the ring morphism $\text{eva}_\alpha:\ZZ[x]\rightarrow\CC$, that is, $\langle\alpha\rangle_\text{ring}=\im(\text{eva}_\alpha)$ where: $$\im(\text{eva}_\alpha)=\left\{\sum_{i=0}^na_i\alpha^i:a_i\in\ZZ,i=0,\ldots, n,n\geq 0\right\}$$
  \end{prop}
  \begin{prop}
    Let $\alpha\in\CC$. Then, the subfield generated by $\alpha$ is equal to the image of the ring morphism $\text{eva}_\alpha:\QQ[x]\rightarrow\CC$, that is, $\langle\alpha\rangle_\text{field}=\im(\text{eva}_\alpha)$ where: $$\im(\text{eva}_\alpha)=\left\{\sum_{i=0}^na_i\alpha^i:a_i\in\QQ,i=0,\ldots, n,n\geq 0\right\}$$
  \end{prop}
  \begin{definition}
    Let $\alpha\in\CC$ and consider $\text{eva}_\alpha:\QQ[x]\rightarrow\CC$. We say that $\alpha$ is \textit{algebraic} if $\ker(\text{eva}_\alpha)=(p(x))$, where $p(x)\in\QQ[x]$ is an irreducible polynomial of degree $d\geq 1$. This polynomial is called \textit{irreducible polynomial of $\alpha$} and is denoted by $\irr(\alpha,\QQ)(x)$. The set of all algebraic complex numbers is denoted by $\overline{\QQ}\subset\CC$.
  \end{definition}
  \begin{definition}
    Let $\alpha\in\CC$ and consider $\text{eva}_\alpha:\QQ[x]\rightarrow\CC$. We say that $\alpha$ is \textit{transcendental} if $\ker(\text{eva}_\alpha)=(0)$, or equivalently, if it is not algebraic.
  \end{definition}
  \begin{theorem}
    The set $\overline{\QQ}\subset\CC$ is countable.
  \end{theorem}
  \subsection{Field extensions}
  \begin{prop}
    Let $K$, $L$ be fields. Then, all field morphism $K\rightarrow L$ is injective.
  \end{prop}
  \begin{definition}
    Let $K$, $L$ be two fields. A \textit{field extension $\quot{L}{K}$} is a field morphism $K\hookrightarrow L$.
  \end{definition}
  \begin{prop}
    Let $K$, $L$ be fields and $\quot{L}{K}$ be a field extension. Then, $L$ is a vector space over $K$. Reciprocally, if $L$ is a vector space over $K$ satisfying: $$(\lambda\cdot 1)\cdot(\mu\cdot 1)=(\lambda\cdot\mu)\cdot 1\qquad\forall\lambda,\mu\in K$$ then the morphism $f:K\rightarrow L$ defined as $f(\lambda)=\lambda\cdot 1$ is a field morphism and $\quot{L}{K}$ is a field extension.
  \end{prop}
  \begin{definition}
    Let $\quot{L}{K}$ be a field extension. We define the \textit{degree of the extension $\quot{L}{K}$} as: $$[L:K]:=\dim_K(L)$$ We say that an extension is \textit{finite} if $[L:K]$ is finite.
  \end{definition}
  \begin{prop}
    Let $K$ be a field, $p(x)\in K[x]$ a monic and irreducible polynomial of degree $d\geq 1$ and $L=\quot{K[x]}{(p(x))}$. Then, $\quot{L}{K}$ is a field extension of degree $d$, and the set $\{1,\bar{x},\ldots,\bar{x}^{d-1}\}$ is a basis of the vector space $L$ over $K$. Furthermore, $\bar{x}\in L$ is a root of $p(x)$ in $L$.
  \end{prop}
  \subsubsection*{Simple and algebraic extensions}
  \begin{definition}
    Let $\quot{L}{K}$ be a field extension and $\alpha\in L$. If $L=K(\alpha)$ then the extension $\quot{L}{K}=\quot{K(\alpha)}{K}$ is called \textit{simple}.
  \end{definition}
  \begin{prop}
    Let $\quot{L}{K}$ be a field extension and $\alpha\in L$. Then:
    \begin{itemize}
      \item If $\alpha$ is algebraic over $K$, then: $$K(\alpha)=K[\alpha]\cong\quot{K[x]}{(\irr(\alpha,K)(x))}$$
      \item If $\alpha$ is transcendental over $K$, then: $$K(\alpha)\cong K(x)$$
    \end{itemize}
  \end{prop}
  \begin{corollary}
    Let $\quot{L}{K}$ be a field extension and $\alpha\in L$. Then, $\quot{K(\alpha)}{K}$ is finite if and only if $\alpha$ is algebraic over $K$. Furthermore in that case: $$[K(\alpha):K]=\deg(\irr(\alpha,K)(x))$$
  \end{corollary}
  \begin{theorem}[Tower formula]
    Let $\quot{F}{L}$ and $\quot{L}{K}$ be field extensions. Then: $$[F:K]=[F:L][L:K]$$
  \end{theorem}
  \begin{prop}
    Let $\quot{L}{K}$ be a field extension and $\alpha\in L$ be algebraic. If $\quot{K'}{K}$ is another field extension, then:
    \begin{enumerate}
      \item $\alpha\in K\iff\irr(\alpha,K)(x)=x-\alpha\iff \deg(\irr(\alpha,K)(x))=1$
      \item $\irr(\alpha,K')(x)\mid\irr(\alpha,K)(x)$ and they are equal if and only if they have the same degree.
      \item $\deg(\irr(\alpha,K)(x))\mid [L:K]$.
    \end{enumerate}
  \end{prop}
  \begin{definition}
    Let $n\in\NN$ and $K_0,\ldots,K_n$ be fields. A \textit{tower of fields} is a sequence of field extension $\quot{K_j}{K_{j-1}}$ for $j=1,\ldots,n$. We will denoted this tower of fields as: $$K_n/K_{n-1}/\cdots/K_0$$
  \end{definition}
  \begin{corollary}
    Let $n\in\NN$ and $$K_n/K_{n-1}/\cdots/K_0$$ be a tower of fields. Then: $$[K_n:K_0]=[K_n:K_{n-1}][K_{n-1}:K_{n-2}]\cdots[K_1:K_0]$$
  \end{corollary}
  \begin{definition}
    Let $\quot{L}{K}$ be a field extension. We say that $\quot{L}{K}$ is \textit{algebraic} if $\forall\alpha\in L$, $\alpha$ is algebraic over $K$.
  \end{definition}
  \begin{definition}
    Let $\quot{L}{K}$ be a field extension. We say that $\quot{L}{K}$ is \textit{purely transcendental} if $\forall\alpha\in L\setminus K$, $\alpha$ is transcendental over $K$.
  \end{definition}
  \begin{lemma}
    Any finite field extension is algebraic.
  \end{lemma}
  \begin{prop}
    Let $\quot{L}{K}$ be a field extension. The following are equivalent:
    \begin{enumerate}
      \item $\quot{L}{K}$ is finite.
      \item $\quot{L}{K}$ is algebraic and there exist $\alpha_1,\ldots,\alpha_n\in L$ such that $L=K(\alpha_1,\ldots,\alpha_n)$.
      \item There exist $\alpha_1,\ldots,\alpha_n\in L$ with $\alpha_i$ algebraic over $K(\alpha_1,\ldots,\alpha_{i-1})$ for $i=1,\ldots,n$ such that $L=K(\alpha_1,\ldots,\alpha_n)$.
    \end{enumerate}
  \end{prop}
  \begin{prop}
    Let $\quot{L}{K}$ and $\quot{F}{L}$ be field extensions. Then:
    \begin{enumerate}
      \item If $\quot{L}{K}$ is algebraic, any subring $R$ such that $K\subseteq R\subseteq L$ is a subfield.
      \item $\quot{L}{K}$ and $\quot{F}{L}$ are algebraic $\iff\quot{F}{K}$ is algebraic.
      \item If $\alpha,\beta\in L$ are algebraic over $K$, then so are $\alpha+\beta$, $\alpha\beta$ and $\alpha\beta^{-1}$ (if $\beta\ne 0$).
      \item The set $$E:=\{\alpha\in L:\alpha\text{ is algebraic over }K\}$$ is a subfield of $L$, the field extension $\quot{E}{K}$ is algebraic and if $L\ne E$, then $\quot{L}{E}$ is purely transcendental.
    \end{enumerate}
  \end{prop}
  \subsubsection*{Morphisms of extensions}
  \begin{definition}
    Let $K$, $L$, $F$ be fields and consider the field extensions $\quot{L}{K}$ and $\quot{F}{K}$ with the morphisms $f:K\hookrightarrow L$ and $g:K\hookrightarrow F$, respectively. We say that a \textit{morphism of extensions between $\quot{L}{K}$ and $\quot{F}{K}$} is a field morphism $h:L\rightarrow F$ such that $h|_K=g$, where $h|_K:=h\circ f$\footnote{If the morphisms $f$ and $g$ are the natural inclusions, we will denote $h|_K:=\id_K$.}. The set of all such morphisms is:
    $$\Mor_K(f,g):=\{h:L\longrightarrow F:h\circ f=g\}$$
    If $f$ and $g$ are the natural inclusions, we will denote: $$\Mor_K(L,F):=\Mor_K(f,g)=\{h:L\longrightarrow F:h|_K=\id_K\}$$
  \end{definition}
  \begin{definition}
    Let $K$, $L$, $F$ be fields and consider the field extensions $\quot{L}{K}$ and $\quot{F}{K}$ with the morphisms $f:K\hookrightarrow L$ and $g:K\hookrightarrow F$, respectively. We define the following sets:
    \begin{align*}
      \Iso_K(f,g):=\{h\in\Mor_K(f,g):h\text{ is bijective}\} \\
      \Aut_K(f):=\Iso_K(f,f)\qquad\Gal(f):=\Aut_K(f)
    \end{align*}
    If $f$ and $g$ are the natural inclusions, we will denote:
    \begin{align*}
      \Iso_K(L,F):=\{h\in\Mor_K(L,F):h\text{ is bijective}\} \\
      \Aut_K(L):=\Iso_K(L,L)\qquad\textstyle\Gal(\quot{L}{K}):=\Aut_K(L)
    \end{align*}
  \end{definition}
  \begin{lemma}
    Let $K$, $L$, $F$ be fields and $f:K\rightarrow L$, $g:K \rightarrow F$ be field morphisms. Let $h\in\Mor_K(f,g)$, $\alpha\in L$, and $p(x)\in K[x]$. Then: $$h(f(p)(\alpha))=g(p)(h(\alpha))$$
  \end{lemma}
  \begin{lemma}
    Let $\quot{L}{K}$, $\quot{F}{K}$ be field extensions and $\alpha\in L$ be algebraic over $K$. Then, we have the following bijection: $$\Mor_K(K(\alpha),F)\cong\{\beta\in F:g(\irr(\alpha,K))(\beta)=0\}$$
  \end{lemma}
  \begin{corollary}
    Let $L=\quot{K(\alpha)}{K}$ be a finite field extension. Then, $\Gal(\quot{L}{K})$ is finite and: $$\textstyle|\Gal(\quot{L}{K})|=|\{\beta\in L:\irr(\alpha,K)(\beta)=0\}|\leq[K(\alpha):K]$$
  \end{corollary}
  \begin{prop}
    Let $K$, $L$ and $F$ be fields and $f:K\rightarrow L$ and $g:K\rightarrow F$ be field morphisms. Then:
    \begin{enumerate}
      \item If $K\overset{f'}{\rightarrow} L'\overset{\varphi}{\rightarrow} F$ is a tower of fields, then: $$\Mor_K(f,g)=\bigsqcup_{h\in\Mor_K(f',g)}\Mor_{L'}(\varphi,h)$$
      \item If $L/L'/K$ is a tower of fields, then:
            \begin{multline*}
              \Mor_K(L,F)=\bigsqcup_{h\in\Mor_K(L',F)}\{\psi\in\Mor_K(L,F)\\
              :\psi|_{L'}=h\}
            \end{multline*}
      \item If $\Iso_K(f,g)\ne\varnothing$, then $\Iso_K(f,g)\cong\Gal(f)$ by sending $h\mapsto h\circ {h_0}^{-1}$, where $h_0\in\Iso_K(f,g)$ is a fixed isomorphism.
      \item If $\Iso_K(L,F)\ne\varnothing$, then $\Iso_K(L,F)\cong\Gal(\quot{L}{K})$ by sending $h\mapsto h\circ {h_0}^{-1}$, where $h_0\in\Iso_K(L,F)$ is a fixed isomorphism.
    \end{enumerate}
  \end{prop}
  \subsubsection*{Splitting field}
  \begin{definition}
    Let $K$ be a field and $p(x)\in K[x]$ be a polynomial such that $\deg p(x)=n\geq 1$. We say that $p(x)$ \textit{factorizes completely} on $K$ if $p(x)=a_n\prod_{i=1}^n(x-a_i)$, where $a_i\in K$ for $i=1,\ldots,n$.
  \end{definition}
  \begin{theorem}
    Let $K$ be a field, $S\subset K[x]$ be a finite set. Then, there exists a finite field extension $\quot{L}{K}$ such that all polynomials in $S$ factorize completely on $L$.
  \end{theorem}
  \begin{theorem}
    Let $K$ be a field, $L=K(\alpha_1,\ldots,\alpha_n)$ and consider a finite field extension $\quot{L}{K}$. Let $f:K\hookrightarrow F$ be a field morphism such that $f(\irr(\alpha_i,K)(x))$ factorizes completely. Then, $$1\leq|\Mor_K(L,F)|\leq [L:K]$$ and, furthermore, $|\Mor_K(L,F)|=[L:K]$ if and only if $f(\irr(\alpha_i,K)(x))$ has no repeated roots on $F$ for all $i=1,\ldots,n$
  \end{theorem}
  \begin{definition}[Splitting field]
    Let $\quot{L}{K}$ be a finite field extension and $p(x)\in K[x]\setminus K$ be such that it factorizes completely on $L$. Let $\alpha_1,\ldots,\alpha_n$ be its roots. We call \textit{splitting field of $p(x)$ in $L$} the subfield $K(\alpha_1,\ldots,\alpha_n)$ of $L$, which is the least subfield where $p(x)$ factorizes completely.
  \end{definition}
  \begin{definition}[Splitting field]
    Let $K$ be a field and $p(x)\in K[x]\setminus K$. We say that a finite field extension $\quot{L}{K}$ is a \textit{splitting field of $p(x)$} if:
    \begin{enumerate}
      \item $p(x)$ factorizes completely on $L$.
      \item For all field extensions $\quot{F}{L}$, with $F\ne L$, then $p(x)$ doesn't factorize completely on $F$.
    \end{enumerate}
  \end{definition}
  \begin{theorem}
    Let $K$ be a field and $p(x)\in K[x]\setminus K$. Then, there exists (up to isomorphism) a unique splitting field of $p(x)$ over $K$
  \end{theorem}
  \begin{theorem}
    Let $K$ be a field, $p(x)\in K[x]\setminus K$ and $\quot{L}{K}$ and $\quot{F}{K}$ be two splitting fields of $p(x)$ over $K$. Then, $[L:K]=[L:K']$ and $$1\leq|\Iso_K(L,F)|\leq [L:K]$$ Furthermore, $|\Iso_K(L,F)|=[L:K]$ if and only if all irreducible factors of $p(x)$ has no repeated roots on $F$.
  \end{theorem}
  \begin{corollary}
    Let $K$ be a field and $p(x)\in K[x]\setminus K$. Then, any two splitting fields of $p(x)$ over $K$ are isomorphic.
  \end{corollary}
  \begin{corollary}
    Let $K$ be a field, $p(x)\in K[x]\setminus K$ and $L$ be the splitting field of $p(x)$ over $K$. Then, $|\Gal(\quot{L}{K})|\leq[L:K]$ and the inequality holds if $p(x)$ has no repeated roots on $L$.
  \end{corollary}
  \begin{corollary}
    Let $\quot{F}{K}$ be a field extension and $p(x)\in K[x]$. Then, the splitting field $L_F$ of $p(x)$ over $F$ contains the splitting field $L$ of $p(x)$ over $K$
  \end{corollary}
\end{multicols}
\end{document}