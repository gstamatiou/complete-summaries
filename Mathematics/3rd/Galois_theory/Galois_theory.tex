\documentclass[../../../main.tex]{subfiles}

\begin{document}
\begin{multicols}{2}[\section{Galois theory}]
  \subsection{Introduction}
  \subsubsection*{Rings, integral domains and fields}
  \begin{prop}
    \hfill
    \begin{enumerate}
      \item A subring of an integral domain is an integral domain.
      \item A field is an integral domain.
      \item A subring of a field is an integral domain.
    \end{enumerate}
  \end{prop}
  \begin{lemma}
    Let $K$ be a field and $R\ne\{0\}$ be a ring. Then, all ring morphisms $f:K\rightarrow R$ are injective.
  \end{lemma}
  \begin{definition}
    Let $K$, $L$ be fields. A \textit{field morphism between $K$ and $L$} is a ring morphism $K\rightarrow L$.
  \end{definition}
  \begin{lemma}
    Let $R$ be a ring. Then, there exists a unique ring morphism $f:\ZZ\rightarrow R$ satisfying:
    \begin{itemize}
      \item $f(1+\overset{(n)}{\cdots}+1)=1_R+\overset{(n)}{\cdots}+1_R$ if $n\geq 1$.
      \item $f(n)=-f(-n)$ if $n\leq -1$.
    \end{itemize}
  \end{lemma}
  \begin{definition}
    Let $R$ be a ring and $f:\ZZ\rightarrow R$ be the ring morphism from $\ZZ$ to $R$. The \textit{characteristic of $R$}, $\ch (R)$, is defined to be the value of $n$ such that $\ker f=\quot{\ZZ}{n\ZZ}$.
  \end{definition}
  \begin{prop}
    Let $K$ be a field. Then, either $\ch(K)\in\PP$ or $\ch(K)=0$.
  \end{prop}
  \begin{definition}
    Let $R$ be a ring. We define the  \textit{polynomial ring $R[x]$} as: $$R[x]:=\{r_0+r_1\cdot x+\cdots+r_n\cdot x^n:r_i\in R\ \forall i\text{ and }n\geq 0\}$$ Moreover, we can iterate this definition to define the polynomial ring in $m$-unknowns: $$R[x_1,\ldots,x_m]=\left(R[x_1,\ldots,x_{m-1}]\right)[x_m]$$
  \end{definition}
  \begin{prop}[Universal property of polynomials in several variables]
    Let $R$, $S$ be two rings, $f:R\rightarrow S$ be a ring morphism and $s_1,\ldots,s_n\in S$ be not necessarily distinct elements of $S$. Then, the function $\text{eva}:R[x_1,\ldots,x_n]\rightarrow S$ defined by: $$\text{eva}(p):=\sum_{i_1,\ldots,i_n\geq 0}f(r_{i_1,\ldots,i_n}){s_1}^{i_1}\cdots {s_n}^{i_n}$$ is the unique ring morphism such that $\text{eva}(r)=f(r)$ $\forall r\in R$ and $\text{eva}(x_i)=s_i$ for $i=1,\ldots,n$. This function is called \textit{evaluation of $s_1,\ldots,s_n$ through $f$}\footnote{For the case $n=1$, we will denote by $\text{eva}_s$ the unique ring morphism $\text{eva}_s:R[x]\rightarrow S$ such that $\text{eva}_s(r)=f(r)$ $\forall r\in R$ and $\text{eva}_s(x)=s$.}.
  \end{prop}
  \subsubsection*{Field of fractions}
  \begin{theorem}
    All integral domains are a subring of a field. More explicitly, if $R$ is an integral domain and $K$ is a field, there exists another field $Q(R)$ and an injective ring morphism $\iota:R\hookrightarrow Q(R)$ so that for all injective ring morphism $f:R\hookrightarrow K$, there exists a unique field morphism $Q(f):Q(R)\rightarrow K$ such that $f=Q(f)\circ\iota$.
  \end{theorem}
  \begin{corollary}
    Let $R$ be an integral domain. The field $Q(R)$ with the injection $\iota$ is unique up to isomorphism, that is if there is a field $Q'(R)$ and an injective ring morphism $\iota':R\hookrightarrow Q(R)$ satisfying the property of above, then there is a unique isomorphism $Q(f):Q(R)\cong Q'(R)$ such that $\iota'=Q(\iota')\circ\iota$. This field $Q(R)$ is called \textit{field of fractions of $R$}.
  \end{corollary}
  \begin{definition}
    Let $K$ be a field. The field of fractions of $K[x]$ is defined as $K(x):=Q(K[x])$ and it is called \textit{field of rational functions}. More generally, the field of fractions of $K[x_1,\ldots,x_n]$ is defined as: $$K(x_1,\ldots,x_n):=Q(K[x_1,\ldots,x_n])$$
  \end{definition}
  \begin{lemma}
    Let $R$ be an integral domain. Then, $R[x]$ is also an integral domain and: $$Q(R[x])\cong Q(R)(x)$$
  \end{lemma}
  \begin{corollary}
    Let $K$ be a field. For all $n\geq 2$ we have: $$K(x_1,\ldots,x_n)\cong K(x_1,\ldots,x_{n-1})(x_n)$$
  \end{corollary}
  \subsubsection*{Subring and subfield generated by a set}
  \begin{definition}
    Let $(R,+,\cdot)$ be a ring and $X\subseteq R$ be a subset of $R$. Let $$P:=\{S\subseteq R: X\subseteq S,(S,+,\cdot)\text{ is a subring of }(R,+,\cdot)\}$$ Then, the smallest subring of $(R,+,\cdot)$ containing $X$ is: $$\langle X\rangle_\text{ring}=\bigcap_{S\in P}S$$
  \end{definition}
  \begin{definition}
    Let $R$ be a ring, $S\subseteq R$ be a subring of $R$ and $A\subseteq R$ be a subset of $R$. Then, the smallest subring of $R$ containing $S$ and $A$ is denoted by $S[A]$.
  \end{definition}
  \begin{lemma}
    Let $A$ be a finite set, $R$ and $S$ be rings and $\text{eva}:R[x_a:a\in A]\rightarrow S$ be the evaluation morphism such that $\text{eva}(r)=r$ $\forall r\in R$ and $\text{eva}(x_a)=a$ $\forall a\in A$. Then, $S[A]=\im(\text{eva})$.
  \end{lemma}
  \begin{definition}
    Let $(K,+,\cdot)$ be a field and $X\subseteq K$ be a subset of $K$. Let $$P:=\{L\subseteq K: X\subseteq L,(L,+,\cdot)\text{ is a subfield of }(R,+,\cdot)\}$$ Then, the smallest subfield of $(K,+,\cdot)$ containing $X$ is: $$\langle X\rangle_\text{field}=\bigcap_{L\in P}L$$
  \end{definition}
  \begin{definition}
    Let $L$ be a field, $K\subseteq L$ be a subfield of $L$ and $A\subseteq L$ be a subset of $L$. Then, the smallest subfield of $L$ containing $K$ and $A$ is denoted by $K(A)$.
  \end{definition}
  \subsubsection*{Algebraic and transcendental numbers}
  \begin{prop}
    Let $\alpha\in\CC$. Then, the subring generated by $\alpha$ is equal to the image of the ring morphism $\text{eva}_\alpha:\ZZ[x]\rightarrow\CC$, that is, $\langle\alpha\rangle_\text{ring}=\im(\text{eva}_\alpha)$ where: $$\im(\text{eva}_\alpha)=\left\{\sum_{i=0}^na_i\alpha^i:a_i\in\ZZ,i=0,\ldots, n,n\geq 0\right\}$$
  \end{prop}
  \begin{prop}
    Let $\alpha\in\CC$. Then, the subfield generated by $\alpha$ is equal to the image of the ring morphism $\text{eva}_\alpha:\QQ[x]\rightarrow\CC$, that is, $\langle\alpha\rangle_\text{field}=\im(\text{eva}_\alpha)$ where: $$\im(\text{eva}_\alpha)=\left\{\sum_{i=0}^na_i\alpha^i:a_i\in\QQ,i=0,\ldots, n,n\geq 0\right\}$$
  \end{prop}
  \begin{definition}
    Let $\alpha\in\CC$ and consider $\text{eva}_\alpha:\QQ[x]\rightarrow\CC$. We say that $\alpha$ is \textit{algebraic} if $\ker(\text{eva}_\alpha)=(p(x))$, where $p(x)\in\QQ[x]$ is an irreducible polynomial of degree $d\geq 1$. This polynomial is called \textit{irreducible polynomial of $\alpha$} and is denoted by $\text{Irr}(\alpha)(x)$. The set of all algebraic complex numbers is denoted by $\overline{\QQ}\subset\CC$.
  \end{definition}
  \begin{definition}
    Let $\alpha\in\CC$ and consider $\text{eva}_\alpha:\QQ[x]\rightarrow\CC$. We say that $\alpha$ is \textit{transcendental} if $\ker(\text{eva}_\alpha)=(0)$, or equivalently, if it is not algebraic.
  \end{definition}
  \begin{theorem}
    The set $\overline{\QQ}\subset\CC$ is countable.
  \end{theorem}
  \begin{prop}
    Let $K$, $L$ be fields. Then, all field morphism $K\rightarrow L$ is injective.
  \end{prop}
  \begin{definition}
    Let $K$, $L$ be two fields. A \textit{field extension $\quot{L}{K}$} is a field morphism $K\hookrightarrow L$.
  \end{definition}
  \begin{prop}
    Let $K$, $L$ be fields and $\quot{L}{K}$ be a field extension. Then, $L$ is a vector space over $K$. Reciprocally, if $L$ is a vector space over $K$ satisfying: $$(\lambda\cdot 1)\cdot(\mu\cdot 1)=(\lambda\cdot\mu)\cdot 1\qquad\forall\lambda,\mu\in K$$ then the morphism $f:K\rightarrow L$ defined as $f(\lambda)=\lambda\cdot 1$ is a field morphism and $\quot{L}{K}$ is a field extension.
  \end{prop}
  \begin{definition}
    Let $K$, $L$ be fields. We say that an extension of field $\quot{L}{K}$ is finite if the dimension of $L$ as a vector space over $K$ is finite. In this case, we define the \textit{degree of the extension} as: $$[L:K]:=\dim_K(L)$$
  \end{definition}
  \begin{prop}
    Let $K$ be a field, $p(x)\in K[x]$ a monic and irreducible polynomial of degree $d\geq 1$. Let $L=\quot{K[x]}{(p(x))}$. Then, $\quot{L}{K}$ is a field extension of degree $d$, and the set $\{1,\bar{x},\ldots,\bar{x}^{d-1}\}$ is a basis of the vector space $L$ over $K$. Furthermore, $\bar{x}\in L$ is a root of $p(x)$ in $L$.
  \end{prop}
\end{multicols}
\end{document}