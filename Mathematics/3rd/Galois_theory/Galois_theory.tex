\documentclass[../../../main_math.tex]{subfiles}
% break line on the "Separability theorem".

\begin{document}
\renewcommand{\col}{\alg}
\begin{multicols}{2}[\section{Galois theory}]
  \subsection{Introduction}
  \subsubsection{Solvability of quadratic, cubic and quartic polynomials}
  \begin{definition}
    A \emph{quadratic polynomial} over a field $K$ is a function of the form:
    $$p(x)=ax^2+bx+c$$ where $a,b,c\in K$.
    A \emph{cubic polynomial} over a field $K$ is a function of the form:
    $$p(x)=ax^3+bx^2+cx+d$$ where $a,b,c,d\in K$.
    A \emph{quartic polynomial} over a field $K$ is a function of the form:
    $$p(x)=ax^4+bx^3+cx^2+dx+e$$ where $a,b,c,d,e\in K$.
  \end{definition}
  \begin{lemma}
    Let $K$ be a field, $n\geq 2$ and $$p(x)=x^n+a_{n-1}x^{n-1}+\cdots+a_1x+a_0$$ where $a_i\in K$ for $i=0,\ldots,n-1$. Then, the change of variable $x=u-\frac{a_{n-1}}{n}$ transforms the previous equation into $$p(u)=u^n+b_{n-2}u^{n-2}+\cdots+b_1u+b_0$$ for some $b_i\in K$ for $i=0,\ldots,n-1$. This new equation is called \emph{depressed equation}.
  \end{lemma}
  \begin{proposition}
    The solutions of the quadratic polynomial $x^2+bx+c$ are: $$\frac{-b\pm\sqrt{b^2-4c}}{2}$$
  \end{proposition}
  \begin{proposition}
    The solutions of the cubic depressed polynomial $x^3+px+q$ are: $$\alpha+\beta:=\sqrt[3]{-\frac{q}{2}+\sqrt{\frac{q^2}{4}+\frac{p^3}{27}}}+\sqrt[3]{-\frac{q}{2}-\sqrt{\frac{q^2}{4}+\frac{p^3}{27}}}$$
    where the cubic roots are chosen such that $\alpha\beta=-p/3$.
  \end{proposition}
  \begin{proposition}
    The solutions of the quartic depressed polynomial $x^4+ax^2+bx+c$ are: $$-S\pm\frac{1}{2}\sqrt{-4S^2-2a+\frac{b}{S}}\ \ \text{and}\ \  S\pm\frac{1}{2}\sqrt{-4S^2-2a-\frac{b}{S}}$$ where
    \begin{align*}
      S=\frac{\sqrt{-\frac{2}{3}a+\frac{1}{3}\left(Q+\frac{\Delta_0}{Q}\right)}}{2}\  & \  Q=\sqrt[3]{\frac{\Delta_1+\sqrt{{\Delta_1}^2-4{\Delta_0}^3}}{2}} \\
      \Delta_0=a^2+12c\quad                                                           & \quad\Delta_1=2a^3+27b^2-72ac
    \end{align*}
  \end{proposition}
  \subsubsection{Rings, integral domains and fields}
  \begin{proposition}
    \hfill
    \begin{enumerate}
      \item A subring of an integral domain is an integral domain.
      \item A field is an integral domain.
      \item A subring of a field is an integral domain.
    \end{enumerate}
  \end{proposition}
  \begin{lemma}
    Let $K$ be a field and $R\ne\{0\}$ be a ring. Then, all ring morphisms $f:K\rightarrow R$ are injective.
  \end{lemma}
  \begin{definition}
    Let $K$, $L$ be fields. A \emph{field morphism} between $K$ and $L$ is a ring morphism $K\rightarrow L$.
  \end{definition}
  \begin{lemma}
    Let $R$ be a ring. Then, there exists a unique ring morphism $f:\ZZ\rightarrow R$ satisfying:
    \begin{itemize}
      \item $f(1+\overset{(n)}{\cdots}+1)=1_R+\overset{(n)}{\cdots}+1_R$ if $n\geq 1$.
      \item $f(n)=-f(-n)$ if $n\leq -1$.
    \end{itemize}
  \end{lemma}
  \begin{definition}
    Let $R$ be a ring and $f:\ZZ\rightarrow R$ be the ring morphism from $\ZZ$ to $R$. The \emph{characteristic} of $R$, $\ch (R)$, is defined to be the value of $n$ such that $\ker f=\quot{\ZZ}{n\ZZ}$.
  \end{definition}
  \begin{proposition}
    Let $K$ be a field. Then, either $\ch K$ is prime or $\ch K=0$.
  \end{proposition}
  \begin{definition}
    Let $R$ be a ring. We define the  \emph{polynomial ring} $R[x]$ as: $$R[x]:=\{r_0+r_1\cdot x+\cdots+r_n\cdot x^n:r_i\in R\ \forall i\text{ and }n\geq 0\}$$ Moreover, we can iterate this definition to define the polynomial ring in $m$ unknowns: $$R[x_1,\ldots,x_m]=\left(R[x_1,\ldots,x_{m-1}]\right)[x_m]$$
  \end{definition}
  \begin{proposition}[Universal property of polynomials in several variables]
    Let $R$, $S$ be two rings, $f:R\rightarrow S$ be a ring morphism and $s_1,\ldots,s_n\in S$ be not necessarily distinct elements of $S$. Then, the function $\varphi_{s_1,\ldots,s_n}:R[x_1,\ldots,x_n]\rightarrow S$ defined by
    \begin{multline*}
      \varphi_{s_1,\ldots,s_n}\left(\sum_{i_1,\ldots,i_n\geq 0}r_{i_1,\ldots,i_n}{x_1}^{i_1}\cdots {x_n}^{i_n}\right)=\\=\sum_{i_1,\ldots,i_n\geq 0}f(r_{i_1,\ldots,i_n}){s_1}^{i_1}\cdots {s_n}^{i_n}
    \end{multline*}
    is the unique ring morphism such that $\varphi_{s_1,\ldots,s_n}(r)=f(r)$ $\forall r\in R$ and $\varphi_{s_1,\ldots,s_n}(x_i)=s_i$ for $i=1,\ldots,n$. This function is called \emph{evaluation} of $s_1,\ldots,s_n$ through $f$.
  \end{proposition}
  \subsubsection{Field of fractions}
  \begin{theorem}[Universal property of the field of fractions]
    All integral domains are a subring of a field. More explicitly, if $R$ is an integral domain and $K$ is a field, there exists another field $Q(R)$\footnote{Recall \cref{AS_field-frac} of \cref{AS_R} for a formal definition of the field $Q(R)$.} and an injective ring morphism $\iota:R\hookrightarrow Q(R)$ so that for all injective ring morphism $f:R\hookrightarrow K$, there exists a unique field morphism $\psi_f:Q(R)\rightarrow K$ defined by
    $$
      \function{\psi_f}{Q(R)}{K}{\displaystyle\frac{a}{b}}{f(a){f(b)}^{-1}}
    $$
    such that $f=\psi_f\circ\iota$.
  \end{theorem}
  \begin{corollary}
    Let $R$ be an integral domain. The field $Q(R)$ with the injection $\iota$ is unique up to isomorphism, that is, if there is a field $Q'(R)$ and an injective ring morphism $\iota':R\hookrightarrow Q'(R)$ satisfying the property of above, then there is a unique isomorphism $\psi_{\iota'}:Q(R)\cong Q'(R)$ such that $\iota'=\psi_{\iota'}\circ\iota$, where $\iota:R\hookrightarrow Q(R)$. This field $Q(R)$ is called \emph{field of fractions} of $R$.
  \end{corollary}
  \begin{definition}
    Let $K$ be a field. The field of fractions of $K[x]$ is defined as $K(x):=Q(K[x])$ and it is called \emph{field of rational functions}. More generally, the field of fractions of $K[x_1,\ldots,x_n]$ is defined as: $$K(x_1,\ldots,x_n):=Q(K[x_1,\ldots,x_n])$$ The elements of $K(x_1,\ldots,x_n)$ are of the form: $$\left\{\frac{p(x_1,\ldots,x_n)}{q(x_1,\ldots,x_n)}:p,q\in K[x_1,\ldots,x_n]\right\}$$
  \end{definition}
  \begin{lemma}
    Let $R$ be an integral domain. Then, $R[x]$ is also an integral domain and: $$Q(R[x])\cong Q(R)(x)$$
  \end{lemma}
  \begin{corollary}
    Let $K$ be a field. For all $n\geq 2$ we have: $$K(x_1,\ldots,x_n)\cong K(x_1,\ldots,x_{n-1})(x_n)$$
  \end{corollary}
  \subsubsection{Subring and subfield generated by a set}
  \begin{definition}
    Let $(R,+,\cdot)$ be a ring and $X\subseteq R$ be a subset of $R$. Let $$P:=\{S\subseteq R: X\subseteq S\land(S,+,\cdot)\leq(R,+,\cdot)\}$$ Then, the \emph{subring generated} by $X$ is the smallest subring of $(R,+,\cdot)$ containing $X$. That is: $$\langle X\rangle_\text{ring}=\bigcap_{S\in P}S$$
  \end{definition}
  \begin{definition}
    Let $R$ be a ring, $S\subseteq R$ be a subring of $R$ and $A\subseteq R$ be a subset of $R$. We denote by $S[A]$ the smallest subring of $R$ containing $S$ and $A$.
  \end{definition}
  \begin{lemma}
    Let $A$ be a finite set, $R$ and $S$ be rings and $\varphi:R[x_a:a\in A]\rightarrow S$ be the evaluation morphism such that $\varphi(r)=r$ $\forall r\in R$ and $\varphi(x_a)=a$ $\forall a\in A$. Then, $S[A]=\im\varphi$.
  \end{lemma}
  \begin{definition}
    Let $(K,+,\cdot)$ be a field and $X\subseteq K$ be a subset of $K$. Let $$P:=\{L\subseteq K: X\subseteq L,(L,+,\cdot)\text{ is a subfield of }(R,+,\cdot)\}$$ Then, the \emph{subfield generated} by $X$ is the smallest subfield of $(K,+,\cdot)$ containing $X$. That is: $$\langle X\rangle_\text{field}=\bigcap_{L\in P}L$$
  \end{definition}
  \begin{definition}
    Let $L$ be a field, $K\subseteq L$ be a subfield of $L$ and $A\subseteq L$ be a subset of $L$. We denote by $K(A)$ the smallest subfield of $L$ containing $K$ and $A$.
  \end{definition}
  \subsubsection{Symmetric polynomials}
  \begin{definition}[Symmetric polynomials]
    Let $R$ be a ring, $n\in\NN$ and $p\in R[x_1,\ldots,x_n]$. We say that $p$ is a \emph{symmetric polynomial} if $\forall \sigma\in \text{S}_n$, we have that $p(x_1,\ldots,x_n)=p(x_{\sigma(1)},\ldots,x_{\sigma(n)})$. We denote by ${R[x_1,\ldots,x_n]}^{\text{S}_n}$ the set of all symmetric polynomials over $R[x_1,\ldots,x_n]$.
  \end{definition}
  \begin{definition}
    Let $R$ be a ring and $n\in\NN$. We define the \emph{elementary symmetric polynomials} $s_1,\ldots,s_n$ as: $$s_k=\sum_{1\leq i_1<\cdots<i_k\leq n}x_{i_1}\cdots x_{i_k}\quad\text{for } k=1,\ldots,n\footnote{For example, for $n=3$ we have: $s_1=x_1+x_2+x_3$, $s_2=x_1x_2+x_1x_3+x_2x_3$ and $s_3=x_1x_2x_3$.}$$
  \end{definition}
  \begin{definition}
    Let $n\in\NN$. We define the \emph{lexicographic order} $<_\text{lex}$ in $\NN^n$ as:
    \begin{multline*}
      (a_1,\ldots,a_n)<_\text{lex}(b_1,\ldots,b_n)\iff\\\iff\exists j\in\NN:a_1=b_1,\ldots,a_j=b_j,a_{j+1}<b_{j+1}
    \end{multline*}
  \end{definition}
  \begin{proposition}
    The pair $(\NN^n,<_\text{lex})$ is a totally ordered set. Moreover, if $x,y,z,t\in\NN^n$ are such that $x<_\text{lex}y$ and $z<_\text{lex}t$, then $x+z<_\text{lex}y+t$.
  \end{proposition}
  \begin{definition}
    Let $R$ be a ring, $n\in\NN$ and $p\in R[x_1,\ldots,x_n]$. Suppose $p$ is of the form: $$p(x_1,\ldots,x_n)=\sum_{\substack{i_1,\ldots,i_n=1\\i_1+\cdots+i_n=n}}^na_{i_1,\ldots,i_n}{x_1}^{i_1}\cdots{x_n}^{i_n}$$ If $p(x_1,\ldots,x_n)\ne 0$, we define the \emph{lexicographic degree} of $p$ as:
    $$
      \deg_{<_\text{lex}}(p):=\max_{<_\text{lex}}\{(i_1,\ldots,i_n):a_{i_1,\ldots,i_n}\ne 0\}\footnote{Here, the notation $\displaystyle \max_{<_\text{lex}}$ means that the maximum is taken with respect to the order $<_\text{lex}$.}
    $$
    If $p(x_1,\ldots,x_n)=0$, we define $\deg_{<_\text{lex}}(p):=-\infty$.
  \end{definition}
  \begin{proposition}
    Let $R$ be a ring, $n\in\NN$ and $p,q\in R[x_1,\ldots,x_n]$. Then:
    \begin{enumerate}
      \item $\deg_{<_\text{lex}}(p+q)\leq\max\{\deg_{<_\text{lex}}(p),\deg_{<_\text{lex}}(q)\}$.
      \item $\deg_{<_\text{lex}}(pq)=\deg_{<_\text{lex}}(p)+\deg_{<_\text{lex}}(q)$.
    \end{enumerate}
  \end{proposition}
  \begin{lemma}[Waring's method]
    Let $R$ be an integral domain and $p\in {R[x_1,\ldots,x_n]}^{\text{S}_n}$. Suppose that $\deg_{<_\text{lex}}(p)=(a_1,\ldots,a_n)$ and let $\lambda\in R\setminus\{0\}$ be the coefficient of ${x_1}^{a_1}\cdots{x_n}^{a_n}$ in $p(x_1,\ldots,x_n)$. Then, $a_1\geq \cdots\geq a_n$ and if $$q:=p-\lambda {s_n}^{a_n}{s_{n-1}}^{a_{n-1}-a_n}{s_{n-2}}^{a_{n-2}-a_{n-1}}\cdots {s_1}^{r_1-r_2}$$ then we have $\deg_{<_\text{lex}}(q)<_\text{lex}\deg_{<_\text{lex}}(p)$.
  \end{lemma}
  \begin{theorem}[Fundamental theorem of symmetric polynomials]
    Let $R$ be a ring and $n\in\NN$. Then: $${R[x_1,\ldots,x_n]}^{\text{S}_n}=R[s_1,\ldots,s_n]$$ That is, every polynomial in $R[x_1,\ldots,x_n]^{\text{S}_n}$ can be expressed uniquely in terms of elementary symmetric polynomials.
  \end{theorem}
  \subsubsection{Cyclotomic polynomials}
  \begin{definition}
    We define the \emph{$n$-th cyclotomic polynomial} as the unique irreducible polynomial $\Phi_n(x)\in\ZZ[x]$ such that $\Phi_n(x)\mid x^n-1$ and $\Phi_n(x)\nmid x^m-1$ for all $m<n$. For example, the first 8 cyclotomic polynomials are:
    \begin{align*}
      \Phi_1(x) & = x - 1                               \\
      \Phi_2(x) & = x + 1                               \\
      \Phi_3(x) & = x^2 + x + 1                         \\
      \Phi_4(x) & = x^2 + 1                             \\
      \Phi_5(x) & = x^4 + x^3 + x^2 + x +1              \\
      \Phi_6(x) & = x^2 - x + 1                         \\
      \Phi_7(x) & = x^6 + x^5 + x^4 + x^3 + x^2 + x + 1 \\
      \Phi_8(x) & = x^4 + 1
    \end{align*}
  \end{definition}
  \begin{proposition}
    Let $n\in\NN$. Then: $$\Phi_n(x)=\prod_{\substack{1\leq k\leq n\\\gcd(k,n)=1}}\left(x-\exp{2\pi\ii\frac{k}{n}}\right)$$
  \end{proposition}
  \begin{theorem}
    Let $n\in\NN$. Then: $$x^n-1=\prod_{d\mid n}\Phi_d(x)$$
  \end{theorem}
  \subsection{Field extensions}
  \begin{proposition}
    Let $K$, $L$ be fields. Then, any field morphism $K\rightarrow L$ is injective.
  \end{proposition}
  \begin{definition}
    Let $K$, $L$ be two fields. A \emph{field extension} $L/K$ is a field morphism $K\hookrightarrow L$.
  \end{definition}
  \begin{proposition}
    Let $L/K$ be a field extension. Then, $L$ is a vector space over $K$. Reciprocally, if $L$ is a vector space over $K$ satisfying: $$(\lambda\cdot 1)\cdot(\mu\cdot 1)=(\lambda\cdot\mu)\cdot 1\qquad\forall\lambda,\mu\in K$$ then the morphism $f:K\rightarrow L$ defined as $f(\lambda)=\lambda\cdot 1$ is a field morphism and $L/K$ is a field extension.
  \end{proposition}
  \begin{definition}
    Let $L/K$ be a field extension. We define the \emph{degree} of the extension $L/K$ as: $$[L:K]:=\dim_K(L)$$ We say that the extension $L/K$ is \emph{finite} if $[L:K]$ is finite. Otherwise, we say that $L/K$ is \emph{infinite}.
  \end{definition}
  \begin{lemma}[Kronecker's lemma]
    Let $K$ be a field, $p(x)\in K[x]$ a monic and irreducible polynomial of degree $d\geq 1$ and $L=\quot{K[x]}{(p(x))}$. Then, $L/K$ is a field extension of degree $d$, and the set $\{1,\overline{x},\ldots,{\overline{x}}^{d-1}\}$ is a basis of the vector space $L$ over $K$. Furthermore, $\overline{x}\in L$ is a root of $p(x)$ in $L$.
  \end{lemma}
  \begin{corollary}
    Let $K$ be a field, $p(x)\in K[x]$ a monic and irreducible polynomial of degree $d\geq 1$. Then, there exists a field extension $L/K$ such that $p(x)$ has a root in $L$.
  \end{corollary}
  \subsubsection{Algebraic and transcendental numbers}
  \begin{definition}
    Let $L/K$ be a field extension and $\alpha\in L$. Consider the ring morphism: $$\function{\varphi_\alpha}{K[x]}{L}{p(x)}{p(\alpha)}$$
    \begin{enumerate}
      \item We say that $\alpha$ is \emph{algebraic} over $K$ if $\ker\varphi_\alpha=(p(x))$, where $p(x)\in K[x]$ is an irreducible polynomial of degree $d\geq 1$. This polynomial is called \emph{irreducible polynomial} of $\alpha$ over $K$ and it is denoted by $\irr(\alpha,K)(x)$.
      \item We say that $\alpha$ is \emph{transcendental} over $K$ if $\ker\varphi_\alpha=(0)$, or equivalently, if it is not algebraic.
    \end{enumerate}
  \end{definition}
  \begin{proposition}
    $\pi$ and $\exp{}$ are transcendental over $\QQ$.
  \end{proposition}
  \begin{proposition}
    Let $L/K$ be a field extension and $\alpha\in L$ be a root of a monic and irreducible polynomial $p(x)\in K[x]$. Then, $\alpha$ is algebraic and $\irr(\alpha,K)(x)=p(x)$.
  \end{proposition}
  \begin{theorem}
    Let $\overline{\QQ}\subset\CC$ be the set of algebraic numbers over $\QQ$. Then, $\overline{\QQ}$ is countable.
  \end{theorem}
  \subsubsection{Simple extensions}
  \begin{definition}
    A field extension $L/K$ is called \emph{simple} if $L=K(\alpha)$ for some $\alpha\in L$. In that case, the element $\alpha$ is called \emph{primitive element} of $L$ over $K$.
  \end{definition}
  \begin{theorem}[Steinitz's theorem]
    Let $L/K$ be a finite field extension. Then, $L/K$ is simple if and only if there is a finite number of intermediate fields between $K$ and $L$.
  \end{theorem}
  \begin{proposition}
    Let $L/K$ be a field extension and $\alpha\in L$. Then:
    \begin{itemize}
      \item If $\alpha$ is algebraic over $K$, then: $$K(\alpha)=K[\alpha]\cong\quot{K[x]}{(\irr(\alpha,K)(x))}$$
      \item If $\alpha$ is transcendental over $K$, then: $$K(\alpha)\cong K(x)$$
    \end{itemize}
  \end{proposition}
  \begin{corollary}
    Let $L/K$ be a field extension and $\alpha\in L$. Then, $K(\alpha)/K$ is finite if and only if $\alpha$ is algebraic over $K$. Furthermore in that case: $$[K(\alpha):K]=\deg(\irr(\alpha,K)(x))$$
  \end{corollary}
  \begin{theorem}[Tower formula]
    Let $F/L$ and $L/K$ be field extensions. Then: $$[F:K]=[F:L][L:K]$$
  \end{theorem}
  \begin{proposition}
    Let $L/K$ be a field extension and $\alpha\in L$ be algebraic. Then:
    \begin{enumerate}
      \item The following statements are equivalent:
            \begin{enumerate}
              \item $\alpha\in K$
              \item $\irr(\alpha,K)(x)=x-\alpha$
              \item $\deg(\irr(\alpha,K)(x))=1$
            \end{enumerate}
      \item If $K'/K$ is another field extension, then: $$\irr(\alpha,K')(x)\mid\irr(\alpha,K)(x)$$ and, moreover, $\irr(\alpha,K')(x)=\irr(\alpha,K)(x)\iff\deg(\irr(\alpha,K')(x))=\deg(\irr(\alpha,K)(x))$.
      \item $\deg(\irr(\alpha,K)(x))\mid [L:K]$
    \end{enumerate}
  \end{proposition}
  \begin{definition}
    Let $n\in\NN$ and $K_0,\ldots,K_n$ be fields. A \emph{tower of fields} is a sequence of field extensions $K_j/K_{j-1}$ for $j=1,\ldots,n$. We will denoted this tower of fields as: $$K_n/K_{n-1}/\cdots/K_0$$
  \end{definition}
  \begin{corollary}
    Let $n\in\NN$ and $K_n/K_{n-1}/\cdots/K_0$ be a tower of fields. Then: $$[K_n:K_0]=[K_n:K_{n-1}][K_{n-1}:K_{n-2}]\cdots[K_1:K_0]$$
  \end{corollary}
  \begin{definition}
    A field extension $L/K$ is called \emph{finitely generated} if there exists $\alpha_1,\ldots,\alpha_n\in L$ such that $L=K(\alpha_1,\ldots,\alpha_n)$.
  \end{definition}
  \begin{definition}
    Let $L/K$, $F/K$ be field extensions. We define the \emph{compositum} of $L$ and $F$, denoted as $LF$, as smallest field containing $L$ and $F$.
  \end{definition}
  \begin{proposition}
    Let $L/K$, $F/K$ be field extensions. Then, $$[FL:K]\leq[F:K][L:K]$$ and the equality holds if the numbers $[F:K]$ are $[L:K]$ coprime.
  \end{proposition}
  \subsubsection{Algebraic extensions}
  \begin{definition}
    Let $L/K$ be a field extension. We say that $L/K$ is \emph{algebraic} if $\forall\alpha\in L$, $\alpha$ is algebraic over $K$.
  \end{definition}
  \begin{definition}
    Let $L/K$ be a field extension. We say that $L/K$ is \emph{purely transcendental} if $\forall\alpha\in L\setminus K$, $\alpha$ is transcendental over $K$.
  \end{definition}
  \begin{lemma}
    Let $L/K$ be a finite field extension. Then, $L/K$ is algebraic.
  \end{lemma}
  \begin{proposition}
    Let $L/K$ be a field extension. The following are equivalent:
    \begin{enumerate}
      \item $L/K$ is finite.
      \item $L/K$ is algebraic and there exist $\alpha_1,\ldots,\alpha_n\in L$ such that $L=K(\alpha_1,\ldots,\alpha_n)$.
      \item There exist $\alpha_1,\ldots,\alpha_n\in L$ with $\alpha_i$ algebraic over $K(\alpha_1,\ldots,\alpha_{i-1})$ for $i=1,\ldots,n$ such that $L=K(\alpha_1,\ldots,\alpha_n)$.
    \end{enumerate}
  \end{proposition}
  \begin{proposition}
    Let $L/F/K$ be a tower of fields such that $F/K$ is algebraic, and $\alpha\in L$. Suppose that $\alpha$ is algebraic over $F$. Then, $\alpha$ is algebraic over $K$.
  \end{proposition}
  \begin{proposition}
    Let $L/F/K$ be a tower of fields. Then:
    \begin{enumerate}
      \item If $L/K$ is algebraic, any subring $R$ such that $K\subseteq R\subseteq L$ is a subfield.
      \item $L/F$ and $F/K$ are algebraic $\iff L/K$ is algebraic.
      \item If $\alpha,\beta\in L$ are algebraic over $K$, then so are $\alpha+\beta$, $\alpha\beta$ and $\alpha\beta^{-1}$ (if $\beta\ne 0$).
      \item The set $$E:=\{\alpha\in L:\alpha\text{ is algebraic over }K\}$$ is a subfield of $L$, the field extension $E/K$ is algebraic and if $L\ne E$, then $L/E$ is purely transcendental.
    \end{enumerate}
  \end{proposition}
  \subsubsection{Morphisms of extensions}
  \begin{definition}
    Let $K$, $L$, $F$ be fields and $f:K\hookrightarrow L$ and $g:K\hookrightarrow F$ be field extensions. A \emph{morphism of field extensions} between $f$ and $g$ (sometimes called \emph{$K$-field morphism}) is a field morphism $h:L\rightarrow F$ such that $g=h\circ f$. We will denote the set of all such morphisms by:
    $$\Mor_K(f,g):=\{h:L\longrightarrow F:h\circ f=g\}$$
    If $f$ and $g$ are the natural inclusions, we will denote: $$\Mor_K(L,F):=\Mor_K(f,g)=\{h:L\longrightarrow F:h|_K=\id_K\}$$
    If $f$ is the natural inclusion but $g$ isn't, we will denote: $$\Mor_K(L,g):=\Mor_K(f,g)=\{h:L\longrightarrow F:h|_K=g\}$$
    Finally, if $g$ is the natural inclusion but $f$ isn't, we will denote: $$\Mor_K(f,F):=\Mor_K(f,g)=\{h:L\longrightarrow F:h\circ f=\id_K\}$$
  \end{definition}
  \begin{definition}
    Let $K$, $L$, $F$ be fields and $f:K\hookrightarrow L$ and $g:K\hookrightarrow F$ be field extensions. We define the following sets:
    \begin{gather*}
      \Iso_K(f,g):=\{h\in\Mor_K(f,g):h\text{ is bijective}\} \\
      \Aut_K(f):=\Iso_K(f,f)
    \end{gather*}
    If $f$ and $g$ are the natural inclusions, we will denote\footnote{And we define $\Iso_K(L,g)$ and $\Iso_K(f,F)$ analogously as we did before.}:
    \begin{gather*}
      \Iso_K(L,F):=\{h\in\Mor_K(L,F):h\text{ is bijective}\} \\
      \Aut_K(L):=\Iso_K(L,L)
    \end{gather*}
  \end{definition}
  \begin{lemma}
    Let $L/K$ be a field extension. Then, $(\Aut_K(L),\circ)$ is a group and it is called \emph{Galois group} of $L/K$. Hence, $\Aut_K(L)$ is also denoted as $\Gal(L/K)$\footnote{For the general case when $f:K\hookrightarrow L$ is a field extension, we define $\Gal(f):=\Aut_K(f)$.}.
  \end{lemma}
  \begin{proposition}
    Let $L/K$ be a finite field extension. Then, $\Gal(L/K)=\Mor_K(L,L)$.
  \end{proposition}
  \begin{lemma}
    Let $K$, $L$, $F$ be fields and $f:K\rightarrow L$, $g:K \rightarrow F$ be field morphisms. Let $h\in\Mor_K(f,g)$, $\alpha\in L$, and $p(x)\in K[x]$. Then: $$h(f(p)(\alpha))=g(p)(h(\alpha))\footnote{Here $f(p)$ denotes the evaluation through $f$ of the polynomial $p(x)$. That is, assuming that $p(x)$ is of the form $p(x)=\sum_{i=1}^na_ix^i$, then $f(p):\sum_{i=1}^na_ix^i\longmapsto\sum_{i=1}^nf(a_i)x^i$.}$$
    If $f$ and $g$ are the natural inclusions, then: $$h(p(\alpha))=p(h(\alpha))$$
  \end{lemma}
  \begin{lemma}
    Let $L/K$, $g:K \hookrightarrow F$ be field extensions and $\alpha\in L$ be algebraic over $K$. Then, we have the bijection $$\Mor_K(K(\alpha),g)\overset{\psi}{\cong}\{\beta\in F:g(\irr(\alpha,K))(\beta)=0\}$$ given by $\psi(h)=h(\alpha)$.
    If $g$ is the natural inclusion, then: $$\Mor_K(K(\alpha),F)\overset{\psi}{\cong}\{\beta\in F:\irr(\alpha,K)(\beta)=0\}$$ given by $\psi(h)=h(\alpha)$.
  \end{lemma}
  \begin{corollary}
    Let $K(\alpha)/K$ be a finite field extension. Then:
    $$\Gal(K(\alpha)/K)\cong\{\beta\in K(\alpha):\irr(\alpha,K)(\beta)=0\}$$
    Therefore, $\Gal(K(\alpha)/K)$ is finite and: $$|\Gal(K(\alpha)/K)|\leq[K(\alpha):K]$$
  \end{corollary}
  \begin{proposition}
    Let $K$, $L$, $F$ be fields and $f:K\hookrightarrow L$ and $g:K\hookrightarrow F$ be field extensions. Then:
    \begin{enumerate}
      \item If $f':K\rightarrow L'$, $\varphi:L'\rightarrow L$ are field extensions, then: $$\Mor_K(f,g)=\bigsqcup_{h\in\Mor_K(f',g)}\Mor_{L'}(\varphi,h)$$
            In particular, if $f$, $g$, $f'$ and $\varphi$ are the natural inclusions, then:
            $$\Mor_K(L,F)=\bigsqcup_{h\in\Mor_K(L',F)}\Mor_{L'}(L,h)$$
      \item If $\Iso_K(f,g)\ne\varnothing$, then $\Iso_K(f,g)\cong\Gal(f)$ by sending $h\mapsto h\circ {h_0}^{-1}$, where $h_0\in\Iso_K(f,g)$ is a fixed isomorphism. Analogously, if $\Iso_K(L,F)\ne\varnothing$, then $\Iso_K(L,F)\cong\Gal(L/K)$.
    \end{enumerate}
  \end{proposition}
  \subsection{Finite fields}
  \begin{definition}[Finite field]
    A \emph{finite field} $F$ is a finite set which is a field.
  \end{definition}
  \begin{proposition}
    Let $F$ be a finite field. Then, $F=p^n$ where $p$ is a prime number and $n\in\NN$.
  \end{proposition}
  \begin{theorem}
    Let $p$ be a prime number and $n\in\NN$. Then, there exists a unique field with $p^n$ elements up to isomorphism which we will denote by $\FF_{p^n}$\footnote{Another commonly used notation to denote the field with $p^n$ elements is $\text{GF}(p^n)$.}.
  \end{theorem}
  \begin{proposition}
    Let $p$ be a prime number and $d,n\in\NN$. Then: $$\FF_{p^d}\subseteq\FF_{p^n}\iff d\mid n$$ And in that case, $[\FF_{p^n}:\FF_{p^d}]=\frac{n}{d}$.
  \end{proposition}
  \begin{theorem}
    Let $p$ be a prime number and $d\in\NN$. We define the set $P_{p,d}$ as:
    \begin{multline*}
      P_{p,d}:=\{f(x)\in\FF_p[x]:\deg(f(x))=d\ \land \\\text{$f(x)$ is monic and irreducible}\}
    \end{multline*}
    Then, for all $n\in\NN$ we have: $$x^{p^n}-x=\prod_{d\mid n}\prod_{f(x)\in P_{p,d}}f(x)$$
  \end{theorem}
  \begin{corollary}
    Let $p$ be a prime number and $d,n\in\NN$. Then: $$p^n=\sum_{d\mid n}d|P_{p,d}|$$
  \end{corollary}
  \begin{corollary}
    For all prime numbers $p$ and for all $n\in\NN$, there exists a monic and irreducible polynomial of degree $n$ in $\FF_p[x]$.
  \end{corollary}
  \begin{corollary}
    Let $p$ be a prime number and $n\in\NN$. Then, $\FF_{p^n}=\FF_p(\alpha)$ for some $\alpha\in\FF_{p^n}$. Thus, the extension $\FF_{p^n}/\FF_p$ is simple.
  \end{corollary}
  \begin{definition}
    Let $p$ be a prime number and $R$ be a ring such that $\ch R=p$. We define the \emph{Frobenius endomorphism} as: $$\function{\text{Frob}_R}{R}{R}{r}{r^p}$$
  \end{definition}
  \begin{theorem}
    Let $p$ be a prime number and $n\in\NN$. Then, $\text{Frob}_{\FF_{p^n}}\in\Gal(\FF_{p^n}/\FF_p)$ and, furthermore: $$\Gal(\FF_{p^n}/\FF_p)=\langle\text{Frob}_{\FF_{p^n}}\rangle\cong\quot{\ZZ}{n\ZZ}$$
  \end{theorem}
  \begin{corollary}
    Let $p$ be a prime number, $n\in\NN$, $q=p^n$ and denote $\text{Frob}_q:={\left(\text{Frob}_{\FF_{p^n}}\right)}^n$. Then, for all $r\in\NN$: $$\Gal(\FF_{q^r}/\FF_q)=\langle\text{Frob}_q\rangle\cong\quot{\ZZ}{r\ZZ}$$
  \end{corollary}
  \begin{definition}[Perfect fields]
    A field $K$ is called \emph{perfect} if either $\ch K=0$ or $\ch K=p>0$ and $\text{Frob}_K\in\Aut(K)$.
  \end{definition}
  \subsection{Algebraic field extensions}
  \subsubsection{Splitting field}
  \begin{definition}
    Let $K$, $L$ be fields and $p(x)\in K[x]$ be a polynomial such that $\deg p(x)=n\geq 1$. We say that $p(x)$ \emph{splits into linear factors} on $L$ if $p(x)=a_n\prod_{i=1}^n(x-a_i)$, where $a_i\in L$ for $i=1,\ldots,n$.
  \end{definition}
  \begin{theorem}[Kronecker's theorem]
    Let $K$ be a field and $S\subset K[x]$ be a finite set. Then, there exists a finite field extension $L/K$ such that all polynomials in $S$ split into linear factors on $L$.
  \end{theorem}
  \begin{theorem}
    Let $K$ be a field and $L=K(\alpha_1,\ldots,\alpha_n)$. Let $f:K\hookrightarrow F$ be a field morphism such that $f(\irr(\alpha_i,K))(x)$ splits into linear factors on $F$ for all $i=1,\ldots,n$. Then, $$1\leq|\Mor_K(L,f)|\leq [L:K]$$ and, furthermore, $|\Mor_K(L,f)|=[L:K]$ if and only if $f(\irr(\alpha_i,K))(x)$ has no repeated roots on $F$ for all $i=1,\ldots,n$.
  \end{theorem}
  \begin{definition}[Splitting field]
    Let $L/K$ be a finite field extension and $p(x)\in K[x]\setminus K$ be such that it splits into linear factors in $L$. Let $\alpha_1,\ldots,\alpha_n$ be their roots. The \emph{splitting field} of $p(x)$ over $K$ is the smallest subfield $K(\alpha_1,\ldots,\alpha_n)$ of $L$ where $p(x)$ splits into linear factors.
  \end{definition}
  \begin{proposition}
    Let $K$ be a field and $p(x)\in K[x]\setminus K$. Then, $L$ is a splitting field of $p(x)$ if and only if $p(x)$ splits into linear factors on $L$ and for all tower of fields $L/F/K$ with $F\ne L$, $p(x)$ doesn't split into linear factors on $F$.
  \end{proposition}
  \begin{theorem}[Existence of the splitting field]
    Let $K$ be a field and $p(x)\in K[x]\setminus K$. Then, there exists a splitting field of $p(x)$ over $K$.
  \end{theorem}
  \begin{theorem}
    Let $K$ be a field, $p(x)\in K[x]\setminus K$ and $L/K$ and $F/K$ be two splitting fields of $p(x)$ over $K$. Then, $[L:K]=[F:K]$ and $$1\leq|\Iso_K(L,F)|\leq [L:K]$$ Furthermore, $|\Iso_K(L,F)|=[L:K]$ if and only if all irreducible factors of $p(x)$ have no repeated roots on $F$.
  \end{theorem}
  \begin{corollary}
    Let $K_1$, $K_2$ be fields, $f:K_1\rightarrow K_2$ be a field isomorphism, $p(x)\in K[x]\setminus K$ and $L_1/K_1$, $L_2/K_2$ be two field extensions. Suppose $L_1$ is the splitting of $p(x)$ over $K_1$ and $L_2$ be the splitting of $f(p)(x)$ over $K_2$. Then, there exists a field isomorphism $\varphi:L_1\rightarrow L_2$ such that $\varphi|_{K_1}=f$.
  \end{corollary}
  \begin{corollary}[Unicity of the splitting field]
    Let $K$ be a field and $p(x)\in K[x]\setminus K$. Then, any two splitting fields of $p(x)$ over $K$ are isomorphic.
  \end{corollary}
  \begin{corollary}
    Let $K$ be a field, $p(x)\in K[x]\setminus K$ and $L$ be the splitting field of $p(x)$ over $K$. Then: $$\left|\Gal(L/K)\right|\leq[L:K]$$ and $\left|\Gal(L/K)\right|=[L:K]$ if and only if $p(x)$ has no repeated roots on $L$.
  \end{corollary}
  \begin{corollary}
    Let $L/K$ be a field extension and $p(x)\in K[x]$. Then, the splitting field of $p(x)$ over $L$ contains the splitting field of $p(x)$ over $K$.
  \end{corollary}
  \begin{proposition}
    Let $p$ be a prime number and $n\in\NN$. Then, $\FF_{p^n}$ is the splitting field of $x^{p^n}-x\in\FF_p[x]$.
  \end{proposition}
  \subsubsection{Normal extensions}
  \begin{definition}
    An algebraic field extension $L/K$ is \emph{normal} if for all irreducible polynomial $p(x)\in K[x]$ we have that if $p(x)$ has a root in $L$, then $p(x)$ splits into linear factors in $L$.
  \end{definition}
  \begin{proposition}
    Let $L/K$ be finite field extension of degree 2. Then, $L/K$ is normal.
  \end{proposition}
  \begin{theorem}
    Let $L/K$ be finite field extension. $L/K$ is normal if and only if $L$ is the splitting field of some polynomial $p(x)\in K[x]\setminus K$.
  \end{theorem}
  \begin{corollary}
    Let $L/K$ be finite field extension. Then, there exists a field extension $F/L$ such that:
    \begin{enumerate}
      \item $F/K$ is finite and normal.
      \item For all field extensions $H/L$ with $H/K$ normal there is at least one $L$-field morphism $f:F\rightarrow H$.
    \end{enumerate}
    The extension $F/L$ is called \emph{normal closure} of $L/K$.
  \end{corollary}
  \begin{corollary}
    Let $L/F/K$ be a tower of fields such that $L/K$ is finite and normal. Then, $L/F$ is also finite and normal.
  \end{corollary}
  \begin{corollary}
    Let $L/F/K$ be a tower of fields such that $L/K$ is finite and normal. Let $f\in\Mor_K(F,L)$. Then, there exists at least one automorphism $\varphi\in\Gal(L/K)$ such that $\varphi|_F=f$.
  \end{corollary}
  \begin{corollary}
    Let $L/K$ be a finite field extension. Then: $$|\Gal(L/K)|\leq[L:K]$$ Hence, $\Gal(L/K)$ is a finite group.
  \end{corollary}
  \begin{corollary}
    Let $L/F/K$ be a tower of fields such that $L/K$ is finite and normal. Then, $F/K$ is normal if and only if $\varphi(F)=F$ $\forall\varphi\in\Gal(L/K)$.
  \end{corollary}
  \subsubsection{Separable polynomials}
  \begin{definition}[Formal derivative]
    Let $R$ be a ring and $p(x)=\sum_{n=0}^da_nx^n\in R[x]$. We define \emph{formal derivative} of $p(x)$ as: $$p'(x):=\sum_{n=1}^dna_nx^{n-1}$$
  \end{definition}
  \begin{proposition}
    Let $R$ be a ring, $a\in R$ and $p(x),q(x)\in R[x]$. Then:
    \begin{enumerate}
      \item ${(p(x)+q(x))}'=p'(x)+q'(x)$
      \item $(ap(x))'=ap'(x)$
      \item $(p(x)q(x))'=p'(x)q(x)+p(x)q'(x)$
      \item $$\deg(p'(x))\leq\deg(p(x))-1$$ And the inequality holds if either $\ch(R)=0$ or $\gcd(\ch(R),\deg(p(x)))=1$.
    \end{enumerate}
  \end{proposition}
  \begin{proposition}
    Let $K$ be a field, $p(x)\in K[x]\setminus K$, $L$ be a splitting field of $p(x)$ over $K$ and $d(x):=\gcd(p(x),p'(x))$. Then:
    $$\{\alpha\in L:d(\alpha)=0\}=\{\alpha\in L:{(x-\alpha)}^2\mid p(x)\}$$
  \end{proposition}
  \begin{definition}
    Let $K$ be a field and $p(x)\in K[x]$. We say that $p(x)$ is \emph{separable} if it doesn't have multiple roots in its splitting field.
  \end{definition}
  \begin{corollary}
    Let $K$ be a field and $p(x)\in K[x]\setminus K$. Then: $$p(x)\text{ is separable}\iff\gcd(p(x),p'(x))=1$$
  \end{corollary}
  \begin{corollary}
    Let $K$ be a field such that $\ch K=0$ and $p(x)\in K[x]$ be an irreducible polynomial. Then, $p(x)$ is separable.
  \end{corollary}
  \begin{lemma}
    Let $K$ be a field such that $\ch K=p>0$ and $p(x)\in K[x]$. Then: $$p'(x)=0\iff\exists q(x)\in K[x]:p(x)=q(x^p)$$
  \end{lemma}
  \begin{corollary}
    Let $K$ be a field such that $\ch K=p>0$, $p(x)\in K[x]$ and $q(x):=p(x^p)$. Then, all roots of $q(x)$ are multiple.
  \end{corollary}
  \begin{corollary}
    Let $K$ be a field such that $\ch K=p>0$, $p(x)\in K[x]$ and $q(x):=p(x^p)+bx$, where $b\in K^*$. Then, all roots of $q(x)$ are simple.
  \end{corollary}
  \begin{theorem}
    Let $K$ be a perfect field. Then, any irreducible polynomial over $K$ is separable.
  \end{theorem}
  \subsubsection{Separable extensions}
  \begin{definition}[Separable extension]
    Let $L/K$ be an algebraic field extension and $\alpha\in L$. We say that $\alpha$ is \emph{separable} over $K$ if $\irr(\alpha,K)(x)$ is separable. We say that $L/K$ is \emph{separable} if and only if $\forall \alpha\in L$, $\alpha$ is separable over $K$.
  \end{definition}
  \begin{corollary}
    Let $K$ be a perfect field. Then, any algebraic extension $L/K$ is separable.
  \end{corollary}
  \begin{theorem}[Separability theorem]
    Let\\ $K(\alpha_1,\ldots,\alpha_n)/K$ be a finite field extension and $f:K\rightarrow L$ a field morphism such that $f(\irr(\alpha_i,K))(x)$ splits into linear factors $\forall i=1,\ldots,n$. Then, the following statements are equivalent:
    \begin{enumerate}
      \item $K(\alpha_1,\ldots,\alpha_n)/K$ is separable.
      \item $\alpha_1,\ldots,\alpha_n$ are separable over $K$.
      \item $|\Mor_K(K(\alpha_1,\ldots,\alpha_n),f)|=[K(\alpha_1,\ldots,\alpha_n):K]$.
    \end{enumerate}
  \end{theorem}
  \begin{corollary}
    Let $K$ be a field and $L$ be the splitting field of a separable polynomial $p(x)\in K[x]$. Then, $L/K$ is separable.
  \end{corollary}
  \begin{proposition}
    Let $L/F/K$ be a tower of fields. Then:
    \begin{enumerate}
      \item $L/F$ and $F/K$ are separable $\iff L/K$ is separable.
      \item The set $$E:=\{\alpha\in L:\alpha\text{ is separable over }K\}$$ is a subfield of $L$, the field extension $E/K$ is separable and if $L\ne E$, then $\forall\beta\in L\setminus E$, $\beta$ is not separable over $E$. In that case, and if the extension $L/E$ is algebraic, we say that $L/E$ is \emph{purely inseparable}.
    \end{enumerate}
  \end{proposition}
  \begin{theorem}[Primitive element theorem]
    Let $L/K$ be a finite and separable field extension. Then, $L/K$ is simple.
  \end{theorem}
  \subsubsection{Galois extensions}
  \begin{definition}
    We say that a field extension $L/K$ is a \emph{Galois extension} (or is \emph{Galois}) if it is normal and separable.
  \end{definition}
  \begin{theorem}
    Let $L/K$ be a finite field extension. Then: $$L/K\text{ is Galois}\iff|\Gal(L/K)|=[L:K]$$
  \end{theorem}
  \begin{lemma}
    Let $L/F/K$ be a tower of fields such that $L/K$ is Galois. Then, $L/F$ is Galois.
  \end{lemma}
  \begin{proposition}
    Let $L/K$ be a Galois extension. Then, $\alpha\in L$ is primitive if and only if $\forall \sigma\in\Gal(L/K)\setminus\{\id\}$, $\sigma(\alpha)\ne\alpha$.
  \end{proposition}
  \subsection{Fundamental theorem of Galois theory}
  \begin{definition}
    Let $L/K$ be a finite field extension and $G$ be a group. We define the following sets:
    \begin{gather*}
      \mathcal{K}(L/K):=\{F\subseteq L:L/F/K\text{ is a tower of fields}\}\\
      \mathcal{S}(G):=\{H\subseteq G:H\text{ is a subgroup of }G\}
    \end{gather*}
  \end{definition}
  \begin{lemma}
    Let $H\in\mathcal{S}(\Gal(L/K))$ and $$L^H:=\{a\in L:\sigma(a)=a\ \forall\sigma\in H\}$$
    Then, $L^H$ is a field (called \emph{fixed field} of $H$) and $L^H\in\mathcal{K}(L/K)$.
  \end{lemma}
  \begin{lemma}
    Let $L/K$ be a finite field extension and $F\in \mathcal{K}(L/K)$. Then, $\Gal(L/F)$ is a subgroup of $\Gal(L/K)$.
  \end{lemma}
  \begin{definition}
    Let $L/K$ be a finite field extension. We define the following functions:
    \begin{gather*}
      \function{\mathcal{F}}{\mathcal{S}(\Gal(L/K))}{\mathcal{K}(L/K)}{H}{L^H}\\
      \function{\mathcal{G}}{\mathcal{K}(L/K)}{\mathcal{S}(\Gal(L/K))}{F}{\Gal(L/F)}
    \end{gather*}
  \end{definition}
  \begin{proposition}
    Let $L/K$ be a finite field extension. Then:
    \begin{enumerate}
      \item $\mathcal{F}(\{\id\})=L$.
      \item $\mathcal{G}(L)=\{\id\}$ and $\mathcal{G}(K)=\Gal(L/K)$.
      \item If $H_1,H_2\in\mathcal{S}(\Gal(L/K))$ are such that $H_1\subseteq H_2$, then $\mathcal{F}(H_1)\supseteq \mathcal{F}(H_2)$.
      \item If $F_1,F_2\in\mathcal{K}(L/K)$ are such that $F_1\subseteq F_2$, then $\mathcal{G}(F_1)\supseteq \mathcal{G}(F_2)$.
      \item $H\subseteq \mathcal{G}(\mathcal{F}(H))$ $\forall H\in \mathcal{S}(\Gal(L/K))$.
      \item $F\subseteq \mathcal{F}(\mathcal{G}(F))$ $\forall F\in \mathcal{K}(L/K)$.
      \item $\mathcal{F}\circ\mathcal{G}\circ\mathcal{F}=\mathcal{F}$.
      \item $\mathcal{G}\circ\mathcal{F}\circ\mathcal{G}=\mathcal{G}$.
    \end{enumerate}
  \end{proposition}
  \begin{lemma}[Artin's lemma]
    Let $L/K$ be a finite field extension and $H$ be a subgroup of $\Gal(L/K)$. Then, $H=\Gal(L/L^H)$ and $|H|=[L:L^H]$.
  \end{lemma}
  \begin{corollary}
    Let $L/K$ be a finite field extension. Then, $\mathcal{G}\circ\mathcal{F}=\id$. Thus, $\mathcal{F}$ is injective and $\mathcal{G}$ is surjective.
  \end{corollary}
  \begin{theorem}[Fundamental theorem of Galois theory]
    Let $L/K$ be a finite and Galois field extension. Then, $\mathcal{F}\circ\mathcal{G}=\id$. Thus, $\mathcal{F}$ and $\mathcal{G}$ are bijective and they are inverses of each other. Furthermore, if $F\in\mathcal{K}(L/K)$, then: $$F/K\text{ is normal}\iff\Gal(L/F)\unlhd\Gal(L/K)$$ And in that case: $$\quot{\Gal(L/K)}{\Gal(L/F)}\cong\Gal(F/K)$$
  \end{theorem}
  \begin{corollary}
    Let $L/K$ be a finite and Galois field extension and $H$ be a subgroup of $\Gal(L/K)$. Then: $$[L^H:K]=\frac{|\Gal(L/K)|}{|H|}$$
  \end{corollary}
  \begin{definition}
    Let $G$ be a group. The \emph{lattice of subgroups} of $G$ is the following graph:
    \begin{itemize}
      \item The vertices of the graph are the subgroups of $G$.
      \item Two vertices (corresponding to two subgroups $H_i$, $H_j$ of $G$) are connected by an edge if $H_i\leq H_j$, with $i\ne j$, and such that there is no $k\ne i,j$ such that $H_i\leq H_k\leq H_j$.
    \end{itemize}
  \end{definition}
  \begin{center}
    \begin{minipage}{\linewidth}
      \centering
      \includestandalone[mode=image|tex,width=0.65\linewidth]{Images/lattice}
      \captionof{figure}{Lattice of subgroups of the group $\quot{\ZZ}{4\ZZ}\times\quot{\ZZ}{2\ZZ}$}
    \end{minipage}
  \end{center}
  \begin{definition}
    Let $K$ be a field, $p(x)\in K[x]$ and $L$ be the splitting field of $p(x)$ over $K$. We denote $\Gal(p(x)/K):=\Gal(L/K)$.
  \end{definition}
  \begin{definition}
    A subgroup $H$ of $\text{S}_n$ is called \emph{transitive} if $\forall i,j\in\{1,\ldots,n\}$, $\exists \sigma\in H$ such that $\sigma(i)=j$.
  \end{definition}
  \begin{lemma}
    The transitive subgroups of $\text{S}_4$, up to isomorphism, are $\text{S}_4$, $\text{A}_4$, $\text{D}_4$, $\text{V}_4$ and $\text{C}_4$, where:
    $$\text{V}_4=\langle (1,2)(3,4),(1,3)(2,4)\rangle\quad\text{and}\quad \text{C}_4=\langle(1,2,3,4)\rangle$$
  \end{lemma}
  \begin{corollary}
    The transitive subgroups of $\text{A}_4$, up to isomorphism, are $\text{A}_4$ and $\text{V}_4$.
  \end{corollary}
  \begin{lemma}
    Let $K$ be a field, $p(x)\in K[x]$ be an irreducible and separable polynomial of degree $n$ and $L$ be its splitting field. Let $\alpha_1,\ldots,\alpha_n\in L$ be the roots of $p(x)$. Then, there exists a unique monomorphism $\iota: \Gal(p(x)/K)\hookrightarrow \text{S}_n$ such that $\sigma(\alpha_i)=\alpha_{\iota(\sigma)(i)}$.
  \end{lemma}
  \begin{lemma}
    Let $K$ be a field, $p(x)\in K[x]$ be an irreducible and separable polynomial of degree $n$ and $\iota: \Gal(p(x)/K)\hookrightarrow \text{S}_n$ be the monomorphism obtained by fixing an order of the roots of $p(x)$ (in its splitting field). Then, $\im(\iota)$ is a transitive subgroup of $\text{S}_n$.
  \end{lemma}
  \begin{definition}
    Let $K$ be a field, $p(x)\in K[x]$ and $\alpha_1,\ldots,\alpha_n$ be the roots of $p(x)$ in its splitting field. We define $\delta(p)$ as: $$\delta(p):=\prod_{1\leq i<j\leq n}(\alpha_j-\alpha_i)$$
    We define the \emph{discriminant} of $p(x)$, $\Disc(p)$, as: $$\Disc(p):=\prod_{1\leq i<j\leq n}{(\alpha_j-\alpha_i)}^2={\delta(p)}^2$$
  \end{definition}
  \begin{proposition}
    Let $K$ be a field, $p(x)\in K[x]$ and $\alpha_1,\ldots,\alpha_n$ be the roots of $p(x)$ in its splitting field. Then, $\Disc (p)\in {K[\alpha_1,\ldots,\alpha_n]}^{\text{S}_n}$.
  \end{proposition}
  \begin{lemma}
    Let $K$ be a field, $p(x)\in K[x]$ be an irreducible and separable polynomial of degree $n$, $L$ be its splitting field and $\alpha_1,\ldots,\alpha_n\in L$ be the roots of $p(x)$. Then, if we think $\Gal(L/K)$ as a subgroup of $\text{S}_n$ via the inclusion $\iota$ of above, we have that $\forall \sigma\in \text{S}_n$, $\sigma(\delta(p))=\sign(\sigma)\delta(p)$.
  \end{lemma}
  \begin{corollary}
    Let $K$ be a field, $p(x)\in K[x]$ be an irreducible and separable polynomial of degree $n$, $L$ be its splitting field and $\alpha_1,\ldots,\alpha_n\in L$ be the roots of $p(x)$. Then, $\Disc(p)\in K$.
  \end{corollary}
  \begin{corollary}
    Let $K$ be a field, $p(x)\in K[x]$ be an irreducible and separable polynomial of degree $n$, $L$ be its splitting field and $\alpha_1,\ldots,\alpha_n\in L$ be the roots of $p(x)$. Then: $$\delta(p)\in K\iff \Gal(L/K)\subseteq \text{A}_n$$
  \end{corollary}
  \begin{proposition}
    Let $f(x)=x^2+bx+c$ and $g(x)=x^3+px+q$. Then:
    \begin{itemize}
      \item $\Disc(f)=b^2-4c$
      \item $\Disc(g)=-4p^3-27q^2$
    \end{itemize}
  \end{proposition}
  \subsection{Fundamental theorem of algebra}
  \begin{definition}
    We say that a field $K$ is \emph{algebraically closed} if each polynomial in $K[x]$ splits into linear factors in $K$.
  \end{definition}
  \begin{proposition}
    Let $K$ be a field. The following statements are equivalent:
    \begin{enumerate}
      \item $K$ is algebraically closed.
      \item If $p(x)\in K[x]$ is irreducible, then $\deg(p(x))=1$.
      \item $K/K$ is the only algebraic extension of $K$.
      \item If $L/K$ is a finite field extension, then $[L:K]=1$.
    \end{enumerate}
  \end{proposition}
  \begin{lemma}
    Let $G$ be a 2-group. Then, $G$ has a normal subgroup of index 2.
  \end{lemma}
  \begin{theorem}
    Let $L/\RR$ be a finite Galois field extension. Then, either $L=\RR$ or $L=\CC$.
  \end{theorem}
  \begin{theorem}[Fundamental theorem of algebra]
    $\CC$ is algebraically closed.
  \end{theorem}
  \begin{theorem}
    Let $K$ be a field. Then, there exists an algebraic field extension $\overline{K}/K$ such that $\overline{K}$ is algebraically closed. This field $\overline{K}$ is called the \emph{algebraic closure} of $K$.
  \end{theorem}
  \subsection{Galois theory of solvable equations}
  \subsubsection{Solvable groups}
  \begin{definition}
    Let $G$ be a finite group. We say $G$ is \emph{solvable} if there is a chain of subgroups $H_i$ of $G$ satisfying: $$\{e\}=H_0\unlhd H_1\unlhd\cdots\unlhd H_n=G$$ and such that $H_i/H_{i-1}$ are abelian for all $i=1,\ldots,n$.
  \end{definition}
  \begin{definition}
    Let $G$ be a group. We say that $G$ is \emph{simple} if its only normal subgroups are the trivial group and the group itself.
  \end{definition}
  \begin{proposition}
    Let $G$ be a solvable group and $H$ be a subgroup of $G$. Then, $H$ is solvable.
  \end{proposition}
  \begin{proposition}
    Let $G$ be a finite group and $H$ be a subgroup of $G$ such that $H\unlhd G$. Then, $G$ is solvable if and only if $H$ and $\quot{G}{H}$ are solvable.
  \end{proposition}
  \begin{proposition}
    Let $G$ be a solvable group. Then, there exists a chain of subgroups $$\{e\}=H_0\unlhd H_1\unlhd\cdots\unlhd H_n=G$$ such that $H_i/H_{i-1}$ are cyclic for all $i=1,\ldots,n$.
  \end{proposition}
  \begin{theorem}
    $\text{A}_n$ is simple for all $n\geq 5$.
  \end{theorem}
  \begin{theorem}
    $\text{S}_n$ and $\text{A}_n$ aren't solvable for all $n\geq 5$.
  \end{theorem}
  \subsubsection{Radical, cyclotomic and cyclic extensions}
  \begin{definition}
    We say that a finite field extension $L/K$ is \emph{radical} if there exist $n\in\NN$ and $\alpha\in L$ such that $L=K(\alpha)$ and $\alpha^n\in K$. Moreover, if $\alpha^n=1$ we say the the extension $L/K$ is \emph{cyclotomic}.
  \end{definition}
  \begin{definition}
    We say that a tower of fields $K_n/K_{n-1}/\cdots/K_0$ is a \emph{radical tower} if $K_i/K_{i-1}$ is a radical extension $\forall i=1,\ldots,n$.
  \end{definition}
  \begin{definition}
    We say that a field extension $L/K$ is \emph{solvable by radicals} if there exists a radical tower of fields $K_n/K_{n-1}/\cdots/K_1/K$ such that $L\subseteq K_n$.
  \end{definition}
  \begin{definition}
    Let $K$ be a field and $p(x)\in K[x]$. We say that $p(x)$ is \emph{solvable by radicals} if the splitting field of $p(x)$ over $K$ is solvable by radicals.
  \end{definition}
  \begin{definition}
    Let $n\in \NN$ and $K$ be a field. A \emph{$n$-th root of unity} is a number $z\in K$ such that $z^n=1$. A \emph{$n$-th primitive root of unity} is a $n$-th root of unity $z\in K$ such that $z^m\ne 1$ for all $m=1,\ldots,n-1$.
  \end{definition}
  \begin{proposition}
    Let $K$ be a field such that $\ch K=0$, $n\geq 2$ and $L$ be the splitting field of $x^n-1$ over $K$. Denote by $\xi_n$ a $n$-th primitive root of unity. Then, $L=K(\xi_n)$ and $\Gal(L/K)\cong H$ for some $H\in\mathcal{S}\left({(\quot{\ZZ}{n\ZZ})}^*\right)$. Furthermore if $K=\QQ$, we have that $\Gal(L/K)\cong {(\quot{\ZZ}{n\ZZ})}^*$.
  \end{proposition}
  \begin{proposition}
    Let $K$ be a field such that $\ch K=0$ and $x^n-1$ splits into linear factors in $K$. Let $K(\alpha)/K$ be a radical extension. Then, $K(\alpha)/K$ is Galois and $\Gal(K(\alpha)/K)\cong\quot{\ZZ}{d\ZZ}$, for some $d$ such that $d\mid n$. Furthermore, $\alpha^d\in K$ and $\irr(\alpha,K)=x^d-\alpha^d$.
  \end{proposition}
  \begin{definition}
    We say that a Galois extension $L/K$ is \emph{abelian} if $\Gal(L/K)$ is abelian. In particular, we say that $L/K$ is \emph{cyclic} if $\Gal(L/K)$ is cyclic.
  \end{definition}
  \begin{lemma}[Dedekind's lemma]
    Let $L$ and $F$ be fields and $f_1,\ldots,f_n:L\rightarrow F$ be distinct field morphisms. Then, if $\lambda_1,\ldots,\lambda_n\in F$ are such that $\lambda_1f_1+\cdots+\lambda_nf_n=0$, then $\lambda_1=\cdots=\lambda_n=0$. In that case, we say that $f_1,\ldots,f_n$ are \emph{$F$-linearly independent}.
  \end{lemma}
  \begin{theorem}
    Let $K$ be a field such that $\ch K=0$ and $x^n-1$ splits into linear factors in $K$. Then, $L/K$ is cyclic of degree $n$ if and only if $L/K$ is radical of degree $n$.
  \end{theorem}
  \begin{lemma}
    Let $F/K$ be a field extension, $p(x)\in K[x]$ be a separable polynomial, $L/K$ be a splitting field of $p(x)$ over $K$ and $E/F$ be a splitting field of $p(x)$ over $F$. Then, $\Gal(E/F)\cong H$ for some $H\in\mathcal{S}(\Gal(L/K))$
  \end{lemma}
  \begin{theorem}
    Let $K$ be a field such that $\ch K=0$, and $p(x)\in K[x]$. Then: $$p(x)\text{ is solvable by radicals}\iff\Gal(p(x)/K)\text{ is solvable}$$
  \end{theorem}
  \begin{lemma}
    Let $K$ be a field, $n\in\NN$, $a_1,\ldots,a_n$ be unknowns and $s_1,\ldots,s_n$ be the elementary symmetric polynomials in the variables $a_1,\ldots,a_n$. Then: $$\Gal(K(a_1,\ldots,a_n)/K(s_1,\ldots,s_n))\cong \text{S}_n$$
  \end{lemma}
  \begin{corollary}
    Let $K$ be a field, $a_1,\ldots,a_n$ be unknowns, $\delta:=\prod_{1\leq i<j\leq n}(a_j-a_i)$ and $s_1,\ldots,s_n$ be the elementary symmetric polynomials in the variables $a_1,\ldots,a_n$.. Then: $${K(a_1,\ldots,a_n)}^{\text{A}_n}=K(s_1,\ldots,s_n)(\delta)$$
  \end{corollary}
  \begin{theorem}[Abel-Ruffini theorem]
    There is no solution in radicals to polynomial equations of degree five or higher with arbitrary coefficients.
  \end{theorem}
  \begin{proposition}
    Let $K$ be a field such that $\ch K=0$, and $p(x)\in K[x]$ be an irreducible polynomial of degree 5. Then:
    $$p(x)\text{ is solvable by radicals}\iff\Gal(p(x)/K)\ncong\text{S}_5,\text{A}_5$$
  \end{proposition}
  \begin{theorem}[Nart-Vila theorem]
    For all $n\geq 2$, $\Gal(x^n-x-1/\QQ)\cong \text{S}_n$.
  \end{theorem}
  \begin{corollary}
    Let $G$ be a finite group. Then, there exists finite field extensions $K/\QQ$ and $L/K$ such that $\Gal(L/K)\cong G$.
  \end{corollary}
  \subsubsection{Biquadratic polynomials}
  \begin{theorem}
    Let $K$ be a field such that $\ch K=2$, $p(x)=x^4+ax^2+b\in K[x]$ be an irreducible polynomial over $K$ and $d:=a^2-4b\in K$. Then:
    \begin{itemize}
      \item If $\sqrt{b}\in K\implies\Gal(p(x)/K)\cong\quot{\ZZ}{2\ZZ}\times\quot{\ZZ}{2\ZZ}$
      \item If $\sqrt{b}\in \sqrt{d}K\implies\Gal(p(x)/K)\cong\quot{\ZZ}{4\ZZ}$
      \item If $\sqrt{b}\notin K\cup\sqrt{d}K\implies\Gal(p(x)/K)\cong\text{D}_4$
    \end{itemize}
  \end{theorem}
\end{multicols}
\end{document}