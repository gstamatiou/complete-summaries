\documentclass[../../../main.tex]{subfiles}


\begin{document}
\renewcommand{\col}{\geo}
\begin{multicols}{2}[\section{Differential geometry}]
  \subsection{Differentiable curves}
  \subsubsection{Inner product of \texorpdfstring{$\RR^n$}{Rn}}
  \begin{proposition}
    Let $\vf{u},\vf{v}\in\RR^n$ be vectors and $\langle\vf{u},\vf{v}\rangle$ be the usual inner product between $\vf{u}$ and $\vf{v}$ in $\RR^n$. Then:
    \begin{itemize}
      \item Cauchy-Schwarz inequality: $$|\langle\vf{u},\vf{v}\rangle|\leq\|\vf{u}\|\|\vf{v}\|$$
      \item Triangular inequality: $$\|\vf{u}+\vf{v}\|\leq\|\vf{u}\|+\|\vf{v}\|$$
      \item Polarization identity: $$\langle\vf{u},\vf{v}\rangle=\frac{1}{2} \left(\|\vf{u}+\vf{v}\|^2-\|\vf{u}\|^2 - \|\vf{v}\|^2\right)$$
    \end{itemize}
  \end{proposition}
  \begin{definition}
    Let $\vf{u},\vf{v}\in\RR^n$ be vectors. We define the angle between $\vf{u}$ and $\vf{v}$ as the unique value $\theta\in[0,\pi]$ such that: $$\cos\theta=\frac{\langle\vf{u},\vf{v}\rangle}{\|\vf{u}\|\|\vf{v}\|}$$
  \end{definition}
  \subsubsection{Parametrized curves}\label{DG_curves}
  \begin{definition}
    Let $U\subseteq\RR^n$ be an open set and $\vf{f}:U\rightarrow\RR^n$ be a differentiable function. We say that $\vf{f}$ is a \emph{local diffeomorphism} if $\forall p\in U$, there exists a neighbourhood $V\subseteq U$ of $p$ such that $\vf{f}|_V:V\rightarrow \vf{f}(V)$ is a diffeomorphism.
  \end{definition}
  \begin{proposition}
    Let $I\subseteq\RR$ be an open interval and $f:I\rightarrow\RR$ be a differentiable function. If $f'(x)\ne 0$ $\forall x\in I$, then $f(I)$ is an open set and $f$ is a diffeomorphism.
  \end{proposition}
  \begin{proposition}
    Let $I\subseteq\RR$ be an open interval and $\vf{\alpha},\vf{\beta}:I\rightarrow\RR^n$ be differentiable functions. Then:
    \begin{enumerate}
      \item ${\langle\vf{\alpha}(t),\vf{\beta}(t)\rangle}'=\langle\vf{\alpha}'(t),\vf{\beta}(t)\rangle+\langle\vf{\alpha}(t),\vf{\beta}'(t)\rangle$
      \item If $t\mapsto\|\vf{\alpha}(t)\|$ is a constant function, then $\vf{\alpha}\perp\vf{\alpha}'$.
    \end{enumerate}
  \end{proposition}
  \begin{definition}
    Let $I\subseteq\RR$ be an open interval and $C\subset\RR^n$ be a curve. A \emph{parametrization} of $C$ of class $\mathcal{C}^k$ is a function $\vf{\alpha}:I\rightarrow\RR^n$ of class $\mathcal{C}^k$ such that $\vf\alpha(I)=C$. The image of $\vf\alpha$, $C$, is called the \emph{trace} of $\vf\alpha$ and it is sometimes denoted by $\vf\alpha^*:=\im(\vf\alpha)$\footnote{Sometimes $\vf\alpha$ is referred to the curve as well as to the parametrization of it.}.
  \end{definition}
  \begin{definition}
    Let $I\subseteq\RR$ be an open interval, $C\subset\RR^n$ be a curve and $\vf{\alpha}:I\rightarrow\RR^n$ be a parametrization of $C$ of class $\mathcal{C}^1$. We define the \emph{tangent vector} of $\vf{\alpha}$ at $t_0\in\RR$ as $\vf{\alpha}'(t_0)$. We say that $\vf{\alpha}$ is \emph{regular} if $\vf{\alpha}'(t)\ne 0$ $\forall t\in I$. In that last case, we define the \emph{tangent line} of $\vf{\alpha}$ at $\vf{\alpha}(t_0)$ as the following parametrized line in $\RR^n$: $$s\longmapsto \vf{\alpha}(t_0)+s\vf{\alpha}'(t_0)$$
  \end{definition}
  \begin{definition}
    Let $C\subset\RR^n$ be a curve. We say that $C$ is a \emph{plane curve} if it is contained in a plane of $\RR^n$.
  \end{definition}
  \begin{definition}\label{DG_reparam}
    Let $I,J\subseteq\RR$ be open intervals, $C\subset\RR^n$ be a curve, $\vf{\alpha}:I\rightarrow\RR^n$ be a regular parametrization of $C$ of class $\mathcal{C}^1$ and $h:J\rightarrow I$ be a diffeomorphism. Then, $\vf{\beta}:=\vf{\alpha}\circ h:J\rightarrow\RR^n$ is a regular parametrization of $C$  satisfying: $$\vf{\beta}'(s)=\vf{\alpha}'(h(s))h'(s)\quad\forall s\in J$$ It is said that $\vf{\beta}$ is a \emph{reparametrization} of $\vf{\alpha}$ and $h$ is a \emph{change of parameter}. Moreover, the reparametrization is \emph{positive} if $h'(s)>0$ $\forall s\in J$ and it is \emph{negative} if $h'(s)<0$ $\forall s\in J$.
  \end{definition}
  \subsubsection{Length of curves}
  \begin{definition}
    Let $I\subseteq\RR$ be an open interval, $C\subset\RR^n$ be a curve, $\vf{\alpha}:I\rightarrow\RR^n$ be a continuous parametrization of $C$, $[a,b]\subset I$, $\mathfrak{P}([a,b])$ be the set of all partitions of $[a,b]$ and $\mathcal{P}=\{t_0,\ldots,t_n\}\in\mathfrak{P}$. We define the \emph{length of the polygonal} with vertices at $\vf{\alpha}(t_i)$, $i=1,\ldots,n$ as: $$L_{a,b}(\vf{\alpha},\mathcal{P})=\sum_{i=1}^n\|\vf{\alpha}(t_i)-\vf{\alpha}(t_{i-1})\|$$ We define $L_{a,b}(\vf{\alpha})$ as:
    $$L_{a,b}(\vf{\alpha}):=\sup\{L_{a,b}(\vf{\alpha},\mathcal{P}):\mathcal{P}\in\mathfrak{P}([a,b])\}\in\RR_{\geq 0}\cup\{+\infty\}$$ If $L_{a,b}(\vf{\alpha})<+\infty$, we say that $C$ is \emph{rectifiable} and that $L_{a,b}(\vf{\alpha})$ is its \emph{length} in $[a,b]$.
  \end{definition}
  \begin{proposition}
    Let $I\subseteq\RR$ be an open interval, $[a,b]\subset I$, $C\subset\RR^n$ be a curve and $\vf{\alpha}:I\rightarrow\RR^n$ be a parametrization $C$ of class $\mathcal{C}^1$. Then, $C$ is rectifiable and: $$L_{a,b}(\vf{\alpha})=\int_a^b\|\vf{\alpha}'(t)\|\dd{t}$$
  \end{proposition}
  \begin{proposition}
    Let $I,J\subseteq\RR$ be open intervals, $[a,b]\subset I$, $C\subset\RR^n$ be a curve, $\vf{\alpha}:I\rightarrow\RR^n$ be a parametrization $C$ of class $\mathcal{C}^1$, $h:J\rightarrow I$ be a diffeomorphism, $\vf{\beta}=\vf{\alpha}\circ h$ be a reparametrization of $\vf\alpha$. Suppose $[c,d]=h^{-1}([a,b])$. Then: $$\int_c^d\|\vf{\beta}'(u)\|\dd{u}=\int_a^b\|\vf{\alpha}'(t)\|\dd{t}$$ That is, the length of a curve does not depend on its parametrization.
  \end{proposition}
  \begin{definition}
    Let $I\subseteq\RR$ be an open interval, $C\subset\RR^n$ be a curve, $\vf{\alpha}:I\rightarrow\RR^n$ be a parametrization $C$ of class $\mathcal{C}^1$ and $t_0\in I$. We define the \emph{arc-length function} of $\vf{\alpha}$ with origin ${t_0}$, the function $s_{t_0}:I\rightarrow\RR$ defined as: $$s_{t_0}(t)=\int_{t_0}^t\|\vf{\alpha}'(u)\|\dd{u}$$
  \end{definition}
  \begin{definition}
    Let $I\subseteq\RR$ be an open interval, $C\subset\RR^n$ be a curve and $\vf{\alpha}:I\rightarrow\RR^n$ be a parametrization $C$ of class $\mathcal{C}^1$. We say that $\vf{\alpha}$ is a \emph{unit-speed parametrization} (or that it is parametrized by \emph{arc-length parameter}) if $\|\vf{\alpha}'(t)\|=1$ $\forall t\in I$.
  \end{definition}
  \begin{proposition}
    Let $I\subseteq\RR$ be an open interval, $C\subset\RR^n$ be a curve and $\vf{\alpha}:I\rightarrow\RR^n$ be a parametrization $C$ of class $\mathcal{C}^1$ and ${t_0}\in I$. Then:
    \begin{enumerate}
      \item $s_{t_0}$ is of class $\mathcal{C}^1$ and $\dv{s_{t_0}}{t}(t_0)=\|\vf{\alpha}'(t_0)\|\geq 0$.
      \item If $\vf{\alpha}$ is regular, then $J:=s_{t_0}(I)\subseteq\RR$ is an open interval and $s_{t_0}:I\rightarrow J$ is a diffeomorphism.
      \item If $\vf{\alpha}$ is regular, then $\vf{\beta}(s_{t_0}):=\vf{\alpha}(t(s_{t_0}))$\footnote{Here, $t(s_{t_0})$ represent the inverse function of $s_{t_0}(t)$.} is an arc-length reparametrization of $\vf{\alpha}$.
    \end{enumerate}
  \end{proposition}
  \begin{proposition}
    Let $I,J\subseteq\RR$ be open intervals, $C\subset\RR^n$ be a curve, $\vf{\alpha}:I\rightarrow\RR^n$ be a regular parametrization of $C$ of class $\mathcal{C}^1$, $h:J\rightarrow I$ be a diffeomorphism and $\vf{\beta}=\vf{\alpha}\circ h$ be a reparametrization of $\vf{\alpha}$. If $\vf{\alpha}$ and $\vf{\beta}$ are arc-length parametrizations, then: $$\vf{\beta}(u)=\vf{\alpha}(\pm u+u_0)$$ for some $u_0\in\RR$.
  \end{proposition}
  \begin{proposition}
    All regular parametrization of curves of class $\mathcal{C}^1$ can be arc-length parametrized.
  \end{proposition}
  \subsubsection{Orientability and cross product}
  \begin{definition}
    Let $V$ be a vector space and $\mathcal{B}_1$ and $\mathcal{B}_2$ be two bases of $V$. We say that $\mathcal{B}_1\sim\mathcal{B}_2$ if $\det\left([\id]_{\mathcal{B}_1,\mathcal{B}_2}\right)>0$. This relation is an equivalence relation on the set of all bases of $V$ which has exactly two connected components.
  \end{definition}
  \begin{definition}
    Let $V$ be a vector space and $\mathcal{B}_1$ and $\mathcal{B}_2$ be two bases of $V$. We say that $\mathcal{B}_1\sim\mathcal{B}_2$ have the \emph{same orientation} if $\det\left([\id]_{\mathcal{B}_1,\mathcal{B}_2}\right)>0$. Otherwise, we say that they have \emph{opposite orientations}. Note that the property of having the same orientation defines an equivalence relation on the set of all bases for $V$.
  \end{definition}
  \begin{definition}
    An \emph{orientation} on a vector space is the choice of one of the two equivalence classes under $\sim$. A vector space with an orientation selected is called an \emph{oriented vector space}, while one not having an orientation selected, is called an \emph{unoriented vector space}. A basis of an oriented vector space which has the orientation chosen is called \emph{positive basis}, while one with the other orientation is called \emph{negative basis}.
  \end{definition}
  \begin{definition}
    Let $V$ be an oriented vector space, $\mathcal{B}$ be a basis of $V$ and $f:V\rightarrow V$ be a linear isomorphism. We say that $f$ is \emph{orientation-preserving} (or \emph{positively oriented}) if $\det \left([f]_{\mathcal{B}}\right)>0$. Analogously, if $\det \left([f]_{\mathcal{B}}\right)<0$ we say that $f$ is \emph{negatively-oriented}.
  \end{definition}
  \begin{definition}
    Let $(\vf{v}_1,\ldots,\vf{v}_n)$ be a basis of $\RR^n$. Suppose for each $i\in\{1,\ldots,n\}$ we have $$\vf{v}_i=\sum_{j=1}^n\lambda_{ij}\vf{e}_1$$ where $\lambda_{ij}\in\RR$ and $(\vf{e}_1,\ldots,\vf{e}_n)$ is the standard basis of $\RR^n$. We define the \emph{determinant} of $(\vf{v}_1,\ldots,\vf{v}_n)$ as: $$\det(\vf{v}_1,\ldots,\vf{v}_n):=
      \begin{vmatrix}
        \lambda_{11} & \cdots & \lambda_{1n} \\
        \vdots       & \ddots & \vdots       \\
        \lambda_{n1} & \cdots & \lambda_{nn} \\
      \end{vmatrix}\footnote{From now on, if we do not explicitly fix a basis it will mean that the standard basis of $\RR^n$ is the chosen one.}$$
  \end{definition}
  \begin{proposition}
    Let $\mathcal{B}=(\vf{v}_1,\ldots,\vf{v}_n)$ be a basis of $\RR^n$ and $\vf{A}\in\mathcal{M}_n(\RR)$. Then:
    $$\det(\vf{A}\vf{v}_1,\ldots,\vf{A}\vf{v}_n)=\det\vf{A}\det(\vf{v}_1,\ldots,\vf{v}_n)$$
  \end{proposition}
  \begin{proposition}
    Let $\vf{v}_1,\ldots,\vf{v}_n$ be vectors of $\RR^n$ and $P$ be the parallelepiped they generate. Then:
    $$\vol P=\abs{\det(\vf{v}_1,\ldots,\vf{v}_n)}$$
  \end{proposition}
  \begin{definition}
    Let $\vf{u}$, $\vf{v}$ be vectors of $\RR^3$. We define the \emph{cross product} of $\vf{u}$ and $\vf{v}$, denoted by $\vf{u}\times\vf{v}$\footnote{Another commonly used notation for the cross product is $\vf{u}\wedge\vf{v}$.}, as the unique vector $\vf{w}$ satisfying: $$\langle\vf{u}\times\vf{v},\vf{w}\rangle=\det(\vf{u},\vf{v},\vf{w})$$
  \end{definition}
  \begin{proposition}
    Let $\vf{u}$, $\vf{v}$ be vectors of $\RR^3$ such that $\vf{u}=\sum_{i=1}^3u_i\vf{e}_i$ and $\vf{v}=\sum_{i=1}^3v_i\vf{e}_i$. Then: $$\vf{u}\times\vf{v}=
      \begin{vmatrix}
        \vf{e}_1 & \vf{e}_2 & \vf{e}_3 \\
        u_1      & u_2      & u_3      \\
        v_1      & v_2      & v_3      \\
      \end{vmatrix}$$
  \end{proposition}
  \begin{proposition}
    Let $\vf{u}$, $\vf{v}$, $\vf{w}$ be vectors of $\RR^3$. Then:
    \begin{enumerate}
      \item $\vf{u}\times\vf{v}=-\vf{v}\times\vf{u}$
      \item $\vf{u}\times\vf{v}=0\iff\vf{u}=\lambda\vf{v}$, for some $\lambda\in\RR$.
      \item $\vf{u}\times\vf{v}\in{\langle\vf{u},\vf{v}\rangle}^{\perp}$
      \item If $\vf{u}$ and $\vf{v}$ are linearly independent, $(\vf{u},\vf{v},\vf{u}\times\vf{v})$ is a positive basis of $\RR^n$.
      \item If $\vf{x}$, $\vf{y}$ are vectors of $\RR^3$, then: $$\langle\vf{u}\times\vf{v},\vf{x}\times\vf{y}\rangle=
              \begin{vmatrix}
                \vf{u}\times\vf{x} & \vf{v}\times\vf{x} \\
                \vf{u}\times\vf{y} & \vf{v}\times\vf{y} \\
              \end{vmatrix}$$
      \item Let $\theta\in[0,\pi]$ be the angle between $\vf{u}$ and $\vf{v}$. Then: $$\|\vf{u}\times\vf{v}\|=\|\vf{u}\|\|\vf{v}\|\sin\theta$$
      \item $(\vf{u}\times\vf{v})\times\vf{w}=\langle\vf{u},\vf{w}\rangle\vf{v}-\langle\vf{v},\vf{w}\rangle\vf{u}$
      \item \emph{Jacobi identity}: $$(\vf{u}\times\vf{v})\times\vf{w}+(\vf{v}\times\vf{w})\times\vf{u}+(\vf{w}\times\vf{u})\times\vf{v}=\vf{0}$$
    \end{enumerate}
  \end{proposition}
  \begin{proposition}
    Let $\vf{\alpha},\vf{\beta}:I\rightarrow\RR^3$ be parametrized curves of class $\mathcal{C}^\infty$. Then:
    $${\left(\vf{\alpha}(t)\times\vf{\beta}(t)\right)}'=\vf{\alpha}'(t)\times\vf{\beta}(t)+\vf{\alpha}(t)\times\vf{\beta}'(t)$$
  \end{proposition}
  \subsubsection{Frenet-Serret formulas}
  \begin{definition}
    Let $I\subseteq\RR$ be an open interval, $C\subset\RR^3$ be a curve and $\vf{\alpha}:I\rightarrow\RR^3$ be an arc-length parametrization of $C$ of class $\mathcal{C}^2$. We define the unit tangent vector of $\vf{\alpha}$ at $s_0\in I$ as: $$\T\alpha(s_0):=\vf{\alpha}'(s_0)$$ Note that $\|\T\alpha\|=1$ and $\T\alpha\perp\T\alpha'$.
  \end{definition}
  \begin{definition}
    Let $I\subseteq\RR$ be an open interval, $C\subset\RR^3$ be a curve and $\vf{\alpha}:I\rightarrow\RR^3$ be an arc-length parametrization of $C$ of class $\mathcal{C}^2$. We define the \emph{curvature} of $\vf{\alpha}$ at $s_0\in I$ as: $$k_{\vf\alpha}(s_0):=\|\vf{\alpha}''(s_0)\|=\|\T\alpha'(s_0)\|$$
  \end{definition}
  \begin{definition}
    Let $I\subseteq\RR$ be an open interval, $C\subset\RR^3$ be a curve and $\vf{\alpha}:I\rightarrow\RR^3$ be an arc-length parametrization of $C$ of class $\mathcal{C}^2$, $s_0\in I$ and suppose that $k_{\vf\alpha}(s_0)\ne 0$. We define the \emph{unit normal vector} of $\vf{\alpha}$ at $s_0$ as: $$\N\alpha(s_0):=\frac{\T\alpha'(s_0)}{k_{\vf\alpha}(s_0)}=\frac{\vf{\alpha}''(s_0)}{\|\vf{\alpha}''(s_0)\|}$$
    Note that $\|\N\alpha\|=1$, $\N\alpha\perp\T\alpha$ and $\T\alpha'(s)=k_{\vf\alpha}(s)\N\alpha(s)$ $\forall s\in I$.
  \end{definition}
  \begin{definition}
    Let $I\subseteq\RR$ be an open interval, $C\subset\RR^3$ be a curve and $\vf{\alpha}:I\rightarrow\RR^3$ be a regular arc-length parametrization of $C$ of class $\mathcal{C}^2$ such that $\vf{\alpha}''(s)\ne 0$ $\forall s\in I$. We define the \emph{binormal vector} of $\vf{\alpha}$ at $s_0\in I$ as:
    $$\B\alpha(s_0)=\T\alpha(s_0)\times\N\alpha(s_0)$$
    Then, the triplet $(\T\alpha(s_0),\ \N\alpha(s_0),\ \B\alpha(s_0))$ is an orthonormal positive basis\footnote{Observe that the binormal vector (together with the tangent and normal vectors) is the unique vector that satisfies this property.}, and the affine frame $\{\vf{\alpha}(s_0); (\T\alpha(s_0),\N\alpha(s_0),\B\alpha(s_0))\}$ is called \emph{Frenet-Serret frame} (or \emph{TNB frame}).
  \end{definition}
  \begin{proposition}
    Let $I\subseteq\RR$ be an open interval, $C\subset\RR^3$ be a curve and $\vf{\alpha}:I\rightarrow\RR^3$ be a regular arc-length parametrization of $C$ of class $\mathcal{C}^3$ such that $\vf{\alpha}''(s)\ne 0$ $\forall s\in I$. Then: $$\B\alpha'(s)=\ta\alpha(s)\N\alpha(s)\quad\forall s\in I$$ This coefficient $\ta\alpha(s)$ is called \emph{torsion} of $\vf{\alpha}$ at $s\in I$.
  \end{proposition}
  \begin{proposition}
    Let $I\subseteq\RR$ be an open interval, $C\subset\RR^3$ be a curve and $\vf{\alpha}:I\rightarrow\RR^3$ be a regular arc-length parametrization of $C$ of class $\mathcal{C}^3$ such that $\vf{\alpha}''(s)\ne 0$ $\forall s\in I$. The following statements are equivalent:
    \begin{enumerate}
      \item $\vf{\alpha}$ is a plane curve.
      \item $\B\alpha=\const$
      \item $\ta\alpha=0$.
    \end{enumerate}
  \end{proposition}
  \begin{theorem}[Frenet-Serret formulas]
    Let $I\subseteq\RR$ be an open interval, $C\subset\RR^3$ be a curve and $\vf{\alpha}:I\rightarrow\RR^3$ be a regular arc-length parametrization of $C$ of class $\mathcal{C}^3$ such that $\vf{\alpha}''(s)\ne 0$ $\forall s\in I$. Then\footnote{Note that an inversion of the orientation of $\vf\alpha$ would change the sign of $\T\alpha$ and $\B\alpha$ but it would preserve the sign of $\N\alpha$, $k_{\alpha}$ and $\ta\alpha$.}:
    $$
      \begin{pmatrix}
        \T\alpha \\
        \N\alpha \\
        \B\alpha \\
      \end{pmatrix}'=
      \begin{pmatrix}
        0              & k_{\vf\alpha} & 0          \\
        -k_{\vf\alpha} & 0             & -\ta\alpha \\
        0              & \ta\alpha     & 0
      \end{pmatrix}
      \begin{pmatrix}
        \T\alpha \\
        \N\alpha \\
        \B\alpha \\
      \end{pmatrix}
    $$
  \end{theorem}
  \begin{definition}
    Let $I\subseteq\RR$ be an open interval, $C\subset\RR^3$ be a curve and $\vf{\alpha}:I\rightarrow\RR^3$ be a regular arc-length parametrization of $C$ of class $\mathcal{C}^3$ such that $\vf{\alpha}''(s)\ne 0$ $\forall s\in I$. and $s_0\in I$. We define the following planes of $\RR^3$:
    \begin{itemize}
      \item \emph{Osculating plane}: plane generated by $\T\alpha(s_0)$ and $\N\alpha(s_0)$ that contains $\vf{\alpha}(s_0)$.
      \item \emph{Normal plane}: plane generated by $\N\alpha(s_0)$ and $\B\alpha(s_0)$ that contains $\vf{\alpha}(s_0)$.
      \item \emph{Rectifying plane}: plane generated by $\T\alpha(s_0)$ and $\B\alpha(s_0)$ that contains $\vf{\alpha}(s_0)$.
    \end{itemize}
  \end{definition}
  \begin{proposition}
    Let $I\subseteq\RR$ be an open interval, $C\subset\RR^3$ be a curve, $\vf{\alpha}:I\rightarrow\RR^3$ be a regular parametrization of $C$ of class $\mathcal{C}^3$ and $h(t)=s(t)$ be the arc-length parameter. Let $\vf{\beta}=(\vf{\alpha}\circ h^{-1})(s)$, which is an arc-length parametrization of $C$. Then, assuming $\vf{\beta}''\ne 0$, we can define the TNB frame of $\vf{\alpha}$ as: $$\T\alpha:=\T\beta\circ h\qquad\N\alpha:=\N\beta\circ h\qquad\B\alpha:=\B\beta\circ h$$
    And the curvature and torsion of $\vf\alpha$ as: $$k_{\vf\alpha}:=k_{\vf\beta}\circ h\qquad\ta\alpha:=\ta\beta\circ h$$
  \end{proposition}
  \begin{lemma}
    Let $I\subseteq\RR$ be an open interval, $C\subset\RR^3$ be a curve, $\vf{\alpha}:I\rightarrow\RR^3$ be a regular parametrization of $C$ of class $\mathcal{C}^3$ and $h(t)=s(t)$ be the arc-length parameter. Let $\vf{\beta}=(\vf{\alpha}\circ h^{-1})(s)$. Then, $\vf{\beta}''=0\iff\vf\alpha'\times\vf\alpha''=0$.
  \end{lemma}
  \begin{lemma}
    Let $I\subseteq\RR$ be an open interval, $C\subset\RR^3$ be a curve, $\vf{\alpha}:I\rightarrow\RR^3$ be a regular parametrization of $C$ of class $\mathcal{C}^3$ such that $\vf{\alpha}'\times \vf{\alpha}''\ne 0$ and $v(t):=\|\vf{\alpha}'(t)\|$. Then:
    \begin{itemize}
      \item $\vf{\alpha}'=v\T\alpha$
      \item $\vf{\alpha}''=v'\T\alpha+k_{\vf\alpha} v^2\N\alpha$
    \end{itemize}
  \end{lemma}
  \begin{theorem}[General Frenet-Serret formulas]
    Let $I\subseteq\RR$ be an open interval, $C\subset\RR^3$ be a curve, $\vf{\alpha}:I\rightarrow\RR^3$ be a regular parametrization of $C$ of class $\mathcal{C}^3$ such that $\vf{\alpha}'\times \vf{\alpha}''\ne 0$ and $v(t):=\|\vf{\alpha}'(t)\|$. Then:
    $$
      \begin{pmatrix}
        \T\alpha \\
        \N\alpha \\
        \B\alpha \\
      \end{pmatrix}'=
      \begin{pmatrix}
        0                & k_{\vf\alpha} v & 0            \\
        -k_{\vf\alpha} v & 0               & -\ta\alpha v \\
        0                & \ta\alpha v     & 0
      \end{pmatrix}
      \begin{pmatrix}
        \T\alpha \\
        \N\alpha \\
        \B\alpha \\
      \end{pmatrix}
    $$
  \end{theorem}
  \begin{corollary}
    Let $I\subseteq\RR$ be an open interval, $C\subset\RR^3$ be a curve and $\vf{\alpha}:I\rightarrow\RR^3$ be a regular parametrization of $C$ of class $\mathcal{C}^3$ such that $\vf{\alpha}'\times \vf{\alpha}''\ne 0$. Then:
    $$\T\alpha=\frac{\vf{\alpha}'}{\|\vf{\alpha}'\|}\qquad\N\alpha=\B\alpha\times\T\alpha\qquad\B\alpha=\frac{\vf{\alpha}'\times\vf{\alpha}''}{\|\vf{\alpha}'\times\vf{\alpha}''\|}$$
    Moreover: $$k_{\vf\alpha}=\frac{\|\vf{\alpha}'\times\vf{\alpha}''\|}{{\|\vf{\alpha}'\|}^3}\qquad\ta\alpha=-\frac{\langle\vf{\alpha}'\times\vf{\alpha}'',\vf{\alpha}'''\rangle}{{\|\vf{\alpha}'\times\vf{\alpha}''\|}^2}$$
  \end{corollary}
  \subsubsection{Contact between curves and surfaces}
  \begin{definition}
    Let $I\subseteq\RR$ be an open interval, $\vf{\alpha},\vf{\beta}:I\rightarrow\RR^n$ be arc-length parametrizations of two curves of class $\mathcal{C}^\infty$ and $s_0\in I$. We say that $\vf{\alpha}$ and $\vf{\beta}$ have \emph{contact} of order $\geq r$ at $s_0$ if $$\lim_{s\to s_0}\frac{\vf{\alpha}(s)-\vf{\beta}(s)}{{(s-s_0)}^r}=\vf{0}$$
    We say that $\vf{\alpha}$ and $\vf{\beta}$ have \emph{contact} of order $r$ at $s_0$ if they have contact of order $\geq r$ but not contact of order $\geq r+1$.
  \end{definition}
  \begin{proposition}
    Let $I\subseteq\RR$ be an open interval, $\vf{\alpha},\vf{\beta}:I\rightarrow\RR^n$ be arc-length parametrizations of two curves of class $\mathcal{C}^\infty$ and $s_0\in I$. Then, $\vf{\alpha}$ and $\vf{\beta}$ have contact of order $\geq r$ at $s_0$ if and only if: $$\vf{\alpha}^{(k)}(s_0)=\vf{\beta}^{(k)}(s_0)\quad\text{for }k=0,\ldots,r$$
  \end{proposition}
  \begin{proposition}
    Let $I\subseteq\RR$ be an open interval, $C\subset\RR^3$ be a curve, $\vf{\alpha}:I\rightarrow\RR^3$ be an arc-length parametrization of $C$ of class $\mathcal{C}^2$ and $s_0\in I$. Then, the \emph{tangent line} at $\vf{\alpha}(s_0)$ is the unique line that has contact of order $\geq 1$ with $\alpha$ at this point. An arc-length parametrization of the tangent line is: $$u\longmapsto \vf\alpha(s_0)+u\T\alpha(s_0)$$
  \end{proposition}
  \begin{proposition}
    Let $I\subseteq\RR$ be an open interval, $C\subset\RR^3$ be a curve and $\vf{\alpha}:I\rightarrow\RR^3$ be an arc-length parametrization of $C$ of class $\mathcal{C}^2$, $s_0\in I$ and suppose that $k_{\vf\alpha}(s_0)\ne 0$. Then, there exists a unique circle of $\RR^3$ that has contact of order $\geq 2$ at $\vf{\alpha}(s_0)$. This circle is called \emph{osculating circle} and its radius (called \emph{radius of curvature}) is $\rho_{\vf{\alpha}}(s_0):=\frac{1}{k_{\vf\alpha}(s_0)}$. Its center is $c(s_0)=\vf{\alpha}(s_0)+\rho_{\vf{\alpha}}(s_0)\N\alpha(s_0)$\footnote{An arc-length parametrization of the osculating circle is, for example: $$u\longmapsto c(s_0)+\rho_{\vf{\alpha}}(s_0)\left(-\cos\left(\frac{u}{\rho_{\vf{\alpha}}(s_0)}\right)\N\alpha(s_0)+\sin\left(\frac{u}{\rho_{\vf{\alpha}}(s_0)}\right)\T\alpha(s_0)\right)$$}.
  \end{proposition}
  \begin{proposition}
    Let $\vf{\alpha}:I\rightarrow\RR^2$ be an regular parametrization of $C$ of class $\mathcal{C}^3$, $s_0\in I$ and suppose that $k_{\vf\alpha}(s_0)\ne 0$. If $\vf\alpha(t)=(x(t),y(t))$, then the center of the osculating circle at $\vf\alpha(s_0)$ has coordinates $(X,Y)$ given by:
    $$X=x+y'\frac{{x'}^2+{y'}^2}{x''y'-x'y''}\quad Y=y-x'\frac{{x'}^2+{y'}^2}{x''y'-x'y''}$$
  \end{proposition}
  \begin{center}
    \begin{minipage}{\linewidth}
      \centering
      \includestandalone[mode=image|tex,width=0.8\linewidth]{Images/oscu-circle}
      \captionof{figure}{Osculating circle of a cycloid at a certain point}
    \end{minipage}
  \end{center}
  \begin{definition}
    Let $I\subseteq\RR$ be an open interval, $C\subset\RR^3$ be a curve and $\vf{\alpha}:I\rightarrow\RR^3$ be an arc-length parametrization of $C$ of class $\mathcal{C}^3$, $s_0\in I$ and suppose that $k_{\vf\alpha}(s_0)\ne 0$. Then, there exists a unique sphere of $\RR^3$ that has contact of order $\geq 3$ at $\vf{\alpha}(s_0)$. This sphere is called \emph{osculating sphere} of $\vf\alpha$ at $\vf{\alpha}(s_0)$ and its center $c(s_0)$ and radius $r(s_0)$ are given by:
    \begin{align*}
      c(s_0)     & =\vf\alpha(s_0)+\rh\alpha(s_0)\N\alpha(s_0)-\frac{\rh\alpha'(s_0)}{\ta\alpha'(s_0)}\B\alpha(s_0) \\
      {r(s_0)}^2 & ={\rh\alpha(s_0)}^2+{\left(\frac{\rh\alpha'(s_0)}{\ta\alpha(s_0)}\right)}^2
    \end{align*}
  \end{definition}
  \subsubsection{Envolute and involute}
  \begin{definition}
    An \emph{envelope} of a family of plane curves is a curve that is tangent to each of the members of the family at some point.
  \end{definition}
  \begin{definition}
    Let $I\subseteq\RR$ be an open interval and $\vf{\alpha},\vf\beta:I\rightarrow\RR^2$ be regular parametrizations of two curves of class $\mathcal{C}^3$ such that $k_{\vf\alpha}(s),\ta\alpha(s)\ne 0$ $\forall s\in I$. We say that $\vf\beta$ is the \emph{evolute} of $\vf\alpha$ if $\vf\beta$ is the envelope of all the normal lines to $\vf\alpha$.
  \end{definition}
  \begin{proposition}
    Let $I\subseteq\RR$ be an open interval, $C\subset\RR^3$ be a curve and $\vf{\alpha}:I\rightarrow\RR^2$ be regular parametrization of $C$ of class $\mathcal{C}^3$ such that $k_{\vf\alpha}(s)\ne 0$ $\forall s\in I$. Then, a parametrization of the evolute of $\vf\alpha$ is: $$t\longmapsto \vf\alpha(t)+\rh\alpha(t)\N\alpha(t)$$
  \end{proposition}
  \begin{definition}
    Let $I\subseteq\RR$ be an open interval and $\vf{\alpha},\vf\beta:I\rightarrow\RR^2$ be regular parametrizations of two curves of class $\mathcal{C}^3$. We say that $\vf\beta$ is the \emph{involute} of $\vf\alpha$ if $\vf\beta$ intersects orthogonally all the tangent lines to $\vf\alpha$.
  \end{definition}
  \begin{proposition}
    Let $I\subseteq\RR$ be an open interval, $C\subset\RR^3$ be a curve, $\vf{\alpha}:I\rightarrow\RR^2$ be regular parametrization of $C$ of class $\mathcal{C}^3$ such that $k_{\vf\alpha}(s)\ne 0$ $\forall t_0\in I$ and $s_0\in I$. Then, a parametrization of the involute of $\vf\alpha$ is: $$t\longmapsto \vf\alpha(t)-\T\alpha(t) \int_{t_0}^t\|\vf{\alpha}'(u)\|\dd{u}$$
  \end{proposition}
  \begin{proposition}
    The evolute of the involute of a curve $C\subset\RR^3$ is the curve $C$ itself.
  \end{proposition}
  \begin{center}
    \begin{minipage}{\linewidth}
      \centering
      \includestandalone[mode=image|tex,width=\linewidth]{Images/involute-evolute}
      \captionof{figure}{Construction of the evolute and involute of a curve}
    \end{minipage}
  \end{center}
  \subsubsection{Curvature of plane curves}
  \begin{lemma}
    Let $I\subseteq\RR$ be an open interval, $a,b:I\rightarrow\RR$ be differentiable functions such that $a^2+b^2=1$, $t_0\in I$ and $\theta_0\in\RR$ be such that $a(t_0)=\cos\theta_0$ and $b(t_0)=\sin\theta_0$. Then, the differentiable function $\theta(t)$ defined as:
    $$\theta(t)=\theta_0+\int_{t_0}^t\left(a(u)b'(u)-a'(u)b(u)\right)\dd{u}$$
    satisfies $a(t)=\cos\theta(t)$, $b(t)=\sin\theta(t)$ and $\theta(t_0)=\theta_0$ $\forall t\in I$.
  \end{lemma}
  \begin{proposition}
    Let $I\subseteq\RR$ be an open interval, $C\subset\RR^3$ be a curve and $\vf{\alpha}:I\rightarrow\RR^2$ be a regular arc-length parametrization of $C$ of class $\mathcal{C}^3$. Then, there is a unique vector $\vf{\hat{N}}_{\vf\alpha}$ such that $(\T\alpha,\vf{\hat{N}}_{\vf\alpha})$ is a positive orthonormal basis of $\RR^2$. Thus, $\T\alpha'\parallel \vf{\hat{N}}_{\vf\alpha}$.
  \end{proposition}
  \begin{definition}
    Let $I\subseteq\RR$ be an open interval, $C\subset\RR^3$ be a curve, $\vf{\alpha}:I\rightarrow\RR^2$ be a regular arc-length parametrization of $C$ of class $\mathcal{C}^3$ and $s_0\in I$. We define the \emph{signed curvature} of $\vf\alpha$ at $\vf\alpha(s_0)$ as the value $\ka{\vf\alpha}(s_0)$ satisfying $\T\alpha'(s_0)=\ka{\vf\alpha}(s_0)\vf{\hat{N}}_{\vf\alpha}(s_0)$\footnote{Using the notation of the last proposition, note that $\vf{\hat{N}}_{\vf\alpha}=\pm\N\alpha$ and therefore $\ka{\vf\alpha}=\pm k_{\vf\alpha}$.}. Moreover: $$\ka{\vf\alpha}=\det(\T\alpha,\T\alpha')$$
  \end{definition}
  \begin{proposition}
    Let $I\subseteq\RR$ be an open interval, $C\subset\RR^3$ be a curve and $\vf{\alpha}:I\rightarrow\RR^2$ be a regular parametrization of $C$ of class $\mathcal{C}^3$. Then, the signed curvature of $\vf\alpha$ is: $$\ka{\vf\alpha}=\frac{\det(\vf\alpha',\vf\alpha'')}{{\norm{\vf\alpha'}}^3}$$
  \end{proposition}
  \subsubsection{Local form of a curve}
  \begin{definition}
    Let $I\subseteq\RR$ be an open interval, $C\subset\RR^3$ be a curve, $\vf{\alpha}:I\rightarrow\RR^2$ be a regular arc-length parametrization of $C$ of class $\mathcal{C}^3$ and $s_0\in I$. Consider the affine frame of reference $\mathcal{R}=\{\vf{\alpha}(s_0);(\T\alpha,\N\alpha,\B\alpha)\}$ and suppose $\vf{\alpha}(s)_{\mathcal{R}}=(x(s),y(s),z(s))$. Then:
    $$
      \begin{cases}
        \displaystyle x(s)\simeq s-\frac{{k_{\vf\alpha}(0)}^2}{6}s^3                              \\
        \displaystyle y(s)\simeq \frac{{k_{\vf\alpha}(0)}^2}{2}s^2-\frac{k_{\vf\alpha}'(0)}{6}s^3 \\
        \displaystyle z(s)\simeq -\frac{k_{\vf\alpha}(0)\ta\alpha(0)}{6}s^3
      \end{cases}
    $$
    This expression of $\vf{\alpha}(s)_{\mathcal{R}}$ is called \emph{local canonical form} of $\vf{\alpha}$ in a neighbourhood of $s_0$.
  \end{definition}
  \begin{corollary}
    Let $I\subseteq\RR$ be an open interval, $C\subset\RR^3$ be a curve, $\vf{\alpha}:I\rightarrow\RR^2$ be a regular arc-length parametrization of $C$ of class $\mathcal{C}^3$ and $s_0\in I$. Then, in the reference $\mathcal{R}=\{\vf{\alpha}(s_0);(\T\alpha,\N\alpha,\B\alpha)\}$ we have:
    \begin{itemize}
      \item If $\tau <0$, at $s=0$ the curve cross the osculating plane towards the direction that points $\B\alpha$ (\emph{dextrorotation}).
      \item If $\tau >0$, at $s=0$ the curve cross the osculating plane towards the opposite direction that points $\B\alpha$ (\emph{levorotation}).
    \end{itemize}
  \end{corollary}
  \subsubsection{Orthogonal group}
  \begin{definition}
    We define that \emph{orthogonal group} as the group of all linear transformations that preserve the inner product. That is: $$\text{O}(n):=\{\vf{A}\in\mathcal{M}_n(\RR):\langle\vf{Au},\vf{Av}\rangle=\langle\vf{u},\vf{v}\rangle\;\forall\vf{u},\vf{v}\in\RR^n\}$$
  \end{definition}
  \begin{proposition}
    Let $\vf{A}\in\text{O}(n)$. Then, $\vf{A}\transpose{\vf{A}}=\vf{I}_n$ and $\det\vf{A}=\pm 1$.
  \end{proposition}
  \begin{definition}
    We define that \emph{special orthogonal group} as: $$\text{SO}(n):=\{\vf{A}\in\text{O}(n):\det\vf{A}=1\}$$
  \end{definition}
  \begin{lemma}
    Let $\vf{A}\in\text{O}(n)$ and $\lambda\in\sigma(\vf{A})$. Then, $\lambda\in\RR\implies\lambda=\pm 1$.
  \end{lemma}
  \begin{proposition}
    Let $\vf{A}\in\text{O}(2)$. Then:
    $$\vf{A}=
      \begin{cases}
        \vspace{0.1cm}
        \begin{pmatrix}
          \cos\omega & -\sin\omega \\
          \sin\omega & \cos\omega  \\
        \end{pmatrix} &
        \text{if } \det\vf{A}=1  \\
        \begin{pmatrix}
          \cos\omega & \sin\omega  \\
          \sin\omega & -\cos\omega \\
        \end{pmatrix} &
        \text{if } \det\vf{A}=-1
      \end{cases}
    $$
    for some $\omega\in\RR$.
  \end{proposition}
  \begin{proposition}
    Let $\vf{A}\in\text{O}(3)$. Then, there exists an orthonormal basis $\mathcal{B}$ of $\RR^3$ such that $${[\id]_{\mathcal{B},\text{Can}(\RR^3)}}^{-1}\vf{A}[\id]_{\mathcal{B},\text{Can}(\RR^3)}=
      \begin{pmatrix}
        \pm 1 & 0                                              & 0 \\
        0     & \multicolumn{2}{c}{\multirow{2}{*}{$\vf{A}'$}}     \\
        0     &                                                &
      \end{pmatrix}$$
    where $\vf{A}'\in\text{O}(2)$.
  \end{proposition}
  \begin{proposition}
    Let $\vf{f}:\RR^n\rightarrow\RR^n$ be an Euclidean motion\footnote{Recall that an Euclidean motion is a function that preserves the distance, that is, if $\vf{f}:\RR^n\rightarrow\RR^n$ is an Euclidean motion, then $\|\vf{f}(\vf{u})-\vf{f}(\vf{v})\|=\|\vf{u}-\vf{v}\|$ $\forall \vf{u},\vf{v}\in\RR^n$.}. Then, $\exists\vf{A}\in\text{O}(n)$ and $\vf{u}\in\RR^n$ such that: $$\vf{f}(\vf{v})=\vf{Av}+\vf{u}$$
  \end{proposition}
  \begin{proposition}
    Let $I\subseteq\RR$ be an open interval, $C\subset\RR^3$ be a curve, $\vf{\alpha}:I\rightarrow\RR^n$ be a parametrization of $C$ of class $\mathcal{C}^3$ and $\vf{A}\in\mathcal{M}_n(\RR)$. Then:
    $${\left(\vf{A}\vf{\alpha}\right)}'(t)=\vf{A}\vf{\alpha}'(t)$$
  \end{proposition}
  \begin{proposition}
    Let $\vf{A}\in\text{SO}(3)$. Then, $\forall \vf{u},\vf{v}\in\RR^3$ we have: $$\vf{A}(\vf{u}\times\vf{v})=\left(\vf{Au}\right)\times\left(\vf{Av}\right)$$
  \end{proposition}
  \begin{corollary}
    Let $I\subseteq\RR$ be an open interval, $C\subset\RR^3$ be a curve, $\vf{\alpha}:I\rightarrow\RR^3$ be an arc-length parametrization of $C$ of class $\mathcal{C}^3$ and $\vf{\beta}:=\vf{A\alpha}+\vf{u}$, where $\vf{A}\in\text{SO}(3)$ and $\vf{u}\in\RR^3$. Then, $\vf{\beta}$ is arc-length parametrized and the TNB frame of $\vf{\beta}$ is:
    $$\T\beta=\vf{A}\T\alpha\qquad\N\beta=\vf{A}\N\beta\qquad\B\beta=\vf{A}\B\alpha$$
    And the curvature and torsion of $\vf\beta$ are: $$k_{\vf\beta}=k_{\vf\alpha}\qquad\ta\beta=\ta\alpha$$
  \end{corollary}
  \subsubsection{Fundamental theorem of curves}
  \begin{theorem}[Fundamental theorem of curves]
    Let $I\subseteq\RR$ be an open interval and $k,\tau:I\rightarrow\RR$ be functions of class $\mathcal{C}^3$ with $k(s)>0$ $\forall s\in I$. Then, there is a curve $C$, arc-length parametrized by $\vf{\alpha}:I\rightarrow\RR^3$ of class $\mathcal{C}^3$, whose curvature and torsion are $k$ and $\tau$, respectively. Moreover, if $\tilde{C}$ is another curve arc-length parametrized by $\vf{\tilde{\alpha}}:I\rightarrow\RR^3$ satisfying these restrictions, then there exists an Euclidean motion that carries $\tilde{C}$ into $C$.
  \end{theorem}
  \subsection{Submanifolds of \texorpdfstring{$\RR^n$}{Rn}}
  \subsubsection{Planar functions}
  \begin{definition}
    Let $U\subseteq\RR^n$ be an open set and $f:U\rightarrow\RR$ be a function. We define the \emph{support} of $f$ as:
    $$\supp(f):=\Cl\left(\{x\in U:f(x)\ne 0\}\right)$$
  \end{definition}
  \begin{lemma}
    Let $x_0\in\RR^n$ and $a,b\in \RR_{>0}$ with $a<b$. Then, there exists a function $\rho:\RR^n\rightarrow[0,1]$ of class $\mathcal{C}^\infty$ such that $\supp(\rho)\subseteq\overline{B(x_0,b)}$ and $\rho|_{B(x_0,a)}=1$.
  \end{lemma}
  \begin{proposition}
    Let $U\subseteq\RR^n$ be an open set and $K\subset U$ be a compact set. Then, there exists a function $\rho:\RR^n\rightarrow[0,1]$ of class $\mathcal{C}^\infty$ such that $\supp(\rho)\subseteq U$ such that $\rho|_K=1$.
  \end{proposition}
  \begin{corollary}
    Let $U\subseteq\RR^n$ be an open set, $K\subset U$ be a compact set and $f:U\rightarrow \RR$ be a function of class $\mathcal{C}^\infty$. Then, there exists a function $\tilde{f}:U\rightarrow \RR$ such that $\tilde{f}|_K=f|_K$ and $\tilde{f}|_{\RR^n\setminus U}=0$.
  \end{corollary}
  \subsubsection{Immersions and submersion}
  \begin{definition}[Immersion]
    Let $n,m\in\NN$ with $n\leq m$, $U\subseteq\RR^n$ be an open set and $\vf{f}:U\rightarrow \RR^m$ be a function of class $\mathcal{C}^\infty$. We say that $\vf{f}$ is an \emph{immersion} at $x_0\in U$ if $\vf{\dd{f}}_{x_0}$ is injective. We say that $\vf{f}$ is an \emph{immersion} (on $U$) if it is an immersion at each point $x\in U$.
  \end{definition}
  \begin{definition}[Submersion]
    Let $n,m\in\NN$ with $n\geq m$, $U\subseteq\RR^n$ be an open set and $\vf{f}:U\rightarrow \RR^m$ be a function of class $\mathcal{C}^\infty$. We say that $\vf{f}$ is a \emph{submersion} at $x_0\in U$ if $\vf{\dd{f}}_{x_0}$ is surjective. We say that $\vf{f}$ is a \emph{submersion} (on $U$) if it is a submersion at each point $x\in U$.
  \end{definition}
  \begin{proposition}
    Let $n,m\in\NN$, $U\subseteq\RR^n$ be an open set and $\vf{f}:U\rightarrow \RR^m$ be a function of class $\mathcal{C}^\infty$. Then:
    \begin{itemize}
      \item If $n\leq m$, then: $$\vf{f}\text{ is immersion}\iff\rank\vf{df}_p=n\ \forall p\in U$$
      \item If $n\geq m$, then: $$\vf{f}\text{ is submersion}\iff\rank\vf{df}_p=m\ \forall p\in U$$
    \end{itemize}
  \end{proposition}
  \begin{theorem}[Local structure of immersions]
    Let $n,m\in\NN$ with $n\leq m$, $U\subseteq\RR^n$ be an open set, $\vf{f}:U\rightarrow \RR^m$ be an immersion at $x_0\in U$ and $\vf{\iota}:\RR^n\rightarrow\RR^m$ be the inclusion map. Then, there exist neighbourhoods $V\subseteq U$ of $x_0$ and $W\subseteq\RR^m$ of $\vf\iota(x_0)$ and a diffeomorphism $\vf{g}:W\rightarrow \vf{g}(W)$ such that the following diagram is commutative, that is, $\vf{f}=\vf{g}\circ\vf\iota$.
    \begin{center}
      \begin{minipage}{\linewidth}
        \centering
        \includestandalone[mode=image|tex,width=0.45\linewidth]{Images/theorem_immersions}
        \captionof{figure}{}
      \end{minipage}
    \end{center}
  \end{theorem}
  \begin{theorem}[Local structure of submersions]
    Let $n,m\in\NN$ with $n\geq m$, $U\subseteq\RR^n$ be an open set, $\vf{f}:U\rightarrow \RR^m$ be a submersion at $x_0\in U$ and $\vf\pi_1:\RR^m\times\RR^{n-m}\rightarrow\RR^m$ be the projection map into the first coordinate. Then, there exists a neighbourhood $V\subseteq U$ of $x_0$ and a diffeomorphism $\vf{g}:V\rightarrow \vf{g}(V)$ such that the following diagram is commutative, that is, $\vf{f}=\vf\pi_1\circ\vf{g}$.
    \begin{center}
      \begin{minipage}{\linewidth}
        \centering
        \includestandalone[mode=image|tex,width=0.45\linewidth]{Images/theorem_submersions}
        \captionof{figure}{}
      \end{minipage}
    \end{center}
  \end{theorem}
  \subsubsection{Submanifolds of \texorpdfstring{$\RR^n$}{Rn}}
  \begin{definition}
    Let $M\subseteq\RR^n$ be a set. We say that $M$ is a \emph{submanifold} of $\RR^n$ of dimension $p$ (and codimension $q:=n-p$) if $\forall z\in M$ there exists a neighbourhood $U\subseteq \RR^n$ of $z$ and a diffeomorphism $\vf{g}:U\rightarrow\vf{g}(U)$ such that: $$\vf{g}(U\cap M)=\vf{g}(U)\cap\left(\RR^p\times\{0\}\right)$$
  \end{definition}
  \begin{theorem}
    Let $M\subseteq\RR^n$ be a set. The following statements are equivalent:
    \begin{enumerate}
      \item $M$ is a submanifold of $\RR^n$ of dimension $p$ and codimension $q$.
      \item $\forall z\in M$ there exists a neighbourhood $U\subseteq \RR^n$ of $z$ and a submersion $\vf\phi:U\rightarrow\RR^q$ such that $U\cap M={\vf\phi}^{-1}(0)$.
      \item $\forall z\in M$ there exists a neighbourhood $V\subseteq \RR^p$ of $z$ and an immersion $\vf\varphi:V\rightarrow\RR^n$ such that $z\in\vf\varphi(V)\subseteq M$ and $\vf\varphi:V\rightarrow\vf\varphi(V)$ is a homeomorphism.
    \end{enumerate}
  \end{theorem}
  \begin{definition}
    Let $M\subseteq\RR^n$ be a submanifold, $V\subseteq \RR^p$ and $\vf\varphi:V\rightarrow\vf\varphi(V)\subseteq M$ be an immersion and a homeomorphism. We say that the pair $(V,\vf\varphi)$ is a \emph{parametrization} of $M$ and the pair $(\vf\varphi(V),{\vf\varphi}^{-1})$, a \emph{coordinate chart} of $M$.
  \end{definition}
  \begin{proposition}
    Let $(V_1,\vf\varphi_1)$, $(V_2,\vf\varphi_2)$ be two pa\-ram\-e\-triza\-tions of a submanifold $M\subseteq\RR^n$. Then, the composition ${\vf\varphi_2}^{-1}\circ\vf\varphi_1$ is differentiable on its domain.
  \end{proposition}
  \begin{proposition}
    Let $M\subseteq\RR^n$ be a submanifold of $\RR^n$ of dimension $p$, $V\subseteq\RR^p$ be an open set and $\vf\varphi:V\rightarrow M$ be a differentiable injective immersion. Then, $\vf\varphi(V)\subseteq M$ is an open set and $\vf\varphi:V\rightarrow \vf\varphi(V)$ is a homeomorphism. Hence, $(V,\vf\varphi)$ is a parametrization of $M$.
  \end{proposition}
  \subsubsection{Surfaces of \texorpdfstring{$\RR^3$}{R3}}
  \begin{definition}
    A submanifold of $\RR^3$ of dimension 2 is called a \emph{regular surface} (or simply \emph{surface}) of $\RR^3$.
  \end{definition}
  \begin{proposition}
    Let $U\subseteq\RR^2$ be an open set an $h:U\rightarrow\RR$ be a function of class $\mathcal{C}^1$. Then, $\graph (h)$ is a surface.
  \end{proposition}
  \begin{proposition}
    Let $S\subseteq\RR^3$ be a set. Then, $S$ is a surface if and only if $\forall z\in S$ there exists an open neighbourhood $U\subseteq \RR^3$ of $z$, a change of the order of the variables $\vf\sigma$, a neighbourhood $V\subseteq \RR^2$ of $\vf\pi_1(\sigma(z))$ (where $\vf\pi_1:\RR^2\times \RR\rightarrow\RR^2$ is the projection map) and a differentiable function $h:V\rightarrow \RR$ such that: $$\vf\sigma(S\cap U)=\graph(h)$$
  \end{proposition}
  \begin{proposition}
    Let $U\subseteq\RR^3$, $f:U\rightarrow\RR$ be a function of class $\mathcal{C}^1$ and $a\in\RR$ such that $\vf{d}f_p\ne\vf{0}$ $\forall p\in f^{-1}(a)$. Then, $f^{-1}(a)$ is a surface.
  \end{proposition}
  \begin{definition}
    Let $(a(u),v(u))$, $u\in I$, be a parametrization of class $\mathcal{C}^1$ of a planar curve $C$. We define the \emph{surface of revolution} created by rotating $C$ around an axis of rotation. The curve $C$ is called \emph{generatrix}. In particular, if we choose the $y$-axis as the axis of rotation, $$\vf\varphi(u,v)=(a(u)\cos v,a(u)\sin v,b(u))\quad (u,v)\in I\times (0,2\pi)$$
    is a parametrization of the induced surface of revolution.
  \end{definition}
  \subsubsection{Differentiable functions}
  \begin{definition}
    Let $S\subseteq\RR^3$ be a surface. We say that a function $\vf{f}:S\rightarrow \RR^n$ is \emph{differentiable} at a point $p\in S$ if there is a local parametrization $(V,\vf\varphi)$ of $S$ with $p\in\vf\varphi(V)$ such that $\vf{f}\circ\vf\varphi$ is differentiable at ${\vf\varphi}^{-1}(p)$. We say that $\vf{f}$ is \emph{differentiable} on $S$ if it is differentiable at each point $p\in S$.
  \end{definition}
  \begin{proposition}
    Let $S\subseteq\RR^3$ be a surface, $\vf{f}:S\rightarrow \RR^n$ be a differentiable function and $p\in S$. Then, there exists an open neighbourhood $U\subseteq\RR^3$ of $p$ and a function $\vf{\tilde{f}}:S\rightarrow \RR^n$ such that $\vf{\tilde{f}}|_{U\cap S}=\vf{f}|_{U\cap S}$.
  \end{proposition}
  \begin{corollary}
    Let $S\subseteq\RR^3$ be a surface, $U\subseteq \RR^n$ and $\vf{f}:U\rightarrow \RR^3$ be a differentiable function such that $\vf{f}(U)\subseteq S$. If $(V,\vf\varphi)$ is a local parametrization of $S$, then ${\vf\varphi}^{-1}\circ\vf{f}$ is also a differentiable function on its domain.
  \end{corollary}
  \begin{corollary}
    Let $S\subseteq\RR^3$ be a surface, $(V,\vf\varphi(u,v))$ be a local parametrization of $S$ and $\vf\alpha:I\rightarrow\RR^3$ be a curve of class $\mathcal{C}^\infty$ such that $\vf\alpha(I)\subset \vf\varphi(V)$. Then, $\vf\alpha$ can be written as $\vf\alpha(t)=\vf\varphi(u(t),v(t))$, where $u(t)$, $v(t)$ are differentiable functions.
  \end{corollary}
  \begin{definition}
    Let $S_1,S_2\subseteq\RR^3$ be surfaces. We say that a function $\vf{f}:S_1\rightarrow S_2$ is differentiable if $\forall p\in S_1$, there exist parametrizations $(V_1,\vf\varphi_1)$ and $(V_2,\vf\varphi_2)$ of $S_1$ and $S_2$ respectively with $p\in\vf\varphi_1$ and $\vf{f}(p)\in\vf\varphi_2$ and such that ${\vf\varphi_2}^{-1}\circ\vf{f}\circ\vf\varphi_1$ is differentiable on its domain\footnote{In particular, note that if $S_1=S_2$ and $\vf{f}=\vf\id$, then ${\vf\varphi_2}^{-1}\circ\vf{\id}\circ\vf\varphi_1={\vf\varphi_2}^{-1}\circ\vf\varphi_1$ is a change of coordinates on the surface.}.
  \end{definition}
  \begin{proposition}
    Let $S_1,S_2\subseteq\RR^3$ be surfaces and $\vf{f}:S_1\rightarrow S_2$ be a function. Then, $\vf{f}$ is differentiable if $\vf{f}\circ\vf{\iota}:S_1\hookrightarrow \RR^3$ is differentiable.
  \end{proposition}
  \subsubsection{Tangent space}
  \begin{definition}
    Let $S\subseteq\RR^3$ be a surface and $p\in S$. If $\vf\alpha:(-\varepsilon,\varepsilon)\rightarrow\RR^3$ is a parametrization of a curve of class $\mathcal{C}^\infty$ such that $\vf\alpha(0)=p$, we say that $\vf\alpha'(0)$ is a \emph{tangent vector} to $S$ at $p$. The set of all such vectors is called \emph{tangent space} (or \emph{tangent plane}) and it is denoted as $T_pS$.
  \end{definition}
  \begin{proposition}
    Let $S\subseteq\RR^3$ be a surface, $p\in S$, $(V,\vf\varphi)$ be a local parametrization of $S$ with $p\in\vf\varphi(V)$ and $f:U\rightarrow\RR$ be a submersion with $S\cap U=f^{-1}(0)$. Then: $$\im\vf{\dd{\varphi}}_{{\vf\varphi}^{-1}(p)}=T_pS=\ker \vf{\dd}{f}_p$$
    Therefore, $T_pS$ is a vector space and $\dim T_pS=2$.
  \end{proposition}
  \begin{proposition}
    Let $S\subseteq\RR^3$ be a surface, $(V,\vf\varphi(u,v))$ be a local parametrization of $S$ and $p=\vf\varphi(u_0,v_0)\in S$. Then, the tangent vectors $$\left(\pdv{\vf\varphi}{u}(u_0,v_0),\pdv{\vf\varphi}{v}(u_0,v_0)\right)\footnote{Usually we will denote these partial derivatives by $\vf\varphi_u=\pdv{\vf\varphi}{u}(u_0,v_0)$ and $\vf\varphi_v=\pdv{\vf\varphi}{v}(u_0,v_0)$, respectively.}$$ form a basis of the tangent plane $T_pS$.
  \end{proposition}
  \begin{lemma}
    Let $U\subseteq \RR^n$ be an open set, $\vf{f}:U\rightarrow\RR^m$ be a differentiable function and $\vf\alpha:(-\varepsilon,\varepsilon)\rightarrow U$ be a parametrization of a curve of class $\mathcal{C}^\infty$ such that $\vf\alpha(0)=p$ and $\vf\alpha'(0)=\vf{v}$. Then: $$\vf{\dd{f}}_p(\vf{v})={(\vf{f}\circ\vf\alpha)}'(0)$$
  \end{lemma}
  \begin{definition}
    Let $S_1,S_2\subseteq\RR^3$ be surfaces, $p\in S_1$ and $\vf{f}:S_1\rightarrow S_2$ be a differentiable function. We define the \emph{tangent function} (or \emph{differential}) of $\vf{f}$ at $p$ as the function:
    $$\function{\vf{\dd{f}}_p}{T_pS_1}{T_{\vf{f}(p)}S_2}{\vf{v}}{{(\vf{f}\circ\vf\alpha)}'(0)}$$ where $\vf\alpha:(-\varepsilon,\varepsilon)\rightarrow S_1$ is a parametrization of a curve of class $\mathcal{C}^\infty$ such that $\vf\alpha(0)=p$ and $\vf\alpha'(0)=\vf{v}$.
  \end{definition}
  \begin{proposition}
    Let $S_1,S_2\subseteq\RR^3$ be surfaces,  $p\in S_1$ and $\vf{f}:S_1\rightarrow S_2$ be a differentiable function. Then, $\vf{\dd{f}}_p$ is linear. Moreover if $(V_1,\vf\varphi_1(u,v))$ and $(V_2,\vf\varphi_2(\tilde{u},\tilde{v}))$ are parametrizations of $S_1$ and $S_2$ respectively, $\tilde{u}=f_1(u,v)$, $\tilde{v}=f_2(u,v)$\footnote{That is, $f_1$ and $f_2$ are the component functions of ${\vf\varphi_2}^{-1}\circ\vf{f}\circ\vf\varphi_1$.} and $\mathcal{B}_1=\left(\pdv{\vf\varphi_1}{u},\pdv{\vf\varphi_1}{v}\right)$, $\mathcal{B}_2=\left(\pdv{\vf\varphi_2}{\tilde{u}},\pdv{\vf\varphi_2}{\tilde{v}}\right)$, we have that: $$[\vf{\dd{f}}_p]_{\mathcal{B}_1,\mathcal{B}_2}=
      \renewcommand\arraystretch{2}
      \begin{pmatrix}
        \displaystyle\pdv{f_1}{u}({\vf{\varphi}}^{-1}(p)) & \displaystyle\pdv{f_1}{v}({\vf{\varphi}}^{-1}(p)) \\
        \displaystyle\pdv{f_2}{u}({\vf{\varphi}}^{-1}(p)) & \displaystyle\pdv{f_2}{v}({\vf{\varphi}}^{-1}(p))
      \end{pmatrix}$$
  \end{proposition}
  \begin{theorem}[Inverse function theorem for surfaces]
    Let $S_1,S_2\subseteq\RR^3$ be surfaces, $p\in S_1$ and $\vf{f}:S_1\rightarrow S_2$ be a differentiable function. Suppose $\vf{\dd{f}}_p$ is an isomorphism. Then, $\vf{f}$ is a diffeomorphism between a neighbourhood $U_1\subseteq S_1$ of $p$ and a neighbourhood $U_2\subseteq S_2$ of $\vf{f}(p)$.
  \end{theorem}
  \subsection{First fundamental form}
  \subsubsection{First fundamental form}
  \begin{definition}
    Let $S\subseteq\RR^3$ be a surface and $p\in S$. We define the \emph{first fundamental form} of $S$ at $p$ as the quadratic form $\I_p:T_pS\times T_pS\rightarrow \RR$ defined by:
    $$\I_p(\vf{v}):={\langle\vf{v},\vf{v}\rangle}_p:={\|\vf{v}\|}^2\footnote{Abusing notation, we will denote the bilinear associated function to $\I_p$ also as $\I_p$. That is, $\I_p(\vf{u},\vf{v})={\langle\vf{u},\vf{v}\rangle}_p$.}$$
  \end{definition}
  \begin{proposition}
    Let $S\subseteq\RR^3$ be a surface, $(V,\vf\varphi(u,v))$ be a local parametrization of $S$ and $p\in S$. Then, in the basis $(\vf\varphi_u,\vf\varphi_v)$ we have:
    $$
      \I_p=\begin{pmatrix}
        E_{\vf\varphi} & F_{\vf\varphi} \\
        F_{\vf\varphi} & G_{\vf\varphi}
      \end{pmatrix}:=\begin{pmatrix}
        {\langle\vf\varphi_u,\vf\varphi_u\rangle}_p & {\langle\vf\varphi_u,\vf\varphi_v\rangle}_p \\
        {\langle\vf\varphi_v,\vf\varphi_u\rangle}_p & {\langle\vf\varphi_v,\vf\varphi_v\rangle}_p
      \end{pmatrix}\footnote{Sometimes we will omit writing the subindex of $E_{\vf\varphi}$, $F_{\vf\varphi}$ and $G_{\vf\varphi}$.}
    $$
    That is, if $\vf{u}=a\vf\varphi_u+b\vf\varphi_v$ and $\vf{v}=c\vf\varphi_u+d\vf\varphi_v$, then
    $${\langle\vf{u},\vf{v}\rangle}_p=\begin{pmatrix}
        a & b
      \end{pmatrix}\begin{pmatrix}
        E_{\vf\varphi} & F_{\vf\varphi} \\
        F_{\vf\varphi} & G_{\vf\varphi}
      \end{pmatrix}\begin{pmatrix}
        c \\
        d
      \end{pmatrix}$$
    and $$\I_p(\vf{u})=a^2 E_{\vf\varphi}+2ab F_{\vf\varphi}+b^2G_{\vf\varphi}$$
  \end{proposition}
  \begin{definition}
    Let $S\subseteq\RR^3$ be a surface, $(V,\vf\varphi(u,v))$ be a local parametrization of $S$ and $p\in S$. We say that the parametrization $(V,\vf\varphi(u,v))$ is \emph{orthogonal} (or that $u$ and $v$ are \emph{orthogonal coordinates}) if $F_{\vf\varphi}=0$.
  \end{definition}
  \begin{proposition}
    Let $S\subseteq\RR^3$ be a surface, $(V,\vf\varphi(u,v))$ be a local parametrization of $S$ and $\vf\alpha:I\rightarrow\vf\varphi(V)$ be a parametrization of a curve of class $\mathcal{C}^\infty$. We can write $\vf\alpha$ as $\vf\alpha(t)=\vf\varphi(u(t),v(t))$. Then:
    $$\|\vf\alpha'(t)\|=\sqrt{{u'(t)}^2 E_{\vf\varphi}+2u'(t)v'(t) F_{\vf\varphi}+{v'(t)}^2G_{\vf\varphi}}$$ where $E_{\vf\varphi}=E_{\vf\varphi}(u(t),v(t))$, $F_{\vf\varphi}=F_{\vf\varphi}(u(t),v(t))$, $G_{\vf\varphi}=G_{\vf\varphi}(u(t),v(t))$. The arc-length parameter is thus:
    $$s(t)=\int_{t_0}^t\sqrt{{u'}^2 E_{\vf\varphi}+2u'v' F_{\vf\varphi}+{v'}^2G_{\vf\varphi}}\dd{\xi}$$
  \end{proposition}
  \begin{proposition}
    Let $S\subseteq\RR^3$ be a surface, $(V,\vf\varphi(u,v))$ be a local parametrization of $S$ and $p\in S$. Then, the angle $\beta$ between the coordinates lines of the parametrization $(V,\vf\varphi(u,v))$ is: $$\cos\beta=\frac{\langle\vf\varphi_u,\vf\varphi_v\rangle}{\|\vf\varphi_u\|\|\vf\varphi_v\|}=\frac{F_{\vf\varphi}}{\sqrt{E_{\vf\varphi}G_{\vf\varphi}}}$$
  \end{proposition}
  \subsubsection{Area}
  \begin{definition}
    Let $S\subseteq\RR^3$ be a surface and $D\subseteq S$ be a subset. We say that $D$ is a \emph{regular domain} (or simply \emph{domain}) if $D$ is open, connected and $\Fr D\subset S$ is the image of a piecewise curve of class $\mathcal{C}^1$. A \emph{region} $R\subseteq S$ is the union of a domain $D$ with its boundary, $R=D\cup\Fr D$.
  \end{definition}
  \begin{definition}
    Let $S\subseteq\RR^3$ be a surface, $(V,\vf\varphi(u,v))$ be a local parametrization of $S$ and $R\subset \vf\varphi(V)$ be a compact region. Let $Q={\vf\varphi}^{-1}(R)\subseteq\RR^2$. We define the area of $R$ as:
    \begin{align*}
      \area(R) & =\iint_Q\|\vf\varphi_u\times\vf\varphi_v\|\dd{u}\dd{v}                                                                                                                                       \\
               & =\iint_Q\sqrt{E_{\vf\varphi}G_{\vf\varphi}-{F_{\vf\varphi}}^2}\dd{u}\dd{v}\footnote{One can check that this definition does not depend on the parametrization $(V,\vf\varphi(u,v))$ of $S$.}
    \end{align*}
  \end{definition}
  \begin{definition}
    Let $S\subseteq\RR^3$ be a surface, $(V,\vf\varphi(u,v))$ be a local parametrization of $S$, $R\subset \vf\varphi(V)$ be a compact region and $f:S\rightarrow \RR$ be a function. Let $Q={\vf\varphi}^{-1}(R)\subseteq\RR^2$. We define the \emph{integral of $f$ over the region $R$} as: $$\iint_Rf\dd{S}:=\iint_Q(f\circ\vf\varphi)\sqrt{E_{\vf\varphi}G_{\vf\varphi}-{F_{\vf\varphi}}^2}\dd{u}\dd{v}\footnote{One can check that this definition does not depend on the parametrization $(V,\vf\varphi(u,v))$ of $S$.}$$
  \end{definition}
  \subsubsection{Isometries}
  \begin{definition}
    Let $S_1,S_2\subseteq\RR^3$ be surfaces and $\vf{f}:S_1\rightarrow S_2$ be a differentiable function. We say that $\vf{f}$ is a \emph{local isometry} if the differential function $\vf{\dd{f}}_p$ is an isometry $\forall p\in S_1$. That is, for each $p\in S_1$ we have:
    $$\langle\vf{v},\vf{w}\rangle_1=\langle\vf{\dd{f}}_p(\vf{v}),\vf{\dd{f}}_p(\vf{w})\rangle_2\quad\forall\vf{v},\vf{w}\in T_pS_1\footnote{Here, ${\langle\cdot,\cdot\rangle}_i$ represents the first fundamental form of $S_i$, $i=1,2$.}$$
    We say that $\vf{f}$ is an \emph{isometry} if it is a local isometry and it is invertible.
  \end{definition}
  \begin{proposition}
    Let $S_1,S_2\subseteq\RR^3$ be surfaces and $\vf{f}:S_1\rightarrow S_2$ be a local isometry. Then, $\vf{\dd{f}}_p$ is a isomorphism.
  \end{proposition}
  \begin{proposition}
    Let $S_1,S_2\subseteq\RR^3$ be surfaces and $\vf{f}:S_1\rightarrow S_2$ be a function of class $\mathcal{C}^1$. Then, $\vf{f}$ is a local isometry if and only if $\vf{f}$ preserves lengths, that is, for any curve $\vf\alpha:I\rightarrow S_1$, we have $L(\vf\alpha)=L(\vf{f}\circ\vf\alpha)$.
  \end{proposition}
  \begin{proposition}
    Let $S_1,S_2\subseteq\RR^3$ be surfaces, $(V,\vf\varphi)$ be a local parametrization of $S_1$, $\vf{f}:S_1\rightarrow S_2$ be a function of class $\mathcal{C}^1$. Then, $(V,\vf\psi=\vf{f}\circ\vf\varphi)$ is a local parametrization of $S_2$ and moreover:
    $$\vf{f}\text{ is an isometry}\iff E_{\vf\varphi}=E_{\vf\psi}, F_{\vf\varphi}=F_{\vf\psi}, G_{\vf\varphi}=G_{\vf\psi}$$
  \end{proposition}
  \begin{corollary}
    Let $S_1,S_2\subseteq\RR^3$ be surfaces and $\vf{f}:S_1\rightarrow S_2$ be an isometry. Then, $\vf{f}$ preserves areas.
  \end{corollary}
  \subsubsection{Conformal maps}
  \begin{definition}
    Let $U,V\subseteq\RR^n$ be open sets, $\vf{f}:U\rightarrow V$ be a function and $p\in U$. We say that $\vf{f}$ is \emph{conformal} (or \emph{angle-preserving}) at $p$ it preserves angles between directed curves through $p$, as well as preserving orientation. We say that $\vf{f}$ is \emph{conformal} (on $U$) if it is conformal at each $p\in U$.
  \end{definition}
  \begin{theorem}
    Let $S_1,S_2\subseteq\RR^3$ be surfaces, $(V,\vf\varphi)$ be a parametrization of $S_1$ and $\vf{f}:S_1\rightarrow S_2$ be a conformal map. Consider the parametrization $(V,\vf\psi=\vf{f}\circ\vf\varphi)$ be a parametrization of $S_2$. Then, there exists a function $\rho:U\rightarrow\RR$ such that:
    $$E_{\vf\varphi}=\rho^2 E_{\vf\psi}\quad F_{\vf\varphi}=\rho^2 F_{\vf\psi}\quad G_{\vf\varphi}=\rho^2 G_{\vf\psi}$$
  \end{theorem}
  \subsection{Second fundamental form}
  \subsubsection{Orientation of surfaces and Gau\ss\ map}
  \begin{definition}
    Let $S\subseteq\RR^3$ be a surface. We say that $S$ is \emph{orientable} if it admits a \emph{normal unit field}, that is, a differentiable function $\vf\nu_S:S\rightarrow S^2\subseteq\RR^3$ such that $\vf\nu_S(p)\in T_pS^\perp$ $\forall p\in S$. This function $\vf\nu_S$\footnote{Unless necessary, we will omit writing the subindex $S$.} is known as \emph{Gau\ss\ map}.
  \end{definition}
  \begin{definition}
    Let $S\subseteq\RR^3$ be an orientable and connected surface. An \emph{orientation} of $S$ is the choice of one of the two ($\vf\nu_S$ or $-\vf\nu_S$) unit normal fields.
  \end{definition}
  \begin{definition}
    Let $S\subseteq\RR^3$ be an orientable surface and $(V,\vf\varphi(u,v))$ be a local parametrization of $S$. We say that $(V,\vf\varphi(u,v))$ is \emph{compatible} with the orientation of $S$ if $$\vf\nu_S=\frac{\vf\varphi_u\times\vf\varphi_v}{\|\vf\varphi_u\times\vf\varphi_v\|}$$
  \end{definition}
  \begin{proposition}
    Let $S\subseteq\RR^3$ be a surface. $S$ is orientable if and only if $S$ can be covered by the images $\vf\varphi_i(V_i)$ of a collection of parametrizations $\{(V_i,\vf\varphi_i):i\in I\}$ of $S$ such that $$\det\vf{\dd}({\vf\varphi_j}^{-1}\circ\vf\varphi_i)>0\quad\forall i,j\in I$$
  \end{proposition}
  \subsubsection{Weingarten endomorphism}
  \begin{definition}
    Let $S\subseteq\RR^3$ be a surface oriented with a normal unit field $\vf\nu$. We define the \emph{Weingarten endomorphsim} of $S$ at the point $p\in S$ as the endomorphism: $$\function{\vf{W}_p}{T_pS}{T_pS}{\vf{v}}{-\vf{\dd{\nu}}_p(\vf{v})}$$
  \end{definition}
  \begin{lemma}
    Let $S\subseteq\RR^3$ be a surface, $(V,\vf\varphi(u,v))$ be a local parametrization of $S$, $\vf\alpha(t)=\vf\varphi(u(t),v(t))$ be a curve on $S$ and $p=\vf{\alpha}(0)$. We denote $\vf\nu(t)=(\vf\nu\circ\vf\alpha)(t)=\vf\nu(u(t),v(t))$. Then:
    \begin{align*}
      \vf{\dd{\nu}}_p(\vf\alpha'(0)) & =\vf{\dd{\nu}}_p(u'(0)\vf\varphi_u+v'(0)\vf\varphi_v) \\
                                     & =u'(0)\vf\nu_u+v'(0)\vf\nu_v
    \end{align*}
    In particular, $\vf{\dd{\nu}}_p(\vf\varphi_u)=\vf\nu_u$ and $\vf{\dd{\nu}}_p(\vf\varphi_v)=\vf\nu_v$.
  \end{lemma}
  \begin{proposition}
    Let $S\subseteq\RR^3$ be a surface oriented with a normal unit field $\vf\nu$ and $p\in S$. Then, the Weingarten endomorphism is auto-adjoint with respect to the first fundamental form. That is: $${\langle \vf{W}_p(\vf{u}),\vf{v}\rangle}_p={\langle \vf{u},\vf{W}_p(\vf{v})\rangle}_p\quad\forall\vf{u},\vf{v}\in T_pS$$
  \end{proposition}
  \begin{proposition}
    Let $S\subseteq\RR^3$ be an orientable surface and $p\in S$. Then, the Weingarten endomorphism has real eigenvalues and it diagonalizes in an orthonormal basis of $T_pS$.
  \end{proposition}
  \begin{definition}
    Let $S\subseteq\RR^3$ be an orientable surface and $p\in S$. We define the \emph{principal directons} of $S$ at $p$ as the eigenspaces of $\vf{W}_p$. We define the \emph{principal curvatures} of $S$ at $p$ as the eigenvalues of $\vf{W}_p$.
  \end{definition}
  \begin{definition}
    Let $S\subseteq\RR^3$ be an orientable surface and $p\in S$. We say that the point $p$ is an \emph{umbilic point} if $\vf{W}_p=\lambda \vf{id}$, for some $\lambda\in\RR$.
  \end{definition}
  \begin{definition}
    Let $S\subseteq\RR^3$ be an orientable surface, $p\in S$ and $k_1$, $k_2$ be the principal curvatures of $S$ at $p$. We define the \emph{Gau\ss\ curvature} of $S$ at $p$ as:
    $$K(p):=\det\vf{W}_p=k_1k_2$$
    We define the \emph{mean curvature} of $S$ at $p$ as:
    $$H(p):=\frac{\trace\vf{W}_p}{2}=\frac{k_1+k_2}{2}$$
  \end{definition}
  \begin{definition}
    Let $S\subseteq\RR^3$ be an orientable surface. We say that $S$ is a \emph{minimal surface} if $H=0$.
  \end{definition}
  \subsubsection{Second fundamental form}
  \begin{definition}
    Let $S\subseteq\RR^3$ be a surface oriented with a normal unit field $\vf\nu$ and $p\in S$. We define the \emph{second fundamental form} of $S$ at $p$ as the quadratic form $\II_p:T_pS\times T_pS\rightarrow \RR$ defined by: $$\II_p(\vf{v})=\I_p(\vf{W}_p(\vf{v}),\vf{v})={\langle\vf{W}_p(\vf{v}),\vf{v}\rangle}_p\footnote{Abusing notation, we will denote the bilinear associated function to $\II_p$ also as $\II_p$. That is, $\II_p(\vf{u},\vf{v})={\langle\vf{W}_p(\vf{u}),\vf{v}\rangle}_p$.}$$
  \end{definition}
  \begin{definition}
    Let $S\subseteq\RR^3$ be a surface oriented with a normal unit field $\vf\nu$, $p\in S$ and $\vf\alpha:I\rightarrow S$ be a regular curve. Suppose $\cos\theta=\langle\vf{N}_{\vf\alpha}(p),\vf\nu(p)\rangle$. We define the \emph{normal curvature} of $\vf\alpha$ at $p$ as: $$k_\text{n}(p):=k_{\vf\alpha}\cos\theta$$
  \end{definition}
  \begin{proposition}[Meusnier's theorem]
    Let $S\subseteq\RR^3$ be an orientable surface, $p\in S$ and $\vf\alpha:I\rightarrow S$ be an arc-length parametrization of a curve $C$ of class $\mathcal{C}^\infty$ such that $\vf\alpha(0)=p$. Then: $$k_\text{n}(p)=\II_p(\vf\alpha'(0))$$
    In particular, $k_\text{n}(p)$ depends only on the tangent line to $\vf\alpha$ at $p$.
  \end{proposition}
  \begin{definition}
    Let $S\subseteq\RR^3$ be an orientable surface, $p\in S$ and $\vf{v}\in T_pS$ with $\|\vf{v}\|=1$. We define the \emph{normal curvature} at $p$ in the direction of $\vf{v}$ as: $$k_\text{n}(\vf{v}):=\II_p(\vf{v})$$
  \end{definition}
  \begin{proposition}
    Let $S\subseteq\RR^3$ be an orientable surface, $p\in S$ and $(\vf{v}_1,\vf{v}_2)$ be an orthonormal basis of $T_pS$, where $\vf{v}_i$ is an eigenvector of eigenvalue $k_i$ of $\vf{W}_p$ for $i=1,2$. Then, for $i=1,2$ we have: $$k_i=k_\text{n}(\vf{v}_i)$$
  \end{proposition}
  \begin{definition}
    Let $S\subseteq\RR^3$ be an orientable surface, $p\in S$ and $\vf{v}\in T_pS$ with $\|\vf{v}\|=1$. We say that the direction of $\vf{v}$ in $T_pS$ is an \emph{asymptotic direction} if $k_\text{n}(\vf{v})=\II_p(\vf{v})=0$.
  \end{definition}
  \begin{definition}
    Let $S\subseteq\RR^3$ be an orientable surface and $C\subset S$ be a curve. We say that $C$ is a \emph{line of curvature} of $S$ if the tangent line at $C$ is a principal direction at each point $p\in C$. We say that $C$ is an \emph{asymptotic line} of $S$ if the tangent line at $C$ is an asymptotic direction at each point $p\in C$.
  \end{definition}
  \begin{proposition}[Olinde Rodrigues' theorem]
    Let $S\subseteq\RR^3$ be an orientable surface and $\vf\alpha:I\rightarrow S$ be a regular parametrization of a curve $C$ of class $\mathcal{C}^\infty$. Let $\vf\nu(t):=(\vf\nu\circ\vf\alpha)(t)$. Then, $C$ is a line of curvature of $S$ if and only if $$\vf\nu'(t)=\lambda(t)\vf\alpha'(t)$$ where $\lambda(t)$ is a differentiable function. In this case, $-\lambda(t)$ is the principal curvature of $S$ in the direction of $\vf\alpha'(t)$.
  \end{proposition}
  \begin{proposition}[Euler's formula]
    Let $S\subseteq\RR^3$ be an orientable surface, $p\in S$ and $(\vf{v}_1,\vf{v}_2)$ be an orthonormal basis of $T_pS$, where $\vf{v}_i$ is an eigenvectors of eigenvalue $k_i$ of $\vf{W}_p$ for $i=1,2$. Then: $$k_\text{n}(\cos\theta \vf{v}_1+\sin\theta \vf{v}_2)=k_1{\left(\cos\theta\right)}^2+k_2{\left(\sin\theta\right)}^2$$
    Hence, we will denote $k_\text{n}(\theta):=k_1{\left(\cos\theta\right)}^2+k_2{\left(\sin\theta\right)}^2$.
  \end{proposition}
  \begin{corollary}
    Let $S\subseteq\RR^3$ be an orientable surface, $p\in S$. Then, the extrema of $k_\text{n}(p)$ are precisely the principal curvatures $k_1$ and $k_2$ at $p$.
  \end{corollary}
  \begin{proposition}
    Let $S\subseteq\RR^3$ be a surface oriented with a normal unit field $\vf\nu$ and $p\in S$. Then, if we invert the orientation of $\vf\nu$, the curvatures $k_1$, $k_2$, $H$ and $k_\text{n}$ change their sign but $K$ remains invariant.
  \end{proposition}
  \begin{proposition}
    Let $S\subseteq\RR^3$ be an orientable surface. Then: $$H=\frac{k_1+k_2}{2}=\frac{1}{2\pi}\int_0^{2\pi}k_\text{n}(\theta)\dd\theta$$
  \end{proposition}
  \begin{proposition}
    Let $S\subseteq\RR^3$ be a surface oriented with a normal unit field $\vf\nu$ and $\vf\alpha:I\rightarrow S$ be a regular curve of class $\mathcal{C}^\infty$. Then: $$k_\text{n}=\frac{\langle\vf\alpha'',\vf\nu\circ\vf\alpha\rangle}{{\|\vf\alpha'\|}^2}$$
  \end{proposition}
  \begin{definition}
    Let $S\subseteq\RR^3$ be an orientable surface and $p\in S$. We say that $p$ is
    \begin{itemize}
      \item an \emph{elliptic point} if $K(p)>0$.
      \item a \emph{hyperbolic point} if $K(p)<0$.
      \item a \emph{parabolic point} if $K(p)=0$ but $\vf{W}_p\ne0$.
      \item a \emph{plane point} if $K(p)=0$ and $\vf{W}_p=0$.
    \end{itemize}
  \end{definition}
  \begin{proposition}
    Let $S\subseteq\RR^3$ be an orientable and connected surface such that all of its points are umbilic. Then, $S$ is contained in a sphere or in a plane.
  \end{proposition}
  \subsubsection{Gau\ss\ map in coordinates}
  \begin{proposition}
    Let $(V,\vf\varphi(u,v))$ be a local pa\-ram\-e\-triza\-tion of a surface $S\subseteq\RR^3$ oriented with a normal unit field $\vf\nu$ and $p\in S$.
    Suppose
    \begin{align*}
      \vf\nu_u & =a_{11}\vf\varphi_u+a_{21}\vf\varphi_v \\
      \vf\nu_v & =a_{12}\vf\varphi_u+a_{22}\vf\varphi_v
    \end{align*}
    Then: $$\vf{\dd{\nu}}_p=-\vf{W}_p=-
      \begin{pmatrix}
        a_{11} & a_{12} \\
        a_{21} & a_{22}
      \end{pmatrix}$$
  \end{proposition}
  \begin{proposition}
    Let $(V,\vf\varphi(u,v))$ be a local pa\-ram\-e\-triza\-tion of a surface $S\subseteq\RR^3$ oriented with a normal unit field $\vf\nu$ and $p\in S$. Then, we have:
    \begin{align*}
      e_{\vf\varphi} & :=-\langle\vf\nu_u,\vf\varphi_u\rangle=\langle \vf\nu,\vf\varphi_{uu}\rangle                                                                            \\
      f_{\vf\varphi} & :=-\langle\vf\nu_v,\vf\varphi_u\rangle=\langle \vf\nu,\vf\varphi_{uv}\rangle=\langle \vf\nu,\vf\varphi_{vu}\rangle=-\langle\vf\nu_u,\vf\varphi_v\rangle \\
      g_{\vf\varphi} & :=-\langle\vf\nu_v,\vf\varphi_v\rangle=\langle \vf\nu,\vf\varphi_{vv}\rangle
    \end{align*}
    Moreover, in the basis $(\vf\varphi_u,\vf\varphi_v)$ we have:
    $$
      \II_p=\begin{pmatrix}
        e_{\vf\varphi} & f_{\vf\varphi} \\
        f_{\vf\varphi} & g_{\vf\varphi}
      \end{pmatrix}\footnote{Sometimes we will omit writing the subindex of $e_{\vf\varphi}$, $f_{\vf\varphi}$ and $g_{\vf\varphi}$.}
    $$
  \end{proposition}
  \begin{proposition}
    Let $(V,\vf\varphi(u,v))$ be a local pa\-ram\-e\-triza\-tion of a surface $S\subseteq\RR^3$ oriented with a normal unit field $\vf\nu$ and $p\in S$. Then:
    $$\vf{W}_p={\I_p}^{-1}\II_p$$
    Hence: $$\vf{W}_p=\frac{1}{E_{\vf\varphi}G_{\vf\varphi}-{F_{\vf\varphi}}^2}
      \begin{pmatrix}
        e_{\vf\varphi}G_{\vf\varphi}-f_{\vf\varphi}F_{\vf\varphi}  & f_{\vf\varphi}G_{\vf\varphi}-g_{\vf\varphi}F_{\vf\varphi}  \\
        -e_{\vf\varphi}F_{\vf\varphi}+f_{\vf\varphi}E_{\vf\varphi} & -f_{\vf\varphi}F_{\vf\varphi}+g_{\vf\varphi}E_{\vf\varphi}
      \end{pmatrix}\!\!$$
  \end{proposition}
  \begin{corollary}
    Let $(V,\vf\varphi(u,v))$ be a local parametrization of a surface $S\subseteq\RR^3$ oriented with a normal unit field $\vf\nu$ and $p\in S$. Then:
    \begin{gather*}
      K =\frac{e_{\vf\varphi}g_{\vf\varphi}-{f_{\vf\varphi}}^2}{E_{\vf\varphi}G_{\vf\varphi}-{F_{\vf\varphi}}^2}               \\
      H =\frac{1}{2}\frac{e_{\vf\varphi}G_{\vf\varphi}-2f_{\vf\varphi}F_{\vf\varphi}+g_{\vf\varphi}E_{\vf\varphi}}{E_{\vf\varphi}G_{\vf\varphi}-{F_{\vf\varphi}}^2}
    \end{gather*}
    Moreover the principal curvatures are given by: $$k_1,k_2=H\pm\sqrt{H^2-K}$$
  \end{corollary}
  \begin{proposition}
    Let $S\subseteq\RR^3$ be an orientable surface, $(V,\vf\varphi(u,v))$ be a parametrization of $S$ and $\vf\alpha:I\rightarrow S$ be a regular parametrization of a curve $C$ of class $\mathcal{C}^\infty$ such that $\vf\alpha(t)=\vf\varphi(u(t),v(t))$. Then:
    \begin{enumerate}
      \item $C$ is an asymptotic line if and only if: $$e_{\vf\varphi}{u'}^2+2f_{\vf\varphi}u'v'+g_{\vf\varphi}{v'}^2=0$$
      \item $C$ is a line of curvature if and only if:
            $$
              \begin{vmatrix}
                {v'}^2         & -u'v'          & {u'}^2         \\
                E_{\vf\varphi} & F_{\vf\varphi} & G_{\vf\varphi} \\
                e_{\vf\varphi} & f_{\vf\varphi} & g_{\vf\varphi}
              \end{vmatrix}=0
            $$
    \end{enumerate}
  \end{proposition}
  \subsubsection{Geometric interpretation of the Gau\ss\ curvature}
  \begin{lemma}
    Let $S\subseteq\RR^3$ be an orientable surface, $p\in S$, $(\vf{v}_1,\vf{v}_2)$ be a basis of $T_pS$ and $\vf{A}:T_pS\rightarrow T_pS$ be a linear isomorphism. Then: $$\vf{A}\vf{v}_1\times\vf{A}\vf{v}_2=\det\vf{A}\left(\vf{v}_1\times\vf{v}_2\right)$$
    In particular, if $\vf\nu$ is a normal unit field of $S$ and $\vf{w}_1,\vf{w}_2\in T_pS$, then: $$\vf{\dd{\nu}}_p(\vf{w}_1)\times\vf{\dd{\nu}}_p(\vf{w}_2)=K(p)\left(\vf{w}_1\times\vf{w}_1\right)$$
    If $\vf{w}_1=\vf\varphi_u$ and $\vf{w}_2=\vf\varphi_v$ we have: $$\vf{\nu}_u\times\vf{\nu}_v=K\left(\vf{\varphi}_u\times\vf{\varphi}_v\right)$$
  \end{lemma}
  \begin{definition}
    Let $S\subseteq\RR^3$ be a surface oriented with a normal unit field $\vf\nu$ and $R\subseteq S$ be a region on $S$ where the curvature $K$ doesn't vanish. We define the signed area of $\vf\nu(R)$ as: $$\area_\text{s}(\vf\nu(R))=\sign(K)\area(\vf\nu(R))$$
  \end{definition}
  \begin{proposition}
    Let $S\subseteq\RR^3$ be an orientable surface, $p\in S$ such that $K(p)\ne 0$ and $V\subseteq S$ be a connected neighbourhood of $p$ where $K$ has constant sign. Let $(B_n)\subseteq V$ be a sequence of regions that converge to $p$. Then: $$K(p)=\lim_{n\to\infty}\frac{\area_\text{s}(\vf\nu(B_n))}{\area(B_n)}$$
  \end{proposition}
  \subsubsection{Ruled surfaces}
  \begin{definition}
    Let $S\subseteq\RR^3$ be a surface is called \emph{ruled surface} if it has a parametrization of the form $$\vf\varphi(u,v)=\vf\alpha(u)+v\bf\beta(u)$$ where $\vf\alpha$ and $\vf\beta$ are curves of $\RR^3$ such that $\abs{\vf\beta}=1$.
  \end{definition}
  \begin{proposition}
    Let $S\subseteq\RR^3$ be a ruled surface. Then, $K\leq 0$.
  \end{proposition}
  \begin{definition}
    Let $S\subseteq\RR^3$ be a surface. We say that $S$ is \emph{developable} if it is ruled and $K=0$.
  \end{definition}
  \begin{proposition}
    Let $S\subseteq\RR^3$ be a developable surface and $(V,\vf\varphi(u,v))$ be a parametrization of $S$. Then, there exists a curve $v=h(u)$ where $\vf\varphi$ stops being regular. This curve is called \emph{regression axis}.
  \end{proposition}
  \subsection{Intrinsic geometry of surfaces}
  \subsubsection{Gau\ss' Theorema Egregium}
  \begin{definition}
    Let $S\subseteq\RR^3$ be an orientable surface and $(V,\vf\varphi(u,v))$ be a parametrization of $S$. Then
    \begin{align*}
      \vf\varphi_{uu} & =\Gamma_{11}^1\vf\varphi_u+\Gamma_{11}^2\vf\varphi_v+e\vf\nu \\
      \vf\varphi_{uv} & =\Gamma_{12}^1\vf\varphi_u+\Gamma_{12}^2\vf\varphi_v+f\vf\nu \\
      \vf\varphi_{vu} & =\Gamma_{21}^1\vf\varphi_u+\Gamma_{21}^2\vf\varphi_v+f\vf\nu \\
      \vf\varphi_{vv} & =\Gamma_{22}^1\vf\varphi_u+\Gamma_{22}^2\vf\varphi_v+g\vf\nu
    \end{align*}
    for some coefficients $\Gamma_{ij}^k$, $i,j,k\in\{1,2\}$. This coefficients are called \emph{Christoffel symbols}\footnote{Observe that $\Gamma_{ij}^k=\Gamma_{ji}^k$  $\forall i,j,k\in\{1,2\}$.}.
  \end{definition}
  \begin{proposition}
    Let $S\subseteq\RR^3$ be an orientable surface and $(V,\vf\varphi(u,v))$ be a parametrization of $S$. Then:
    \begin{align*}
      \begin{pmatrix}
        E & F \\
        F & G
      \end{pmatrix}
      \begin{pmatrix}
        \Gamma_{11}^1 \\
        \Gamma_{11}^2
      \end{pmatrix} & =\begin{pmatrix}
                         \frac{1}{2}E_u \\
                         F_u-\frac{1}{2}E_v
                       \end{pmatrix}  \\
      \begin{pmatrix}
        E & F \\
        F & G
      \end{pmatrix}
      \begin{pmatrix}
        \Gamma_{12}^1 \\
        \Gamma_{12}^2
      \end{pmatrix} & =\begin{pmatrix}
                         \frac{1}{2}E_v \\
                         \frac{1}{2}G_u
                       \end{pmatrix}     \\
      \begin{pmatrix}
        E & F \\
        F & G
      \end{pmatrix}
      \begin{pmatrix}
        \Gamma_{22}^1 \\
        \Gamma_{22}^2
      \end{pmatrix} & =\begin{pmatrix}
                         F_v-\frac{1}{2}G_u \\
                         \frac{1}{2}G_v
                       \end{pmatrix}
    \end{align*}
    That is, the Christoffel symbols only depend on the coefficients of the first fundamental form.
  \end{proposition}
  \begin{proposition}
    Let $S\subseteq\RR^3$ be an orientable surface and $(V,\vf\varphi(u,v))$ be a parametrization of $S$. Then:
    \begin{enumerate}
      \item $\displaystyle {\left(\Gamma_{12}^2\right)}_u-{\left(\Gamma_{11}^2\right)}_v +\Gamma_{12}^1\Gamma_{11}^2-\Gamma_{11}^1\Gamma_{12}^2+\Gamma_{12}^2\Gamma_{12}^2-\Gamma_{11}^2\Gamma_{22}^2=-EK$
      \item $\displaystyle {\left(\Gamma_{12}^1\right)}_u-{\left(\Gamma_{11}^1\right)}_v +\Gamma_{12}^1\Gamma_{12}^2-\Gamma_{11}^2\Gamma_{22}^1=FK$
      \item $\displaystyle {\left(\Gamma_{22}^2\right)}_u-{\left(\Gamma_{12}^2\right)}_v +\Gamma_{22}^1\Gamma_{11}^2-\Gamma_{12}^1\Gamma_{12}^2=-FK$
      \item $\displaystyle {\left(\Gamma_{22}^1\right)}_u-{\left(\Gamma_{12}^1\right)}_v +\Gamma_{22}^1\Gamma_{11}^1+\Gamma_{22}^2\Gamma_{12}^1-\Gamma_{12}^1\Gamma_{12}^1-\Gamma_{12}^2\Gamma_{22}^1=GK$
    \end{enumerate}
    These equations are called \emph{Gau\ss\ equations}. Moreover, we have:
    \begin{enumerate}\setcounter{enumi}{4}
      \item $\displaystyle e_v-f_u=e\Gamma_{12}^1+f\left(\Gamma_{12}^2-\Gamma_{11}^1\right)-g\Gamma_{11}^2$
      \item $\displaystyle f_v-g_u=e\Gamma_{22}^1+f\left(\Gamma_{22}^2-\Gamma_{12}^1\right)-g\Gamma_{12}^2$
    \end{enumerate}
    These equations are called \emph{Codazzi-Mainardi equations}.
  \end{proposition}
  \begin{corollary}
    Let $S\subseteq\RR^3$ be an orientable surface. Then, its Gau\ss\space curvature depends only on the coefficients of the first fundamental form.
  \end{corollary}
  \begin{theorem}[Gau\ss' Theorema Egregium]
    The Gau\ss\space curvature is invariant under local isometries between surfaces.
  \end{theorem}
  \begin{theorem}[Bonnet's theorem]
    Let $V\subseteq\RR^2$ be an open set and $E,F,G,e,f,g:V\rightarrow\RR$ be functions of class $\mathcal{C}^\infty$ such that $E,G,EG-F^2>0$ and such that they satisfy the Gau\ss\space  and Codazzi-Mainardi equations. Then $\forall p\in V$, there exists a neighbourhood $U\subseteq V$ of $p$ and an immersion $\vf\varphi:U\rightarrow\RR^3$ such that $S:=\vf\varphi(U)$ is a regular surface whose first and second fundamental forms coefficients are $E$, $F$, $G$ and $e$, $f$, $g$, respectively. Moreover, if $\vf\psi:U\rightarrow\RR^3$ satisfy the same conditions, then there exist $\vf{A}\in\text{O}(3)$ and $\vf{C}\in\RR^3$ such that $\vf\psi=\vf{A\varphi}+\vf{C}$.
  \end{theorem}
  \begin{proposition}
    Let $S\subseteq\RR^3$ be an orientable surface and $(V,\vf\varphi(u,v))$ be an orthogonal parametrization of $S$. Then: $$K=-\frac{1}{2\sqrt{EG}}\left[{\left(\frac{E_v}{\sqrt{EG}}\right)}_v+{\left(\frac{G_u}{\sqrt{EG}}\right)}_u\right]$$
  \end{proposition}
  \subsubsection{Parallel transport}
  \begin{definition}
    Let $S\subseteq\RR^3$ be a surface and $U\subseteq S$ be an open set. A \emph{vector field} tangent to $S$ defined on $U$ is a correspondence $\vf{X}$ that at each point $p\in U$ it assigns a tangent vector $\vf{X}(p)=:\vf{X}_p\in T_pS$. We say that $\vf{X}$ is \emph{differentiable} at $p\in U$ if there is a parametrization $\vf\varphi(u,v)$ of $S$ whose image contains $p$ such that $$\vf{X}=a\vf\varphi_u+b\vf\varphi_v$$ for some differentiable functions $a(u,v)$, $b(u,v)$ at $p$. We say that $\vf{X}$ is \emph{differentiable} if it is differentiable at each point $p\in U$\footnote{From now on, all the vector fields considered will be differentiable so sometimes we will omit to say it explicitly.}.
  \end{definition}
  \begin{definition}
    Let $S\subseteq\RR^3$ be a surface, $\vf{X}$ be a differentiable vector field tangent to $S$, $p\in S$ and $\vf{w}\in T_pS$. Let $\vf\alpha:(-\varepsilon,\varepsilon)\rightarrow S$ a parametrized curve of class $\mathcal{C}^\infty$ with $\vf\alpha(0)=p$ and $\vf\alpha'(0)=\vf{w}$. We denote $\vf{X}(t):=(\vf{X}\circ\vf\alpha)(t)$. We define the \emph{covariant derivative} of $\vf{X}$ at the point $p$ in the direction of $\vf{w}$, denoted as $\frac{\text{D}\vf{X}}{\dd{t}}(0)$, as the orthogonal projection $\pi^\perp$ of $\vf{X}'(0)$ over the vector field $T_pS$. That is: $$\frac{\text{D}\vf{X}}{\dd{t}}(0)=\pi^\perp\left(\vf{X}'(0)\right)$$
  \end{definition}
  \begin{proposition}
    Let $S\subseteq\RR^3$ be a surface, $\vf{X}$ be a differentiable vector field tangent to $S$, $p\in S$ and $\vf{w}\in T_pS$. Let $\vf\alpha:(-\varepsilon,\varepsilon)\rightarrow S$ a parametrized curve of class $\mathcal{C}^\infty$ with $\vf\alpha(0)=p$ and $\vf\alpha'(0)=\vf{w}$. Suppose $(V,\vf\varphi(u,v))$ is a parametrization $\vf\varphi(u,v)$ of $S$ whose image contains $p$. Suppose $\vf{X}(t)=a(u(t),v(t))\vf\varphi_u+b(u(t),v(t))\vf\varphi_v=a(t)\vf\varphi_u+b(t)\vf\varphi_v$ Then:
    \begin{multline}\label{DG_covderivative}
      \frac{\text{D}\vf{X}}{\dd{t}}(0)=\left(a'+\Gamma_{11}^1au'+\Gamma_{12}^1av'+\Gamma_{21}^1bu'+\Gamma_{22}^1bv'\right)\vf\varphi_u\\
      +\left(b'+\Gamma_{11}^2au'+\Gamma_{12}^2av'+\Gamma_{21}^2bu'+\Gamma_{22}^2bv'\right)\vf\varphi_v
    \end{multline}
  \end{proposition}
  \begin{definition}
    Let $S\subseteq\RR^3$ be a surface and $\vf\alpha:I\rightarrow S$ be a curve of class $\mathcal{C}^\infty$. A \emph{vector field} tangent to $S$ along $\vf\alpha$ is a correspondence $\vf{X}$ that at each $t\in U$ it assigns a tangent vector $\vf{X}(t)=:\vf{X}_{\vf\alpha(t)}\in T_{\vf\alpha(t)}S$. We say that $\vf{X}$ is \emph{differentiable} if at each local chart $(V,\vf\varphi(u,v))$ we have: $$\vf{X}(t)=a(t)\vf\varphi_u+b(t)\vf\varphi_v$$ for some differentiable functions $a(t)$, $b(t)$. We define its covariant derivative (along $\vf\alpha$) as the vector field $\frac{\text{D}\vf{X}}{\dd{t}}$ defined by \cref{DG_covderivative}, which is differentiable along $\vf\alpha$.
  \end{definition}
  \begin{definition}
    Let $S\subseteq\RR^3$ be a surface and $\vf{X}$ be a vector field tangent to $S$ along a curve $\vf\alpha:I\rightarrow S$ of class $\mathcal{C}^\infty$. We say that $\vf{X}$ is \emph{parallel} if: $$\frac{\text{D}\vf{X}}{\dd{t}}=0$$
  \end{definition}
  \begin{proposition}
    Let $S\subseteq\RR^3$ be a surface and $\vf{X}$, $\vf{Y}$ be vector fields tangent to $S$ along a curve $\vf\alpha:I\rightarrow S$ of class $\mathcal{C}^\infty$. Then, $t\mapsto\langle \vf{X}(t),\vf{Y}(t)\rangle$ is constant. In particular, the norms $\|\vf{X}(t)\|$, $\|\vf{Y}(t)\|$ as well as the angle between $\vf{X}(t)$ and $\vf{Y}(t)$ are constant.
  \end{proposition}
  \begin{proposition}
    Let $S\subseteq\RR^3$ be a surface and $\vf\alpha I\rightarrow S$ be a parametrized curve of class $\mathcal{C}^\infty$. Then, given $t_0\in I$ and $\vf{w}\in T_{\vf\alpha(t_0)}S$ there exists a unique parallel vector field $\vf{X}$ along $\vf\alpha$ such that $\vf{X}(t_0)=\vf{w}$. This vector field is called \emph{parallel transport} of the vector $\vf{w}$ along $\vf{\alpha}$ and it is defined on the entire interval $I$.
  \end{proposition}
  \subsubsection{Geodesics}
  \begin{definition}
    Let $S\subseteq\RR^3$ be a surface and $\vf\alpha I\rightarrow S$ be a parametrized curve of class $\mathcal{C}^\infty$. We say that $\vf\alpha$ is a \emph{geodesic} of $S$ if the tangent vector $\vf{\alpha}'$ is parallel along $\vf\alpha$. That is, if it satisfies: $$\frac{\text{D}\vf{\alpha}'}{\dd{t}}=0$$
  \end{definition}
  \begin{proposition}
    Let $\vf\alpha$ be a geodesic of a surface $S\subseteq\RR^3$ be a surface. Then, $\|\vf\alpha'\|$ is constant.
  \end{proposition}
  \begin{proposition}
    Let $S\subseteq\RR^3$ be an orientable surface, $(V,\vf\varphi(u,v))$ be a local parametrization of $S$ and $\vf\alpha I\rightarrow S$ be a parametrized curve of class $\mathcal{C}^\infty$ with $\alpha^*\subset\vf\varphi(V)$. Suppose $\vf\alpha(t)=\vf\varphi(u(t),v(t))$, for some differentiable functions $u,v:I\rightarrow\RR$. Then, $\vf\alpha$ is a geodesic of $S$ if and only if:
    $$\left\{
      \begin{aligned}
        u''+\Gamma_{11}^1{(u')}^2+2\Gamma_{12}^1u'v'+\Gamma_{22}^1{(v')}^2 & =0 \\
        v''+\Gamma_{11}^2{(u')}^2+2\Gamma_{12}^2u'v'+\Gamma_{22}^2{(v')}^2 & =0
      \end{aligned}
      \right.
    $$
  \end{proposition}
  \begin{proposition}
    Let $S\subseteq\RR^3$ be a surface, $p\in S$ and $\vf{v}\in T_pS$. Then, there exists $\varepsilon>0$ and a curve $\vf\alpha:(-\varepsilon,\varepsilon)\rightarrow S$ of class $\mathcal{C}^\infty$ such that it is a geodesic, $\vf\alpha(0)=p$ and $\vf\alpha'(0)=\vf{v}$.
  \end{proposition}
  \subsubsection{Geodesic curvature}
  \begin{definition}
    Let $S\subseteq\RR^3$ be a surface oriented with a normal unit field $\vf\nu$ and $\vf{X}$ be a unit vector field tangent to $S$ along a curve $\vf\alpha:I\rightarrow S$ of class $\mathcal{C}^\infty$. We define the by $\vf{\overline{X}}$ the unique unit vector field along $\vf\alpha$ such that $(\vf{X},\vf{\overline{X}},\vf\nu)$ is a positive orthonormal basis of $\RR^3$\footnote{Or equivalently such that $(\vf{X},\vf{\overline{X}})$ is a positive orthonormal basis of $T_{\vf\alpha(t)}S$.}.
  \end{definition}
  \begin{definition}
    Let $S\subseteq\RR^3$ be a surface oriented with a normal unit field $\vf\nu$ and $\vf{X}$ be a unit vector field tangent to $S$ along a curve $\vf\alpha:I\rightarrow S$ of class $\mathcal{C}^\infty$. Since $\vf{X}$ is a unit field, we have: $$\frac{\text{D}\vf{X}}{\dd{t}}=\lambda\vf{\overline{X}}=\lambda\vf\nu\times\vf{X}$$
    We define the \emph{algebraic value} of the covariant derivative of $\vf{X}$ at time $t$ as: $$\left[\frac{\text{D}\vf{X}}{\dd{t}}(t)\right]:=\lambda(t)\footnote{Note that the sign of $\left[\frac{\text{D}\vf{X}}{\dd{t}}(t)\right]$ does depend on the orientation of $\vf\nu$.}$$
  \end{definition}
  \begin{definition}
    Let $S\subseteq\RR^3$ be an orientable surface and $\vf\alpha:I\rightarrow S$ be a regular arc-length parametrization of a curve $C$ of class $\mathcal{C}^\infty$. The algebraic value of the covariant derivative of $\vf\alpha'$ at $\vf\alpha(s)$ is: $$k_{\text{g}}(s):=\left[\frac{\text{D}\vf{\alpha}'}{\dd{s}}(s)\right]$$
    This value $k_{\text{g}}$ is called \emph{geodesic curvature} of $C$ at $\vf\alpha(s)$.
  \end{definition}
  \begin{proposition}
    Let $S\subseteq\RR^3$ be an orientable surface and $\vf\alpha:I\rightarrow S$ be a regular arc-length parametrization of a curve $C$ of class $\mathcal{C}^\infty$. Then: $$C\text{ is a geodesic} \iff k_{\text{g}}=0$$
  \end{proposition}
  \begin{proposition}
    Let $S\subseteq\RR^3$ be a surface oriented with a normal unit field $\vf\nu$ and $\vf{X}$ be a unit vector field tangent to $S$ along a curve $\vf\alpha:I\rightarrow S$ of class $\mathcal{C}^\infty$. Then:
    $$\left[\frac{\text{D}\vf{X}}{\dd{t}}(t)\right]=\left\langle\dv{\vf{X}}{t},\vf{\overline{X}}\right\rangle=\left\langle\dv{\vf{X}}{t},\vf\nu\times\vf{X}\right\rangle$$
    In particular if $\vf\alpha$ is arc-length parametrized, then: $$k_\text{g}=\langle\vf\alpha'',\vf\nu\times\vf\alpha'\rangle$$ Or more generally: $$k_\text{g}=\frac{\langle\vf\alpha'',\vf\nu\times\vf\alpha'\rangle}{{\|\vf\alpha'\|}^3}$$
  \end{proposition}
  \begin{proposition}
    Let $S\subseteq\RR^3$ be an orientable surface and $\vf\alpha:I\rightarrow S$ be a regular arc-length parametrization of a curve $C$ of class $\mathcal{C}^\infty$. Then: $$k^2={k_{\text{g}}}^2+{k_\text{n}}^2$$
  \end{proposition}
  \begin{proposition}
    Let $S\subseteq\RR^3$ be an orientable surface and $\vf{X}$, $\vf{Y}$ be two unit vector fields tangent to $S$ along a curve $\vf\alpha:I\rightarrow S$ of class $\mathcal{C}^\infty$. Then: $$\left[\frac{\text{D}\vf{Y}}{\dd{t}}\right]-\left[\frac{\text{D}\vf{X}}{\dd{t}}\right]=\dv{\theta}{t}$$ where $\theta$ is a differentiable determination of the angle between $\vf{X}$ and $\vf{Y}$.
  \end{proposition}
  \begin{corollary}
    Let $S\subseteq\RR^3$ be an orientable surface, $\vf{X}$ be a parallel unit vector field along an arc-length parametrized curve $\vf\alpha:I\rightarrow S$ of class $\mathcal{C}^\infty$ and $\theta$ is a differentiable determination of the angle between $\vf{X}$ and $\vf{\alpha}'$. Then: $$k_\text{g}(s)=\left[\frac{\text{D}\vf{X}}{\dd{s}}(s)\right]$$
  \end{corollary}
  \begin{proposition}
    Let $S\subseteq\RR^3$ be an orientable surface and $(V,\vf\varphi(u,v))$ be an orthogonal parametrization of $S$ compatible with the orientation. Let $\vf{X}$ be a unit vector field tangent to $S$ along a curve $\vf\alpha:I\rightarrow S$ of class $\mathcal{C}^\infty$. Suppose $\vf\alpha(t)=\vf\varphi(u(t),v(t))$. Then: $$\left[\frac{\text{D}\vf{X}}{\dd{t}}\right]=\frac{1}{2\sqrt{EG}}\left(G_uv'-E_vu'\right)+\dv{\theta}{t}$$
    where $\theta$ is the angle from $\vf\varphi_u$ to $\vf{X}$. In particular if the curve $\vf\alpha$ is arc-length parametrized, then:
    $$k_\text{g}=\frac{1}{2\sqrt{EG}}\left(G_uv'-E_vu'\right)+\dv{\theta}{t}$$
  \end{proposition}
  \begin{theorem}[Liouville's formula]
    Let $S\subseteq\RR^3$ be an orientable surface and $(V,\vf\varphi(u,v))$ be an orthogonal parametrization of $S$ compatible with the orientation. Let $\vf\alpha:I\rightarrow S$ be an arc-length parametrized curve of class $\mathcal{C}^\infty$ such that $\vf\alpha(t)=\varphi(u(t),v(t))$ and let $\theta=\theta(s)$ be the angle between $\varphi_u$ and $\vf\alpha'(s)$. Then: $$k_\text{g}={(k_\text{g})}_1\cos\theta+{(k_\text{g})}_2\sin\theta+\dv{\theta}{s}$$
    where ${(k_\text{g})}_1$ and ${(k_\text{g})}_2$ denote the geodesic curvature of the coordinate lines $v=\const$ and $u=\const$, respectively.
  \end{theorem}
  \subsection{Differential forms}
  \subsubsection{Vector fields of \texorpdfstring{$\RR^n$}{Rn}}
  \begin{definition}
    Let $p\in\RR^n$. We denote by $T_p\RR^n$ the vector space defined by: $$T_p\RR^n:=\{(p,\vf{v}):\vf{v}\in\RR^n\}$$
  \end{definition}
  \begin{definition}
    Let $U\subseteq\RR^n$ be an open set. A \emph{vector field} defined on $U$ is a correspondence $\vf{X}$ that at each point $p\in U$ it assigns a vector $\vf{X}(p)=:\vf{X}_p\in T_p\RR^n$.
  \end{definition}
  \begin{definition}
    Let $(\vf{e}_1,\ldots,\vf{e}_n)$ be the standard basis of $\RR^n$. For $i=1,\ldots,n$, we define the following vector fields: $$\function{\vf{E}_i:=\pdv{}{x^i}}{\RR^n}{T_p\RR^n}{p}{(p,\vf{e}_i)}$$
  \end{definition}
  \begin{proposition}
    Let $U\subseteq\RR^n$ be an open set. Then, for each $p\in\RR^n$, $(\vf{E}_1,\ldots,\vf{E}_n)$ is a basis of $T_p\RR^n$. Consequently, given a vector field $\vf{X}$ defined on $U$, it can be uniquely written as: $$\vf{X}=\sum_{i=1}^nX^i\vf{E}_i=\sum_{i=1}^nX^i\pdv{}{x^i}\footnote{The superscript notation is used due to historical resons (Einstein notation). See \url{https://en.wikipedia.org/wiki/Einstein_notation} for further information.}$$
  \end{proposition}
  \begin{definition}
    Let $U\subseteq\RR^n$ be an open set and $\vf{X}=\sum X^i\pdv{}{x^i}$ be a vector field defined on $U$. We say that $\vf{X}$ is \emph{differentiable} at $p\in U$ if the components $X^i$ are differentiable at $p$. We say that $\vf{X}$ is \emph{differentiable} on $U$ if it is differentiable at each point $p\in U$. We denote by $\mathcal{X}(U)$ the set of all differentiable vector fields defined on $U$\footnote{Note that $\mathcal{X}(U)$ is a $\RR$-vector space.}.
  \end{definition}
  \begin{definition}
    Let $U\subseteq\RR^n$ be an open set and $\vf{X},\vf{Y}\in\mathcal{X}(U)$. We define the inner product of $\vf{X}$ and $\vf{Y}$ as: $$\function{\langle\vf{X},\vf{Y}\rangle}{U}{\RR}{p}{\langle\vf{X}_p,\vf{Y}_p\rangle}$$
  \end{definition}
  \begin{definition}
    Let $U\subseteq\RR^n$ be an open set, $f\in\mathcal{C}^\infty(U)$ and $\vf{X}=\sum X^i\pdv{}{x^i}\in\mathcal{X}(U)$. We denote by $\vf{X}f$ the differentiable function defined as: $$\vf{X}f(p):=\vf{X}_pf:=\sum_{i=1}^n X^i(p)\pdv{f}{x^i}\footnote{Observe that $\vf{X}_pf$ is the partial derivative of $f$ at $p$ in the direction of $\vf{X}_p$.}$$
  \end{definition}
  \begin{lemma}
    Let $U\subseteq\RR^n$ be an open set and $\vf{X}\in\mathcal{X}(U)$. Then, the function $$\function{\vf{X}}{\mathcal{C}^\infty(U)}{\mathcal{C}^\infty(U)}{f}{\vf{X}f}$$ satisfies the following properties:
    \begin{enumerate}
      \item It is $\RR$-linear.
      \item $\vf{X}(fg)=(\vf{X}f)g+f(\vf{X}g)$ for all $f,g\in\mathcal{C}^\infty(U)$.
    \end{enumerate}
  \end{lemma}
  \begin{lemma}
    Let $U\subseteq\RR^n$ be an open set, $f\in\mathcal{C}^\infty(U)$ and $\vf{X}\in\mathcal{X}(U)$. Then: $$\vf{X}f=\langle\vf{X},\grad f\rangle$$
  \end{lemma}
  \begin{definition}
    Let $U\subseteq\RR^n$ be an open set and $\vf{X}=\sum X^i\pdv{}{x^i}\in\mathcal{X}(U)$. We say that a parametrized curve $\vf{\gamma}:I\rightarrow\RR^n$ is an \emph{integral curve} of $\vf{X}$ if: $$\vf\gamma'(t)=\vf{X}_{\vf\gamma(t)}\qquad \forall t\in I$$
    That is, the integral curve $\vf\gamma(t)=(x^1(t),\ldots,x^n(t))$ of $\vf{X}$ satisfies the following system of odes:
    $$
      \left\{
      \begin{aligned}
        {(x^1)}' & =\vf{X}^1(\vf\gamma(t)) \\
                 & \;\;\vdots              \\
        {(x^n)}' & =\vf{X}^n(\vf\gamma(t)) \\
      \end{aligned}
      \right.
    $$
  \end{definition}
  \begin{proposition}
    Let $U\subseteq\RR^n$ be an open set, $F\in\mathcal{C}^\infty(U)$ and $\vf{X}\in\mathcal{X}(U)$. We say that $F$ if a \emph{first integral} of $\vf{X}$ if:
    \begin{enumerate}
      \item $\vf{\dd}F_p\ne 0$ $\forall p\in U$
      \item $F$ is constant over the integral curves of $\vf{X}$. That is, $\vf{X}F=0$.
    \end{enumerate}
  \end{proposition}
  \begin{proposition}
    Let $n\geq 2$, $U\subseteq\RR^n$ be an open set, $p\in U$ and $\vf{X}\in\mathcal{X}(U)$. Suppose that $\vf{X}_p\ne 0$. Then, there exists a neighbourhood $V\subseteq U$ of $p$ and a differential function $F:V\rightarrow\RR$ such that $F$ is a first integral of $\vf{X}$.
  \end{proposition}
  \begin{definition}
    Let $S\subseteq\RR^3$ be a regular surface. A \emph{vector field} defined on $S$ is a correspondence $\vf{X}$ that at each point $p\in S$ it assigns a vector $\vf{X}(p)=:\vf{X}_p\in T_p\RR^3$. If there is a parametrization $\vf\varphi(u,v)$ of $S$, we can write
    $$\vf{X}=\vf{X}(u,v)=\sum_{i=1}^3X^i(u,v)\pdv{}{x^i}$$
    where $X^i(u,v):=(X^i\circ\vf\varphi)(u,v)$. We say that $\vf{X}$ is \emph{differentiable} if it the functions $X^i(u,v)$ are differentiable. We say that $\vf{X}$ is \emph{tangent} to $S$ if $\vf{X}_p\in T_pS$ $\forall p\in S$. In this case we can write: $$\vf{X}=\tilde{X}^1\vf\varphi_u+\tilde{X}^2\vf\varphi_v$$
  \end{definition}
  \begin{proposition}
    Let $S\subseteq\RR^3$ be a regular surface and $\vf{X}$, $\vf{Y}$ be tangent differential vector fields to $S$ such that at some point $p\in S$, the vectors $\vf{X}_p$, $\vf{Y}_p$ are linearly independent. Then, there exists a local parametrization $(V,\vf\varphi(u,v))$ of $S$ such that $p\in\vf\varphi(V)$ and $$\vf{X}=\lambda\vf\varphi_u\qquad\vf{Y}=\mu\vf\varphi_v$$ for some differentiable functions $\lambda$, $\mu$.
  \end{proposition}
  \begin{corollary}
    Let $S\subseteq\RR^3$ be a regular surface and $p\in S$. Then, there exists a local orthogonal parametrization $(V,\vf\varphi(u,v))$ of $S$ such that $p\in\vf\varphi(V)$.
  \end{corollary}
  \subsubsection{Multilinear algebra}
  \begin{definition}
    Let $V_1,\ldots,V_k,W$ be a vector spaces over a field $K$. A \emph{multilinear map} (or \emph{$k$-linear map}) is a function $$f:V_1\times\cdots\times V_k\longrightarrow W$$ that is linear separately in each variable. The value $k$ is called \emph{degree} of the multilinear map.
  \end{definition}
  \begin{definition}
    Let $V$ be a vector space of dimension $n$ and $\omega:V\times\overset{(k)}{\cdots}\times V\longrightarrow \RR$ be a $k$-linear map. We say that $\omega$ is \emph{alternating} if $$\omega(\vf{u}_{\sigma(1)},\ldots,\vf{u}_{\sigma(k)})=\sign(\sigma)\omega(\vf{u}_1,\ldots,\vf{u}_k)\qquad\forall\sigma\in \text{S}_k$$
    We denote by $\Lambda^kV^*$ the vector space of the alternating $k$-linear maps. The elements of $\Lambda^kV^*$ are called \emph{multilinear forms}\footnote{Here $V^*$ denotes the dual space of $V$ (see \cref{LA_dual}).}. By agreement we denote $\Lambda^0V^*:=\RR$ and: $$\Lambda^*V^*:=\bigoplus_{k=0}^{n}\Lambda^kV^*$$
  \end{definition}
  \begin{definition}
    Let $V$ be a vector space, $\alpha\in\Lambda^pV^*$ and $\beta\in\Lambda^qV^*$. We define its \emph{exterior product} as the multilinear map $\alpha\wedge\beta$ defined as:
    \begin{multline*}
      \alpha\wedge\beta(\vf{u}_1,\ldots\vf{u}_{p+q})=\frac{1}{p!q!}\sum_{\sigma\in \text{S}_{p+q}}\sign(\sigma)\cdot\\\cdot\alpha(\vf{u}_{\sigma(1)},\ldots,\vf{u}_{\sigma(p)})\beta(\vf{u}_{\sigma(p+1)},\ldots,\vf{u}_{\sigma(p+q)})
    \end{multline*}
  \end{definition}
  \begin{proposition}
    Let $V$ be a vector space and $\alpha\in\Lambda^pV^*$, $\beta\in\Lambda^qV^*$ and $\gamma\in\Lambda^rV^*$. Then:
    \begin{enumerate}
      \item $\alpha\wedge\beta\in\Lambda^{p+q}V^*$ (that is, $\alpha\wedge\beta$ is alternating)
      \item $\alpha\wedge(\beta\wedge\gamma)=(\alpha\wedge\beta)\wedge\gamma$
      \item $\alpha\wedge\beta={(-1)}^{pq}\beta\wedge\alpha$
    \end{enumerate}
  \end{proposition}
  \begin{proposition}
    Let $V$ be a vector space and $\omega^1,\ldots,\omega^k\in V^*=\Lambda^1V^*$. Then: $$\omega^1\wedge\cdots\wedge\omega^k(\vf{u}_1,\ldots,\vf{u}_k)=
      \begin{vmatrix}
        \omega^1(\vf{u}_1) & \cdots & \omega^1(\vf{u}_k) \\
        \vdots             & \ddots & \vdots             \\
        \omega^k(\vf{u}_1) & \cdots & \omega^k(\vf{u}_k)
      \end{vmatrix}$$
  \end{proposition}
  \begin{corollary}
    Let $(\vf{e}_1,\ldots,\vf{e}_n)$ be the standard basis of $\RR^n$ and $({\vf{e}_1}^*,\ldots,{\vf{e}_n}^*)$ be its associated dual basis. Then:
    $${\vf{e}_1}^*\wedge\cdots\wedge{\vf{e}_n}^*(\vf{u}_1,\ldots,\vf{u}_n)=\det(\vf{u}_1,\ldots,\vf{u}_n)$$
  \end{corollary}
  \begin{proposition}
    Let $(\vf{e}_1,\ldots,\vf{e}_n)$ be the standard basis of $E:=\RR^n$ and $({\vf{e}_1}^*,\ldots,{\vf{e}_n}^*)$ be its associated dual basis. Then, given $\omega\in\Lambda^kE^*$ we have that: $$\omega(\vf{u}_1,\ldots,\vf{u}_k)=\sum_{j_1<\cdots<j_k}A_{j_1,\ldots,j_k}\omega(\vf{e}_{j_1},\ldots,\vf{e}_{j_k})$$
    where $A_{j_1,\ldots,j_k}:={\vf{e}_{j_1}}^*\wedge\cdots\wedge{\vf{e}_{j_k}}^*(\vf{u}_1,\ldots,\vf{u}_k)$. Thus: $$\omega=\sum_{j_1<\cdots<j_k}\omega(\vf{e}_{j_1},\ldots,\vf{e}_{j_k}){\vf{e}_{j_1}}^*\wedge\cdots\wedge{\vf{e}_{j_k}}^*$$
  \end{proposition}
  \begin{corollary}
    Let $(\vf{e}_1,\ldots,\vf{e}_n)$ be the standard basis of $E:=\RR^n$ and $({\vf{e}_1}^*,\ldots,{\vf{e}_n}^*)$ be its associated dual basis. Then, the set $$\{{\vf{e}_{j_1}}^*\wedge\cdots\wedge{\vf{e}_{j_k}}^*:j_1<\cdots<j_k,j_i\in\NN\ \forall i=1,\ldots,k\}$$
    is a basis of $\Lambda^kE^*$. In particular, $\dim \Lambda^kE^*=\binom{n}{k}$.
  \end{corollary}
  \begin{corollary}
    Let $(\vf{e}_1,\ldots,\vf{e}_n)$ and $(\vf{v}_1,\ldots,\vf{v}_n)$ be the standard basis and an arbitrary basis of $E:=\RR^n$ with associated dual basis $({\vf{e}_1}^*,\ldots,{\vf{e}_n}^*)$ and $({\vf{v}_1}^*,\ldots,{\vf{v}_n}^*)$, respectively. Suppose $\vf{v}_i=\sum_{j=1}^na_{ij}\vf{e}_i$ for $i=1,\ldots,n$ and define $\vf{A}:=(a_{ij})\in\mathcal{M}_n(\RR)$. Then: $${\vf{v}_1}^*\wedge\cdots\wedge{\vf{v}_n}^*=\frac{1}{\det\vf{A}}{\vf{e}_1}^*\wedge\cdots\wedge{\vf{e}_n}^*$$
  \end{corollary}
  \begin{proposition}
    Let $\vf{v}_1,\ldots,\vf{v}_n\in\RR^n$ and $P$ be the parallelepiped they generate. Then:
    $$\vol P=\abs{\det(\vf{v}_1,\ldots,\vf{v}_n)}$$
  \end{proposition}
  \subsubsection{Differential forms}
  \begin{definition}
    Let $U\subseteq\RR^n$ be an open set. A \emph{differential $k$-form} on $U$ is a differentiable function $\omega:U\rightarrow\Lambda^k{(\RR^n)}^*\cong\RR^{\binom{n}{k}}$\footnote{An element $\omega(p)=:\omega_p$ can be though as an element of $\Lambda^k{(T_p\RR^n)}^*$.}. We will denote the dual basis of $\left(\pdv{}{x^1},\ldots,\pdv{}{x^n}\right)$ by $(\dd{x^1},\ldots,\dd{x^n})$. Thus, a differential $k$-form can be written as:
    $$\omega=\sum_{j_1<\cdots<j_k}\omega_{j_1,\ldots,j_k}\dd{x^{j_1}}\wedge\cdots\wedge\dd{x^{j_k}}$$
    We denote by $\Omega^k(U)$ the set of all differential $k$-forms defined on $U$ with the agreement that $\Omega^0(U):=\mathcal{C}^\infty(U)$.
  \end{definition}
  \begin{proposition}
    Let $U\subseteq\RR^n$ be an open set. Then, $\Omega^k(U)$ is a $\RR$-vector space.
  \end{proposition}
  \begin{definition}
    Let $U\subseteq\RR^n$ be an open set and $\omega\in \Omega^k(U)$. This form $\omega$ defines a $\mathcal{C}^\infty(U)$-multilinear alternating function, which we'll denote also by $\omega$, given by: $$\function{\omega}{\mathcal{X}(U)\times\overset{(k)}{\cdots}\times\mathcal{X}(U)}{\mathcal{C}^\infty(U)}{(\vf{X}_1,\ldots,\vf{X}_k)}{\omega(\vf{X}_1,\ldots,\vf{X}_k)}$$ where $\omega(\vf{X}_1,\ldots,\vf{X}_k)$ is the function defined by: $$\omega(\vf{X}_1,\ldots,\vf{X}_k)(p)=\omega_p({(\vf{X}_1)}_p,\ldots,{(X_k)}_p)$$
  \end{definition}
  \begin{definition}
    Let $U\subseteq\RR^n$ be an open set and $h\in\mathcal{C}^\infty$. We define its \emph{differential} as the differential 1-form $\dd{h}$ given by: $$\dd{h}=\sum_{i=1}^n\pdv{h}{x^i}\dd{x^i}$$
  \end{definition}
  \begin{proposition}
    Let $U\subseteq\RR^n$ be an open set, $h\in\mathcal{C}^\infty(U)$ and $\vf{X}\in\mathcal{X}(U)$. Then: $$\dd{h}(\vf{X})=\vf{X}h\footnote{In particular, note that $\dd{x^i}\left(\pdv{}{x^j}\right)=\delta_{ij}$.}$$
  \end{proposition}
  \begin{definition}
    Let $U\subseteq\RR^n$, $V\subseteq\RR^m$ be open sets, $\vf{f}:U\rightarrow V$ be a differentiable function and $k\geq 1$. We define the \emph{pull-back} by $\vf{f}$ as the linear function $$\vf{f^*}:\Omega^k(V)\rightarrow\Omega^k(U)$$ defined by $$(\vf{f^*}\omega)(\vf{X}_1,\ldots,\vf{X}_k)(p)=\omega_{f(p)}(\vf{df}_p(\vf{X}_1),\ldots,\vf{df}_p(\vf{X}_k))$$
    If $k=0$, we define \emph{pull-back} by $\vf{f}$ as $\vf{f^*}h=h\circ\vf{f}$.
  \end{definition}
  \begin{proposition}
    Let $U\subseteq\RR^n$, $V\subseteq\RR^m$, $W\subseteq\RR^r$ be open sets, $\vf{f}:U\rightarrow V$, $\vf{g}:V\rightarrow W$ be differentiable functions, $\omega,\eta\in\Omega^k(V)$, $h\in\Omega^0(V)$ and $a,b\in\RR$. Then:
    \begin{enumerate}
      \item $\vf{f^*}(a\omega+b\eta)=a\vf{f^*}\omega+b\vf{f^*}\eta$
      \item $\vf{f^*}(\omega\wedge\eta)=\vf{f^*}\omega\wedge \vf{f^*}\eta$
      \item ${(\vf{g}\circ \vf{f})}^{\vf{*}}=\vf{f^*}\circ \vf{g^*}$
      \item $\vf{f^*}\dd{h}=\dd{(h\circ \vf{f^*})}=\dd{(\vf{f^*}h)}$
    \end{enumerate}
  \end{proposition}
  \begin{corollary}
    Let $U\subseteq\RR^n$, $V\subseteq\RR^m$ be open sets, $\vf{f}=(f^1,\ldots,f^m):U\rightarrow V$ be a differentiable function, $\omega,\eta\in\Omega^k(V)$ and $h\in\Omega^0(V)$. Then:
    \begin{enumerate}
      \item $\vf{f^*}(h\omega)=(h\circ \vf{f})\vf{f^*}\omega$
      \item $\vf{f^*}\dd{y^i}=\dd{(y^j\circ \vf{f})}=\dd{f^j}$
      \item If $\omega=\sum_{j_1<\cdots<j_k}\omega_{j_1,\ldots,j_k}\dd{y^{j_1}}\wedge\cdots\wedge\dd{y^{j_k}}$, then: $$\vf{f^*}\omega=\sum_{j_1<\cdots<j_k}(\omega_{j_1,\ldots,j_k}\circ \vf{f})\dd{f^{j_1}}\wedge\cdots\wedge\dd{f^{j_k}}$$
    \end{enumerate}
  \end{corollary}
  \begin{definition}
    Let $U\subseteq\RR^n$ be an open set, $\vf{X}\in\mathcal{X}(U)$ and $\omega\in\Omega^k(U)$, where $k\geq 1$.
    We define the \emph{interior product} of $\omega$ by $\vf{X}$ as the differential $(k-1)$-form $\iota_{\vf{X}}\omega\in\Omega^{k-1}(U)$ defined by:
    $$\iota_{\vf{X}}\omega(\vf{Y}_2,\ldots,\vf{Y}_k)=\omega(\vf{X},\vf{Y}_2,\ldots,\vf{Y}_k)$$
    By agreement, we define $\iota_{\vf{X}}h=0$ if $h\in\Omega^0(U)$.
  \end{definition}
  \begin{proposition}
    Let $U\subseteq\RR^n$ be an open set, $\vf{X}\in\mathcal{X}(U)$, $\alpha\in\Omega^k(U)$, and $\beta\in\Omega^m(U)$. Then: $$\iota_{\vf{X}}(\alpha\wedge\beta)=(\iota_{\vf{X}}\alpha)\wedge\beta+{(-1)}^k\alpha\wedge(\iota_{\vf{X}}\beta)$$
  \end{proposition}
  \begin{definition}
    The differential $n$-form $\eta\in\Omega^n(\RR^n)$ defined by $$\eta=\dd{x^1}\wedge\cdots\wedge\dd{x^n}$$ is called \emph{volume element} of $\RR^n$.
  \end{definition}
  \begin{lemma}
    Let $(\vf{u}_1,\ldots,\vf{u}_n)$ be an orthonormal positive basis of $T_p\RR^n$, $p\in\RR^n$, and let $\eta$ be the volume element. Then: $$\eta_p(\vf{u}_1,\ldots,\vf{u}_n)=1$$
  \end{lemma}
  \begin{proposition}
    Let $U,V\subseteq\RR^n$ be open sets and $\vf{f}:U\rightarrow V$ be a differentiable function. Then: $$\vf{f^*}\eta=J\vf{f}\cdot\eta$$ where $J\vf{f}$ is the Jacobian of $\vf{f}$.
  \end{proposition}
  \begin{definition}
    Let $U\subseteq\RR^n$ be an open set and $k\geq 0$. We define the \emph{exterior differential} as the linear function $\dd:\Omega^k(U)\rightarrow\Omega^{k+1}(U)$ defined as follows: If $$\omega=\sum_{j_1<\cdots<j_k}\omega_{j_1,\ldots,j_k}\dd{x^{j_1}}\wedge\cdots\wedge\dd{x^{j_k}}$$
    then: $$\dd{\omega}:=\sum_{m,j_1<\cdots<j_k}\pdv{\omega_{j_1,\ldots,j_k}}{x^m}\dd{x^m}\wedge\dd{x^{j_1}}\wedge\cdots\wedge\dd{x^{j_k}}$$
  \end{definition}
  \begin{proposition}
    Let $U,V\subseteq\RR^n$ be open sets, $\vf{f}:U\rightarrow V$ be a differentiable function and $\alpha,\beta\in\Omega^k(U)$. Then:
    \begin{enumerate}
      \item $\dd{(\alpha\wedge\beta)}=\dd{\alpha}\wedge\beta+{(-1)}^k\alpha\wedge\dd{\beta}$
      \item $\dd^2=0$
      \item $\vf{f^*}\circ\dd{}=\dd{}\circ\vf{f^*}$
    \end{enumerate}
  \end{proposition}
  \begin{definition}
    Let $U\subseteq\RR^n$ be an open set and $\omega\in\Omega^k(U)$. We say that $\omega$ is \emph{closed} if $\dd{\omega}=0$.
  \end{definition}
  \begin{definition}
    Let $U\subseteq\RR^n$ be an open set and $\omega\in\Omega^k(U)$. We say that $\omega$ is \emph{exact} if $\exists \tilde{\omega}\in\Omega^{k-1}(U)$ such that $\dd{\tilde{\omega}}=\omega$.
  \end{definition}
  \subsection{Integration}
  \subsubsection{Submanifolds of \texorpdfstring{$\RR^n$}{Rn}}
  \begin{definition}
    Let $M\subseteq\RR^n$ be a submanifold and $p\in M$. If $\vf\alpha:(-\varepsilon,\varepsilon)\rightarrow\RR^3$ is a parametrization of a curve of class $\mathcal{C}^\infty$ such that $\vf\alpha(0)=p$, we say that $\vf\alpha'(0)$ is a \emph{tangent vector} to $M$ at $p$. The set of all such vectors is called \emph{tangent space} to $M$ at $p$ and it is denoted as $T_pM$. Moreover, $T_pM$ is a vector space of dimension $\dim M$.
  \end{definition}
  \begin{definition}
    Let $M\subseteq\RR^n$ be a submanifold of dimension $k$ and $U\subseteq S$ be an open set. A \emph{vector field} defined on $U$ is a correspondence $\vf{X}$ that at each point $p\in U$ it assigns a tangent vector $\vf{X}(p)=:\vf{X}_p\in T_p\RR^n$. We say that $\vf{X}$ is \emph{differentiable} at $p\in U$ if there is a parametrization $\vf\varphi(u^1,\ldots,u^k)$ of $M$ whose image contains $p$ such that $$\vf{X}=\sum X^i\vf\varphi_{u^i}$$ for some functions $X^1,\ldots,X^k$ differentiable at $p$. We say that $\vf{X}$ is \emph{differentiable} if it is differentiable at each point $p\in U$. We say that $\vf{X}$ is \emph{tangent} to $M$ if $\vf{X}_p\in T_pM$ $\forall p\in U$. We denote by $\mathcal{X}(U)$ the set of all differentiable vector fields on $U$ that are tangent to $M$.
  \end{definition}
  \begin{definition}
    Let $M\subseteq\RR^n$ be a submanifold of dimension $k$ and $U\subseteq M$ be an open set and $(V,\vf\varphi(u^1,\ldots,u^k))$ be a local parametrization of $M$ with $\vf\varphi(V)=U$. A \emph{differential $\ell$-form} on $U$ is a differentiable function $\omega:U\rightarrow\Lambda^k{(T_pM)}^*\cong\RR^{\binom{k}{\ell}}$. We denote by $\Omega^\ell(U)$ the set of all differential $\ell$-forms defined on $U$ with the agreement that $\Omega^0(U):=\mathcal{C}^\infty(U)$\footnote{All the definitions made in the previous section about differential forms can be applied, conveniently modifed, to submanifolds.}. We will denote the dual basis of $\left(\pdv{}{u^1},\ldots,\pdv{}{u^k}\right):=\left(\pdv{\vf\varphi}{u^1},\ldots,\pdv{\vf\varphi}{u^k}\right)$ by $(\dd{u^1},\ldots,\dd{u^k})$. Thus, a differential $k$-form can be written uniquely as:
    $$\omega=\sum_{j_1<\cdots<j_k}\omega_{j_1,\ldots,j_\ell}\dd{u^{j_1}}\wedge\cdots\wedge\dd{u^{j_\ell}}$$
  \end{definition}
  \begin{definition}
    Let $M\subseteq\RR^n$ be a submanifold of dimension $k$ and $U\subseteq M$ be an open set, $(V,\vf\varphi(u^1,\ldots,u^k))$ be a local parametrization of $M$ with $\vf\varphi(V)=U$ and $\omega\in\Omega^\ell(U)$. We define the \emph{exterior differential} as the linear function $\dd:\Omega^k(U)\rightarrow\Omega^{\ell+1}(U)$ defined as follows: if $$\omega=\sum_{j_1<\cdots<j_\ell}\omega_{j_1,\ldots,j_\ell}\dd{x^{j_1}}\wedge\cdots\wedge\dd{x^{j_\ell}}$$
    then: $$\dd{\omega}=\sum_{m,j_1<\cdots<j_\ell}\pdv{\omega_{j_1,\ldots,j_\ell}}{x^m}\dd{x^m}\wedge\dd{x^{j_1}}\wedge\cdots\wedge\dd{x^{j_\ell}}$$
  \end{definition}
  \begin{proposition}
    Let $M\subseteq\RR^n$ be a submanifold of dimension $k$ and $U\subseteq \RR^n$ be an open set and $\omega\in\Omega^\ell(U)$. Then, $\omega$ induces a differential form $\omega_M\in\Omega^\ell(V)$, where $V=U\cap M$, defined by: $$\omega_M(\vf{X}_1,\ldots,\vf{X}_\ell)(p)=\omega_p({(\vf{X}_1)}_p,\ldots,{(\vf{X}_\ell)}_p)$$
    for all $p\in V$ and all $\vf{X}_i\in\mathcal{X}(V)$.
    The expression in the coordinates $u^1,\ldots,u^k$ from a parametrization $\vf\varphi$ is: $\omega_M=\vf{\varphi^*}\omega$.
  \end{proposition}
  \subsubsection{Manifolds with boundary}
  \begin{definition}
    Let $k\in\NN$. We define the set $\HH^k$ as: $$\HH^k:=\{(x^1,\ldots,x^k)\in\RR^k:x^k\geq 0\}$$
    Note that $\Fr{\HH^k}=\{(x^1,\ldots,x^k)\in\RR^k:x^k=0\}$.
  \end{definition}
  \begin{definition}
    Let $U\subseteq \HH^k$ be an open set and $\vf{f}:U\rightarrow\RR^m$ be a function. We say that $\vf{f}$ is \emph{differentiable} if $\forall p\in U$ there exists a neighbourhood $W\subseteq \RR^k$ of $p$ and a differentiable function $\vf{\tilde{f}}:W\rightarrow\RR^m$ such that $\vf{\tilde{f}}|_{V\cap W}=\vf{f}|_{V\cap W}$. In this case, we define the \emph{differential} of $\vf{f}$ at a point $p\in U$ as $\vf{df}_p=\vf{d\tilde{f}}_p$.
  \end{definition}
  \begin{definition}
    Let $M\subseteq\RR^n$ be a set. We say that $M$ is a \emph{submanifold with boundary} of dimension $k$ if $\forall p\in M$ there is an open neighbourhood $U\subseteq\RR^n$ of $p$, an open set $V\subset \HH^k$ and a differentiable function $\vf\varphi:V\rightarrow\RR^n$ such that:
    \begin{enumerate}
      \item $\vf\varphi(V)=U\cap M$ and $\vf\varphi:V\rightarrow U\cap M$ is a homeomorphism.
      \item $\vf\varphi$ is an immersion.
    \end{enumerate}
    In these conditions, the pair $(V,\vf\varphi)$ is called local parametrization of $M$.
  \end{definition}
  \begin{proposition}
    Let $M\subseteq\RR^n$ be a submanifold with boundary of dimension $k$, $(V,\vf\varphi)$ be a local parametrization of it, $W\subseteq \RR^m$ be an open set and $\vf{f}:W\rightarrow\RR^n$ be a differentiable function such that $\vf{f}(W)\subseteq M$. Then, ${\vf\varphi}^{-1}\circ\vf{f}$ is differentiable.
  \end{proposition}
  \begin{lemma}
    Let $M\subseteq\RR^n$ be a submanifold with boundary of dimension $k$, $(V,\vf\varphi)$ be a local parametrization of it, $x\in V\cap \Fr{\HH^k}$ and $p:=\vf\varphi(x)$. Then the set $$\{\vf\alpha'(0):\vf\alpha:[0,\varepsilon)\rightarrow M\text{ is differentiable with }\vf\alpha(0)=p\}$$ is equal to the image $\vf{\dd}\vf\varphi_x(\HH^k)\subseteq T_p\RR^n$.
  \end{lemma}
  \begin{definition}
    Let $M\subseteq\RR^n$ be a submanifold with boundary of dimension $k$. We say that $p\in M$ is an \emph{interior point} of $M$ is there exists a local parametrization $(V,\vf\varphi)$ of $M$ such that $p\in\vf\varphi(V\setminus\Fr{\HH^k})$. We call \emph{boundary} of $M$ the set: $$\Fr{M}=M\setminus\{p\in M:p\text{ is interior}\}$$
  \end{definition}
  \begin{proposition}
    Let $M\subseteq\RR^n$ be a submanifold with boundary of dimension $k$. Then, $\Fr{M}$ is a manifold (without boundary) of dimension $k-1$.
  \end{proposition}
  \begin{proposition}
    Let $M\subseteq\RR^n$ be a set. $M$ is a submanifold with boundary of dimension $n$ if and only if $\forall p\in M$ there exists an open neighbourhood $U\subseteq \RR^n$ of $p$ and a function $F:U\rightarrow\RR$ such that:
    \begin{enumerate}
      \item $U\cap M=\{x\in U:F(x)\leq 0\}$
      \item $F$ is a submersion on the points of $F^{-1}(0)$.
    \end{enumerate}
  \end{proposition}
  \begin{definition}
    Let $M\subseteq\RR^n$ be a submanifold with boundary of dimension $k$, $p\in M$ and $(V,\vf\varphi(u^1,\ldots,u^k))$ be a local parametrization of $M$ such that $p\in\vf\varphi(V)$. We define the \emph{tangent space} of $M$ at $p$ as the following vector subspace $T_pM$ of $T_p\RR^n$: $$T_pM:=\left\langle{\left(\vf\varphi_{u^1}\right)}_p,\ldots,{\left(\vf\varphi_{u^k}\right)}_p\right\rangle$$
  \end{definition}
  \begin{definition}
    Let $M\subseteq\RR^n$ be a submanifold with boundary of dimension $k$, $p\in \Fr{M}$. We define the \emph{set of interior vectors} $T_p^\text{i}M$ of $T_pM$ as the set:
    $$\{\vf\alpha'(0):\vf\alpha:[0,\varepsilon)\rightarrow M\text{ is differentiable with }\vf\alpha(0)=p\}$$
    Note that $T_p^\text{i}M$ is a closed subspace of $T_pM$ and its boundary is $T_p\Fr{M}\cong\RR^{k-1}$.
    The elements of $T_pM\setminus T_p^\text{i}M$ are called \emph{exterior vectors}.
  \end{definition}
  \begin{lemma}
    Let $M\subseteq\RR^n$ be a submanifold with boundary of dimension $k$, $p\in M$ and $(V,\vf\varphi(u^1,\ldots,u^k))$ be a local parametrization of $M$ such that $p\in\vf\varphi(V)$. A vector $\vf{w}\in T_pM$ can be written as: $$\vf{w}=a^1{\left(\pdv{\vf\varphi}{u^1}\right)}_p+\cdots+a^k{\left(\pdv{\vf\varphi}{u^k}\right)}_p$$ Then:
    $$
      \begin{cases}
        \vf{w}\text{ is interior}\iff a^k\geq 0 \\
        \vf{w}\text{ is exterior}\iff a^k< 0    \\
      \end{cases}
    $$
  \end{lemma}
  \subsubsection{Oritentability}
  \begin{definition}
    Let $M\subseteq\RR^n$ be a submanifold\footnote{From now on the submanifolds may be with boundary or not.} of dimension $k$. We say that $M$ is \emph{orientable} if we can assign an orientation to each tangent space $T_pM$ of $M$ in a way that $\forall p\in M$, there is a local parametrization $(V,\vf\varphi)$ of $M$ such that $p\in\vf\varphi(V)$ and $\vf{d\varphi}_x:T_x\RR^k\rightarrow T_{\vf\varphi(x)}M$ is positive-oriented (or negative-oriented). In these conditions, we say that $(V,\vf\varphi)$ is \emph{compatible} with the orientation of $M$.
  \end{definition}
  \begin{proposition}
    Let $M\subseteq\RR^n$ be a submanifold of dimension $k$. $M$ is orientation if and only if there exists an atlas $\{(V_\alpha,\vf\varphi_\alpha):\alpha\in A\}$ such that $\forall \alpha,\beta\in A$ we have: $$J({\vf\varphi_\beta}^{-1}\circ\vf\varphi_\alpha)>0$$
  \end{proposition}
  \begin{proposition}
    Let $M\subseteq\RR^n$ be an orientated submanifold of dimension $k$. Then, there exists a unique differential $k$-form $\eta_M\in\Omega^k(M)$ such that if $(\vf{e}_1,\ldots,\vf{e}_k)$ is a orthonormal positive basis of $T_pM$, then $\eta_M(\vf{e}_1,\ldots,\vf{e}_k)=1$. This form $\eta_M$ is called \emph{volume element} of $M$.
  \end{proposition}
  \begin{proposition}
    Let $M\subseteq\RR^n$ be a submanifold of dimension $k$. $M$ is orientable if and only if there exists a differential $k$-form $\eta\in\Omega^k(M)$ that isn't zero at each point of $M$
  \end{proposition}
  \begin{lemma}
    Let $M\subseteq\RR^n$ be a submanifold of dimension $k$ with $\Fr{M}\ne\varnothing$. Then, there is a global exterior vector filed $\vf\nu$ defined on $\Fr{M}$, that is, a vector field such that $\forall p\in\Fr{M}$, $\vf\nu_p\notin T_p^\text{i}M$ (in particular $\vf\nu_p\ne 0$).
  \end{lemma}
  \begin{definition}
    Let $M\subseteq\RR^n$ be a submanifold of dimension $k$ with $\Fr{M}\ne\varnothing$. We call \emph{unit normal exterior vector field} the field $\vf\nu_{\Fr{M}}$ univocally determined by being unit, exterior and perpendicular to $T_p\Fr{M}$ at each point $p\in M$.
  \end{definition}
  \begin{proposition}
    Let $M\subseteq\RR^n$ be a submanifold with boundary of dimension $k$. If $M$ is orientable, so it is $\Fr{M}$.
  \end{proposition}
  \begin{definition}
    Let $M\subseteq\RR^n$ be an orientated submanifold of dimension $k$ with $\Fr{M}\ne\varnothing$. We say that a basis $(\vf{e}_1,\ldots,\vf{e}_{k-1})$ of $T_p\Fr{M}$ is \emph{positive} if $(\vf\nu_{\Fr{M}},\vf{e}_1,\ldots,\vf{e}_{k-1})$ is a positive basis of $T_pM$. This choice determines an orientation on $\Fr{M}$, which is called \emph{orientation induced by $M$}.
  \end{definition}
  \begin{proposition}
    Let $M\subseteq\RR^n$ be an orientated submanifold of dimension $k$ with $\Fr{M}\ne\varnothing$, $\eta_M$ be the volume element of $M$ and $\vf\nu_{\Fr{M}}$ be the unit normal exterior vector field. Then the the volume element of $\Fr{M}$ associated with the orientation of $\Fr{M}$ induced by the one of $M$ is: $$\eta_{\Fr{M}}=\iota_{\vf\nu}\eta_M$$
  \end{proposition}
  \begin{proposition}
    Let $S\subseteq\RR^3$ be a regular surface oriented with a vector field $\vf\nu_S$. Then, the area element of $S$ is given by $\eta_S=\iota_{\vf\nu}\eta$, where $\eta=\dd{x}\wedge\dd{y}\wedge\dd{z}$. Moreover if $\vf\varphi(u,v)$ is a local parametrization of $S$ compatible with the orientation, then: $$\eta_S=\vf\varphi^*\eta_S=\sqrt{E_{\vf\varphi}G_{\vf\varphi}-{F_{\vf\varphi}}^2}\dd{u}\wedge\dd{v}$$
  \end{proposition}
  \subsubsection{Integration of differential forms}
  \begin{definition}
    Let $M\subseteq\RR^n$ be an orientated submanifold of dimension $k$ and $\omega\in\Omega^\ell(M)$ We define the support of $\omega$ as: $$\supp (\omega):=\overline{\{p\in M:\omega_p\ne 0\}}$$
  \end{definition}
  \begin{definition}
    Let $U\subseteq \RR^k$ be an open set and $\omega=h\dd{u^1}\wedge\cdots\wedge\dd{u^k}\in\Omega^k(\RR^k)$ where $h=h(u^1,\ldots,u^k)$ and such that $\supp(\omega)\subset U$ is compact. We define the \emph{integral} of $\omega$ on $U$ as: $$\int_U\omega=\int_Uh\dd{u^1}\wedge\cdots\wedge\dd{u^k}:=\int_Uh\dd{u^1}\cdots\dd{u^k}$$
  \end{definition}
  \begin{definition}
    Let $M\subseteq\RR^n$ be an orientated submanifold of dimension $k$, $(U,\vf\varphi(u^1,\ldots,u^k))$ be a local parametrization of $M$ compatible with the orientation and $\omega\in\Omega^k(M)$ be such that $\supp(\omega)\subset \vf\varphi(U)$ is compact. We define the \emph{integral} of $\omega$ on $M$ as: $$\int_M\omega:=\int_U\vf\varphi^*\omega=\int_U\omega\left(\pdv{\vf\varphi}{u^1},\ldots,\pdv{\vf\varphi}{u^k}\right)\dd{u^1}\cdots\dd{u^k}\footnote{It can be seen that this definition does not depend on the parametrization $(U,\vf\varphi(u^1,\ldots,u^k))$.}$$
    If $h:U\rightarrow M$ is a differentiable function and $\supp(h)\subset \vf\varphi(U)$, then we define the \emph{integral} of $h$ on $M$ as:
    $$\int_Mh:=\int_Mh\eta_M=\int_U(h\circ\omega)\vf\varpi^*\eta_M$$
    where $\eta_M$ is the volume element of $M$.
  \end{definition}
  \begin{definition}
    Let $M\subseteq\RR^n$ be an orientated submanifold of dimension $k$, $(U,\vf\varphi(u^1,\ldots,u^k))$ be a local parametrization of $M$ compatible with the orientation and $R\subset M$ be a compact region contained in $\vf\varphi(U)$. Let $Q:=\vf\varphi^{-1}(R)\subset U$. We define the volume of $R$ as: $$\vol(R)=\int_R\eta_M:=\int_Q\vf\varphi^*\eta_M$$
  \end{definition}
  \begin{proposition}
    Let $K\subseteq \RR^n$ be a compact set, $\{V_\alpha:\alpha\in A\}$ be an open cover of $K$. Then, there exist differentiable and non-negative functions $\rho_1,\ldots,\rho_m\in\mathcal{C}^\infty(\RR^n)$ such that:
    \begin{enumerate}
      \item $\sum_{i=1}^m\rho_i(x)=1$ $\forall x\in K$
      \item For all $i\in\{1,\ldots,m\}$, $\exists \alpha\in A$ such that $\supp(\rho_i)\subset V_\alpha$
    \end{enumerate}
    In these conditions the set $\{\rho_i:i=1,\ldots,m\}$ is called a \emph{partition of unity} of $K$ subordinated to $\{V_\alpha:\alpha\in A\}$.
  \end{proposition}
  \begin{definition}
    Let $k\geq 1$ and $M\subseteq\RR^n$ be an orientated submanifold. We define the set $\Omega_\text{c}^k(M)$ as the vector space of all differential $k$-forms of $M$ with compact support.
  \end{definition}
  \begin{definition}
    Let $M\subseteq\RR^n$ be an orientated submanifold of dimension $k$, $\{(V_\alpha,\vf\varphi_\alpha):\alpha\in A\}$ and an atlas of $M$ compatible with the orientation. Given $\omega\in\Omega_\text{c}^k(M)$, let $\{\rho_1,\ldots,\rho_m\}$ be a partition of unity of $K$ subordinated to $\{\vf\varphi(V_\alpha)\}$. We define the \emph{integral} of $\omega$ on $M$ as:
    $$\int_M\omega=\sum_{i= 1}^m\int_M\rho_i\omega=\sum_{i= 1}^m\int_{U_\alpha}{\vf\varphi_\alpha}^*(\rho_i\omega)$$
    where $\alpha\in A$ is such that $\supp(\rho_i)\subset \vf\varphi(V_\alpha)$\footnote{It can be seen that this definition doesn't depend neither on the atlas nor on the partition of unity chosen.}.
  \end{definition}
  \begin{proposition}
    Let $M\subseteq\RR^n$ be an orientated submanifold of dimension $k$, $\{(V_i,\vf\varphi_i):i=1,\ldots,m\}$ be a finite set of parametrizations of $M$ compatibles with the orientation and such that:
    \begin{enumerate}
      \item $M\setminus\bigcup_{i= 1}^m\vf\varphi_i(U_i)$ is union of submanifolds of dimension less than $<k$.
      \item $\vf\varphi_i(U_i)\cap\vf\varphi_j(U_j)=\varnothing$ $\forall i\ne j$.
    \end{enumerate}
    Then, given $\omega\in \Omega_\text{c}^k(M)$, we have: $$\int_M\omega=\sum_{i= 1}^m\int_{U_i}{\vf\varphi_i}^*\omega$$
  \end{proposition}
  \begin{theorem}[Change of variables]
    Let $M\subseteq\RR^n$, $M'\subseteq \RR^m$ be two orientated submanifolds of dimension $k$, $\vf{F}:M\rightarrow M'$ be a orientation-preserving diffeomorphism and $\omega\in \Omega_\text{c}^k(M')$. Then: $$\int_{M'}\omega=\int_M\vf{F}^*\omega$$
  \end{theorem}
  \subsubsection{Stokes theorem}
  \begin{proposition}
    Let $\omega\in\Omega_\text{c}^{k-1}(\HH^k)$ and consider $\Fr{\HH^k}$ with the orientation induced by the one of $\HH^k$. Then:
    \begin{enumerate}
      \item If $\supp{\omega}\cap \Fr{\HH^k}=\varnothing$, then $\displaystyle\int_{\HH^k}\dd{\omega}=0$.
      \item If $\supp{\omega}\cap \Fr{\HH^k}\ne\varnothing$, then $\displaystyle\int_{\HH^k}\dd{\omega}=\int_{\Fr{\HH^k}}\omega$.
    \end{enumerate}
  \end{proposition}
  \begin{theorem}[Stokes theorem]
    Let $M\subseteq\RR^n$ be an orientated submanifold of dimension $k$ and $\omega\in\Omega_\text{c}^{k-1}(M)$. Then: $$\int_M\dd{\omega}=\int_{\Fr{M}}\omega$$
  \end{theorem}
  \begin{corollary}
    Let $M\subseteq\RR^n$ be an orientated submanifold of dimension $k$ with $\Fr{M}=\varnothing$ and $\omega\in\Omega_\text{c}^{k-1}(M)$. Then: $$\int_M\dd{\omega}=0$$
  \end{corollary}
  \begin{corollary}[Green's formula]
    Let $D\subseteq\RR^2$ be a regular domain (manifold of dimension 2 with boundary) and $P,Q:D\rightarrow\RR$ be differentiable functions. Then: $$\iint_D\left(\pdv{Q}{x}-\pdv{P}{y}\right)\dd{x}\dd{y}=\int_{\Fr{D}}P\dd{x}+Q\dd{y}$$
  \end{corollary}
  \subsubsection{Vector calculus}
  \begin{definition}
    Let $U\subseteq\RR^3$, $f\in\mathcal{C}^\infty(U)$ and $\vf{X}\in\mathcal{X}(U)$. We define the following differential forms on $U$:
    \begin{align*}
      \omega_{\vf{X}}^1 & =X^1\dd{x}+X^2\dd{y}+X^3\dd{z}                                     \\
      \omega_{\vf{X}}^2 & =X^1\dd{y}\wedge\dd{z}+X^2\dd{z}\wedge\dd{z}+X^3\dd{x}\wedge\dd{y} \\
      \omega_{f}^3      & =f\dd{x}\wedge\dd{y}\wedge\dd{z}
    \end{align*}
  \end{definition}
  \begin{lemma}
    Let $U\subseteq\RR^3$, $f\in\mathcal{C}^\infty(U)$ and $\vf{X},\vf{Y},\vf{Z}\in\mathcal{X}(U)$. Then:
    \begin{enumerate}
      \item $\omega_{\vf{X}}^1(\vf{Y})=\langle\vf{X},\vf{Y}\rangle$
      \item $\omega_{\vf{X}}^2(\vf{Y},\vf{Z})=\langle\vf{X},\vf{Y}\times\vf{Z}\rangle=\det(\vf{X},\vf{Y},\vf{Z})$
      \item $\omega_{f}^3(\vf{X},\vf{Y},\vf{Z})=f\det(\vf{X},\vf{Y},\vf{Z})$
    \end{enumerate}
  \end{lemma}
  \begin{proposition}
    Let $U\subseteq\RR^3$, $f\in\mathcal{C}^\infty(U)$ and $\vf{X}\in\mathcal{X}(U)$. Then:
    $$\dd{f}=\omega_{\grad{f}}^1\qquad\dd{\omega_{\vf{X}}^1}=\omega_{\rot{\vf{X}}}^2\qquad\dd{\omega_{\vf{X}}^2}=\omega_{\div{\vf{X}}}^3$$
  \end{proposition}
  \begin{corollary}
    Let $U\subseteq\RR^3$, $f\in\mathcal{C}^\infty(U)$ and $\vf{X}\in\mathcal{X}(U)$. Then:
    \begin{enumerate}
      \item $\rot(\grad f)=0$
      \item $\div(\rot\vf{X})=0$
    \end{enumerate}
  \end{corollary}
\end{multicols}
\end{document}