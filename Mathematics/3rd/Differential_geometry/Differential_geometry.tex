\documentclass[../../../main.tex]{subfiles}

\begin{document}
\begin{multicols}{2}[\section{Differential geometry}]
  \subsection{Differentiable curves}
  \subsubsection{Inner product of \texorpdfstring{$\RR^n$}{Rn}}
  \begin{proposition}
    Let $\vf{u},\vf{v}\in\RR^n$ be vectors and $\langle\vf{u},\vf{v}\rangle$ be the usual inner product between $\vf{u}$ and $\vf{v}$ in $\RR^n$. Then:
    \begin{itemize}
      \item Cauchy-Schwarz inequality: $$|\langle\vf{u},\vf{v}\rangle|\leq\|\vf{u}\|\|\vf{v}\|$$
      \item Triangular inequality: $$\|\vf{u}+\vf{v}\|\leq\|\vf{u}\|+\|\vf{v}\|$$
      \item Polarization identity: $$\langle\vf{u},\vf{v}\rangle=\frac{1}{2} \left(\|\vf{u}+\vf{v}\|^2-\|\vf{u}\|^2 - \|\vf{v}\|^2\right)$$
    \end{itemize}
  \end{proposition}
  \begin{definition}
    Let $\vf{u},\vf{v}\in\RR^n$ be vectors. We define the angle between $\vf{u}$ and $\vf{v}$ as the unique value $\theta\in[0,\pi]$ such that: $$\cos\theta=\frac{\langle\vf{u},\vf{v}\rangle}{\|\vf{u}\|\|\vf{v}\|}$$
  \end{definition}
  \subsubsection{Parametrized curves}
  \begin{definition}
    Let $U\subseteq\RR^n$ be an open set and $\vf{f}:U\rightarrow\RR^n$ be a differentiable function. We say that $\vf{f}$ is a \emph{local diffeomorphism} if $\forall p\in U$, there exists a neighborhood $V\subseteq U$ of $p$ such that $\vf{f}|_V:V\rightarrow \vf{f}(V)$ is a diffeomorphism.
  \end{definition}
  \begin{proposition}
    Let $I\subseteq\RR$ be an open interval and $f:I\rightarrow\RR$ be a differentiable function. If $f'(x)\ne 0$ $\forall x\in I$, then $f(I)$ is an open set and $f$ is a diffeomorphism.
  \end{proposition}
  \begin{proposition}
    Let $I\subseteq\RR$ be an open interval and $\vf{\alpha},\vf{\beta}:I\rightarrow\RR^n$ be differentiable functions. Then:
    \begin{enumerate}
      \item $\dv{}{t}\langle\vf{\alpha}(t),\vf{\beta}(t)\rangle=\langle\vf{\alpha}'(t),\vf{\beta}(t)\rangle+\langle\vf{\alpha}(t),\vf{\beta}'(t)\rangle$
      \item If $t\mapsto\|\vf{\alpha}(t)\|$ is a constant function, then $\vf{\alpha}\perp\vf{\alpha}'$.
    \end{enumerate}
  \end{proposition}
  \begin{definition}
    Let $I\subseteq\RR$ be an open interval. A \emph{parametrized curve of class $\mathcal{C}^k$} in $\RR^n$ is a function $\vf{\alpha}:I\rightarrow\RR^n$ of class $\mathcal{C}^k$ .
  \end{definition}
  \begin{definition}
    Let $\vf{\alpha}:I\rightarrow\RR^n$ be a parametrized curve of class $\mathcal{C}^1$. We define the \emph{tangent vector} of $\vf{\alpha}$ at $t_0\in\RR$ as $\vf{\alpha}'(t_0)$. We say that $\vf{\alpha}$ is \emph{regular} if $\vf{\alpha}'(t)\ne 0$ $\forall t\in I$. In that last case, we define the \emph{tangent line} of $\vf{\alpha}$ at $\vf{\alpha}(t_0)$ as the following parametrized line in $\RR^n$: $$s\longmapsto \vf{\alpha}(t_0)+s\vf{\alpha}'(t_0)$$
  \end{definition}
  \begin{definition}
    Let $\vf{\alpha}:I\rightarrow\RR^n$ be a parametrized curve of class $\mathcal{C}^1$. We say that $\vf{\alpha}$ is a \emph{plane curve} is it is contained in a plane of $\RR^n$.
  \end{definition}
  \begin{definition}
    Let $\vf{\alpha}:I\rightarrow\RR^n$ be a regular parametrized curve of class $\mathcal{C}^1$, $I,J\subseteq\RR$ be open intervals and $h:J\rightarrow I$ be a diffeomorphism. Then, $\vf{\beta}:=\vf{\alpha}\circ h:J\rightarrow\RR^n$ is a curve satisfying: $$\vf{\beta}'(s)=\vf{\alpha}'(h(s))h'(s)\quad\forall s\in J$$ Hence, $\vf{\beta}$ is regular. In that case, we say that $\vf{\beta}$ is a \emph{reparametrization} of $\vf{\alpha}$ and $h$ is a \emph{change of parameter}. Moreover, the parametrization is \emph{positive} if $h'(s)>0$ $\forall s\in J$ and it is \emph{negative} if $h'(s)<0$ $\forall s\in J$.
  \end{definition}
  \subsubsection{Length of curves}
  \begin{definition}
    Let $\vf{\alpha}:I\rightarrow\RR^n$ be a parametrized of a curve of class $\mathcal{C}^0$, $[a,b]\subset I$, $\mathfrak{P}([a,b])$ be the set of all partitions of $[a,b]$ and $\mathcal{P}=\{t_0,\ldots,t_n\}\in\mathfrak{P}$. We define the \emph{length of the polygonal} with vertices at $\vf{\alpha}(t_i)$, $i=1,\ldots,n$ as: $$L_{a,b}(\vf{\alpha},\mathcal{P})=\sum_{i=1}^n\|\vf{\alpha}(t_i)-\vf{\alpha}(t_{i-1})\|$$ We define the $L_{a,b}(\vf{\alpha})$ as:
    $$L_{a,b}(\vf{\alpha}):=\sup\{L_{a,b}(\vf{\alpha},\mathcal{P}):\mathcal{P}\in\mathfrak{P}([a,b])\}\in\RR_{\geq 0}\cup\{+\infty\}$$ If $L_{a,b}(\vf{\alpha})<+\infty$, we say that $\vf{\alpha}$ is \emph{rectifiable} and that $L_{a,b}(\vf{\alpha})$ is its \emph{length} in $[a,b]$ of $\vf{\alpha}$.
  \end{definition}
  \begin{proposition}
    Let $\vf{\alpha}:I\rightarrow\RR^n$ be a parametrized curve of class $\mathcal{C}^1$. Then, $\vf{\alpha}$ is rectifiable and: $$L_{a,b}(\vf{\alpha})=\int_a^b\|\vf{\alpha}'(t)\|\dd{t}$$
  \end{proposition}
  \begin{proposition}
    Let $\vf{\alpha}:I\rightarrow\RR^n$ be a parametrized curve of class $\mathcal{C}^1$, $\vf{\beta}=\vf{\alpha}\circ h$ be a reparametrization of $\alpha$ and $[a,b]\subset \im h=I$. Suppose $[c,d]=h^{-1}([a,b])$. Then: $$\int_c^d\|\vf{\beta}'(u)\|\dd{u}=\int_a^b\|\vf{\alpha}'(t)\|\dd{t}$$ That is, the length of a curve does not depend on its parametrization.
  \end{proposition}
  \begin{definition}
    Let $\vf{\alpha}:I\rightarrow\RR^n$ be a parametrized curve of class $\mathcal{C}^1$ and $a\in I$. We define the \emph{arc-length function} of $\vf{\alpha}$ with origin $a$, the function $s_a:I\rightarrow\RR$ defined as: $$s_a(t)=\int_a^t\|\vf{\alpha}'(u)\|\dd{u}$$
  \end{definition}
  \begin{definition}
    Let $\vf{\alpha}:I\rightarrow\RR^n$ be a parametrized curve of class $\mathcal{C}^1$. We say that $\vf{\alpha}$ is a \emph{unit-speed parametrization} (or that it is parametrized by \emph{arc-length parameter}) if $\|\vf{\alpha}(t)\|=1$ $\forall t\in I$
  \end{definition}
  \begin{proposition}
    Let $\vf{\alpha}:I\rightarrow\RR^n$ be a parametrized curve of class $\mathcal{C}^1$ and $a\in I$.
    \begin{enumerate}
      \item $s_a$ is of class $\mathcal{C}^1$ and $\dv{s_a}{t}(t_0)=\|\vf{\alpha}'(t_0)\|\geq 0$.
      \item If $\vf{\alpha}$ is regular, then $J:=s_a(I)\subseteq\RR$ is an open interval and $s_a:I\rightarrow J$ is a diffeomorphism.
      \item If $\vf{\alpha}$ is regular, then $\vf{\beta}(s_a):=\vf{\alpha}(t(s_a))$\footnote{Here, $t(s_a)$ represent the inverse function of $s_a(t)$.} is an arc-length reparametrization of $\vf{\alpha}$.
    \end{enumerate}
  \end{proposition}
  \begin{proposition}
    Let $\vf{\alpha}:I\rightarrow\RR^n$ be a regular parametrized curve of class $\mathcal{C}^1$ and $\vf{\beta}=\vf{\alpha}\circ h$ be a reparametrization of $\vf{\alpha}$. If $\vf{\alpha}$ and $\vf{\beta}$ are arc-length parametrizations, then: $$\vf{\beta}(u)=\vf{\alpha}(\pm u+u_0)$$ for some $u_0\in\RR$.
  \end{proposition}
  \begin{proposition}
    All regular parametric curves of class $\mathcal{C}^1$ can be arc-length parametrized.
  \end{proposition}
  \begin{proposition}
    The length of any regular parametric curve of class $\mathcal{C}^1$ does not depend on its parametrization.
  \end{proposition}
  \begin{definition}
    Let $\vf{\alpha},\vf{\beta}:I\rightarrow\RR^n$ be arc-length parametrized curves of class $\mathcal{C}^\infty$ and $t_0\in I$. We say that $\vf{\alpha}$ and $\vf{\beta}$ have \emph{contact} of order $\geq r$ at $t_0$ if $$\lim_{t\to t_0}\frac{\vf{\alpha}(t)-\vf{\beta}(t)}{{(t-t_0)}^r}=0$$
    We say that $\vf{\alpha}$ and $\vf{\beta}$ have \emph{contact} of order $r$ at $t_0$ if they have contact of order $\geq r$ but not contact of order $\geq r+1$.
  \end{definition}
  \begin{proposition}
    Let $\vf{\alpha},\vf{\beta}:I\rightarrow\RR^n$ be arc-length parametrized curves of class $\mathcal{C}^\infty$ and $t_0\in I$. Then, $\vf{\alpha}$ and $\vf{\beta}$ have contact of order $\geq r$ at $t_0$ if and only if: $$\vf{\alpha}^{(k)}(t_0)=\vf{\beta}^{(k)}(t_0)\quad\text{for }k=0,\ldots,r$$
  \end{proposition}
  \subsubsection{Orientability and cross product}
  \begin{definition}
    Let $V$ be a vector space and $\mathcal{B}_1$ and $\mathcal{B}_2$ be two bases of $V$. We say that $\mathcal{B}_1\sim\mathcal{B}_2$ if $\det\left([\id]_{\mathcal{B}_1,\mathcal{B}_2}\right)>0$. This relation is an equivalence relation on the set of all bases for $V$ which has two exactly connected components.
  \end{definition}
  \begin{definition}
    Let $V$ be a vector space and $\mathcal{B}_1$ and $\mathcal{B}_2$ be two bases of $V$. We say that $\mathcal{B}_1\sim\mathcal{B}_2$ have the \emph{same orientation} if $\det\left([\id]_{\mathcal{B}_1,\mathcal{B}_2}\right)>0$. Otherwise, we say that they have \emph{opposite orientations}. Note that the property of having the same orientation defines an equivalence relation on the set of all bases for $V$.
  \end{definition}
  \begin{definition}
    An \emph{orientation} on a vector space is the choice of one of the two equivalence classes under $\sim$. A vector space with an orientation selected is called an \emph{oriented vector space}, while one not having an orientation selected, is called an \emph{unoriented vector space}. A basis of an oriented vector space which has the orientation chosen is called \emph{positive basis}, while one with the other orientation is called \emph{negative basis}.
  \end{definition}
  \begin{definition}
    Let $V$ be an oriented vector space, $\mathcal{B}$ be a basis of $V$ and $f:V\rightarrow V$ be a linear isomorphism. We say that $f$ is \emph{orientation-preserving} if $\det \left([f]_{\mathcal{B}}\right)>0$. Analogously, if $\det \left([f]_{\mathcal{B}_1,\mathcal{B}_2}\right)<0$ we say that $f$ is \emph{not orientation-preserving}.
  \end{definition}
  \begin{definition}
    Let $(\vf{v}_1,\ldots,\vf{v}_n)$ be a basis of $\RR^n$. Suppose for each $i\in\{1,\ldots,n\}$ we have $$\vf{v}_i=\sum_{j=1}^n\lambda_{ij}\vf{e}_1$$ where $\lambda_{ij}\in\RR$ and $(\vf{e}_1,\ldots,\vf{e}_n)$ is the standard basis of $\RR^n$. We define the \emph{determinant} of $(\vf{v}_1,\ldots,\vf{v}_n)$ as: $$\det(\vf{v}_1,\ldots,\vf{v}_n):=
      \begin{pmatrix}
        \lambda_{11} & \cdots \lambda_{1n} \\
        \vdots       & \ddots \vdots       \\
        \lambda_{n1} & \cdots \lambda_{nn} \\
      \end{pmatrix}\footnote{From now on, if we do not explicitly fix a basis it will mean that the standard basis of $\RR^n$ is the chosen one.}$$
  \end{definition}
  \begin{proposition}
    Let $\mathcal{B}=(\vf{v}_1,\ldots,\vf{v}_n)$ be a basis of $\RR^n$ and $\vf{A}\in\mathcal{M}_n(\RR)$. Then:
    $$\det(\vf{A}\vf{v}_1,\ldots,\vf{A}\vf{v}_n)=\det\vf{A}\det(\vf{v}_1,\ldots,\vf{v}_n)$$
  \end{proposition}
  \begin{proposition}
    Let $\vf{v}_1,\ldots,\vf{v}_n$ be vectors of $\RR^n$ and $P$ be the parallelepiped they generate. Then:
    $$\vol P=\abs{\det(\vf{v}_1,\ldots,\vf{v}_n)}$$
  \end{proposition}
  \begin{definition}
    Let $\vf{u}$, $\vf{v}$ be vectors of $\RR^3$. We define the \emph{cross product} of $\vf{u}$ and $\vf{v}$, denoted by $\vf{u}\times\vf{v}$\footnote{Another commonly used notation for the cross product is $\vf{u}\wedge\vf{v}$.}, as the unique vector $\vf{w}$ satisfying: $$\langle\vf{u}\times\vf{v},\vf{w}\rangle=\det(\vf{u},\vf{v},\vf{w})$$
  \end{definition}
  \begin{proposition}
    Let $\vf{u}$, $\vf{v}$ be vectors of $\RR^3$ such that $\vf{u}=\sum_{i=1}^3u_i\vf{e}_i$ and $\vf{v}=\sum_{i=1}^3v_i\vf{e}_i$. Then: $$\vf{u}\times\vf{v}=
      \begin{vmatrix}
        \vf{e}_1 & \vf{e}_2 & \vf{e}_3 \\
        u_1      & u_2      & u_3      \\
        v_1      & v_2      & v_3      \\
      \end{vmatrix}$$
  \end{proposition}
  \begin{proposition}
    Let $\vf{u}$, $\vf{v}$, $\vf{w}$ be vectors of $\RR^3$. Then:
    \begin{enumerate}
      \item $\vf{u}\times\vf{v}=-\vf{v}\times\vf{u}$
      \item $\vf{u}\times\vf{v}=0\iff\vf{u}=\lambda\vf{v}$, for some $\lambda\in\RR$.
      \item $\vf{u}\times\vf{v}\in{\langle\vf{u},\vf{v}\rangle}^{\perp}$
      \item If $\vf{u}$ and $\vf{v}$ are linearly independent, $(\vf{u},\vf{v},\vf{u}\times\vf{v})$ is a positive basis of $\RR^n$.
      \item If $\vf{x}$, $\vf{y}$ are vectors of $\RR^3$, then: $$\langle\vf{u}\times\vf{v},\vf{x}\times\vf{y}\rangle=
              \begin{vmatrix}
                \vf{u}\times\vf{x} & \vf{v}\times\vf{x} \\
                \vf{u}\times\vf{y} & \vf{v}\times\vf{y} \\
              \end{vmatrix}$$
      \item Let $\theta\in[0,\pi]$ the angle between $\vf{u}$ and $\vf{v}$. Then: $$\|\vf{u}\times\vf{v}\|=\|\vf{u}\|\|\vf{v}\|\sin\theta$$
      \item $(\vf{u}\times\vf{v})\times\vf{w}=\langle\vf{u},\vf{w}\rangle\vf{v}-\langle\vf{v},\vf{w}\rangle\vf{u}$
      \item \emph{Jacobi identity}: $$(\vf{u}\times\vf{v})\times\vf{w}+(\vf{v}\times\vf{w})\times\vf{u}+(\vf{w}\times\vf{u})\times\vf{v}=\vf{0}$$
    \end{enumerate}
  \end{proposition}
  \begin{proposition}
    Let $\vf{\alpha},\vf{\beta}:I\rightarrow\RR^3$ be parametrized curves of class $\mathcal{C}^\infty$. Then:
    $$\dv{}{t}\left(\vf{\alpha}(t)\times\vf{\beta}(t)\right)=\vf{\alpha}'(t)\times\vf{\beta}(t)+\vf{\alpha}(t)\times\vf{\beta}'(t)$$
  \end{proposition}
  \subsubsection{Frenet-Serret formulas}
  \begin{definition}
    Let $\vf{\alpha}:I\rightarrow\RR^3$ be an arc-length parametrized curve of class $\mathcal{C}^\infty$. We define the unit tangent vector of $\vf{\alpha}$ at $s_0\in I$ as $$\T\alpha(s_0):=\vf{\alpha}'(s_0)$$ Note that $\|\T\alpha\|=1$ and $\T\alpha\perp\T\alpha'$.
  \end{definition}
  \begin{definition}
    Let $\vf{\alpha}:I\rightarrow\RR^3$ be an arc-length parametrized curve of class $\mathcal{C}^\infty$. We define the \emph{curvature} of $\vf{\alpha}$ at $s_0\in I$ as: $$\ka\alpha(s_0):=\|\vf{\alpha}''(s_0)\|=\|\T\alpha'(s_0)\|$$
  \end{definition}
  \begin{definition}
    Let $\vf{\alpha}:I\rightarrow\RR^3$ be an arc-length parametrized curve of class $\mathcal{C}^\infty$, $s_0\in I$ and suppose that $\ka\alpha(s_0)\ne 0$. Then, we define the \emph{unit normal vector} of $\vf{\alpha}$ at $s_0$ as $$\N\alpha(s_0):=\frac{\T\alpha'(s_0)}{\ka\alpha(s_0)}=\frac{\vf{\alpha}''(s_0)}{\|\vf{\alpha}''(s_0)\|}$$
    Note that $\|\N\alpha\|=1$, $\N\alpha\perp\T\alpha$ and $\T\alpha'(s)=\ka\alpha(s)\N\alpha(s)$ $\forall s\in I$.
  \end{definition}
  \begin{proposition}
    Let $\vf{\alpha}:I\rightarrow\RR^3$ be an arc-length parametrized curve of class $\mathcal{C}^\infty$, $s_0\in I$ and suppose that $\ka\alpha(s_0)\ne 0$. Then, there exists a unique circle of $\RR^3$ that has contact of order $\geq 2$ at $\vf{\alpha}(s_0)$. This cercle is called \emph{osculating circle} and its radius (called \emph{radius of curvature}) is $\rho_{\vf{\alpha}}(s_0)=\frac{1}{\ka\alpha(s_0)}$. Its center is $Q=\vf{\alpha}(s_0)+\rho_{\vf{\alpha}}(s_0)\N\alpha(s_0)$\footnote{An arc-length parametrization of the osculating circle would be $$u\longmapsto Q+\rho_{\vf{\alpha}}(s_0)\left(-\cos\left(\frac{u}{\rho_{\vf{\alpha}}(s_0)}\right)\N\alpha(s_0)+\sin\left(\frac{u}{\rho_{\vf{\alpha}}(s_0)}\right)\T\alpha(s_0)\right)$$}.
  \end{proposition}
  \begin{definition}
    Let $\vf{\alpha}:I\rightarrow\RR^3$ be a regular arc-length parametrized curve of class $\mathcal{C}^\infty$ such that $\vf{\alpha}''(s)\ne 0$ $\forall s\in I$. We define the \emph{binormal vector}of $\vf{\alpha}$ at $s_0\in I$ as:
    $$\B\alpha(s_0)=\T\alpha(s_0)\times\N\alpha(s_0)$$
    Then, the triplet $(\T\alpha(s),\N\alpha(s),\B\alpha(s))$ is an orthonormal positive basis\footnote{Observe that the binormal vector (together with the tangent and normal vectors) is the unique vector that satisfies this property.}, and the affine frame $\{\vf{\alpha}(s_0); (\T\alpha(s),\N\alpha(s),\B\alpha(s))\}$ is called \emph{Frenet-Serret frame} (or \emph{TNB frame}).
  \end{definition}
  \begin{proposition}
    Let $\vf{\alpha}:I\rightarrow\RR^3$ be a regular arc-length parametrized curve of class $\mathcal{C}^\infty$ such that $\vf{\alpha}''(s)\ne 0$ $\forall s\in I$. Then, $$\B\alpha'(s)=\ta\alpha(s)\N\alpha(s)\quad\forall s\in I$$ This coefficient $\ta\alpha(s)$ is called \emph{torsion} of $\vf{\alpha}$ at $s\in I$.
  \end{proposition}
  \begin{proposition}
    Let $\vf{\alpha}:I\rightarrow\RR^3$ be a regular arc-length parametrized curve of class $\mathcal{C}^\infty$ such that $\vf{\alpha}''(s)\ne 0$ $\forall s\in I$. The following statements are equivalent:
    \begin{enumerate}
      \item $\vf{\alpha}$ is a plane curve.
      \item $\B\alpha=\const$
      \item $\ta\alpha=0$.
    \end{enumerate}
  \end{proposition}
  \begin{theorem}[Frenet-Serret formulas]
    Let $\vf{\alpha}:I\rightarrow\RR^3$ be a regular arc-length parametrized curve of class $\mathcal{C}^\infty$ such that $\vf{\alpha}''(s)\ne 0$ $\forall s\in I$. Then:
    $$
      \begin{pmatrix}
        \T\alpha \\
        \N\alpha \\
        \B\alpha \\
      \end{pmatrix}'=
      \begin{pmatrix}
        0          & \ka\alpha & 0          \\
        -\ka\alpha & 0         & -\ta\alpha \\
        0          & \ta\alpha & 0
      \end{pmatrix}
      \begin{pmatrix}
        \T\alpha \\
        \N\alpha \\
        \B\alpha \\
      \end{pmatrix}
    $$
  \end{theorem}
  \begin{definition}
    Let $\vf{\alpha}:I\rightarrow\RR^3$ be a regular arc-length parametrized curve of class $\mathcal{C}^\infty$ such that $\vf{\alpha}''(s)\ne 0$ $\forall s\in I$. and $s_0\in I$. We define the following planes of $\RR^3$:
    \begin{itemize}
      \item \emph{Osculating plane}: plane generated by $\T\alpha(s_0)$ and $\N\alpha(s_0)$ that contains $\vf{\alpha}(s_0)$.
      \item \emph{Normal plane}: plane generated by $\N\alpha(s_0)$ and $\B\alpha(s_0)$ that contains $\vf{\alpha}(s_0)$.
      \item \emph{Rectifying plane}: plane generated by $\T\alpha(s_0)$ and $\B\alpha(s_0)$ that contains $\vf{\alpha}(s_0)$.
    \end{itemize}
  \end{definition}
  \begin{proposition}
    Let $\vf{\alpha}:I\rightarrow\RR^3$ be a regular parametrized curve of class $\mathcal{C}^\infty$ and $h(t)=s(t)$ be the arc-length parameter. Let $\vf{\beta}=(\vf{\alpha}\circ h^{-1})(s)$, which is an arc-length parametrization of $\vf{\alpha}$. Then, assuming $\vf{\beta}''\ne 0$, we can define the TNB frame of $\vf{\alpha}$ as: $$\T\alpha:=\T\beta\circ h\qquad\N\alpha:=\N\beta\circ h\qquad\B\alpha:=\B\beta\circ h$$
    And the curvature and torsion of $\vf\alpha$ as: $$\ka\alpha:=\ka\beta\circ h\qquad\ta\alpha:=\ta\beta\circ h$$
  \end{proposition}
  \begin{lemma}
    Let $\vf{\alpha}:I\rightarrow\RR^3$ be a regular parametrized curve of class $\mathcal{C}^\infty$ and $h(t)=s(t)$ be the arc-length parameter. Let $\vf{\beta}=(\vf{\alpha}\circ h^{-1})(s)$. Then, $\vf{\beta}''=0\iff\vf\alpha'\times\vf\alpha''=0$.
  \end{lemma}
  \begin{proposition}
    Let $\vf{\alpha}:I\rightarrow\RR^3$ be a regular parametrized curve of class $\mathcal{C}^\infty$ such that $\vf{\alpha}'\times \vf{\alpha}''\ne 0$ and $v(t):=\|\vf{\alpha}'(t)\|$. Then:
    \begin{itemize}
      \item $\vf{\alpha}'=v\T\alpha$
      \item $\vf{\alpha}''=v'\T\alpha'+\ka\alpha v^2\N\alpha$
    \end{itemize}
  \end{proposition}
  \begin{theorem}[General Frenet-Serret formulas]
    Let $\vf{\alpha}:I\rightarrow\RR^3$ be a regular parametrized curve of class $\mathcal{C}^\infty$ such that $\vf{\alpha}'\times \vf{\alpha}''\ne 0$ and $v(t):=\|\vf{\alpha}'(t)\|$. Then:
    $$
      \begin{pmatrix}
        \T\alpha \\
        \N\alpha \\
        \B\alpha \\
      \end{pmatrix}'=
      \begin{pmatrix}
        0            & \ka\alpha v & 0            \\
        -\ka\alpha v & 0           & -\ta\alpha v \\
        0            & \ta\alpha v & 0
      \end{pmatrix}
      \begin{pmatrix}
        \T\alpha \\
        \N\alpha \\
        \B\alpha \\
      \end{pmatrix}
    $$
  \end{theorem}
  \begin{corollary}
    Let $\vf{\alpha}:I\rightarrow\RR^3$ be a regular parametrized curve of class $\mathcal{C}^\infty$ such that $\vf{\alpha}'\times \vf{\alpha}''\ne 0$. Then:
    $$\T\alpha=\frac{\vf{\alpha}'}{\|\vf{\alpha}'\|}\qquad\N\alpha=\B\alpha\times\T\alpha\qquad\B\alpha=\frac{\vf{\alpha}'\times\vf{\alpha}''}{\|\vf{\alpha}'\times\vf{\alpha}''\|}$$
    Moreover: $$\ka\alpha=\frac{\|\vf{\alpha}'\times\vf{\alpha}''\|}{{\|\vf{\alpha}'\|}^3}\qquad\ta\alpha=-\frac{\langle\vf{\alpha}'\times\vf{\alpha}'',\vf{\alpha}'''\rangle}{{\|\vf{\alpha}'\times\vf{\alpha}''\|}^2}$$
  \end{corollary}
  \subsubsection{Curvature of plane curves}
  \begin{lemma}
    Let $a,b:I\rightarrow\RR$ be differentiable functions such that $a^2+b^2=1$, $t_0\in I$ and $\theta_0\in\RR$ be such that $a(t_0)=\cos\theta_0$ and $b(t_0)=\sin\theta_0$. Then, the differentiable function $\theta(t)$ defined as:
    $$\theta(t)=\theta_0+\int_{t_0}^t\left(a(t)b'(u)-a'(u)b(u)\right)\dd{u}$$
    satisfies $a(t)=\cos\theta(t)$, $b(t)=\sin\theta(t)$ and $\theta(t_0)=\theta_0$ $\forall t\in I$.
  \end{lemma}
  \begin{proposition}
    Let $\vf{\alpha}:I\rightarrow\RR^2$ be a regular arc-length parametrized curve of class $\mathcal{C}^\infty$. Then, there is a unique vector $\vf{\hat{N}}_{\vf\alpha}$ such that $(\T\alpha,\vf{\hat{N}}_{\vf\alpha})$ is a positive orthonormal basis of $\RR^2$. Thus, $\T\alpha'\parallel \vf{\hat{N}}_{\vf\alpha}$ and we define the \emph{signed curvature} of $\vf\alpha$ as the value $k_{\vf\alpha}$ satisfying $\T\alpha'=k_{\vf\alpha}\vf{\hat{N}}_{\vf\alpha}$\footnote{Note that, $\vf{\hat{N}}_{\vf\alpha}=\pm\N\alpha$ and $k_{\vf\alpha}=\pm\ka\alpha$.}. Moreover: $$k_{\vf\alpha}=\det(\T\alpha,\T\alpha')$$
  \end{proposition}
  \begin{proposition}
    Let $\vf{\alpha}:I\rightarrow\RR^2$ be a regular parametrized curve of class $\mathcal{C}^\infty$. Then, the signed curvature of $\vf\alpha$ is: $$k_{\vf\alpha}=\frac{\det(\vf\alpha',\vf\alpha'')}{{\norm{\vf\alpha'}}^3}$$
  \end{proposition}
  \subsubsection{Local form of a curve}
  \begin{definition}
    Let $\vf{\alpha}:I\rightarrow\RR^2$ be a regular arc-length parametrized curve of class $\mathcal{C}^\infty$ and $s_0\in I$. Choose a frame of reference $\mathcal{R}$ such that $\vf{\alpha}(s_0)\mapsto (0,0,0)$ and $(\T\alpha,\N\alpha,\B\alpha)\mapsto(\vf{e}_1,\vf{e}_2,\vf{e}_3)$ and suppose $\vf{\alpha}(s)_{\mathcal{R}}=(x(s),y(s),z(s))_\mathcal{R}$. Then:
    $$\left\{
      \def\arraystretch{2}
      \begin{array}{l}
        \displaystyle x(s)\simeq s-\frac{{\ka\alpha(0)}^2}{6}s^3                          \\
        \displaystyle y(s)\simeq \frac{{\ka\alpha(0)}^2}{2}s^2-\frac{\ka\alpha'(0)}{6}s^3 \\
        \displaystyle z(s)\simeq -\frac{\ka\alpha(0)\ta\alpha(0)}{6}s^3                   \\
      \end{array}
      \right.
    $$
    This expression of $\vf{\alpha}(s)_{\mathcal{R}}$ is called \emph{local canonical form} of $\vf{\alpha}$ in a neighbourhood of $s_0$.
  \end{definition}
  \subsubsection{Orthogonal group}
  \begin{definition}
    We define that \emph{orthogonal group} as the group of all linear transformations that preserve the inner product. That is: $$\text{O}(n):=\{\vf{A}\in\mathcal{M}_n(\RR):\langle\vf{Au},\vf{Av}\rangle=\langle\vf{u},\vf{v}\rangle\;\forall\vf{u},\vf{v}\in\RR^n\}$$
  \end{definition}
  \begin{proposition}
    Let $\vf{A}\in\text{O}(n)$. Then, $\det\vf{A}=\pm$ and $\vf{A}\transpose{A}=\vf{I}_n$.
  \end{proposition}
  \begin{definition}
    We define that \emph{special orthogonal group} as: $$\text{SO}(n):=\{\vf{A}\in\text{O}(n):\det\vf{A}=1\}$$
  \end{definition}
  \begin{lemma}
    Let $\vf{A}\in\text{O}(n)$ and $\lambda\in\sigma(\vf{A})$. Then, $\lambda\in\RR\implies\lambda=\pm 1$.
  \end{lemma}
  \begin{proposition}
    Let $\vf{A}\in\text{O}(n)$. Then:
    $$\left\{
      \begin{array}{lll}
        \vspace{0.1cm}
        \vf{A}=\begin{pmatrix}
          \cos\omega & -\sin\omega \\
          \sin\omega & \cos\omega  \\
        \end{pmatrix} &
        \text{ if }                       & \det\vf{A}=1  \\
        \vf{A}=\begin{pmatrix}
          \cos\omega & \sin\omega  \\
          \sin\omega & -\cos\omega \\
        \end{pmatrix} &
        \text{ if }                       & \det\vf{A}=-1
      \end{array}
      \right.
    $$
    for some $\omega\in\RR$.
  \end{proposition}
  \begin{proposition}
    Let $\vf{A}\in\text{O}(3)$. Then, there exists a orthonormal basis $\mathcal{B}$ of $\RR^3$ such that $$[\id]_{\text{Can}(\RR^3),\mathcal{B}}\vf{A}[\id]_{\text{Can}(\RR^3),\mathcal{B}}=\begin{pmatrix}
        \pm 1 & 0                                              & 0 \\
        0     & \multicolumn{2}{c}{\multirow{2}{*}{$\vf{A}'$}}     \\
        0     &                                                &
      \end{pmatrix}$$
    where $\vf{A}'\in\text{O}(2)$.
  \end{proposition}
  \begin{proposition}
    Let $\vf{f}:\RR^n\rightarrow\RR^n$ be an Euclidean motion\footnote{Recall that an Euclidean motion is a function that preserves the distance, that is, if $\vf{f}:\RR^n\rightarrow\RR^n$ is an Euclidean motion, then $\|\vf{f}(p)-\vf{f}(q)\|=\|p-q\|$ $\forall p,q\in\RR^n$.}. Then, $\exists\vf{A}\in\text{O}(n)$ and $\vf{u}\in\RR^n$ such that: $$\vf{f}(\vf{v})=\vf{Av}+\vf{u}$$
  \end{proposition}
  \begin{proposition}
    Let $\vf{\alpha}:I\rightarrow\RR^n$ be a parametrized curve of class $\mathcal{C}^\infty$ and $\vf{A}\in\mathcal{M}_n(\RR)$. Then:
    $${\left(\vf{A}\vf{\alpha}\right)}^{'}(t)=\vf{A}\vf{\alpha}'(t)$$
  \end{proposition}
  \begin{proposition}
    Let $\vf{A}\in\text{SO}(3)$. Then, $\forall \vf{u},\vf{v}\in\RR^3$ we have: $$\vf{A}\left(\vf{u}\times\vf{v}\right)=\left(\vf{Au}\right)\times\left(\vf{Av}\right)$$
  \end{proposition}
  \begin{corollary}
    Let $\vf{\alpha}:I\rightarrow\RR^3$ be an arc-length parametrized curve of class $\mathcal{C}^\infty$ and $\vf{\beta}:=\vf{A\alpha}+\vf{u}$, where $\vf{A}\in\text{SO}(3)$ and $\vf{u}\in\RR^3$. Then, $\vf{\beta}$ is arc-length parametrized and th TNB frame of $\vf{\beta}$ is:
    $$\T\beta=\vf{A}\T\alpha\qquad\N\beta=\vf{A}\N\beta\qquad\B\beta=\vf{A}\B\alpha$$
    And the curvature and torsion of $\vf\beta$ are: $$\ka\beta:=\ka\alpha\qquad\ta\beta:=\ta\alpha$$
  \end{corollary}
  \subsubsection{Fundamental theorem of curves}
  \begin{theorem}[Fundamental theorem of curves]
    Let $\kappa,\tau:I\rightarrow\RR$ be functions of class $\mathcal{C}^\infty$ with $\kappa(s)>0$ $\forall s\in I$. Then, there is an arc-length parametrized curve $\vf{\alpha}:I\rightarrow\RR^3$ of class $\mathcal{C}^\infty$ whose curvature and torsion are $\kappa$ and $\tau$, respectively. Moreover, if $\tilde{\vf{\alpha}}:I\rightarrow\RR^3$ is another curve satisfying these restrictions, then there exists an Euclidean motion that carries $\tilde{\vf{\alpha}}$ to $\vf{\alpha}$\footnote{In matrix form, $\exists\vf{A}\in\text{SO}(3)$ and $\vf{u}\in\RR^3$ such that $\vf{\alpha}=\vf{A}\tilde{\vf{\alpha}}+\vf{u}$.}.
  \end{theorem}
  \subsection{Submanifolds of \texorpdfstring{$\RR^n$}{Rn}}
  \subsubsection{Planar functions}
  \begin{definition}
    Let $U\subseteq\RR^n$ be an open set and $f:U\rightarrow\RR$ be a function. We define the \emph{support} of $f$ as:
    $$\supp(f):=\Cl\left(\{x\in U:f(x)\ne 0\}\right)$$
  \end{definition}
\end{multicols}
\end{document}