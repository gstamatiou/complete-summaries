\documentclass[../../../main.tex]{subfiles}

\begin{document}
\begin{multicols}{2}[\section{Differential geometry}]
  \subsection{Differentiable curves}
  \subsubsection{Inner product of \texorpdfstring{$\RR^n$}{Rn}}
  \begin{proposition}
    Let $\vf{u},\vf{v}\in\RR^n$ be vectors and $\langle\vf{u},\vf{v}\rangle$ be the usual inner product between $\vf{u}$ and $\vf{v}$ in $\RR^n$. Then:
    \begin{itemize}
      \item Cauchy-Schwarz inequality: $$|\langle\vf{u},\vf{v}\rangle|\leq\|\vf{u}\|\|\vf{v}\|$$
      \item Triangular inequality: $$\|\vf{u}+\vf{v}\|\leq\|\vf{u}\|+\|\vf{v}\|$$
      \item Polarization identity: $$\langle\vf{u},\vf{v}\rangle=\frac{1}{2} \left(\|\vf{u}+\vf{v}\|^2-\|\vf{u}\|^2 - \|\vf{v}\|^2\right)$$
    \end{itemize}
  \end{proposition}
  \begin{definition}
    Let $\vf{u},\vf{v}\in\RR^n$ be vectors. We define the angle between $\vf{u}$ and $\vf{v}$ as the unique value $\theta\in[0,\pi]$ such that: $$\cos\theta=\frac{\langle\vf{u},\vf{v}\rangle}{\|\vf{u}\|\|\vf{v}\|}$$
  \end{definition}
  \subsubsection{Parametrized curves}\label{DG_curves}
  \begin{definition}
    Let $U\subseteq\RR^n$ be an open set and $\vf{f}:U\rightarrow\RR^n$ be a differentiable function. We say that $\vf{f}$ is a \emph{local diffeomorphism} if $\forall p\in U$, there exists a neighbourhood $V\subseteq U$ of $p$ such that $\vf{f}|_V:V\rightarrow \vf{f}(V)$ is a diffeomorphism.
  \end{definition}
  \begin{proposition}
    Let $I\subseteq\RR$ be an open interval and $f:I\rightarrow\RR$ be a differentiable function. If $f'(x)\ne 0$ $\forall x\in I$, then $f(I)$ is an open set and $f$ is a diffeomorphism.
  \end{proposition}
  \begin{proposition}
    Let $I\subseteq\RR$ be an open interval and $\vf{\alpha},\vf{\beta}:I\rightarrow\RR^n$ be differentiable functions. Then:
    \begin{enumerate}
      \item $\dv{}{t}\langle\vf{\alpha}(t),\vf{\beta}(t)\rangle=\langle\vf{\alpha}'(t),\vf{\beta}(t)\rangle+\langle\vf{\alpha}(t),\vf{\beta}'(t)\rangle$
      \item If $t\mapsto\|\vf{\alpha}(t)\|$ is a constant function, then $\vf{\alpha}\perp\vf{\alpha}'$.
    \end{enumerate}
  \end{proposition}
  \begin{definition}
    Let $I\subseteq\RR$ be an open interval. A \emph{parametrized curve of class $\mathcal{C}^k$} in $\RR^n$ is a function $\vf{\alpha}:I\rightarrow\RR^n$ of class $\mathcal{C}^k$. The image of $\vf\alpha$ is called the \emph{trace} of the curve and it is denoted by $\vf\alpha^*:=\im(\vf\alpha)$.
  \end{definition}
  \begin{definition}
    Let $\vf{\alpha}:I\rightarrow\RR^n$ be a parametrized curve of class $\mathcal{C}^1$. We define the \emph{tangent vector} of $\vf{\alpha}$ at $t_0\in\RR$ as $\vf{\alpha}'(t_0)$. We say that $\vf{\alpha}$ is \emph{regular} if $\vf{\alpha}'(t)\ne 0$ $\forall t\in I$. In that last case, we define the \emph{tangent line} of $\vf{\alpha}$ at $\vf{\alpha}(t_0)$ as the following parametrized line in $\RR^n$: $$s\longmapsto \vf{\alpha}(t_0)+s\vf{\alpha}'(t_0)$$
  \end{definition}
  \begin{definition}
    Let $\vf{\alpha}:I\rightarrow\RR^n$ be a parametrized curve of class $\mathcal{C}^1$. We say that $\vf{\alpha}$ is a \emph{plane curve} is it is contained in a plane of $\RR^n$.
  \end{definition}
  \begin{definition}\label{DG_reparam}
    Let $\vf{\alpha}:I\rightarrow\RR^n$ be a regular parametrized curve of class $\mathcal{C}^1$, $I,J\subseteq\RR$ be open intervals and $h:J\rightarrow I$ be a diffeomorphism. Then, $\vf{\beta}:=\vf{\alpha}\circ h:J\rightarrow\RR^n$ is a curve satisfying: $$\vf{\beta}'(s)=\vf{\alpha}'(h(s))h'(s)\quad\forall s\in J$$ Hence, $\vf{\beta}$ is regular. In that case, we say that $\vf{\beta}$ is a \emph{reparametrization} of $\vf{\alpha}$ and $h$ is a \emph{change of parameter}. Moreover, the parametrization is \emph{positive} if $h'(s)>0$ $\forall s\in J$ and it is \emph{negative} if $h'(s)<0$ $\forall s\in J$.
  \end{definition}
  \subsubsection{Length of curves}
  \begin{definition}
    Let $\vf{\alpha}:I\rightarrow\RR^n$ be a parametrized of a curve of class $\mathcal{C}^0$, $[a,b]\subset I$, $\mathfrak{P}([a,b])$ be the set of all partitions of $[a,b]$ and $\mathcal{P}=\{t_0,\ldots,t_n\}\in\mathfrak{P}$. We define the \emph{length of the polygonal} with vertices at $\vf{\alpha}(t_i)$, $i=1,\ldots,n$ as: $$L_{a,b}(\vf{\alpha},\mathcal{P})=\sum_{i=1}^n\|\vf{\alpha}(t_i)-\vf{\alpha}(t_{i-1})\|$$ We define the $L_{a,b}(\vf{\alpha})$ as:
    $$L_{a,b}(\vf{\alpha}):=\sup\{L_{a,b}(\vf{\alpha},\mathcal{P}):\mathcal{P}\in\mathfrak{P}([a,b])\}\in\RR_{\geq 0}\cup\{+\infty\}$$ If $L_{a,b}(\vf{\alpha})<+\infty$, we say that $\vf{\alpha}$ is \emph{rectifiable} and that $L_{a,b}(\vf{\alpha})$ is its \emph{length} in $[a,b]$ of $\vf{\alpha}$.
  \end{definition}
  \begin{proposition}
    Let $\vf{\alpha}:I\rightarrow\RR^n$ be a parametrized curve of class $\mathcal{C}^1$. Then, $\vf{\alpha}$ is rectifiable and: $$L_{a,b}(\vf{\alpha})=\int_a^b\|\vf{\alpha}'(t)\|\dd{t}$$
  \end{proposition}
  \begin{proposition}
    Let $\vf{\alpha}:I\rightarrow\RR^n$ be a parametrized curve of class $\mathcal{C}^1$, $\vf{\beta}=\vf{\alpha}\circ h$ be a reparametrization of $\alpha$ and $[a,b]\subset \im h=I$. Suppose $[c,d]=h^{-1}([a,b])$. Then: $$\int_c^d\|\vf{\beta}'(u)\|\dd{u}=\int_a^b\|\vf{\alpha}'(t)\|\dd{t}$$ That is, the length of a curve does not depend on its parametrization.
  \end{proposition}
  \begin{definition}
    Let $\vf{\alpha}:I\rightarrow\RR^n$ be a parametrized curve of class $\mathcal{C}^1$ and $a\in I$. We define the \emph{arc-length function} of $\vf{\alpha}$ with origin $a$, the function $s_a:I\rightarrow\RR$ defined as: $$s_a(t)=\int_a^t\|\vf{\alpha}'(u)\|\dd{u}$$
  \end{definition}
  \begin{definition}
    Let $\vf{\alpha}:I\rightarrow\RR^n$ be a parametrized curve of class $\mathcal{C}^1$. We say that $\vf{\alpha}$ is a \emph{unit-speed parametrization} (or that it is parametrized by \emph{arc-length parameter}) if $\|\vf{\alpha}(t)\|=1$ $\forall t\in I$
  \end{definition}
  \begin{proposition}
    Let $\vf{\alpha}:I\rightarrow\RR^n$ be a parametrized curve of class $\mathcal{C}^1$ and $a\in I$.
    \begin{enumerate}
      \item $s_a$ is of class $\mathcal{C}^1$ and $\dv{s_a}{t}(t_0)=\|\vf{\alpha}'(t_0)\|\geq 0$.
      \item If $\vf{\alpha}$ is regular, then $J:=s_a(I)\subseteq\RR$ is an open interval and $s_a:I\rightarrow J$ is a diffeomorphism.
      \item If $\vf{\alpha}$ is regular, then $\vf{\beta}(s_a):=\vf{\alpha}(t(s_a))$\footnote{Here, $t(s_a)$ represent the inverse function of $s_a(t)$.} is an arc-length reparametrization of $\vf{\alpha}$.
    \end{enumerate}
  \end{proposition}
  \begin{proposition}
    Let $\vf{\alpha}:I\rightarrow\RR^n$ be a regular parametrized curve of class $\mathcal{C}^1$ and $\vf{\beta}=\vf{\alpha}\circ h$ be a reparametrization of $\vf{\alpha}$. If $\vf{\alpha}$ and $\vf{\beta}$ are arc-length parametrizations, then: $$\vf{\beta}(u)=\vf{\alpha}(\pm u+u_0)$$ for some $u_0\in\RR$.
  \end{proposition}
  \begin{proposition}
    All regular parametric curves of class $\mathcal{C}^1$ can be arc-length parametrized.
  \end{proposition}
  \begin{proposition}
    The length of any regular parametric curve of class $\mathcal{C}^1$ does not depend on its parametrization.
  \end{proposition}
  \subsubsection{Orientability and cross product}
  \begin{definition}
    Let $V$ be a vector space and $\mathcal{B}_1$ and $\mathcal{B}_2$ be two bases of $V$. We say that $\mathcal{B}_1\sim\mathcal{B}_2$ if $\det\left([\id]_{\mathcal{B}_1,\mathcal{B}_2}\right)>0$. This relation is an equivalence relation on the set of all bases for $V$ which has two exactly connected components.
  \end{definition}
  \begin{definition}
    Let $V$ be a vector space and $\mathcal{B}_1$ and $\mathcal{B}_2$ be two bases of $V$. We say that $\mathcal{B}_1\sim\mathcal{B}_2$ have the \emph{same orientation} if $\det\left([\id]_{\mathcal{B}_1,\mathcal{B}_2}\right)>0$. Otherwise, we say that they have \emph{opposite orientations}. Note that the property of having the same orientation defines an equivalence relation on the set of all bases for $V$.
  \end{definition}
  \begin{definition}
    An \emph{orientation} on a vector space is the choice of one of the two equivalence classes under $\sim$. A vector space with an orientation selected is called an \emph{oriented vector space}, while one not having an orientation selected, is called an \emph{unoriented vector space}. A basis of an oriented vector space which has the orientation chosen is called \emph{positive basis}, while one with the other orientation is called \emph{negative basis}.
  \end{definition}
  \begin{definition}
    Let $V$ be an oriented vector space, $\mathcal{B}$ be a basis of $V$ and $f:V\rightarrow V$ be a linear isomorphism. We say that $f$ is \emph{orientation-preserving} if $\det \left([f]_{\mathcal{B}}\right)>0$. Analogously, if $\det \left([f]_{\mathcal{B}_1,\mathcal{B}_2}\right)<0$ we say that $f$ is \emph{not orientation-preserving}.
  \end{definition}
  \begin{definition}
    Let $(\vf{v}_1,\ldots,\vf{v}_n)$ be a basis of $\RR^n$. Suppose for each $i\in\{1,\ldots,n\}$ we have $$\vf{v}_i=\sum_{j=1}^n\lambda_{ij}\vf{e}_1$$ where $\lambda_{ij}\in\RR$ and $(\vf{e}_1,\ldots,\vf{e}_n)$ is the standard basis of $\RR^n$. We define the \emph{determinant} of $(\vf{v}_1,\ldots,\vf{v}_n)$ as: $$\det(\vf{v}_1,\ldots,\vf{v}_n):=
      \begin{pmatrix}
        \lambda_{11} & \cdots \lambda_{1n} \\
        \vdots       & \ddots \vdots       \\
        \lambda_{n1} & \cdots \lambda_{nn} \\
      \end{pmatrix}\footnote{From now on, if we do not explicitly fix a basis it will mean that the standard basis of $\RR^n$ is the chosen one.}$$
  \end{definition}
  \begin{proposition}
    Let $\mathcal{B}=(\vf{v}_1,\ldots,\vf{v}_n)$ be a basis of $\RR^n$ and $\vf{A}\in\mathcal{M}_n(\RR)$. Then:
    $$\det(\vf{A}\vf{v}_1,\ldots,\vf{A}\vf{v}_n)=\det\vf{A}\det(\vf{v}_1,\ldots,\vf{v}_n)$$
  \end{proposition}
  \begin{proposition}
    Let $\vf{v}_1,\ldots,\vf{v}_n$ be vectors of $\RR^n$ and $P$ be the parallelepiped they generate. Then:
    $$\vol P=\abs{\det(\vf{v}_1,\ldots,\vf{v}_n)}$$
  \end{proposition}
  \begin{definition}
    Let $\vf{u}$, $\vf{v}$ be vectors of $\RR^3$. We define the \emph{cross product} of $\vf{u}$ and $\vf{v}$, denoted by $\vf{u}\times\vf{v}$\footnote{Another commonly used notation for the cross product is $\vf{u}\wedge\vf{v}$.}, as the unique vector $\vf{w}$ satisfying: $$\langle\vf{u}\times\vf{v},\vf{w}\rangle=\det(\vf{u},\vf{v},\vf{w})$$
  \end{definition}
  \begin{proposition}
    Let $\vf{u}$, $\vf{v}$ be vectors of $\RR^3$ such that $\vf{u}=\sum_{i=1}^3u_i\vf{e}_i$ and $\vf{v}=\sum_{i=1}^3v_i\vf{e}_i$. Then: $$\vf{u}\times\vf{v}=
      \begin{vmatrix}
        \vf{e}_1 & \vf{e}_2 & \vf{e}_3 \\
        u_1      & u_2      & u_3      \\
        v_1      & v_2      & v_3      \\
      \end{vmatrix}$$
  \end{proposition}
  \begin{proposition}
    Let $\vf{u}$, $\vf{v}$, $\vf{w}$ be vectors of $\RR^3$. Then:
    \begin{enumerate}
      \item $\vf{u}\times\vf{v}=-\vf{v}\times\vf{u}$
      \item $\vf{u}\times\vf{v}=0\iff\vf{u}=\lambda\vf{v}$, for some $\lambda\in\RR$.
      \item $\vf{u}\times\vf{v}\in{\langle\vf{u},\vf{v}\rangle}^{\perp}$
      \item If $\vf{u}$ and $\vf{v}$ are linearly independent, $(\vf{u},\vf{v},\vf{u}\times\vf{v})$ is a positive basis of $\RR^n$.
      \item If $\vf{x}$, $\vf{y}$ are vectors of $\RR^3$, then: $$\langle\vf{u}\times\vf{v},\vf{x}\times\vf{y}\rangle=
              \begin{vmatrix}
                \vf{u}\times\vf{x} & \vf{v}\times\vf{x} \\
                \vf{u}\times\vf{y} & \vf{v}\times\vf{y} \\
              \end{vmatrix}$$
      \item Let $\theta\in[0,\pi]$ the angle between $\vf{u}$ and $\vf{v}$. Then: $$\|\vf{u}\times\vf{v}\|=\|\vf{u}\|\|\vf{v}\|\sin\theta$$
      \item $(\vf{u}\times\vf{v})\times\vf{w}=\langle\vf{u},\vf{w}\rangle\vf{v}-\langle\vf{v},\vf{w}\rangle\vf{u}$
      \item \emph{Jacobi identity}: $$(\vf{u}\times\vf{v})\times\vf{w}+(\vf{v}\times\vf{w})\times\vf{u}+(\vf{w}\times\vf{u})\times\vf{v}=\vf{0}$$
    \end{enumerate}
  \end{proposition}
  \begin{proposition}
    Let $\vf{\alpha},\vf{\beta}:I\rightarrow\RR^3$ be parametrized curves of class $\mathcal{C}^\infty$. Then:
    $$\dv{}{t}\left(\vf{\alpha}(t)\times\vf{\beta}(t)\right)=\vf{\alpha}'(t)\times\vf{\beta}(t)+\vf{\alpha}(t)\times\vf{\beta}'(t)$$
  \end{proposition}
  \subsubsection{Frenet-Serret formulas}
  \begin{definition}
    Let $\vf{\alpha}:I\rightarrow\RR^3$ be an arc-length parametrized curve of class $\mathcal{C}^\infty$. We define the unit tangent vector of $\vf{\alpha}$ at $s_0\in I$ as: $$\T\alpha(s_0):=\vf{\alpha}'(s_0)$$ Note that $\|\T\alpha\|=1$ and $\T\alpha\perp\T\alpha'$.
  \end{definition}
  \begin{definition}
    Let $\vf{\alpha}:I\rightarrow\RR^3$ be an arc-length parametrized curve of class $\mathcal{C}^\infty$. We define the \emph{curvature} of $\vf{\alpha}$ at $s_0\in I$ as: $$k_\alpha(s_0):=\|\vf{\alpha}''(s_0)\|=\|\T\alpha'(s_0)\|$$
  \end{definition}
  \begin{definition}
    Let $\vf{\alpha}:I\rightarrow\RR^3$ be an arc-length parametrized curve of class $\mathcal{C}^\infty$, $s_0\in I$ and suppose that $k_\alpha(s_0)\ne 0$. Then, we define the \emph{unit normal vector} of $\vf{\alpha}$ at $s_0$ as: $$\N\alpha(s_0):=\frac{\T\alpha'(s_0)}{k_\alpha(s_0)}=\frac{\vf{\alpha}''(s_0)}{\|\vf{\alpha}''(s_0)\|}$$
    Note that $\|\N\alpha\|=1$, $\N\alpha\perp\T\alpha$ and $\T\alpha'(s)=k_\alpha(s)\N\alpha(s)$ $\forall s\in I$.
  \end{definition}
  \begin{definition}
    Let $\vf{\alpha}:I\rightarrow\RR^3$ be a regular arc-length parametrized curve of class $\mathcal{C}^\infty$ such that $\vf{\alpha}''(s)\ne 0$ $\forall s\in I$. We define the \emph{binormal vector} of $\vf{\alpha}$ at $s_0\in I$ as:
    $$\B\alpha(s_0)=\T\alpha(s_0)\times\N\alpha(s_0)$$
    Then, the triplet $(\T\alpha(s_0),\ \N\alpha(s_0),\ \B\alpha(s_0))$ is an orthonormal positive basis\footnote{Observe that the binormal vector (together with the tangent and normal vectors) is the unique vector that satisfies this property.}, and the affine frame $\{\vf{\alpha}(s_0); (\T\alpha(s_0),\N\alpha(s_0),\B\alpha(s_0))\}$ is called \emph{Frenet-Serret frame} (or \emph{TNB frame}).
  \end{definition}
  \begin{proposition}
    Let $\vf{\alpha}:I\rightarrow\RR^3$ be a regular arc-length parametrized curve of class $\mathcal{C}^\infty$ such that $\vf{\alpha}''(s)\ne 0$ $\forall s\in I$. Then: $$\B\alpha'(s)=\ta\alpha(s)\N\alpha(s)\quad\forall s\in I$$ This coefficient $\ta\alpha(s)$ is called \emph{torsion} of $\vf{\alpha}$ at $s\in I$.
  \end{proposition}
  \begin{proposition}
    Let $\vf{\alpha}:I\rightarrow\RR^3$ be a regular arc-length parametrized curve of class $\mathcal{C}^\infty$ such that $\vf{\alpha}''(s)\ne 0$ $\forall s\in I$. The following statements are equivalent:
    \begin{enumerate}
      \item $\vf{\alpha}$ is a plane curve.
      \item $\B\alpha=\const$
      \item $\ta\alpha=0$.
    \end{enumerate}
  \end{proposition}
  \begin{theorem}[Frenet-Serret formulas]
    Let $\vf{\alpha}:I\rightarrow\RR^3$ be a regular arc-length parametrized curve of class $\mathcal{C}^\infty$ such that $\vf{\alpha}''(s)\ne 0$ $\forall s\in I$. Then:
    $$
      \begin{pmatrix}
        \T\alpha \\
        \N\alpha \\
        \B\alpha \\
      \end{pmatrix}'=
      \begin{pmatrix}
        0         & k_\alpha  & 0          \\
        -k_\alpha & 0         & -\ta\alpha \\
        0         & \ta\alpha & 0
      \end{pmatrix}
      \begin{pmatrix}
        \T\alpha \\
        \N\alpha \\
        \B\alpha \\
      \end{pmatrix}
    $$
  \end{theorem}
  \begin{definition}
    Let $\vf{\alpha}:I\rightarrow\RR^3$ be a regular arc-length parametrized curve of class $\mathcal{C}^\infty$ such that $\vf{\alpha}''(s)\ne 0$ $\forall s\in I$. and $s_0\in I$. We define the following planes of $\RR^3$:
    \begin{itemize}
      \item \emph{Osculating plane}: plane generated by $\T\alpha(s_0)$ and $\N\alpha(s_0)$ that contains $\vf{\alpha}(s_0)$.
      \item \emph{Normal plane}: plane generated by $\N\alpha(s_0)$ and $\B\alpha(s_0)$ that contains $\vf{\alpha}(s_0)$.
      \item \emph{Rectifying plane}: plane generated by $\T\alpha(s_0)$ and $\B\alpha(s_0)$ that contains $\vf{\alpha}(s_0)$.
    \end{itemize}
  \end{definition}
  \begin{proposition}
    Let $\vf{\alpha}:I\rightarrow\RR^3$ be a regular parametrized curve of class $\mathcal{C}^\infty$ and $h(t)=s(t)$ be the arc-length parameter. Let $\vf{\beta}=(\vf{\alpha}\circ h^{-1})(s)$, which is an arc-length parametrization of $\vf{\alpha}$. Then, assuming $\vf{\beta}''\ne 0$, we can define the TNB frame of $\vf{\alpha}$ as: $$\T\alpha:=\T\beta\circ h\qquad\N\alpha:=\N\beta\circ h\qquad\B\alpha:=\B\beta\circ h$$
    And the curvature and torsion of $\vf\alpha$ as: $$k_\alpha:=k_\beta\circ h\qquad\ta\alpha:=\ta\beta\circ h$$
  \end{proposition}
  \begin{lemma}
    Let $\vf{\alpha}:I\rightarrow\RR^3$ be a regular parametrized curve of class $\mathcal{C}^\infty$ and $h(t)=s(t)$ be the arc-length parameter. Let $\vf{\beta}=(\vf{\alpha}\circ h^{-1})(s)$. Then, $\vf{\beta}''=0\iff\vf\alpha'\times\vf\alpha''=0$.
  \end{lemma}
  \begin{proposition}
    Let $\vf{\alpha}:I\rightarrow\RR^3$ be a regular parametrized curve of class $\mathcal{C}^\infty$ such that $\vf{\alpha}'\times \vf{\alpha}''\ne 0$ and $v(t):=\|\vf{\alpha}'(t)\|$. Then:
    \begin{itemize}
      \item $\vf{\alpha}'=v\T\alpha$
      \item $\vf{\alpha}''=v'\T\alpha'+k_\alpha v^2\N\alpha$
    \end{itemize}
  \end{proposition}
  \begin{theorem}[General Frenet-Serret formulas]
    Let $\vf{\alpha}:I\rightarrow\RR^3$ be a regular parametrized curve of class $\mathcal{C}^\infty$ such that $\vf{\alpha}'\times \vf{\alpha}''\ne 0$ and $v(t):=\|\vf{\alpha}'(t)\|$. Then:
    $$
      \begin{pmatrix}
        \T\alpha \\
        \N\alpha \\
        \B\alpha \\
      \end{pmatrix}'=
      \begin{pmatrix}
        0           & k_\alpha v  & 0            \\
        -k_\alpha v & 0           & -\ta\alpha v \\
        0           & \ta\alpha v & 0
      \end{pmatrix}
      \begin{pmatrix}
        \T\alpha \\
        \N\alpha \\
        \B\alpha \\
      \end{pmatrix}
    $$
  \end{theorem}
  \begin{corollary}
    Let $\vf{\alpha}:I\rightarrow\RR^3$ be a regular parametrized curve of class $\mathcal{C}^\infty$ such that $\vf{\alpha}'\times \vf{\alpha}''\ne 0$. Then:
    $$\T\alpha=\frac{\vf{\alpha}'}{\|\vf{\alpha}'\|}\qquad\N\alpha=\B\alpha\times\T\alpha\qquad\B\alpha=\frac{\vf{\alpha}'\times\vf{\alpha}''}{\|\vf{\alpha}'\times\vf{\alpha}''\|}$$
    Moreover: $$k_\alpha=\frac{\|\vf{\alpha}'\times\vf{\alpha}''\|}{{\|\vf{\alpha}'\|}^3}\qquad\ta\alpha=-\frac{\langle\vf{\alpha}'\times\vf{\alpha}'',\vf{\alpha}'''\rangle}{{\|\vf{\alpha}'\times\vf{\alpha}''\|}^2}$$
  \end{corollary}
  \subsubsection{Contact between curves and surfaces}
  \begin{definition}
    Let $\vf{\alpha},\vf{\beta}:I\rightarrow\RR^n$ be arc-length parametrized curves of class $\mathcal{C}^\infty$ and $t_0\in I$. We say that $\vf{\alpha}$ and $\vf{\beta}$ have \emph{contact} of order $\geq r$ at $t_0$ if $$\lim_{t\to t_0}\frac{\vf{\alpha}(t)-\vf{\beta}(t)}{{(t-t_0)}^r}=\vf{0}$$
    We say that $\vf{\alpha}$ and $\vf{\beta}$ have \emph{contact} of order $r$ at $t_0$ if they have contact of order $\geq r$ but not contact of order $\geq r+1$.
  \end{definition}
  \begin{proposition}
    Let $\vf{\alpha},\vf{\beta}:I\rightarrow\RR^n$ be arc-length parametrized curves of class $\mathcal{C}^\infty$ and $t_0\in I$. Then, $\vf{\alpha}$ and $\vf{\beta}$ have contact of order $\geq r$ at $t_0$ if and only if: $$\vf{\alpha}^{(k)}(t_0)=\vf{\beta}^{(k)}(t_0)\quad\text{for }k=0,\ldots,r$$
  \end{proposition}
  \begin{proposition}
    Let $\vf{\alpha}:I\rightarrow\RR^3$ be an arc-length parametrized curve of class $\mathcal{C}^\infty$ and $s_0\in I$. Then, the \emph{tangent line} at $\vf{\alpha}(s_0)$ is the unique line that has contact of order $\geq 1$ with $\alpha$ at this point. An arc-length parametrization of the tangent line is, for example: $$u\longmapsto \vf\alpha(s_0)+u\T\alpha(s_0)$$
  \end{proposition}
  \begin{proposition}
    Let $\vf{\alpha}:I\rightarrow\RR^3$ be an arc-length parametrized curve of class $\mathcal{C}^\infty$, $s_0\in I$ and suppose that $k_\alpha(s_0)\ne 0$. Then, there exists a unique circle of $\RR^3$ that has contact of order $\geq 2$ at $\vf{\alpha}(s_0)$. This circle is called \emph{osculating circle} and its radius (called \emph{radius of curvature}) is $\rho_{\vf{\alpha}}(s_0):=\frac{1}{k_\alpha(s_0)}$. Its center is $C(s_0)=\vf{\alpha}(s_0)+\rho_{\vf{\alpha}}(s_0)\N\alpha(s_0)$\footnote{An arc-length parametrization of the osculating circle is, for example: $$u\longmapsto C(s_0)+\rho_{\vf{\alpha}}(s_0)\left(-\cos\left(\frac{u}{\rho_{\vf{\alpha}}(s_0)}\right)\N\alpha(s_0)+\sin\left(\frac{u}{\rho_{\vf{\alpha}}(s_0)}\right)\T\alpha(s_0)\right)$$}.
  \end{proposition}
  \begin{proposition}
    Let $\vf{\alpha}:I\rightarrow\RR^2$ be an regular parametrized curve of class $\mathcal{C}^\infty$, $s_0\in I$ and suppose that $k_\alpha(s_0)\ne 0$. If $\vf\alpha(t)=(x(t),y(t))$, then the center of the osculating circle at $\vf\alpha(s_0)$ has coordinates $(X,Y)$ given by:
    $$X=x+y'\frac{{x'}^2+{y'}^2}{x''y'-x'y''}\quad Y=y-x'\frac{{x'}^2+{y'}^2}{x''y'-x'y''}$$
  \end{proposition}
  \begin{center}
    \begin{minipage}{\linewidth}
      \centering
      \includestandalone[mode=image|tex,width=0.8\linewidth]{Images/oscu-circle}
      \captionof{figure}{Osculating circle of a cycloid at some point}
    \end{minipage}
  \end{center}
  \begin{definition}
    Let $\vf{\alpha}:I\rightarrow\RR^3$ be an arc-length parametrized curve of class $\mathcal{C}^\infty$, $s_0\in I$ and suppose that $k_\alpha(s_0)\ne 0$. Then, there exists a unique sphere of $\RR^3$ that has contact of order $\geq 3$ at $\vf{\alpha}(s_0)$. This sphere is called \emph{osculating sphere} of $\vf\alpha$ at $\vf{\alpha}(s_0)$ and its center $\vf{c}(s_0)$ and radius $r(s_0)$ are given by:
    \begin{align*}
      \vf{c}(s_0) & =\vf\alpha(s_0)+\rh\alpha(s_0)\N\alpha(s_0)-\frac{\rh\alpha'(s_0)}{\ta\alpha'(s_0)}\B\alpha(s_0) \\
      {r(s_0)}^2  & =\rh\alpha(s_0)+{\left(\frac{\rh\alpha'(s_0)}{\ta\alpha(s_0)}\right)}^2
    \end{align*}
  \end{definition}
  \subsubsection{Envolute and involute}
  \begin{definition}
    An \emph{envelope} of a family of plane curves is a curve that is tangent to each of the members of the family at some point.
  \end{definition}
  \begin{definition}
    Let $\vf{\alpha},\vf\beta:I\rightarrow\RR^2$ be regular parametrized curves of class $\mathcal{C}^\infty$ such that $k_\alpha(s),\ta\alpha(s)\ne 0$ $\forall s\in I$. We say that $\vf\beta$ is the \emph{evolute} of $\vf\alpha$ if $\vf\beta$ is the envelope of all the normal lines to $\vf\alpha$.
  \end{definition}
  \begin{proposition}
    Let $\vf{\alpha}:I\rightarrow\RR^2$ be regular parametrized curves of class $\mathcal{C}^\infty$ such that $k_\alpha(s)\ne 0$ $\forall s\in I$. Then, a parametrization of the evolute of $\vf\alpha$ is: $$t\longmapsto \vf\alpha(t)+\rh\alpha\N\alpha(t)$$
  \end{proposition}
  \begin{definition}
    Let $\vf{\alpha},\vf\beta:I\rightarrow\RR^2$ be regular parametrized curves of class $\mathcal{C}^\infty$. We say that $\vf\beta$ is the \emph{involute} of $\vf\alpha$ if $\vf\beta$ is intersects orthogonally all the tangent lines to $\vf\alpha$.
  \end{definition}
  \begin{proposition}
    Let $\vf{\alpha}:I\rightarrow\RR^2$ be regular parametrized curves of class $\mathcal{C}^\infty$ such that $k_\alpha(s)\ne 0$ $\forall t_0\in I$ and $s_0\in I$. Then, a parametrization of the involute of $\vf\alpha$ is: $$t\longmapsto \vf\alpha(t)-\T\alpha(t) \int_{t_0}^t\|\vf{\alpha}'(u)\|\dd{u}$$
  \end{proposition}
  \begin{proposition}
    The evolute of the involute of a curve $\vf\alpha$ is the curve $\vf\alpha$ itself.
  \end{proposition}
  \begin{center}
    \begin{minipage}{\linewidth}
      \centering
      \includestandalone[mode=image|tex,width=\linewidth]{Images/involute-evolute}
      \captionof{figure}{Construction of the evolute and involute of a curve}
    \end{minipage}
  \end{center}
  \subsubsection{Curvature of plane curves}
  \begin{lemma}
    Let $a,b:I\rightarrow\RR$ be differentiable functions such that $a^2+b^2=1$, $t_0\in I$ and $\theta_0\in\RR$ be such that $a(t_0)=\cos\theta_0$ and $b(t_0)=\sin\theta_0$. Then, the differentiable function $\theta(t)$ defined as:
    $$\theta(t)=\theta_0+\int_{t_0}^t\left(a(t)b'(u)-a'(u)b(u)\right)\dd{u}$$
    satisfies $a(t)=\cos\theta(t)$, $b(t)=\sin\theta(t)$ and $\theta(t_0)=\theta_0$ $\forall t\in I$.
  \end{lemma}
  \begin{proposition}
    Let $\vf{\alpha}:I\rightarrow\RR^2$ be a regular arc-length parametrized curve of class $\mathcal{C}^\infty$. Then, there is a unique vector $\vf{\hat{N}}_{\vf\alpha}$ such that $(\T\alpha,\vf{\hat{N}}_{\vf\alpha})$ is a positive orthonormal basis of $\RR^2$. Thus, $\T\alpha'\parallel \vf{\hat{N}}_{\vf\alpha}$.
  \end{proposition}
  \begin{definition}
    Let $\vf{\alpha}:I\rightarrow\RR^2$ be a regular arc-length parametrized curve of class $\mathcal{C}^\infty$ ans $s_0\in I$. We define the \emph{signed curvature} of $\vf\alpha$ at $\vf\alpha(s_0)$ as the value $\ka{\vf\alpha}(s_0)$ satisfying $\T\alpha'(s_0)=\ka{\vf\alpha}(s_0)\vf{\hat{N}}_{\vf\alpha}(s_0)$\footnote{Using the notation of the last proposition, note that $\vf{\hat{N}}_{\vf\alpha}=\pm\N\alpha$ and therefore $\ka{\vf\alpha}=\pm k_\alpha$.}. Moreover: $$\ka{\vf\alpha}=\det(\T\alpha,\T\alpha')$$
  \end{definition}
  \begin{proposition}
    Let $\vf{\alpha}:I\rightarrow\RR^2$ be a regular parametrized curve of class $\mathcal{C}^\infty$. Then, the signed curvature of $\vf\alpha$ is: $$\ka{\vf\alpha}=\frac{\det(\vf\alpha',\vf\alpha'')}{{\norm{\vf\alpha'}}^3}$$
  \end{proposition}
  \subsubsection{Local form of a curve}
  \begin{definition}
    Let $\vf{\alpha}:I\rightarrow\RR^2$ be a regular arc-length parametrized curve of class $\mathcal{C}^\infty$ and $s_0\in I$. Consider the affine frame of reference $\mathcal{R}=\{\vf{\alpha}(s_0);(\T\alpha,\N\alpha,\B\alpha)\}$ and suppose $\vf{\alpha}(s)_{\mathcal{R}}=(x(s),y(s),z(s))$. Then:
    $$\left\{
      \def\arraystretch{2}
      \begin{array}{l}
        \displaystyle x(s)\simeq s-\frac{{k_\alpha(0)}^2}{6}s^3                         \\
        \displaystyle y(s)\simeq \frac{{k_\alpha(0)}^2}{2}s^2-\frac{k_\alpha'(0)}{6}s^3 \\
        \displaystyle z(s)\simeq -\frac{k_\alpha(0)\ta\alpha(0)}{6}s^3                  \\
      \end{array}
      \right.
    $$
    This expression of $\vf{\alpha}(s)_{\mathcal{R}}$ is called \emph{local canonical form} of $\vf{\alpha}$ in a neighbourhood of $s_0$.
  \end{definition}
  \begin{corollary}
    Let $\vf{\alpha}:I\rightarrow\RR^2$ be a regular arc-length parametrized curve of class $\mathcal{C}^\infty$ and $s_0\in I$. Then, in the reference $\mathcal{R}=\{\vf{\alpha}(s_0);(\T\alpha,\N\alpha,\B\alpha)\}$ we have:
    \begin{itemize}
      \item If $\tau <0$, at $s=0$ the curve cross the osculating plane towards the direction that points $\B\alpha$ \emph{dextrorotation}.
      \item If $\tau <0$, at $s=0$ the curve cross the osculating plane towards the opposite direction that points $\B\alpha$ \emph{levorotation}.
    \end{itemize}
  \end{corollary}
  \subsubsection{Orthogonal group}
  \begin{definition}
    We define that \emph{orthogonal group} as the group of all linear transformations that preserve the inner product. That is: $$\text{O}(n):=\{\vf{A}\in\mathcal{M}_n(\RR):\langle\vf{Au},\vf{Av}\rangle=\langle\vf{u},\vf{v}\rangle\;\forall\vf{u},\vf{v}\in\RR^n\}$$
  \end{definition}
  \begin{proposition}
    Let $\vf{A}\in\text{O}(n)$. Then, $\det\vf{A}=\pm$ and $\vf{A}\transpose{\vf{A}}=\vf{I}_n$.
  \end{proposition}
  \begin{definition}
    We define that \emph{special orthogonal group} as: $$\text{SO}(n):=\{\vf{A}\in\text{O}(n):\det\vf{A}=1\}$$
  \end{definition}
  \begin{lemma}
    Let $\vf{A}\in\text{O}(n)$ and $\lambda\in\sigma(\vf{A})$. Then, $\lambda\in\RR\implies\lambda=\pm 1$.
  \end{lemma}
  \begin{proposition}
    Let $\vf{A}\in\text{O}(n)$. Then:
    $$\left\{
      \begin{array}{lll}
        \vspace{0.1cm}
        \vf{A}=\begin{pmatrix}
          \cos\omega & -\sin\omega \\
          \sin\omega & \cos\omega  \\
        \end{pmatrix} &
        \text{ if }                       & \det\vf{A}=1  \\
        \vf{A}=\begin{pmatrix}
          \cos\omega & \sin\omega  \\
          \sin\omega & -\cos\omega \\
        \end{pmatrix} &
        \text{ if }                       & \det\vf{A}=-1
      \end{array}
      \right.
    $$
    for some $\omega\in\RR$.
  \end{proposition}
  \begin{proposition}
    Let $\vf{A}\in\text{O}(3)$. Then, there exists a orthonormal basis $\mathcal{B}$ of $\RR^3$ such that $${[\id]_{\mathcal{B},\text{Can}(\RR^3)}}^{-1}\vf{A}[\id]_{\mathcal{B},\text{Can}(\RR^3)}=\begin{pmatrix}
        \pm 1 & 0                                              & 0 \\
        0     & \multicolumn{2}{c}{\multirow{2}{*}{$\vf{A}'$}}     \\
        0     &                                                &
      \end{pmatrix}$$
    where $\vf{A}'\in\text{O}(2)$.
  \end{proposition}
  \begin{proposition}
    Let $\vf{f}:\RR^n\rightarrow\RR^n$ be an Euclidean motion\footnote{Recall that an Euclidean motion is a function that preserves the distance, that is, if $\vf{f}:\RR^n\rightarrow\RR^n$ is an Euclidean motion, then $\|\vf{f}(p)-\vf{f}(q)\|=\|p-q\|$ $\forall p,q\in\RR^n$.}. Then, $\exists\vf{A}\in\text{O}(n)$ and $\vf{u}\in\RR^n$ such that: $$\vf{f}(\vf{v})=\vf{Av}+\vf{u}$$
  \end{proposition}
  \begin{proposition}
    Let $\vf{\alpha}:I\rightarrow\RR^n$ be a parametrized curve of class $\mathcal{C}^\infty$ and $\vf{A}\in\mathcal{M}_n(\RR)$. Then:
    $${\left(\vf{A}\vf{\alpha}\right)}^{'}(t)=\vf{A}\vf{\alpha}'(t)$$
  \end{proposition}
  \begin{proposition}
    Let $\vf{A}\in\text{SO}(3)$. Then, $\forall \vf{u},\vf{v}\in\RR^3$ we have: $$\vf{A}\left(\vf{u}\times\vf{v}\right)=\left(\vf{Au}\right)\times\left(\vf{Av}\right)$$
  \end{proposition}
  \begin{corollary}
    Let $\vf{\alpha}:I\rightarrow\RR^3$ be an arc-length parametrized curve of class $\mathcal{C}^\infty$ and $\vf{\beta}:=\vf{A\alpha}+\vf{u}$, where $\vf{A}\in\text{SO}(3)$ and $\vf{u}\in\RR^3$. Then, $\vf{\beta}$ is arc-length parametrized and th TNB frame of $\vf{\beta}$ is:
    $$\T\beta=\vf{A}\T\alpha\qquad\N\beta=\vf{A}\N\beta\qquad\B\beta=\vf{A}\B\alpha$$
    And the curvature and torsion of $\vf\beta$ are: $$k_\beta:=k_\alpha\qquad\ta\beta:=\ta\alpha$$
  \end{corollary}
  \subsubsection{Fundamental theorem of curves}
  \begin{theorem}[Fundamental theorem of curves]
    Let $k_ppa,\tau:I\rightarrow\RR$ be functions of class $\mathcal{C}^\infty$ with $k_ppa(s)>0$ $\forall s\in I$. Then, there is an arc-length parametrized curve $\vf{\alpha}:I\rightarrow\RR^3$ of class $\mathcal{C}^\infty$ whose curvature and torsion are $k_ppa$ and $\tau$, respectively. Moreover, if $\tilde{\vf{\alpha}}:I\rightarrow\RR^3$ is another curve satisfying these restrictions, then there exists an Euclidean motion that carries $\tilde{\vf{\alpha}}$ to $\vf{\alpha}$\footnote{In matrix form, $\exists\vf{A}\in\text{SO}(3)$ and $\vf{u}\in\RR^3$ such that $\vf{\alpha}=\vf{A}\tilde{\vf{\alpha}}+\vf{u}$.}.
  \end{theorem}
  \subsection{Submanifolds of \texorpdfstring{$\RR^n$}{Rn}}
  \subsubsection{Planar functions}
  \begin{definition}
    Let $U\subseteq\RR^n$ be an open set and $f:U\rightarrow\RR$ be a function. We define the \emph{support} of $f$ as:
    $$\supp(f):=\Cl\left(\{x\in U:f(x)\ne 0\}\right)$$
  \end{definition}
  \begin{lemma}
    Let $x_0\in\RR^n$ and $a,b\in \RR_{>0}$ with $a<b$. Then, there exists a function $\rho:\RR^n\rightarrow[0,1]$ of class $\mathcal{C}^\infty$ such that $\supp(\rho)\subseteq\overline{B(x_0,b)}$ and $\rho|_{B(x_0,a)}=1$.
  \end{lemma}
  \begin{proposition}
    Let $U\subseteq\RR^n$ be an open set and $K\subset U$ be a compact set. Then, there exists a function $\rho:\RR^n\rightarrow[0,1]$ of class $\mathcal{C}^\infty$ such that $\supp(\rho)\subseteq U$ such that $\rho|_K=1$.
  \end{proposition}
  \begin{corollary}
    Let $U\subseteq\RR^n$ be an open set, $K\subset U$ be a compact set and $f:U\rightarrow \RR$ be a function of class $\mathcal{C}^\infty$. Then, there exists a function $\tilde{f}:U\rightarrow \RR$ such that $\tilde{f}|_K=f|_K$ and $\tilde{f}|_{\RR^n\setminus U}=0$.
  \end{corollary}
  \subsubsection{Immersions and submersion}
  \begin{definition}[Immersion]
    Let $U\subseteq\RR^n$ be an open set and $\vf{f}:U\rightarrow \RR^m$ be a function of class $\mathcal{C}^\infty$. We say that $\vf{f}$ is an \emph{immersion} at $x_0\in U$ if $\vf{Df}(x_0)$ is injective. We say that $\vf{f}$ is an \emph{immersion} (on $U$) if it is an immersion at each point $x\in U$.
  \end{definition}
  \begin{definition}[Submersion]
    Let $U\subseteq\RR^n$ be an open set and $\vf{f}:U\rightarrow \RR^m$ be a function of class $\mathcal{C}^\infty$. We say that $\vf{f}$ is a \emph{submersion} at $x_0\in U$ if $\vf{Df}(x_0)$ is surjective. We say that $\vf{f}$ is a \emph{submersion} (on $U$) if it is a submersion at each point $x\in U$.
  \end{definition}
  \begin{theorem}[Local structure of immersions]
    Let $U\subseteq\RR^n$ be an open set, $\vf{f}:U\rightarrow \RR^m$ be an immersion at $x_0\in U$ and $\vf{\iota}:\RR^n\rightarrow\RR^m$ be the inclusion map. Then, there exist neighbourhoods $V\subseteq U$ of $x_0$ and $W\subseteq\RR^m$ of $\vf\iota(x_0)$ and a diffeomorphism $\vf{g}:W\rightarrow \vf{g}(W)$ such that the following diagram is commutative, that is, $\vf{f}=\vf{g}\circ\vf\iota$.
    \begin{center}
      \begin{minipage}{\linewidth}
        \centering
        \includestandalone[mode=image|tex,width=0.45\linewidth]{Images/theorem_immersions}
        \captionof{figure}{}
      \end{minipage}
    \end{center}
  \end{theorem}
  \begin{theorem}[Local structure of submersions]
    Let $U\subseteq\RR^n$ be an open set, $\vf{f}:U\rightarrow \RR^m$ be a submersion at $x_0\in U$ and $\vf\pi_1:\RR^m\times\RR^{n-m}\rightarrow\RR^m$ be the projection map into the first coordinate. Then, there exists a neighbourhood $V\subseteq U$ of $x_0$ and a diffeomorphism $\vf{g}:V\rightarrow \vf{g}(V)$ such that the following diagram is commutative, that is, $\vf{f}=\vf\pi_1\circ\vf{g}$.
    \begin{center}
      \begin{minipage}{\linewidth}
        \centering
        \includestandalone[mode=image|tex,width=0.45\linewidth]{Images/theorem_submersions}
        \captionof{figure}{}
      \end{minipage}
    \end{center}
  \end{theorem}
  \subsubsection{Submanifolds of \texorpdfstring{$\RR^n$}{Rn}}
  \begin{definition}
    Let $M\subseteq\RR^n$ be a set. We say that $M$ is a \emph{submanifold} of $\RR^n$ of dimension $p$ (and codimension $q:=n-p$) if $\forall z\in M$ there exists a neighbourhood $U\subseteq \RR^n$ of $z$ and a diffeomorphism $\vf{g}:U\rightarrow\vf{g}(U)$ such that: $$\vf{g}(U\cap M)=\vf{g}(U)\cap\left(\RR^p\times\{0\}\right)$$
  \end{definition}
  \begin{theorem}
    Let $M\subseteq\RR^n$ be a set. The following statements are equivalent:
    \begin{enumerate}
      \item $M$ is a submanifold of $\RR^n$ of dimension $p$ and codimension $q$.
      \item $\forall z\in M$ there exists a neighbourhood $U\subseteq \RR^n$ of $z$ and a submersion $\vf\phi:U\rightarrow\RR^q$ such that $U\cap M={\vf\phi}^{-1}(0)$.
      \item $\forall z\in M$ there exists a neighbourhood $V\subseteq \RR^p$ of $z$ and an immersion $\vf\varphi:V\rightarrow\RR^n$ such that $z\in\vf\varphi(V)\subseteq M$ and $\vf\varphi:V\rightarrow\vf\varphi(V)$ is a homeomorphism.
    \end{enumerate}
  \end{theorem}
  \begin{definition}
    Let $M\subseteq\RR^n$ be a submanifold, $V\subseteq \RR^p$ and $\vf\varphi:V\rightarrow\vf\varphi(V)\subseteq M$ be an immersion and a homeomorphism. We say that the pair $(V,\vf\varphi)$ is a \emph{parametrization} of $M$ and the pair $(\vf\varphi(V),{\vf\varphi}^{-1})$, a \emph{coordinate chart}.
  \end{definition}
  \begin{proposition}
    Let $(V_1,\vf\varphi_1)$, $(V_2,\vf\varphi_2)$ be two parametrizations of a submanifold $M\subseteq\RR^n$. Then, the composition ${\vf\varphi_2}^{-1}\circ\vf\varphi_1$ is differentiable on its domain.
  \end{proposition}
  \begin{proposition}
    Let $M\subseteq\RR^n$ be a submanifold of $\RR^n$ of dimension $p$, $V\subseteq\RR^p$ be an open set and $\vf\varphi:V\rightarrow M$ be a differentiable injective immersion. Then, $\vf\varphi(\Omega)\subseteq M$ is an open set and $\vf\varphi:V\rightarrow \vf\varphi(V)$ is a homeomorphism. Hence, $(V,\vf\varphi)$ is a parametrization of $M$.
  \end{proposition}
  \subsubsection{Surfaces of \texorpdfstring{$\RR^3$}{R3}}
  \begin{definition}
    A submanifold of $\RR^3$ of dimension 2 is called a \emph{regular surface} (or simply \emph{surface}) of $\RR^3$.
  \end{definition}
  \begin{proposition}
    Let $S\subseteq\RR^3$ be a set. Then, $S$ is a surface if and only if $\forall z\in S$ there exists an open neighbourhood $U\subseteq \RR^3$ of $z$, a change of variables $\vf\sigma$, a neighbourhood $V\subseteq \RR^2$ of $\vf\pi_1(\sigma(z))$ (where $\vf\pi_1:\RR^2\times \RR\rightarrow\RR^2$ is the projection map) and a differentiable function $h:V\rightarrow \RR$ such that: $$\vf\sigma(S\cap U)=\graph(h)$$
  \end{proposition}
  \subsubsection{Differentiable functions}
  \begin{definition}
    Let $S\subseteq\RR^3$ be a surface. We say that a function $\vf{f}:S\rightarrow \RR^n$ is \emph{differentiable} at a point $p\in S$ if there is a local parametrization $(V,\vf\varphi)$ of $S$ with $p\in\vf\varphi(V)$ such that $\vf{f}\circ\vf\varphi$ is differentiable at ${\vf\varphi}^{-1}(p)$. We say that $\vf{f}$ is \emph{differentiable} on $S$ if it is differentiable at each point $p\in S$.
  \end{definition}
  \begin{proposition}
    Let $S\subseteq\RR^3$ be a surface, $\vf{f}:S\rightarrow \RR^n$ be a differentiable function and $p\in S$. Then, there exists an open neighbourhood $U\subseteq\RR^3$ of $p$ and a $\vf{\tilde{f}}:S\rightarrow \RR^n$ such that $\vf{\tilde{f}}|_{U\cap S}=\vf{f}|_{U\cap S}$.
  \end{proposition}
  \begin{corollary}
    Let $S\subseteq\RR^3$ be a surface, $U\subseteq \RR^n$ and $\vf{f}:U\rightarrow \RR^3$ be a differentiable function such that $\vf{U}\subseteq S$. If $(V,\vf\varphi)$ is a local parametrization of $S$, then ${\vf\varphi}^{-1}\circ\vf{f}$ is also a differentiable function on its domain.
  \end{corollary}
  \begin{corollary}
    Let $S\subseteq\RR^3$ be a surface, $(V,\vf\varphi(u,v))$ be a local parametrization of $S$ and $\vf\alpha:I\rightarrow\RR^3$ be a curve of class $\mathcal{C}^\infty$ such that $\vf\alpha(I)\subset \vf\varphi(V)$. Then, $\vf\alpha$ can be written as $\vf\alpha(t)=\vf\varphi(u(t),v(t))$, where $u(t)$, $v(t)$ are differentiable functions.
  \end{corollary}
  \begin{definition}
    Let $S_1,S_2\subseteq\RR^3$ be surfaces. We say that a function $\vf{f}:S_1\rightarrow S_2$ is differentiable if $\forall p\in S_1$, there exist parametrizations $(V_1,\vf\varphi_1)$ and $(V_2,\vf\varphi_2)$ of $S_1$ and $S_2$ respectively with $p\in\vf\varphi_1$ and $\vf{f}(p)\in\vf\varphi_2$ and such that ${\vf\varphi_2}^{-1}\circ\vf{f}\circ\vf\varphi_1$ is differentiable on its domain.
  \end{definition}
  \subsubsection{Tangent space}
  \begin{definition}
    Let $S\subseteq\RR^3$ be a surface and $p\in S$. If $\vf\alpha:(-\varepsilon,\varepsilon)\rightarrow\RR^3$ is a curve of class $\mathcal{C}^\infty$ such that $\vf\alpha(0)=p$, we say that $\vf\alpha'(0)$ is a \emph{tangent vector} to $S$ at $p$. The set of all such vectors is called \emph{tangent spaces} (or \emph{tangent plane}) and it is denoted as $T_pS$.
  \end{definition}
  \begin{proposition}
    Let $S\subseteq\RR^3$ be a surface, $p\in S$, $(V,\vf\varphi)$ be a local parametrization of $S$ with $p\in\vf\varphi(V)$ and $f:U\rightarrow\RR$ be a submersion with $S\cap U=f^{-1}(0)$. Then: $$\im\vf{D\varphi}({\vf\varphi}^{-1}(p))=T_pS=\ker \vf{D\varphi}(p)$$
    Therefore, $\dim T_pS=2$.
  \end{proposition}
  \begin{proposition}
    Let $S\subseteq\RR^3$ be a surface, $(V,\vf\varphi(u,v))$ be a local parametrization of $S$ and $p=\vf\varphi(u_0,v_0)\in S$. Then, the tangent vectors $$\left(\pdv{\vf\varphi}{u}(u_0,v_0),\pdv{\vf\varphi}{v}(u_0,v_0)\right)\footnote{Usually we will denote these partial derivatives by $\vf\varphi_u=\pdv{\vf\varphi}{u}(u_0,v_0)$ and $\vf\varphi_v=\pdv{\vf\varphi}{v}(u_0,v_0)$, respectively.}$$ form a basis of the tangent plane $T_pS$.
  \end{proposition}
  \begin{lemma}
    Let $U\subseteq \RR^n$ be an open set, $\vf{f}:U\rightarrow\RR^m$ be a differentiable function and $\vf\alpha:(-\varepsilon,\varepsilon)\rightarrow U$ be a curve of class $\mathcal{C}^\infty$ such that $\vf\alpha(0)=p$ and $\vf\alpha'(0)=\vf{v}$. Then: $$\vf{Df}(p)\vf{v}={(\vf{f}\circ\vf\alpha)}'(0)$$
  \end{lemma}
  \begin{definition}
    Let $S_1,S_2\subseteq\RR^3$ be surfaces, $p\in S_1$ and $\vf{f}:S_1\rightarrow S_2$ be a differentiable function. We define the \emph{tangent function} (or \emph{differential}) of $\vf{f}$ at $p$ as the function:
    $$\function{\vf{Df}_p}{T_pS_1}{T_{\vf{f}(p)}S_2}{\vf{v}}{{(\vf{f}\circ\vf\alpha)}'(0)}$$ where $\vf\alpha:(-\varepsilon,\varepsilon)\rightarrow S_1$ is a curve of class $\mathcal{C}^\infty$ such that $\vf\alpha(0)=p$ and $\vf\alpha'(0)=\vf{v}$.
  \end{definition}
  \begin{proposition}
    Let $S_1,S_2\subseteq\RR^3$ be surfaces,  $p\in S_1$ and $\vf{f}:S_1\rightarrow S_2$ be a differentiable function. Then, $\vf{Df}_p$ is linear. Moreover if $(V_1,\vf\varphi_1(u,v))$ and $(V_2,\vf\varphi_2(\tilde{u},\tilde{v}))$ are parametrizations of $S_1$ and $S_2$ respectively, $\tilde{u}=f_1(u,v)$, $\tilde{v}=f_2(u,v)$\footnote{That is, $f_1$ and $f_2$ are the component functions of ${\vf\varphi_2}^{-1}\circ\vf{f}\circ\vf\varphi_1$.} and $\mathcal{B}_1=\left(\pdv{\vf\varphi_1}{u},\pdv{\vf\varphi_1}{v}\right)$, $\mathcal{B}_2=\left(\pdv{\vf\varphi_2}{\tilde{u}},\pdv{\vf\varphi_2}{\tilde{v}}\right)$, we have that $$[\vf{Df}_p]_{\mathcal{B}_1,\mathcal{B}_2}=
      \renewcommand\arraystretch{2}
      \begin{pmatrix}
        \displaystyle\pdv{f_1}{u}({\vf{\varphi}}^{-1}(p)) & \displaystyle\pdv{f_1}{v}({\vf{\varphi}}^{-1}(p)) \\
        \displaystyle\pdv{f_2}{u}({\vf{\varphi}}^{-1}(p)) & \displaystyle\pdv{f_2}{v}({\vf{\varphi}}^{-1}(p))
      \end{pmatrix}$$
  \end{proposition}
  \begin{proposition}
    Let $S_1,S_2\subseteq\RR^3$ be surfaces, $p\in S_1$ and $\vf{f}:S_1\rightarrow S_2$ be a differentiable function. Suppose $\vf{Df}_p$ is an isomorphism. Then, $\vf{f}$ is a diffeomorphism between a neighbourhood $U_1\subseteq S_1$ of $p$ and a neighbourhood $U_2\subseteq S_2$ of $\vf{f}(p)$.
  \end{proposition}
  \subsection{First fundamental form}
  \subsubsection{First fundamental form}
  \begin{definition}
    Let $S\subseteq\RR^3$ be a surface and $p\in S$. We define the \emph{first fundamental form} of $S$ at $p$ as the quadratic form $\I_p$ of $T_pS$ defined by:
    $$\I_p(\vf{v}):={\langle\vf{v},\vf{v}\rangle}_p:={\|\vf{v}\|}^2$$
  \end{definition}
  \begin{proposition}
    Let $S\subseteq\RR^3$ be a surface, $(V,\vf\varphi(u,v))$ be a local parametrization of $S$ and $p\in S$. Then, in the basis $(\vf\varphi_u,\vf\varphi_v)$ we have:
    $$
      \I_p=\begin{pmatrix}
        E & F \\
        F & G
      \end{pmatrix}:=\begin{pmatrix}
        {\langle\vf\varphi_u,\vf\varphi_u\rangle}_p & {\langle\vf\varphi_u,\vf\varphi_v\rangle}_p \\
        {\langle\vf\varphi_v,\vf\varphi_u\rangle}_p & {\langle\vf\varphi_v,\vf\varphi_v\rangle}_p
      \end{pmatrix}
    $$
    That is, if $\vf{u}=a\vf\varphi_u+b\vf\varphi_v$ and $\vf{v}=c\vf\varphi_u+d\vf\varphi_v$, then
    $${\langle\vf{u},\vf{v}\rangle}_p=\begin{pmatrix}
        a & b
      \end{pmatrix}\begin{pmatrix}
        E & F \\
        F & G
      \end{pmatrix}\begin{pmatrix}
        c \\
        d
      \end{pmatrix}$$
    and $$\I_p(\vf{u})=a^2 E+2ab F+b^2G$$
  \end{proposition}
  \begin{definition}
    Let $S\subseteq\RR^3$ be a surface, $(V,\vf\varphi(u,v))$ be a local parametrization of $S$ and $p\in S$. We say that the parametrization $(V,\vf\varphi(u,v))$ is orthogonal or that $u$ and $v$ are \emph{orthogonal cooridnates} if $F=0$.
  \end{definition}
  \begin{proposition}
    Let $S\subseteq\RR^3$ be a surface, $(V,\vf\varphi(u,v))$ be a local parametrization of $S$ and $\vf\alpha:I\rightarrow\vf\varphi(V)$. We can write $\vf\alpha$ as $\vf\alpha(t)=\vf\varphi(u(t),v(t))$. Then:
    $$\|\vf\alpha'(t)\|=\sqrt{{u'(t)}^2 E+2u'(t)v'(t) F+{v'(t)}^2G}$$ where $E=E(u(t),v(t))$, $F=F(u(t),v(t))$, $G=G(u(t),v(t))$. The arc-length parameter is thus:
    $$s(t)=\int_{t_0}^t\sqrt{{u'}^2 E+2u'v' F+{v'}^2G}\dd \xi$$
  \end{proposition}
  \begin{proposition}
    Let $S\subseteq\RR^3$ be a surface, $(V,\vf\varphi(u,v))$ be a local parametrization of $S$ and $p\in S$. Then, the angle $\beta$ between the coordinates lines of the parametrization $(V,\vf\varphi(u,v))$ is: $$\beta=\frac{\langle\vf\varphi_u,\vf\varphi_v\rangle}{\|\vf\varphi_u\|\|\vf\varphi_v\|}=\frac{F}{\sqrt{EG}}$$
  \end{proposition}
  \subsubsection{Area}
  \begin{definition}
    Let $S\subseteq\RR^3$ be a surface and $D\subseteq S$ be a subset. We say that $D$ is a \emph{regular domain} (or simply \emph{domain}) if $D$ is open, connected and $\Fr D\subset S$ is the image of a $\mathcal{C}^1$-piecewise curve. A \emph{region} $R\subseteq S$ is the union of a domain $D$ with its boundary, $R=D\cup\Fr D$.
  \end{definition}
  \begin{definition}
    Let $S\subseteq\RR^3$ be a surface, $(V,\vf\varphi(u,v))$ be a local parametrization of $S$ and $R\subset \vf\varphi(V)$ be a compact region. Let $Q={\vf\varphi}^{-1}(R)\subseteq\RR^2$. We define the area of $R$ as:
    $$\area(R)=\iint_Q\|\vf\varphi_u\times\vf\varphi_v\|\dd{u}\dd{v}=\iint_Q\sqrt{EG-F^2}\dd{u}\dd{v}\footnote{One can check that this definition does not depend on the parametrization $(V,\vf\varphi(u,v))$ of $S$.}$$
  \end{definition}
  \begin{definition}
    Let $S\subseteq\RR^3$ be a surface, $(V,\vf\varphi(u,v))$ be a local parametrization of $S$, $R\subset \vf\varphi(V)$ be a compact region and $f:S\rightarrow \RR$ be a function. Let $Q={\vf\varphi}^{-1}(R)\subseteq\RR^2$. We define the \emph{integral of $f$ over the region $R$} as: $$\iint_Rf\dd{S}:=\iint_Q(f\circ\vf\varphi)\sqrt{EG-F^2}\dd{u}\dd{v}\footnote{One can check that this definition does not depend on the parametrization $(V,\vf\varphi(u,v))$ of $S$.}$$
  \end{definition}
  \subsubsection{Isometries}
  \begin{definition}
    Let $S_1,S_2\subseteq\RR^3$ be surfaces and $\vf{f}:S_1\rightarrow S_2$ be a differentiable function. We say that $\vf{f}$ is a \emph{local isometry} if the differential function $\vf{Df}_p$ is an isometry $\forall p\in S_1$. That is, for each $p\in S_1$ we have:
    $$\langle\vf{v},\vf{w}\rangle_1=\langle\vf{Df}_p(\vf{v}),\vf{Df}_p(\vf{w})\rangle_2\quad\forall\vf{v},\vf{w}\in T_pS_1$$
    We say that $\vf{f}$ is an \emph{isometry} if it is a local isometry and it is invertible.
  \end{definition}
  \begin{proposition}
    Let $S_1,S_2\subseteq\RR^3$ be surfaces and $\vf{f}:S_1\rightarrow S_2$ be a local isometry. Then, $\vf{Df}_p$ is a diffeomorphism.
  \end{proposition}
  \begin{proposition}
    Let $S_1,S_2\subseteq\RR^3$ be surfaces and $\vf{f}:S_1\rightarrow S_2$ be a function of class $\mathcal{C}^1$. Then, $\vf{f}$ is a local isometry if and only if $\vf{f}$ preserves lengths, that is, for any curve $\vf\alpha:I\rightarrow S_1$, we have $L(\vf\alpha)=L(f\circ\vf\alpha)$.
  \end{proposition}
  \begin{proposition}
    Let $S_1,S_2\subseteq\RR^3$ be surfaces, $(V,\vf\varphi)$ be a local parametrization of $S_1$, $\vf{f}:S_1\rightarrow S_2$ be a function of class $\mathcal{C}^1$. Then, $(V,\vf\psi=\vf{f}\circ\vf\varphi)$ is a local parametrization of $S_2$. Let $E_{\vf\varphi}$, $F_{\vf\varphi}$, $G_{\vf\varphi}$ be the coefficients of the first fundamental form of $S_1$ and $E_{\vf\psi}$, $F_{\vf\psi}$, $G_{\vf\psi}$ be the coefficients of the first fundamental form of $S_2$. Then:
    $$\vf{f}\text{ is isometry}\iff E_{\vf\varphi}=E_{\vf\psi}, F_{\vf\varphi}=F_{\vf\psi}\text{ and }G_{\vf\varphi}=G_{\vf\psi}$$
  \end{proposition}
  \begin{corollary}
    Let $S_1,S_2\subseteq\RR^3$ be surfaces and $\vf{f}:S_1\rightarrow S_2$ be an isometry. Then, $\vf{f}$ preserves areas.
  \end{corollary}
  \subsection{Second fundamental form}
  \subsubsection{Orientation of surfaces and Gau\ss\ map}
  \begin{definition}
    Let $S\subseteq\RR^3$ be a surface. We say that $S$ is \emph{orientable} if it admits a \emph{normal unit field}, that is, a differentiable function $\vf\nu_S:S\rightarrow S^2\subseteq\RR^3$ such that $\vf\nu_S(p)\in T_pS^\perp$ $\forall p\in S$. This function $\vf\nu_S$\footnote{Unless necessary, we will omit writing the subindex $S$.} is known as \emph{Gau\ss\ map}.
  \end{definition}
  \begin{definition}
    Let $S\subseteq\RR^3$ be an orientable and connected surface. An \emph{orientation} of $S$ is the choice of one of the two ($\vf\nu_S$ or $-\vf\nu_S$) unit normal fields.
  \end{definition}
  \begin{definition}
    Let $S\subseteq\RR^3$ be an orientable surface and $(V,\vf\varphi(u,v))$ be a local parametrization of $S$. We say that $(V,\vf\varphi(u,v))$ is \emph{compatible} with the orientation of $S$ if $$\vf\nu_S=\frac{\vf\varphi_u\times\vf\varphi_v}{\|\vf\varphi_u\times\vf\varphi_v\|}$$
  \end{definition}
  \begin{proposition}
    Let $S\subseteq\RR^3$ be a surface. $S$ is orientable if and only if $S$ can be covered by the images $\vf\varphi_i(V_i)$ of a collection of parametrizations $\{(V_i,\vf\varphi_i):i\in I\}$ of $S$ such that $$\det\vf{D}({\vf\varphi_j}^{-1}\circ\vf\varphi_i)>0\quad\forall i,j\in I$$
  \end{proposition}
  \subsubsection{Weingarten endomorphism}
  \begin{definition}
    Let $S\subseteq\RR^3$ be a surface oriented with a normal unit field $\vf\nu$. We define the \emph{Weingarten endomorphsim} of $S$ at the point $p\in S$ as the endomorphism: $$\function{\vf{W}_p}{T_pS}{T_pS}{\vf{v}}{-\vf{D\vf\nu}_p(\vf{v})}$$
  \end{definition}
  \begin{lemma}
    Let $S\subseteq\RR^3$ be a surface, $(V,\vf\varphi(u,v))$ be a local parametrization of $S$, $\vf\alpha(t)=\vf\varphi(u(t),v(t))$ be a curve on $S$ and $p=\vf{\alpha}(0)$. We denote $\vf\nu(t)=(\vf\nu\circ\vf\alpha)(t)=\vf\nu(t)=\vf\nu(u(t),v(t))$. Then:
    \begin{align*}
      \vf{D\nu}_p(\vf\alpha'(0)) & =\vf{D\nu}_p(u'(0)\vf\varphi_u+v'(0)\vf\varphi_v) \\
                                 & =u'(0)\vf\nu_u+v'(0)\vf\nu_v
    \end{align*}
    In particular, $\vf{D\nu}_p(\vf\varphi_u)=\vf\nu_u$ and $\vf{D\nu}_p(\vf\varphi_v)=\vf\nu_v$.
  \end{lemma}
  \begin{proposition}
    Let $S\subseteq\RR^3$ be a surface oriented with a normal unit field $\vf\nu$ and $p\in S$. Then, the Weingarten endomorphism is auto-adjoint with respect to the first fundamental form. That is: $${\langle \vf{W}_p(\vf{u}),\vf{v}\rangle}_p={\langle \vf{u},\vf{W}_p(\vf{v})\rangle}_p\quad\forall\vf{u},\vf{v}\in T_pS$$
  \end{proposition}
  \begin{proposition}
    Let $S\subseteq\RR^3$ be an orientable surface and $p\in S$. Then, the Weingarten endomorphism has real eigenvalues and it diagonalizes in a orthonormal basis of $T_pS$.
  \end{proposition}
  \begin{definition}
    Let $S\subseteq\RR^3$ be an orientable surface and $p\in S$. We define the \emph{principal directons} of $S$ at $p$ as the eigenspaces of $\vf{W}_p$. We define the \emph{principal curvatures} of $S$ at $p$ as the eigenvalues of $\vf{W}_p$. The point $p$ is called \emph{umbilic point} if $\vf{W}_p=\lambda \vf{id}$, for some $\lambda\in\RR$.
  \end{definition}
  \begin{definition}
    Let $S\subseteq\RR^3$ be an orientable surface, $p\in S$ and $k_1$, $k_2$ be de principal curvatures of $S$ at $p$. We define the \emph{Gau\ss\ curvature} of $S$ at $p$ as:
    $$K(p):=\det\vf{W}_p=k_1k_2$$
    We define the \emph{mean curvature} of $S$ at $p$ as:
    $$H(p):=\frac{\trace\vf{W}_p}{2}=\frac{k_1+k_2}{2}$$
  \end{definition}
  \subsubsection{Second fundamental form}
  \begin{definition}
    Let $S\subseteq\RR^3$ be a surface oriented with a normal unit field $\vf\nu$ and $p\in S$. We define the \emph{second fundamental form} of $S$ at $p$ as the quadratic form: $$\II_p(\vf{v})={\langle\vf{W}_p(\vf{v}),\vf{v}\rangle}_p\footnote{Making an abuse of notation, we will denote the bilinear associated function to $\II_p$ also as $\II_p$.}$$
  \end{definition}
  \begin{definition}
    Let $S\subseteq\RR^3$ be a surface oriented with a normal unit field $\vf\nu$, $p\in S$ and $\vf\alpha:I\rightarrow S$ be a regular curve. Suppose $\cos\theta=\langle\vf{N}_{\vf\alpha}(p),\nu(p)\rangle$. We define the \emph{normal curvature} of $\vf\alpha\subset S$ at $p$ as: $$k_\text{n}(p):=k_{\vf\alpha}\cos\theta$$
  \end{definition}
  \begin{proposition}[Meusnier's theorem]
    Let $S\subseteq\RR^3$ be an orientable surface, $p\in S$ and $\vf\alpha:I\rightarrow S$ be an arc-length parametrized curve such that $\vf\alpha(0)=p$. Then: $$k_\text{n}=\II_p(\vf\alpha'(0))$$
    In particular, $k_\text{n}$ only depends on the tangent line to $\vf\alpha$ at $p$.
  \end{proposition}
  \begin{definition}
    Let $S\subseteq\RR^3$ be an orientable surface, $p\in S$ and $\vf{v}\in T_pS$ with $\|\vf{v}\|=1$. We define the \emph{normal curvature} at $p$ in the direction of $\vf{v}$ as: $$k_\text{n}(\vf{v}):=\II_p(\vf{v})$$
  \end{definition}
  \begin{proposition}
    Let $S\subseteq\RR^3$ be an orientable surface, $p\in S$ and $(\vf{v}_1,\vf{v}_2)$ be an orthonormal basis of $T_pS$, where $\vf{v}_i$ is an eigenvectors of eigenvalue $k_i$ of $\vf{W}_p$ for $i=1,2$. Then, for $i=1,2$ we have: $$k_i=k_\text{n}(\vf{v}_i)$$
  \end{proposition}
  \begin{definition}
    Let $S\subseteq\RR^3$ be an orientable surface, $p\in S$ and $\vf{v}\in T_pS$ with $\|\vf{v}\|=1$. We day that the direction of $\vf{v}$ in $T_pS$ is an \emph{asymptotic direction} if $k_\text{n}(\vf{v})=0$.
  \end{definition}
  \begin{definition}
    Let $S\subseteq\RR^3$ be an orientable surface and $\vf\alpha:I\rightarrow S$ be a curve. We say that $\vf\alpha^*$ is a \emph{line of curvature} of $S$ if the tangent line at $\vf\alpha^*$ is a principal direction at each point $p\in\vf\alpha^*\subset S$. We say that $\vf\alpha^*$ is an \emph{asymptotic line} of $S$ if the tangent line at $\vf\alpha^*$ is an asymptotic direction at each point $p\in\vf\alpha^*\subset S$.
  \end{definition}
  \begin{proposition}[Olinde-Rodrigues' theorem]
    Let $S\subseteq\RR^3$ be an orientable surface and $\vf\alpha:I\rightarrow S$ be a regular curve. Let $\vf\nu(t):=(\vf\nu\circ\vf\alpha)(t)$. Then, $\vf{\alpha}^*$ is a line of curvature of $S$ if and only if $$\vf\nu'(t)=\lambda(t)\vf\alpha'(t)$$ where $\lambda(t)$ is a differentiable function. In this case, $-\lambda(t)$ is the principal curvature of $S$ in the direction of $\vf\alpha'(t)$.
  \end{proposition}
  \begin{proposition}[Euler's formula]
    Let $S\subseteq\RR^3$ be an orientable surface, $p\in S$ and $(\vf{v}_1,\vf{v}_2)$ be an orthonormal basis of $T_pS$, where $\vf{v}_i$ is an eigenvectors of eigenvalue $k_i$ of $\vf{W}_p$ for $i=1,2$. Then: $$k_\text{n}(\cos\theta \vf{v}_1+\sin\theta \vf{v}_2)=k_1{\left(\cos\theta\right)}^2+k_2{\left(\sin\theta\right)}^2$$
    Hence, we will denote $k_\text{n}(\theta):=k_1{\left(\cos\theta\right)}^2+k_2{\left(\sin\theta\right)}^2$.
  \end{proposition}
  \begin{corollary}
    Let $S\subseteq\RR^3$ be an orientable surface, $p\in S$. Then, the extrema of $k_\text{n}(p)$ are precisely the principal curvatures $k_1$ and $k_2$ at $p$.
  \end{corollary}
  \begin{proposition}
    Let $S\subseteq\RR^3$ be an orientable surface. Then: $$H=\frac{k_1+k_2}{2}=\frac{1}{2\pi}\int_0^{2\pi}k_\text{n}(\theta)\dd\theta$$
  \end{proposition}
  \begin{definition}
    Let $S\subseteq\RR^3$ be an orientable surface and $p\in S$. We say that $p$ is
    \begin{itemize}
      \item an \emph{elliptic point} if $K(p)>0$.
      \item a \emph{hyperbolic point} if $K(p)<0$.
      \item a \emph{parabolic point} if $K(p)=0$ but $\vf{W}_p\ne0$.
      \item a \emph{plane point} if $K(p)=0$ and $\vf{W}_p=0$.
    \end{itemize}
  \end{definition}
  \begin{proposition}
    Let $S\subseteq\RR^3$ be an orientable and connected surface such that all of its points are umbilic. Then, $S$ is contained in a sphere or in a plane.
  \end{proposition}
  \subsubsection{Gau\ss\ map in coordinates}
  \begin{proposition}
    Let $(V,\vf\varphi(u,v))$ be a local pa\-ram\-e\-triza\-tion of a surface $S\subseteq\RR^3$ oriented with a normal unit field $\vf\nu$ and $p\in S$.
    Suppose
    \begin{align*}
      \vf\nu_u & =a_{11}\vf\varphi_u+a_{21}\vf\varphi_v
      \vf\nu_v & =a_{12}\vf\varphi_u+a_{22}\vf\varphi_v
    \end{align*}
    Then: $$\vf{D\nu}_p=-\vf{W}_p=\begin{pmatrix}
        a_{11} & a_{12} \\
        a_{21} & a_{22}
      \end{pmatrix}$$
  \end{proposition}
  \begin{proposition}
    Let $(V,\vf\varphi(u,v))$ be a local pa\-ram\-e\-triza\-tion of a surface $S\subseteq\RR^3$ oriented with a normal unit field $\vf\nu$ and $p\in S$. Then, we have:
    \begin{align*}
      e & :=-\langle\vf\nu_u,\vf\varphi_u\rangle=\langle \vf\nu,\vf\varphi_{uu}\rangle                                                                            \\
      f & :=-\langle\vf\nu_v,\vf\varphi_u\rangle=\langle \vf\nu,\vf\varphi_{uv}\rangle=\langle \vf\nu,\vf\varphi_{vu}\rangle=-\langle\vf\nu_u,\vf\varphi_v\rangle \\
      g & :=-\langle\vf\nu_v,\vf\varphi_v\rangle=\langle \vf\nu,\vf\varphi_{vv}\rangle
    \end{align*}
    Moreover, in the basis $(\vf\varphi_u,\vf\varphi_v)$ we have:
    $$
      \II_p=\begin{pmatrix}
        e & f \\
        f & g
      \end{pmatrix}
    $$
  \end{proposition}
  \begin{proposition}
    Let $(V,\vf\varphi(u,v))$ be a local pa\-ram\-e\-triza\-tion of a surface $S\subseteq\RR^3$ oriented with a normal unit field $\vf\nu$ and $p\in S$. Then:
    $$\vf{W}_p={\I_p}^{-1}\II_p$$
    Hence: $$\vf{W}_p=\frac{1}{EG-F^2}
      \begin{pmatrix}
        eG-fF  & fG-gF  \\
        -eF+fE & -fF+gE
      \end{pmatrix}$$
  \end{proposition}
  \begin{corollary}
    Let $(V,\vf\varphi(u,v))$ be a local parametrization of a surface $S\subseteq\RR^3$ oriented with a normal unit field $\vf\nu$ and $p\in S$. Then:
    \begin{gather*}
      K =\frac{eg-f^2}{EG-F^2}               \\
      H =\frac{1}{2}\frac{eG-2fF+gE}{EG-F^2}
    \end{gather*}
    Moreoverthe principal curvatures are given by: $$k_1,k_2=H\pm\sqrt{H^2-K}$$
  \end{corollary}
  \begin{proposition}
    Let $S\subseteq\RR^3$ be an orientable surface, $(V,\vf\varphi(u,v))$ be a parametrization of $S$ and $\vf\alpha:I\rightarrow S$ be a regular curve such that $\vf\alpha(t)=\vf\varphi(u(t),v(t))$. Then:
    \begin{enumerate}
      \item $\vf\alpha^*$ is an asymptotic line if and only if: $$e{u'}^2+2fu'v'+g{v'}^2=0$$
      \item $\vf\alpha^*$ is a line of curvature if and only if:
            $$
              \begin{vmatrix}
                {v'}^2 & -u'v' & {u'}^2=0 \\
                E      & F     & G        \\
                e      & f     & g
              \end{vmatrix}
            $$
    \end{enumerate}
  \end{proposition}
  \subsubsection{Geometric interpretation of Gau\ss\ curvature}
  \begin{lemma}
    Let $S\subseteq\RR^3$ be an orientable surface, $p\in S$, $(\vf{v}_1,\vf{v}_2)$ be a basis of $T_pS$ and $\vf{A}:T_pS\rightarrow T_pS$ be a linear isomorphism. Then: $$\vf{A}\vf{v}_1\times\vf{A}\vf{v}_2=\det\vf{A}\left(\vf{v}_1\times\vf{v}_2\right)$$
    In particular, if $\vf\nu$ is a normal unit field of $S$ and $\vf{w}_1,\vf{w}_2\in T_pS$, then: $$\vf{D}\vf\nu_p(\vf{w}_1)\times\vf{D}\vf\nu_p(\vf{w}_2)=K(p)\left(\vf\nu_p(\vf{w}_1)\times\vf\nu_p(\vf{w}_1)\right)$$
  \end{lemma}
  \begin{definition}
    Let $S\subseteq\RR^3$ be a surface oriented with a normal unit field $\vf\nu$ and $R\subseteq S$ be a region on $S$ where the curvature $K$ doesn't vanish. We define the signed area of $\vf\nu(R)$ as: $$\area_\text{s}(\vf\nu(R))=\sign(K)\area(\vf\nu(R))$$
  \end{definition}
  \begin{proposition}
    Let $S\subseteq\RR^3$ be an orientable surface, $p\in S$ such that $K(p)\ne 0$ and $V\subseteq S$ be a connected neighbourhood of $p$ where $K$ has constant sign. Let $(B_n)\subseteq V$ be a sequence of regions that converge to $p$. Then: $$K(p)=\lim_{n\to\infty}\frac{\area_\text{s}(\vf\nu(B_n))}{\area(B_n)}$$
  \end{proposition}
  \subsection{Intrinsic geometry of surfaces}
  \subsubsection{Gau\ss' Theorema Egregium}
\end{multicols}
\end{document}