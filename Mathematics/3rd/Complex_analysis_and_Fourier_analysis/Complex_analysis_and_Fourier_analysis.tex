\documentclass[../../../main.tex]{subfiles}


\begin{document}
\renewcommand{\col}{\ana}
\begin{multicols}{2}[\section{Complex analysis and Fourier analysis}]
  \subsection{Complex numbers}
  \subsubsection{Definition of complex numbers}
  \begin{definition}
    Consider $x^2+1\in\RR[x]$ and the ring $R:=\quot{\RR[x]}{(x^2+1)}$. Then, $R$ is a commutative field, which we will denote by $\CC$, whose elements are of the form $a+b\overline{x}=:a+b\ii$, $a,b\in\RR$\footnote{Such expression of a complex number is called \emph{Cartesian form} of a complex number.}. This field is called \emph{field of complex numbers}.
  \end{definition}
  \begin{proposition}
    Let $a_1+b_1\ii,a_2+b_2\ii\in\CC$, $a_1,a_2,b_1,b_2\in\RR$\footnote{From now on, we will omit to say that these values are real numbers. If they aren't, we will explicitly remark it.}. Then:
    \begin{itemize}
      \item $(a_1+b_1\ii)+(a_2+b_2\ii)=(a_1+a_2)+(b_1+b_2)\ii$
      \item $(a_1+b_1\ii)\cdot(a_2+b_2\ii)=(a_1a_2-b_1b_2)+(a_1b_2+a_2b_1)\ii$
      \item $\displaystyle\frac{a_1+b_1\ii}{a_2+b_2\ii}=\frac{a_1a_2+b_1b_2}{{a_2}^2+{b_2}^2}+\frac{a_2b_1-a_1b_2}{{a_2}^2+{b_2}^2}\ii$, provided that $a_2,b_2\ne 0$.
    \end{itemize}
  \end{proposition}
  \begin{theorem}
    $\CC$ is not an ordered field.
  \end{theorem}
  \subsubsection{Complex conjugate, modulus and argument}
  \begin{definition}
    Let $z=a+b\ii\in\CC$. We define the \emph{complex conjugate} (or simply \emph{conjugate}) of $z$ as $\overline{z}:=a-b\ii$.
  \end{definition}
  \begin{proposition}
    Let $z,w\in\CC$. Then:
    \begin{enumerate}
      \item $\overline{\overline{z}}=z$
      \item $\overline{z+w}=\overline{z}+\overline{w}$
      \item $\overline{z\cdot w}=\overline{z}\cdot\overline{w}$
      \item $\displaystyle\overline{\left(\frac{z}{w}\right)}=\frac{\overline{z}}{\overline{w}}$, provided that $w\ne 0$.
      \item $z\in\RR\iff \overline{z}=z$
    \end{enumerate}
  \end{proposition}
  \begin{definition}
    Let $z=a+b\ii\in\CC$. We define the \emph{real part} of $z$ as $\Re z:=a$. We define the \emph{imaginary part} of $z$ as $\Im z:=b$.
  \end{definition}
  \begin{proposition}
    Let $z\in\CC$. Then: $$\Re z=\frac{z+\overline{z}}{2}\quad\text{and}\quad\Im z=\frac{z-\overline{z}}{2\ii}$$
  \end{proposition}
  \begin{definition}
    Let $z=a+b\ii\in\CC$. We define the \emph{modulus} of $z$ as: $$\abs{z}:=\sqrt{a^2+b^2}$$
  \end{definition}
  \begin{proposition}
    Let $z,w\in\CC$. Then:
    \begin{enumerate}
      \item $\abs{z}\geq 0$
      \item $\abs{z}=0\iff z=0$
      \item $z\overline{z}={\abs{z}}^2$
      \item $z^{-1}=\frac{1}{{\abs{z}}^2}\cdot\overline{z}$
      \item $\abs{\Re z},\abs{\Im z}\leq \abs{z}\leq\abs{\Re z}+\abs{\Im z}$
      \item $\abs{z\cdot w}=\abs{z}\cdot \abs{w}$
      \item $\abs{z^n}={\abs{z}}^n$ $\forall n\in\ZZ$
      \item $\displaystyle\abs{\frac{z}{w}}=\frac{\abs{z}}{\abs{w}}$, provided that $w\ne 0$.
      \item ${\abs{z\pm w}}^2={\abs{z}}^2+{\abs{w}}^2\pm 2\Re(z\overline{w})$
      \item $\abs{z\pm w}\leq \abs{z}+ \abs{w}$
            %and the equality holds if and only if $z=\lambda w$, $\lambda\in\RR_{\geq 0}$.
      \item $\abs{\abs{z}-\abs{w}}\leq \abs{z\pm w}$
            % and the equality holds if and only if $z=\lambda w$, $\lambda\in\RR_{\geq 0}$.
    \end{enumerate}
  \end{proposition}
  \begin{corollary}
    Let $n\in\NN$ and $z_1, \ldots, z_n \in \CC$. Then:
    \begin{itemize}
      \item $\displaystyle \abs{\sum_{i=1}^n{z_i}} \leq \sum_{i=1}^n{\abs{z_i}}$
      \item $\displaystyle \abs{z_1 \cdots z_n} = \abs{z_1} \cdots \abs{z_n}$
      \item $\displaystyle \abs{\Re(z_1 \cdots z_n)},\abs{\Im(z_1 \cdots z_n)} \leq \abs{z_1} \cdots \abs{z_n}$
    \end{itemize}
  \end{corollary}
  \begin{definition}
    Let $z \in \CC^*$. We define the \textit{argument} of $z$, denoted by $\arg{z}$, as the real number $\theta$ satisfying: $$z = \abs{z}(\cos{\theta} + \ii\sin{\theta})$$ Note that $\arg{z}$ is not unique. Because of that, we say that $\arg z$ is a \emph{multivalued function}.
  \end{definition}
  \begin{definition}
    Let $U\subseteq\CC$ be an open set and $f: U\rightarrow\CC^*$ be a function. A \emph{determination of the argument of $f$} is a continuous function $g:U\rightarrow \RR$ such that $f(z)=\abs{f(z)}\exp{\ii g(z)}$ $\forall z\in U$.
  \end{definition}
  \begin{definition}
    Let $z \in \CC^*$. We define the \textit{principal argument} of $z$ as the unique real number $\theta$ satisfying:
    $$\Arg z := \{\theta \in (-\pi, \pi]:z = \abs{z}(\cos{\theta} + \ii\sin{\theta})\}$$
    Note that this determination of the argument is not continuous.
  \end{definition}
  \begin{proposition}
    Let $z=a+b\ii\in\CC$. Then:
    $$\Arg z=
      \begin{cases}
        \arctan\left(\frac{y}{x}\right)     & \text{if }x>0         \\
        \arctan\left(\frac{y}{x}\right)+\pi & \text{if }x<0,y\geq 0 \\
        \arctan\left(\frac{y}{x}\right)-\pi & \text{if }x<0,y<0     \\
        \sign{(y)}\frac{\pi}{2}             & \text{if }x=0
      \end{cases}
    $$
  \end{proposition}
  \subsubsection{Metric topology of \texorpdfstring{$\CC$}{C}}
  \begin{proposition}
    Consider the distance $d$ defined as: $$\function{d}{\CC\times\CC}{\RR}{(z,w)}{\abs{z-w}}$$ Then, $(\CC,d)$ is a metric space\footnote{In order to simplify the notation we will refer to $(\CC,d)$ simply as $\CC$.}.
  \end{proposition}
  \begin{proposition}
    Thinking complex numbers as an ordered pair of real numbers, the topology of $\CC$ induced by $d$ is the same as the ordinary topology of $\RR^2$.
  \end{proposition}
  \begin{definition}
    We define the \emph{extended complex plane} as $\CC_\infty:=\CC\cup\{\infty\}$. We define the \emph{extended real numbers} as $\RR_\infty:=\RR\cup\{\infty\}$. The topologies added to those sets are the ones given by the one-point compactification\footnote{See \cref{TOP_alex}.}.
  \end{definition}
  \begin{definition}[Stereographic projection]
    The \emph{stereographic projection} is the function $p:S^2\rightarrow\CC_\infty$ defined as:
    $$p(x_1,x_2,x_3)=
      \begin{cases}
        \frac{x_1}{1-x_3}+\ii\frac{x_2}{1-x_3} & \text{if }x_3\ne 1 \\
        \infty                                 & \text{if }x_3= 1
      \end{cases}$$
    The inverse of the stereographic projection $p^{-1}:\CC_\infty\rightarrow S^2$ is:
    $$p^{-1}(z)=
      \begin{cases}
        \left(\frac{z+\overline{z}}{1+\abs{z}^2},\frac{z-\overline{z}}{\ii(1+{\abs{z}}^2)},\frac{{\abs{z}}^2-1}{1+{\abs{z}}^2}\right) & \text{if }z\in\CC  \\
        (0,0,1)                                                                                                                       & \text{if }z=\infty
      \end{cases}$$
  \end{definition}
  \begin{center}
    \begin{minipage}{\linewidth}
      \centering
      \includestandalone[mode=image|tex,width=\linewidth]{Images/stereo}
      \captionof{figure}{Stereographic projection}
    \end{minipage}
  \end{center}
  \subsection{Sequences and series}
  \subsubsection{Sequences}
  \begin{definition}
    A \textit{sequence of complex numbers} is a function of the form $$\function{}{\NN}{\CC}{n}{z_n}$$ In general, we will denote that sequence by $(z_n)$.
  \end{definition}
  \begin{definition}
    Let $(z_n)\in\CC$ be a sequence. A subsequence of $(z_n)$ is a sequence $(z_{k_n})$, where $(k_n)\in\NN$ is an increasing sequence of natural numbers.
  \end{definition}
  \begin{definition}
    Let $(z_n)\in\CC$ be a sequence. We say that $(z_n)$ has \emph{limit} $z\in\CC$ (or it \textit{converges} to $z$) if $\forall\varepsilon>0$, $\exists n_0\in\NN$ such that $$\abs{z_n - z} < \varepsilon\quad \forall n > n_0$$ In that case, we will write $\displaystyle\lim_{n \to \infty} z_n = z$ and we say that sequence is \emph{convergent}. Otherwise, we say that the sequence is \emph{divergent}.
  \end{definition}
  \begin{definition}
    Let $(z_n)\in\CC$ be a sequence. We say that $(z_n)$ is \emph{bounded} if $\exists M\in\RR$ such that $\abs{z_n}\leq M$ $\forall n\in\NN$.
  \end{definition}
  \begin{definition}
    Let $(z_n)\in\CC$ be a sequence. We say that $(z_n)$ is \emph{Cauchy} if $\forall\varepsilon>0$, $\exists n_0\in\NN$ such that $$\abs{z_n - z_m} < \varepsilon\quad \forall n,m > n_0$$
  \end{definition}
  \begin{proposition}
    Let $(z_n)\in\CC$ be a convergent sequence. Then, $(z_n)$ is bounded and Cauchy.
  \end{proposition}
  \begin{proposition}
    Let $(z_n)\in\CC$ be a sequence. Then, $(z_n)$ is convergent if and only if all its subsequences are convergent.
  \end{proposition}
  \begin{proposition}
    Let $(z_n),(w_n)\in\CC$ be two convergent sequences whose limits are $z,w\in \CC$, respectively. Then:
    \begin{enumerate}
      \item $\displaystyle\lim_{n\to\infty}z_n+w_n=z+w$
      \item $\displaystyle\lim_{n\to\infty}z_nw_n=zw$
      \item $\displaystyle\lim_{n\to\infty}\frac{z_n}{w_n}=\frac{z}{w}$, provided that $w_n\ne 0$ $\forall n\in\NN$.
    \end{enumerate}
  \end{proposition}
  \begin{definition}
    Let $(z_n)\in\CC$ be a sequence such that $z_n=x_n+y_n\ii$ $\forall n\in\NN$, where $x_n,y_n\in\RR$. Then:
    \begin{enumerate}
      \item $(z_n)$ is convergent if and only if $(x_n)$ and $(y_n)$ are convergent. In that case, we have: $$\lim_{n\to\infty}z_n=\lim_{n\to\infty}x_n+\ii\lim_{n\to\infty}y_n$$
      \item $(z_n)$ is Cauchy if and only if $(x_n)$ and $(y_n)$ are Cauchy.
    \end{enumerate}
  \end{definition}
  \begin{theorem}
    $\CC$ is a complete metric space.
  \end{theorem}
  \subsubsection{Series}
  \begin{definition}
    Let $(z_n)\in\CC$ be a sequence. A \emph{numeric series of complex numbers} is an expression of the form $$\sum_{n=1}^\infty z_n$$ We call $z_n$ \emph{general term of the series} and $\displaystyle S_N:=\sum_{n=1}^N z_n$, for all $N\in\NN $, \emph{$N$-th partial sum of the series}\footnote{Sometimes we will write $\sum z_n$ to refer to $\displaystyle\sum_{n=1}^\infty z_n$.}.
  \end{definition}
  \begin{definition}
    We say the series of complex numbers $\sum z_n$ is \emph{convergent} if $\displaystyle S=\lim_{N\to\infty}S_N$ exists and it is finite. In that case, $S$ is called the \emph{sum of the series}. If the previous limit doesn't exist or it is infinite, we say the series is \emph{divergent}\footnote{We will use the notation $\sum z_n<\infty$ or $\sum z_n=+\infty$ to express that the series converges or diverges, respectively.}.
  \end{definition}
  \begin{definition}
    Let $\sum z_n$ be a series of complex numbers. A \emph{reordering} of $\sum z_n$ is any series $\sum z_{\sigma(n)}$, where $\sigma:\NN\rightarrow\NN$ be a bijective function.
  \end{definition}
  \begin{proposition}
    Let $(z_n)\in\CC$ be a sequence and $\sum z_n$ be a convergent series. Then, $\displaystyle\lim_{n\to\infty}z_n =0$.
  \end{proposition}
  \begin{proposition}
    Let $(z_n)\in\CC$ be a sequence such that such that $z_n=x_n+y_n\ii$ $\forall n\in\NN$, where $x_n,y_n\in\RR$, and $\sum z_n$ be a series. Then, $\sum z_n<\infty$ if and only if $\sum x_n<\infty$ and $\sum y_n<\infty$. In that case, we have: $$\sum_{n=1}^\infty z_n=\sum_{n=1}^\infty x_n+\ii\sum_{n=1}^\infty y_n$$
  \end{proposition}
  \begin{proposition}
    Let $\sum z_n=z\in\CC$ and $\sum w_n=w\in\CC$ be two series, and $\lambda\in\CC$. Then:
    \begin{enumerate}
      \item $\sum (z_n+w_n)$ is convergent and $\sum (z_n+w_n)=z+w$.
      \item $\sum (\lambda z_n)$ is convergent and $\sum (\lambda z_n)=\lambda z$.
    \end{enumerate}
  \end{proposition}
  \begin{lemma}[Abel's summation formula]
    Let $(z_n),(w_n)\in\CC$ be two sequence. Let $S_N:=\sum_{n=1}^N z_n$. Then:
    \begin{equation*}
      \sum_{n=1}^N z_nw_n=S_Nw_N+\sum_{n=1}^{N-1}S_n(w_n-w_{n+1})
    \end{equation*}
  \end{lemma}
  \begin{definition}
    Let $\sum z_n$ be a series of complex numbers. We say that $\sum z_n$ is \emph{absolutely convergent} if $\sum \abs{z_n}<\infty$\footnote{Note that since $\sum \abs{z_n}$ is a sequence of real numbers, all the criteria for convergence of numeric series of real numbers are, thus, applicable.}.
  \end{definition}
  \begin{proposition}
    Let $\sum z_n$ be a series of complex numbers.
    \begin{enumerate}
      \item $\sum \abs{z_n}<\infty\implies\sum z_n<\infty$
      \item $\sum \abs{z_n}=z<\infty\implies\forall\sigma\in S(\NN),\ \sum z_{\sigma (n)}=w<\infty$ for some $w\in\CC$.
    \end{enumerate}
  \end{proposition}
  \begin{definition}[Cauchy product]
    Let $\sum z_n$, $\sum w_n$ be absolutely convergent series of complex numbers. We define the \emph{product} of $\sum z_n$ and $\sum w_n$ as the series $\sum p_n$, where $p_n=\sum_{k=0}^nz_kw_{n-k}$.
  \end{definition}
  \begin{proposition}
    Let $\sum z_n$, $\sum w_n$ be absolutely convergent series of complex numbers. Then, the product of these series is absolutely convergent and satisfy: $$\sum_{n=1}^\infty p_n=\left(\sum_{n=1}^\infty z_n\right)\left(\sum_{n=1}^\infty w_n\right)$$
  \end{proposition}
  \subsection{Complex functions}
  \subsubsection{Continuity}
  \begin{definition}
    Let $D \subseteq \CC$ be a set. We define a \textit{complex function}\footnote{Due to the close similarity between these kind of functions and the multivariate functions, we will only expose the most remarkable results about continuity of complex functions.} as a function of the form:
    $$\function{f}{D}{\CC}{z}{f(z)}$$
  \end{definition}
  \begin{definition}
    Let $D \subseteq \CC$ be a set and $f:D\rightarrow\CC$ be a function. We say that $f$ is \emph{continuous} at $z_0\in D$ if and only if $\forall \varepsilon>0$ $\exists\delta>0$ such that $\abs{f(z)-f(z_0)}<\varepsilon$ whenever $\abs{z-z_0}<\delta$. We say that $f$ is continuous on $D$ if it is continuous at $z$ $\forall z\in D$.
  \end{definition}
  \begin{definition}
    Let $D \subseteq \CC$ be a set and $f:D\rightarrow\CC$ be a function. We define the function $\Re f$ as the function: $$\function{\Re f}{D}{\CC}{z}{\Re(f(z))}$$
    Analogously, we define the function $\Im f$ as the function: $$\function{\Im f}{D}{\CC}{z}{\Im(f(z))}$$
  \end{definition}
  \begin{proposition}
    Let $D \subseteq \CC$ be a set, $f:D\rightarrow\CC$ be a function and $z_0 \in D$. Then, $f = \Re f + \ii \Im f$ is continuous at $z_0$ if and only if $\Re f$ and $\Im f$ are continuous at $z_0$.
  \end{proposition}
  \begin{proposition}
    Let $D \subseteq \CC$ be a set, $f:D\rightarrow\CC$ be a function and $z_0 \in D$. Then, $f$ is continuous at $z_0$ if and only if for all sequence $(w_n)\in D$ convergent to $z_0$, the sequence $(f(w_n))$ converges to $f(z_0)$.
  \end{proposition}
  \begin{proposition}
    Let $D \subseteq \CC$ be a set, $f,g:D \subseteq \CC \longrightarrow \CC$ be two continuous function at a point $z_0 \in D$ and $\lambda \in \CC$. Then, $\lambda f$, $f+g$, and $fg$ are continuous at $z_0$. Moreover, if $g(z_0) \neq 0$, then $f/g$ is continuous at $z_0$.
  \end{proposition}
  \subsubsection{Sequences of functions}
  \begin{definition}
    Let $D\subseteq\CC$. A set $$(f_n(z))=\{f_1(z),f_2(z),\ldots,f_n(z),\ldots\}$$ is a \emph{sequence of complex functions} if $f_i:D\rightarrow\CC $ is a complex function $\forall i\in\NN$. In this case we say the sequence $(f_n(z))$, or simply $(f_n)$, is well-defined on $D$\footnote{The majority of definitions and results of sequences of real-valued functions can be extended conveniently to sequences of complex functions. So in this document, we will only expose the most important ones.}.
  \end{definition}
  \begin{definition}
    Let $(f_n)\in D\subseteq\CC$ be a sequence of functions and $f:D\rightarrow\CC$. We say $(f_n)$ \emph{converges pointwise} to $f$ on $D$ if $\forall z\in D$, $\displaystyle\lim_{n\to\infty}f_n(z)=f(z)$.
  \end{definition}
  \begin{definition}
    Let $(f_n)\in D\subseteq\CC$ be a sequence of functions and $f:D\rightarrow\CC$. We say $(f_n)$ \emph{converges uniformly} to $f$ on $D$ if $\forall\varepsilon>0$, $\exists n_0$ such that $\abs{f_n(z)-f(z)}<\varepsilon$ $\forall n\geq n_0$ and $\forall z\in D$.
  \end{definition}
  \begin{lemma}
    Let $(f_n)\in D\subseteq\CC$ be a sequence of functions. $(f_n)$ converges uniformly to $f:D\rightarrow\CC$ on $D$ if and only if $\displaystyle \lim_{n\to\infty}\sup\left\{\abs{f_n(z)-f(z)}:z\in D\right\}=0$.
  \end{lemma}
  \begin{theorem}[Cauchy's test]
    A sequence of functions $(f_n)\in D\subseteq\CC$ converges uniformly to $f:D\rightarrow\CC$ on $D\subseteq\CC$ if and only if $\forall\varepsilon>0$ $\exists n_0\in\NN$ such that  $\forall n,m\geq n_0$, we have: $$\sup\left\{\abs{f_n(z)-f_m(z)}:z\in D\right\}< \varepsilon$$
  \end{theorem}
  \begin{theorem}
    Let $(f_n)\in D\subseteq\CC$ be a sequence of continuous functions. If $(f_n)$ converges uniformly to $f:D\rightarrow\CC$ on $D$, then $f$ is continuous on $D$.
  \end{theorem}
  \subsubsection{Series of functions}
  \begin{definition}
    Let $(f_n)\in D\subseteq\CC$ be a sequence of functions. The expression $$\sum_{n=1}^\infty f_n(z)$$ is the \emph{series of functions} associated with $(f_n)$\footnote{The majority of definitions and results of series of real-valued functions can be extended conveniently to series of complex functions. So in this document, we will only expose the most important ones.}.
  \end{definition}
  \begin{definition}
    A series of functions $\sum f_n(z)$ defined on $D\subseteq\CC$ \emph{converges pointwise} on $D$ if the sequence of partials sums $$F_N(z)=\sum_{n=1}^Nf_n(z)$$ converges pointwise on $D$. If the pointwise limit of $(F_N)$ is $F(z)$, we say $F$ is the \emph{sum of the series in a pointwise sense}.
  \end{definition}
  \begin{definition}
    A series of functions $\sum f_n(z)$ defined on $D\subseteq\CC$ \emph{converges uniformly} on $D$ if the sequence of partials sums $$F_N(z)=\sum_{n=1}^Nf_n(z)$$ converges uniformly on $D$. If the uniform limit of $(F_N)$ is $F(z)$, we say $F$ is the \emph{sum of the series in an uniform sense}.
  \end{definition}
  \begin{theorem}[Cauchy's test]
    A series of functions $\sum f_n(z)$ defined on $D\subseteq\CC$ converges uniformly on $D$ if and only if $\forall\varepsilon>0$ $\exists n_0$ such that $\forall  M, N\geq n_0$ (with $N\leq M$), we have: $$\sup\left\{\abs{\sum_{n=N}^Mf_n(z)}:z\in D\right\}< \varepsilon$$
  \end{theorem}
  \begin{corollary}
    Let $(f_n)\in D\subseteq\CC$ be a sequence of functions. If $\sum f_n(z)$ is uniformly convergent on $D\subseteq\CC$, then $(f_n)$ converges uniformly to zero on $D$.
  \end{corollary}
  \begin{theorem}
    Let $(f_n)\in D\subseteq\CC$ be a sequence of continuous functions. If $\sum f_n(z)$ is uniformly convergent on $D\subseteq\CC$, then its sum function is also continuous on $D$.
  \end{theorem}
  \begin{theorem}[Weierstra\ss\space M-test]
    Let $(f_n)\in D\subseteq\CC$ be a sequence of functions such that $\abs{f_n(z)}\leq M_n$ $\forall z\in D$ and suppose that $\sum M_n<\infty$. Then, $\sum f_n(z)$ converges uniformly on $D$.
  \end{theorem}
  \begin{theorem}[Dirichlet's test]
    Let $(f_n)\in X\subseteq \CC$ and $(g_n)\in Y\subseteq \CC$ be two sequences of functions and $F_N:=\sum_{n=1}^Nf_n(z)$. Suppose:
    \begin{enumerate}
      \item $(F_N)$ is uniformly bounded on $X$.
      \item $(g_n(z))$ is a monotone sequence of real numbers and converges uniformly to 0 on $Y$.
    \end{enumerate}
    Then, $\sum f_n(z)g_n(z)$ converges uniformly on $X\times Y$.
  \end{theorem}
  \begin{theorem}[Abel's test]
    Let $(f_n)\in X\subseteq \CC$ and $(g_n)\in Y\subseteq \CC$ be two sequences of functions. Suppose:
    \begin{enumerate}
      \item $\sum f_n(z)$ is uniformly convergent on $X$.
      \item $(g_n)$ is a monotone and bounded sequence of real numbers.
    \end{enumerate}
    Then, $\sum f_n(z)g_n(z)$ converges uniformly on $X\times Y$.
  \end{theorem}
  \begin{theorem}[Dedekind's test]
    Let $(f_n)\in X\subseteq \CC$ and $(g_n)\in Y\subseteq \CC$ be two sequences of functions and $F_N:=\sum_{n=1}^Nf_n(z)$. Suppose:
    \begin{enumerate}
      \item $(F_N)$ is uniformly bounded on $X$.
      \item $(g_n)$ converges uniformly to 0 on $Y$.
      \item $\sum\abs{g_n(z)-g_{n+1}(z)}$ converges uniformly on $Y$.
    \end{enumerate}
    Then, $\sum f_n(z)g_n(z)$ converges uniformly on $X\times Y$.
  \end{theorem}
  \begin{theorem}[Du Bois-Reymond's test]
    Let $(f_n)\in X\subseteq \CC$ and $(g_n)\in Y\subseteq \CC$ be two sequences of functions. Suppose:
    \begin{enumerate}
      \item $\sum f_n(z)$ is uniformly convergent on $X$.
      \item $\sum\abs{g_n(z)-g_{n+1}(z)}<\infty$ $\forall z\in Y$.
    \end{enumerate}
    Then, $\sum f_n(z)g_n(z)$ converges uniformly on $X\times Y$.
  \end{theorem}
  \subsubsection{Power series}
  \begin{definition}
    Let $(a_n)\in\CC$ be a sequence and $z_0\in\CC$. A \emph{complex power series} centered at $z_0$ is a series of functions of the form: $$\sum_{n=0}^\infty a_n{(z-z_0)}^n$$
  \end{definition}
  \begin{theorem}[Cauchy-Hadamard theorem]
    Let $\sum a_n{(z-z_0)}^n$ be a complex power series. Then, $\exists! R\in[0,\infty]$ defined as $$R=\left(\limsup_{n\to\infty}\sqrt[n]{\abs{a_n}}\right)^{-1}\in[0,\infty]$$
    that satisfies the following properties:
    \begin{enumerate}
      \item If $\abs{z-z_0}<R\implies\sum a_n{(z-z_0)}^n$ converges absolutely.
      \item If $0\leq r<R\implies\sum a_n{(z-z_0)}^n$ converges absolutely and uniformly on $\{z\in\CC:\abs{z-z_0}\leq r\}$.
      \item If $\abs{z-z_0}>R\implies\sum\abs{a_n}{\abs{z-z_0}}^n$ diverges.
    \end{enumerate}
    The number $R$ is called \emph{radius of convergence} of series.
  \end{theorem}
  \begin{theorem}[Abel's theorem]
    Let $\sum a_n{z}^n$ be a complex power series that converges uniformly on $D\subseteq \CC$. Then, the series $\sum a_n{z}^n$ converges uniformly on $\{r\zeta\in\CC:r\in[0,1]\land\zeta\in D\}$, which is a \emph{cone} with base $D$.
  \end{theorem}
  \begin{corollary}[Abel's theorem]
    Let $\sum a_n{z}^n$ be a complex power series with radius of convergence $R\in(0,\infty)$ and $I\subseteq [0,2\pi)$ be a non-empty connected set. Suppose that the series $\sum a_n{z}^n$ is uniformly convergent on $D:=\{R\exp{\ii \theta}\in\CC: \theta\in I\}$. Then, $f(z):=\sum a_n{z}^n$ converges uniformly on the cone $C$ with base $D$. In particular, we have: $$\lim_{\substack{z\to R\exp{\ii \theta}\\z\in C}}f(z)=\sum_{n=0}^\infty a_nR^n\exp{\ii n\theta}\qquad\forall\theta\in I$$
  \end{corollary}
  \begin{proposition}
    Let $f:\CC\rightarrow\CC$ be the sum function of a complex power series. Then $f$ is continuous on the domain of convergence of the series.
  \end{proposition}
  \subsubsection{Exponential and logarithmic functions}
  \begin{definition}
    For all $z \in \CC$, we define the \textit{complex exponential function} as: $$\exp{z}:=\sum_{n=0}^\infty\frac{z^n}{n!}$$
  \end{definition}
  \begin{proposition}
    The radius of convergence of $\exp{z}$ is infinite, and its image is $\CC^*$.
  \end{proposition}
  \begin{proposition}
    Let $z,w\in\CC$. Then:
    \begin{enumerate}
      \item $\exp{z+w}=\exp{z}\exp{w}$
      \item $\overline{\exp{z}}=\exp{\overline{z}}$
      \item $\abs{\exp{z}}=\exp{\Re z}$
    \end{enumerate}
  \end{proposition}
  \begin{corollary}[Euler's formula]
    Let $x\in\RR$. Then: $$\exp{\ii x}=\cos x+\ii \sin x$$
  \end{corollary}
  \begin{proposition}
    Let $z, w \in \CC$ and $n\in\ZZ$. Then:
    \begin{enumerate}
      \item $\arg(zw) = \arg z + \arg w$
      \item $\arg(z^n) = n \arg{z}$
    \end{enumerate}
  \end{proposition}
  \begin{definition}[Polar form]
    Let $z\in\CC$, $r=\abs{z}$ and $\theta=\arg z$. We define the \emph{polar form} of $z$ as: $$z = r\exp{\ii\theta}=r(\cos{\theta} + \ii\sin{\theta})$$
  \end{definition}
  \begin{corollary}
    Let $z,w\in\CC$. Then:
    \begin{enumerate}
      \item $\exp{z}=1\iff z=2\pi k\ii$, $k\in\ZZ$.
      \item $\exp{z}$ is periodic of period $2\pi i$.
      \item $\exp{z}=\exp{w}\iff z=w+2\pi \ii k$, $k\in\ZZ$.
    \end{enumerate}
  \end{corollary}
  \begin{corollary}
    Let $x\in\RR$. Then: $$\cos x=\sum_{n=0}^\infty\frac{(-1)^nx^{2n}}{(2n)!}\qquad\sin x=\sum_{n=0}^\infty\frac{(-1)^nx^{2n+1}}{(2n+1)!}$$
  \end{corollary}
  \begin{proposition}[De Moivre's formula]
    Let $\theta\in\RR$ and $n\in\ZZ$. Then: $${(\cos{\theta} + \ii\sin{\theta})}^n = \cos(n\theta) + \ii\sin(n\theta)$$
  \end{proposition}
  \begin{theorem}
    Let $n\in\NN$. Then, there are $n$ $n$-th roots of any complex number $z\in \CC^*$. Assuming $z=r(\cos\theta + \ii\sin\theta)$, these roots are: $$\sqrt[n]{r}\left[\cos\left(\frac{\theta}{n}+\frac{2\pi}{n}k\right)+\ii\sin\left(\frac{\theta}{n}+\frac{2\pi}{n}k\right)\right]$$ for $k=0,\ldots,n-1$.
  \end{theorem}
  \begin{proposition}
    Let $z\in\CC^*$. Then, the equation $\exp{w}=z$ has infinitely many solutions.
  \end{proposition}
  \begin{definition}
    Let $z\in \CC$. We define a \emph{complex natural logarithm} of $z$ as a solution to the equation $\exp{w}=z$. That is: $$\ln z:=\ln \abs{z}+\ii\arg z$$ Note that $\ln z$ is a multivalued function.
    We define the \emph{principal value} of $\ln z$ as: $$\Ln z:=\ln \abs{z}+\ii\Arg z$$
  \end{definition}
  \subsubsection{Trigonometric functions}
  \begin{definition}
    Let $z\in \CC$. We define the \emph{complex sine} and \emph{complex cosine} respectively as:
    \begin{align*}
      \cos z & =\frac{\exp{\ii z}+\exp{-\ii z}}{2}=\sum_{n=0}^\infty\frac{(-1)^nz^{2n}}{(2n)!}        \\
      \sin z & =\frac{\exp{\ii z}-\exp{-\ii z}}{2\ii}=\sum_{n=0}^\infty\frac{(-1)^nz^{2n+1}}{(2n+1)!}
    \end{align*}
  \end{definition}
  \begin{proposition}
    Let $z,w\in\CC$. Then:
    \begin{enumerate}
      \item ${\left(\cos z\right)}^2+{\left(\sin z\right)}^2=1$
      \item $\cos(-z)=\cos z$, $\sin(-z)=-\sin z$
      \item $\cos(z \pm w) = \cos z \cos w \mp \sin z \sin w$
      \item $\sin(z \pm w) = \sin z \cos w \pm \cos z \sin w$
    \end{enumerate}
  \end{proposition}
  \begin{proposition}
    The functions $\cos z$, $\sin z$ are unbounded and periodic of period $2\pi$.
  \end{proposition}
  \begin{proposition}
    Let $z\in \CC$. Then:
    \begin{itemize}
      \item $\cos z=0\implies z=\frac{\pi}{2}+\pi k\in\RR$, for some $k\in\ZZ$.
      \item $\sin z=0\implies z=\pi k\in\RR$, for some $k\in\ZZ$.
    \end{itemize}
  \end{proposition}
  \begin{definition}
    We define the \emph{complex tangent}, \emph{complex secant}, \emph{complex cosecant} and \emph{complex cotangent} respectively as:
    \begin{align*}
      \tan z & = \frac{\sin z}{\cos z}=-\ii\frac{\exp{\ii z}-\exp{-\ii z}}{\exp{\ii z}+\exp{-\ii z}}\quad \forall z\in\CC\setminus\bigcup_{k\in\ZZ}\left\{\frac{\pi}{2}+\pi k\right\} \\
      \sec z & = \frac{1}{\cos z}=\frac{2}{\exp{\ii z}+\exp{-\ii z}}\quad \forall z\in\CC\setminus\bigcup_{k\in\ZZ}\left\{\frac{\pi}{2}+\pi k\right\}                                 \\
      \csc z & = \frac{1}{\sin z}=\frac{2\ii}{\exp{\ii z}-\exp{-\ii z}}\quad\forall z\in\CC\setminus\bigcup_{k\in\ZZ}\{\pi k\}                                                        \\
      \cot z & = \frac{1}{\tan z}=\ii\frac{\exp{\ii z}+\exp{-\ii z}}{\exp{\ii z}-\exp{-\ii z}}\quad\forall z\in\CC\setminus\bigcup_{k\in\ZZ}\{\pi k\}
    \end{align*}
  \end{definition}
  \begin{definition}
    Let $z\in \CC$. We define the \emph{complex hyperbolic sine} and \emph{complex hyperbolic cosine} respectively as:
    \begin{align*}
      \cosh z & =\frac{\exp{z}+\exp{-z}}{2}=\sum_{n=0}^\infty\frac{z^{2n}}{(2n)!}     \\
      \sinh z & =\frac{\exp{z}-\exp{-z}}{2}=\sum_{n=0}^\infty\frac{z^{2n+1}}{(2n+1)!}
    \end{align*}
  \end{definition}
  \begin{proposition}
    Let $z,w\in\CC$. Then:
    \begin{enumerate}
      \item ${\left(\cosh z\right)}^2-{\left(\sinh z\right)}^2=1$
      \item $\cosh(-z)=\cosh z$, $\sinh(-z)=-\sinh z$
      \item $\cosh{(z \pm w)} = \cosh z \cosh w \pm \sinh z \sin w$
      \item $\sinh{(z \pm w)} = \sinh z \cosh w \pm \cosh z \sin w$
    \end{enumerate}
  \end{proposition}
  \begin{proposition}
    Let $z=x+\ii y\in \CC$. Then:
    \begin{enumerate}
      \item $\cos z = \cos x \cosh y - \ii \sin x \sinh y$
      \item $\sin z = \sin x \cosh y + \ii \cos x \sinh y$
    \end{enumerate}
  \end{proposition}
  \begin{definition}
    We define the \emph{complex hyperbolic tangent}, \emph{complex hyperbolic secant}, \emph{complex hyperbolic cosecant} and \emph{complex hyperbolic cotangent} respectively as:
    \begin{align*}
      \tanh z & = \frac{\sinh z}{\cosh z}=\frac{\exp{z}-\exp{-z}}{\exp{z}+\exp{-z}}\quad \forall z\in\CC\setminus\bigcup_{k\in\ZZ}\left\{\ii\left(\frac{\pi}{2}+\pi k\right)\right\} \\
      \sech z & = \frac{1}{\cosh z}=\frac{2}{\exp{z}+\exp{-z}}\quad \forall z\in\CC\setminus\bigcup_{k\in\ZZ}\left\{\ii\left(\frac{\pi}{2}+\pi k\right)\right\}                      \\
      \csch z & = \frac{1}{\sinh z}=\frac{2}{\exp{z}-\exp{-z}}\quad\forall z\in\CC\setminus\bigcup_{k\in\ZZ}\{\pi k\ii\}                                                             \\
      \coth z & = \frac{1}{\tanh z}=\frac{\exp{z}+\exp{-z}}{\exp{z}-\exp{-z}}\quad\forall z\in\CC\setminus\bigcup_{k\in\ZZ}\{\pi k\ii\}
    \end{align*}
  \end{definition}
  \subsection{Complex differentiation}
  \subsubsection{Holomorphic functions}
  \begin{definition}
    Let $U\subseteq \CC$ be an open set, $z_0\in U$ and $f:U\rightarrow \CC$ be a function. We say that $f$ is \emph{$\CC$-differentiable} at $z_0$ if the following limit exists: $$\lim_{z \to z_0} \frac{f(z) - f(z_0)}{z - z_0} = \lim_{h \to 0} \frac{f(z_0 + h) - f(z_0)}{h}$$ In that case, the limit is called \textit{derivative of $f$ at $z_0$} and it is denoted by $f'(z_0)$.
  \end{definition}
  \begin{proposition}
    Let $U\subseteq \CC$ be an open set, $z_0\in U$ and $f:U\rightarrow \CC$ be a function. Then, $f$ is $\CC$-differentiable at $z_0$ if and only if $\exists a+b\ii\in\CC$ such that $\forall\varepsilon>0$ $\exists\delta>0$ such that $$|f(z_0+h)-f(z_0)-(a+b\ii)h|\leq \varepsilon \abs{h}$$ whenever $|h|<\delta$.
  \end{proposition}
  \begin{proposition}
    Let $U\subseteq \CC$ be an open set, $z_0 \in U$ and $f:U\rightarrow \CC$ be a $\CC$-differentiable function at $z_0$. Then, $f$ is continuous at $z_0$.
  \end{proposition}
  \begin{proposition}
    Let $U\subseteq \CC$ be an open set, $z_0 \in U$, $f,g:U\rightarrow \CC$ be two $\CC$-differentiable functions at $z_0$ and $\alpha,\beta\in \CC$. Then:
    \begin{enumerate}
      \item $\alpha f + \beta g$ is $\CC$-differentiable at $z_0$ and: $${(\alpha f+ \beta g)}'(z_0) = \alpha f'(z_0) + \beta g'(z_0)$$
      \item $fg$ is $\CC$-differentiable at $z_0$ and: $${(fg)}'(z_0) = f'(z_0)g(z_0) + f(z_0)g'(z_0)$$
      \item If $g(z_0) \neq 0$, then $f/g$ is $\CC$-differentiable at $z_0$ and:
            $${\left(\frac{f}{g}\right)}'(z_0) = \frac{f'(z_0)g(z_0) - f(z_0) g'(z_0)}{{g(z_0)}^2}$$
    \end{enumerate}
  \end{proposition}
  \begin{theorem}[Chain rule]
    Let $U,V\subseteq \CC$ be open sets, $z_0\in U$, $f:U\rightarrow\CC$ be a $\CC$-differentiable function at $z_0$ such that $f(U)\subseteq V$, and $g:V\rightarrow\CC$ be a $\CC$-differentiable function at $f(z_0)$. Then, $g\circ f$ is $\CC$-differentiable at $z_0$ and: $${(g\circ f)}'(z_0)=g'(f(z_0))f'(z_0)$$
  \end{theorem}
  \begin{proposition}
    Let $z\in\CC$. Then:
    \begin{itemize}
      \item ${\left(\exp{z}\right)}'=\exp{z}$
      \item ${\left(\cos{z}\right)}'=-\sin{z}$
      \item ${\left(\sin{z}\right)}'=\cos{z}$
      \item ${\left(\tan{z}\right)}'=1+{(\tan{z})}^2$
      \item ${\left(\cosh{z}\right)}'=\sinh{z}$
      \item ${\left(\sinh{z}\right)}'=\cosh{z}$
      \item ${\left(\tanh{z}\right)}'=1-{(\tanh{z})}^2$
    \end{itemize}
  \end{proposition}
  \begin{definition}
    Let $U\subseteq \CC$ be an open set and $f:U\rightarrow \CC$ be a function. We say that $f$ is \emph{holomorphic} on $U$ if $f$ is $\CC$-differentiable at each $z\in U$. We denote the set of all holomorphic functions on $U$ by $\mathcal{H}(U)$.
  \end{definition}
  \begin{definition}
    We say that $f:\CC\rightarrow\CC$ is an \emph{entire function} if $f$ is holomorphic on the whole complex plane.
  \end{definition}
  \begin{proposition}
    Let $n\in\NN\cup\{0\}$ and $f:\CC\rightarrow\CC$ be a function defined as  $f(z)=z^n$. Then, $f$ is holomorphic and $f'(z)=nz^{n-1}$ $\forall z\in\CC$.
  \end{proposition}
  \begin{corollary}
    Let $p(z)=\frac{f(z)}{g(z)}\in\CC(z)$ be a rational function. Then, $p(z)$ is a holomorphic on the open set $\CC\setminus Z(g)$, where $Z(g)=\{z\in\CC:g(z)=0\}$.
    In particular, if $p(z)\in\CC[z]$ is a polynomial, $p(z)$ is holomorphic on $\CC$.
  \end{corollary}
  \begin{theorem}
    Let $\sum_{n=0}^\infty a_n{(z-z_0)}^n$ be a complex power series with radius of convergence $R\in[0,\infty]$. Then, the series $\sum_{n=1}^\infty na_n{(z-z_0)}^{n-1}$ has the same radius of convergence $R$ and if $f(z)=\sum_{n=0}^\infty a_n{(z-z_0)}^n$ for $|z-z_0|<R$, then $f$ is holomorphic and: $$f'(z)=\sum_{n=1}^\infty na_n{(z-z_0)}^{n-1}\quad\text{for } |z-z_0|<R$$
  \end{theorem}
  \begin{corollary}
    Let $\sum_{n=0}^\infty a_n{(z-z_0)}^n$ be a complex power series with radius of convergence $R\in[0,\infty]$. Then, the series $\sum_{n=k}^\infty n(n-1)\cdots(n-k+1)a_n{(z-z_0)}^{n-k}$ has the same radius of convergence $R$ for all $k\in\NN \cup\{0\}$ and if $f(z)=\sum_{n=0}^\infty a_n{(z-z_0)}^n$ for $|z-z_0|<R$, then $f^{(k)}$ is holomorphic and: $$f^{(k)}(z)=\sum_{n=k}^\infty n(n-1)\cdots(n-k+1)a_n{(z-z_0)}^{n-k}$$ for $|z-z_0|<R$ and $\forall k\in\NN \cup\{0\}$. In particular $f^{(k)}(z_0)=k!a_k$ $\forall k\in\NN\cup\{0\}$.
  \end{corollary}
  \begin{proposition}
    Let $U\subseteq\CC$ be a connected open set and $f\in\mathcal{H}(U)$ such that $f'(z)=0$ $\forall z\in U$. Then, $f$ is constant.
  \end{proposition}
  \subsubsection{Determination of the logarithm}
  \begin{definition}
    Let $U\subseteq\CC$ be an open set and $f:U\longrightarrow\CC^*$ and $g:U\rightarrow\CC$ be functions. We say that $g$ is a \emph{determination} of $\ln f(z)$ if $g$ is continuous on $U$ and $\exp{g(z)}=f(z)$ $\forall z\in U$. In particular, we say that $g$ is a \emph{determination of the logarithm} if $g$ is continuous on $U$ and $\exp{g(z)}=z$ $\forall z\in U$.
  \end{definition}
  \begin{proposition}
    Let $\theta\in[0,2\pi)$, $L_\theta:=\{r\exp{\ii\theta}:r\in\RR_{\geq 0}\}$ and $U=\CC\setminus L_\theta$. Then, there exists a determination of the logarithm $g:U\rightarrow\CC$ defined as:
    $$g(z)=\ln |z|+\ii\arg(z)\quad\arg(z)\in(\theta,\theta+2\pi)$$
  \end{proposition}
  \begin{theorem}
    Let $U,V\subseteq\CC$ be open sets, $f:U\longrightarrow\CC^*$ be a function and $g\in\mathcal{H}(V)$. Suppose $f(U)\subseteq V$, $g(f(z))=z$ and $g'(f(z))\ne 0$ $\forall z\in U$. Then, $f$ is holomorphic on $U$ and: $$f'(z)=\frac{1}{g'(f(z))}\quad\forall z\in U$$
  \end{theorem}
  \begin{proposition}
    Let $U,V_1,V_2\subseteq\CC$ be open sets, $f:U\rightarrow\CC^*$ be a continuous function and $g_1:V_1\longrightarrow\CC^*$, $g_2:V_2\longrightarrow\CC^*$ be two determinations of the logarithm. If $W\subseteq V_1\cap V_2\ne\varnothing$ is connected, then $\exists k\in\ZZ$ such that: $$g_2(z)=g_1(z)+2\pi \ii k\qquad\forall z\in W$$
  \end{proposition}
  \begin{corollary}
    Let $U\subseteq\CC$ be an open set, $f:U\longrightarrow\CC^*$ be a holomorphic function and $g:U\rightarrow\CC$ be a determination of $\ln f$. Then, $g$ is holomorphic and: $$g'(z)=\frac{f'(z)}{f(z)}\quad \forall z\in U$$
    In particular, if $g$ is a determination of the logarithm, then $g'(z)=\frac{1}{z}$.
  \end{corollary}
  \subsubsection{Cauchy-Riemann equations}
  \begin{definition}
    Let $U\subseteq \CC$ be an open set, $z_0 \in U$ and $f:U\rightarrow \CC$ be a function. We define the partial derivatives of $f$ at $z_0$ as
    \begin{align*}
      \pdv{f}{x}(z_0) & =\lim_{t\to 0}\frac{f(z_0+t)-f(z_0)}{t}     \\
      \pdv{f}{y}(z_0) & =\lim_{t\to 0}\frac{f(z_0+\ii t)-f(z_0)}{t}
    \end{align*}
    whenever the limits exist.
  \end{definition}
  \begin{proposition}
    Let $U\subseteq \CC$ be an open set and $f:U\rightarrow \CC$ be a function such that $f(z)=u(z)+\ii v(z)$ $\forall z\in U$ with $u,v:U\rightarrow\RR$. Then, if $z=x+\ii y$, we have:
    \begin{align*}
      \pdv{f}{x}(z)=\pdv{u}{x}(z)+\ii\pdv{v}{x}(z) \\
      \pdv{f}{y}(z)=\pdv{u}{y}(z)+\ii\pdv{v}{y}(z)
    \end{align*}
  \end{proposition}
  \begin{definition}
    Let $U\subseteq \CC$ be set and $f:U\rightarrow \CC$ be a function such that $f(z)=u(z)+\ii v(z)$ $\forall z\in U$ with $u,v:U\rightarrow\RR$. We define the \emph{associated multivalued function} of $f$ as the function $\vf{F}:\RR^2\rightarrow\RR^2$ defined by $\vf{F}(x,y)=(u(x,y),v(x,y)):=(u(x+\ii y),v(x+\ii y))$.
  \end{definition}
  \begin{definition}
    Let $U\subseteq \CC$ be an open set, $z_0\in U$, $f:U\rightarrow \CC$ be a function. We say that $f$ is \emph{$\RR$-differentiable} at $z_0\in U$ if and only if there exists a $\RR$-linear function $\vf{L}:\CC\rightarrow\RR^2$ such that:
    $$\lim_{h+k\ii\to 0}\frac{|f(z_0+h+k\ii)-f(z_0)-\vf{L}(h+k\ii)|}{|h+k\ii|}=0\footnote{Here we shall think the outcomes of $\vf{L}$ inside $\CC$ instead of $\RR^2$.}$$
    In that case, the function $\vf{L}$ is called \emph{differential} of $f$ at $z_0$ and it is denoted by $\vf{D}f(z_0)$\footnote{That is, the $\RR$-differentiability is the usual one if we think $f$ inside $\RR^2$ instead of inside $\CC$.}.
  \end{definition}
  \begin{proposition}
    Let $U\subseteq \CC$ be an open set, $z_0\in U$ and $f:U\rightarrow \CC$ be a $\RR$-differentiable function at $z_0$ such that $f(z)=u(z)+\ii v(z)$ $\forall z\in U$ with $u,v:U\rightarrow\RR$. Then:
    $$\vf{D}f(z_0)=
      \begin{pmatrix}
        \pdv{u}{x}(z_0) & \pdv{u}{y}(z_0) \\
        \pdv{v}{x}(z_0) & \pdv{v}{y}(z_0)
      \end{pmatrix}\footnote{From now on, if we use the matrix notation in the complex plane, that should be interpreted as the first row being the real part and the second row being the imaginary part.}
    $$
  \end{proposition}
  \begin{theorem}[Cauchy-Riemann theorem]
    Let $U\subseteq \CC$ be an open set,  $f:U\rightarrow \CC$ be a function and $z_0\in U$. Then, $f$ is $\CC$-differentiable and $f'(z_0)=a+b\ii$ if and only if $f$ is $\RR$-differentiable and:
    $$\vf{D}f(z_0)=
      \begin{pmatrix}
        a & -b \\
        b & a
      \end{pmatrix}$$
    Which is equivalent to:
    \begin{equation}\label{CAFA_cauchyrieamann}
      \pdv{u}{x}(z_0)=\pdv{v}{y}(z_0)\qquad\pdv{v}{x}(z_0)=-\pdv{u}{y}(z_0)
    \end{equation}
    These equations are called \emph{Cauchy-Riemann equations}.
  \end{theorem}
  \begin{corollary}
    Let $U\subseteq \CC$ be an open set, $u,v:U\rightarrow\CC$ be functions such that their partial derivatives exist, they are continuous and they satisfy \cref{CAFA_cauchyrieamann}. Then, $f=u+\ii v$ is holomorphic on $U$.
  \end{corollary}
  \begin{definition}
    Let $U\subseteq \CC$ be an open set, $f:U\rightarrow \CC$ be a function. We define the \emph{Wirtinger operators} as:
    \begin{align*}
      \partial f:=\pdv{f}{z}                       & :=\frac{1}{2}\left(\pdv{f}{x}-\ii\pdv{f}{y}\right) \\
      \overline{\partial} f:=\pdv{f}{\overline{z}} & :=\frac{1}{2}\left(\pdv{f}{x}+\ii\pdv{f}{y}\right)
    \end{align*}
  \end{definition}
  \begin{proposition}
    Let $U\subseteq \CC$ be an open set, $f,g:U\rightarrow \CC$ be $\RR$-differentiable functions, $z\in U$ and $w=g(z)$. Then:
    \begin{enumerate}
      \item $\partial(f\circ g)(z)=\partial f(w)\partial g(z)+\overline{\partial}f(w)\partial \overline{g}(z)$
      \item $\overline{\partial}(f\circ g)(z)=\partial f(w)\overline{\partial} g(z)+\overline{\partial}f(w)\overline{\partial} \overline{g}(z)$
    \end{enumerate}
  \end{proposition}
  \begin{proposition}
    Let $U\subseteq \CC$ be an open set, $z_0\in U$ and $f:U\rightarrow \CC$ be a $\CC$-differentiable function at $z_0$. Then, the Cauchy-Riemann equations of \cref{CAFA_cauchyrieamann} can also be written as: $$\overline{\partial} f(z_0)=0$$
  \end{proposition}
  \begin{proposition}
    Let $U\subseteq \CC$ be an open set, $f:U\rightarrow \CC$ be a function and $z_0\in U$. We say that $f$ is $\RR$-differentiable at $z_0\in U$ if and only if there exist $x,y\in \CC$ such that: $$\lim_{h\to 0}\frac{f(z_0+h)-f(z_0)-xh-y\overline{h}}{h}=0$$
    In that case, we have $x=\partial f(z_0)$ and $y=\overline{\partial} f(z_0)$.
  \end{proposition}
  \begin{proposition}
    Let $U\subseteq \CC$ be an open set and $f\in\mathcal{H}(U)$ such that $f=u+\ii v$ with $u,v:U\rightarrow\RR$. Then:
    $${|f'(z)|}^2=\det \vf{D}(u,v)(z)={\left(\pdv{u}{x}\right)}^2+{\left(\pdv{u}{y}\right)}^2$$
  \end{proposition}
  \begin{proposition}
    Let $U\subseteq \CC$ be an open connected set and $f\in\mathcal{H}(U)$ such that it satisfies one of the following properties:
    \begin{itemize}
      \item $\Re f=\const$
      \item $\Im f=\const$
      \item $\abs{f}=\const$
      \item $f(U)$ is contained in a line.
      \item $f(U)$ is contained in a circle.
    \end{itemize}
    Then, $f$ is constant.
  \end{proposition}
  \subsubsection{Harmonic functions}
  \begin{definition}
    Let $U\subseteq \CC\cong\RR^2$ be an open set and $f:U\rightarrow \RR$ be a function of class $\mathcal{C}^2$. We say that $f$ is \emph{harmonic} if $\laplacian f(z)=0$ $\forall z\in U$\footnote{Recall \cref{FOSV_laplacian}.}.
  \end{definition}
  \begin{proposition}
    Let $U\subseteq \CC$ be an open set and $f\in\mathcal{H}(U)$ such that $f=u+\ii v$ with $u,v:U\rightarrow\RR$. Then, $u$ and $v$ are harmonic.
  \end{proposition}
  \begin{definition}
    Let $S\subseteq\RR^n$ be a subset and $s_0\in S$. We say that $S$ is a \emph{star domain} with respect to $s_0$ if $\forall s\in S$, the line segment from $s_0$ to $s$ lies in $S$.
  \end{definition}
  \begin{definition}
    Let $U\subseteq\RR^n$ be a subset. We say that $U$ is \emph{convex} if it is a star domain with respect to $x\in U$, for all $x\in U$.
  \end{definition}
  \begin{proposition}
    Let $U\subseteq \CC$ be an open star domain with respect to $z_0=x_0+\ii y_0\in U$ and $u:U\rightarrow\RR$ be a harmonic function. Then, there exists a harmonic function $v:U\rightarrow\RR$ such that $f=u+\ii v$ is holomorphic. That function $v$ is:
    $$v(x+\ii y)=\int_{y_0}^y\pdv{u}{x}(x+\ii t)\dd{t}-\int_{x_0}^x\pdv{u}{y}(t+\ii y_0)\dd{t}+C$$
    for some constant $C\in\RR$.
  \end{proposition}
  \subsection{Möbius transformations}
  \subsubsection{Möbius transformations}
  \begin{definition}[Möbius transformations]
    A \emph{Möbius transformation}\footnote{Also called linear fractional transformation, homography or homographic transformation.} of the complex plane is a rational function $f:\CC_\infty\rightarrow\CC_\infty$ of the form: $$f(z)=\frac{az+b}{cz+d}$$ with $a,b,c,d\in\CC$ and $ad-bc\ne 0$. The special values $f(\infty)$ and $f\left(-\frac{d}{c}\right)$ are defined conveniently as $f(\infty)=\frac{a}{c}$ and $f\left(-\frac{d}{c}\right)=\infty$. The set of all Möbius transformations is denoted by $\mathcal{M}$.
  \end{definition}
  \begin{proposition}
    Let $f,g\in\mathcal{M}$ such that $$f(z)=\frac{az+b}{cz+d}\quad\text{and}\quad g(z)=\frac{a'z+b'}{c'z+d'}$$
    Then, $f(z)=g(z)\iff \exists\lambda\in\CC^*$ such that $a'=\lambda a$, $b'=\lambda b$, $c'=\lambda c$ and $d'=\lambda d$.
  \end{proposition}
  \begin{proposition}
    Let $f\in\mathcal{M}$. Then, $f$ is a homeomorphism of $\CC_\infty$ and moreover if $\circ$ denotes the composition of Möbius transformations, $(\mathcal{M},\circ)$ is a non-abelian group. Moreover we have the following group morphism:
    $$\function{}{\GL_2(\CC)}{\mathcal{M}}{\begin{pmatrix}
          a & b \\
          c & d
        \end{pmatrix}}{\displaystyle\frac{az+b}{cz+d}}$$
  \end{proposition}
  \begin{definition}
    Let $f\in\mathcal{M}$ and $a,\exp{\ii\theta}\in\CC$. We say that $f$ is
    \begin{itemize}
      \item a \emph{translation} if $f(z)=z+a$.
      \item a \emph{rotation} if $f(z)=\exp{\ii\theta}z$.
      \item a \emph{dilatation} if $a\ne 0$ and $f(z)=az$.
      \item an \emph{inversion} if $f(z)=\frac{1}{z}$.
    \end{itemize}
  \end{definition}
  \begin{theorem}
    Let $f\in\mathcal{M}$. Then, $f$ is a composition of a finite number of translations and dilatations and an inversion.
  \end{theorem}
  \begin{proposition}
    Let $f=\frac{az+b}{cz+d}\in\mathcal{M}$. Then, $z\in\CC_\infty$ is a fixed point for $f$ if and only if: $$c^2z^2+(d-a)z-b=0$$
    Hence, $f$ can have at most two fixed points.
  \end{proposition}
  \begin{corollary}
    Let $f,g\in\mathcal{M}$ such that they coincide in three points. Then, $f=g$.
  \end{corollary}
  \subsubsection{Cross ratio}
  \begin{definition}
    Let $z_2,z_3,z_4\in\CC_\infty$ be distinct points and $z\in\CC_\infty$. We define the \emph{cross ratio} of $z$, $z_2$, $z_3$ and $z_4$, denoted by $(z,z_2,z_3,z_4)$ as the image under the unique Möbius transformation $f\in\mathcal{M}$ such that $f(z_2)=1$, $f(z_3)=0$ and $f(z_4)=\infty$.
  \end{definition}
  \begin{proposition}
    Let $z_2,z_3,z_4\in\CC_\infty$ be distinct points and $z\in\CC_\infty$. Then: $$(z,z_2,z_3,z_4)=\frac{z-z_3}{z-z_4}\cdot\frac{z_2-z_4}{z_2-z_3}$$
  \end{proposition}
  \begin{corollary}
    Let $z_1,z_2,z_3\in\CC_\infty$ and $w_1,w_2,w_3\in\CC_\infty$ be two triplets of distinct points. Then, $\exists! f\in\mathcal{M}$ such that: $$f(z_1)=w_1\quad f(z_2)=w_2\quad f(z_3)=w_3$$
  \end{corollary}
  \begin{theorem}
    Let $z_1,z_2,z_3\in\CC_\infty$ be distinct points and $f\in\mathcal{M}$. Then: $$(z,z_2,z_3,z_4)=(f(z),f(z_2),f(z_3),f(z_4))$$
  \end{theorem}
  \subsubsection{Circles in \texorpdfstring{$\CC_\infty$}{Coo}}
  \begin{definition}
    We define a \emph{circle} in $\CC_\infty$ as a circle in $\CC$ or the set $r\cup\{\infty\}$, where $r$ is a line in $\CC$\footnote{Note that with this definition $\RR_\infty$ is a circle.}.
  \end{definition}
  \begin{proposition}
    Given three points of $\CC_\infty$, there exists a unique circle that passes through them.
  \end{proposition}
  \begin{proposition}
    Let $f\in\mathcal{M}$. Then, $f(\RR_\infty)$ is a circle of $\CC_\infty$.
  \end{proposition}
  \begin{proposition}
    Let $z_1,z_2,z_3,z_4\in\CC_\infty$ be distinct points. Then, these points lie on a circle if and only if $(z_1,z_2,z_3,z_4)\in\RR_\infty$
  \end{proposition}
  \begin{theorem}
    Let $f\in\mathcal{M}$ and $C\subset\CC_\infty$ be a circle. Then, $f(C)\subset\CC_\infty$ is also a circle.
  \end{theorem}
  \subsubsection{Orientation and symmetry}
  \begin{definition}
    Let $C\subset\CC_\infty$ be a circle and $z_2,z_3,z_4\in C$. An \emph{orientation} of $C$ is an ordered triplet $(z_2,z_3,z_4)$\footnote{The orientation is determined by going from $z_2$ to $z_3$ without passing through $z_4$, by going form $z_3$ to $z_4$ without passing through $z_2$ and by going form $z_4$ to $z_2$ without passing through $z_3$, all these travels always on $C$.}. This determines a partition of the plane into three sets:
    \begin{itemize}
      \item \emph{Right side} of $C$: $\{z\in\CC_\infty:\Im(z,z_2,z_3,z_4)>0\}$
      \item \emph{Left side} of $C$: $\{z\in\CC_\infty:\Im(z,z_2,z_3,z_4)<0\}$
      \item \emph{Center} of $C$: $\{z\in\CC_\infty:\Im(z,z_2,z_3,z_4)=0\}$
    \end{itemize}
  \end{definition}
  \begin{theorem}[Orientation principle]
    Let $C\subset\CC_\infty$ be a circle, $(z_1,z_2,z_3)$ be an orientation of $C$, and $f\in\mathcal{M}$. Then, $C':=f(C)$ is a circle and $f$ carries the right/left side of $C$ to the right/left side of $C'$.
  \end{theorem}
  \begin{definition}
    Let $C\subset\CC_\infty$ be a circle, $z_2,z_3,z_4\in C$ and $z,z^*\in\CC_\infty$. We say that $z$ and $z^*$ are symmetrics with respect to $C$ if: $$(z,z_2,z_3,z_4)=\overline{(z^*,z_2,z_3,z_4)}\footnote{It can be seen that this definition does not depend on the triplet $(z_2,z_3,z_4)$ chosen.}$$ The function $$\function{R_C}{\CC_\infty}{\CC_\infty}{z}{z^*}$$ is called \emph{reflection} with respect to $C$.
  \end{definition}
  \begin{theorem}[Symmetry principle]
    Let $C\subset\CC_\infty$ be a circle, $f\in\mathcal{M}$ and $z,z^*\in\CC_\infty$. Then, if $z$, $z^*$ are symmetric with respect to $C$, then $f(z)$ and $f(z^*)$ are symmetric with respect to $f(C)$.
  \end{theorem}
  \begin{proposition}
    Let $C\subset\CC$ be a circle of center $a$ and radius $R$. Then: $$R_C(z)=a+\frac{R^2}{\abs{z-a}^2}(z-a)$$
  \end{proposition}
  \begin{corollary}
    Let $C_1,C_2\subset\CC_\infty$ be two circles. Then, there exists $f\in\mathcal{M}$ such that $f(C_1)=C_2$. Moreover if we fix the image of three points of $C_1$, the transformation $f$ is unique. Moreover, $f(\Int C_1)$ maps to either $\Int C_2$ or $\Ext C_2$ and the same happens with $\Ext C_2$.
  \end{corollary}
  \begin{proposition}
    Let $a\in\CC$ with $\abs{a}<1$ and $\theta\in\RR$. Then, the Möbius transformations
    $$f(z)=\exp{\ii \theta}\frac{z-a}{1-\overline{a}z}$$
    are the unique holomorphic and bijective functions that carry $D(0,1)$ to $D(0,1)$.
  \end{proposition}
  \subsection{Integration}
  \begin{definition}
    Let $f:[a,b]\subset\RR\rightarrow\CC$ be a function such that $f=u+\ii v$ with $u,v:[a,b]\rightarrow\RR$. We define the \emph{integral} of $f$ on the interval $[a,b]$ as: $$\int_a^bf(t)\dd{t}:=\int_a^bu(t)\dd{t}+\ii\int_a^bv(t)\dd{t}$$
  \end{definition}
  \begin{proposition}
    Let $f,g:[a,b]\subset\RR\rightarrow\CC$ be functions and $\alpha\in\CC$. Then:
    \begin{enumerate}
      \item $\displaystyle\int_a^b(f+g)(t)\dd{t}=\int_a^bf(t)\dd{t}+\int_a^bg(t)\dd{t}$
      \item $\displaystyle\int_a^b(\alpha f)(t)\dd{t}=\alpha \int_a^bf(t)\dd{t}$
      \item $\displaystyle\left|\int_a^bf(t)\dd{t}\right|\leq\int_a^b|f(t)|\dd{t}$
      \item $\displaystyle \int_b^af(t)\dd{t}=-\int_a^bf(t)\dd{t}$
    \end{enumerate}
  \end{proposition}
  \begin{proposition}
    Let $f:[a,b]\subset\RR\rightarrow\CC$ be a function and $\varphi:[a,b]\rightarrow\RR$ be a change of variable of class $\mathcal{C}^1$ such that $\varphi([a,b])\subseteq[a,b]$. Then: $$\int_a^bf(\varphi(t))\varphi'(t)\dd{t}=\int_{\varphi(a)}^{\varphi(b)}f(t)\dd{t}$$
  \end{proposition}
  \subsubsection{Curves}
  \begin{definition}
    Let $\gamma_1:[a,b]\rightarrow\CC$, $\gamma_2:[c,d]\rightarrow\CC$ be two curves\footnote{Recall \cref{DG_curves}.} of class $\mathcal{C}^1$ such that $\gamma_1(t_1)=\gamma_2(t_2)=z_0\in\CC$ for some $t_1\in[a,b]$ and $t_2\in[c,d]$. Suppose that ${\gamma_1}'(t_1),{\gamma_2}'(t_2)\ne 0$. We define the angle between $\gamma_1$ and $\gamma_2$ at $z_0$ as $$\arg{\gamma_1}'(t_1)-\arg{\gamma_2}'(t_2)$$ which does not depend on the determination of the argument.
  \end{definition}
  \begin{definition}
    Let $U\subseteq \CC$ be an open set, $f:U\rightarrow\CC$ be a function and $z_0\in\CC$. We say that $f$ is \emph{conformal} at $z_0$ if it preserves the oriented angle between curves that intersect at $z_0$. We say that $f$ is \emph{conformal} if it is conformal at each point $z\in U$.
  \end{definition}
  \begin{definition}
    Let $U,V\subseteq \CC$ be open sets and $f:U\rightarrow V$ be a function. We say that $f$ is a \emph{conformal representation} between $U$ and $V$ if $f$ is bijective, holomorphic and its inverse is also holomorphic.
  \end{definition}
  \begin{theorem}
    Let $U\subseteq \CC$ be an open set and $f\in\mathcal{H}(U)$. Then, $f$ is conformal at the points $z\in U$ such that $f'(z)\ne 0$.
  \end{theorem}
  \begin{definition}
    Let $\gamma:[a,b]\rightarrow U$ be a piecewise path of class $\mathcal{C}^1$. We define the inverse path $\gamma^-:[a,b]\rightarrow\CC$ as $\gamma^-(t)=\gamma(a+b-t)$
  \end{definition}
  \begin{definition}
    Let $\gamma:[a,b]\rightarrow U$ be an injective curve. We say that $\gamma$ is a triangular path if $\gamma^*=\Fr T$, where $T\subset\CC$ is a triangle. The domain delimited by $T$ is denoted by $\overline{D(\gamma)}:=\overline{T}$.
  \end{definition}
  \subsubsection{Line integration}
  \begin{definition}
    Let $U\subseteq \CC$ be an open set, $f:U\subset\RR\rightarrow\CC$ be a continuous function, $\gamma:[a,b]\rightarrow U$ be a rectifiable curve and $\{t_0,\ldots,t_n\}$ be a partition of $[a,b]$. We define the \emph{line integral} of $f$ along $\gamma$ as: $$\int_\gamma f(z)\dd{z}:=\lim_{n\to\infty}\sum_{j=0}^{n-1}f(\gamma(\eta_j))(\gamma(t_{j+1})-\gamma(t_j))$$ where $\eta_j\in[t_j,t_{j+1}]$ $\forall j\in\{0,1,\ldots,n-1\}$.
  \end{definition}
  \begin{definition}
    Let $U\subseteq \CC$ be an open set, $f:U\subset\RR\rightarrow\CC$ be a continuous function and $\gamma:[a,b]\rightarrow U$ be a piecewise curve of class $\mathcal{C}^1$. Suppose $\{t_0,\ldots,t_n\}$ is partition of $[a,b]$ with the property that $\gamma\in\mathcal{C}^1([t_j,t_{j+1}])$ $\forall j\in\{0,1,\ldots,n-1\}$. Then: $$\int_\gamma f(z)\dd{z}=\sum_{j=0}^{n-1}\int_{t_j}^{t_{j+1}}f(\gamma(t))\gamma'(t)\dd{t}$$
    In particular, if $\gamma\in\mathcal{C}^1([a,b])$, then: $$\int_\gamma f(z)\dd{z}=\int_a^bf(\gamma(t))\gamma'(t)\dd{t}$$
  \end{definition}
  \begin{theorem}
    Let $U\subseteq \CC$ be an open set, $f:U\subset\RR\rightarrow\CC$ be a continuous function and $\gamma:[a,b]\rightarrow U$ be a piecewise curve of class $\mathcal{C}^1$. Suppose there exists a function $F\in\mathcal{H}(U)$ such that $F'(z)=f(z)$ $\forall z\in U$. Then:
    $$\int_\gamma f(z)\dd{z}=F(\gamma(b))-F(\gamma(a))$$ In particular if $\gamma$ is a closed curve, then $\int_\gamma f(z)\dd{z}=0$.
  \end{theorem}
  \begin{corollary}
    There is no determination of the logarithm in any set of the form $\{z\in\CC:0<r\leq\abs{z}\leq R\}$ for $r,R\in\RR_{> 0}$ with $r<R$.
  \end{corollary}
  \begin{proposition}
    Let $U\subseteq \CC$ be an open set, $f:U\subset\RR\rightarrow\CC$ be a continuous function and $\gamma:[a,b]\rightarrow U$ be a piecewise curve of class $\mathcal{C}^1$. Let $L(\gamma)$ be the length of $\gamma$. Then: $$L(\gamma)=\int_a^b\abs{\gamma'(t)}\dd{t}$$
  \end{proposition}
  \begin{proposition}
    Let $U\subseteq \CC$ be an open set, $f:U\subset\RR\rightarrow\CC$ be a continuous function, $\gamma_1:[a,b]\rightarrow U$ be a piecewise curve of class $\mathcal{C}^1$ and $\gamma_2$ be a reparametrization of $\gamma_1$.
    \begin{itemize}
      \item If the parametrization is positive\footnote{Recall \cref{DE_reparam}.}, then: $$\int_{\gamma_1}f(z)\dd{z}=\int_{\gamma_2}f(z)\dd{z}$$
      \item If the parametrization is negative, then: $$\int_{\gamma_1}f(z)\dd{z}=-\int_{\gamma_2}f(z)\dd{z}$$
            In particular, for any piecewise curve of class $\mathcal{C}^1$ $\gamma:[a,b]\rightarrow U$ we have: $$\int_{\gamma}f(z)\dd{z}=-\int_{\gamma^-}f(z)\dd{z}$$
    \end{itemize}
  \end{proposition}
  \begin{definition}
    Let $U\subseteq \CC$ be an open set, $f:U\subset\RR\rightarrow\CC$ be a continuous function and $\gamma:[a,b]\rightarrow\CC$ be a piecewise curve of class $\mathcal{C}^1$ with respect to the partition $\{t_0,\ldots,t_n\}$ of $[a,b]$. We define the line integral with respect to the length as: $$\int_{\gamma}f(z)\abs{\dd{z}}:=\sum_{j=0}^{n-1}\int_{t_j}^{t_{j+1}}f(\gamma(t))|\gamma'(t)|\dd{t}$$
  \end{definition}
  \begin{definition}
    Let $U\subseteq \CC$ be an open set, $f:U\subset\RR\rightarrow\CC$ be a continuous function and $\gamma:[a,b]\rightarrow\CC$ be a piecewise curve of class $\mathcal{C}^1$. We define the following line integral with respect to $\overline{z}$: $$\int_{\gamma}f(z)\dd{\overline{z}}:=\overline{\int_{\gamma}\overline{f(z)}\dd{z}}$$
  \end{definition}
  \begin{proposition}
    Let $U\subseteq \CC$ be an open set, $f:U\subset\RR\rightarrow\CC$ be a continuous function and $\gamma:[a,b]\rightarrow\CC$ be a piecewise curve of class $\mathcal{C}^1$. Then:
    \begin{enumerate}
      \item $\displaystyle\int_{\gamma}\abs{\dd{z}}=L(\gamma)$
      \item $\displaystyle\left|\int_{\gamma}f(z)\dd{z}\right|\leq\int_{\gamma}|f(z)|\abs{\dd{z}}$
    \end{enumerate}
  \end{proposition}
  \subsubsection{Local Cauchy theory}
  \begin{theorem}[Goursat's theorem]
    Let $U\subseteq \CC$ be an open set, $f\in\mathcal{H}(U)$ and $\gamma$ be a triangular path such that $\overline{D(\gamma)}\subseteq U$. Then: $$\int_\gamma f(z)\dd{z}=0$$
  \end{theorem}
  \begin{theorem}[Local Cauchy's integral theorem]
    Let $U\subseteq \CC$ be a convex open set, $f\in\mathcal{H}(U)$ and $\gamma:[a,b]\rightarrow\CC$ be a closed piecewise curve of class $\mathcal{C}^1$ with $\gamma^*\subset U$. Then: $$\int_\gamma f(z)\dd{z}=0$$
  \end{theorem}
  \begin{lemma}
    Let $U\subseteq \CC$ be an open and convex set, $z_0\in U$, $f\in\mathcal{H}(U\setminus\{z_0\})$ and $\gamma:[a,b]\rightarrow\CC$ be a closed piecewise curve of class $\mathcal{C}^1$ such that $\gamma^*\subseteq U\setminus\{z_0\}$. Suppose $\displaystyle\lim_{z\to z_0}f(z)(z-z_0)=0$. Then: $$\int_\gamma f(z)\dd{z}=0$$
  \end{lemma}
  \begin{definition}
    We define the \emph{Fresnel integrals} $S(z)$ and $C(z)$ as:
    $$S(z)=\int_0^z\sin(\zeta^2)\dd{\zeta}\qquad C(z)=\int_0^z\cos(\zeta^2)\dd{\zeta}$$
  \end{definition}
  \subsubsection{Index of a curve}
  \begin{theorem}
    Let $\gamma:[a,b]\rightarrow\CC$ be a closed curve and $z_0\notin\gamma^*$. Then, there exist an open disc $D_0$ centered at $z_0$ such that $D_0\cap\gamma^*=\varnothing$, and a determination $\Phi:[a,b]\times D_0\rightarrow\CC$ of $\ln(\gamma(t)-z_0)$. Moreover, given $t\in[a,b]$ the function $z\mapsto \Phi(t,z)$ is holomorphic and for each $z\in D_0$ the function $t\mapsto\Phi(t,z)$ has the same differentiability as $\gamma$.
  \end{theorem}
  \begin{definition}
    Let $\gamma:[a,b]\rightarrow\CC$ be a closed curve, $z_0\notin\gamma^*$ and $\Phi:[a,b]\times D_0\rightarrow\CC$ be a determination of $\ln(\gamma(t)-z_0)$. We define the \emph{index} (or \emph{winding number}) of $\gamma$ with respect to $z_0$, denoted by $\Ind(\gamma,z_0)$, as: $$\Ind(\gamma,z_0):=\frac{\Phi(b,z_0)-\Phi(a,z_0)}{2\pi\ii}\footnote{Intuitively it can be though as the number of turns made by $\gamma$ around $z_0$.}$$
  \end{definition}
  \begin{proposition}
    Let $\gamma:[a,b]\rightarrow\CC$ be a closed curve and $z_0\notin\gamma^*$. Then:
    \begin{enumerate}
      \item The value $\Ind(\gamma,z_0)\in\ZZ$ does not depend on the determination of $\ln(\gamma(t)-z_0)$ chosen.
      \item $\Ind(\gamma,z_0)\in\ZZ$.
      \item $\Ind(\gamma^-,z_0)=-\Ind(\gamma,z_0)$.
      \item The function $$\function{}{\CC\setminus\gamma^*}{\ZZ}{z}{\Ind(\gamma,z)}$$ is continuous and therefore it is constant on each connected component of $\CC\setminus\gamma^*$.
      \item If $z\in\CC\setminus\gamma^*$ is in the unbounded component of $\CC\setminus\gamma^*$, then $\Ind(\gamma,z)=0$.
    \end{enumerate}
  \end{proposition}
  \begin{theorem}
    Let $\gamma:[a,b]\rightarrow\CC$ be a closed piecewise curve of class $\mathcal{C}^1$ and $z_0\notin\gamma^*$. Then:
    $$\Ind(\gamma,z_0)=\frac{1}{2\pi\ii}\int_\gamma\frac{\dd{z}}{z-z_0}$$
  \end{theorem}
  \begin{theorem}[Local Cauchy's integral formula]
    Let $U\subseteq\CC$ be an open and convex set, $f\in\mathcal{H}(U)$, $\gamma:[a,b]\rightarrow\CC$ be a closed piecewise curve of class $\mathcal{C}^1$ and $z_0\notin\gamma^*$. Then:
    $$f(z_0)\cdot\Ind(\gamma,z_0)=\frac{1}{2\pi\ii}\int_\gamma\frac{f(z)}{z-z_0}\dd{z}$$
  \end{theorem}
  \begin{corollary}
    Let $U\subseteq\CC$ be an open and convex set, $f\in\mathcal{H}(U)$, $z_0\in\CC$ and $r\in\RR_{\geq 0}$. Then:
    $$f(z_0)=\frac{1}{2\pi}\int_0^{2\pi}f(z_0+r\exp{\ii\theta})\dd{\theta}$$
  \end{corollary}
  \begin{lemma}
    Let $U\subseteq\CC$ be an open, $(f_n)\in U\subseteq\CC$ be a sequence of functions, $f:U\rightarrow\CC$ such that $(f_n)$ converge uniformly to $f$ over compact sets in $U$, and $\gamma:[a,b]\rightarrow\CC$ be a closed piecewise curve of class $\mathcal{C}^1$. Then:
    $$\lim_{n\to\infty}\int_\gamma f_n(z)\dd{z}=\int_\gamma f(z)\dd{z}$$
  \end{lemma}
  \begin{corollary}
    Let $z_0\in\CC$ and $f(z)=\sum a_n{(z-z_0)}^n$. Then:
    $$\int_0^z f(\zeta)\dd{\zeta}=\sum_{n=0}^\infty a_n\frac{{(z-z_0)}^{n+1}}{n+1}$$
  \end{corollary}
  \subsubsection{Analytic functions}
  \begin{definition}
    Let $U\subseteq\CC$ be an open set and $f:U\rightarrow \CC$ be a function. We say that $f$ is \emph{analytic} on $U$ if for each $z_0\in U$, there exists a power series $\sum_{n=0}^\infty a_n{(z-z_0)}^n$ with radius of convergence $R_{z_0}$ such that $$f(z)=\sum_{n=0}^\infty a_n{(z-z_0)}^n$$ in a neighbourhood of $z$.
  \end{definition}
  \begin{theorem}
    Let $U\subseteq\CC$ be an open set and $f\in\mathcal{H}(U)$. Then, $\forall z_0\in U$ there exists a power series $\sum a_n{(z-z_0)}^n$ with radius of convergence $R_{z_0}\geq d(z_0,\Fr U)>0$ such that: $$f(z)=\sum_{n=0}^\infty a_n{(z-z_0)}^n$$ That is, $f$ is analytic.
  \end{theorem}
  \begin{corollary}
    Let $U\subseteq\CC$ be an open set and $f:U\rightarrow \CC$ be a function. Then, $f$ is analytic on $U$ if and only if $f$ is holomorphic on $U$.
  \end{corollary}
  \begin{corollary}[Local Cauchy's integral formula for derivatives]
    Let $U\subseteq\CC$ be an open set, $f\in\mathcal{H}(U)$ and $z_0\in U$, $r\in\RR_{>0}$ such that $\overline{D(z_0,r)}\subset U$. Let $\gamma(t)=z_0+r\exp{\ii t}$, $t\in[0,2\pi]$. Then:
    $$f^{(n)}(z_0)=\frac{n!}{2\pi\ii}\int_\gamma\frac{f(z)}{{(z-z_0)}^{n+1}}\dd{z}$$
  \end{corollary}
  \begin{corollary}
    Let $f\in\mathcal{H}(\CC)$. Then, for all $z_0\in\CC$ there exists a power series $\sum a_n{(z-z_0)}^n$ with infintie radius of convergence such that: $$f(z)=\sum_{n=0}^\infty a_n{(z-z_0)}^n$$
  \end{corollary}
  \begin{corollary}
    Let $f(z)=\sum a_n{(z-z_0)}^n$ be a power series with radius of convergence $R\in(0,\infty)$. Then, $f$ is analytic on $D(z_0,R)$ and $\forall w_0\in D(z_0,R)$ we have $$f(z)=\sum_{n=0}^\infty b_n{(z-w_0)}^n$$ whenever $\abs{z-w_0}<R-|z_0-w_0|$. Here the coefficients $b_n$ can be determined with the formula $b_n=\frac{f^{(n)}(w_0)}{n!}$.
  \end{corollary}
  \subsubsection{Some important theorems}
  \begin{proposition}[Cauchy's inequality]
    Let $U\subseteq\CC$ be an open set, $f\in\mathcal{H}(U)$ and $z_0\in U$, $r\in\RR_{>0}$ such that $\overline{D(z_0,r)}\subset U$. Let $M=\sup\{\abs{f(z)}:\abs{z-z_0}=r\}$. Then:
    $$\abs{f^{(n)}(z_0)}\leq\frac{n!M}{r^n}$$
  \end{proposition}
  \begin{theorem}[Liouville's theorem]
    Let $f\in\mathcal{H}(\CC)$ and bounded. Then, $f$ is constant.
  \end{theorem}
  \begin{corollary}
    Let $f\in\mathcal{H}(\CC)$ such that $\Re(f(z))\geq 0$ $\forall z\in\CC$. Then, $f$ is constant.
  \end{corollary}
  \begin{theorem}[Fundamental theorem of algebra]
    Let $p(z)=a_0+a_1z\cdots+a_nz^n\in\CC[z]$ with $a_n\ne 0$. Then, $\exists \alpha\in \CC$ such that $p(\alpha)=0$.
  \end{theorem}
  \begin{corollary}
    $\CC$ is an algebraically closed field.
  \end{corollary}
  \begin{proposition}[Cardano-Vieta's formulas]
    Let $p(z)=a_0+a_1z\cdots+a_nz^n\in\CC[z]$ with $a_n\ne 0$ and with roots $\alpha_1,\ldots,\alpha_n$. Then, we have the following relations:
    \begin{align*}
      \sum_{i=1}^n\alpha_i                   & =                    -\frac{a_{n-1}}{a_n}  \\
      \sum_{1\leq i<j\leq n}\alpha_i\alpha_j & =  \frac{a_{n-2}}{a_n}                     \\
                                             & \;\;\vdots                                 \\
      \alpha_1\cdots\alpha_n                 & =                  {(-1)}^n\frac{a_0}{a_n}
    \end{align*}
  \end{proposition}
  \begin{theorem}[Morera's theorem]
    Let $U\subseteq\CC$ be an open set, $f:U\rightarrow\CC$ be a continuous function such that $\displaystyle \int_\gamma f(z)\dd{z}=0$ for all triangular path $\gamma$ with $\gamma^*\subset G$. Then, $f$ is analytic.
  \end{theorem}
  \begin{theorem}[Weierstra\ss' theorem]
    Let $U\subseteq\CC$ be an open set, $(f_n)\in\mathcal{H}(U)$ be a sequence of functions such that they converge uniformly to $f:U\rightarrow\CC$ over compact sets of $U$. Then, $f\in\mathcal{H}(U)$ and $\forall k\in \NN$, ${f_n}^{(k)}$ converge uniformly to $f^{(k)}$ over compact sets.
  \end{theorem}
  \begin{corollary}
    Let $U\subseteq\CC$ be an open set, $(f_n)\in\mathcal{H}(U)$ be a sequence of functions such that they converge uniformly to 0 over compact sets of $U$, and $f(z)=\sum_{n=0}^\infty f_n(z)$. Then, $f\in\mathcal{H}(U)$ and: $$f^{(k)}(z)=\sum_{n=0}^\infty {f_n}^{(k)}(z)$$
  \end{corollary}
  \begin{theorem}[Maximum modulus principle]
    Let $U\subseteq\CC$ be an open connected set, $f\in\mathcal{H}(U)$ and $a\in U$ such that $\abs{f(a)}\geq \abs{f(z)}$ $\forall z\in \overline{D(a,r)}\subset U$, for some $r\in\RR_{>0}$. Then, $f$ is constant.
  \end{theorem}
  \begin{corollary}
    Let $U\subseteq\CC$ be an open bounded set and $f\in\mathcal{H}(U)$. Then, $\abs{f}$ has an absolute maximum on $\overline{U}$ and $$\max\{\abs{f(z)}:z\in\overline{U}\}=\max\{\abs{f(z)}:z\in\Fr U\}$$
  \end{corollary}
  \subsubsection{Zeros of holomorphic functions}
  \begin{theorem}
    Let $U\subseteq\CC$ be an open connected set and $f\in\mathcal{H}(U)$. Then, the following statements are equivalent:
    \begin{enumerate}
      \item $f(z)=0$.
      \item $\exists z_0\in U$ such that $f^{(n)}(z_0)=0$ $\forall n\in\NN\cup\{0\}$.
      \item ${\{z\in U:f(z)=0\}}'\cap U\ne\varnothing$.
    \end{enumerate}
  \end{theorem}
  \begin{corollary}
    Let $U\subseteq\CC$ be an open connected set and $f\in\mathcal{H}(U)$ such that $f$ is not identically zero. Then, if $z_0\in U$ is a zero of $f$, $\exists m\in\NN$ and $g\in\mathcal{H}(U)$ such that $g(z_0)\ne 0$ and: $$f(z)={(z-z_0)}^{m}g(z)$$ The value of $m$ is called \emph{multiplicity} of $z_0$. In particular, the zeros of $f$ are isolated.
  \end{corollary}
  \begin{theorem}[Analytic contiuation theorem]
    Let $U\subseteq\CC$ be an open connected set and $f,g\in\mathcal{H}(U)$ such that ${\{z\in U:f(z)=g(z)\}}'\cap U\ne\varnothing$. Then, $f(z)=g(z)$ $\forall z\in U$.
  \end{theorem}
  \subsubsection{General Cauchy theory}
  \begin{definition}
    Let $\gamma_1:[a,b]\rightarrow\CC$, $\gamma_2:[c,d]\rightarrow\CC$ be two paths and $n\in\ZZ$. We define the path $\gamma_1+\gamma_2$ as the concatenation of $\gamma_1$ and $\gamma_2$. That is:
    $$
      (\gamma_1+\gamma_2)(t):=
      \begin{cases}
        \gamma_1(2t)   & \text{if }\frac{a}{2}\leq t\leq \frac{b}{2}  \\
        \gamma_2(2t-b) & \text{if }\frac{c+b}{2} <t\leq \frac{d+b}{2}
      \end{cases}
    $$
    We define the path $n\gamma_1$ as the path:
    $$
      (n\gamma_1)(t):=
      \begin{cases}
        \gamma_1+\overset{(n)}{\cdots}+\gamma_1               & \text{if }n\geq 0 \\
        {\gamma_1}^-+\overset{(\abs{n})}{\cdots}+{\gamma_1}^- & \text{if }n<0
      \end{cases}
    $$
  \end{definition}
  \begin{definition}
    Let $\gamma_1,\ldots,\gamma_k$ be paths. A \emph{chain of paths} is a linear combination $$\gamma=n_1\gamma_1+\cdots+n_k\gamma_k$$ where $n_i\in\ZZ$ $\forall i=1,\ldots,k$. We define the \emph{image} of $\gamma$, $\gamma^*$, as:
    $$\gamma^*:=\bigcup_{i=1}^k{\gamma_i}^*$$
  \end{definition}
  \begin{definition}
    Let $\gamma=n_1\gamma_1+\cdots+n_k\gamma_k$ be a chain of paths. We say that $\gamma$ is a \emph{cycle} if the paths $\gamma_i$ are closed $\forall i=1,\ldots,k$.
  \end{definition}
  \begin{definition}
    Let $\gamma=n_1\gamma_1+\cdots+n_k\gamma_k$ be a cycle and $z_0\in\CC\setminus\gamma^*$. We define the \emph{index} of $\gamma$ with respect to $z_0$ as: $$\Ind(\gamma,a):=\sum_{i=1}^kn_i\Ind(\gamma_i,a)$$
  \end{definition}
  \begin{definition}
    Let $U\subseteq \CC$ be an open set and $\gamma$ be a cycle such that $\gamma^*\subset U$. We say that $\gamma$ is \emph{homologous to zero} on $U$, and we denoted it by $\gamma\homoleg{U} 0$, if $\Ind(\gamma,z)=0$ $\forall z\in\CC\setminus U$.
  \end{definition}
  \begin{lemma}
    Let $U\subseteq\CC$ be an open set and $f\in\mathcal{H}(U)$. Consider the function $\varphi:U\times U\rightarrow\CC$ defined as:
    $$
      \varphi(z,w)=
      \begin{cases}
        \frac{f(z)-f(w)}{z-w} & \text{if }z\ne w \\
        f'(z)                 & \text{if }z=w
      \end{cases}
    $$
    Then, $\varphi$ is continuous and $\forall w_0\in U$ the function $\varphi(\cdot,w_0)$ is holomorphic on $U$.
  \end{lemma}
  \begin{lemma}
    Let $U,V\subseteq\CC$ be open sets, $g:U\times V\rightarrow\CC$ be a continuous function and $\gamma$ be a piecewise path of class $\mathcal{C}^1$ with $\gamma^*\subset V$. Suppose that given $w_0\in V$ we have $g(\cdot,w_0)\in\mathcal{H}(U)$. Then, $F(z):=\int_\gamma g(z,w)\dd{w}$ is holomorphic on $U$ and $$F'(z)=\int_\gamma\pdv{g}{z}(z,w)\dd{w}$$
  \end{lemma}
  \begin{theorem}[General Cauchy's integral formula]
    Let $U\subseteq\CC$ be an open set, $f\in\mathcal{H}(U)$, $\gamma$ be a piecewise cycle of class $\mathcal{C}^1$ with $\gamma^*\subset U$ and $z_0\in U\setminus\gamma^*$. Suppose that $\gamma\homoleg{U} 0$. Then:
    $$f(z_0)\cdot\Ind(\gamma,z_0)=\frac{1}{2\pi\ii}\int_\gamma\frac{f(z)}{z-z_0}\dd{z}$$
    If $\gamma=n_1\gamma_1+\cdots+n_k\gamma_k$, the previous formula is equivalent to:
    $$f(z_0)\sum_{i=1}^kn_i\Ind(\gamma_i,z_0)=\frac{1}{2\pi\ii}\sum_{i=1}^kn_i\int_{\gamma_i}\frac{f(z)}{z-z_0}\dd{z}$$
  \end{theorem}
  \begin{theorem}[Cauchy's integral theorem]
    Let $U\subseteq \CC$ be an open set, $f\in\mathcal{H}(U)$ and $\gamma:[a,b]\rightarrow\CC$ be a piecewise cycle of class $\mathcal{C}^1$ with $\gamma^*\subset U$. Suppose that $\gamma\homoleg{U} 0$. Then: $$\int_\gamma f(z)\dd{z}=0$$
  \end{theorem}
  \begin{corollary}[General Cauchy's integral formula for derivatives]
    Let $U\subseteq\CC$ be an open set, $f\in\mathcal{H}(U)$, $\gamma$ be a piecewise cycle of class $\mathcal{C}^1$ with $\gamma^*\subset U$ and $z_0\in U\setminus\gamma^*$. Suppose that $\gamma\homoleg{U} 0$. Then:
    $$f^{(n)}(z_0)\cdot\Ind(\gamma,z_0)=\frac{n!}{2\pi\ii}\int_\gamma\frac{f(z)}{{(z-z_0)}^{n+1}}\dd{z}$$
  \end{corollary}
  \begin{theorem}
    Let $U\subseteq\CC$ be an open connected set and $f\in\mathcal{H}(U)$. Let $z_1,\ldots,z_n$ be the zeros of $f$ (counting repetitions) and $\gamma\homoleg{U} 0$ be a piecewise cycle of class $\mathcal{C}^1$ such that $\gamma^*\cap\{z_1,\ldots,z_n\}=\varnothing$. Then:
    $$\frac{1}{2\pi\ii}\int_{\gamma}\frac{f'(z)}{f(z)}\dd z=\sum_{j=1}^n\Ind(\gamma,z_j)=:\sum\Ind(\gamma,f^{-1}(0))$$
    where the last expression is a new notation that we will sometimes use in order to simplify the lecture.
  \end{theorem}
  \begin{corollary}
    Let $U\subseteq\CC$ be an open connected set, $f\in\mathcal{H}(U)$, $w\in \CC$ and $\gamma\homoleg{U} 0$ be a piecewise cycle of class $\mathcal{C}^1$ such that $\gamma^*\cap f^{-1}(w)=\varnothing$. Then:
    $$\frac{1}{2\pi\ii}\int_{\gamma}\frac{f'(z)}{f(z)-w}\dd z=\sum\Ind(\gamma,f^{-1}(w))$$
  \end{corollary}
  \begin{proposition}
    Let $a\in\CC$, $r\in\RR_{>0}$ and $f\in\mathcal{H}(D(a,r))$ be a non constant function. Suppose $f(a)=b\in \CC$ and so we can write $f(z)-b={(z-a)}^mg(z)$ with $m\in\NN$ and $g\in\mathcal{H}(D(a,r))$. Then, there exists $\varepsilon>0$ and $\delta>0$ such that $D(a,\varepsilon)\subset D(a,r)$ and all the points of $D(b,\delta)\setminus\{b\}$ have $m$ preimages in $D(a,\varepsilon)$ of multiplicity 1.
  \end{proposition}
\end{multicols}
\end{document}