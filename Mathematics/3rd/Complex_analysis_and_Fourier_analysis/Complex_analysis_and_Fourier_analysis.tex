\documentclass[../../../main.tex]{subfiles}

\begin{document}
\begin{multicols}{2}[\section{Complex analysis and Fourier analysis}]
  \subsection{Complex numbers}
  \subsubsection{Definition of complex numbers}
  \begin{definition}
    Consider $x^2+1\in\RR[x]$ and the ring $R:=\quot{\RR[x]}{(x^2+1)}$. Then, $R$ is a commutative field, which we will denote by $\CC$, whose elements are of the form $a+b\overline{x}=:a+b\ii$, $a,b\in\RR$\footnote{Such expression of a complex number is called \emph{Cartesian form} of a complex number.}. This field is called \emph{field of complex numbers}.
  \end{definition}
  \begin{proposition}
    Let $a_1+b_1\ii,a_2+b_2\ii\in\CC$, $a_1,a_2,b_1,b_2\in\RR$\footnote{From now on, we will omit to say that these values are real numbers. If they aren't, we will explicitly remark it.}. Then:
    \begin{itemize}
      \item $(a_1+b_1\ii)+(a_2+b_2\ii)=(a_1+a_2)+(b_1+b_2)\ii$
      \item $(a_1+b_1\ii)\cdot(a_2+b_2\ii)=(a_1a_2-b_1b_2)+(a_1b_2+a_2b_1)\ii$
      \item $\displaystyle\frac{a_1+b_1\ii}{a_2+b_2\ii}=\frac{a_1a_2+b_1b_2}{{a_2}^2+{b_2}^2}+\frac{a_2b_1-a_1b_2}{{a_2}^2+{b_2}^2}\ii$, provided that $a_2,b_2\ne 0$.
    \end{itemize}
  \end{proposition}
  \begin{theorem}
    $\CC$ is not an ordered field.
  \end{theorem}
  \subsubsection{Complex conjugate, modulus and argument}
  \begin{definition}
    Let $z=a+b\ii\in\CC$. We define the \emph{complex conjugate} (or simply \emph{conjugate}) of $z$ as $\overline{z}:=a-b\ii$.
  \end{definition}
  \begin{proposition}
    Let $z,w\in\CC$. Then:
    \begin{enumerate}
      \item $\overline{\overline{z}}=z$
      \item $\overline{z+w}=\overline{z}+\overline{w}$
      \item $\overline{z\cdot w}=\overline{z}\cdot\overline{w}$
      \item $\displaystyle\overline{\left(\frac{z}{w}\right)}=\frac{\overline{z}}{\overline{w}}$, provided that $w\ne 0$.
      \item $z\in\RR\iff \overline{z}=z$
    \end{enumerate}
  \end{proposition}
  \begin{definition}
    Let $z=a+b\ii\in\CC$. We define the \emph{real part} of $z$ as $\Re z:=a$. We define the \emph{imaginary part} of $z$ as $\Im z:=b$.
  \end{definition}
  \begin{proposition}
    Let $z\in\CC$. Then: $$\Re z=\frac{z+\overline{z}}{2}\quad\text{and}\quad\Im z=\frac{z-\overline{z}}{2\ii}$$
  \end{proposition}
  \begin{definition}
    Let $z=a+b\ii\in\CC$. We define the \emph{modulus} of $z$ as: $$\abs{z}:=\sqrt{a^2+b^2}$$
  \end{definition}
  \begin{proposition}
    Let $z,w\in\CC$. Then:
    \begin{enumerate}
      \item $\abs{z}\geq 0$
      \item $\abs{z}=0\iff z=0$
      \item $z\overline{z}={\abs{z}}^2$
      \item $z^{-1}=\frac{1}{{\abs{z}}^2}\cdot\overline{z}$
      \item $\abs{\Re z},\abs{\Im z}\leq \abs{z}\leq\abs{\Re z}+\abs{\Im z}$
      \item $\abs{z\cdot w}=\abs{z}\cdot \abs{w}$
      \item $\abs{z^n}={\abs{z}}^n$ $\forall n\in\ZZ$
      \item $\displaystyle\abs{\frac{z}{w}}=\frac{\abs{z}}{\abs{w}}$, provided that $w\ne 0$.
      \item ${\abs{z\pm w}}^2={\abs{z}}^2\abs{z}^2+{\abs{w}}^2\pm 2\Re(z\overline{w})$
      \item $\abs{z+w}\leq \abs{z}+ \abs{w}$ and the equality holds if and only if $z=\lambda w$, $\lambda\in\RR_{\geq 0}$.
      \item $\abs{\abs{z}-\abs{w}}\leq \abs{z-w}$ and the equality holds if and only if $z=\lambda w$, $\lambda\in\RR_{\geq 0}$.
    \end{enumerate}
  \end{proposition}
  \begin{corollary}
    Let $n\in\NN$ and $z_1, \ldots, z_n \in \CC$. Then:
    \begin{itemize}
      \item $\displaystyle \abs{\sum_{i=1}^n{z_i}} \leq \sum_{i=1}^n{\abs{z_i}}$
      \item $\displaystyle \abs{z_1 \cdots z_n} = \abs{z_1} \cdots \abs{z_n}$
      \item $\displaystyle \abs{\Re(z_1 \cdots z_n)},\abs{\Im(z_1 \cdots z_n)} \leq \abs{z_1} \cdots \abs{z_n}$
    \end{itemize}
  \end{corollary}
  \begin{definition}
    Let $z \in \CC^*$. We define the \textit{argument} of $z$, denoted by $\arg{z}$, as the real number $\theta$ satisfying: $$z = \abs{z}(\cos{\theta} + \ii\sin{\theta})$$ Note that $\arg{z}$ is not unique. We define the \textit{principal argument} of $z$ as the unique real number $\theta$ satisfying:
    $$\Arg z := \{\theta \in (-\pi, \pi]:z = \abs{z}(\cos{\theta} + \ii\sin{\theta})\}$$
  \end{definition}
  \begin{proposition}
    Let $z=a+b\ii\in\CC$. Then:
    $$\Arg z=\left\{
      \begin{array}{lcl}
        \arctan\left(\frac{y}{x}\right)     & \text{if} & x>0         \\
        \arctan\left(\frac{y}{x}\right)+\pi & \text{if} & x<0,y\geq 0 \\
        \arctan\left(\frac{y}{x}\right)-\pi & \text{if} & x<0,y<0     \\
        \sign{(y)}\frac{\pi}{2}             & \text{if} & x=0
      \end{array}
      \right.$$
  \end{proposition}
  \begin{proposition}
    Let $z, w \in \CC$ and $n\in\ZZ$. Then:
    \begin{enumerate}
      \item $\arg(zw) = \arg z + \arg w$
      \item $\arg(z^n) = n \arg{z}$
    \end{enumerate}
  \end{proposition}
  \begin{definition}[Polar form]
    Let $z\in\CC$, $r=\abs{z}$ and $\theta=\arg z$. We define the \emph{polar form} of $z$ as: $$z = r(\cos{\theta} + \ii\sin{\theta})$$
  \end{definition}
  \subsubsection{Metric topology of \texorpdfstring{$\CC$}{C}}
  \begin{proposition}
    Consider the distance $d$ defined as: $$\function{d}{\CC\times\CC}{\RR}{(z,w)}{\abs{z-w}}$$ Then, $(\CC,d)$ is a metric space\footnote{In order to simplify the notation we will refer to $(\CC,d)$ simply as $\CC$.}.
  \end{proposition}
  \begin{proposition}
    Thinking complex numbers as an ordered pair of real numbers, the topology of $\CC$ induced by $d$ is the same as the ordinary topology of $\RR^2$.
  \end{proposition}
  \subsection{Sequences and series}
  \subsubsection{Sequences}
  \begin{definition}
    A \textit{sequence of complex numbers} is a function of the form $$\function{}{\NN}{\CC}{n}{z_n}$$ In general, we will denote that sequence by $(z_n)$.
  \end{definition}
  \begin{definition}
    Let $(z_n)\in\CC$ be a sequence. A subsequence of $(z_n)$ is a sequence $(z_{k_n})$, where $(k_n)\in\NN$ is an increasing sequence of natural numbers.
  \end{definition}
  \begin{definition}
    Let $(z_n)\in\CC$ be a sequence. We say that $(z_n)$ has \emph{limit} $z\in\CC$ (or it \textit{converges} to $z$) if $\forall\varepsilon>0$, $\exists n_0\in\NN$ such that $$\abs{z_n - z} < \varepsilon\quad \forall n > n_0$$ In that case, we will write $\displaystyle\lim_{n \to \infty} z_n = z$ and the sequence is called \emph{convergent}. Otherwise, we say that the sequence is \emph{divergent}.
  \end{definition}
  \begin{definition}
    Let $(z_n)\in\CC$ be a sequence. We say that $(z_n)$ is \emph{bounded} if $\exists M\in\RR$ such that $\abs{z_n}\leq M$ $\forall n\in\NN$.
  \end{definition}
  \begin{definition}
    Let $(z_n)\in\CC$ be a sequence. We say that $(z_n)$ is \emph{Cauchy} if $\forall\varepsilon>0$, $\exists n_0\in\NN$ such that $$\abs{z_n - z_m} < \varepsilon\quad \forall n,m > n_0$$
  \end{definition}
  \begin{proposition}
    Let $(z_n)\in\CC$ be a convergent sequence. Then, $(z_n)$ is bounded and Cauchy.
  \end{proposition}
  \begin{proposition}
    Let $(z_n)\in\CC$ be a sequence. Then, $(z_n)$ is convergent if and only if all its subsequences are convergent.
  \end{proposition}
  \begin{proposition}
    Let $(z_n),(w_n)\in\CC$ be two convergent sequences whose limits are $z,w\in \CC$, respectively. Then:
    \begin{enumerate}
      \item $\displaystyle\lim_{n\to\infty}z_n+w_n=z+w$
      \item $\displaystyle\lim_{n\to\infty}z_nw_n=zw$
      \item $\displaystyle\lim_{n\to\infty}\frac{z_n}{w_n}=\frac{z}{w}$, provided that $w_n\ne 0$ $\forall n\in\NN$.
    \end{enumerate}
  \end{proposition}
  \begin{definition}
    Let $(z_n)\in\CC$ be a sequence such that $z_n=x_n+y_n\ii$ $\forall n\in\NN$, where $x_n,y_n\in\RR$. Then:
    \begin{enumerate}
      \item $(z_n)$ is convergent if and only if $(x_n)$ and $(y_n)$ are convergent. In that case, we have: $$\lim_{n\to\infty}z_n=\lim_{n\to\infty}x_n+\ii\lim_{n\to\infty}y_n$$
      \item $(z_n)$ is Cauchy if and only if $(x_n)$ and $(y_n)$ are Cauchy.
    \end{enumerate}
  \end{definition}
  \begin{theorem}
    $\CC$ is a complete metric space.
  \end{theorem}
  \subsubsection{Series}
  \begin{definition}
    Let $(z_n)\in\CC$ be a sequence. A \emph{numeric series of complex numbers} is an expression of the form $$\sum_{n=1}^\infty z_n$$ We call $z_n$ \emph{general term of the series} and $\displaystyle S_N:=\sum_{n=1}^N z_n$, for all $N\in\NN $, \emph{$N$-th partial sum of the series}\footnote{From now on we will write $\sum z_n$ to refer $\displaystyle\sum_{n=1}^\infty z_n$.}.
  \end{definition}
  \begin{definition}
    We say the series of complex numbers $\sum z_n$ is \emph{convergent} if $\displaystyle S=\lim_{N\to\infty}S_N$ exists and it is finite. In that case, $S$ is called the \emph{sum of the series}. If the previous limit doesn't exists or it is infinite, we say the series is \emph{divergent}\footnote{We will use the notation $\sum a_n<\infty$ or $\sum a_n=+\infty$ to express that the series converges or diverges, respectively.}.
  \end{definition}
  \begin{definition}
    Let $\sum z_n$ be a series of complex numbers.  A \emph{reordering} of $\sum z_n$ is any series $\sum z_{\sigma(n)}$, where $\sigma:\NN\rightarrow\NN$ be a bijective function.
  \end{definition}
  \begin{proposition}
    Let $(z_n)\in\CC$ be a sequence and $\sum z_n$ be a convergent series. Then, $\displaystyle\lim_{n\to\infty}z_n =0$.
  \end{proposition}
  \begin{proposition}
    Let $(z_n)\in\CC$ be a sequence such that such that $z_n=x_n+y_n\ii$ $\forall n\in\NN$, where $x_n,y_n\in\RR$, and $\sum z_n$ be a series. Then, $\sum z_n<\infty$ if and only if $\sum x_n$ and $\sum y_n$. In that case, we have: $$\sum_{n=1}^\infty z_n=\sum_{n=1}^\infty x_n+\ii\sum_{n=1}^\infty y_n$$
  \end{proposition}
  \begin{proposition}
    Let $\sum z_n=z\in\CC$ and $\sum w_n=w\in\CC$ be two series, and $\lambda\in\CC$. Then:
    \begin{enumerate}
      \item $\sum (z_n+w_n)$ is convergent and $\sum (z_n+w_n)=z+w$.
      \item $\sum (\lambda z_n)$ is convergent and $\sum (\lambda z_n)=\lambda z$.
    \end{enumerate}
  \end{proposition}
  \begin{lemma}[Abel's summation formula]
    Let $(z_n),(w_n)\in\CC$ be two sequence. Let $S_N:=\sum_{n=1}^N z_n$. Then:
    \begin{equation*}
      \sum_{n=1}^N z_nw_n=S_Nw_N+\sum_{n=1}^{N-1}S_n(w_n-w_{n+1})
    \end{equation*}
  \end{lemma}
  \begin{definition}
    Let $\sum z_n$ be a series of complex numbers. We say that $\sum z_n$ is \emph{absolutely convergent} if $\sum \abs{z_n}<\infty$\footnote{Note that since $\sum \abs{z_n}$ is a sequence of real numbers, all the criteria for convergence of numeric series of real numbers are, thus, aplicable.}.
  \end{definition}
  \begin{proposition}
    Let $\sum z_n$ be a series of complex numbers.
    \begin{enumerate}
      \item $\sum \abs{z_n}<\infty\implies\sum z_n<\infty$
      \item $\sum \abs{z_n}=z<\infty\implies\forall\sigma\in S(\NN),\ \sum z_{\sigma (n)}=w<\infty$ for some $w\in\CC$.
    \end{enumerate}
  \end{proposition}
  \begin{definition}[Cauchy product]
    Let $\sum z_n$, $\sum w_n$ be absolutely convergent series of complex numbers. We define the \emph{product} of $\sum z_n$ and $\sum w_n$ as the series $\sum p_n$, where $p_n=\sum_{k=0}^nz_kw_{n-k}$.
  \end{definition}
  \begin{proposition}
    Let $\sum z_n$, $\sum w_n$ be absolutely convergent series of complex numbers. Then, the product of these series is absolutely convergent and satisfy: $$\sum_{n=1}^\infty p_n=\left(\sum_{n=1}^\infty z_n\right)\left(\sum_{n=1}^\infty w_n\right)$$
  \end{proposition}
  \subsection{Complex functions}
  \subsubsection{Continuity}
  \begin{definition}
    Let $D \subseteq \CC$ be a set. We define a \textit{complex function}\footnote{Due to the close similarity between these kind of functions and the multivariate functions, we will only expose the most remarkable results about continuity of complex functions.} as a function of the form:
    $$\function{f}{D}{\CC}{z}{f(z)}$$
  \end{definition}
  \begin{definition}
    Let $D \subseteq \CC$ be a set and $f:D\rightarrow\CC$ be a function. We say that $f$ is \emph{continuos} at $z_0\in D$ if and only if $\forall \varphi>0$ $\exists\delta>0$ such that $\abs{f(z)-f(z_0)}<\delta$ whenever $\abs{z-z_0}<\delta$. We say that $f$ is continuos on $D$ is it is continuos at $z$ $\forall x\in D$.
  \end{definition}
  \begin{definition}
    Let $D \subseteq \CC$ be a set and $f:D\rightarrow\CC$ be a function. We define the function $\Re f$ as the function: $$\function{\Re f}{D}{\CC}{z}{\Re(f(z))}$$
    Analogously, we define the function $\Im f$ as the function: $$\function{\Im f}{D}{\CC}{z}{\Im(f(z))}$$
  \end{definition}
  \begin{proposition}
    Let $D \subseteq \CC$ be a set $f:D\rightarrow\CC$ be a function and $z_0 \in D$. Then, $f = \Re f + \ii \Im f$ is continuous at $z_0$ if and only if $\Re f$ and $\Im f$ are continuous at $z_0$.
  \end{proposition}
  \begin{proposition}
    Let $D \subseteq \CC$ be a set $f:D\rightarrow\CC$ be a function and $z_0 \in D$. Then, $f$ is continuous at $z_0$ if and only if for all sequence $(w_n)\in D$ convergent to $z_0$, the sequence $(f(w_n))$ converges to $f(z_0)$.
  \end{proposition}
  \begin{proposition}
    Let $D \subseteq \CC$ be a set and $f,g:D \subseteq \CC \longrightarrow \CC$ be two continuous function at a point $z_0 \in D$ and $\lambda \in \CC$. Then, $\lambda f$, $f+g$, and $fg$ are continuous at $z_0$. Moreover, if $g(z_0) \neq 0$ then, $f/g$ is continuous at $z_0$.
  \end{proposition}
  \subsubsection{Sequences of functions}
  \begin{definition}
    Let $D\subseteq\CC$. A set $$(f_n(z))=\{f_1(z),f_2(z),\ldots,f_n(z),\ldots\}$$ is a \emph{sequence of complex functions} if $f_i:D\rightarrow\CC $ is a complex function. In this case we say the sequence $(f_n(z))$, or simply $(f_n)$, is well-defined on $D$\footnote{The majority of definitions and results of sequences of real-valued functions can be extended conveniently to sequences of complex functions. So in this document, we will only expose the most important ones.}.
  \end{definition}
  \begin{definition}
    Let $(f_n)\in D\subseteq\CC$ be a sequence of functions and $f:D\rightarrow\CC$. We say $(f_n)$ \emph{converges pointwise} to $f$ on $D$ if $\forall z\in D$, $\displaystyle\lim_{n\to\infty}f_n(z)=f(z)$
  \end{definition}
  \begin{definition}
    Let $(f_n)\in D\subseteq\CC$ be a sequence of functions and $f:D\rightarrow\CC$. We say $(f_n)$ \emph{converges uniformly} to $f$ on $D$ if $\forall\varepsilon>0$, $\exists n_0:\abs{f_n(z)-f(z)}<\varepsilon$ $\forall n\geq n_0$ and $\forall z\in D$.
  \end{definition}
  \begin{lemma}
    Let $(f_n)\in D\subseteq\CC$ be a sequence of functions. $(f_n)$ converges uniformly to $f:D\rightarrow\CC$ on $D$ if and only if $\displaystyle \lim_{n\to\infty}\sup\left\{\abs{f_n(z)-f(z)}:z\in D\right\}=0$.
  \end{lemma}
  \begin{theorem}[Cauchy's test]
    A sequence of functions $(f_n)\in D\subseteq\CC$ converges uniformly to $f:D\rightarrow\CC$ on $D\subseteq\CC$ if and only if $\forall\varepsilon>0$ $\exists n_0\in\NN$ such that  $\forall n,m\geq n_0$, we have: $$\sup\left\{\abs{f_n(z)-f_m(z)}:z\in D\right\}< \varepsilon$$
  \end{theorem}
  \begin{theorem}
    Let $(f_n)\in D\subseteq\CC$ be a sequence of continuous functions. If $(f_n)$ converges uniformly to $f:D\rightarrow\CC$ on $D$, then $f$ is continuous on $D$.
  \end{theorem}
  \subsubsection{Series of functions}
  \begin{definition}
    Let $(f_n)\in D\subseteq\CC$ be a sequence of functions. The expression $$\sum_{n=1}^\infty f_n(z)$$ is the \emph{series of functions} associated with $(f_n)$\footnote{The majority of definitions and results of series of real-valued functions can be extended conveniently to series of complex functions. So in this document, we will only expose the most important ones.}.
  \end{definition}
  \begin{definition}
    A series of functions $\sum f_n(z)$ defined on $D\subseteq\CC$ \emph{converges pointwise} on $D$ if the sequence of partials sums $$F_N(z)=\sum_{n=1}^Nf_n(z)$$ converges pointwise. If the pointwise limit of $(F_N)$ is $F(z)$, we say $F$ is the \emph{sum of the series in a pointwise sense}.
  \end{definition}
  \begin{definition}
    A series of functions $\sum f_n(z)$ defined on $D\subseteq\CC$ \emph{converges uniformly} on $D$ if the sequence of partials sums $$F_N(z)=\sum_{n=1}^Nf_n(z)$$ converges uniformly. If the uniform limit of $(F_N)$ is $F(z)$, we say $F$ is the \emph{sum of the series in an uniform sense}.
  \end{definition}
  \begin{theorem}[Cauchy's test]
    A series of functions $\sum f_n(z)$ defined on $D\subseteq\CC$ converges uniformly if and only if $\forall\varepsilon>0$ $\exists n_0$ such that $\forall  M, N\geq n_0$ (with $N\leq M$), we have: $$\sup\left\{\abs{\sum_{n=N}^Mf_n(z)}:z\in D\right\}< \varepsilon$$
  \end{theorem}
  \begin{corollary}
    Let $(f_n)\in D\subseteq\CC$ be a sequence of functions. If $\sum f_n(z)$ is uniformly convergent on $D\subseteq\CC$, then $(f_n)$ converges uniformly to zero on $D$.
  \end{corollary}
  \begin{theorem}
    Let $(f_n)\in D\subseteq\CC$ be a sequence of continuous functions. If $\sum f_n(z)$ is uniformly convergent on $D\subseteq\CC$, then its sum function is also continuous on $D$.
  \end{theorem}
  \begin{theorem}[Weierstra\ss\space M-test]
    Let $(f_n)\in D\subseteq\CC$ be a sequence of functions such that $\abs{f_n(z)}\leq M_n$ $\forall z\in D$ and suppose that $\sum M_n<\infty$. Then, $\sum f_n(z)$ converges uniformly on $D$.
  \end{theorem}
  \begin{theorem}[Dirichlet's test]
    Let $(f_n)\in X\subseteq \CC$ and $(g_n)\in Y\subseteq \CC$ be two sequences of functions and $F_N:=\sum_{n=1}^Nf_n(z)$. Suppose:
    \begin{enumerate}
      \item $(F_N)$ is uniformly bounded on $X$.
      \item $(g_n(z))$ is a monotone sequence of real numbers and converges uniformly to 0 on $Y$.
    \end{enumerate}
    Then, $\sum f_n(z)g_n(z)$ converges uniformly on $X\cap Y$.
  \end{theorem}
  \begin{theorem}[Abel's test]
    Let $(f_n)\in X\subseteq \CC$ and $(g_n)\in Y\subseteq \CC$ be two sequences of functions. Suppose:
    \begin{enumerate}
      \item $\sum f_n(z)$ is uniformly convergent on $X$.
      \item $(g_n)$ is a monotone and bounded sequence of real numbers.
    \end{enumerate}
    Then, $\sum f_n(z)g_n(z)$ converges uniformly on $X\cap Y$.
  \end{theorem}
  \begin{theorem}[Dedekind's test]
    Let $(f_n)\in X\subseteq \CC$ and $(g_n)\in Y\subseteq \CC$ be two sequences of functions and $F_N:=\sum_{n=1}^Nf_n(z)$. Suppose:
    \begin{enumerate}
      \item $(F_N)$ is uniformly bounded on $X$.
      \item $(g_n)$ converges uniformly to 0 on $Y$.
      \item $\sum\abs{g_n(z)-g_{n+1}(z)}$ converges uniformly on $Y$.
    \end{enumerate}
    Then, $\sum f_n(z)g_n(z)$ converges uniformly on $X\cap Y$.
  \end{theorem}
  \begin{theorem}[Du Bois-Reymond's test]
    Let $(f_n)\in X\subseteq \CC$ and $(g_n)\in Y\subseteq \CC$ be two sequences of functions. Suppose:
    \begin{enumerate}
      \item $\sum f_n(z)$ is uniformly convergent on $X$.
      \item $\sum\abs{g_n(z)-g_{n+1}(z)}<\infty$ $\forall z\in Y$.
    \end{enumerate}
    Then, $\sum f_n(z)g_n(z)$ converges uniformly on $X\cap Y$.
  \end{theorem}
  \subsubsection{Power series}
  \begin{definition}
    Let $(a_n)\in\CC$ be a sequence and $z_0\in\CC$. A \emph{complex power series} centred at $z_0$ is a series of functions of the form $$\sum_{n=0}^\infty a_n{(z-z_0)}^n$$
  \end{definition}
  \begin{proposition}
    Let $\sum a_n{(z-z_0)}^n$ be a complex power series. Suppose there exists an $x_1\in\RR $ such that $\sum a_n(x_1-x_0)^n<\infty$. Then, $\sum a_n{(z-z_0)}^n$ converges uniformly on any closed interval $I\subset A=\{x\in\RR :\abs{z-z_0}<\abs{x_1-x_0}\}$.
  \end{proposition}
  \begin{theorem}[Cauchy-Hadamard theorem]
    Let $\sum a_n{(z-z_0)}^n$ be a complex power series. Then, $\exists! R\in[0,\infty]$ defined as $$R=\left(\limsup_{n\to\infty}\sqrt[n]{\abs{a_n}}\right)^{-1}\in[0,\infty]$$
    that satisfies the following properties:
    \begin{enumerate}
      \item If $\abs{z-z_0}<R\implies\sum a_n{(z-z_0)}^n$ converges absolutely.
      \item If $0\leq r<R\implies\sum a_n{(z-z_0)}^n$ converges absolutely and uniformly on $\{z\in\CC:\abs{z-z_0}\leq r\}$.
      \item If $\abs{z-z_0}>R\implies\sum\abs{a_n}{\abs{z-z_0}}^n$ diverges.
    \end{enumerate}
    The number $R$ is called \emph{radius of convergence} of series.
  \end{theorem}
  \begin{theorem}[Abel's theorem]
    Let $\sum a_n{(z-z_0)}^n$ be a complex power series that converges uniformly in $D\subseteq \CC$. Then, the series $\sum a_n{(z-z_0)}^n$ converges uniformly on $\{r\zeta\in\CC:r\in[0,1]\land\zeta\in D\}$, which is a \emph{cone} with base $D$.
  \end{theorem}
  \begin{corollary}[Abel's theorem]
    Let $\sum a_n{(z-z_0)}^n$ be a complex power series with radius of convergence $R\in(0,\infty)$ and $I\subseteq [0,2\pi)$ be a non-empty connected set. Suppose that the series $\sum a_n{(z-z_0)}^n$ is uniformly convergent on $D:=\{R\exp{\ii \theta}\in\CC: \theta\in I\}$. Then, $f(z):=\sum a_n{(z-z_0)}^n$ converges uniformly on the cone $C$ with base $D$. In particular, we have: $$\lim_{\substack{z\to R\exp{\ii \theta}\\z\in C}}f(z)=\sum_{n=0}^\infty a_nR^n\exp{\ii n\theta}\qquad\forall\theta\in J$$
  \end{corollary}
  \begin{proposition}
    Let $f:\CC\rightarrow\CC$ be the sum function of a complex power series. Then $f$ is continuous on the domain of convergence of the series.
  \end{proposition}
  \subsubsection{Exponential and logarithmic functions}
  \begin{definition}
    For all $z \in \CC$, we define the \textit{complex exponential function} as: $$\exp{z}:=\sum_{n=0}^\infty\frac{z^n}{n!}$$
  \end{definition}
  \begin{proposition}
    The radius of convergence of $\exp{z}$ is infinite and its image is $\CC^*$.
  \end{proposition}
  \begin{proposition}
    Let $z,w\in\CC$. Then:
    \begin{enumerate}
      \item $\exp{z+w}=\exp{z}\exp{w}$
      \item $\overline{\exp{z}}=\exp{\overline{z}}$
      \item $\abs{\exp{z}}=\exp{\Re z}$
    \end{enumerate}
  \end{proposition}
  \begin{corollary}[Euler's formula]
    Let $x\in\RR$. Then: $$\exp{\ii x}=\cos x+\ii \sin x$$
  \end{corollary}
  \begin{corollary}
    Let $z,w\in\CC$. Then:
    \begin{enumerate}
      \item $\exp{z}=1\iff z=2\pi k\ii$, $k\in\ZZ$.
      \item $\exp{z}$ is periodic of period $2\pi i$.
      \item $\exp{z}=\exp{w}\iff z=w+2\pi \ii k$, $k\in\ZZ$.
    \end{enumerate}
  \end{corollary}
  \begin{corollary}
    Let $x\in\RR$. Then: $$\cos x=\sum_{n=0}^\infty\frac{(-1)^nx^{2n}}{(2n)!}\qquad\sin x=\sum_{n=0}^\infty\frac{(-1)^nx^{2n+1}}{(2n+1)!}$$
  \end{corollary}
  \begin{proposition}[De Moivre's formula]
    Let $\theta\in\RR$ and $n\in\ZZ$. Then: $${(\cos{\theta} + \ii\sin{\theta})}^n = \cos(n\theta) + \ii\sin(n\theta)$$
  \end{proposition}
  \begin{theorem}
    Let $n\in\NN$. Then, there are $n$ $n$-th roots of any complex number $z\in \CC^*$. Assuming $z=r(\cos\theta + \ii\sin\theta)$, these roots are: $$\sqrt[n]{r}\left[\cos\left(\frac{\theta}{n}+\frac{2\pi}{n}k\right)+\ii\sin\left(\frac{\theta}{n}+\frac{2\pi}{n}k\right)\right]$$ for $k=0,\ldots,n-1$.
  \end{theorem}
  \begin{proposition}
    Let $z\in\CC^*$. Then, the equation $\exp{w}=z$ has infinitely many solutions.
  \end{proposition}
  \begin{definition}
    Let $z\in \CC$. We define a \emph{complex natural logarithm} of $z$ as a solution to the equation $\exp{w}=z$. That is: $$\ln z:=\ln \abs{z}+\ii\arg z$$
    We define the \emph{principal value} of $\ln z$ as: $$\Ln z:=\ln \abs{z}+\ii\Arg z$$
  \end{definition}
  \begin{proposition}
    Let $z\in \CC$. Then, $\exp{\Ln z}=\Ln{\exp{z}}=z$.
  \end{proposition}
  \subsubsection{Trigonometric functions}
  \begin{definition}
    Let $z\in \CC$. We define the \emph{complex sine} and \emph{complex cosine} respectively as:
    \begin{align*}
      \cos z & =\frac{\exp{\ii z}+\exp{-\ii z}}{2}=\sum_{n=0}^\infty\frac{(-1)^nz^{2n}}{(2n)!}        \\
      \sin z & =\frac{\exp{\ii z}-\exp{-\ii z}}{2\ii}=\sum_{n=0}^\infty\frac{(-1)^nz^{2n+1}}{(2n+1)!}
    \end{align*}
  \end{definition}
  \begin{proposition}
    Let $z,w\in\CC$. Then:
    \begin{enumerate}
      \item ${\left(\cos z\right)}^2+{\left(\sin z\right)}^2=1$
      \item $\cos(-z)=\cos z$, $\sin(-z)=-\sin z$
      \item $\cos(z \pm w) = \cos z \cos w \mp \sin z \sin w$
      \item $\sin(z \pm w) = \sin z \cos w \pm \cos z \sin w$
    \end{enumerate}
  \end{proposition}
  \begin{proposition}
    The functions $\cos z$, $\sin z$ are unbounded and periodic of period $2\pi$.
  \end{proposition}
  \begin{proposition}
    Let $z\in \CC$. Then:
    \begin{itemize}
      \item $\cos z=0\implies z\in\RR$
      \item $\sin z=0\implies z\in\RR$
    \end{itemize}
  \end{proposition}
  \begin{definition}
    We define the \emph{complex tangent} and \emph{complex secant}, \emph{complex cosecant} and \emph{complex cotangent} respectively as:
    \begin{align*}
      \tan z & = \frac{\sin z}{\cos z}=-\ii\frac{\exp{\ii z}-\exp{-\ii z}}{\exp{\ii z}+\exp{-\ii z}}\quad \forall z\in\CC\setminus\bigcup_{k\in\ZZ}\left\{\frac{\pi}{2}+\pi k\right\} \\
      \sec z & = \frac{1}{\cos z}=\frac{2}{\exp{\ii z}+\exp{-\ii z}}\quad \forall z\in\CC\setminus\bigcup_{k\in\ZZ}\left\{\frac{\pi}{2}+\pi k\right\}                                 \\
      \csc z & = \frac{1}{\sin z}=\frac{2\ii}{\exp{\ii z}-\exp{-\ii z}}\quad\forall z\in\CC\setminus\bigcup_{k\in\ZZ}\{\pi k\}                                                        \\
      \cot z & = \frac{1}{\tan z}=\ii\frac{\exp{\ii z}+\exp{-\ii z}}{\exp{\ii z}-\exp{-\ii z}}\quad\forall z\in\CC\setminus\bigcup_{k\in\ZZ}\{\pi k\}
    \end{align*}
  \end{definition}
  \begin{definition}
    Let $z\in \CC$. We define the \emph{complex hyperbolic sine} and \emph{complex hyperbolic cosine} respectively as:
    \begin{align*}
      \cosh z & =\frac{\exp{z}+\exp{-z}}{2}=\sum_{n=0}^\infty\frac{z^{2n}}{(2n)!}     \\
      \sinh z & =\frac{\exp{z}-\exp{-z}}{2}=\sum_{n=0}^\infty\frac{z^{2n+1}}{(2n+1)!}
    \end{align*}
  \end{definition}
  \begin{proposition}
    Let $z,w\in\CC$. Then:
    \begin{enumerate}
      \item ${\left(\cosh z\right)}^2-{\left(\sinh z\right)}^2=1$
      \item $\cosh(-z)=\cosh z$, $\sinh(-z)=-\sinh z$
      \item $\cosh{(z \pm w)} = \cosh z \cosh w \pm \sinh z \sin w$
      \item $\sinh{(z \pm w)} = \sinh z \cosh w \pm \cosh z \sin w$
    \end{enumerate}
  \end{proposition}
  \begin{proposition}
    Let $z=x+\ii y\in \CC$. Then:
    \begin{enumerate}
      \item $\cos z = \cos x \cosh y - \ii \sin x \sinh y$
      \item $\sin z = \sin x \cosh y + \ii \cos x \sinh y$
    \end{enumerate}
  \end{proposition}
  \begin{definition}
    We define the \emph{complex hyperbolic tangent} and \emph{complex hyperbolic secant}, \emph{complex hyperbolic cosecant} and \emph{complex hyperbolic cotangent} respectively as:
    \begin{align*}
      \tanh z & = \frac{\sinh z}{\cosh z}=\frac{\exp{z}-\exp{-z}}{\exp{z}+\exp{-z}}\quad \forall z\in\CC\setminus\bigcup_{k\in\ZZ}\left\{\ii\left(\frac{\pi}{2}+\pi k\right)\right\} \\
      \sech z & = \frac{1}{\cosh z}=\frac{2}{\exp{z}+\exp{-z}}\quad \forall z\in\CC\setminus\bigcup_{k\in\ZZ}\left\{\ii\left(\frac{\pi}{2}+\pi k\right)\right\}                      \\
      \csch z & = \frac{1}{\sinh z}=\frac{2}{\exp{z}-\exp{-z}}\quad\forall z\in\CC\setminus\bigcup_{k\in\ZZ}\{\pi k\ii\}                                                             \\
      \coth z & = \frac{1}{\tanh z}=\frac{\exp{z}+\exp{-z}}{\exp{z}-\exp{-z}}\quad\forall z\in\CC\setminus\bigcup_{k\in\ZZ}\{\pi k\ii\}
    \end{align*}
  \end{definition}
  \subsection{Holomorphic functions}
  \begin{definition}
    Let $U\subseteq \CC$ be an open set, $f:U\rightarrow \CC$ be a function and $z_0\in U$. We say that $f$ is \emph{$\CC$-differentiable} at $z_0$ if the following limit exists: $$\lim_{z \to z_0} \frac{f(z) - f(z_0)}{z - z_0} = \lim_{h \to 0} \frac{f(z_0 + h) - f(z_0)}{h}$$ In that case, the limit is called \textit{derivative of $f$ at $z_0$} and it is denoted by $f'(z_0)$.
  \end{definition}
  \begin{proposition}
    Let $U\subseteq \CC$ be an open set, $f:U\rightarrow \CC$ be a function and $z_0\in U$. Then, $f$ is $\CC$-differentiable at $z_0$ if and only if $\exists a+b\ii\in\CC$ such that $\forall\varepsilon>0$ $\exists\delta>0$ such that $$|f(z_0+s)-f(z_0)-(a+b\ii)s|\leq \varepsilon s$$ whenever $|s|<\delta$.
  \end{proposition}
  \begin{proposition}
    Let $U\subseteq \CC$ be an open set, $z_0 \in U$, $f,g:U\rightarrow \CC$ be two $\CC$-differentiable functions at $z_0$ and $\alpha,\beta\in \CC$. Then:
    \begin{enumerate}
      \item $\alpha f + \beta g$ is $\CC$-differentiable at $z_0$ and: $${(\alpha f+ \beta g)}'(z_0) = \alpha f'(z_0) + \beta g'(z_0)$$
      \item $fg$ is $\CC$-differentiable at $z_0$ and: $${(fg)}'(z_0) = f'(z_0)g(z_0) + f(z_0)g'(z_0)$$
      \item If $g(z_0) \neq 0$, then $f/g$ is $\CC$-differentiable at $z_0$ and:
            $${\left(\frac{f}{g}\right)}'(z_0) = \frac{f'(z_0)g(z_0) - f(z_0) g'(z_0)}{{g(z_0)}^2}$$
    \end{enumerate}
  \end{proposition}
  \begin{proposition}
    Let $z\in\CC$. Then:
    \begin{itemize}
      \item ${\left(\exp{z}\right)}'=\exp{z}$
      \item ${\left(\cosh{z}\right)}'=\sinh{z}$
      \item ${\left(\sinh{z}\right)}'=\cosh{z}$
      \item ${\left(\tanh{z}\right)}'=1+{(\tanh{z})}^2$
    \end{itemize}
  \end{proposition}
  \begin{definition}
    Let $U\subseteq \CC$ be an open set and $f:U\rightarrow \CC$ be a function. We say that $f$ is \emph{holomorphic} on $U$ is $f$ is $\CC$-differentiable at each $z\in U$. We denote the set of all holomorphic functions on $U$ by $\mathcal{H}(U)$.
  \end{definition}
  \begin{definition}
    We say that $f:\CC\rightarrow\CC$ is an \emph{entire function} if $f$ is holomorphic on the whole complex plane.
  \end{definition}
  \begin{proposition}
    Let $U\subseteq \CC$ be an open set, $f:U\rightarrow \CC$ be a function and $z_0 \in \CC$ a point. If $f$ is $\CC$-differentiable at $z_0$, then it is continuous at $z_0$.
  \end{proposition}
  \begin{proposition}
    Let $n\in\NN\cup\{0\}$ and $f:\CC\rightarrow\CC$ be a function defined as  $f(z)=z^n$. Then, $f$ is holomorphic and $f'(z)=nz^{n-1}$ $\forall z\in\CC$.
  \end{proposition}
  \begin{corollary}
    Let $p(z)=\frac{f(z)}{g(z)}\in\CC(z)$ be a rational function. Then, $p(z)$ is a holomorphic on the open set $\CC\setminus Z(g)$, where $Z(g)=\{z\in\CC:g(z)=0\}$.
    In particular, if $p(z)\in\CC[z]$ is a polynomial, $p(z)$ is holomorphic on $\CC$.
  \end{corollary}
  \begin{theorem}[Chain rule]
    Let $U,V\subseteq \CC$ be open sets, $z_0\in U$, $f:U\rightarrow\CC$ be a $\CC$-differentiable function at $z_0$ such that $f(U)\subseteq V$, and $g:V\rightarrow\CC$ be $\CC$-differentiable function at $f(z_0)$. Then, $g\circ f$ is $\CC$-differentiable at $z_0$ and: $${(g\circ f)}'(z_0)=g'(f(z_0))f'(z_0)$$
  \end{theorem}
  \begin{theorem}
    Let $\sum_{n=0}^\infty a_n{(z-z_0)}^n$ be a complex power series with radius of convergence $R\in[0,\infty]$. Then, the series $\sum_{n=1}^\infty na_n{(z-z_0)}^{n-1}$ has the same radius of convergence $R$ and if $f(z)=\sum_{n=0}^\infty a_n{(z-z_0)}^n$ for $|z-z_0|<R$, then $f$ is holomorphic and: $$f'(z)=\sum_{n=1}^\infty na_n{(z-z_0)}^{n-1}\quad\text{for } |z-z_0|<R$$
  \end{theorem}
  \begin{corollary}
    Let $\sum_{n=0}^\infty a_n{(z-z_0)}^n$ be a complex power series with radius of convergence $R\in[0,\infty]$. Then, the series $\sum_{n=k}^\infty n(n-1)\cdots(n-k+1)a_n{(z-z_0)}^{n-k}$ has the same radius of convergence $R$ for all $k\in\NN \cup\{0\}$ and if $f(z)=\sum_{n=0}^\infty a_n{(z-z_0)}^n$ for $|z-z_0|<R$, then $f^(k)$ is holomorphic and: $$f^{(k)}(z)=\sum_{n=k}^\infty n(n-1)\cdots(n-k+1)a_n{(z-z_0)}^{n-k}$$ for $|z-z_0|<R$ and $\forall k\in\NN \cup\{0\}$. In particular $f^{(k)}(0)=k!a_k$ $\forall k\in\NN\cup\{0\}$.
  \end{corollary}
  \begin{proposition}
    Let $U\subseteq\CC$ be a connected open set and $f\in\mathcal{H}(U)$ such that $f'(z)=0$ $\forall z\in U$. Then, $f$ is constant.
  \end{proposition}
  \subsubsection{Logarithm determination}
  \begin{definition}
    Let $U\subseteq\CC^*$ be an open set and $f:U\longrightarrow\CC^*$ and $g:U\rightarrow\CC$ be functions. We say that $g$ is a \emph{determination} of $\ln f(z)$ if $g$ is continuos on $U$ and $\exp{g(z)}=f(z)$ $\forall z\in U$.
  \end{definition}
  \begin{theorem}
    Let $U,V\subseteq\CC^*$ be open sets, $f:U\longrightarrow\CC^*$ be a function and $g:V\rightarrow\CC$ be and holomorphic function. Suppose $f(U)\subseteq V$, $g(f(z))=z$ and $g'(f(z))\ne 0$ $\forall z\in U$. Then, $f$ is holomorphic on $U$ and: $$f'(z)=\frac{1}{g'(f(z))}$$
  \end{theorem}
  \begin{corollary}
    Let $U,V\subseteq\CC^*$ be open sets, $g:V\rightarrow\CC$ be and holomorphic function and $f:U\longrightarrow\CC^*$ be a determination of the logarithm. Then: $$f'(z)=\frac{1}{z}$$
    In particular, ${\left(\ln z\right)}'=\frac{1}{z}$.
  \end{corollary}
\end{multicols}
\end{document}