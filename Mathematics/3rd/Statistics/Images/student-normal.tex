\documentclass{standalone}
\usepackage{../../../../preamble_formulas}

\def\n{} % level

\begin{document}
\begin{tikzpicture}
  \begin{axis}[
      xmin = - 5, xmax = 5, %
      ymin = 0, ymax = 0.4, %
      width = 10cm, %sets the width of the figure
      height = 7.5cm,  %sets the height of the figure
      legend cell align = {left}, %
      %legend style={} % position of the legend box and anchor is the point on the box to be fitted exactly at the point of cs:<>,<>. Options are anchor=center,south west,south east,north west,north east,north,south,west... Example: legend style={at={(axis cs:0,-0.5)},anchor=center}
    ]
    % t Student (degree 1)
    \addplot[
      domain=-5:5, %Domain of the fucntion
      samples=200, %This parameter determines the number of point to be plotted for the function, while bigger the number better looks the function.
      smooth,
      thick, %Stroke of the function. Options: ultra thin, very thin, thin, semithick, thick, very thick, ultra thick.
      color_blue1, %Color of the function.
    ]{(1+x^2)^(-1)/pi};
    \addlegendentry{$t_1$}
    % t Student (degree 2, 4, 8)
    \foreach \i/\j in {2/color_blue2,4/color_blue3,8/color_blue4}{
        \edef\temp{\noexpand\addplot[
            domain=-5:5, %Domain of the fucntion
            samples=200, %This parameter determines the number of point to be plotted for the function, while bigger the number better looks the function.
            smooth,
            thick, %Stroke of the function. Options: ultra thin, very thin, thin, semithick, thick, very thick, ultra thick.
            \j, %Color of the function.
          ]{\i!/(sqrt(\i)*(\i/2)!*((\i-2)/2)!*2^(\i))*(1+x^2/\i)^(-(\i+1)/2)};
          \noexpand\addlegendentry{$t_\i$}
        }
        \temp
      }
    % Normal
    \addplot[
      domain=-5:5, %Domain of the fucntion
      samples=200, %This parameter determines the number of point to be plotted for the function, while bigger the number better looks the function.
      smooth, %f we use this option, the compiler makes an interpolation between the point plotted to get a soft appearance for the function.
      thick, %Stroke of the function. Options: ultra thin, very thin, thin, semithick, thick, very thick, ultra thick.
      color_green2 %Color of the function.
    ]{1/(2*pi)^(1/2)*exp{-x^2/2}};
    \addlegendentry{$N(0,1)$}
    %\legend{$N(0,1)$,$t_1$,$t_2$,$t_4$,$t_8$}
  \end{axis}
\end{tikzpicture}
\end{document}