\documentclass[../../../main.tex]{subfiles}
% break in parametric statistical model

\begin{document}
\begin{multicols}{2}[\section{Statistics}]
  \subsection{Point estimation}
  \subsubsection{Stadistical models}
  \begin{definition}
    Let $(\Omega,\mathcal{A},\Prob)$ be a probability space\footnote{From now on we will assume that the random variables are defined always in the same probability space $(\Omega,\mathcal{A},\Prob)$, so we will omit to say that.}, $\Theta$ be a set, $n\in\NN$ and $x_1,\ldots,x_n$ be a collection of data that we may assume that they are the outcomes of a random vector $\vf{X}_n=(X_1,\ldots,X_n)$ defined on $(\Omega,\mathcal{A},\Prob)$. Suppose, moreover, that the outcomes of $\vf{X}_n$ are in a set $\mathcal{X}\subseteq\RR^n$, the law $\vf{X}_n$ is one in the set $\mathcal{P}=\{\Prob_\theta^{\vf{X}_n}:\theta\in\Theta\}$ and $\mathcal{F}$ is a $\sigma$-algebra over $\mathcal{X}$\footnote{That is, $\mathcal{P}$ denotes a family of probability distributions of $\vf{X}_n$ in $(\mathcal{X},\mathcal{F})$, indeXed by $\theta\in\Theta$. Note that we denote that distribution of $\vf{X}_n$ by $\Prob^{\vf{X}_n}$ to distinguish it from the probability distribution $\Prob_{\vf{X}_n}$ in $(\Omega,\mathcal{A},\Prob)$.}. We define a \emph{statistical model} as the triplet $(\mathcal{X},\mathcal{F},\mathcal{P})$\footnote{Often we will take $\mathcal{F}=\mathcal{B}(\mathcal{X})$.}. The set $\mathcal{X}$ is called \emph{sample space}, and the set $\Theta$, \emph{parameter space}. The random vector $\vf{X}_n$ is called \emph{random sample}. If, moreover, $X_1,\ldots,X_n$ are i.i.d. random variables, $\vf{X}_n$ is called a \emph{simple random sample}. The value $(x_1,\ldots,x_n)\in\mathcal{X}$ is called a \emph{realization} of $(X_1,\ldots,X_n)$.
  \end{definition}
  \begin{definition}
    Let $(\mathcal{X},\mathcal{F},\{\Prob_\theta^{\vf{X}_n}:\theta\in\Theta\})$ be a statistical model. We say $\mathcal{P}=\{\Prob_\theta^{\vf{X}_n}:\theta\in\Theta\})$ is \emph{identificable} if the function $$\function{}{\Theta}{\mathcal{P}}{\theta}{\Prob_\theta^{\vf{X}_n}}$$ is injective\footnote{From now on, we will suppose that all the sets $\mathcal{P}$ are always identificable.}.
  \end{definition}
  \begin{definition}
    A statistical model $(\mathcal{X},\mathcal{F},\{\Prob_\theta^{\vf{X}_n}:\theta\in\Theta\})$ is said to be \emph{parametric} if $\Theta\subseteq \RR^d$ for some $d\in\NN$\footnote{There are cases where $\Theta$ is not a subset of $\RR^d$. For example, we could have $\Theta=\{f:\RR\rightarrow\RR_{\geq 0} : \int_{-\infty}^{+\infty}f(x)\dd{x}=1\}$.}.
  \end{definition}
  \subsubsection{Stadistics and estimators}
  \begin{definition}[Statistic]
    Let $(\mathcal{X},\mathcal{F},\{\Prob_\theta^{\vf{X}_n}:\theta\in\Theta\})$ be a statistical model. We define a \emph{statistic} as a Borel measurable function. That is, $\vf{T}$ can be written as $\vf{T}=\vf{h}(X_1,\ldots,X_n)$, where $\vf{h}:\mathcal{X}\rightarrow\RR^m$ is a Borel measurable function. Hence, $\vf{T}$ is a random vector. The value $m$ is the \emph{dimension} of the statistic.
  \end{definition}
  \begin{definition}
    Let $(\mathcal{X},\mathcal{F},\{\Prob_\theta^{\vf{X}_n}:\theta\in\Theta\})$ be a statistical model. We define the \emph{sample mean} as the statistic: $$T(X_1,\ldots,X_n)=\frac{1}{n}\sum_{i=1}^nX_i=:\overline{X}_n$$
  \end{definition}
  \begin{definition}
    Let $(\mathcal{X},\mathcal{F},\{\Prob_\theta^{\vf{X}_n}:\theta\in\Theta\})$ be a statistical model. We define the \emph{sample variance} as the statistic: $$T(X_1,\ldots,X_n)=\frac{1}{n}\sum_{i=1}^n{(X_i-\overline{X}_n)}^2:={s_n}^2$$ We define the \emph{corrected sample variance} as the statistic:
    $$T(X_1,\ldots,X_n)=\frac{1}{n-1}\sum_{i=1}^n{(X_i-\overline{X}_n)}^2=:\tilde{s}_n{}^2$$
  \end{definition}
  \begin{proposition}
    Let $X_1,\ldots,X_n$ be random variables. Then: $${s_n}^2=\frac{1}{n}\sum_{i=1}^n{X_i}^2-{\overline{X}_n}^2$$
  \end{proposition}
  \begin{definition}
    Let $(\mathcal{X},\mathcal{F},\{\Prob_{\theta}^{\vf{X}_n}:{\theta}\in\Theta\subseteq \RR\})$ be a statistical model, ${\theta} \in\Theta$ and $g:\Theta\rightarrow\Theta$ be a function. An \emph{estimator} of $g({\theta})$ is a statistic ${\hat\theta}$ whose outcomes are in $\Theta$ and does not depend on any unknown parameter. It is used to give an estimation of the (supposedly unknown) parameter $g({\theta})$.
  \end{definition}
  \subsubsection{Properties of estimators}
  \begin{definition}[Bias]
    Let $(\mathcal{X},\mathcal{F},\{\Prob_{\theta}^{\vf{X}_n}:{\theta}\in\Theta\subseteq \RR\})$ be a parametric statistical model, $g:\Theta\rightarrow\Theta$ be a function and ${\hat\theta}$ be an integrable estimator of $g({\theta})\in\Theta$. We define the \emph{bias} of ${\hat\theta}$ with respect to ${\theta}$ as: $$\bias({\hat\theta}):=\Exp({\hat\theta})-g({\theta})$$ We say that ${\hat\theta}$ is an \emph{unbiased estimator} of $g({\theta})$ if $\bias({\hat\theta})=0$ $\forall\theta\in\Theta$. Otherwise, we say that it is a \emph{biased estimator} of $g({\theta})$.
  \end{definition}
  \begin{proposition}
    Let $(\mathcal{X},\mathcal{F},\{\Prob_{\theta}^{\vf{X}_n}:{\theta}\in\Theta\subseteq \RR\})$ be a parametric statistical model such that $X_1,\ldots,X_n$ are square-integrable\footnote{That is, with finite 2nd moments.} i.i.d. random variables with mean $\mu$ and variance $\sigma^2$. Then: $$\Exp(\overline{X}_n)=\mu\quad\text{and}\quad\Var{(\overline{X}_n)}=\frac{\sigma^2}{n}$$
    Hence, the estimator $\overline{X}_n$ of $\mu$ is unbiased.
  \end{proposition}
  \begin{proposition}
    Let $(\mathcal{X},\mathcal{F},\{\Prob_{\theta}^{\vf{X}_n}:{\theta}\in\Theta\subseteq \RR\})$ be a parametric statistical model such that $X_1,\ldots,X_n$ are square-integrable i.i.d. random variables with mean $\mu$ and variance $\sigma^2$. Then: $$\Exp({s_n}^2)=\frac{n-1}{n}\sigma^2\quad\text{and}\quad\Exp(\tilde{s}_n{}^2)=\sigma^2$$
    Hence, the estimator $\tilde{s}_n{}^2$ of $\sigma^2$ is unbiased whereas the estimator ${s_n}^2$ of $\sigma^2$ is biased.
  \end{proposition}
  \begin{definition}
    Let $(\mathcal{X},\mathcal{F},\{\Prob_{\theta}^{\vf{X}_n}:{\theta}\in\Theta\subseteq \RR\})$ be a parametric statistical model, $g:\Theta\rightarrow\Theta$ be a function and ${\hat\theta}$ be an integrable estimator of $g({\theta})\in\Theta$. The \emph{mean squared error} (\emph{m.s.e.}) of ${\hat\theta}$ is the function: $$\MSE({\hat\theta}):=\Exp\left({({\hat\theta}-g({\theta}))}^2\right)$$
  \end{definition}
  \begin{proposition}
    Let $(\mathcal{X},\mathcal{F},\{\Prob_{\theta}^{\vf{X}_n}:{\theta}\in\Theta\subseteq \RR\})$ be a parametric statistical model, $g:\Theta\rightarrow\Theta$ be a function and ${\hat\theta}$ be an integrable estimator of $g({\theta})\in\Theta$. Then: $$\MSE({\hat\theta})=\Var({\hat\theta})+{(\bias({\hat\theta}))}^2$$
  \end{proposition}
  \begin{definition}
    Let $(\mathcal{X},\mathcal{F},\{\Prob_{\theta}^{\vf{X}_n}:{\theta}\in\Theta\subseteq \RR\})$ be a parametric statistical model, $g:\Theta\rightarrow\Theta$ be a function and ${\hat\theta}$, ${\tilde\theta}$ be estimators of $g({\theta})\in\Theta$. We say that ${\hat\theta}$ is \emph{more efficient than} ${\tilde\theta}$ if $$\Var{({\hat\theta})}<\Var{({\tilde\theta})}\quad\forall{\theta}\in\Theta$$
  \end{definition}
  \begin{definition}
    Let $(\mathcal{X},\mathcal{F},\{\Prob_{\theta}^{\vf{X}_n}:{\theta}\in\Theta\subseteq \RR\})$ be a parametric statistical model and ${\hat\theta}$ be an square integrable estimator of ${\theta}\in\Theta$. We say that ${\hat\theta}$ is a \emph{minimum-variance unbiased estimator} (\emph{MVUE}) if it is an unbiased estimator that has lower variance than any other unbiased estimator $\forall {\theta}\in\Theta$.
  \end{definition}
  \begin{proposition}
    Let $(\mathcal{X},\mathcal{F},\{\Prob_{\theta}^{\vf{X}_n}:{\theta}\in\Theta\subseteq \RR\})$ be a parametric statistical model. Then, the MVUE is unique almost surely.
  \end{proposition}
  \subsubsection{Sufficient statistics}
  \begin{definition}
    Let $(\mathcal{X},\mathcal{F},\{\Prob_\theta^{\vf{X}_n}:\theta\in\Theta\})$ be a statistical model and $\vf{T}$ be an statistic. We say that $\vf{T}$ is \emph{sufficient} for $\theta\in\Theta$ if the joint conditional distribution of $(X_1,\ldots,X_n)$ given $\vf{T}(X_1,\ldots,X_n)=\vf{t}$ does not depend on $\theta$.
  \end{definition}
  \subsubsection{Asymptotic properties}
  \begin{definition}
    For each $n\in\NN$, let $(\mathcal{X},\mathcal{F},\{\Prob_{\theta}^{\vf{X}_n}:\theta\in\Theta\subseteq \RR\})$ be a parametric statistical model with $X_1,\ldots,X_n$ being i.i.d., $g:\Theta\rightarrow\Theta$ be a function and ${\hat\theta}$ be an estimator of $g({\theta})\in\Theta$. We say that the sequence $({{\hat\theta}}_n)$ is a \emph{weakly consistent estimator} of $g(\theta)$ if $${{\hat\theta}}_n\overset{\Prob}{\longrightarrow}g(\theta)$$
  \end{definition}
  \begin{definition}
    For each $n\in\NN$, let $(\mathcal{X},\mathcal{F},\{\Prob_{\theta}^{\vf{X}_n}:\theta\in\Theta\subseteq \RR\})$ be a parametric statistical model with $X_1,\ldots,X_n$ being i.i.d., $g:\Theta\rightarrow\Theta$ be a function and ${\hat\theta}$ be an estimator of $g({\theta})\in\Theta$. We say that the sequence $({{\hat\theta}}_n)$ is a \emph{strongly consistent estimator} of $g(\theta)$ if $${{\hat\theta}}_n\overset{\text{a.s.}}{\longrightarrow}g(\theta)$$
  \end{definition}
  \begin{definition}
    For each $n\in\NN$, let $(\mathcal{X},\mathcal{F},\{\Prob_{\theta}^{\vf{X}_n}:\theta\in\Theta\subseteq \RR\})$ be a parametric statistical model with $X_1,\ldots,X_n$ being i.i.d., $g:\Theta\rightarrow\Theta$ be a function and ${\hat\theta}$ be an estimator of $g({\theta})\in\Theta$. We say that the sequence $({{\hat\theta}}_n)$ is a \emph{consistent estimator in $L^2$} of $g(\theta)$ if $$\lim_{n\to\infty}\Exp\left({({\hat\theta}_n-g({\theta}))}^2\right)=\lim_{n\to\infty}\MSE({\hat\theta}_n)=0$$
  \end{definition}
  \begin{proposition}
    For each $n\in\NN$, let $(\mathcal{X},\mathcal{F},\{\Prob_{\theta}^{\vf{X}_n}:\theta\in\Theta\subseteq \RR\})$ be a parametric statistical model with $X_1,\ldots,X_n$ being i.i.d., $g:\Theta\rightarrow\Theta$ be a function and ${\hat\theta}$ be a consistent estimator in $L^2$ of $g({\theta})\in\Theta$. Then, ${{\hat\theta}}_n$ is a weakly consistent estimator of $g(\theta)$.
  \end{proposition}
  \begin{definition}
    For each $n\in\NN$, let $(\mathcal{X},\mathcal{F},\{\Prob_{\theta}^{\vf{X}_n}:\theta\in\Theta\subseteq \RR\})$ be a parametric statistical model with $X_1,\ldots,X_n$ being i.i.d., $g:\Theta\rightarrow\Theta$ be a function and ${{\hat\theta}}_n$ be an estimator of $g(\theta)\in\Theta$. We say that the sequence $({{\hat\theta}}_n)$ is an \emph{asymptotically unbiased estimator} of $g(\theta)$ if $$\Exp({{\hat\theta}}_n)\longrightarrow g(\theta)$$
  \end{definition}
  \begin{definition}
    For each $n\in\NN$, let $(\mathcal{X},\mathcal{F},\{\Prob_{\theta}^{\vf{X}_n}:\theta\in\Theta\subseteq \RR\})$ be a parametric statistical model with $X_1,\ldots,X_n$ being i.i.d. whose variance is $\sigma^2$, $g:\Theta\rightarrow\Theta$ be a function and ${{\hat\theta}}_n$ be an estimator of $g(\theta)\in\Theta$. We say that the sequence $({\hat\theta}_n)$ is \emph{an asymptotically normal estimator} of $g(\theta)$ with asymptotically variance $\frac{\sigma^2}{n}$ if $$\sqrt{n}({\hat\theta}_n-g(\theta))\overset{\text{d}}{\longrightarrow}N(0,\sigma^2)$$
    for all $\theta\in\Theta$. In that case, we denote it by ${\hat\theta}_n\sim AN\left(g(\theta),\frac{\sigma^2}{n}\right)$.
  \end{definition}
  \subsubsection{Methods of estimation}
  \begin{definition}[Method of moments]
    Let $(\mathcal{X},\mathcal{F},\{\Prob_{\vf\theta}^{\vf{X}_n}:\vf{\theta}\in\Theta\subseteq \RR^d\})$ be a parametric statistical model such that $X_1,\ldots,X_n$ are i.i.d. random variables, and $\mu_k$ be $k$-th moment of each of them. Suppose $\vf{\theta}=(\theta_1,\ldots,\theta_d)$. Then, given a realization $\vf{x}_n=(x_1,\ldots,x_n)\in\mathcal{X}$ of $\vf{X}_n$, a estimator $\vf{\tilde{\theta}}(\vf{x}_n)=(\tilde{\theta}_1(\vf{x}_n),\ldots,\tilde{\theta}_d(\vf{x}_n))$ of $\vf{\theta}$ is given by the solution of the following system:
    $$
      \left\{
      \begin{aligned}
        \frac{1}{n}\sum_{i=1}^nx_i     & =\mu_1(\theta_1,\ldots,\theta_d) \\
        \frac{1}{n}\sum_{i=1}^n{x_i}^2 & =\mu_2(\theta_1,\ldots,\theta_d) \\
                                       & \;\;\vdots                       \\
        \frac{1}{n}\sum_{i=1}^n{x_i}^d & =\mu_d(\theta_1,\ldots,\theta_d) \\
      \end{aligned}
      \right.
    $$
  \end{definition}
  \begin{definition}[Likelihood]
    Let $(\mathcal{X},\mathcal{F},\{\Prob_{\theta}^{\vf{X}_n}:{\theta}\in\Theta\subseteq \RR\})$ be a parametric statistical model, $\vf{x}_n\in\mathcal{X}$ be a realization of $\vf{X}_n$.
    \begin{enumerate}
      \item For the discrete case, let $p_{\vf{X}_n}(\vf{x};\theta)$ be the pmf of $\Prob_{\theta}^{\vf{X}_n}$. In this case, we define the \emph{likelihood function} as the function:
            $$\function{L(\cdot;\vf{x}_n)}{\Theta}{\RR}{\theta}{p_{\vf{X}_n}(\vf{x}_n;\theta)}$$
      \item For the continuous case, let $f_{\vf{X}_n}(\vf{x};\theta)$ be the pdf of $\Prob_{\theta}^{\vf{X}_n}$. In this case, we define the \emph{likelihood function} as the function:
            $$\function{L(\cdot;\vf{x}_n)}{\Theta}{\RR}{\theta}{f_{\vf{X}_n}(\vf{x}_n;\theta)}$$
    \end{enumerate}
  \end{definition}
  \begin{definition}[Maximum likelihood method]
    Let $(\mathcal{X},\mathcal{F},\{\Prob_{\theta}^{\vf{X}_n}:\theta\in\Theta\})$ be a statistical model and $\vf{x}_n\in\mathcal{X}$ be a realization of $\vf{X}_n$. A \emph{maximum likelihood estimator} (\emph{MLE}) of $\theta\in\Theta$ is the estimator $\hat{\theta}$ such that: $$L(\hat{\theta};\vf{x}_n)=\max\{L(\theta;\vf{x}_n):\theta\in\Theta\}\footnote{Note that sometimes this estimator is not unique or may not even exist.}$$
  \end{definition}
  \begin{definition}
    Let $(\mathcal{X},\mathcal{F},\{\Prob_{\vf\theta}^{\vf{X}_n}:\vf\theta\in\Theta\subseteq \RR^d\})$ be a parametric statistical model, $\vf{x}_n\in\mathcal{X}$ be a realization of $\vf{X}_n$. We define the \emph{log-likelihood function} as: $$\ell(\vf\theta;\vf{x}_n):=\ln L(\vf\theta;\vf{x}_n)$$
    We define the \emph{score function} as: $$\vf{S}(\vf\theta;\vf{x}_n):=\pdv{\ell}{\vf\theta}(\vf\theta,\vf{x}_n)$$
  \end{definition}
  \begin{proposition}
    Let $(\mathcal{X},\mathcal{F},\{\Prob_{\vf\theta}^{\vf{X}_n}:\vf{\theta}\in\Theta\subseteq \RR^d\})$ be a parametric statistical model, $\vf{x}_n\in\mathcal{X}$ be a realization of $\vf{X}_n$ Then, a MLE $\vf{\hat\theta}$ of $\vf\theta$ is the one that satisfies:
    $$\pdv{L}{\vf\theta}(\vf{\hat\theta};\vf{x}_n)=\vf{0}$$
    Or equivalently, $\pdv{\ell}{\vf\theta}(\vf{\hat\theta};\vf{x}_n)=\vf{0}$.
  \end{proposition}
  \begin{proposition}[Invariance of the MLE]
    Let $(\mathcal{X},\mathcal{F},\{\Prob_{\theta}^{\vf{X}_n}:\theta\in\Theta\})$ be a statistical model, $\Lambda$ be a parameter space and $g:\Theta\rightarrow\Lambda$ be a measurable function. Suppose $\hat\theta$ is a MLE of $\theta$. Then, $g(\hat\theta)$ is a MLE of $g(\theta)$.
  \end{proposition}
  \subsubsection{Regular statistical models}
  \begin{definition}
    A statistical model $(\mathcal{X},\mathcal{F},\{\Prob_{\theta}^{\vf{X}_n}:\theta\in\Theta\})$ is said to be \emph{regular} if it satisfies the following conditions:
    \begin{enumerate}
      \item $\Theta$ is open.
      \item The support of $\Prob_{\theta}^{\vf{X}_n}$ does not depend on $\theta$.
      \item The function $L(\theta;\vf{x}_n)$ is two times differentiable with respect to $\theta$ $\forall \vf{x}_n\in\mathcal{X}$ (except in a set of probability zero) and moreover:
            \begin{enumerate}
              \item For the discrete case: $$\pdv[2]{}{\theta}\sum_{\vf{x}_n\in\mathcal{X}}L(\theta;\vf{x}_n)=\sum_{\vf{x}_n\in\mathcal{X}}\pdv[2]{L}{\theta}(\theta;\vf{x}_n)$$
              \item\label{EST_regular} For the continuous case: $$\pdv[2]{}{\theta}\int_{\mathcal{X}}L(\theta;\vf{x}_n)\dd{\vf{x}_n}=\int_{\mathcal{X}}\pdv[2]{L}{\theta}(\theta;\vf{x}_n)\dd{\vf{x}_n}$$
            \end{enumerate}
      \item For all $\theta\in\Theta$, we have: $$0<\int_\mathcal{X}{\left(\pdv[2]{\ell}{\theta}(\theta;\vf{x}_n)\right)}^2 f_{\vf{X}_n}(\vf{x};\vf\theta)\dd{\vf{x}}<\infty$$
    \end{enumerate}
  \end{definition}
  \begin{definition}
    Let $(\mathcal{X},\mathcal{F},\{\Prob_{\vf\theta}^{\vf{X}_n}:\vf\theta\in\Theta\subseteq\RR^d\})$ be a regular parametric statistical model, $\vf{x}_n\in\mathcal{X}$ be a realization of $\vf{X}_n$. We define the \emph{observed information} of the model as:
    $$\vf{J}(\vf\theta;\vf{x}_n)=-\pdv[2]{\ell}{\vf\theta}(\vf\theta;\vf{x}_n)$$
    We define the \emph{Fisher information} of the model as: $$\vf{I}(\vf\theta)=\Exp(\vf{J}(\vf\theta;\vf{X}_n))=-\Exp\left(\pdv[2]{\ell}{\vf\theta}(\vf\theta;\vf{X}_n)\right)\footnote{Since generally $\vf{J}(\vf\theta;\vf{X}_n)$ will be a matrix, the expectation of $\vf{J}(\vf\theta;\vf{X}_n)$ is taken component by component.}$$
  \end{definition}
  \begin{proposition}
    Let $(\mathcal{X},\mathcal{F},\{\Prob_{\vf\theta}^{\vf{X}_n}:\vf\theta\in\Theta\subseteq\RR^d\})$ be a regular parametric statistical model, $\vf{x}_n\in\mathcal{X}$ be a realization of $\vf{X}_n$. Then, $\Exp(\vf{S}(\vf\theta;\vf{X}_n))=0$ and $$\vf{I}(\vf\theta)=\Var(\vf{S}(\vf\theta;\vf{X}_n))=\Exp\left[{\left(\pdv{\ell}{\vf\theta}(\vf\theta;\vf{X}_n)\right)}^2\right]$$
    for all $\vf\theta\in\Theta$.
  \end{proposition}
  \begin{proposition}
    Let $(\mathcal{X},\mathcal{F},\{\Prob_{\vf\theta}^{X_1}:\vf\theta\in\Theta\})$ be a regular parametric statistical model of one observation $x_1\in \mathcal{X}$. Then, the model corresponding to $n$ i.i.d. observations $x_1,\ldots,x_n$ is regular and $$\vf{I}(\vf\theta)=n \vf{I}_1(\vf\theta)$$
    where $\vf{I}_1(\vf\theta)$ denotes the Fisher information in the model with one observation.
  \end{proposition}
  \begin{definition}
    Let $(\mathcal{X},\mathcal{F},\{\Prob_{\theta}^{\vf{X}_n}:\theta\in\Theta\})$ be a statistical model and $T$ be a statistic. We say that $T$ is \emph{regular} if
    \begin{enumerate}
      \item for the discrete case: $$\pdv{}{\theta}\sum_{\vf{x}_n\in\mathcal{X}}T(\vf{x}_n)L(\theta;\vf{x}_n)=\sum_{\vf{x}_n\in\mathcal{X}}T(\vf{x}_n)\pdv{L}{\theta}(\theta;\vf{x}_n)$$
      \item for the continuous case: $$\pdv{}{\theta}\int_{\mathcal{X}}T(\vf{x}_n)L(\theta;\vf{x}_n)\dd{\vf{x}_n}=\int_{\mathcal{X}}T(\vf{x}_n)\pdv{L}{\theta}(\theta;\vf{x}_n)\dd{\vf{x}_n}$$
    \end{enumerate}
    for all $\theta\in\Theta$.
  \end{definition}
  \begin{theorem}[Cramér-Rao bound]
    Let $(\mathcal{X}, \mathcal{F}, \{\Prob_{\theta}^{\vf{X}_n}: \theta\in\Theta\subseteq\RR\})$ be a regular parametric statistical model, $\vf{x}_n\in\mathcal{X}$ be a realization of $\vf{X}_n$, $g:\Theta\rightarrow\Theta$ be a differentiable function and ${\hat\theta}$ be a regular and unbiased estimator of $g({\theta})\in\Theta$. Then: $$\Var(\hat\theta)\geq \frac{{g'(\theta)}^2}{I(\theta)}$$
  \end{theorem}
  \begin{definition}
    Let $(\mathcal{X},\mathcal{F},\{\Prob_{\theta}^{\vf{X}_n}:\theta\in\Theta\})$ be a regular statistical model, $\vf{x}_n\in\mathcal{X}$ be a realization of $\vf{X}_n$, $g:\Theta\rightarrow\Theta$ be a differentiable function and ${\hat\theta}$ be a regular and unbiased estimator of $g({\theta})\in\Theta$. We say that $\hat\theta$ is an \emph{efficient estimator} of $\theta$ if $$\Var(\hat\theta)=\frac{{g'(\theta)}^2}{I(\theta)}$$
  \end{definition}
  \begin{proposition}
    Let $(\mathcal{X},\mathcal{F},\{\Prob_{\theta}^{\vf{X}_n}:\theta\in\Theta\})$ be a regular statistical model, $g:\Theta\rightarrow\Theta$ be a function and ${\hat\theta}$ be a regular and unbiased estimator of $g({\theta})\in\Theta$. Then, $\hat\theta$ is MVUE in the class of regular estimators.
  \end{proposition}
  \begin{theorem}
    Let $(\mathcal{X},\mathcal{F},\{\Prob_{\theta}^{\vf{X}_n}:\theta\in\Theta\})$ be a regular statistical model, $\vf{x}_n\in\mathcal{X}$ be a realization of $\vf{X}_n$ and ${\hat\theta}$ be a MLE strongly consistent estimator of ${\theta}\in\Theta$. Suppose that $\pdv[2]{\ell}{\theta}$ is a continuos function of $\theta$ and that $$\left|\pdv[2]{\ell}{\theta}(\tilde{\theta};\vf{x}_n)\right|<h(\vf{x};\theta)$$
    with $\int_\mathcal{X}h(\vf{x};\theta)L(\theta;\vf{x})\dd{\vf{x}}<\infty$, for all $\tilde{\theta}$ in a neighborhood of $\theta$. Then:
    $$\hat\theta\overset{\text{d}}{\longrightarrow}N\left(\theta,\frac{1}{I(\theta)}\right)$$
  \end{theorem}
  \subsection{Distributions relating \texorpdfstring{$N(\mu,\sigma^2)$}{N(mu,sigma2)}}
  \subsubsection{Standard normal distribution}
  \begin{definition}
    We denote by $\Phi(t)$ the cdf of a standard normal distribution $N(0,1)$.
  \end{definition}
  \begin{definition}[Quantile]
    We define quantile function $Q(p)$ of a distribution as the inverse of the cmf. In particular, we denote the quantile of a standard normal distribution as $z_p:=Q(p)$.
  \end{definition}
  \subsubsection{\texorpdfstring{$\chi^2$}{chi2}-distribution}
  \begin{definition}
    Let $n\in\NN$ and $X_1,\ldots,X_n$ be independent random variables such that $X_i\sim \text{Gamma}(\alpha_i,\beta)$ for $i=1,\ldots,n$. Then: $$\sum_{i=1}^nX_i\sim\text{Gamma}\left(\sum_{i=1}^n\alpha_i,\beta\right)$$
  \end{definition}
  \begin{corollary}
    Let $n\in\NN$ and $Z_1,\ldots,Z_n$ be i.i.d. random variable with standard normal distribution. Then $${Z_1}^2+\cdots+{Z_n}^2\sim\text{Gamma}\left(\frac{n}{2},\frac{1}{2}\right)$$
  \end{corollary}
  \begin{definition}
    We define the \emph{chi-squared distribution with $n$ degrees of freedom}, denoted as ${\chi_n}^2$, as the distribution $${\chi_n}^2:=\text{Gamma}\left(\frac{n}{2},\frac{1}{2}\right)$$ which is the distribution of ${Z_1}^2+\cdots+{Z_n}^2$, where $Z_1,\ldots,Z_n\sim N(0,1)$ are i.i.d. random variables. Its pdf is:
    $$f_{{\chi_n}^2}(x)=\frac{1}{2^\frac{n}{2}\Gamma\left(\frac{n}{2}\right)}x^{\frac{n}{2}-1}\exp{-\frac{x}{2}}\vf{1}_{(0,\infty)}(x)$$
  \end{definition}
  \subsubsection{Student's \texorpdfstring{$t$}{t}-distribution}
  \begin{definition}
    Let $n\in\NN$ and $Z\sim N(0,1)$ and $Y\sim{\chi_n}^2$ be independent random variables. We define the \emph{Student's $t$-distribution with $n$ degrees of freedom} as the distribution of: $$\frac{Z}{\sqrt{Y/n}}$$
  \end{definition}
  \begin{proposition}
    Let $n\in\NN$. Then, the pdf of $t_n$ is: $$f_{t_n}(x)=\frac{\Gamma\left(\frac{n+1}{2}\right)}{\sqrt{\pi n}\Gamma\left(\frac{n}{2}\right)}{\left(1+\frac{x^2}{n}\right)}^{-\frac{n+1}{2}}$$
  \end{proposition}
  \subsubsection{Fisher's theorem}
  \begin{theorem}
    Let $(\mathcal{X},\mathcal{F},\{\Prob_\theta^{\vf{X}_n}:\theta\in\Theta\})$ be a parametric statistical model and suppose $X_1,\ldots,X_n\sim N(\mu,\sigma^2)$ are i.i.d. random variables. Then:
    \begin{enumerate}
      \item $\overline{X}_n\sim N\left(\mu,\frac{\sigma^2}{n}\right)$
      \item $\tilde{s}_n{}^2\sim\frac{\sigma^2}{n-1}{\chi_{n-1}}^2$
      \item $\overline{X}_n$ and $\tilde{s}_n{}^2$ are independent.
    \end{enumerate}
  \end{theorem}
  \begin{corollary}
    Let $n\in\NN$ and $X_1,\ldots,X_n\sim N(\mu,\sigma^2)$ be i.i.d. random variables. Then: $$\frac{\overline{X}_n-\mu}{\frac{\tilde{s}_n}{\sqrt{n}}}\sim t_{n-1}$$
  \end{corollary}
  \begin{corollary}
    Let $n\in\NN$ and $X\sim t_n$ be a random variable. Then: $$X\overset{\text{d}}{\longrightarrow }N\left(0,1\right)$$
    Hence, $N(0,1)=t_\infty$.
  \end{corollary}
  \begin{corollary}
    Let $(\mathcal{X},\mathcal{F},\{\Prob_\theta^{\vf{X}_n}:\theta\in\Theta\})$ be a parametric statistical model and suppose $X_1,\ldots,X_n\sim N(\mu,\sigma^2)$ are i.i.d. random variables. Then, the estimators $\overline{X}_n$ of $\mu$ and $\tilde{s}_n{}^2$ of $\sigma^2$ are unbiased and consistent.
  \end{corollary}
  \begin{center}
    \begin{minipage}{\linewidth}
      \centering
      \includestandalone[mode=image|tex,width=0.95\linewidth]{Images/student-normal}
      \captionof{figure}{Probability density function of 4 Student's $t$-distribution together with a standard normal $N(0,1)=t_{\infty}$.}
    \end{minipage}
  \end{center}
\end{multicols}
\end{document}