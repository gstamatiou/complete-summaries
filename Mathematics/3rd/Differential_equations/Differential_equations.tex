\documentclass[../../../main.tex]{subfiles}


\begin{document}
\renewcommand{\col}{\apl}
\begin{multicols}{2}[\section{Differential equations}]
  Along this document we will often write the points in $\RR^n$, $n\geq 2$, in bold face (as well as the vectors) in order to be consistent when handling points and vectors together.
  \subsection{Space of continuous and bounded functions}
  \begin{definition}
    Let $X$, $Y$ be topological spaces. We define the following sets:
    \begin{gather*}
      \mathcal{C}(X,Y)=\{f:X\longrightarrow Y:f\text{ is continuous}\}\\
      \mathcal{C}_\text{b}(X,\RR^n)=\{f\in\mathcal{C}(X,\RR^n):f\text{ is bounded}\}
    \end{gather*}
  \end{definition}
  \begin{theorem}
    Let $X$ be a topological space and $f\in\mathcal{C}_\text{b}(X,\RR^n)$. We define the norm of $f$ as: $$\|f\|:=\sup\{\|f(x)\|:x\in X\}$$ and a distance $d$ in $\mathcal{C}_\text{b}(X,\RR^n)$ as: $$d(f,g):=\|f-g\|\quad\forall f,g\in\mathcal{C}_\text{b}(X,\RR^n)$$
    Then, $(\mathcal{C}_\text{b}(X,\RR^n),d)$ is a complete metric space.
  \end{theorem}
  \begin{theorem}
    Let $X$ be a topological space and $C\subseteq\RR^n$ be a closed subset. Then, $(\mathcal{C}(X,C),d)$ is a complete metric space.
  \end{theorem}
  \begin{corollary}
    Let $K\subset \RR^n$ be a compact subset and $C\subseteq\RR^n$ be a closed subset. Then, $(\mathcal{C}(K,C),d)$ is a complete metric space.
  \end{corollary}
  \begin{corollary}
    Let $D\subset\RR^n$ be a closed set and $X=\mathcal{C}([a,b],D)$. Then $(X,d)$ is also a complete metric space.
  \end{corollary}
  \subsection{Ordinary differential equations}
  \begin{definition}
    An \emph{ordinary differential equation} (\emph{ode}) of $m$ unknowns and of order $n$ in \emph{implicit form} is an expression of the form: $$\vf{f}\left(t,\vf{x}(t),\vf{x}'(t),\vf{x}''(t),\ldots,\vf{x}^{(n)}(t)\right)=\vf{0}$$
    where $\vf{x}:U\subseteq\RR\rightarrow\RR^m$ is a vector-valued function of one variable $t\in\RR$ (which is called \emph{independent variable}) and $\vf{f}:\Omega\subseteq\RR\times\RR^{m\cdot(n+1)}\rightarrow\RR^m$, where both $U$ and $\Omega$ are open sets. The same ordinary differential equation in \emph{explicit form} is an expression of the form: $$\vf{x}^{(n)}(t)=\vf{g}\left(t,\vf{x}(t),\vf{x}'(t),\vf{x}''(t),\ldots,\vf{x}^{(n-1)}(t)\right)$$
    where $\vf{g}:\Omega\subseteq\RR\times\RR^{m\cdot n}\rightarrow\RR^m$\footnote{Sometimes we will write $\vf{x}^{(n)}=\vf{g}\left(t,\vf{x},\vf{x}',\ldots,\vf{x}^{(n-1)}\right)$ instead of $\vf{x}^{(n)}(t)=\vf{g}\left(t,\vf{x}(t),\vf{x}'(t),\ldots,\vf{x}^{(n-1)}(t)\right)$ in order to simplify the notation.}.
  \end{definition}
  \begin{definition}
    Consider the following ode of $m$ unknowns and of order $n$:
    \begin{equation}\label{DE_ode1}
      \vf{x}^{(n)}(t)=\vf{f}\left(t,\vf{x}(t),\vf{x}'(t),\ldots,\vf{x}^{(n-1)}(t)\right)
    \end{equation}
    We say that $\vf{\varphi}:I\subseteq\RR\rightarrow\RR^m$ is a \emph{solution of the ode} if:
    \begin{itemize}
      \item $\vf{\varphi}$ is $n$ times differentiable on $I$.
      \item $\displaystyle\left\{\left(t,\vf{\varphi}(t),\vf{\varphi}'(t),\ldots,\vf{\varphi}^{(n-1)}(t)\right):t\in I\right\}\subseteq\domain \vf{f}$
      \item For all $t\in I$ we have:
            $$\vf{\varphi}^{(n)}(t)=\vf{f}\left(t,\vf{\varphi}(t),\vf{\varphi}'(t),\ldots,\vf{\varphi}^{(n-1)}(t)\right)$$
    \end{itemize}
    The set of all solutions of an ode is called \emph{general solution of the ode}.
  \end{definition}
  \begin{proposition}
    Consider an ode of $m$ unknowns and order $n$ of the form of \cref{DE_ode1}. Then, we can transform this ode to an ode of $m\cdot n$ unknowns and order 1 in the following way\footnote{Therefore, we will mainly study the odes of order 1.}. Define $\vf{y}_i=\vf{x}^{(i-1)}$ for $i=1,\ldots,n$. Therefore, the functions $\vf{y}_i$ must satisfy:
    \begin{equation*}
      \left\{
      \begin{aligned}
        {\vf{y}_1}'     & =\vf{y}_2                                                        \\
        {\vf{y}_2}'     & =\vf{y}_3                                                        \\
                        & \;\;\vdots                                                       \\
        {\vf{y}_{n-1}}' & =\vf{y}_{n-2}                                                    \\
        {\vf{y}_n}'     & =\vf{f}\left(t,\vf{y}_1(t),\vf{y}_2(t),\ldots,\vf{y}_n(t)\right) \\
      \end{aligned}
      \right.
    \end{equation*}
    This is called a \emph{system of ordinary differential equations} (of order 1) or a \emph{differential system}.
  \end{proposition}
  \begin{definition}
    We say that an ode is \emph{autonomous} if it doesn't depend on the independent variable, that is, if it is of the form: $$\vf{x}'=\vf{f}(\vf{x})$$ Otherwise, we say that an ode is \emph{non-autonomous}.
  \end{definition}
  \begin{definition}
    We say that an ode of order $n$ is \emph{linear} if it is of the form:
    \begin{equation}\label{DE_linear0}
      a_0(t)\vf{x}+a_1(t)\vf{x}'+\cdots+a_n(t)\vf{x}^{(n)}=\vf{b}(t)
    \end{equation}
    where $a_i\in\mathcal{C}(I,\RR)$ for $i=0,\ldots,n$ and $\vf{b}\in\mathcal{C}(I,\RR^m)$ are arbitrary functions which do not need to be linear. We say that the linear ode of \cref{DE_linear0} is \emph{homogeneous} if $\vf{b}(t)=\vf{0}$ $\forall t\in I$. We say that linear ode of \cref{DE_linear0} is of \emph{constant coefficients} if $a_i(t):=a_{i0}\in\RR$ $\forall t\in I$ and $\forall i=0,\ldots,n$.
  \end{definition}
  \begin{definition}[Initial value problem]
    Let $U\subseteq\RR\times\RR^n$ be an open set and $\vf{f}:U\rightarrow\RR^n$ be a function. Given $(t_0,\vf{x}_0)\in U$, the \emph{initial value problem} (\emph{ivp}) (or \emph{Cauchy problem}) consists in finding a solution of the ode $$\vf{x}'=\vf{f}(t,\vf{x})$$ with \emph{initial conditions} $\vf{x}(t_0)=\vf{x}_0$.
  \end{definition}
  \subsubsection{Methods for solving odes}
  \begin{method}[Separation of variables]
    Let $f:(a,b)\rightarrow\RR$, $g:(c,d)\rightarrow\RR$ be continuous functions such that $f(x)\ne 0$ $\forall x\in (a,b)$. Consider the ode $x'=f(x)g(t)$. To find the solution of this ode, proceed as follows:
    $$x'=f(x)g(t)\iff \int\frac{\dd{x}}{f(x)}=C+\int g(t)\dd{t}$$ where the constant $C$ is determined with the initial conditions of the ode.
  \end{method}
  \begin{method}[Variation of constants]
    Let $I\subset \RR$ be an interval, $a,b\in\mathcal{C}(I,\RR)$. Consider the ode $x'=a(t)x+b(t)$. To find the solution of this ode, proceed as follows:
    \begin{enumerate}
      \item Find the solution of the associated homogeneous system with the separation of variables method. Let's say that is $\varphi(t)c$, where $c\in\RR$.
      \item Try to find a general solution of the form $\varphi(t)c(t)$:
            \begin{multline*}
              \left(\varphi(t)c(t)\right)'=a(t)\varphi(t)c(t)+b(t)\iff\\\varphi(t)'c(t)+\varphi(t)c(t)'=a(t)\varphi(t)c(t)+b(t)\iff\\\varphi(t)c(t)'=b(t)\iff c(t)=d+\int\varphi(t)^{-1}b(t)\dd{t}
            \end{multline*}
            where $d\in\RR$. Hence, the general solution will be: $$\varphi(t)\left(d+\int\varphi(t)^{-1}b(t)\dd{t}\right)$$
    \end{enumerate}
  \end{method}
  \begin{method}[Characteristic equation]
    Consider the following ode of order $n$ of constant coefficients:
    \begin{equation}\label{DE_char}
      a_n x^{(n)} + a_{n-1}x^{(n-1)} + \cdots + a_1 x' + a_0 x = 0
    \end{equation}
    We define the \emph{characteristic equation} of that system as the equation: $$p(r):=a_n r^n + a_{n-1}r^{n-1} + \cdots + a_1 r + a_0 = 0$$
    In order to find the solution of this ode, we need to find the solutions to $p(r)=0$. So suppose $p$ has $s$ distinct real roots and $2(m-s)$ distinct complex roots.
    $$\lambda_1,\ldots,\lambda_s,\lambda_{s+1},\overline{\lambda_{s+1}},\ldots,\lambda_{m},\overline{\lambda_m}$$
    Here, $\lambda_i\in\RR$ $\forall i=1,\ldots,s$ and $\lambda_{i}=\alpha_i+\ii\beta_i\in\CC$ $\forall i=s+1,\ldots,m$. Assume, each of these roots have multiplicity $k_i\in\NN$. Then, the general solution to \cref{DE_char} is:
    \begin{multline*}
      \varphi(t)=\sum_{i=1}^s\left(c_{i,1}+c_{i,2}t+\cdots+c_{i,k_i}t^{k_i-1}\right)\exp{\lambda_i t}+\\
      +\sum_{i=s+1}^m\sum_{j=1}^{k_i}c_{i,j}t^{j-1}\exp{\alpha_i t}\left(\cos(\beta_i t)+\sin(\beta_i t)\right)
    \end{multline*}
    where $c_{i,j}\in\RR$ are constants.
  \end{method}
  \begin{corollary}
    Consider the following ode of order $n$ of constant coefficients:
    \begin{equation}\label{DE_char2}
      x'' + px' q = 0
    \end{equation}
    Let $\lambda_1,\lambda_2$ be the roots of the polynomial $p(r)=r^2+pr+q$.
    Then, the general solution to \cref{DE_char2} is:
    \begin{itemize}
      \item If $p^2-4q>0$: $$\varphi(t)=c_1\exp{\lambda_1t}+c_2\exp{\lambda_2t}$$
      \item If $p^2-4q=0$, then $\lambda_1=\lambda_2$ and the solution is: $$\varphi(t)=c_1\exp{\lambda_1t}+c_2t\exp{\lambda_1t}$$
      \item If $p^2-4q<0$, then $\lambda_1=\alpha+\ii\beta\in\CC$ and the solution is: $$\varphi(t)=\exp{\alpha t}\left[c_1\cos(\beta t)+c_2\sin(\beta t)\right]$$
    \end{itemize}
  \end{corollary}
  \begin{method}[Reducible linear ode of second order]
    Let $I\subset \RR$ be an interval, $a,b,c,d\in\mathcal{C}(I,\RR)$. Consider the system of odes:
    \begin{equation}\label{DE_ode-complex}
      \left\{
      \begin{aligned}
        x' & =a(t)x-b(t)y+c(t) \\
        y' & =b(t)x+a(t)y+d(t)
      \end{aligned}
      \right.
    \end{equation}
    In order to find the solution of this ode, consider the change of variable $z=x+\ii y$. Then, \cref{DE_ode-complex} becomes:
    $$z'=[a(t)+\ii b(t)]z+c(t)+\ii d(t)$$ which is a linear ode of order 1 and can be easily solved.
  \end{method}
  \begin{method}[Bernoulli differential equation]
    Let $p,q\in\mathcal{C}((a,b),\RR)$ and $\alpha\in\RR$. Consider the \emph{Bernoulli differential equation}:
    \begin{equation}\label{DE_bernoulli}
      x'+p(t)x=q(t)x^\alpha
    \end{equation}
    If $\alpha=0,1$ the ode is linear. So suppose $\alpha\ne 0,1$. In order to solve it, consider the change of variable $y=x^{1-\alpha}$. Then, \cref{DE_bernoulli} becomes:
    $$y'+(1-\alpha)p(t)y=(1-\alpha)q(t)$$  which is a linear ode of order 1 and can be easily solved.
  \end{method}
  \begin{method}[Riccati differential equation]
    Let $q_0,q_1,q_2\in\mathcal{C}((a,b),\RR)$. Consider the \emph{Riccati differential equation}:
    \begin{equation}\label{DE_riccati}
      x'=q_0(t)+q_1(t)x+q_2(t)x^2
    \end{equation}
    Suppose we have found a particular solution $x_1(t)$ of the ode of \cref{DE_riccati}. In order to find the general solution, consider the change of variable $x=x_1(t)+\frac{1}{y}$. Then, \cref{DE_riccati} becomes:
    $$y'+[q_1(t)+2q_2(t)x_1(t)]y=-q_2(t)$$ which is a linear ode of order 1 and can be easily solved.
  \end{method}
  \begin{method}[Integrating factor]
    Consider the ode: $$p(t,x)+q(t,x)x'=0\iff p(t,x)\dd{t}+q(t,x)\dd{x}=0$$ where $p,q\in\mathcal{C}^1(U,\RR)$ and $U\subseteq\RR^2$ is an open set.
    An \emph{integrating factor} $\mu(t,x)\in\mathcal{C}^1(U)$, $\mu(t,x)\ne 0$, is a function so that $$\mu(t,x)p(t,x)\dd{t}+\mu(t,x)q(t,x)\dd{x}$$ is an exact differential ($\dd{\Phi(t,x)}$) of a function $\Phi(t,x)$, that is:
    \begin{gather}
      \label{DE_ifactor1}\frac{\partial\Phi}{\partial t}(t,x)=\mu(t,x)p(t,x)\\
      \label{DE_ifactor2}\frac{\partial\Phi}{\partial x}(t,x)=\mu(t,x)q(t,x)
    \end{gather}
    So we need that: $$\frac{\partial}{\partial x}\left(\mu(t,x)p(t,x)\right)=\frac{\partial}{\partial t}\left(\mu(t,x)q(t,x)\right)$$
    From here, in certain cases, we will be able to find $\mu(x,y)$ and, therefore, $\Phi(t,x)$ by integrating \cref{DE_ifactor1,DE_ifactor2}.
  \end{method}
  \subsection{Existence and uniqueness of solutions}
  \begin{proposition}
    Let $f:(a,b)\rightarrow\RR$ be a continuous function such that $f(x)\ne 0$ $\forall x\in(a,b)$. Then, the ivp
    $$
      \begin{cases}
        x'      =f(x) \\
        x(t_0)  =x_0
      \end{cases}
    $$
    has a unique solution $\forall t_0\in\RR$ and $\forall x_0\in(a,b)$.
  \end{proposition}
  \begin{proposition}
    Let $f:(a,b)\rightarrow\RR$, $g:(c,d)\rightarrow\RR$ be continuous functions such that $f(x)\ne 0$ $\forall x\in(a,b)$. Then, the ivp
    $$\begin{cases}
        x'      =f(x)g(t) \\
        x(t_0)  =x_0
      \end{cases}$$
    has a unique solution $\forall t_0\in(c,d)$ and $\forall x_0\in(a,b)$.
  \end{proposition}
  \begin{proposition}
    Let $I\subseteq\RR$ be an interval and $a:I\rightarrow\RR$ and $b:I\rightarrow\RR$ be continuous functions. Then, the ivp
    $$\begin{cases}
        x'      =a(t)x+b(t) \\
        x(t_0)  =x_0
      \end{cases}$$
    has a unique solution $\forall t_0\in I$ and $\forall x_0\in\RR$\footnote{See \cref{DE_sol-lin} for the solution.}.
  \end{proposition}
  \subsubsection{Lipschitz continuity}
  \begin{definition}
    Let $\vf{f}:U\subseteq\RR\times\RR^n\rightarrow\RR^m$ be a function. We say that $\vf{f}$ is \emph{Lipschitz continuous with respect to the second variable} if $\exists L\in\RR_{>0}$ such that: $$\|\vf{f}(t,\vf{x})-\vf{f}(t,\vf{y})\|\leq L\|\vf{x}-\vf{y}\|\qquad\forall (t,\vf{x}),(t,\vf{y})\in U$$
  \end{definition}
  \begin{definition}
    Let $\vf{f}:U\subseteq\RR\times\RR^n\rightarrow\RR^m$ be a function. We say that $\vf{f}$ is \emph{locally Lipschitz continuous with respect to the second variable} if $\forall (t_0,\vf{x}_0)\in U$ there exists a neighbourhood $V$ of $(t_0,\vf{x}_0)$ such that $f|_V$ is Lipschitz continuous with respect to the second variable.
  \end{definition}
  \begin{proposition}
    Let $U\subseteq\RR\times\RR^n$ be an open set and $\vf{f}:U\subseteq\RR\times\RR^n\rightarrow\RR^n$ be a function. Then:
    \begin{enumerate}
      \item If $\vf{f}$ is locally Lipschitz continuous with respect to the second variable, then it is continuous with respect to the second variable.
      \item If $\vf{f}$ is Lipschitz continuous with respect to the second variable, then it is uniformly continuous with respect to the second variable.
      \item If $\vf{f}$ is continuous, $U$ is compact and $\vf{f}$ is locally Lipschitz continuous with respect to the second variable, then $\vf{f}$ is Lipschitz continuous with respect to the second variable.
    \end{enumerate}
  \end{proposition}
  \begin{proposition}
    Let $U\subseteq\RR\times\RR^n$ be an open and convex set and $\vf{f}:U\subseteq\RR\times\RR^n\rightarrow\RR^n$ be a function of class $\mathcal{C}^1$. Then:
    \begin{enumerate}
      \item $\vf{f}$ is locally Lipschitz continuous with respect to the second variable.
      \item $\vf{f}$ is Lipschitz continuous with respect to the second variable if and only if $\vf{Df}$ is bounded.
    \end{enumerate}
  \end{proposition}
  \subsubsection{Picard theorem}
  \begin{proposition}
    Let $U\subseteq\RR\times\RR^n$ be an open set and $\vf{f}:U\rightarrow\RR^n$ be a continuous function. Let $I\subseteq\RR$ be an open interval, $t_0\in I$ and $\vf{x}_0\in\RR^n$ be such that $(t_0,\vf{x}_0)\in U$. Then, a continuous function $\vf{\varphi}:I\rightarrow\RR^n$ is a solution of the ivp
    \begin{equation}
      \begin{cases}
        \vf{x}'=\vf{f}(t,\vf{x}) \\
        \vf{x}(t_0)=\vf{x}_0
      \end{cases}
      \label{DE_ivp}
    \end{equation}
    if and only if $$\vf{\varphi}(t)=\vf{x}_0+\int_{t_0}^tf(s,\vf{\varphi}(s))\dd{s}\quad\forall t\in I$$
  \end{proposition}
  \begin{definition}
    An \emph{operator} is a function whose domain is a set of functions.
  \end{definition}
  \begin{definition}
    Let $U\subseteq\RR\times\RR^n$ be an open set, $(t_0,\vf{x}_0)\in U$, $\vf{f}:U\rightarrow\RR^n$ be a continuous function and $I$ be a closed interval. We define the operator
    $$
      \function{\vf{T}}
      {\mathcal{C}(I,\RR^n)}
      {\mathcal{C}(I,\RR^n)}
      {\vf{\varphi}}
      {\displaystyle\vf{T}\vf{\varphi}(t)=\vf{x}_0+\int_{t_0}^tf(s,\vf{\varphi}(s))\dd{s}\footnotemark}
    $$
  \end{definition}
  \begin{theorem}[Banach fixed-point theorem]\footnotetext{Note that the fixed points of this operator are precisely the solutions of the ivp of \cref{DE_ivp}.}
    Let $(X,d)$ be a complete metric space and $f:(X,d)\rightarrow (X,d)$ be a contraction. Then, $f$ has a unique fixed point $p\in X$\footnote{Furthermore, $p$ can be found as follows: start with an arbitrary element $x_0\in X$ and define a sequence $(x_n)$ by $x_n=f(x_{n-1})$ for $n\geq 1$. Then, $\displaystyle\lim_{n\to\infty} x_n=p$.}.
  \end{theorem}
  \begin{corollary}
    Let $(X,d)$ be a complete metric space and $f:(X,d)\rightarrow (X,d)$ be a function. If there exists $m\in\NN$ such that $f^m$ is a contraction, then $f$ has a unique fixed point $p\in X$.
  \end{corollary}
  \begin{definition}
    Let $t_0\in\RR$, $\vf{x}_0\in\RR^n$ and $a,b\in\RR_{>0}$. We define the following sets: $$I_a(t_0):=[t_0-a,t_0+a]\subset\RR\;\;\text{and}\;\;\overline{B}_{b}(\vf{x}_0):=\overline{B}(\vf{x}_0,b)\subset\RR^n$$
  \end{definition}
  \begin{theorem}[Picard theorem]\label{DE_picard}
    Let $t_0\in\RR$, $\vf{x}_0\in\RR^n$, $a,b\in\RR_{>0}$, $\vf{f}:I_a(t_0)\times\overline{B}_{b}(\vf{x}_0)\subset\RR\times\RR^n\rightarrow\RR^n$ be a continuous function and Lipschitz continuous with respect to the second variable, and define: $$M:=\max\{\|\vf{f}(t,x)\|:(t,x)\in I_a(t_0)\times\overline{B}_{b}(\vf{x}_0)\}$$ Then, the ivp of \cref{DE_ivp} has a unique solution $\vf{\varphi}:I_\alpha(t_0)\rightarrow\overline{B}_{b}(\vf{x}_0)$, where $\alpha:=\min\left\{a,\frac{b}{M}\right\}$.
  \end{theorem}
  \begin{corollary}
    Let $I\subset \RR$ be a closed interval, $t_0\in I$, $\vf{x}_0\in\RR^n$ and $\vf{f}:I\times\RR^n\rightarrow\RR^n$ be a continuous function and Lipschitz continuous with respect to the second variable. Then, the ivp of \cref{DE_ivp} has a unique solution $\vf{\varphi}:I\rightarrow\RR^n$.
  \end{corollary}
  \begin{corollary}[Picard iteration process]
    Suppose we want to solve the ivp of \cref{DE_ivp}. That is, we look for a solution $\vf{\varphi}(t)$. Let $\vf{\varphi}_0$ be a fixed function (usually chosen to be $\vf{\varphi}_0=\vf{x}_0$) and define
    $$\vf{\varphi}_{n+1}(t)=\vf{T}\vf{\varphi}_n(t)=\vf{x}_0+\int_{t_0}^tf(s,\vf{\varphi}_n(s))\dd{s}$$
    for all $n\geq 0$. Then, $\displaystyle\vf{\varphi}(t)=\lim_{n\to\infty}\vf{\varphi}_n(t)$.
  \end{corollary}
  \begin{corollary}
    Let $U\subseteq\RR\times\RR^n$ be an open set and $\vf{f}:U\rightarrow\RR^n$ be a continuous function and locally Lipschitz continuous with respect to the second variable. Then, $\forall(t_0,\vf{x}_0)\in U$, there exists $\alpha(t_0,\vf{x}_0)\in\RR_{>0}$ and a neighbourhood $V_{t_0,\vf{x}_0}=I_{a(t_0,\vf{x}_0)}(t_0)\times\overline{B}_{b(t_0,\vf{x}_0)}(\vf{x}_0)$ of $(t_0,\vf{x}_0)$ in $U$ such that the ivp of \cref{DE_ivp} has a unique solution $\vf{\varphi}_{t_0,\vf{x}_0}$ defined on $I_{\alpha(t_0,\vf{x}_0)}\subseteq I_{a(t_0,\vf{x}_0)}$ with $\graph(\vf{\varphi}_{t_0,\vf{x}_0})\subset V_{t_0,\vf{x}_0}$.
  \end{corollary}
  \begin{proposition}
    Let $I\subseteq\RR$ be an interval and $\vf{f}:I\times\RR^n\rightarrow\RR^n$ be a continuous function and Lipschitz continuous with respect to the second variable. Then, $\forall(t_0,\vf{x}_0)\in I\times\RR^n$ there is a unique solution of the ivp of \cref{DE_ivp} defined on $I$.
  \end{proposition}
  \begin{corollary}
    Let $I\subseteq\RR$ be an interval and $\vf{A}:I\rightarrow\mathcal{L}(\RR^n,\RR^n)$ and $\vf{b}:I\rightarrow\RR^n$ be continuous functions. Then, for all $(t_0,\vf{x}_0)\in I\times\RR^n$ the ivp
    $$
      \begin{cases}
        \vf{x}'=\vf{A}(t)\vf{x}+\vf{b}(t) \\
        \vf{x}(t_0)=\vf{x}_0
      \end{cases}
    $$
    has a unique solution defined on $I$.
  \end{corollary}
  \begin{theorem}
    Let $f:[t_0,t_1]\times\RR\rightarrow\RR$ be a continuous function and $x_0\in \RR$. Suppose that $f$ is decreasing with respect to the second variable. Then, the ivp
    \begin{equation*}
      \begin{cases}
        x'=f(t,x) \\
        x(t_0)=x_0
      \end{cases}
    \end{equation*}
    has a unique solution defined on $[t_0,t_1^*]$, where $t_1^*\leq t_1$.
  \end{theorem}
  \subsubsection{Peano theorem}
  \begin{definition}
    Let $(X,d)$ be a metric space and $F\subset\mathcal{C}(X,\RR^n)$ be a subset. We say that $F$ is \emph{pointwise bounded} if: $$\forall x\in X\;\exists M_x>0\text{ such that }\|\vf{f}(x)\|\leq M_x\quad\forall\vf{f}\in F$$
    We say that $F$ is \emph{uniformly bounded} if: $$\exists M>0\text{ such that }\|\vf{f}(x)\|\leq M \quad\forall\vf{f}\in F\text{ and }\forall x\in X$$
  \end{definition}
  \begin{definition}
    Let $(X,d)$ be a metric space and $F\subset\mathcal{C}(X,\RR^n)$ be a subset. We say that $F$ is \emph{equicontinuous at a point} $x_0\in X$ if $\forall \varepsilon>0$ $\exists \delta>0$ such that $\forall x\in X$ with $d(x,x_0)<\delta$ we have: $$\|\vf{f}(x)-\vf{f}(x_0)\|<\varepsilon\quad\forall\vf{f}\in F$$
    We say that $F$ is \emph{pointwise equicontinuous} if it is equicontinuous at each point of $X$. Finally, we say that $F$ is \emph{uniformly equicontinuous} if $\forall \varepsilon>0$ $\exists \delta>0$ such that $\forall x,y\in X$ with $d(x,y)<\delta$ we have: $$\|\vf{f}(x)-\vf{f}(y)\|<\varepsilon\quad\forall\vf{f}\in F$$
  \end{definition}
  \begin{proposition}
    Let $(X,d)$ be a metric space and $F\subset\mathcal{C}_\text{b}(X,\RR^n)$ be a subset. Suppose that $\vf{f}$ is Lipschitz continuous for all $\vf{f}\in F$. Then, $F$ is uniformly equicontinuous.
  \end{proposition}
  \begin{theorem}[Arzelà-Ascoli theorem]
    Let $(X,d)$ be a compact metric space and $(\vf{f}_m)$ be a sequence of functions such that $\vf{f}_m\in\mathcal{C}(X,\RR^n)$ $\forall m\geq 1$. If the sequence is pointwise equicontinuous and pointwise bounded, then there exists a subsequence $(\vf{f}_{m_k})$ that converges on $\mathcal{C}(X,\RR^n)$.
  \end{theorem}
  \begin{corollary}
    Let $(X,d)$ be a compact metric space, $D\subset\RR^n$ be a closed set and $(\vf{f}_m)$ be a sequence of functions such that $\vf{f}_m\in\mathcal{C}(X,D)$ $\forall m\geq 1$. If the sequence is pointwise equicontinuous and pointwise bounded, then there exists a subsequence $(\vf{f}_{m_k})$ that converges on $\mathcal{C}(X,D)$.
  \end{corollary}
  \begin{theorem}[Peano theorem]
    Let $t_0\in\RR$, $\vf{x}_0\in\RR^n$, $a,b\in\RR_{>0}$, $\vf{f}:I_a(t_0)\times\overline{B}_{b}(\vf{x}_0)\subset\RR\times\RR^n\rightarrow\RR^n$ be a continuous function, and define: $$M:=\max\{\|\vf{f}(t,x)\|:(t,x)\in I_a(t_0)\times\overline{B}_{b}(\vf{x}_0)\}$$ Then, the ivp of \cref{DE_ivp} has at least one solution $\vf{\varphi}:I_\alpha(t_0)\rightarrow\RR^n$, where $\alpha:=\min\left\{a,\frac{b}{M}\right\}$.
  \end{theorem}
  \begin{corollary}
    Let $U\subseteq\RR\times\RR^n$ be an open set, $K\subset U$ be a compact set and $\vf{f}:U\rightarrow\RR^n$ be a continuous function. Then, $\exists\alpha\in\RR_{>0}$ such that $\forall (t_0,\vf{x}_0)\in K$, the ivp of \cref{DE_ivp} has a solution defined in $I_\alpha(t_0)$.
  \end{corollary}
  \subsubsection{Maximal solutions}
  \begin{definition}
    Let $U\subseteq\RR\times\RR^n$ be an open set, $(t_0,\vf{x}_0)\in U$ and $\vf{f}:U\rightarrow\RR^n$ be a continuous function. We define the set $\mathcal{S}(U,\vf{f},t_0,\vf{x}_0)$ as:
    \begin{multline*}
      \mathcal{S}(U,\vf{f},t_0,\vf{x}_0):=\{(I,\vf{\varphi}):I\subseteq\RR\text{ is an interval},t_0\in I\\\text{and }\vf{\varphi}:I\rightarrow\RR^n\text{ is a solution of the ivp of \cref{DE_ivp}}\}
    \end{multline*}
  \end{definition}
  \begin{definition}
    We define the relation $\leq$ defined on $\mathcal{S}(U,\vf{f},t_0,\vf{x}_0)$ in the following way. For $(I,\vf{\varphi}),(J,\vf{\psi})\in \mathcal{S}(U,\vf{f},t_0,\vf{x}_0)$: $$(J,\vf{\psi})\leq (I,\vf{\varphi})\iff J\subset I\text{ and }\vf{\varphi}|_J=\vf{\psi}\footnote{It can be seen that $\leq$ is a partial (but not total) order relation.}$$ In this case, we say that $(I,\vf{\varphi})$ is an \emph{extension} of $(J,\vf{\psi})$.
  \end{definition}
  \begin{definition}
    Let $(A,\leq )$ be a poset. Then, $m\in A$ is a \emph{maximal element} if and only if $\forall a\in A$ with $m \leq  a$ we have $m=a$.
  \end{definition}
  \begin{definition}
    Consider the poset $(\mathcal{S}(U,\vf{f},t_0,\vf{x}_0),\leq)$. We say that a solution $(I,\vf{\varphi})$ is \emph{maximal} if for all extensions $(J,\vf{\psi})$ of $(I,\vf{\varphi})$ we have $I=J$ and $\vf{\varphi}=\vf{\psi}$.
  \end{definition}
  \begin{definition}
    Let $(A,\leq )$ be a poset and $C\subseteq A$ be a subset of $A$. We say that $C$ is a \emph{chain} if it is totally ordered in the inherited order, that is, if it is partially ordered and $\forall x,y\in C$ we have either $x\leq y$ or $y\leq x$.
  \end{definition}
  \begin{definition}
    Let $(A,\leq )$ be a poset, $x\in A$ and $B\subseteq A$ be a subset. $x$ is an \emph{upper bound} of $B$ if and only if $b\leq x$ $\forall b\in B$.
  \end{definition}
  \begin{definition}
    Let $(A,\leq )$ be a poset and $B\subseteq A$ be a subset. Then, $g\in A$ is a \emph{greatest element} of $B$ if $g\in B$ and $\forall b\in B$ we have $b \leq  g$.
  \end{definition}
  \begin{lemma}[Zorn's lemma]
    Let $(A,\leq )$ be a poset. If every chain $C\subseteq A$ has an upper bound in $A$, then $A$ contains at least one maximal element.
  \end{lemma}
  \begin{theorem}
    Let $U\subseteq\RR\times\RR^n$ be an open set, $(t_0,\vf{x}_0)\in U$ and $\vf{f}:U\rightarrow\RR^n$ be a continuous function. Consider the poset $(\mathcal{S}(U,\vf{f},t_0,\vf{x}_0),\leq)$. Then, $\mathcal{S}(U,\vf{f},t_0,\vf{x}_0)$ has maximal elements. Furthermore, if $(I,\vf{\varphi})$ is a maximal solution, then $I$ is open.
  \end{theorem}
  \begin{proposition}
    Let $U\subseteq\RR\times\RR^n$ be an open set and $\vf{f}:U\rightarrow\RR^n$ be such that $\forall(t_0,\vf{x}_0)\in U$ the ivp of \cref{DE_ivp} has a unique solution defined in a neighbourhood of $t_0$. Then, $\forall(t_0,\vf{x}_0)\in U$ the ivp of \cref{DE_ivp} has a unique maximal solution.
  \end{proposition}
  \begin{lemma}[Wintner lemma]
    Let $U\subseteq\RR\times\RR^n$ be an open set, $\vf{f}:U\rightarrow\RR^n$ be a continuous function, $\vf{\varphi}:I\rightarrow\RR^n$ be a solution of $\vf{x}'=\vf{f}(t,\vf{x})$ and $(b,y)\in U$ be an accumulation point of $\vf{\varphi}$. Then, $\displaystyle\lim_{t\to b}\vf{\varphi}(t)=y$ and the solution can be extended up to $b$.
  \end{lemma}
  \begin{corollary}
    Let $U\subseteq\RR\times\RR^n$ be an open set, $\vf{f}:U\rightarrow\RR^n$ be a continuous function and $\vf{\varphi}:(a,b)\rightarrow\RR^n$ be a maximal solution of $\vf{x}'=\vf{f}(t,\vf{x})$. If $b<\infty$, then for all compact set $K\subset U$, $\exists t_0<\infty$ such that $(t,\vf{\varphi}(t))\notin K$ $\forall t\in[t_0,b)$. In that case, we say that $\vf{\varphi}$ \emph{tends to the boundary} of $U$.
  \end{corollary}
  \subsection{Linear differential equations}
  \begin{definition}
    Let $I\subseteq\RR$ be an interval. A \emph{system of linear differential equations} is an expression of the form:
    \begin{equation}\label{DE_linear}
      \vf{x}'=\vf{A}(t)\vf{x}+\vf{b}(t)
    \end{equation}
    where $\vf{A}:I\rightarrow\mathcal{L}(\RR^n,\RR^n)$ and $\vf{b}:I\rightarrow\RR^n$ are continuous functions.
    We say that linear equation of \cref{DE_linear} is \emph{homogeneous} if $\vf{b}(t)=\vf{0}$ $\forall t\in I$. We say that linear equation of \cref{DE_linear} is of \emph{constant coefficients} if $\vf{A}(t)=\vf{A}$ $\forall t\in I$, where $\vf{A}\in\mathcal{M}_n(\RR)$.
  \end{definition}
  \begin{definition}
    Let $I\subseteq\RR$ be an interval, $t_0\in I$, $\vf{x}_0\in\RR^n$ and consider the ode of \cref{DE_linear}. We define the \emph{flow of the linear ode} as the function:
    $$
      \function{\vf{\phi}}{I\times I\times \RR^n}{\RR^n}{(t,t_0,\vf{x}_0)}{\vf{\varphi}_{(t_0,\vf{x}_0)}(t)}
    $$
    where $\vf{\varphi}_{(t_0,\vf{x}_0)}$ is the solution of \cref{DE_linear} with initial conditions $(t_0,\vf{x}_0)$.
  \end{definition}
  \begin{proposition}
    Let $I\subseteq\RR$ be an interval and $a,b\in\mathcal{C}(I,\RR)$. Then, the general solution of the ivp
    $$\begin{cases}
        x'      =a(t)x+b(t) \\
        x(t_0)  =x_0
      \end{cases}$$
    is given by:
    \begin{equation}\label{DE_sol-lin}
      \varphi(t,t_0,x_0)=\exp{\int_{t_0}^ta(s)\dd{s}}\left(x_0+\int_{t_0}^tb(u)\exp{-\int_{t_0}^ua(s)\dd{s}}\dd{u}\right)
    \end{equation}
    for all $t\in I$.
  \end{proposition}
  \subsubsection{Homogeneous systems}
  \begin{theorem}
    Let $I\subseteq\RR$ be an interval and $\vf{A}\in\mathcal{C}(I,\mathcal{L}(\RR^n))$. We define $\mathcal{A}_n$ as the set of all solutions of the linear ode:
    \begin{equation}\label{DE_homo}
      \vf{x}'=\vf{A}(t)\vf{x}
    \end{equation} Then, $\mathcal{A}_n$ is a vector space of dimension $n$ and for each $t_0\in I$, the function
    $$
      \function{\vf{\xi}_{t_0}}{\RR^n}{\mathcal{A}_n}{\vf{x}_0}{\vf{\varphi}(\cdot,t_0,\vf{x}_0)}
    $$
    is an isomorphism.
  \end{theorem}
  \begin{corollary}
    Let $I\subseteq\RR$ be an interval, $t_0\in I$, $(\vf{v}_1,\ldots,\vf{v}_n)$ be a basis of $\RR^n$ and $\vf{\varphi}_1,\ldots,\vf{\varphi}_n\in\mathcal{A}_n$ be such that: $$\vf{\varphi}_i=\vf{\xi}_{t_0}(\vf{v}_i)\quad\text{for } i=1,\ldots,n$$
    Then, $(\vf{\varphi}_1,\ldots,\vf{\varphi}_n)$ is a basis of $\mathcal{A}_n$.
  \end{corollary}
  \begin{corollary}
    Let $I\subseteq\RR$ be an interval and $\vf{\psi}\in\mathcal{A}_n$. Suppose $\exists t_0\in I$ such that $\vf{\psi} (t_0)=0$. Then, $\vf{\psi}=\vf{0}$.
  \end{corollary}
  \begin{corollary}
    Let $I\subseteq\RR$ be an interval, $m,n\in\NN$ with $m\leq n$, $\vf{\varphi}_1,\ldots,\vf{\varphi}_m\in\mathcal{A}_n$ and $t_0\in I$ such that the vectors $\vf{\varphi}_1(t_0),\ldots,\vf{\varphi}_m(t_0)$ are linearly independent. Then, $\vf{\varphi}_1,\ldots,\vf{\varphi}_m$ are linearly independent.
  \end{corollary}
  \begin{corollary}
    Let $s,t,w\in\RR$. Consider the function
    $$
      \function{\vf{\phi}_s^t}{\RR^n}{\RR^n}{\vf{x}}{(\vf{\xi}_s(\vf{x}))(t)}
    $$
    Then, $\vf{\phi}_s^t$ is an isomorphism and satisfies:
    \begin{enumerate}
      \item $\vf{\phi}_s^s=\vf{\id}$
      \item $\vf{\phi}_s^t\circ\vf{\phi}_w^s=\vf{\phi}_w^t$
      \item ${\left[\vf{\phi}_s^t\right]}^{-1}=\vf{\phi}_t^s$
    \end{enumerate}
  \end{corollary}
  \begin{definition}
    Let $I\subseteq \RR$ be an interval, $\vf{A}\in\mathcal{C}(I,\mathcal{L}(\RR^n))$ and $\vf{M}(t)=(m_{ij}(t))\in\mathcal{M}_n(\RR)$. We say that $\vf{M}(t)$ is a \emph{matrix solution} of the ode of \cref{DE_homo} if $\vf{\varphi}_j={(m_{1j}(t),\ldots,m_{nj}(t))}^\mathrm{T}\in\mathcal{A}_n$ for $j=1,\ldots,n$. We say that $\vf{M}(t)$ is a \emph{fundamental matrix solution} of the ode of \cref{DE_homo} if $\vf{M}(t)$ is a matrix solution and $\vf{\varphi}_1,\ldots,\vf{\varphi}_n$ are linearly independent.
  \end{definition}
  \begin{proposition}
    Let $I\subseteq \RR$ be an interval, $\vf{A}\in\mathcal{C}(I,\mathcal{L}(\RR^n))$ and $\vf{M}(t)\in\mathcal{M}_n(\RR)$. Then:
    \begin{enumerate}
      \item $\vf{M}(t)$ is a matrix solution of the ode of \cref{DE_homo} $\iff\vf{M}'(t)=\vf{A}(t)\vf{M}(t)$\footnote{By definition, if $\vf{M}(t)=(m_{ij}(t))$, then $\vf{M}'(t):=({m_{ij}}'(t))$.}.
      \item $\vf{M}(t)$ is a matrix solution of the ode of \cref{DE_homo} $\iff\forall \vf{c}\in\RR^n$, $\vf{M}(t)\vf{c}\in\mathcal{A}_n$.
      \item If $\vf{M}(t)$ is a matrix solution of the ode of \cref{DE_homo}, then $\forall \vf{C}\in\mathcal{M}_n(\RR)$, $\vf{M}(t)\vf{C}$ is a matrix solution of the ode of \cref{DE_homo}.
      \item If $\vf{M}(t)$ is a fundamental matrix solution of the ode of \cref{DE_homo}, then $\det\vf{M}(t)\ne 0$ $\forall t\in I$.
      \item $\vf{M}(t)$ is a fundamental matrix solution of the ode of \cref{DE_homo} $\iff\vf{M}(t)$ is a matrix solution of the ode of \cref{DE_homo} and $\exists t_0\in I$ such that $\det\vf{M}(t_0)\ne 0$.
    \end{enumerate}
  \end{proposition}
  \begin{proposition}
    Let $I\subseteq \RR$ be an interval, $\vf{A}\in\mathcal{C}(I,\mathcal{L}(\RR^n))$ and $\vf{\Phi}(t),\vf{\psi}(t)\in\mathcal{M}_n(\RR)$ be matrix solutions of the ode of \cref{DE_homo} such that $\vf{\Phi}(t)$ is fundamental. Then, $\exists! \vf{C}\in\mathcal{M}_n(\RR)$ such that $\vf{\psi}(t)=\vf{\Phi}(t)\vf{C}$. Moreover, $\vf{\psi}(t)$ is fundamental if and only if $\det \vf{C}\ne 0$.
  \end{proposition}
  \subsubsection{Non-homogeneous linear systems}
  \begin{proposition}
    Let $I\subseteq \RR$ be an interval, $\vf{A}\in\mathcal{C}(I,\mathcal{L}(\RR^n))$ and $\vf{b}\in\mathcal{C}(I,\RR^n)$. Suppose $\vf{\phi}(t,t_0,\vf{x}_0)$ is the flow of the ode of \cref{DE_linear}. Then, $$\vf{\phi}(t,t_0,\vf{x}_0)=\Phi(t)\left[{\Phi(t_0)}^{-1}\vf{x}_0+\int_{t_0}^t{\Phi(s)}^{-1}\vf{b}(s)\dd{s}\right]$$ where $\Phi(t)$ is a fundamental matrix of the associated homogeneous system.
  \end{proposition}
  \begin{corollary}
    Let $I\subseteq \RR$ be an interval, $\vf{A}\in\mathcal{C}(I,\mathcal{L}(\RR^n))$ and $\vf{b}\in\mathcal{C}(I,\RR^n)$. Then, the general solution $\vf{\varphi}(t)$ of the ode of \cref{DE_homo} can be written as: $$\vf{\varphi}(t)=\vf{\varphi}_\mathrm{h}(t)+\vf{\varphi}_\mathrm{p}(t)$$ where $\vf{\varphi}_\mathrm{h}(t)$ is the general solution to the associated homogeneous system and $\vf{\varphi}_\mathrm{p}(t)$ is a particular solution of \cref{DE_homo}.
  \end{corollary}
  \begin{proposition}[Liouville's formula]
    Let $I\subseteq \RR$ be an interval, $\vf{A}\in\mathcal{C}(I,\mathcal{L}(\RR^n))$, $\vf{\Phi}(t)\in\mathcal{M}_n(\RR)$ be a matrix solution of the ode of \cref{DE_homo} and $t_0\in I$. Then, for all $t\in I$ we have: $$\det(\Phi(t))=\det (\Phi(t_0))\exp{\int_{t_0}^t\trace(\vf{A}(s))\dd{s}}$$
  \end{proposition}
  \subsubsection{Constant coefficients linear systems}
  \begin{lemma}
    Let $I\subseteq\RR$ be a compact interval and $\vf{f}:I\times\RR^n\rightarrow\RR^n$ be a continuous function and Lipschitz continuous with respect to the second variable. Let $\vf{\varphi}:I\rightarrow\RR^n$ be the solution of the ivp of \cref{DE_ivp}. Then, $\forall\vf{\psi}\in\mathcal{C}(I,\RR^n)$ the sequence $(\vf{T}^m\vf{\psi})$ converges uniformly to $\vf{\varphi}$ on $I$.
  \end{lemma}
  \begin{theorem}
    Let $\vf{A}\in\mathcal{M}_n(\RR)$ and $\vf{\Phi}(t)\in\mathcal{M}_n(\RR)$ be a matrix solution of the ode
    \begin{equation}\label{DE_coef-constants}
      \vf{x}'=\vf{A}\vf{x}
    \end{equation}
    such that $\Phi(0)=\vf{I}_n$. Then:
    \begin{enumerate}
      \item For all $t,s\in\RR$, then $\Phi(t+s)=\Phi(t)\Phi(s)$.
      \item ${\Phi(t)}^{-1}=\Phi(-t)$.
      \item The series $\displaystyle\sum_{k=0}^\infty \frac{\vf{A}^kt^k}{k!}$ converges uniformly on compact sets.
    \end{enumerate}
  \end{theorem}
  \begin{definition}
    Let $\vf{A}\in\mathcal{M}_n(\RR)$ and $t\in\RR$. We define the \emph{matrix exponential} $\exp{\vf{A}t}$ as: $$\exp{\vf{A}t}=\sum_{k=0}^\infty\frac{\vf{A}^kt^k}{k!}$$
  \end{definition}
  \begin{proposition}
    Let $\vf{A}\in\mathcal{M}_n(\RR)$ and $t,s\in \RR$. Then, the matrix exponential $\exp{\vf{A}t}$ is a fundamental matrix of the ode of \cref{DE_coef-constants} and has the following properties:
    \begin{enumerate}
      \item $\exp{\vf{A}\cdot 0}=\vf{I}_n$
      \item $\exp{\vf{A}(t+s)}=\exp{\vf{A}t}\exp{\vf{A}s}$
      \item ${\left(\exp{\vf{A}t}\right)}^{-1}=\exp{-\vf{A}t}$
      \item $\left(\exp{\vf{A}t}\right)'=\vf{A}\exp{\vf{A}t}=\exp{\vf{A}t}\vf{A}$
      \item If $\vf{\Phi}(t)$ is an arbitrary fundamental matrix of the ode of \cref{DE_coef-constants}, then: $$\exp{\vf{A}t}=\vf{\Phi}(t){\vf{\Phi}(0)}^{-1}$$
    \end{enumerate}
  \end{proposition}
  \begin{lemma}
    Let $\vf{A},\vf{B},\vf{C}\in\mathcal{M}_n(\RR)$. Then:
    \begin{enumerate}
      \item If $\vf{B}\vf{C}=\vf{C}\vf{A}$, then: $$\exp{\vf{B}t}\vf{C}=\vf{C}\exp{\vf{A}t}$$
      \item If $\vf{A}\vf{B}=\vf{B}\vf{A}$, then: $$\exp{\vf{A}t}\vf{B}=\vf{B}\exp{\vf{A}t}\quad\text{and}\quad\exp{(\vf{A}+\vf{B})t}=\exp{\vf{A}t}\exp{\vf{B}t}$$
    \end{enumerate}
  \end{lemma}
  \begin{corollary}
    Let $t\in\RR$, $\vf{A}\in\mathcal{M}_n(\RR)$ and $\vf{J}\in\mathcal{M}_n(\RR)$ be the Jordan form of $\vf{A}$ such that $\vf{A}=\vf{C}\vf{J}\vf{C}^{-1}$ for some matrix $\vf{C}\in\GL_n(\RR)$. Then: $$\exp{\vf{A}t}=\vf{C}\exp{\vf{J}t}\vf{C}^{-1}$$
  \end{corollary}
  \begin{proposition}
    Let $\vf{A}\in\mathcal{M}_n(\RR)$ and $t\in\RR$. If $\lambda$ is an eigenvalue of $\vf{A}$ with associated eigenvector $\vf{v}$, then $\exp{\lambda t}$ is an eigenvalue of $\exp{\vf{A}t}$ with associated eigenvector $\vf{v}$. That is, $\exp{\vf{A}t}\vf{v}=\exp{\lambda t}\vf{v}$. Hence, $\vf{\varphi}(t)=\exp{\lambda t}\vf{v}$ is a solution of the ivp:
    $$
      \begin{cases}
        \vf{x}'      =\vf{A}\vf{x} \\
        \vf{x}(0)  =\vf{v}
      \end{cases}
    $$
  \end{proposition}
  \begin{corollary}
    Let $\vf{A}\in\mathcal{M}_n(\RR)$ and $t\in\RR$ and consider the linear ode of \cref{DE_coef-constants}. If $(\vf{v}_1,\ldots,\vf{v}_n)$ is a basis of eigenvectors with associated eigenvalues $\lambda_1,\ldots,\lambda_n$, respectively, then $(\vf{\varphi}_1,\ldots,\vf{\varphi}_n)$, where $\vf{\varphi}_i=\exp{\lambda_it}\vf{v}_i$ for $i=1,\ldots,n$, is a basis of $\mathcal{A}_n$.
  \end{corollary}
  \begin{lemma}
    Let $\vf{A}=\diag(\lambda_1,\ldots,\lambda_n)\in\mathcal{M}_n(\RR)$ and $t\in\RR$. Then:
    $$\exp{\vf{A}t}=\diag(\exp{\lambda_1 t},\ldots,\exp{\lambda_n t})$$
  \end{lemma}
  \begin{proposition}
    Let $\vf{A}\in\mathcal{M}_n(\RR)$ and $\lambda=\alpha+\ii\beta\in\CC\setminus\RR$ be an eigenvalue of $\vf{A}$ with associated eigenvector $\vf{v}=\vf{u}+\ii\vf{w}\in\CC^n$. Then:
    \begin{multline*}
      \exp{\vf{A}t}\vf{v}=\exp{\vf{A}t}\vf{u}+\ii\exp{\vf{A}t}\vf{w}=\exp{\alpha t}\left[\cos(\beta t)\vf{u}-\sin(\beta t)\vf{w}\right]+\\+\ii\exp{\alpha t}\left[\sin(\beta t)\vf{u}+\cos(\beta t)\vf{w}\right]
    \end{multline*}
    and $\exp{\vf{A}t}\vf{u}$, $\exp{\vf{A}t}\vf{w}$ are linearly independent solutions of the ode of \cref{DE_coef-constants} with initial conditions $\vf{x}(0)=\vf{u}$ and $\vf{x}(0)=\vf{w}$, respectively.
  \end{proposition}
  \begin{definition}
    Let $\vf{A}\in\mathcal{M}_n(\RR)$. A vector $\vf{w}\in\RR^n$ is a \emph{generalized eigenvector} of rank $m$ of $\vf{A}$ corresponding to the eigenvalue $\lambda\in\RR$ if: $${(\vf{A}-\lambda\vf{I}_n)}^m\vf{w}=0\quad\text{but}\quad{(\vf{A}-\lambda\vf{I}_n)}^{m-1}\vf{w}\ne 0$$
    The set spanned by all generalized eigenvectors of $\lambda$ is called \emph{generalized eigenspace} of $\lambda$.
  \end{definition}
  \begin{proposition}
    Let $\vf{A}\in\mathcal{M}_n(\RR)$ and $\lambda\in\sigma(A)$. Then, the dimension of the generalized eigenspace is the algebraic multiplicity of $\lambda$.
  \end{proposition}
  \begin{lemma}
    Let $\vf{A}\in\mathcal{M}_n(\RR)$ and $\vf{v}_1\in\RR^n$ be an eigenvector of $\vf{A}$ with associated eigenvalue $\lambda$. We define $\vf{v}_2,\ldots,\vf{v}_m\in\RR^n$ in the following way: $$(\vf{A}-\lambda\vf{I}_n)\vf{v}_k=\vf{v}_{k-1}\qquad k=2,\ldots, m$$
    That is, $\vf{v}_k$ is a generalized eigenvector of rank $k$ of $\vf{A}$ with associated eigenvalue $\lambda$. Then,
    $$
      \left\{
      \begin{aligned}
        \vf{\varphi}_1 & =\exp{\lambda t}\vf{v}_1                                                                  \\
        \vf{\varphi}_2 & =\exp{\lambda t}\left(\vf{v}_2+t\vf{v}_1\right)                                           \\
        \vf{\varphi}_3 & =\exp{\lambda t}\left(\vf{v}_3+t\vf{v}_2+\frac{t^2}{2}\vf{v}_1\right)                     \\
                       & \;\;\vdots                                                                                \\
        \vf{\varphi}_m & =\exp{\lambda t}\left(\vf{v}_m+t\vf{v}_{m-1}+\cdots+\frac{t^{m-1}}{(m-1)!}\vf{v}_1\right) \\
      \end{aligned}
      \right.
    $$
    are solutions of the ode of \cref{DE_coef-constants}. Furthermore, if $\vf{v}_1,\ldots,\vf{v}_k$ are linearly independent, then so are $\vf{\varphi}_1,\ldots,\vf{\varphi}_k$.
  \end{lemma}
  \begin{corollary}
    Let $\vf{A}\in\mathcal{M}_n(\RR)$ and $\sigma(\vf{A})=\{\lambda_1,\ldots,\lambda_n\}$ be the spectrum of $\vf{A}$ such that:
    \begin{itemize}
      \item $\lambda_1,\ldots,\lambda_{2k}\in\CC\setminus\RR$, $\lambda_{k+i}=\overline{\lambda_i}$ and $\lambda_i=\alpha_i+\ii\beta_i$, $\alpha_i,\beta_i\in\RR$ for $i=1,\ldots,k$.
      \item $\lambda_{2k+1},\ldots,\lambda_n\in\RR$
    \end{itemize}
    Then, the general solution of the ode of \cref{DE_coef-constants} is of the form:
    \begin{multline*}
      \vf{\varphi}(t)=\sum_{i=1}^k\exp{\alpha_i t}\left(\vf{P}_i(t)\cos(\beta_i t)+\vf{Q}_i(t)\sin(\beta_i t)\right)+\\+\sum_{i=2k+1}^n\exp{\lambda_i t}\vf{R}_i(t)
    \end{multline*}
    where $\vf{P}_i,\vf{Q}_i,\vf{R}_i\in\RR^n[t]$ and $\deg \vf{P}_i,\deg\vf{Q}_i,\deg\vf{R}_i<n$ $\forall i$.
  \end{corollary}
  \subsection{Dependence on initial conditions and parameters}
  \begin{definition}
    Let $U\subseteq\RR\times\RR^n\times\RR^p$ be an open set and $\vf{f}:U\rightarrow\RR^n$ be a continuous function. Suppose that the ivp:
    \begin{equation}\label{DE_ivp_lambda}
      \begin{cases}
        \vf{x}'      =\vf{f}(t,\vf{x},\vf{\lambda}) \\
        \vf{x}(t_0)  =\vf{x}_0
      \end{cases}
    \end{equation}
    has a unique maximal solution $\vf{\varphi}_{(t_0,\vf{x}_0,\vf{\lambda})}(t)$ defined on an interval $I_{(t_0,\vf{x}_0,\vf{\lambda})}$. We define the \emph{flow} of the ode $\vf{x}'=\vf{f}(t,\vf{x},\vf{\lambda})$ as: $$\function{\vf{\phi}}{I_{(t_0,\vf{x}_0,\vf{\lambda})}\times\RR\times\RR^n\times\RR^p}{\RR^n}{(t,t_0,\vf{x}_0,\vf{\lambda})}{\vf{\varphi}_{(t_0,\vf{x}_0,\vf{\lambda})}(t)}$$
  \end{definition}
  \subsubsection{Continuous and Lipschitz continuous dependence}
  \begin{lemma}
    Let $X$ be a compact metric space and $(\vf{\varphi}_m)$ be a sequence of functions $\vf{\varphi}_m:X\rightarrow\RR^n$ such that they are pointwise equicontinuous and pointwise bounded. Suppose that all convergent partial subsequences of $(\vf{\varphi}_m)$ have the same limit $\vf{\varphi}$. Then, $(\vf{\varphi}_m)$ converges uniformly to $\vf{\varphi}$.
  \end{lemma}
  \begin{proposition}
    Let $U\subseteq\RR\times\RR^n$ be an open set and $\vf{f}_m:U\rightarrow\RR^n$ be continuous function for $m\in\NN$ and such that for all compact $K\subset U$, the sequence $(\vf{f}_m|_K)$ converge uniformly to a function $\vf{f}_0|_K$. Let $((t_m,\vf{x}_m))\subset U$ be a sequence such that $\displaystyle\lim_{m\to\infty}(t_m,\vf{x}_m)=(t_0,\vf{x}_0)$. Suppose that for all $m\geq 0$ the ivp
    \begin{equation*}
      \begin{cases}
        \vf{x}'      =\vf{f}_m(t,\vf{x}) \\
        \vf{x}(t_m)  =\vf{x}_m
      \end{cases}
    \end{equation*}
    has a unique maximal solution $\vf{\varphi}_m$ defined on $I_m$. Then, for all $[a,b]\subset I_0$ with $t_0\in(a,b)$, $\exists m_0\in\NN$ such that $[a,b]\subset I_m$ $\forall m>m_0$. Furthermore, the sequence $\left(\vf{\varphi}_m|_{[a,b]}\right)_{m>m_0}$ converges uniformly to $\vf{\varphi}_0|_{[a,b]}$.
  \end{proposition}
  \begin{theorem}
    Let $U\subseteq\RR\times\RR^n$ be an open set and $\vf{f}:U\rightarrow\RR^n$ be a continuous function. Suppose that each ivp of the form of \cref{DE_ivp} has a unique maximal solution. Then, the flow $\vf{\phi}(t,t_0,\vf{x}_0)$ is a continuous function defined in an open set.
  \end{theorem}
  \begin{theorem}
    Let $U\subseteq\RR\times\RR^n\times\RR^p$ be an open set and $\vf{f}:U\rightarrow\RR^n$ be a continuous function. Suppose that each ivp of the form of \cref{DE_ivp_lambda} has a unique maximal solution. Then, the flow $\vf{\phi}(t,t_0,\vf{x}_0,\vf{\lambda})$ is a continuous function defined in an open set $V\subseteq I_{(t_0,\vf{x}_0,\vf{\lambda})}\times\RR\times\RR^n\times\RR^p$.
  \end{theorem}
  \begin{lemma}[Grönwall's lemma]
    Let $u,v,w:[a,b)\rightarrow\RR$ be continuous functions such that $v(t)\geq 0$ $\forall t\in[a,b)$ and satisfying: $$u(t)\leq w(t)+\int_a^tv(s)u(s)\dd{s}\quad\forall t\in[a,b)$$
    Then: $$u(t)\leq w(t)+\int_a^tw(s)v(s)\exp{\int_s^tv(r)\dd{r}}\dd{s}\quad\forall t\in[a,b)$$
    If, moreover, $w\in\mathcal{C}^1((a,b))$, then: $$u(t)\leq w(a)\exp{\int_a^tv(r)\dd{r}}+\int_a^tw'(s)\exp{\int_s^tv(r)\dd{r}}\dd{s}\quad\forall t\in[a,b)$$
  \end{lemma}
  \begin{proposition}
    Let $U\subseteq\RR\times\RR^n$ be an open set and $\vf{f}:U\rightarrow\RR^n$ be a continuous function and Lipschitz continuous with respect to the second variable with Lipschitz constant $L$. Let $\vf{\phi}$ be the flow of the ode $\vf{x}'=\vf{f}(t,\vf{x})$. Then, $\forall (t_0,\vf{x}_1),(t_0,\vf{x}_2)\in U$ and $\forall t\in I_{(t_0,\vf{x}_1)}\cap I_{(t_0,\vf{x}_2)}$, we have:
    $$\|\vf{\phi}(t,t_0,\vf{x}_2)-\vf{\phi}(t,t_0,\vf{x}_1)\|\leq\exp{L|t-t_0|}\|\vf{x}_2-\vf{x}_1\|$$
    Thus, $\vf{\phi}$ is locally Lipschitz continuous with respect to the third variable.
  \end{proposition}
  \subsubsection{Differentiable dependence}
  \begin{theorem}[Dependence on $\vf{x}_0$]
    Let $U\subseteq\RR\times\RR^n\times\RR^p$ be an open set and $\vf{f}:U\rightarrow\RR^n$ be a continuous function and of class $\mathcal{C}^1$ with respect to the second variable. Suppose that the flow $\vf{\phi}(t,t_0,\vf{x}_0,\vf{\lambda})$ of $\vf{x}'=\vf{f}(t,\vf{x},\vf{\lambda})$ is defined on an open set $V\subseteq\RR\times\RR\times\RR^n\times\RR^p$. Then, $\forall (t,t_0,\vf{x}_0,\vf{\lambda})\in V$, $\vf{\phi}$ is differentiable with respect to $\vf{x}_0$ and $\vf{D}_3\vf{\phi}(t,t_0,\vf{x}_0,\vf{\lambda})$\footnote{Here, $\vf{D}_3\vf{\phi}(t,t_0,\vf{x}_0,\vf{\lambda})=\frac{\partial\vf{\phi}}{\partial \vf{x}_0}(t,t_0,\vf{x}_0,\vf{\lambda})$ denotes the matrix $\left(\frac{\partial\vf{\phi}_i}{\partial{\vf{x}_0}_j}(t,t_0,\vf{x}_0,\vf{\lambda})\right)\in\mathcal{M}_n(\RR)$, where ${\vf{x}_0}_j$ denotes the $j$-th component of $\vf{x}_0$ and $\vf{\phi}_i$ denotes the $i$-th component of $\vf{\phi}$.} is continuous on $V$. Furthermore, $\vf{D}_3\vf{\phi}(t,t_0,\vf{x}_0,\vf{\lambda})$ satisfies the following ivp:
    \begin{equation*}
      \begin{cases}
        \vf{M}'      =\vf{D}_2\vf{f}(t,\vf{\phi}(t,t_0,\vf{x}_0,\vf{\lambda}),\vf{\lambda})\vf{M} \\
        \vf{M}(t_0)  =\vf{I}_n
      \end{cases}
    \end{equation*}
    Or, equivalently, $\frac{\partial\vf{\phi}}{\partial {\vf{x}_0}_i}(t,t_0,\vf{x}_0,\vf{\lambda})=\vf{D}_3\vf{\phi}(t,t_0,\vf{x}_0,\vf{\lambda})\vf{e}_i$ satisfies the following ivp:
    $$
      \begin{cases}
        \vf{y}'      =\vf{D}_2\vf{f}(t,\vf{\phi}(t,t_0,\vf{x}_0,\vf{\lambda}),\vf{\lambda})\vf{y} \\
        \vf{y}(t_0)  =\vf{e}_i
      \end{cases}\quad\text{for }i=1,\ldots,n
    $$
    These kinds of equations are called \emph{variational equations}.
  \end{theorem}
  \begin{theorem}[Dependence on $t_0$]
    Let $U\subseteq\RR\times\RR^n\times\RR^p$ be an open set and $\vf{f}:U\rightarrow\RR^n$ be a continuous function and of class $\mathcal{C}^1$. Suppose that the flow $\vf{\phi}(t,t_0,\vf{x}_0,\vf{\lambda})$ of $\vf{x}'=\vf{f}(t,\vf{x},\vf{\lambda})$ is defined on an open set $V\subseteq\RR\times\RR\times\RR^n\times\RR^p$. Then, $\forall (t,t_0,\vf{x}_0,\vf{\lambda})\in V$, $\vf{\phi}$ is differentiable with respect to $t_0$ and $\vf{D}_2\vf{\phi}(t,t_0,\vf{x}_0,\vf{\lambda})$ is continuous on $V$. Furthermore, $\vf{D}_2\vf{\phi}(t,t_0,\vf{x}_0,\vf{\lambda})$ satisfies the following ivp:
    \begin{equation*}
      \begin{cases}
        \vf{y}'      =\vf{D}_2\vf{f}(t,\vf{\phi}(t,t_0,\vf{x}_0,\vf{\lambda}),\vf{\lambda})\vf{y} \\
        \vf{y}(t_0)  = -\vf{f}(t_0,\vf{x}_0,\vf{\lambda})
      \end{cases}
    \end{equation*}
  \end{theorem}
  \begin{theorem}[Dependence on $\vf{\lambda}$]
    Let $U\subseteq\RR\times\RR^n\times\RR^p$ be an open set and $\vf{f}:U\rightarrow\RR^n$ be a continuous function and of class $\mathcal{C}^1$ with respect to the second and third variable. Suppose that the flow $\vf{\phi}(t,t_0,\vf{x}_0,\vf{\lambda})$ of $\vf{x}'=\vf{f}(t,\vf{x},\vf{\lambda})$ is defined on an open set $V\subseteq\RR\times\RR\times\RR^n\times\RR^p$. Then, $\forall (t,t_0,\vf{x}_0,\vf{\lambda})\in V$, $\vf{\phi}$ is differentiable with respect to $\vf{\lambda}$ and $\vf{D}_4\vf{\phi}(t,t_0,\vf{x}_0,\vf{\lambda})$ is continuous on $V$. Furthermore, $\vf{D}_4\vf{\phi}(t,t_0,\vf{x}_0,\vf{\lambda})$ satisfies the following ivp:
    \begin{equation*}
      \begin{cases}
        \vf{M}'      =\vf{D}_2\vf{f}(t,\vf{\phi}(t,t_0,\vf{x}_0,\vf{\lambda}),\vf{\lambda})\vf{M}+ \vf{B} \\
        \vf{M}(t_0)  =\vf{0}
      \end{cases}
    \end{equation*}
    where $\vf{B}=\vf{D}_3\vf{f}(t,\vf{\phi}(t,t_0,\vf{x}_0,\vf{\lambda}),\vf{\lambda})$. Equivalently, $\frac{\partial\vf{\phi}}{\partial \vf{\lambda}_i}(t,t_0,\vf{x}_0,\vf{\lambda})$ satisfies the ivp:
    $$
      \begin{cases}
        \vf{y}'      =\vf{D}_2\vf{f}(t,\vf{\phi}(t,t_0,\vf{x}_0,\vf{\lambda}),\vf{\lambda})\vf{y}+\vf{B}\vf{e}_i \\
        \vf{y}(t_0)  =\vf{0}
      \end{cases}\;\text{for }i=1,\ldots,n
    $$
  \end{theorem}
  \subsubsection{Higher order dependence}
  \begin{theorem}
    Let $U\subseteq\RR\times\RR^n$ be an open set and $\vf{f}:U\rightarrow\RR^n$ be a continuous function and of class $\mathcal{C}^k$, $k\in\NN$. Suppose that the flow $\vf{\phi}(t,t_0,\vf{x}_0)$ of $\vf{x}'=\vf{f}(t,\vf{x})$ is defined on an open set $V\subseteq\RR\times\RR\times\RR^n$. Then, $\vf{\phi}(t,t_0,\vf{x}_0)$ is of class $\mathcal{C}^k$ on $V$.
  \end{theorem}
  \subsection{Qualitative theory of autonomous systems}
  \subsubsection{Introduction to dynamical systems}
  \begin{definition}
    A \emph{dynamical system} is a triplet $(G,X,\Psi)$, where $G$ is a topological abelian group\footnote{That is, $G$ is an abelian group with an inherited topological structure.}, $X$ is a topological space and $\Psi:G\times X\rightarrow X$ is a function such that:
    \begin{itemize}
      \item $\Psi$ is continuous.
      \item $\Psi(0,x)=x$ $\forall x\in X$.
      \item $\Psi(t,\Psi(s,x))=\Psi(t+s,x)$ $\forall s,t\in G$ and $\forall x\in X$.
    \end{itemize}
    We say that a dynamical system $(G,X,\Psi)$ is \emph{discrete} if $G=\ZZ$ and we say that it is \emph{continuous} if $G=\RR$.
  \end{definition}
  \begin{definition}
    Let $(G,X,\Psi)$ be a dynamical system and $x\in X$. The \emph{orbit} through $x$ is defined as: $$\gamma(x)=\gamma_{\Psi}(x):=\{\Psi(t,x):t\in G\}\footnote{In general, if the context is clear we will use the notation $\gamma(x)$ instead of $\gamma_{\Psi}(x)$.}$$ Moreover if $G=\ZZ$ or $G=\RR$ we define the \emph{positive semi-orbit} through $x$ and the \emph{negative semi-orbit} through $x$ as the following respective sets:
    \begin{gather*}
      \gamma^+(x)={\gamma_{\Psi}}^+(x):=\{\Psi(t,x):t\in G_{\geq 0}\}\\
      \gamma^-(x)={\gamma_{\Psi}}^-(x):=\{\Psi(t,x):t\in G_{\leq 0}\}
    \end{gather*}
  \end{definition}
  \begin{definition}
    Let $(G,X,\Psi)$ be a dynamical system. Then, we have an equivalence relation $\sim$ on $X$ given by $$x\sim y\iff\gamma(x)=\gamma(y)\quad\forall x,y\in X$$
    which creates a partition of $X$, called \emph{phase portrait}.
  \end{definition}
  \begin{definition}
    The \emph{phase space} of an ode or system of odes is the space in which all possible states of a system are represented with each possible state corresponding to one unique point in the phase space.
  \end{definition}
  \begin{center}
    \begin{minipage}[b]{0.475\linewidth}
      \centering
      \includestandalone[mode=image|tex,width=\linewidth]{Images/phase_space}
      \captionof{figure}{Phase space of the system $\{x'=-y,y'=x:(x(0),y(0))=(r,0)\}$.}
    \end{minipage}\hfill
    \begin{minipage}[b]{0.475\linewidth}
      \centering
      \includestandalone[mode=image|tex,width=\linewidth]{Images/vector_field}
      \captionof{figure}{Vector field of the system $\{x'=x,y'=x+y\}$ together with two orbits.}
    \end{minipage}
  \end{center}
  \begin{center}
    \begin{minipage}{\linewidth}
      \centering
      \includestandalone[mode=image|tex,width=0.75\linewidth]{Images/phase_portrait}
      \captionof{figure}{Phase portrait of the system $\{x'=x/2,y'=x+y/2\}$.}
    \end{minipage}
  \end{center}
  \begin{definition}
    Let $(G,X,\Psi)$ be a dynamical system and $x\in X$. We define the following function: $$\function{\Psi_x}{G}{\gamma(x)}{t}{\Psi(t,x)}$$
  \end{definition}
  \begin{lemma}
    Let $\vf{f}:\RR^n\rightarrow\RR^n$ be a continuous function such that the flow $\vf{\phi}(t,t_0,\vf{x}_0)$ of the ode $\vf{x}' =\vf{f}(\vf{x})$ is defined for all $t\in\RR$. Then, $(\RR,\RR^n,\vf{\Psi})$ is a dynamical system, where $\vf{\Psi}(t,\vf{x})=\vf{\phi}(t,0,\vf{x})$. Furthermore, note that $\vf{\gamma}(\vf{x})=\im(\vf{\phi}(\cdot,0,\vf{x}))$.
  \end{lemma}
  \begin{lemma}
    Let $\vf{f}:\RR^n\rightarrow\RR^n$ be a continuous function such that $\exists M,N\in\RR_{\geq 0}$ with $\|\vf{f}(\vf{x})\|\leq M\|\vf{x}\|+N$. Then, the solutions of the ode $\vf{x}' =\vf{f}(\vf{x})$ are defined for all $t\in\RR$.
  \end{lemma}
  \begin{definition}
    Let $\vf{f},\vf{g}:\RR^n\rightarrow\RR^n$ be continuous functions and $\vf{x}' =\vf{f}(\vf{x})$, $\vf{x}' =\vf{g}(\vf{x})$ be two odes for which we have existence and uniqueness of solutions. We say that these two odes are \emph{equivalent} if there exists $\vf{h}:\RR^n\rightarrow\RR^n$ such that $\vf{h}(\vf{x})\geq 0$ and $\vf{f}(\vf{x})=\vf{h}(\vf{x})\vf{g}(\vf{x})$ $\forall \vf{x}\in\RR^n$. Therefore, $\vf{f}$ and $\vf{g}$ have the same orbits oriented in the same way.
  \end{definition}
  \begin{corollary}
    Let $\vf{f}:\RR^n\rightarrow\RR^n$ be a continuous function such that the ode $\vf{x}' =\vf{f}(\vf{x})$ has existence and uniqueness of solutions for all initial conditions. Then, there exists a continuous function $\vf{g}:\RR^n\rightarrow\RR^n$ such that the autonomous odes induced by $\vf{f}$ and $\vf{g}$ are equivalent and the flow of the ode $\vf{x}' =\vf{g}(\vf{x})$ is defined $\forall t\in\RR$.
  \end{corollary}
  \begin{lemma}
    Let $H$ be a proper subgroup of $\RR$ which is closed. Then, $\exists T\in\RR_{\geq 0}$ such that $H=T\ZZ$.
  \end{lemma}
  \begin{proposition}
    Let $(\RR,\RR^n,\vf{\Psi})$ be a dynamical system and $\vf{\gamma}(\vf{x})$ be an orbit. Then, there are 3 possible cases for $\vf{\gamma}(\vf{x})$:
    \begin{enumerate}
      \item $\vf{\gamma}(\vf{x})=\{\vf{x}\}$.
      \item $\vf{\gamma}(\vf{x})\cong S^1$.
      \item $\vf{\gamma}(\vf{x})$ is homeomorphic to an injective and continuous image of $\RR$.
    \end{enumerate}
  \end{proposition}
  \begin{definition}
    Let $(\RR,\RR^n,\vf{\Psi})$ be a dynamical system and $\vf{p}\in\RR^n$. We say that $\vf{p}\in\RR^n$ is a \emph{critical point} or \emph{singular point} if $\vf{\gamma}(\vf{p})=\{\vf{p}\}$. Otherwise, we say that $\vf{p}$ is \emph{non-singular} or \emph{regular}.
  \end{definition}
  \begin{definition}
    Let $(\RR,\RR^n,\vf{\Psi})$ be a dynamical system and $\vf{\gamma}(\vf{x})$ be an orbit of $(\RR,\RR^n,\vf{\Psi})$. We say that $\vf{\gamma}(\vf{x})$ is \emph{periodic} of period $T>0$ if $\vf{\gamma}(\vf{x})\cong S^1$ and $\ker\vf{\Psi_{\vf{x}}}=T\ZZ$.
  \end{definition}
  \begin{proposition}
    Let $(\RR,\RR^n,\vf{\Psi})$ be a dynamical system such that $\vf{\Psi}(t,\vf{x})=\vf{\phi}(t,0,\vf{x})$, where $\vf{\phi}(t,t_0,\vf{x}_0)$ is the flow of the ode $\vf{x}' =\vf{f}(\vf{x})$. Let $\vf{p}\in\RR^n$. Then, the following statements are equivalent:
    \begin{enumerate}
      \item $\{\vf{p}\}$ is a critical point.
      \item $\vf{\phi}(t,0,\vf{p})=\vf{p}$.
      \item $\vf{f}(\vf{p})=0$.
    \end{enumerate}
  \end{proposition}
  \begin{definition}
    Let $(\RR,\RR^n,\vf{\Psi})$ be a dynamical system and $\vf{x}\in\RR^n$. We say that $\vf{y}\in\RR^n$ is an \emph{$\alpha$-limit point} of $\vf{x}$ if there exists a sequence $(t_n)\subset\RR$ such that $\displaystyle\lim_{n\to\infty}t_n=-\infty$ and $\displaystyle\lim_{n\to\infty}\vf{\Psi}(t_n,\vf{x})=\vf{y}$.
    The set of all $\alpha$-limit points of $\vf{x}$ is called \emph{$\alpha$-limit set}, and it is denoted by $\alpha(\vf{x})$. For an orbit $\vf{\gamma}$ of $(\RR,\RR^n,\vf{\Psi})$, we say that $\vf{y}$ is an \emph{$\omega$-limit point} of $\vf{\gamma}$, it is a $\omega$-limit point of some point on the orbit $\vf{\gamma}$. The set of such $\alpha$-limit points will be denoted as $\alpha(\vf{\gamma})$.
  \end{definition}
  \begin{definition}
    Let $(\RR,\RR^n,\vf{\Psi})$ be a dynamical system and $\vf{x}\in\RR^n$. We say that $\vf{y}\in\RR^n$ is an \emph{$\omega$-limit point} of $\vf{x}$ if there exists a sequence $(t_n)\subset\RR$ such that $\displaystyle\lim_{n\to\infty}t_n=+\infty$ and $\displaystyle\lim_{n\to\infty}\vf{\Psi}(t_n,\vf{x})=\vf{y}$.
    The set of all $\omega$-limit points of $\vf{x}$ is called \emph{$\omega$-limit set}, and it is denoted by $\omega(\vf{x})$. For an orbit $\vf{\gamma}$ of $(\RR,\RR^n,\vf{\Psi})$, we say that $\vf{y}$ is an \emph{$\omega$-limit point} of $\vf{\gamma}$, it is a $\omega$-limit point of some point on the orbit $\vf{\gamma}$. The set of such $\omega$-limit points will be denoted as $\omega(\vf{\gamma})$.
  \end{definition}
  \begin{proposition}
    Let $(\RR,\RR^n,\vf{\Psi})$ be a dynamical system and $\vf{x}\in\RR^n$. Then: $$\Cl(\vf{\gamma}(\vf{x}))=\alpha(\vf{x})\cup\vf{\gamma}(\vf{x})\cup\omega(\vf{x})$$
  \end{proposition}
  \begin{definition}
    Let $(\RR,\RR^n,\vf{\Psi})$ be a dynamical system and $A\subseteq \RR^n$ be a subset. We say that $A$ is \emph{invariant} if $\vf{\gamma}(\vf{x})\subseteq A$ $\forall\vf{x}\in A$. We say that $A$ is \emph{positively invariant} if ${\vf{\gamma}}^+(\vf{x})\subseteq A$ $\forall\vf{x}\in A$. Analogously, we say that $A$ is \emph{negatively invariant} if ${\vf{\gamma}}^-(\vf{x})\subseteq A$ $\forall\vf{x}\in A$.
  \end{definition}
  \begin{proposition}
    Let $(\RR,\RR^n,\vf{\Psi})$ be a dynamical system, $\vf{p}\in \RR^n$ and $\vf\gamma$ be an orbit of the system such that ${\vf{\gamma}}^+$ is contained in a compact set. Then:
    \begin{itemize}
      \item $\omega(\vf{p})\ne\varnothing$.
      \item $\omega(\vf{p})$ is compact.
      \item $\omega(\vf{p})$ is invariant.
      \item $\omega(\vf{p})$ is connected.
      \item If $\omega(\vf{\gamma})\subseteq\vf{\gamma}\implies\omega(\vf{\gamma})=\vf{\gamma}$, then $\vf{\gamma}$ is either a critical point or a period orbit.
    \end{itemize}
  \end{proposition}
  \begin{proposition}
    Let $(\RR,\RR^n,\vf{\Psi})$ be a dynamical system, $\vf{p}\in \RR^n$ and $\vf\gamma$ be an orbit of the system such that ${\vf{\gamma}}^-$ is contained in a compact set. Then:
    \begin{itemize}
      \item $\alpha(\vf{p})\ne\varnothing$.
      \item $\alpha(\vf{p})$ is compact.
      \item $\alpha(\vf{p})$ is invariant.
      \item $\alpha(\vf{p})$ is connected.
      \item If $\alpha(\vf{\gamma})\subseteq\vf{\gamma}\implies\alpha(\vf{\gamma})=\vf{\gamma}$, then $\vf{\gamma}$ is either a critical point or a period orbit.
    \end{itemize}
  \end{proposition}
  \begin{definition}
    Let $(\RR,\RR^n,\vf{\Psi})$ be a dynamical system and $K\subset \RR^n$ be a compact set. We say that $K$ is \emph{positively stable} if for all neighbourhood $U$ of $K$, there exists a neighbourhood $V$ of $K$ with $V\subseteq U$ and such that $\forall \vf{x}\in V$, ${\vf{\gamma}}^+(\vf{x})\subset U$. Analogously, we say that $K$ is \emph{negatively stable} if for all neighbourhood $U$ of $K$, there exists a neighbourhood $V$ of $K$ with $V\subseteq U$ and such that $\forall \vf{x}\in V$, ${\vf{\gamma}}^-(\vf{x})\subset U$.
  \end{definition}
  \begin{definition}
    Let $(\RR,\RR^n,\vf{\Psi})$ be a dynamical system and $K\subset \RR^n$ be a compact set. We say that $K$ is \emph{attracting} if there exists a neighbourhood $U$ of $K$ such that $\forall \vf{x}\in U$, $\omega(\vf{x})\subset K$.  We say that $K$ is \emph{repelling} if there exists a neighbourhood $U$ of $K$ such that $\forall \vf{x}\in U$, $\alpha(\vf{x})\subset K$. We say that $K$ is \emph{asymptotically stable} if it is both attracting and positively stable.
  \end{definition}
  \begin{proposition}
    Let $(\RR,\RR^n,\vf{\Psi})$ be a dynamical system and $K\subset \RR^n$ be a compact set. Suppose that $K$ is positively stable. Then, $K$ is positively invariant.
  \end{proposition}
  \begin{definition}
    Let $(\RR,\RR^n,\vf{\Psi})$ be a dynamical system and $\vf{p}\in\RR^n$. We say that $\vf{p}$ is a \emph{center} if there exists a neighbourhood $U$ of $\vf{p}$ such that if $\vf{\gamma}(\vf{x})\subset U$, then $\vf{\gamma}(\vf{x})$ is periodic. The largest neighbourhood with this property is called \emph{basin} of the center.
  \end{definition}
  \begin{proposition}
    Let $(\RR,\RR^n,\vf{\Psi})$ be a dynamical system and $\vf{p}\in\RR^n$ be a center. Then:
    \begin{enumerate}
      \item $\vf{p}$ is positively and negatively stable.
      \item $\vf{p}$ is not attracting.
    \end{enumerate}
  \end{proposition}
  \subsubsection{Equivalence and conjugacy of dynamical systems}
  \begin{definition}
    Let $(G,X,\Psi_1)$ and $(G,X,\Psi_2)$ be dynamical systems and $r\in\NN\cup\{0,\infty\}$. We say that $(G,X,\Psi_1)$ and $(G,X,\Psi_2)$ are \emph{equivalent dynamical systems} of class $\mathcal{C}^r$ if there exists a diffeomorphism $h:X\rightarrow Y$ of class $\mathcal{C}^r$ such that $\forall x\in X$, $h(\gamma_{\Psi_1}(x))=\gamma_{\Psi_2}(h(x))$ and preserving the orientation of the orbits. In particular, if $r=0$ we say that $(G,X,\Psi_1)$ and $(G,X,\Psi_2)$ are \emph{topologically equivalent}. That diffeomorphism $h$ is called an \emph{equivalence} (of class $\mathcal{C}^r$) between $(G,X,\Psi_1)$ and $(G,X,\Psi_2)$.
  \end{definition}
  \begin{definition}
    Let $(G,X,\Psi_1)$ and $(G,X,\Psi_2)$ be dynamical systems and $r\in\NN\cup\{0,\infty\}$. We say that $(G,X,\Psi_1)$ and $(G,X,\Psi_2)$ are \emph{conjugate dynamical systems} of class $\mathcal{C}^r$ if there exists a diffeomorphism $h:X\rightarrow Y$ of class $\mathcal{C}^r$ such that $\forall (t,x)\in G\times X$, $h(\Psi_1(t,x))=\Psi_2(t,h(x))$. In particular, if $r=0$ we say that $(G,X,\Psi_1)$ and $(G,X,\Psi_2)$ are \emph{topologically conjugate}. That diffeomorphism $h$ is called a \emph{conjugacy} (of class $\mathcal{C}^r$) between $(G,X,\Psi_1)$ and $(G,X,\Psi_2)$.
  \end{definition}
  \begin{proposition}
    Let $(G,X,\Psi_1)$ and $(G,X,\Psi_2)$ be dynamical systems and $h$ be a conjugacy of class $\mathcal{C}^r$ between them. Then, $h$ is an equivalence of class $\mathcal{C}^r$ between $(G,X,\Psi_1)$ and $(G,X,\Psi_2)$.
  \end{proposition}
  \begin{proposition}
    Two dynamical systems induced by two equivalent odes are equivalent (as a dynamical systems).
  \end{proposition}
  \begin{proposition}
    Let $(G,X,\Psi_1)$ and $(G,X,\Psi_2)$ be dynamical systems and $h:X\rightarrow Y$ be an equivalence of class $\mathcal{C}^r$ between them. Then:
    \begin{enumerate}
      \item $h$ preserves the type of orbit. More precisely if $p\in X$, we have:
            \begin{enumerate}
              \item If $p$ is a critical point, then so it is $h(p)$.
              \item If $\gamma(p)$ is a periodic orbit, then so it is $h(\gamma(p))$\footnote{Note that the period of $\gamma(p)$ and $h(\gamma(p))$ may be different.}.
              \item If $\gamma(p)$ is the injective and continuous image of $\RR$, then so it is $h(\gamma(p))$.
            \end{enumerate}
      \item If $p\in X$ is a critical point of $\Psi_1$, we have:
            \begin{enumerate}
              \item If $p$ is attracting for $(G,X,\Psi_1)$, then so it is $h(p)$ for $(G,X,\Psi_2)$.
              \item If $p$ is repelling for $(G,X,\Psi_1)$, then so it is $h(p)$ for $(G,X,\Psi_2)$.
              \item If $p$ is positively stable for $(G,X,\Psi_1)$, then so it is $h(p)$ for $(G,X,\Psi_2)$.
              \item If $p$ is asymptotically stable for $(G,X,\Psi_1)$, then so it is $h(p)$ for $(G,X,\Psi_2)$.
            \end{enumerate}
    \end{enumerate}
  \end{proposition}
  \begin{proposition}
    A conjugacy between two dynamical systems preserves the period of periodic orbits.
  \end{proposition}
  \begin{proposition}
    Let $\alpha,\beta\in\RR$ such that $\alpha\beta>0$. Consider the function $h:\RR\rightarrow\RR$ defined by: $$h(x)=
      \begin{cases}
        x^{\beta/\alpha}      & \text{if }x\geq 0 \\
        -{|x|}^{\beta/\alpha} & \text{if }x< 0
      \end{cases}$$
    Then, $h$ is a topological conjugation between the systems induced by the odes $x'=\alpha x$ and $y'=\beta y$.
  \end{proposition}
  \begin{proposition}
    Let $\vf{A},\vf{B}\in\mathcal{M}_n(\RR)$ be similar matrices, that is, $\exists\vf{P}\in\mathcal{M}_n(\RR)$ such that $\vf{B}=\vf{P}\vf{A}\vf{P}^{-1}$. Then, the function $$\function{\vf{h}}{\RR^n}{\RR^n}{\vf{x}}{\vf{Px}}$$ is a conjugation between the systems induced by the odes $\vf{x}'=\vf{A}\vf{x}$ and $\vf{y}'=\vf{B}\vf{y}$.
  \end{proposition}
  \subsubsection{Local equivalence and conjugacy of dynamical systems}
  \begin{definition}
    Let $U\subseteq \RR^n$ be an open set and $\vf{f}:U\rightarrow\RR^n$ be a vector field of class $\mathcal{C}^1$. For all $\vf{x}_0\in U$, let $\vf{\varphi}_{\vf{x}_0}(t)$ be the maximal solution to the ivp
    $$
      \begin{cases}
        \vf{x}'      =\vf{f}(t,\vf{x}) \\
        \vf{x}(0)  =\vf{x}_0
      \end{cases}
    $$
    We define the \emph{flow} of the vector field $\vf{f}$ as $\vf{\phi}(t,\vf{x}):=\vf{\varphi}_{\vf{x}}(t)$.
  \end{definition}
  \begin{proposition}
    Let $U\subseteq \RR^n$ be an open set and $\vf{f}:U\rightarrow\RR^n$ be a vector field of class $\mathcal{C}^1$. Then, the flow $\vf{\phi}(t,\vf{x})$ of $\vf{f}$ defines locally a dynamical system.
  \end{proposition}
  \begin{definition}
    Let $U,V\subseteq \RR^n$ be open sets and $\vf{f}:U\rightarrow\RR^n$, $\vf{g}:V\rightarrow\RR^n$ be vector fields of class $\mathcal{C}^1$, and $r\in\NN\cup\{0,\infty\}$. We say that $\vf{f}$ and $\vf{g}$ are \emph{equivalent} of class $\mathcal{C}^r$ if there exists a diffeomorphism $\vf{h}:U\rightarrow V$ of class $\mathcal{C}^r$ such that $\forall \vf{x}\in U$, $\vf{h}(\vf{\gamma}_{\vf{f}}(\vf{x}))=\vf{\gamma}_{\vf{g}}(\vf{h}(\vf{x}))$ and preserving the orientation of the orbits. In particular, if $r=0$ we say that $\vf{f}$ and $\vf{g}$ are \emph{locally topologically equivalent}. That diffeomorphism $\vf{h}$ is called a \emph{local equivalence} (of class $\mathcal{C}^r$) between $\vf{f}$ and $\vf{g}$.
  \end{definition}
  \begin{definition}
    Let $U,V\subseteq \RR^n$ be open sets and $\vf{f}:U\rightarrow\RR^n$, $\vf{g}:V\rightarrow\RR^n$ be vector fields of class $\mathcal{C}^1$ with flows $\vf{\phi}$ and $\vf{\psi}$, respectively, and $r\in\NN\cup\{0,\infty\}$. We say that $\vf{f}$ and $\vf{g}$ are \emph{conjugate} of class $\mathcal{C}^r$ if there exists a diffeomorphism $\vf{h}:U\rightarrow V$ of class $\mathcal{C}^r$ such that $\vf{h}(\vf{\phi}(t,\vf{x}))=\vf{\psi}(t,\vf{h}(\vf{x}))$ $\forall(t,\vf{x})\in \domain(\vf{\phi})$ when the equation is well-defined\footnote{That is, $\forall(t,\vf{x})\in \domain(\vf{\phi})$ such that $(t,\vf{h}(\vf{x}))\in \domain(\vf{\psi})$.}. In particular, if $r=0$ we say that $\vf{f}$ and $\vf{g}$ are \emph{locally topologically conjugate}. That diffeomorphism $\vf{h}$ is called a \emph{local conjugacy} (of class $\mathcal{C}^r$) between $\vf{f}$ and $\vf{g}$.
  \end{definition}
  \begin{lemma}
    Let $U,V\subseteq \RR^n$ be open sets and $\vf{f}:U\rightarrow\RR^n$, $\vf{g}:V\rightarrow\RR^n$ be vector fields of class $\mathcal{C}^1$ and $\vf{h}:U\rightarrow V$ be a diffeomorphism of class $\mathcal{C}^1$. Then, $\vf{h}$ is a conjugacy of class $\mathcal{C}^1$ if and only if $\vf{Dh}(\vf{x})(\vf{f}(\vf{x}))=\vf{g}(\vf{h}(\vf{x}))$ $\forall \vf{x}\in U$.
  \end{lemma}
  \begin{proposition}
    Let $U,V\subseteq \RR^n$ be open sets and $\vf{f}:U\rightarrow\RR^n$, $\vf{g}:V\rightarrow\RR^n$ be vector fields. Suppose that $\vf{h}$ is a conjugacy between $\vf{x}'=\vf{f}(\vf{x})$ and $\vf{y}'=\vf{g}(\vf{y})$. Then, $\vf{x}'=-\vf{f}(\vf{x})$ and $\vf{y}'=-\vf{g}(\vf{y})$ are conjugate by $\vf{h}$.
  \end{proposition}
  \begin{definition}
    Let $U\subseteq\RR^n$ be an open set and $\vf{f}:U\rightarrow\RR^n$ be a vector field of class $\mathcal{C}^r$, $r\geq 1$, and $A\subseteq \RR^{n-1}$ be an open set. We say that function $\vf{s}:A\rightarrow U$ of class $\mathcal{C}^r$ is a \emph{local transversal section} of $\vf{f}$ of class $\mathcal{C}^r$ if $\forall a\in A$, $\langle\im \vf{Dh}(a),\vf{f}(\vf{s}(\vf{a}))\rangle=\RR^n$\footnote{A plane version would be that $\vf{s}:A\subseteq \RR\rightarrow U\subseteq \RR^2$ is a \emph{local transversal section} of $\vf{f}$ if $\forall a\in A$, $\vf{s}'(a)$ and $\vf{f}(\vf{s}(\vf{a}))$ are linearly independent.}. Take $\Sigma:=\vf{s}(A)$. If $\vf{s}:A\rightarrow\Sigma$ is a homeomorphism, we say that $\Sigma$ is a \emph{transversal section} of $\vf{f}$ of class $\mathcal{C}^r$.
  \end{definition}
  \begin{theorem}[Flow box theorem]
    Let $U\subseteq \RR^n$ be an open set, $\vf{f}:U\rightarrow\RR^n$ be a vector field of class $\mathcal{C}^r$, $r\geq 1$, $\vf{p}\in U$ be a non-singular point of $\vf{f}$ and $\vf{s}:A\rightarrow\Sigma$ be a transversal section of $\vf{f}$ of class $\mathcal{C}^r$ with $\vf{s}(\vf{0})=\vf{p}$. Then, there exists a neighbourhood $V\subseteq U$ of $\vf{p}$ and a diffeomorphism $\vf{h}:V\rightarrow(-\varepsilon,\varepsilon)\times B$ of class $\mathcal{C}^r$, where $\varepsilon>0$ and $B\subseteq\RR^{n-1}$ is an open ball centered at $\vf{0}=\vf{s}^{-1}(\vf{p})$, such that:
    \begin{itemize}
      \item $\vf{h}(\Sigma\cap V)=\{0\}\times B$.
      \item $\vf{h}$ is a conjugacy of class $\mathcal{C}^r$ between $\vf{f}|_V$ and $\vf{g}:(-\varepsilon,\varepsilon)\times B\rightarrow\RR^n$ defined as $\vf{g}(\vf{x})=(1,0\ldots,0)$ $\forall \vf{x}\in\domain\vf{g}$.
    \end{itemize}
  \end{theorem}
  \begin{lemma}
    Let $U,V\subseteq \RR^n$ be open sets and $\vf{f}:U\rightarrow\RR^n$, $\vf{g}:V\rightarrow\RR^n$ be vector fields of class $\mathcal{C}^1$ and $\vf{p}\in U$ be a critical point of $\vf{f}$. Suppose that $\vf{h}$ is a conjugacy of class $\mathcal{C}^1$ between $\vf{f}$ and $\vf{g}$. Then, the vector fields $\vf{Df}(\vf{p})$ and $\vf{Dg}(\vf{h}(\vf{p}))$ are conjugate by $\vf{Dh}(\vf{p})$. In particular, $\sigma(\vf{Df}(\vf{p}))=\sigma(\vf{Dg}(\vf{h}(\vf{p})))$.
  \end{lemma}
  \begin{definition}
    Let $U\subseteq \RR^2$ be an open set, $\vf{f}:U\rightarrow\RR^2$ be a vector field of class $\mathcal{C}^1$ and $\vf{p}\in U$ be a critical point of $\vf{f}$. Suppose $\sigma(\vf{Df}(\vf{p}))=\{\lambda_1,\lambda_2\}$. We say that $\vf{p}$ is a
    \begin{itemize}
      \item \emph{stable node} if $\lambda_1,\lambda_2\in\RR_{<0}$ and $\lambda_1\ne\lambda_2$.
      \item \emph{unstable node} if $\lambda_1,\lambda_2\in\RR_{>0}$ and $\lambda_1\ne\lambda_2$.
      \item \emph{saddle point} if $\lambda_1,\lambda_2\in\RR$ and $\lambda_1\lambda_2<0$.
      \item \emph{stable star} if $\lambda_1,\lambda_2\in\RR_{<0}$, $\lambda_1=\lambda_2$ and $\vf{Df}(\vf{p})\sim
              \begin{pmatrix}
                \lambda_1 & 0         \\
                0         & \lambda_1
              \end{pmatrix}$.
      \item \emph{unstable star} if $\lambda_1,\lambda_2\in\RR_{>0}$, $\lambda_1=\lambda_2$ and $\vf{Df}(\vf{p})\sim
              \begin{pmatrix}
                \lambda_1 & 0         \\
                0         & \lambda_1
              \end{pmatrix}$.
      \item \emph{stable degenerated node} if $\lambda_1,\lambda_2\in\RR_{<0}$, $\lambda_1=\lambda_2$ and $\vf{Df}(\vf{p})\sim
              \begin{pmatrix}
                \lambda_1 & 0         \\
                1         & \lambda_1
              \end{pmatrix}$.
      \item \emph{unstable degenerated node} if $\lambda_1,\lambda_2\in\RR_{>0}$, $\lambda_1=\lambda_2$ and $\vf{Df}(\vf{p})\sim
              \begin{pmatrix}
                \lambda_1 & 0         \\
                1         & \lambda_1
              \end{pmatrix}$.
      \item \emph{stable focus} (or \emph{sink}) if $\lambda_1,\lambda_2\in\CC$ and $\Re(\lambda_1)<0$.
      \item \emph{unstable focus} (or \emph{source}) if $\lambda_1,\lambda_2\in\CC$ and $\Re(\lambda_1)>0$.
      \item \emph{center} if $\lambda_1,\lambda_2\in\CC$, $\Re(\lambda_1)=0$ and $\vf{p}$ is surrounded by periodic orbits.
    \end{itemize}
  \end{definition}
  \begin{definition}
    Let $\vf{A}\in\mathcal{M}_2(\RR)$ and consider the linear system induced by $\vf{A}$ such that the origin is a saddle point, and $E_1$, $E_2$ be the eigenspaces of $\vf{A}$. We say that the four orbits contained in $E_1\cup E_2$ (without taking into account the singular point $\vf{0}$) are the \emph{saddle separatrices} of the linear system.
  \end{definition}
  \begin{center}
    \begin{minipage}{\linewidth}
      \centering
      \includestandalone[mode=image|tex,width=\linewidth]{Images/singularities_dim2}
      \captionof{figure}{Classification of singular points of a linear dynamical system of dimension 2, induced by the equation $\vf{x}'=\vf{A}\vf{x}$, $\vf{A}\in\mathcal{M}_2(\RR)$ in terms of $D=\det\vf{A}$ and $T=\trace\vf{A}$.}
    \end{minipage}
  \end{center}
  \begin{definition}
    Let $U\subseteq \RR^n$ be an open set, $\vf{f}:U\rightarrow\RR^n$ be a vector field and consider the differential system induced by $\vf{f}$. Let $\vf{\gamma}$ be an orbit of that system. We say that $\vf{\gamma}$ is a \emph{homoclinic orbit} if it joins a saddle point to itself. We say that $\vf{\gamma}$ is a \emph{heteroclinic orbit} if it joins joins two different singular points.
  \end{definition}
  \begin{center}
    \begin{minipage}{\linewidth}
      \centering
      \includestandalone[mode=image|tex,width=\linewidth]{Images/homo_hetero}
      \captionof{figure}{A homoclinic orbit (blue) and a heteroclinic orbit (green) of the differential system $x''=\sin(x)+x\cos(x)$.}
    \end{minipage}
  \end{center}
  \begin{definition}
    Let $U\subseteq \RR^2$ be an open set, $\vf{f}:U\rightarrow\RR^2$ be a vector field of class $\mathcal{C}^1$ and $\vf{p}\in U$ be a critical point of $\vf{f}$. We say that $\vf{p}$ has
    \begin{itemize}
      \item an \emph{elliptic sector} if a side of $\vf{p}$ is locally as \cref{DE_esector}.
      \item a \emph{hyperbolic sector} if a side of $\vf{p}$ is locally as \cref{DE_hsector}.
      \item  an \emph{attracting parabolic sector} if a side of $\vf{p}$ is locally as \cref{DE_apsector}.
      \item a \emph{repelling parabolic sector} if a side of $\vf{p}$ is locally as \cref{DE_rpsector}.
    \end{itemize}
    The union of all sectors that form a neighbourhood of $\vf{p}$ is called \emph{sectorial decomposition}.
    \begin{center}
      \begin{minipage}[b]{0.475\linewidth}
        \centering
        \includestandalone[mode=image|tex,width=0.75\linewidth]{Images/esector}
        \captionof{figure}{Elliptic sector}
        \label{DE_esector}
      \end{minipage}\hfill
      \begin{minipage}[b]{0.475\linewidth}
        \centering
        \includestandalone[mode=image|tex,width=0.75\linewidth]{Images/hsector}
        \captionof{figure}{Hyperbolic sector}
        \label{DE_hsector}
      \end{minipage}
    \end{center}
    \begin{center}
      \begin{minipage}[b]{0.475\linewidth}
        \centering
        \includestandalone[mode=image|tex,width=0.75\linewidth]{Images/apsector}
        \captionof{figure}{Attracting parabolic sector}
        \label{DE_apsector}
      \end{minipage}\hfill
      \begin{minipage}[b]{0.475\linewidth}
        \centering
        \includestandalone[mode=image|tex,width=0.75\linewidth]{Images/rpsector}
        \captionof{figure}{Repelling parabolic sector}
        \label{DE_rpsector}
      \end{minipage}
    \end{center}
  \end{definition}
  \begin{proposition}
    Any critical point of an analytic differential system in the plane can either be:
    \begin{itemize}
      \item A focus.
      \item A center.
      \item A finite collection of elliptic sectors, hyperbolic sectors and/or parabolic sectors.
    \end{itemize}
  \end{proposition}
  \subsubsection{Hamiltonian systems}
  \begin{definition}
    Let $U\subseteq \RR^n$ be an open set, $\vf{f}:U\rightarrow\RR^n$ be a vector field and $H:U\rightarrow\RR$ be a non-constant function. We say that $H$ is a \emph{first integral} for the differential system $\vf{x}'=\vf{f}(\vf{x})$ if for each solution $\vf{\varphi}(t)$ of that system, we have $H(\vf{\varphi(t)})=\const$ Thus, the phase trajectory of a solution $\vf{\varphi}(t)$ to $\vf{x}'=\vf{f}(\vf{x})$ lies on a level surface of $H$. In particular, if $n=2$, $\vf{\varphi}(t)$ will be a level curve of $H$.
  \end{definition}
  \begin{proposition}
    Let $U\subseteq \RR^n$ be an open set, $\vf{f}:U\rightarrow\RR^n$ be a vector field such that $\vf{f}=(f_1,\ldots,f_n)$ and $H:U\rightarrow\RR$ be a non-constant function. Then, $H$ is a first integral for the differential system $\vf{x}'=\vf{f}(\vf{x})$ if and only if: $$\pdv{H}{x_1}(\vf{x})f_1(\vf{\vf{x}})+\cdots+\pdv{H}{x_n}(\vf{x})f_n(\vf{\vf{x}})=0\quad\forall\vf{x}\in U$$
  \end{proposition}
  \begin{definition}
    Let $U\subseteq \RR^n$ be an open set, $\vf{f}:U\rightarrow\RR^n$ be a vector field and $H_1,\ldots,H_k:U\rightarrow\RR$ be $k\leq n-1$ first integrals for the differential system $\vf{x}'=\vf{f}(\vf{x})$. We say that $H_1,\ldots,H_k$ are \emph{functionally independent} (or simply \emph{independent}) if for all $\vf{x}\in U$ (except for maybe a set of zero area), we have: $$\rank
      \begin{pmatrix}
        \pdv{U_1}{x_1}(\vf{x}) & \cdots & \pdv{U_1}{x_n}(\vf{x}) \\
        \vdots                 & \ddots & \vdots                 \\
        \pdv{U_k}{x_1}(\vf{x}) & \cdots & \pdv{U_k}{x_n}(\vf{x}) \\
      \end{pmatrix}=k$$
  \end{definition}
  \begin{proposition}
    Let $U\subseteq \RR^n$ be an open set, $\vf{f}:U\rightarrow\RR^n$ be a vector field and $H_1,\ldots,H_k$ be $k\leq n-1$ functionally independent first integrals for the differential system $\vf{x}'=\vf{f}(\vf{x})$. Then, the number of unknowns of the system can be reduced to $n-k$.
  \end{proposition}
  \begin{definition}
    Let $U\subseteq \RR^{2n}$ be an open set, $H:U\rightarrow\RR$ be a function and $(\vf{x},\vf{y}):=(x_1,\ldots,x_n,y_1,\ldots,y_n)$. The differential system
    \begin{equation*}
      \left\{
      \begin{aligned}
        {x_1}' & =-\pdv{H}{y_1}(\vf{x}) \\
               & \;\;\vdots             \\
        {x_n}' & =-\pdv{H}{y_n}(\vf{x}) \\
        {y_1}' & =\pdv{H}{x_1}(\vf{x})  \\
               & \;\;\vdots             \\
        {y_n}' & =\pdv{H}{x_1}(\vf{x})  \\
      \end{aligned}
      \right.
    \end{equation*}
    is called \emph{Hamiltonian system} in $2n$ unknowns and $H$ is called \emph{Hamiltonian} of the system.
  \end{definition}
  \begin{proposition}
    Let $U\subseteq \RR^{2n}$ be an open set and $H:U\rightarrow\RR$ be the Hamiltonian of its Hamiltonian associated system. Then, $H$ is a first integral of that differential system.
  \end{proposition}
  \begin{theorem}
    Let $U\subseteq \RR^{2}$ be an open set, $H:U\rightarrow\RR$ be the Hamiltonian of its Hamiltonian associated system and $\vf{p}\in U$ be a singular point. Then, $\vf{p}$ is either a saddle point or a center.
  \end{theorem}
  \subsubsection{Local structure of hyperbolic critical points}
  \begin{definition}
    Let $U\subseteq \RR^n$ be an open set, $\vf{f}:U\rightarrow\RR^n$ be a vector field of class $\mathcal{C}^r$ with $r\geq 1$ and $\vf{p}\in U$ be a critical point of $\vf{f}$. We say that $\vf{p}$ is \emph{hyperbolic critical point} if $\Re(\lambda)\ne 0$ $\forall \lambda\in\sigma(\vf{Df}(\vf{p}))$.
  \end{definition}
  \begin{theorem}[Hartman-Grobman theorem]
    Let $U\subseteq \RR^n$ be an open set, $\vf{f}:U\rightarrow\RR^n$ be a vector field of class $\mathcal{C}^1$ and $\vf{p}\in U$ be a hyperbolic critical point of $\vf{f}$. Let $\vf{g}:\RR^n\rightarrow\RR^n$ be the vector field defined as $\vf{g}(\vf{x})=\vf{Df}(\vf{p})(\vf{x})$. Then, there exist neighbourhoods $V\subseteq U$ of $\vf{p}$ and $W\subseteq \RR^n$ of $\vf{f}(\vf{p})=\vf{0}$ such that $\vf{f}|_V$ and $\vf{g}|_W$ topologically conjugate.
  \end{theorem}
  \begin{corollary}
    Let $U,V\subseteq \RR^n$ be open sets and $\vf{f}:U\rightarrow\RR^n$, $\vf{g}:V\rightarrow\RR^n$ be vector fields of class $\mathcal{C}^1$ and $\vf{p}\in U$ be a hyperbolic critical point of $\vf{f}$. Suppose that $\vf{h}$ is a conjugacy of class $\mathcal{C}^1$ between $\vf{f}$ and $\vf{g}$. Then, $\vf{h}(\vf{p})$ is a hyperbolic critical point of $\vf{g}$.
  \end{corollary}
  \begin{definition}
    Let $\vf{A}\in\mathcal{M}_n(\RR)$. We say that $\vf{A}$ is \emph{hyperbolic matrix} if $\Re(\lambda)\ne 0$ $\forall\lambda\in\sigma(\vf{A})$.
  \end{definition}
  \begin{proposition}
    Let $\vf{A}\in\mathcal{M}_n(\RR)$ be a hyperbolic matrix. Then, $\vf{0}\in\RR^n$ is the unique critical point of $\vf{x}'=\vf{A}\vf{x}$ and it is hyperbolic.
  \end{proposition}
  \begin{definition}
    Let $\vf{A}\in\mathcal{M}_n(\RR)$. We define the \emph{stability number} of $\vf{A}$ as: $$\iota(\vf{A}):=|\{\lambda\in\sigma(\vf{A}):\Re(\lambda)<0\}|$$
  \end{definition}
  \begin{theorem}
    Let $\vf{A}\in\mathcal{M}_n(\RR)$ be a hyperbolic matrix such that $\iota(\vf{A})=n$. Then, $\vf{x}'=\vf{A}\vf{x}$ and $\vf{y}'=-\vf{y}$, $\vf{y}\in\RR^n$, are topologically conjugate. In particular, the origin is attracting.
  \end{theorem}
  \begin{corollary}
    Let $\vf{A}\in\mathcal{M}_n(\RR)$ be a hyperbolic matrix such that $\iota(\vf{A})=0$. Then, $\vf{x}'=\vf{A}\vf{x}$ and $\vf{y}'=\vf{y}$, $\vf{y}\in\RR^n$, are topologically conjugate. In particular, the origin is repelling.
  \end{corollary}
  \begin{corollary}
    Let $\vf{A}\in\mathcal{M}_n(\RR)$ be a hyperbolic matrix such that $\iota(\vf{A})=k$. Then, $\vf{x}'=\vf{A}\vf{x}$ and $$\{\vf{y}'=-\vf{y}, \vf{z}'=\vf{z}:\vf{y}\in\RR^k,{z}\in\RR^{n-k}\}$$ are topologically conjugate. In particular, the origin is neither attracting nor repelling.
  \end{corollary}
  \begin{theorem}
    Let $\vf{A},\vf{B}\in\mathcal{M}_n(\RR)$ be hyperbolic matrices. Then, $\vf{x}'=\vf{A}\vf{x}$ and $\vf{y}'=\vf{B}\vf{y}$ are topologically conjugate if and only if $\iota(\vf{A})=\iota(\vf{B})$.
  \end{theorem}
  \begin{corollary}
    Let $\vf{A}\in\mathcal{M}_n(\RR)$ be a hyperbolic matrix. Then:
    \begin{itemize}
      \item $\vf{0}$ is attracting for $\vf{x}'=\vf{A}\vf{x}\iff\iota(\vf{A})=n$.
      \item $\vf{0}$ is repelling for $\vf{x}'=\vf{A}\vf{x}\iff\iota(\vf{A})=0$.
    \end{itemize}
  \end{corollary}
  \begin{theorem}
    Let $U\subseteq \RR^n$ be an open set, $\vf{f}:U\rightarrow\RR^n$ be a vector field of class $\mathcal{C}^1$ and $\vf{p}\in U$ be a critical point of $\vf{f}$. Then:
    \begin{enumerate}
      \item If $\iota(\vf{Df}(\vf{p}))=n$, then $\vf{p}$ is asymptotically stable for the dynamical system induced by $\vf{x}'=\vf{f}(\vf{x})$.
      \item If $\iota(\vf{Df}(\vf{p}))=0$, then $\vf{p}$ is repelling and negatively stable for the dynamical system induced by $\vf{x}'=\vf{f}(\vf{x})$.
      \item If $\vf{p}$ is positively stable, then $\Re(\lambda)\leq 0$ $\forall\lambda\in\sigma(\vf{Df}(\vf{p}))$.
      \item If $\vf{p}$ is negatively stable, then $\Re(\lambda)\geq 0$ $\forall\lambda\in\sigma(\vf{Df}(\vf{p}))$.
    \end{enumerate}
  \end{theorem}
  \subsection{Qualitative theory of planar differential systems}
  \subsubsection{Polynomial vectors fields}
  \begin{definition}
    Let $p,q\in\RR[x,y]$. The system of odes
    \begin{equation}\label{DE_poly}
      \left\{
      \begin{aligned}
        x' & =p(x,y) \\
        y' & =q(x,y)
      \end{aligned}
      \right.
    \end{equation}
    is called a \emph{polynomial system}. The field $\vf{f}=(p,q)$ is called \emph{polynomial vector field}. We define the \emph{degree} of that system as $n:=\max\{\deg p,\deg q\}$. Another commonly used notation for expressing the vector field is through the operator
    \begin{equation}\label{DE_oper}
      \vf{X}:=p(x,y)\pdv{}{x}+q(x,y)\pdv{}{y}
    \end{equation}
  \end{definition}
  \begin{definition}
    Let $f\in\RR[x,y]$ be a polynomial. An \emph{algebraic curve} is the set of points satisfying the equation $f(x,y)=0$.
  \end{definition}
  \begin{definition}
    Let $f(x,y)=0$ be an algebraic curve, $p,q\in\RR[x,y]$ and consider the polynomial system of degree $n$ of \cref{DE_poly}. We say that $f(x,y)=0$ is an \emph{invariant algebraic curve} under the system of \cref{DE_poly} if
    \begin{equation}\label{DE_invarcurve}
      \pdv{f}{x}(x,y)p(x,y)+\pdv{f}{y}q(x,y)=k(x,y)f(x,y)
    \end{equation}
    where $k\in\RR[x,y]$ is called \emph{cofactor} of the invariant curve $f(x,y)=0$\footnote{Note that $\deg k\leq n-1$.}. The \cref{DE_invarcurve} can be written as: $$\vf{X}f=kf$$
    where $\vf{X}$ is the operator defined in \cref{DE_oper}.
  \end{definition}
  \begin{proposition}
    Let $f(x,y)=0$ be an algebraic curve, $p,q\in\RR[x,y]$ and consider the polynomial system of degree $n$ of \cref{DE_poly}. Then, the invariant curve $f(x,y)=0$ is a set of orbits of the differential system of \cref{DE_poly}.
  \end{proposition}
  \subsubsection{Local structure of periodic orbits}
  \begin{definition}[Poincaré map]
    Let $U\subseteq \RR^n$ be an open set, $\vf{f}:U\rightarrow\RR^n$ be a vector field of class $\mathcal{C}^1$ with flow $\vf{\phi}(t,\vf{x})$, $\vf{p}\in U$ and $\vf{\gamma}(\vf{p})$ be a periodic orbit of period $T$ that passes through $\vf{p}$. Let $\Sigma$ be a transversal section at $\vf{p}$. For each $\vf{q}\in\Sigma$ (close enough to $\vf{p}$) such that the trajectory $\vf{\phi}(t,\vf{q})$ intersects $\Sigma$ in a distinct point from $\vf{q}$, we define the \emph{Poincaré map} as the function $\pi:\Sigma_0\subset \Sigma\rightarrow\Sigma$ sending $\vf{q}$ to the first point where $\vf{\phi}(t,\vf{q})$ intersects $\Sigma$ (different from $\vf{q}$).
  \end{definition}
  \begin{definition}
    Let $U\subseteq \RR^2$ be an open set, $\vf{f}:U\rightarrow\RR^2$ be a vector field of class $\mathcal{C}^1$ and $\vf{\gamma}$ be a periodic orbit. We say that $\vf{\gamma}$ is a \emph{limit cycle} if there exists a neighbourhood $V$ of $\vf{\gamma}$ such that $\vf{\gamma}$ is the only periodic orbit in $V$.
  \end{definition}
  \begin{definition}
    Let $U\subseteq \RR^2$ be an open set, $\vf{f}:U\rightarrow\RR^2$ be a vector field of class $\mathcal{C}^1$ and $\vf{\gamma}$ be a periodic orbit. We denote by $\Ext(\vf{\gamma})$ the set of points which belong to the unbounded component of $\RR^2\setminus\vf{\gamma}$, and by $\Int(\vf{\gamma})$ the set of points which belong to the bounded component of $\RR^2\setminus\vf{\gamma}$.
  \end{definition}
  \begin{proposition}
    Let $U\subseteq \RR^2$ be an open set, $\vf{f}:U\rightarrow\RR^2$ be a vector field of class $\mathcal{C}^1$, $\vf{\gamma}$ be a limit cycle and $V$ be a neighbourhood of $\vf{\gamma}$. Then, $\vf{\gamma}$ is exactly one of the following three types of limit cycles:
    \begin{itemize}
      \item $\vf{\gamma}$ is \emph{stable} if $\omega(\vf{q})=\vf{\gamma}$ $\forall \vf{q}\in V$.
      \item $\vf{\gamma}$ is \emph{unstable} if $\alpha(\vf{q})=\vf{\gamma}$ $\forall \vf{q}\in V$.
      \item $\vf{\gamma}$ is \emph{semi-stable} if either
            \begin{multline*}
              \{\omega(\vf{q})=\vf{\gamma}\quad\forall \vf{q}\in V\cap\Ext(\vf{\gamma})\} \land \\\land\{\alpha(\vf{q})=\vf{\gamma}\quad\forall \vf{q}\in V\cap\Int(\vf{\gamma})\}
            \end{multline*}
            or
            \begin{multline*}
              \{\omega(\vf{q})=\vf{\gamma}\quad\forall \vf{q}\in V\cap\Int(\vf{\gamma})\}\land\\\land \{\alpha(\vf{q})=\vf{\gamma}\quad\forall \vf{q}\in V\cap\Ext(\vf{\gamma})\}
            \end{multline*}
    \end{itemize}
  \end{proposition}
  \begin{definition}
    Let $U\subseteq \RR^2$ be an open set, $\vf{f}:U\rightarrow\RR^2$ be a vector field of class $\mathcal{C}^1$ and $\vf{\gamma}$ be a periodic orbit of period $T$. We say that $\vf\gamma$ is a \emph{hyperbolic periodic orbit} if $$I(\vf\gamma):=\int_0^T\div\vf{f}(\vf\gamma(t))\dd{t}\ne 0$$
  \end{definition}
  \begin{theorem}
    Let $U\subseteq \RR^2$ be an open set, $\vf{f}:U\rightarrow\RR^2$ be a vector field of class $\mathcal{C}^1$ and $\vf{\gamma}$ be a periodic orbit of period $T$. Then:
    \begin{itemize}
      \item $I(\vf\gamma)>0\implies\vf\gamma(t)$ is an unstable limit cycle.
      \item $I(\vf\gamma)<0\implies\vf\gamma(t)$ is a stable limit cycle.
    \end{itemize}
  \end{theorem}
  \subsubsection{Poincaré-Bendixson theorem}
  \begin{lemma}
    Let $U\subseteq \RR^2$ be an open set, $\vf{f}:U\rightarrow\RR^2$ be a vector field of class $\mathcal{C}^1$, $\Sigma$ be a transversal section of $\vf{f}$, $\vf\gamma$ be an orbit of $\vf{f}$ and $\vf{p}\in\Sigma\cap\omega(\vf{\gamma})$. Suppose that $\vf\varphi(t)$ is the flux of the system. Then, $\exists (t_n)\in\RR$ such that $\vf\varphi(t_n)\in\Sigma$ and $\displaystyle\lim_{n\to\infty}\vf\varphi(t_n)=\vf{p}$.
  \end{lemma}
  \begin{lemma}
    Let $U\subseteq \RR^2$ be an open set, $\vf{f}:U\rightarrow\RR^2$ be a vector field of class $\mathcal{C}^1$, $\Sigma$ be a transversal section of $\vf{f}$, $\vf\gamma$ be an orbit of $\vf{f}$ and $\vf{p}\in\Sigma\cap\omega(\vf{\gamma})$. Then, ${\vf\gamma}^+(\vf{p})$ intersect $\Sigma$ in a (finite or infinite) monotone sequence of points.
  \end{lemma}
  \begin{lemma}
    Let $U\subseteq \RR^2$ be an open set, $\vf{f}:U\rightarrow\RR^2$ be a vector field of class $\mathcal{C}^1$, $\Sigma$ be a transversal section of $\vf{f}$ and $\vf{p}\in U$. Then, $\left|\Sigma\cap\omega(\vf{p})\right|$ is either 0 or 1.
  \end{lemma}
  \begin{lemma}
    Let $U\subseteq \RR^2$ be an open set, $\vf{f}:U\rightarrow\RR^2$ be a vector field of class $\mathcal{C}^1$, $\vf{p}\in U$ be such that ${\vf{\gamma}}^+(\vf{p})$ is contained in a compact set, and $\vf{\gamma}$ be an orbit such that $\vf\gamma\subseteq \omega(\vf{p})$. If $\omega(\vf{p})$ contains only non-singular points, then $\omega(\vf{p})$ is a periodic orbit and $\vf\gamma=\omega(\vf{p})$.
  \end{lemma}
  \begin{theorem}[Poincaré-Bendixson theorem]
    Let $U\subseteq \RR^2$ be an open set, $\vf{f}:U\rightarrow\RR^2$ be a vector field of class $\mathcal{C}^1$ and $\vf{p}\in U$ be such that ${\vf{\gamma}}^+(\vf{p})$ is contained in a compact set. Suppose that $\vf{f}$ has a finite number of singular points. Then:
    \begin{enumerate}
      \item If $\omega(\vf{p})$ contains only non-singular points, then $\omega(\vf{p})$ is a periodic orbit.
      \item If $\omega(\vf{p})$ contains only singular points, then $\omega(\vf{p})$ is a singular point.
      \item If $\omega(\vf{p})$ contains both singular and non-singular points, then $\omega(\vf{p})$ is a collection of singular points together with homoclinic and heteroclinic orbits connecting those points.
    \end{enumerate}
  \end{theorem}
  \begin{corollary}[Poincaré-Bendixson theorem]
    Let $U\subseteq \RR^2$ be an open set, $\vf{f}:U\rightarrow\RR^2$ be a vector field of class $\mathcal{C}^1$ and $\vf{p}\in U$ be such that ${\vf{\gamma}}^-(\vf{p})$ is contained in a compact set. Suppose that $\vf{f}$ has a finite number of singular points. Then:
    \begin{enumerate}
      \item If $\alpha(\vf{p})$ contains only non-singular points, then $\alpha(\vf{p})$ is a periodic orbit.
      \item If $\alpha(\vf{p})$ contains only singular points, then $\alpha(\vf{p})$ is a singular point.
      \item If $\alpha(\vf{p})$ contains both singular and non-singular points, then $\alpha(\vf{p})$ is a collection of singular points together with homoclinic and heteroclinic orbits connecting those points.
    \end{enumerate}
  \end{corollary}
  \begin{corollary}
    Let $U\subseteq \RR^2$ be an open set, $\vf{f}:U\rightarrow\RR^2$ be a vector field of class $\mathcal{C}^1$ and ${\vf\gamma}$ be a periodic orbit of $\vf{f}$. Then, there is at least one singular point in $\Int(\vf\gamma)$.
  \end{corollary}
  \subsubsection{Lyapunov stability}
  \begin{definition}
    Let $U\subseteq \RR^n$ be an open set and $\vf{f}:U\rightarrow\RR^n$ be a vector field of class $\mathcal{C}^1$ and $\vf{p}\in U$ be a critical point of $\vf{f}$. We say that $\vf{p}$ is \emph{Lyapunov stable} if the set $\{\vf{p}\}$ is positively stable.
  \end{definition}
  \begin{definition}
    Let $U\subseteq \RR^n$ be an open set, $\vf{f}:U\rightarrow\RR^n$ be a vector field of class $\mathcal{C}^1$ and $\vf{p}\in U$ be a critical point of $\vf{f}$. We say that a function $V:U\rightarrow\RR$ of class $\mathcal{C}^1$ is a \emph{Lyapunov function} for $\vf{p}$ if there exists a neighbourhood $\tilde{U}\subseteq U$ of $\vf{p}$ such that:
    \begin{itemize}
      \item $V(\vf{p})=0$ and $V(\vf{x})>0$ $\forall \vf{x}\in\tilde{U}\setminus\{\vf{p}\}$
      \item $\grad V(\vf{q})\cdot \vf{f}(\vf{q})\leq 0$ $\forall \vf{q}\in \tilde{U}$
    \end{itemize}
    If instead of the second condition we have
    \begin{itemize}
      \item $\grad V(\vf{q})\cdot \vf{f}(\vf{q})<0$ $\forall \vf{q}\in \tilde{U}\setminus\{\vf{p}\}$
    \end{itemize}
    we say that $V$ is a \emph{strict Lyapunov function} for $\vf{p}$.
  \end{definition}
  \begin{theorem}[Lyapunov's theorem]
    Let $U\subseteq \RR^n$ be an open set and $\vf{f}:U\rightarrow\RR^n$ be a vector field of class $\mathcal{C}^1$ and $\vf{p}\in U$ be a critical point of $\vf{f}$.
    \begin{itemize}
      \item If there exists a Lyapunov function for $\vf{p}$ in a neighbourhood of $\vf{p}$, then $\vf{p}$ is Lyapunov stable.
      \item If there exists a strict Lyapunov function for $\vf{p}$ in a neighbourhood of $\vf{p}$, then $\vf{p}$ is asymptotically stable.
    \end{itemize}
  \end{theorem}
  \begin{theorem}[Bendixson's theorem]
    Let $U\subseteq \RR^2$ be an open set and $\vf{f}:U\rightarrow\RR^2$ be a vector field of class $\mathcal{C}^1$ such that $\div \vf{f}$ has constant sign in a simply connected region $R$ and is not identically zero on any subregion of $R$ with positive area. Then, the system $\vf{x}'=\vf{f}(\vf{x})$ does not have periodic orbits that lie entirely on $R$.
  \end{theorem}
  \begin{theorem}[Bendixson-Dulac theorem]
    Let $U\subseteq \RR^2$ be an open set and $\vf{f}:U\rightarrow\RR^2$ be a vector field of class $\mathcal{C}^1$. Suppose that there exists a simply connected region $R$ and a function $h:R\rightarrow\RR$ of class $\mathcal{C}^1$ such that $\div (h\vf{f})$ has constant sign on $R$ and is not identically zero on any subregion of $R$ with positive area. Then, the system $\vf{x}'=\vf{f}(\vf{x})$ doesn't have periodic orbits that lie entirely on $R$.
  \end{theorem}
  \begin{theorem}[Generalized Bendixson-Dulac theorem]
    Let $U\subseteq \RR^2$ be an open set, $n\in\NN\cup\{0\}$ and $\vf{f}:U\rightarrow\RR^2$ be a vector field of class $\mathcal{C}^1$. Suppose that there exists a subset $R\subseteq U$ homeomorphic to a disk with $n$ holes and a function $h:R\rightarrow\RR$ of class $\mathcal{C}^1$ such that $\div (h\vf{f})$ has constant sign on $R$ and is not identically zero on any subregion of $R$ with positive area. Then, the system $\vf{x}'=\vf{f}(\vf{x})$ has at most $n$ periodic orbits that lie entirely on $R$.
  \end{theorem}
  \subsubsection{Poincaré compactification}
  \begin{definition}
    Let $\vf{f}:\RR^2\rightarrow\RR^2$ be a vector field of class $\mathcal{C}^1$. Consider the sphere $S^2$ and the plane $\Pi=\{(x_1,x_2,x_3)\in\RR^3:x_3=1\}\cong\RR^2$. Let
    \begin{gather*}
      H_+:=S^2\cap\{(x_1,x_2,x_3)\in\RR^3:x_3>0\}\\
      H_-:=S^2\cap\{(x_1,x_2,x_3)\in\RR^3:x_3<0\}
    \end{gather*}
    For each point $p\in\Pi$, the line joining $p$ and $(0,0,0)$ intersects $S^2$ in two points. We define the following functions
    \begin{gather*}
      \function{\vf{g}_+}{\Pi}{H_+}{(x_1,x_2,1)}{\left(\frac{x_1}{\sqrt{1+x_1^2+x_2^2}},\frac{x_2}{\sqrt{1+x_1^2+x_2^2}},\frac{1}{\sqrt{1+x_1^2+x_2^2}}\right)}\\
      \function{\vf{g}_-}{\Pi}{H_-}{(x_1,x_2,1)}{\left(\frac{-x_1}{\sqrt{1+x_1^2+x_2^2}},\frac{-x_2}{\sqrt{1+x_1^2+x_2^2}},\frac{-1}{\sqrt{1+x_1^2+x_2^2}}\right)}
    \end{gather*}
    which are diffeomorphisms. The induced vector field $\vf{\tilde{f}}$ defined in $S^2\setminus S^1:=H_+\cup H_-$ is\footnote{The idea behind this concept is to study the asymptotic behaviour of the orbits of the system $\vf{x}'=\vf{f}(\vf{x})$. In order to do so, we would like to extend the field $\vf{\tilde{f}}$ to the equator of $S^2$ ($\{(x_1,x_2,x_3)\in S^2:x_3=0\}$). And that set would correspond to the infinity in $\RR^2$.}:
    $$\vf{\tilde{f}}(\vf{y})=
      \begin{cases}
        \vf{D}\vf{g}_+(\vf{x}) \vf{f}(\vf{x}) & \text{if }\vf{y}=\vf{g}_+(\vf{x})\in H_+ \\
        \vf{D}\vf{g}_-(\vf{x}) \vf{f}(\vf{x}) & \text{if }\vf{y}=\vf{g}_-(\vf{x})\in H_-
      \end{cases}
    $$
  \end{definition}
  \begin{proposition}
    Let $\vf{f}=(p,q):\RR^2\rightarrow\RR^2$ be a polynomial vector field of degree $d$, $\vf{\tilde{f}}$ be the induced vector field on $S^2\setminus S^1$ and $\rho:S^2\rightarrow\RR$ be the function defined as $\rho(y_1,y_2,y_3)={y_3}^{d-1}$. Then, the field $\rho\vf{\tilde{f}}$ can be extended analytically to $S^2$ with the equator of $S^2$ remaining invariant.
  \end{proposition}
  \begin{corollary}
    Let $\vf{f}=(p,q):\RR^2\rightarrow\RR^2$ be a polynomial vector field of degree $d$ and consider the local charts $(U_i,\vf\phi_i)$, $(V_i,\vf\psi_i)$ for $i=1,2,3$ defined as:
    \begin{gather*}
      U_i=\{(x_1,x_2,x_3)\in S^2:x_i>0\}\\
      V_i=\{(x_1,x_2,x_3)\in S^2:x_i<0\}
    \end{gather*}
    and
    \begin{gather*}
      \function{\vf\phi_i}{U_i}{\RR^2}{(y_1,y_2,y_3)}{\left(\frac{y_j}{y_i},\frac{y_k}{y_i}\right)}\\
      \function{\vf\psi_i}{V_i}{\RR^2}{(y_1,y_2,y_3)}{\left(\frac{y_j}{y_i},\frac{y_k}{y_i}\right)}
    \end{gather*}
    with $j,k\ne i$, $j<k$ and $i=1,2,3$. Then, the extended vector field defined on each $(U_i,\vf\phi_i)$ is:
    \begin{itemize}
      \item On $(U_1,\vf\phi_1)$, if $(u,v)=\left(\frac{x_2}{x_1},\frac{1}{x_1}\right)$, then:
            $$
              \left\{
              \begin{aligned}
                u' & =v^d\left[-up\left(\frac{1}{v},\frac{u}{v}\right)+q\left(\frac{1}{v},\frac{u}{v}\right)\right] \\
                v' & =-v^{d+1}p\left(\frac{1}{v},\frac{u}{v}\right)
              \end{aligned}
              \right.
            $$
      \item On $(U_2,\vf\phi_2)$, if $(u,v)=\left(\frac{x_3}{x_2},\frac{1}{x_2}\right)$, then:
            $$
              \left\{
              \begin{aligned}
                u' & =v^d\left[p\left(\frac{u}{v},\frac{1}{v}\right)-uq\left(\frac{u}{v},\frac{1}{v}\right)\right] \\
                v' & =-v^{d+1}q\left(\frac{u}{v},\frac{1}{v}\right)
              \end{aligned}
              \right.
            $$
      \item On $(U_3,\vf\phi_3)$, if $(u,v)=\left(\frac{x_1}{x_3},\frac{1}{x_3}\right)$, then:
            $$
              \left\{
              \begin{aligned}
                u' & =p(u,v) \\
                v' & =q(u,v)
              \end{aligned}
              \right.
            $$
    \end{itemize}
    The extended vector field defined on each $(V_i,\vf\psi_i)$ is the one defined on each $(U_i,\vf\phi_i)$ multiplied by ${(-1)}^{d-1}$.
    This extension is called \emph{Poincaré compactification} of $\vf{f}$.
  \end{corollary}
  \begin{definition}
    We define the \emph{Poincaré disk} as the orthogonal projection $\vf\pi: \overline{H_+}\rightarrow \overline{D(0,1)}$.
  \end{definition}
  \subsubsection{Integrability theory of polynomial systems}
  \begin{definition}
    Let $U\subseteq\RR^2$ be an open set, $p,q\in\RR[x,y]$ and consider the polynomial system of \cref{DE_poly}. We say that a function $R:U\rightarrow\RR$ is an \emph{integrating factor} if $$\div (Rp,Rq)=\pdv{(Rp)}{x}+\pdv{(Rq)}{y}=0$$
  \end{definition}
  \begin{lemma}
    Let $U\subseteq\RR^2$ be an open set, $R:U\rightarrow\RR$ be a differentiable function, $p,q\in\RR[x,y]$ and consider the polynomial system of \cref{DE_poly}. Then, $R$ is an integrating factor if and only if: $$\vf{X}R=p\pdv{R}{x}+q\pdv{R}{y}=-R\div(p,q)$$
    where $\vf{X}$ is the operator defined in \cref{DE_oper}.
  \end{lemma}
  \begin{proposition}\label{DE_intfactor}
    Let $U\subseteq\RR^2$ be an open set, $p,q\in\RR[x,y]$ and consider the polynomial system of \cref{DE_poly}. Suppose that system admits an integrating factor $R:U\rightarrow\RR$. Then, the system admits a first integral $H:U\rightarrow\RR$ given by:
    $$H(x,y)=-\int R(x,y)p(x,y)\dd{y}+h(x)$$
    where $h(x)$ satisfy: $$h'(x)=R(x,y)q(x,y)+\pdv{}{x}\left(\int R(x,y)p(x,y)\dd{y}\right)$$
  \end{proposition}
  \begin{definition}
    Let $p,q,g,h\in\RR[x,y]$ and consider the polynomial system of \cref{DE_poly} of degree $d$ and let $\vf{X}$ be the vector field operator of that system (defined by \cref{DE_oper}). We say that $\exp{\frac{g(x,y)}{h(x,y)}}$ is an \emph{exponential factor} with cofactor $k(x,y)\in\RR[x,y]$ if $\deg k\leq d-1$ and: $$\vf{X}\exp{\frac{g(x,y)}{h(x,y)}}=k(x,y)\exp{\frac{g(x,y)}{h(x,y)}}$$
  \end{definition}
  \begin{theorem}[Darboux theorem]
    Let $p,q\in\RR[x,y]$ and consider the polynomial system of \cref{DE_poly} of degree $d$, $f_i(x,y)=0$ be invariant algebraic curves with cofactors $k_i(x,y)$ for $i=1,\ldots,r$ and $\exp{\frac{g_j(x,y)}{h_j(x,y)}}$ be exponential factors with cofactors $\ell_j(x,y)$ for $j=1,\ldots,s$. Then:
    \begin{enumerate}
      \item If $\exists \lambda_i,\mu_j\in\RR$, $i=1,\ldots,r$ and $j=1,\ldots,s$, not all zero such that $\sum_{i=1}^r\lambda_ik_i+\sum_{j=1}^s\mu_j\ell_j=0$, then
            \begin{equation}\label{DE_firstint}
              H={f_1}^{\lambda_1}\cdots{f_r}^{\lambda_r}\exp{\mu_1\frac{g_1(x,y)}{h_1(x,y)}}\cdots\exp{\mu_s\frac{g_s(x,y)}{h_s(x,y)}}
            \end{equation}
            is a first integral for the system.
      \item If $r+s\geq \frac{d(d+1)}{2}+1$, then $\exists \lambda_i,\mu_j\in\RR$, $i=1,\ldots,r$ and $j=1,\ldots,s$, not all zero such that $\sum_{i=1}^p\lambda_ik_i+\sum_{j=1}^q\mu_j\ell_j=0$. And so, the system has the first integral defined in \cref{DE_firstint}.
      \item If $r+s\geq \frac{d(d+1)}{2}+2$, then the system has a rational first integral. Consequently all trajectories of the system are contained in
            invariant algebraic curves.
      \item If $\exists \lambda_i,\mu_j\in\RR$, $i=1,\ldots,r$ and $j=1,\ldots,s$, not all zero such that $\sum_{i=1}^p\lambda_ik_i+\sum_{j=1}^q\mu_j\ell_j=-\div(p,q)$, then
            \begin{equation*}
              R={f_1}^{\lambda_1}\cdots{f_r}^{\lambda_r}\exp{\mu_1\frac{g_1(x,y)}{h_1(x,y)}}\cdots\exp{\mu_s\frac{g_s(x,y)}{h_s(x,y)}}
            \end{equation*}
            is an integrating factor for the system. And so the system also admits a first integral by \cref{DE_intfactor}.
    \end{enumerate}
  \end{theorem}
  \subsubsection{Index of paths and homotopy}
  \begin{definition}
    Let $\vf\gamma:[a,b]\rightarrow\RR^2$ be a closed path, $q\in \RR\setminus\vf\gamma^*$ and $L$ be a ray with vertex at $q$. Consider a continuous determination $\varphi:[a,b]\rightarrow\RR$ of the angle (measured counterclockwisely) between $\vf\gamma(t)$ and $L$. Then, we define the \emph{index} of $q$ with respect to $\vf\gamma$ as: $$\Ind(\vf\gamma,q):=\frac{\varphi(b)-\varphi(a)}{2\pi}$$
  \end{definition}
  \begin{proposition}
    Let $q_1,q_2\in\RR^2$ and $\vf\gamma:I\rightarrow\RR^2$ be a closed path such that the segment $\overline{q_1q_2}$ does not intersect $\vf\gamma^*$. Then, $\Ind(\vf\gamma,q_1)=\Ind(\vf\gamma,q_2)$.
  \end{proposition}
  \begin{corollary}
    Let $\vf\gamma:I\rightarrow\RR^2$ be a closed path. Then, all points in the same conencted component of $\RR^2\setminus\vf\gamma^*$ have the same index.
  \end{corollary}
  \begin{proposition}
    Let $\vf\gamma_1,\vf\gamma_2:I\rightarrow\RR^2$ be two closed paths and $q\in\RR^2$ be such that $q\notin\overline{\vf\gamma_1(t)\vf\gamma_2(t)}$ $\forall t\in I$. Then, $\Ind(\vf\gamma_1,q)=\Ind(\vf\gamma_2,q)$.
  \end{proposition}
  \begin{proposition}
    Let $\vf\gamma_1,\vf\gamma_2:I\rightarrow\RR^2$ be two closed paths and $q\in\RR^2$ be such that $q\notin{\vf\gamma_1}^*\cup{\vf\gamma_2}^*$ and $\|\vf\gamma_1(t)-\vf\gamma_2(t)\|<\|q-\vf\gamma_2(t)\|$ $\forall t\in I$. Then, $\Ind(\vf\gamma_1,q)=\Ind(\vf\gamma_2,q)$.
  \end{proposition}
  \begin{definition}
    Let $\vf\gamma_1,\vf\gamma_2:I\rightarrow\RR^2$ be two closed paths. We say that the are \emph{homotopic}, and we denote it by $\vf\gamma_1\homotop\vf\gamma_2$, if there exists a continuous function $\vf{h}:I\times[0,1]\rightarrow\RR^2$ such that:
    \begin{enumerate}
      \item $\vf\gamma_1(t)=\vf{h}(t,0)$ $\forall t\in I$
      \item $\vf\gamma_2(t)=\vf{h}(t,1)$ $\forall t\in I$
      \item $\vf{h}(0,s)=\vf{h}(1,s)$ $\forall s\in [0,1]$
    \end{enumerate}
    Such function $\vf{h}$ is called the \emph{homotopy} between $\vf\gamma_1$ and $\vf\gamma_2$.
  \end{definition}
  \begin{lemma}
    Being homotopic is an equivalence relation.
  \end{lemma}
  \begin{proposition}
    Let $\vf\gamma_1,\vf\gamma_2:I\rightarrow\RR^2$ be two closed homotopic paths, $\vf{h}:I\times[0,1]\rightarrow\RR^2$ be the respective homotopy and $q\in\RR^2$ be such that $q\notin \im(\vf{h})$\footnote{From now on we will denote $q\notin \im(\vf{h})$ as $q\in\RR^2\setminus \overline{D(\vf\gamma_1-\vf\gamma_2)}$, where $D(\vf\gamma_1-\vf\gamma_2)$ is the domain enclosed between $\vf\gamma_1$ and $\vf\gamma_2$.}. Then, $\Ind(\vf\gamma_1,q)=\Ind(\vf\gamma_2,q)$.
  \end{proposition}
  \begin{definition}
    Let $\vf\gamma:I\rightarrow\RR^2$ be a closed path. We say that $\vf\gamma$ is \emph{contractible} if it is homotopic to the constant path ${\vf\alpha}(t)=a\in\RR^2$.
  \end{definition}
  \begin{proposition}
    Let $\vf\gamma:I\rightarrow\RR^2$ be a closed contractible path and $q\in\RR^2\setminus \overline{D(\vf\gamma)}$, where $D(\vf\gamma)$ is the domain enclosed by $\vf\gamma$. Then, $\Ind(\vf\gamma,q)=0$.
  \end{proposition}
  \begin{proposition}
    Let $\vf\gamma:I\rightarrow\RR^2$ be a closed path which is homotopic to the path $\vf\alpha_n:=q+\exp{2\pi\ii n t}$ (thought in $\RR^2$), $n\in\ZZ$ and $q\in\RR^2\setminus \overline{D(\vf\gamma-\vf\alpha_n)}$. Then, $\Ind(\vf\gamma,q)=n$.
  \end{proposition}
  \begin{proposition}
    Let $\vf\gamma:I\rightarrow\RR^2$ be a closed path and $q\in\RR^2\setminus\vf\gamma^*$ be such that $\Ind(\vf\gamma,q)=n$. Then, $\vf\gamma\homotop\vf\alpha_n$.
  \end{proposition}
  \begin{theorem}
    Let $\vf\gamma_1,\vf\gamma_2:I\rightarrow\RR^2$ be two closed paths and $q\in\RR^2\setminus \overline{D(\vf\gamma_1-\vf\gamma_2)}$. Then, $\vf\gamma_1\sim\vf\gamma_2\iff\Ind(\vf\gamma_1,q)=\Ind(\vf\gamma_2,q)$
  \end{theorem}
  \begin{theorem}
    Let $\vf{f}:\overline{D(0,1)}=[0,1]\times[0,2\pi]\rightarrow\RR^2$ be a continuous function, $\vf\gamma(t)=\vf{f}(1,2\pi t)$ with $t\in[0,1]$ and $q\in\RR^2\setminus \vf{f}(S^1)$ be such that $\Ind(\vf\gamma,q)\ne 0$. Then, $q\in \im\vf{f}$.
  \end{theorem}
  \subsubsection{Poincaré-Hopf theorem}
  \begin{definition}
    Let $U\subseteq \RR^2$ be an open set, $\vf{X}:U\rightarrow\RR^2$ be a differentible vector field and $\vf\alpha$ be the path defined on the boundary of a closed disk $D\subseteq U$. Let $\vf\gamma(t):=(\vf{X}\circ\vf\alpha)(t)$ and $q\in\Int(\vf\gamma)$. We define the \emph{index} of $\vf{X}$ on $\Fr{D}$ as: $$\Ind_{\Fr{D}}(\vf{X}):=\Ind(\vf\gamma,q)\footnote{It can be seen that this definition doesn't depend on the point $q$ inside $\vf\gamma$ chosen.}$$
  \end{definition}
  \begin{definition}
    Let $U\subseteq \RR^2$ be an open set, $\vf{X}:U\rightarrow\RR^2$ be a differentible vector field and $p\in U$ be an isolated singular point (on the set of all singular points). Let $D$ be a disk that surrounds only that singular point $p$. We define the \emph{index} of $p$ as: $$\Ind_{p}(\vf{X}):=\Ind_{\Fr{D}}(\vf{X})\footnote{It can be seen that this definition doesn't depend on the disk $D$ chosen.}$$
  \end{definition}
  \begin{proposition}
    Let $\overline{D}\subseteq \RR^2$ be a closed disk and $\vf{X}:\overline{D}\rightarrow\RR^2$ be a continuous vector field such that $\vf{X}(q)\ne 0$ $\forall q\in\Fr{\overline{D}}$. Suppose that $\vf{X}$ has a finite number of singular points $p_1,\ldots,p_n$. Then: $$\sum_{i= 1}^n\Ind_{p_i}(\vf{X})=\Ind_{\Fr{D}}(\vf{X})$$
  \end{proposition}
  \begin{definition}
    Let $U\subseteq S^2$ be an open set. A \emph{tangent vector field} defined on $S^2$ is a vector field $\vf{X}$ such that $\vf{X}(q)\in T_pS^2$ $\forall q\in U$\footnote{Recall \cref{DG_tangent}.}.
  \end{definition}
  \begin{definition}
    Let $U\subseteq S^2$ be an open set and $\vf{X}:U\rightarrow\RR^3$ be a tangent vector field and $p$ be a singular point of $\vf{X}$. Suppose (rotating the sphere if necessary) that $p$ is in one of its poles. Let $\vf{\tilde{X}}$ be the field created from the stereographic projection from $-p$ to the equator plane. We define the index of $p$ with respect to the field $\vf{X}$ as: $$\Ind_p(X)=\Ind_0(\vf{\tilde{X}})$$
  \end{definition}
  \begin{theorem}[Poincaré-Hopf theorem]
    Consider a continuous vector field $\vf{X}$ on a compact manifold $M$ with a finite number of singular points. Then, the sum of their indices is $\chi(M)$.
  \end{theorem}
  \begin{corollary}[Poincaré-Hopf theorem on $S^2$]
    Consider a continuous vector field $\vf{X}$ on $S^2$ with a finite number of singular points. Then, the sum of their indices is 2.
  \end{corollary}
  \begin{proposition}[Poincaré index formula]
    Let $U\subseteq \RR^2$ be an open set, $\vf{X}:U\rightarrow\RR^2$ be a differentible vector field and $p$ be a singular point with a finite finite sectorial decomposition. Denote by $e$ the number of elliptic sectors; by $h$, the number of hyperbolic sectors, and by $p$, the number of parabolic sectors. Then: $$\Ind_{p}(\vf{X})=\frac{e-h}{2}+1$$
  \end{proposition}
  \begin{corollary}
    Every tangent vector field $\vf{X}$ defined on $S^2$ has singular points.
  \end{corollary}
  \subsection{Introduction to partial differential equations}
  \begin{definition}
    Let $U\subseteq \RR^n$ be an open set. A \emph{partial differential equation} (\emph{pde}) of order $k$ is an expression of the form $$F\left(\vf{x},u(\vf{x}),\pdv{u}{\vf{x}},\ldots,\pdv[k]{u}{\vf{x}}\right)=0$$ where $\vf{x}=(x_1,\ldots,x_n)$, $F:U\times\RR\times\RR^{n^1}\times\cdots\times\RR^{n^k}\rightarrow\RR$ is a given function and $u:U\rightarrow \RR$ is an unknown function. The function $u$ is called \emph{solution} of the pde defined by $F$.
  \end{definition}
  \subsubsection{Quasilinear partial differential equations}
  \begin{definition}
    Let $U\subseteq \RR^n$ be an open set and $u:U\rightarrow\RR$ be a function. A \emph{quasilinear pde} is an expression of the form:
    \begin{equation}\label{DE_pde1}
      p_1(\vf{x},u)\pdv{u}{x_1}+\cdots+p_n(\vf{x},u)\pdv{u}{x_n}=q(\vf{x},u)
    \end{equation}
  \end{definition}
  \begin{theorem}
    Let $U\subseteq \RR^n$ be an open set and $u:U\rightarrow\RR$ be a function and consider the pde of \cref{DE_pde1}. Let $H_1,\ldots,H_{n}$ be the $n$ independent first integrals of the system:
    $$
      \left\{
      \begin{aligned}
        {x_1}' & =p_1(x_1,\ldots,x_n,u) \\
               & \;\;\vdots             \\
        {x_n}' & =p_n(x_1,\ldots,x_n,u) \\
        {u}'   & = q(x_1,\ldots,x_n,u)
      \end{aligned}
      \right.
    $$
    Then, for any function $F:\RR^n\rightarrow\RR$ of class $\mathcal{C}^1$, the implicit equation $$F(H_1(\vf{x},u),\ldots,H_n(\vf{x},u))=0$$ is a solution to \cref{DE_pde1}.
  \end{theorem}
  \subsubsection{Heat, wave and Laplace equations}
  \begin{definition}[Heat equation]
    Let $u:\RR\times\RR\rightarrow\RR$ be an unknown function. The \emph{heat equation} is the pde defined by $$\pdv{u}{t}=k\pdv[2]{u}{x}$$ where $k\in\RR$.
  \end{definition}
  \begin{proposition}
    Consider a bar of line $L\in\RR_{>0}$ whose temperature can be modeled by a function $u:\RR\times\RR\rightarrow\RR$, and $f:[0,L]\rightarrow\RR$ be a function. Then, the solution $u(x,t)$ to the heat equation with boundary conditions $u(x,0)=f(x)$ and $u(0,t)=u(L,t)=0$ is: $$u(x,t)=\sum_{n=1}^\infty b_n\sin\left(\frac{\pi n x}{L}\right)\exp{-\frac{n^2\pi^2k}{L^2}t}$$ where $\displaystyle b_n=\frac{1}{L}\int_{-L}^Lf(x)\sin\left(\frac{\pi n x}{L}\right)\dd{x}$.
  \end{proposition}
  \begin{definition}[Wave equation]
    Let $u:\RR\times\RR\rightarrow\RR$ be an unknown function. The \emph{Wave equation} is the pde defined by $$\pdv[2]{u}{t}=c^2\pdv[2]{u}{x}$$ where $c\in\RR$.
  \end{definition}
  \begin{proposition}
    Consider a string of line $L\in\RR_{>0}$ whose position can be modeled by a function $u:\RR\times\RR\rightarrow\RR$, and $f,g:[0,L]\rightarrow\RR$ be functions. Then, the solution $u(x,t)$ to the wave equation with boundary conditions $u(x,0)=f(x)$, $u_t(x,0)=g(x)$ and $u(0,t)=u(L,t)=0$ is: $$u(x,t)=\sum_{n=0}^\infty \sin\left(\frac{\pi n x}{L}\right)\left[a_n\cos\left(\frac{\pi n c}{L}t\right)+ b_n\sin\left( \frac{\pi n c}{L}t\right)\right]$$ where:
    \begin{align*}
      a_n & =\frac{1}{L}\int_{-L}^Lf(x)\cos\left(\frac{\pi n x}{L}\right)\dd{x}       \\
      b_n & =\frac{1}{\pi n c}\int_{-L}^Lg(x)\sin\left(\frac{\pi n x}{L}\right)\dd{x}
    \end{align*}
  \end{proposition}
  \begin{proposition}
    Let $u(x,t)$ be a solution to the wave equation. Then, $\exists F,G:\RR\rightarrow\RR$ such that: $$u(x,t)=F(x+ct)+G(x-ct)$$
  \end{proposition}
  \begin{proposition}[D'Alembert formula]
    Let $f,g:[0,L]\rightarrow\RR$ be functions. The solution $u(x,t)$ to the wave equation with boundary conditions $u(x,0)=f(x)$ and $u_t(x,0)=g(x)$ is: $$u(x,t)=\frac{f(x-ct)+f(x+ct)}{2}+\frac{1}{2c}\int_{x-ct}^{x+ct}g(s)\dd{s}$$
  \end{proposition}
  \begin{definition}[Laplace equation]
    Let $u:\RR\times\RR\rightarrow\RR$ be an unknown function. The \emph{Laplace equation} is the pde defined by: $$\pdv[2]{u}{x}+\pdv[2]{u}{y}=\laplacian u=0$$
  \end{definition}
  \begin{proposition}
    The Laplacian of a function $u:(0,\infty)\times[0,2\pi]\rightarrow\RR$ in polar coordinates $(r,\theta)$ is: $$\laplacian u=u_{rr}+\frac{u_r}{r}+\frac{u_{\theta\theta}}{r^2}$$
  \end{proposition}
  \begin{proposition}[Dirichlet problem]
    Let $f:[0,2\pi]\rightarrow\RR$ be a continuous function such that $f(0)=f(2\pi)$. Then, there exists a continuous function $v:\overline{D(0,\rho)}\rightarrow\RR$ such that:
    \begin{enumerate}
      \item $v(r,0)=v(r,2\pi)$ $\forall r\in[0,\rho]$
      \item $v\in\mathcal{C}^2(D(0,\rho)\setminus\{0\})$ and $\laplacian v=0$.
      \item $v(\rho,\theta)=f(\theta)$ $\forall\theta\in[0,2\pi]$
    \end{enumerate}
    An example of such function is:
    $$v(r,\theta)=\sum_{n=0}^\infty \frac{r^n}{\rho^n}\left[a_n\cos\left(n\theta\right)+ b_n\sin\left(n\theta\right)\right]$$ where:
    \begin{align*}
      a_n & =\frac{1}{\pi}\int_{0}^{2\pi} f(\theta)\cos\left(n\theta\right)\dd{\theta} \\
      b_n & =\frac{1}{\pi}\int_{0}^{2\pi} f(\theta)\sin\left(n\theta\right)\dd{\theta}
    \end{align*}
  \end{proposition}
\end{multicols}
\end{document}