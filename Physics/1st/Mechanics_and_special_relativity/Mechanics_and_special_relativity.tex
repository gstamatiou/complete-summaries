\documentclass[../../../main.tex]{subfiles}


\begin{document}
\begin{multicols}{2}[\section{Mechanics and special relativity}]
  \subsection{Mechanics}
  \subsubsection{Kinematics}
  \begin{definition}
    The \textit{equation of movement} of any particle is of the form:
    $$\vectorfunction{r}(t)=x(t)\vectorfunction{e}_x+y(t)\vectorfunction{e}_y+z(t)\vectorfunction{e}_z$$ where $x(t)$, $y(t)$, $z(t)$ are the movements equations of the particle along $x$-, $y$- and $z$-axis, respectively.
  \end{definition}
  \begin{definition}
    Consider a particle with movement equation $\vectorfunction{r}(t)$. Then, the \textit{average velocity over any time interval $\Delta t=t_2-t_1$} is:
    $$\vectorfunction{v}_\text{avg}=\frac{\Delta\vectorfunction{r}(t)}{\Delta t}=\frac{\vectorfunction{r}(t_2)-\vectorfunction{r}(t_1)}{t_2-t_1}$$
    If we take the limit when $\Delta t\to0$ (or $t_2\to t_1$), we get the \textit{instantaneous velocity at time $t_1$}: $$\vectorfunction{v}(t_1)=\lim_{\Delta t\to 0}\frac{\Delta\vectorfunction{r}(t)}{\Delta t}=\dot{\vectorfunction{r}}(t_1)=\dot{x}(t_1)\vectorfunction{e}_x+\dot{y}(t_1)\vectorfunction{e}_y+\dot{z}(t_1)\vectorfunction{e}_z$$
  \end{definition}
  \begin{definition}
    The \textit{speed} of a particle moving at a velocity $\vectorfunction{v}(t)$ is:
    $$v(t)=\|\vectorfunction{v}(t)\|$$
  \end{definition}
  \begin{definition}
    Consider a particle moving at a velocity $\vectorfunction{v}(t)$. Then the \textit{average acceleration over any time interval $\Delta t=t_2-t_1$} is:
    $$\vectorfunction{a}_\text{avg}=\frac{\Delta\vectorfunction{v}(t)}{\Delta t}=\frac{\vectorfunction{v}(t_2)-\vectorfunction{v}(t_1)}{t_2-t_1}$$
    If we take the limit when $\Delta t\to0$ (or $t_2\to t_1$), we get the \textit{instantaneous acceleration at time $t_1$}: $$\vectorfunction{a}(t_1)=\lim_{\Delta t\to 0}\frac{\Delta\vectorfunction{v}(t)}{\Delta t}=\ddot{\vectorfunction{r}}(t_1)=\ddot{x}(t_1)\vectorfunction{e}_x+\ddot{y}(t_1)\vectorfunction{e}_y+\ddot{z}(t_1)\vectorfunction{e}_z$$
  \end{definition}
  \begin{prop}[Uniform linear motion]
    Consider a particle moving at a constant speed $v$ along a straight line. If at time $t=0$ it is in the position $x_0$, then: $$x(t)=x_0+vt$$
  \end{prop}
  \begin{prop}[Accelerated linear motion]
    Consider a particle moving at a constant acceleration $a$ along a straight line. If at time $t=0$ it is in the position $x_0$ with velocity $v_0$, then:
    $$\dot{x}(t)=v_0+at\qquad x(t)=x_0+v_0t+\frac{1}{2}at^2$$
  \end{prop}
  \begin{definition}
    Suppose a particle is at Cartesian coordinates $(x,y)$ and at polar coordinates $(r,\varphi)$. Then, polar unit vectors are defined as:
    \begin{gather*}
      \vectorfunction{e}_r=\cos\varphi\vectorfunction{e}_x+\sin\varphi\vectorfunction{e}_y\\
      \vectorfunction{e}_\varphi=-\sin\varphi\vectorfunction{e}_x+\cos\varphi\vectorfunction{e}_y
    \end{gather*}
  \end{definition}
  \begin{definition}
    The equations of the circular movement are the following: $$\vectorfunction{r}(t)=r\vectorfunction{e}_r\quad\;\dot{\vectorfunction{r}}(t)=r\dot{\varphi}(t)\vectorfunction{e}_\varphi\quad\;\ddot{\vectorfunction{r}}(t)=r\ddot{\varphi}(t)\vectorfunction{e}_\varphi-r{\dot{\varphi}(t)}^2\vectorfunction{e}_r$$ where we have supposed that $r$ is constant. We define the \textit{angular velocity $\omega(t)$} as $\omega(t):=\dot{\varphi}(t)$ and the \textit{angular acceleration $\alpha(t)$} as $\alpha(t):=\ddot{\varphi}(t)$. The first term of $\ddot{\vectorfunction{r}}(t)$ is called \textit{tangential acceleration} and its magnitude is $a_t:=r\alpha$. The second term is called \textit{normal acceleration} and its magnitude is $a_n:=r\omega^2$.
  \end{definition}
  \begin{definition}
    Consider a particle moving along the trajectory $\vectorfunction{r}(t)$. We define Frenet vectors as:
    \begin{enumerate}
      \item First Frenet vector: $\displaystyle\vectorfunction{e}_1(t)=\frac{\dot{\vectorfunction{r}}(t)}{\|\dot{\vectorfunction{r}}(t)\|}$
      \item Second Frenet vector: $\displaystyle\vectorfunction{e}_2(t)=\frac{\dot{\vectorfunction{e}}_1(t)}{\|\dot{\vectorfunction{e}}_1(t)\|}$
    \end{enumerate}
    Note that the first vector is tangent to the trajectory at each point and the second one is normal to the trajectory at each point.

    From this definition we have: $$\dot{\vectorfunction{r}}(t)=v(t)\vectorfunction{e}_1\qquad\ddot{\vectorfunction{r}}(t)=a_t(t)\vectorfunction{e}_1+a_n(t)\vectorfunction{e}_2(t)$$ We also define the \textit{curvature $\kappa(t)$} and \textit{radius of curvature $R(t)$} as: $$\frac{1}{\kappa(t)}=R(t):=\frac{\|\dot{\vectorfunction{r}}(t)\|}{\|\dot{\vectorfunction{e}}_1(t)\|}$$ Finally, the normal acceleration is: $$a_n(t)=\frac{v(t)^2}{R(t)}$$
  \end{definition}
  \begin{prop}[Curvature]
    Consider a particle moving along a two-dimensional trajectory and let $\Delta\varphi$ be the angle the trajectory has curved when traveling a distance $\Delta s$. Then the \textit{average curvature along $\Delta s$} is: $$\kappa_\text{avg}=\frac{\Delta \varphi}{\Delta s}$$ If we take the limit when $\Delta s\to 0$ we have: $$\kappa=\lim_{\Delta s\to 0}\frac{\Delta \varphi}{\Delta s}=\dv{\varphi}{s}$$
  \end{prop}
  \begin{prop}[Arc length]
    The total distance traveled by a particle moving along a curve $\vectorfunction{r}(t)$ between the instants $t_1$ and $t_2$ is: $$\int_{t_1}^{t_2}\|\dot{\vectorfunction{r}}(t)\|\dd t$$
  \end{prop}
  \begin{prop}[Projectile motion]
    The equations of a projectile motion like the one in \cref{projectile} are:
    \begin{align*}
      x(t)=x_0+v_0\cos\theta t\quad & \quad y(t)=y_0+v_0\sin\theta t-\frac{1}{2}gt^2 \\
      v_x(t)=v_0\cos\theta\quad     & \quad v_y(t)=v_0\sin\theta-gt
    \end{align*}
    \begin{center}
      \begin{minipage}{\linewidth}
        \centering
        \includestandalone[mode=image|tex,width=\linewidth]{Images/projectile}
        \captionof{figure}{}
        \label{projectile}
      \end{minipage}
    \end{center}
  \end{prop}
  \subsubsection{Dynamics}
  \begin{law}[Newton's laws]
    \hfill
    \begin{enumerate}
      \item An object at rest will stay at rest and an object in motion will stay in motion unless acted on by a net external force. That is: $$\sum\vectorfunction{F}=0\iff\dv{\vectorfunction{v}}{t}=0$$
      \item The rate of change of momentum of a body over time is directly proportional to the force applied, and occurs in the same direction as the applied force. That is: $$\vectorfunction{F}=\dv{\vectorfunction{p}}{t}$$
      \item If one object $A$ exerts a force $\vectorfunction{F}_A$ on a second object $B$, then $B$ simultaneously exerts a force $\vectorfunction{F}_B$ on $A$ and the two forces are equal in magnitude and opposite in direction: $$\vectorfunction{F}_A=-\vectorfunction{F}_B$$
    \end{enumerate}
  \end{law}
  \begin{prop}[Gravity force]
    Any two object with mass $m_1$ and $m_2$ exerts an attracting force called \textit{gravity}:
    $$\vectorfunction{F}_{21}=-G\frac{m_1m_2}{{|\vectorfunction{r}_{12}|}^3}\vectorfunction{r}_{12}$$
    where $\vectorfunction{F}_{21}$ is the force applied on object 2 exerted by object 1, $\vectorfunction{r}_{12}$ is the vector distance from object 1 to object 2.
  \end{prop}
  \begin{prop}[Elastic force]
    Consider an object attached to a string of natural length $x_0$ as shown in the \cref{elastic_force}.
    \begin{center}
      \begin{minipage}{\linewidth}
        \centering
        \includestandalone[mode=image|tex,width=0.6\linewidth]{Images/elastic_force}
        \captionof{figure}{}
        \label{elastic_force}
      \end{minipage}
    \end{center}
    If we displace the object a distance of $x$ from its equilibrium position, the resulting elastic force is: $$\vectorfunction{F}=-k\vectorfunction{x}$$ where $k$ is the spring constant. Moreover, ignoring the friction, the mass starts to oscillate and this oscillation have the following equations:
    \begin{gather*}
      x(t)=A\cos(\omega t+\phi)\\
      \dot{x}(t)=-\omega A\sin(\omega t+\phi)\\
      \ddot{x}(t)=-\omega^2 A\cos(\omega t+\phi)=-\omega^2x(t)\\
      \omega=\sqrt{\frac{k}{m}}\qquad T=\frac{2\pi}{\omega}\qquad\nu=\frac{1}{T}
    \end{gather*}
    where $A$ is the amplitude, $\phi$ is the initial phase, $\omega$ is the angular frequency, $T$ is the period and $\nu$ is the frequency.
  \end{prop}
  \begin{prop}
    Consider an object on a surface that undergo a normal force $\vectorfunction{F}_N$ and it is pulled by a net force of magnitude $F$. Then the magnitude of the frictional force is:
    $$
      F_f=\left\{
      \begin{array}{rcl}
        F        & \text{if} & F\leq\mu_sF_N \\
        \mu_kF_N & \text{if} & F>\mu_sF_N
      \end{array}\right.
    $$
    where $\mu_s$ is the \textit{static coefficient of friction} and $\mu_k$ is the \textit{kinetic coefficient of friction}.
  \end{prop}
  \begin{prop}[Inertial forces]
    Consider two general reference frames $\mathcal{R}$ and $\mathcal{R}'$ (separated by $\vectorfunction{R}(t)$) and suppose that we observe a particle of mass $m$ at position $\vectorfunction{r}(t)$ from $\mathcal{R}$ and at position $\vectorfunction{r}'(t)$ from $\mathcal{R}'$, as shown in the figure:
    \begin{center}
      \begin{minipage}{\linewidth}
        \centering
        \includestandalone[mode=image|tex,width=0.6\linewidth]{Images/inertial_force}
        \captionof{figure}{}
      \end{minipage}
    \end{center}
    Then for a general $\vectorfunction{R}(t)$ we have $\vectorfunction{r}'(t)=\vectorfunction{r}(t)-\vectorfunction{R}(t)$ and therefore $\ddot{\vectorfunction{r}}'(t)=\ddot{\vectorfunction{r}}(t)-\ddot{\vectorfunction{R}}(t)$. If we assume that $\mathcal{R}$ is inertial, then $$\vectorfunction{F}(t)-m\ddot{\vectorfunction{R}}(t)=m\ddot{\vectorfunction{r}}'(t)$$ If $\mathcal{R}'$ is not inertial, Newton's second law is not satisfied. In this case, we denote the term $-m\ddot{\vectorfunction{R}}(t)$ as an \textit{inertial force} or \textit{fictitious force}: $\vectorfunction{F}_\text{iner}(t):=-m\ddot{\vectorfunction{R}}(t)$\footnote{Note that if $\mathcal{R}'$ is inertial, Newton's second law is still satisfied because $\vectorfunction{R}(t)=\vectorfunction{V}t$ and therefore $-m\ddot{\vectorfunction{R}}(t)=0$.}.
  \end{prop}
  \begin{prop}[Galilean transformation]
    Consider two reference frames $\mathcal{R}$ and $\mathcal{R}'$. Using the previous notation, suppose $\vectorfunction{R}(t)=Vt\vectorfunction{e}_x$. Then:
    \begin{gather*}
      \begin{aligned}
        x' & =x-Vt & \hspace{1cm} v_x' & =v_x-V \\
        y' & =y    & v_y'              & =v_y   \\
        z' & =z    & v_z'              & =v_z   \\
      \end{aligned}\\
      t'=t
    \end{gather*}
  \end{prop}
  \subsubsection{Statics}
  \begin{definition}[Linear momentum of a particle]
    Consider a particle of mass $m$ moving at a velocity of $\vectorfunction{v}$. We define its \textit{linear momentum} as:
    $$\vectorfunction{p}=m\vectorfunction{v}$$
  \end{definition}
  \begin{prop}[Linear momentum of a system of particles]
    Consider a system of $N$ particles which interact with themselves (internal forces) and also with external forces. The \textit{linear momentum of the system} is: $$\vectorfunction{P}=\sum_{a=1}^N\vectorfunction{p}_a$$ Moreover if the net external force is $\vectorfunction{F}_\text{ext}$ we have:
    $$\dv{\vectorfunction{P}}{t}=\vectorfunction{F}_\text{ext}$$
  \end{prop}
  \begin{prop}[Center of masses]
    The \textit{center of masses} (CM) of a system of $N$ particles is: $$\vectorfunction{R}=\frac{1}{M}\sum_{i=1}^Nm_i\vectorfunction{r}_i$$ where $\displaystyle M=\sum_{i=1}^Nm_i$. Differentiating the last equality we get
    $$M\dot{\vectorfunction{R}}=\vectorfunction{P}\qquad M\ddot{\vectorfunction{R}}=\dot{\vectorfunction{P}}=\vectorfunction{F}_\text{ext}$$
    If the mass distribution is continuous with the density $\rho(\vectorfunction{r})$ within a solid $\Omega$, the center of mass is: $$\vectorfunction{R}=\frac{1}{M}\iiint_\Omega\rho(\vectorfunction{r})\vectorfunction{r}\dd V$$ where $\displaystyle M=\iiint_\Omega\rho(\vectorfunction{r})\dd V$.
  \end{prop}
  \begin{prop}[Angular momentum]
    Consider a particle with linear momentum $\vectorfunction{p}$ situated at position $\vectorfunction{r}$ with respect to the origin $O$. We define its \textit{angular momentum} as: $$\vectorfunction{L}=\vectorfunction{r}\crossprod\vectorfunction{p}$$ The angular momentum of a system of $N$ particles is: $$\vectorfunction{L}_\text{sys}=\sum_{i=1}^N\vectorfunction{L}_i$$
  \end{prop}
  \begin{prop}[Torque]
    Consider a particle at position $\vectorfunction{r}$ with respect to the origin $O$ and let $\vectorfunction{F}$ be a force acting on the particle. We define the \textit{torque} as: $$\vectorfunction{\tau}=\vectorfunction{r}\crossprod\vectorfunction{F}$$ The torque of a system of $N$ particles is: $$\vectorfunction{\tau}_\text{ext}=\sum_{i=1}^N\vectorfunction{\tau}_i$$
  \end{prop}
  \begin{prop}
    Relating the torque and angular momentum of a particle and a system of particles we have:
    $$\dot{\vectorfunction{L}}=\vectorfunction{\tau}\qquad\dot{\vectorfunction{L}}_\text{sys}=\vectorfunction{\tau}_\text{ext}$$ Therefore, if $\vectorfunction{\tau}_\text{ext}=0$, then $\vectorfunction{L}_\text{sys}=\text{const}.$
  \end{prop}
  \begin{definition}[Mechanical equilibrium]
    The conditions of mechanical equilibrium are: $$\vectorfunction{F}_\text{ext}=0\quad\text{and}\quad\vectorfunction{\tau}_\text{ext}=0$$
  \end{definition}
  \subsubsection{Work and energy}
  \begin{definition}[Work]
    The \textit{work} of a constant force $\vectorfunction{F}$ acting on a particle that moves throughout a straight distance $\Delta\vectorfunction{r}$ is: $$ W=\vectorfunction{F}\cdot\Delta\vectorfunction{r}$$ If the force is not necessary constant and the particle moves along a curve $c$, we have: $$W=\int_c\vectorfunction{F}\cdot \dd\vectorfunction{r}$$
  \end{definition}
  \begin{definition}[Power]
    The \textit{power} is defined as $$P=\dv{W}{t}$$ If $\Delta W$ is the amount of work performed during a period of time of duration $\Delta t$, the \textit{average power} is: $$P=\frac{\Delta W}{\Delta t}$$ From the first definition we can deduce the following general formula: $$P=\vectorfunction{F}\cdot\vectorfunction{v}$$
  \end{definition}
  \begin{definition}[Kinetic energy]
    The \textit{kinetic energy} of a particle of mass $m$ moving at a speed $v$ is: $$K=\frac{1}{2}mv^2$$
  \end{definition}
  \begin{theorem}
    The total work done on a particle is:
    $$W=\Delta K$$
  \end{theorem}
  \begin{definition}[Conservative forces]
    A force is \textit{conservative} if for any path $c$ connecting points $A$ and $B$, the work necessary to move a particle from $A$ to $B$ does not depend on $c$.
  \end{definition}
  \begin{prop}
    The work done by a conservative force can be expressed as a variation of a function called \textit{potential energy}.
  \end{prop}
  \begin{prop}[Potential energy]
    If a force $\vectorfunction{F}$ is conservative, we define the potential energy as: $$U(\vectorfunction{r})=-\int_{\vectorfunction{r}_0}^{\vectorfunction{r}}\vectorfunction{F}\cdot d\vectorfunction{r}$$ where $\vectorfunction{r}_0$ is a reference point and can be chosen arbitrarily. It can be easily seen that: $$W=-\Delta U$$
  \end{prop}
  \begin{prop}[Mechanical energy]
    The mechanical energy of a particle (with kinetic energy $K$) subjected to a conservative force of potential energy $U$ is: $$E=K+U$$
  \end{prop}
  \begin{theorem}[Conservation of mechanical energy]
    For a particle subjected to a conservative force we have: $$\Delta E=0$$ That is, $E$ is constant. If there are non-conservative forces acting on the particle we have: $$\Delta E=W_{\text{nc}}$$ where $W_\text{nc}$ is the work done by non-conservative forces.
  \end{theorem}
  \begin{prop}[Examples of potential energies]
    \hfill
    \begin{enumerate}
      \item Elastic potential energy of a spring: $$U=\frac{1}{2}kx^2$$ where $x$ is the distance the spring has been stretched.
      \item Gravitational potential energy of a solid of mass $m$: $$U=-\frac{GM_Tm}{r}$$ where $M_T$ is the Earth mass, $r$ is the distance form the center of the earth to the position of the solid and $G$ is the gravitational constant. Note that if $r=R_T+h$, $h>0$ and $\frac{r}{R_T}=1+\frac{h}{R_T}\approx 1$, then: $$U=mgh$$ where $R_T$ is the radius of earth and $g$ is the surface gravity.
    \end{enumerate}
  \end{prop}
  \subsubsection{Rotation}
  \begin{definition}
    Consider a system of $N$ particles that spin around a reference axis at an angular velocity $\vectorfunction{\omega}$. The \textit{moment of inertia} $I$ with respect to the axis is: $$I=\sum_{i=1}^Nm_i{r_i}^2$$
    where $m_i$ is the mass of the $i$-th particle and $r_i$ is the distance between that particle and the axis. Moreover we have: $$\vectorfunction{L}_\text{sys}=I\vectorfunction{\omega}$$
  \end{definition}
  \begin{prop}
    For a rigid body of moment of inertia $I$ that spins around a reference axis at an angular velocity $\vectorfunction{\omega}$ we have: $$\vectorfunction{\tau}_\text{ext}=I\dot{\vectorfunction{\omega}}=I\vectorfunction{\alpha}$$
  \end{prop}
  \begin{prop}
    Consider a system of particles whose CM is at a distance $\vectorfunction{R}(t)$ from a fixed point $O$. If $\vectorfunction{P}$ is the linear momentum of the CM, we have: $$\vectorfunction{L}_O=\vectorfunction{L}_\text{CM}+\vectorfunction{R}\crossprod\vectorfunction{P}$$ where $\vectorfunction{L}_O$ is the angular momentum of the system with respect to the point $O$ and $\vectorfunction{L}_\text{CM}$ is the angular momentum of the system with respect to the CM. Moreover if $\vectorfunction{F}_\text{ext}$ is the total external force applied onto the system, $\vectorfunction{\tau}_\text{O,ext}$ is the torque done by the forces with respect to the point $O$ and $\vectorfunction{\tau}_\text{CM,ext}$ is the torque done by the forces with respect to the CM, we have: $$\vectorfunction{\tau}_\text{O,ext}=\vectorfunction{\tau}_\text{CM,ext}+\vectorfunction{R}\crossprod\vectorfunction{F}_\text{ext}$$ Finally, we deduce: $$\dot{\vectorfunction{L}}_{CM}=\vectorfunction{\tau}_\text{CM,ext}$$
  \end{prop}
  \begin{prop}
    Consider a system of particles with total mass $M$. Suppose the moment of inertia of the system with respect to the CM is $I_\text{CM}$ and that the speed of the CM is $V$. If the angular velocity of the system around the CM is $\omega$, the kinetic energy of rotation will be: $$K=\frac{1}{2}MV^2+\frac{1}{2}I_\text{CM}\omega^2$$
  \end{prop}
  \begin{theorem}[Parallel axis theorem]
    Consider a body of mass $m$ that is rotating around an axis that passes through the body's center of mass. Let $I_\text{CM}$ be the moment of inertia with respect of that axis. Suppose there is another axis parallel to the previous one and separated each other a distance of $d$. Then, the moment of inertia of the body with respect to this latter axis $I$ will be:
    $$I=I_\text{CM}+md^2$$
  \end{theorem}
  \subsection{Special relativity}
  \begin{definition}
    A \textit{inertial frame of reference} is a frame of reference in which a particle remains at rest or in uniform linear motion.
  \end{definition}
  \begin{principle}[First postulate]
    The laws of physics take the same form in all inertial frames of reference.
  \end{principle}
  \begin{principle}[Second postulate]
    The speed of light, $c$, is a constant, independent of the relative motion of the source.
  \end{principle}
  \begin{definition}[Lorentz factor]
    For an object moving at speed $v$, \textit{Lorentz factor} is defined as:
    $$\gamma=\frac{1}{\sqrt{1-\beta^2}}$$ where $\beta=v/c$.
  \end{definition}
  \begin{prop}[Time dilation]
    Consider two frames of reference in uniform relative motion with velocity $v$ such that one of them has a clock. If $\Delta t_0$ is the time interval between two events made in the same location and measured in the frame in which the clock is at rest (\textit{proper time}), then the time measured by the other frame is:
    $$\Delta t=\gamma\Delta t_0$$
  \end{prop}
  \begin{prop}[Length contraction]
    Consider two frames of reference in uniform relative motion with velocity $v$ such that one of them has an object. If $L_0$ is length of the object measured instantaneously in the frame in which the object is at rest (\textit{proper length}), then the length measured by the other frame is:
    $$L=\frac{L_0}{\gamma}$$
  \end{prop}
  \begin{prop}[Lorentz transformations]
    Consider coordinates $(x,y,z,t)$ and $(x',y',z',t')$ of a single arbitrary event measured in two coordinate systems $S$ and $S'$, in uniform relative motion ($S'$ is moving at velocity $\vectorfunction{v}=(v,0,0)$ with respect to $S$) in their common $x$ and $x'$ directions and  with their spatial origins coinciding at time $t=t'=0$. Then:
    \begin{align*}
      x'  & =\gamma(x-\beta ct) & x  & =\gamma(x'+\beta ct') \\
      y'  & =y                  & y  & =y'                   \\
      z'  & =z                  & z  & =z'                   \\
      ct' & =\gamma(ct-\beta x) & ct & =\gamma(ct'+\beta x')
    \end{align*}
  \end{prop}
  \begin{prop}[Lorentz transformations of velocities]
    In a situation similar to the previous one, if an object is moving at a velocity $\vectorfunction{u}=(u_x,u_y,u_z)$ in $S$ and $\vectorfunction{u'}=(u_x',u_y',u_z')$ in $S'$, we have:
    \begin{align*}
      u_x' & =\frac{u_x-v}{1-u_xv/c^2}                   & u_x & =\frac{u_x'+v}{1+u_x'v/c^2}                  \\
      u_y' & =\frac{u_y}{\gamma \left(1-u_xv/c^2\right)} & u_y & =\frac{u_y'}{\gamma \left(1+u_xv/c^2\right)} \\
      u_z' & =\frac{u_z}{\gamma \left(1-u_xv/c^2\right)} & u_z & =\frac{u_z'}{\gamma \left(1+u_xv/c^2\right)}
    \end{align*}
  \end{prop}
  \begin{prop}[Matrix form of Lorentz transformations]
    We can write the Lorentz transformations as:
    $$\begin{pmatrix}
        x' \\
        ct'
      \end{pmatrix}=\begin{pmatrix}
        \gamma       & -\beta\gamma \\
        -\beta\gamma & \gamma
      \end{pmatrix}\begin{pmatrix}
        x \\
        ct
      \end{pmatrix}$$ If $$\Lambda:=\begin{pmatrix}
        \gamma       & -\beta\gamma \\
        -\beta\gamma & \gamma
      \end{pmatrix},\text{ then }\Lambda^{-1}=\begin{pmatrix}
        \gamma      & \beta\gamma \\
        \beta\gamma & \gamma
      \end{pmatrix}$$ and we obtain the inverse transformations.
  \end{prop}
  \begin{prop}[Lorentz invariant]
    The factor $s^2$, defined as follows, is invariant in any inertial frame of reference. $$s^2=c^2t^2-x^2-y^2-z^2$$
  \end{prop}
  \begin{prop}[Types of events]
    There are three types of events: \textit{timelike}, \textit{lightlike} and \textit{spacelike}.
    \begin{itemize}
      \item $s^2>0\implies\mathit{timelike}$
      \item $s^2=0\implies\mathit{lightlike}$
      \item $s^2<0\implies\mathit{spacelike}$
    \end{itemize}
    Timelike and lightlike events are in causal relation with the origin (that is, it is possible to send a light signal from the origin to the point or vice versa), while \textit{spacelike} events are not.
  \end{prop}
  \begin{center}
    \begin{minipage}{\linewidth}
      \centering
      \includestandalone[mode=image|tex,width=\linewidth]{Images/minkowski}
      \captionof{figure}{Minkowski diagram}
    \end{minipage}
  \end{center}
  \begin{prop}[Relativistic Doppler effect]
    Suppose a frame of reference where the receiver is at rest and the source is moving at speed $\beta$ forming an angle $\phi$ with the light direction (measured in receiver frame). Then:
    \begin{gather}
      \label{dopp1}\nu_R=\frac{\nu_S}{\gamma(1-\beta\cos\phi)}\\
      \lambda_R=\gamma\lambda_S(1-\beta\cos\phi)
    \end{gather} where $\nu_S$ is the frequency measured by the source and $\nu_R$ is the frequency measured by the receiver, and analogously with wavelengths $\lambda_S$ and $\lambda_R$.\newline Relation between the angles $\phi$ and $\phi'$, where $\phi'$ is the angle between the velocity and the light direction measured in source frame:
    $$\tan\frac{\phi'}{2}=\sqrt{\frac{1+\beta}{1-\beta}}\tan\frac{\phi}{2}$$
  \end{prop}
  \begin{center}
    \begin{minipage}{\linewidth}
      \centering
      \includestandalone[mode=image|tex,width=0.6\linewidth]{Images/doppler}
      \captionof{figure}{General case of Doppler effect}
    \end{minipage}
  \end{center}
  \begin{corollary}
    There are three important cases to consider:
    \begin{itemize}
      \item The source moves away, that is making $\phi=\pi$ in equation \eqref{dopp1} (\textit{Redshift}):
            $$\nu_R=\nu_S\sqrt{\frac{1-\beta}{1+\beta}}$$
      \item The source gets close, that is making $\phi=0$ in equation \eqref{dopp1} (\textit{Blueshift}):
            $$\nu_R=\nu_S\sqrt{\frac{1+\beta}{1-\beta}}$$
      \item The source moves transversely, that is making $\phi=\pi/2$ in equation \eqref{dopp1}:$$\nu_R=\nu_S/\gamma$$
    \end{itemize}
  \end{corollary}
  \begin{prop}[Relativistic mass]
    If $m_0$ is the mass of an object at rest, then the mass of an object at a velocity $\beta$ is: $$m=\gamma m_0$$ The mass $m_0$ is invariant.
  \end{prop}
  \begin{prop}[Relativistic momentum]
    The relativistic momentum for a particle with mass at rest $m_0$ and moving at a velocity of $\vectorfunction{v}$ is given by: $$\vectorfunction{p}=\gamma m_0\vectorfunction{v}$$
  \end{prop}
  \begin{prop}[Relativistic energy]
    The relativistic energy of a particle is: $$E=mc^2=\gamma m_0c^2$$ On the other hand, $E=K+m_0c^2$, where $K$ is the kinetic energy of a particle and $m_0c^2$ its rest energy. Moreover we can express the energy of a particle in terms of its momentum:
    $$E=mc^2=\sqrt{p^2c^2+m_0^2c^4}$$
  \end{prop}
  \begin{prop}[Photon energy and momentum]
    For a photon of frequency $\nu$, energy $E$ and linear momentum $p$, we have:
    $$E=h\nu\qquad p=\frac{h\nu}{c}$$
  \end{prop}
  \begin{prop}[Lorentz transformations of energy and momentum]
    Consider a particle that have energy $E$ and momentum $\vectorfunction{p}=(p_x,p_y,p_z)$ in a frame of reference $S$ and have energy $E'$ and momentum $\vectorfunction{p}'=(p_x',p_y',p_z')$ in frame of reference $S'$. These frames are in uniform relative motion ($S'$ is moving at velocity $\vectorfunction{v}=(v,0,0)$ with respect to $S$) and their spatial origins coincide at time $t=t'=0$. Then:
    \begin{align*}
      E'    & =\gamma(E-\beta cp_x) & E    & =\gamma(E'+\beta cp_x') \\
      cp_x' & =\gamma(cp_x-\beta E) & cp_x & =\gamma(cp_x'+\beta E') \\
      p_y'  & =p_y                  & p_y  & =p_y'                   \\
      p_z'  & =p_z                  & p_z  & =p_z'
    \end{align*}
  \end{prop}
  \begin{prop}[Compton scattering]
    Consider a photon with wavelength $\lambda$ colliding with a particle at rest of mass $m_0$ (usually an electron). As a result of the collision, the photon energy decrease and therefore its wavelength increase (let's say the scattered photon has wavelength $\lambda'$). If the scattered photon is moving at an angle $\theta$ with respect to initial direction, we have:
    $$\lambda'-\lambda=\frac{h}{m_0c}(1-\cos\theta)$$
    \begin{center}
      \begin{minipage}{\linewidth}
        \centering
        \includestandalone[mode=image|tex,width=0.5\linewidth]{Images/compton}
        \captionof{figure}{Compton scattering}
      \end{minipage}
    \end{center}
  \end{prop}
  \subsection{Fluids}
  \begin{definition}
    A \textit{fluid} is a substance that continually flows under an applied external force.
  \end{definition}
  \begin{definition}
    The \textit{viscosity} of a fluid is a measure of its resistance to deformation at a given rate. We say a fluid is \textit{ideal} if we don't consider viscosity.
  \end{definition}
  \begin{prop}[Density]
    The density of a fluid of mass $m$ that occupies a volume $V$ is: $$\rho=\frac{m}{V}$$ The density depends on temperature and pressure\footnote{This variation is typically small for solids and liquids but much greater for gases.}.
  \end{prop}
  \begin{definition}
    A fluid is said to be \textit{incompressible} if its density doesn't varies with the pressure.
  \end{definition}
  \begin{prop}[Pressure]
    Consider a point $x$ and a small sphere centered at $x$. Then, the pressure $p(x)$ at point $x$ is: $$p(x)=\frac{\sum F_N}{S}$$ where $\sum F_N$ is the sum of normal forces and $S$ is the surface which the forces are applied to. The SI unit of pressure is the Pascal: $1\;\text{Pa}=1\;\text{N}/\text{m}^2$.
  \end{prop}
  \begin{prop}[Hydrostatic pressure]
    Consider a static fluid with constant density $\rho$ and let $p_0$ be the pressure on its surface. Then, the pressure $p$ on a depth $h$ is
    $$p=p_0+\rho gh$$
  \end{prop}
  \begin{prop}[Pascal's principle]
    Any pressure applied to the surface of a fluid is transmitted uniformly throughout the fluid in all directions, in such a way that initial variations in pressure are not changed. $$p_1=\frac{F_1}{S_1}=\frac{F_2}{S_2}=p_2$$
  \end{prop}
  \begin{prop}[Archimedes' principle]
    Any object (of mass $m$), totally or partially immersed in a fluid of density $\rho$, is buoyed up by a force equal to the weight of the fluid displaced by the object, that is: $$F_B:=\rho gV_\text{dis}$$ where $F_E$ is called the \textit{buoyancy} and $V_\text{dis}$ is the volume of the liquid displaced\footnote{Note that if $F_B-mg>0$, the object rises to the surface of the liquid; if $F_B-mg<0$, the object sinks, and if $F_B-mg=0$, the object is neutrally buoyant, that is, it remains in place without either rising or sinking.}.
  \end{prop}
  \begin{definition}
    We define the \textit{discharge of a fluid} as: $$Q=Sv$$ where $S$ is the cross-sectional area of the portion of the channel occupied by the flow and $v$ is the average flow velocity. If the velocity is not constant, then: $$Q=\iint_S\vectorfunction{v}\cdot d\vectorfunction{S}$$
  \end{definition}
  \begin{prop}[Continuity equation]
    Consider an incompressible fluid moving throughout a channel. Then, the volume per unit of time is conserved, that is, the discharge is conserved. Mathematically: $$Q_1=S_1v_1=S_2v_2=Q_2$$
  \end{prop}
  \begin{definition}
    \textit{Laminar flow} is a fluid motion that occurs when a fluid flows in parallel layers, with no disruption between those layers. \textit{Turbulent flow} is a fluid motion characterized by chaotic changes in pressure and flow velocity.
  \end{definition}
  \begin{center}
    \begin{minipage}{\linewidth}
      \centering
      \includestandalone[mode=image|tex,width=\linewidth]{Images/laminar_turbulent}
      \captionof{figure}{}
    \end{minipage}
  \end{center}
  \begin{prop}[Bernolli's principle]
    Consider an incompressible and ideal fluid of density $\rho$ with steady laminar flow. Then: $$p+\rho gh+\frac{1}{2}\rho v^2=\text{const.}$$ where $p$ is the pressure at a point on a streamline; $h$, the elevation of the point from a reference frame, and $v$, the fluid flow speed at the chosen point.
  \end{prop}
  \begin{prop}[Lift force]
    If the air has density $\rho$ and an object of cross-sectional area $S$ is moving at a velocity of $v$ relative to the air, then the lift force is: $$F_L=\frac{1}{2}C_L\rho Sv^2$$ where $C_L$ is the \textit{lift coefficient}. From that we deduce that the minimum velocity for lifting is: $$F_L=mg\implies v_\text{min}=\sqrt{\frac{2mg}{C_L\rho S}}$$
  \end{prop}
  \begin{prop}[Viscosity]
    Consider a fluid trapped between two plates of area $S$, one fixed and the other one in parallel motion at constant speed $v$. If we suppose a laminar flow, each layer of fluid moves faster than the one just below it and so this creates a friction force  resisting their relative motion. An external force $F$ is therefore required in order to keep the top plate moving at constant speed. This force is given by: $$F=\eta\frac{vS}{z}$$ where $z$ is the separation between the plates and $\eta$ is the viscosity of the fluid ($[\eta]=$ Pa $\cdot$ s).
  \end{prop}
  \begin{center}
    \begin{minipage}{\linewidth}
      \centering
      \includestandalone[mode=image|tex,width=0.6\linewidth]{Images/viscous_force}
      \captionof{figure}{}
    \end{minipage}
  \end{center}
  \begin{prop}[Velocity of a fluid in a channel]
    Consider a fluid with viscosity $\eta$ in laminar flow so that the layer in contact with the wall of the channel (of radius $r$) is at rest. Let $p_1$ be the pressure at one point of the channel and $p_2$ be the pressure at another point separated a distance $L$ along the $x$-axis from the previous point. Then, the speed of each layer of fluid at a distance $x$ from the center of the channel is:
    $$v(y)=\frac{p_1-p_2}{4\eta L}(r^2-y^2)$$ The average speed and maximal speed of the fluid are:
    \begin{equation}
      v_\text{avg}=\frac{p_1-p_2}{8\eta L}r^2\qquad v_\text{max}=\frac{p_1-p_2}{4\eta L}r^2
      \label{eq1}
    \end{equation}
  \end{prop}
  \begin{prop}[Poiseuille's law]
    In conditions of the equation \eqref{eq1}, we have: $$Q=Sv_\text{avg}=\frac{\pi}{8\eta }\frac{p_1-p_2}{L}r^4\implies\Delta p=\frac{8\eta}{\pi}\frac{L}{r^4}Q$$ If we denote $\displaystyle R_f:=\frac{8\eta}{\pi}\frac{L}{r^4}$ the hydrodynamic resistance, we can write Poiseuille's law as follows: $$\Delta p=R_f Q,$$ which is an analogy of Ohm's law\footnote{In that case, $R_f$ would play the role of electric resistance; $Q$, the role of intensity of the current, and $\Delta p$, the role of electric potential difference.}.
  \end{prop}
  \begin{prop}[Resistance in fluids]
    Consider $n$ channels each of resistance $R_i$. The total resistance will be:
    \begin{itemize}
      \item Connected in series: $$R_T=\sum_{i=1}^nR_i$$
      \item Connected in parallel: $$\frac{1}{R_T}=\sum_{i=1}^n\frac{1}{R_i}$$
    \end{itemize}
  \end{prop}
  \begin{prop}[Dissipated power]
    Consider a fluid that passes throughout a channel of resistance $R_f$. If the discharge of the fluid is $Q$ in a section where the pressure difference is $\Delta p$, the \textit{dissipated power} will be:
    $$P=\Delta pQ=R_fQ^2$$
  \end{prop}
  \begin{prop}[Drag forces]
    An object moving at a velocity $v$ in a fluid of density $\rho$ and viscosity $\eta$ creates drag forces:
    \begin{itemize}
      \item For low speeds and high viscosity, viscous forces predominate:\par
            $$F=k\eta vr$$
            where $k=6\pi$ if the object is an sphere and $r$ is its radius.
      \item For high speeds and low viscosity, inertial forces predominate:
            $$F=\frac{1}{2}C_a\rho Sv^2$$
            where $C_a$ is the aerodynamic coefficient and $S$ the cross-sectional area.
    \end{itemize}
  \end{prop}
  \begin{prop}[Terminal velocity]
    An object falling (by gravity) inside a fluid attains a maximum velocity (terminal velocity) when its weight equals the drag force. We have two cases to consider:
    \begin{itemize}
      \item For viscous forces: $$v_t=\frac{mg}{k\eta r}$$
      \item For inertial forces: $$v_t=\sqrt{\frac{2mg}{C_a\rho S}}$$
    \end{itemize}
  \end{prop}
  \begin{prop}[Reynolds number]
    The Reynolds number helps to predict flow patterns in different fluid flow situations.
    $$\text{Re}=\frac{\rho vD}{\eta}\approx\frac{F_{\text{inertial}}}{F_{\text{viscous}}}$$
    where $v$ is the flow speed and $D$ is the diameter of the object.
    \begin{align*}
      \text{Re}<2000 & \implies\text{laminar flow}   \\
      \text{Re}>3000 & \implies\text{turbulent flow}
    \end{align*}
  \end{prop}


\end{multicols}
\end{document}