\documentclass[../../../main_physics.tex]{subfiles}

\begin{document}
\renewcommand{\col}{\phy}
\begin{multicols}{2}[\section{Classical mechanics}]
  \subsection{Motion in one dimension}
  \subsubsection{Integration of Newton's 2nd law}
  \begin{proposition}[Newton's 2nd law]
    Consider a particle with constant mass $m$ that moves in one dimension. Then, it satisfies: $$\ddot{x}(t)=\frac{1}{m}F(x(t),\dot{x}(t),t)$$
    where we have supposed the force function $F$ is known and $x(t)$ is the position of the particle as a function of time\footnote{Sometimes, in order to simplify the notation, we will write $x$ instead of $x(t)$ (and similarly with the velocity and other magnitudes).}. We also suppose that the initial position and initial velocity, denoted by $x(t_0)=x_0$ and $\dot{x}(t_0)=\dot{x}_0$ respectively, are known.
  \end{proposition}
  \begin{proposition}[Integration of Newton's 2nd law]
    Consider a force $F$ that only depends on time, position or velocity. Then, the position and velocity of the particle subjected to $F$ are:
    \begin{itemize}
      \item Time dependence:
            \begin{gather*}
              \dot{x}(t)=\dot{x}_0+\int_{t_0}^t\frac{F(s)}{m}\dd{s}\\
              x(t)=x_0+\int_{t_0}^t\dot{x}(s)\dd{s}
            \end{gather*}
      \item Position dependence:
            \begin{gather*}
              {\dot{x}(x)}^2={\dot{x}(x_0)}^2+2\int_{x_0}^x\frac{F(s)}{m}\dd{s}\\
              x(t)=g^{-1}(t)
            \end{gather*}
            where $\displaystyle g(x):=\int_{x_0}^x\frac{1}{\dot{x}(s)}\dd{s}$.
      \item Velocity dependence:
            \begin{gather*}
              \dot{x}(t)=h^{-1}(t)\\
              x(t)=x_0+\int_{t_0}^th^{-1}(s)\dd{s}
            \end{gather*}
            where $\displaystyle h(\dot{x}):=\int_{\dot{x}_0}^{\dot{x}}\frac{m}{F(s)}\dd{s}$.
    \end{itemize}
  \end{proposition}
  \subsubsection{Variable mass}
  \begin{proposition}[Mass accretion formula]
    Consider two objects of masses $m(t)$ and $dm$ and velocities $\vf{v}(t)$ and $\vf{u}(t)$ respectively, which in an interval of time $\dd{t}$ the second one collide with the first one and become a unique object. If $\vf{F}^\text{ext}$ is the external force acting to the system, we have:
    \begin{equation}
      \vf{F}^\text{ext}=m\dot{\vf{v}}+(\vf{v}-\vf{u})\dot{m}=\dot{\vf{p}}-\dot{m}\vf{u}
      \label{CM_mass}
    \end{equation}
    where $\vf{p}$ is the momentum of the object that gains mass\footnote{The formula is also valid for the case when the object is losing mass, i.e.\ $\dot{m}<0$.}.
  \end{proposition}
  \begin{definition}[Rocket motion]\label{CM_rocket}
    Consider a rocket moving at a velocity $\vf{v}$ that expels gas (to propel itself) at a velocity $\vf{c}$ with respect to the rocket. Suppose the mass of the rocket is $m(t)$ and let $m_0:=m(t_0)$. If $\vf{u}=\vf{v}+\vf{c}$ is the velocity of the gas with respect to an external frame of reference and $\vf{F}^\text{ext}$ is the net external force acting on the rocket, by \mcref{CM_mass} we have:
    $$m\dot{\vf{v}}=\vf{F}^\text{ext}+\dot{m}\vf{c}$$
  \end{definition}
  \begin{proposition}[Rocket without gravity]
    In this case, and using the hypothesis of \mcref{CM_rocket}, we have $\vf{F}^\text{ext}=0$ and if we suppose $\vf{v}=v\vf{j}$ and $\vf{c}=-c\vf{j}$, we have:
    \begin{equation}
      m\dv{v}{t}=-c\dv{m}{t}\implies v=c\log\frac{m_0}{m}
      \label{CM_rockg0}
    \end{equation}
    Consider now the discrete case, i.e.\ when the function $\dot{m}$ is not differentiable. For that we can consider instantaneous ejections of $\Delta m=(m_0-m_f)/n$ amount of mass where $m_f$ is the mass of the rocket after $n$ ejections of mass. For this case, we have: $$v=c\sum_{k=1}^n\frac{(m_0-m_f)/n}{m_f+k(m_0-m_f)/n}=c\sum_{k=1}^n\frac{\Delta m}{m_f+k\Delta m}\footnote{Obviously if we tend $n$ to infinity we get \mcref{CM_rocket}.}$$
  \end{proposition}
  \begin{proposition}[Rocket with gravity]
    In this case, and using the hypothesis of \mcref{CM_rocket}, we have $\vf{F}^\text{ext}=-mg\vf{j}$. Suppose $\vf{v}=v\vf{j}$, $\vf{c}=-c\vf{j}$ and, for simplicity, consider only the case when $\dot{m}=-\beta$, $\beta>0$. Therefore, we obtain:
    \begin{equation}
      m\frac{\dd{v}}{\dd{t}}=-mg+c\beta\implies v=c\ln\frac{m_0}{m}-\frac{g}{\beta}(m_0-m)
      \label{CM_rockg1}
    \end{equation}
    Observe that if $m_0g>\beta c$, then $\dv{v}{t}$ would be negative, which is not possible. Therefore, in this case the formula is not correct if we are considering the rocket launch. In this case the formula becomes:
    \begin{equation}
      v=c\ln\frac{\beta c}{mg}-c+\frac{g}{\beta}m
      \label{CM_rockg2}
    \end{equation}
    Because of $\dot{m}=-\beta\implies m(t)=m_0-\beta t$, we can express \mcref{CM_rockg1,CM_rockg2} respectively as:
    \begin{gather*}
      v(t)=c\ln\frac{m_0}{m_0-\beta t}-gt\\
      v(t)=c\ln\frac{\beta c}{m_0g-g\beta t}-c+\frac{g}{\beta}m_0-gt
    \end{gather*}
  \end{proposition}
  \subsection{Oscillations}
  \subsubsection{Simple harmonic oscillator}
  \begin{proposition}
    Consider the differential equation $$\ddot{x}+{\omega_0}^2 x=0$$ with initial values $x(0)=x_0$ and $\dot{x}(0)=\dot{x}_0$. Then, the general solution is:
    \begin{equation}
      x(t)=x_0\cos(\omega_0t)+\frac{\dot{x}_0}{\omega_0}\sin(\omega_0t)=A\cos(\omega_0t+\phi)
      \label{CM_mhs}
    \end{equation} where $\displaystyle A=\sqrt{{x_0}^2+{\left(\frac{\dot{x}_0}{\omega_0}\right)}^2}$ and $\displaystyle \phi=-\arctan\left(\frac{\dot{x}_0}{\omega_0x_0}\right)$. Such constants $\omega_0$ ($[\omega_0]=\text{rad}\cdot \text{s}^{-1}$), $A$ ($[A]=\text{m}$) and $\phi$ ($[\phi]=\text{rad}$) are called \emph{angular frequency}, \emph{amplitude} and \emph{initial phase}, respectively. Observe that the function in \mcref{CM_mhs} is periodic with period $T=\frac{2\pi}{\omega_0}$ ($[T]=\text{s}$) and frequency $\nu=T^{-1}=\frac{\omega_0}{2\pi}$ ($[\nu]=\text{s}^{-1}=:\text{Hz}$).
  \end{proposition}
  \begin{proposition}
    Consider a mass $m$ subjected to a spring of elastic constant $k$ (see \mcref{CM_fig1}). If we displace the mass a distance $A$ from its equilibrium state, the mass will start to oscillate and its position $x(t)$ as a function of time will be:
    $$-kx=m\ddot{x}\implies x(t)=A\cos(\omega_0t+\phi)$$ where $\omega_0=\sqrt{k/m}$. The energies of this systems are:
    \begin{gather*}
      K=\frac{1}{2}kA^2{\left[\sin(\omega_0 t+\phi)\right]}^2\quad U=\frac{1}{2}kA^2{\left[\cos(\omega_0 t+\phi)\right]}^2\\
      E=\frac{1}{2}kA^2
    \end{gather*}
    \begin{center}
      \begin{minipage}{\linewidth}
        \centering
        \includestandalone[mode=image|tex,width=0.4\linewidth]{Images/spring}
        \captionof{figure}{Mass subjected to a spring}
        \label{CM_fig1}
      \end{minipage}
    \end{center}
  \end{proposition}
  \begin{definition}
    Let $U(x)$ be a potential function of class $\mathcal{C}^2(\RR)$. We say $x_0$ is a \emph{point of stable equilibrium} if $U$ attains a minimum in $x_0$. Analogously, we say $x_0$ is a \emph{point of unstable equilibrium} if $U$ attains a maximum in $x_0$.
  \end{definition}
  \begin{proposition}
    Let $U(x)$ be a potential function of class $\mathcal{C}^2(\RR)$ associated with a particle of mass $m$, and $x_0$ be a point of stable equilibrium. Suppose the particles is at the position $x_0$. Then, if we disturb slightly the particle, it will start to oscillate at a frequency $\omega_0=\sqrt{\frac{U''(x_0)}{m}}$.
  \end{proposition}
  \begin{proposition}
    Consider a mass hanging from a spring and let $y(t)$ be the position of the mass measured from initial string's length (without the mass) to the position of the mass at time $t$ (see \mcref{CM_fig2}). If we disturb the system with an external force so that the mass starts to oscillate, we have: $$mg-ky=m\ddot{y}\implies y(t)=\frac{mg}{k}+A\cos(\omega_0t+\phi)$$ where $\omega_0=\sqrt{\frac{k}{m}}$.
    \begin{center}
      \begin{minipage}{\linewidth}
        \centering
        \includestandalone[mode=image|tex,width=\linewidth]{Physics/2nd/Classical_mechanics/Images/springs}
        \captionof{figure}{Mass hanging from a spring}
        \label{CM_fig2}
      \end{minipage}
    \end{center}
  \end{proposition}
  \begin{proposition}
    Consider a simple pendulum with a mass $m$ attached to one of the endpoints of the string (massless and of length $\ell$). If we disturb the system a small angle with an external force so that the mass starts to oscillate, then assuming that $\theta(t)$ denotes the angle of the string with respect to the stable equilibrium we have:
    $$\ddot{\theta}+\frac{g}{\ell}\theta=0\implies\theta(t)=A\cos(\omega_0t+\phi)$$ where $\omega_0=\sqrt{\frac{g}{\ell}}$.
    \begin{center}
      \begin{minipage}{\linewidth}
        \centering
        \includestandalone[mode=image|tex,width=0.5\linewidth]{Physics/2nd/Classical_mechanics/Images/simple_pendulum}
        \captionof{figure}{Simple pendulum}
      \end{minipage}
    \end{center}
  \end{proposition}
  \begin{proposition}
    Consider a physical pendulum of mass $m$ and let $I_e$ be the moment of inertia of the body with respect to the axis $e$ (see \mcref{CM_fig3}). If we disturb the system a small angle with an external force so that the mass starts to oscillate, then assuming that $\theta(t)$ denotes the angle between the axis $e$ and the line passing through the fixed point and the CM, we have:
    $$\ddot{\theta}+\frac{mgD}{I_e}\theta=0\implies\theta(t)=A\cos(\omega_0t+\phi)$$ where $\omega_0=\sqrt{\frac{mgD}{I_e}}$.
    \begin{center}
      \begin{minipage}{\linewidth}
        \centering
        \includestandalone[mode=image|tex,width=0.4\linewidth]{Physics/2nd/Classical_mechanics/Images/physical_pendulum}
        \captionof{figure}{Physical pendulum}
        \label{CM_fig3}
      \end{minipage}
    \end{center}
  \end{proposition}
  \begin{proposition}
    Consider the LC circuit as shown in \mcref{CM_LC}. We have that:
    $$L\ddot{q}+\frac{1}{C}q=0\implies q(t)=A\cos(\omega_0t+\phi)$$ where $\omega_0=\frac{1}{\sqrt{LC}}$. Differentiating $q(t)$, we get: $$I(t)=\dv{q}{t}=-A\omega_0\sin(\omega_0t+\phi)$$
    \begin{center}
      \begin{minipage}{\linewidth}
        \centering
        \includestandalone[mode=image|tex,width=0.5\linewidth]{Physics/2nd/Classical_mechanics/Images/LC}
        \captionof{figure}{LC circuit}
        \label{CM_LC}
      \end{minipage}
    \end{center}
  \end{proposition}
  \subsubsection{Damped harmonic oscillator}
  \begin{proposition}
    Consider the following differential equation $$\ddot{x}+2\beta\dot{x}+{\omega_0}^2 x=0$$ with initials values of $x(0)=x_0$ and $\dot{x}(0)=\dot{x}_0$. Then we have three cases for the general solution:
    \begin{itemize}
      \item If $\beta<\omega_0$ (\emph{underdamped}):
            \begin{equation}
              x(t)=e^{-\beta t}\left(c_1\cos\Tilde{\omega}t+c_2\sin\Tilde{\omega}t\right)
              \label{CM_b<w}
            \end{equation}
      \item If $\beta=\omega_0$ (\emph{critically damped}):
            \begin{equation}
              x(t)=e^{-\beta t}\left(c_1+c_2t\right)
              \label{CM_b=w}
            \end{equation}
      \item If $\beta>\omega_0$ (\emph{overdamped}):
            \begin{equation}
              x(t)=c_1e^{-(\beta+\Tilde{\omega})t}+c_2e^{-(\beta-\Tilde{\omega})t}
              \label{CM_b>w}
            \end{equation}
    \end{itemize}
    Here $c_1$, $c_2$ are constants depending on the initial values and we have defined $\displaystyle \Tilde{\omega}=\sqrt{\left|{\omega_0}^2-\beta^2\right|}$.
  \end{proposition}
  \begin{proposition}[Energy of damped harmonic oscillator]
    Relating the energy of a damped harmonic oscillator, we have:
    $$E=\frac{\mu}{2}\left(\dot{x}^2+{\omega_0}^2x^2\right)$$ where $\mu$ is a constant.
  \end{proposition}
  \begin{definition}[Quality factor]
    The \emph{quality factor} is defined as follows: $$Q:=\frac{\omega_0}{2\beta}$$
  \end{definition}
  \begin{proposition}
    We can write $\tilde{\omega}$ in terms of $Q$ as follows: $$\Tilde{\omega}=\omega_0\sqrt{\left|1-\frac{1}{4Q^2}\right|}$$
  \end{proposition}
  \begin{proposition}[Underdamped harmonic oscillator: $\beta<\omega_0$]
    Coefficients $c_1$, $c_2$ of the general solution of \mcref{CM_b<w} are: $$c_1=x_0,\quad c_2=\frac{\dot{x}_0+\beta x_0}{\Tilde{\omega}}$$ On the other hand, \mcref{CM_b<w} can be simplified to $$x(t)=Ae^{-\beta t}\cos(\Tilde{\omega}t+\phi)$$ where $$A=\sqrt{{x_0}^2+{\left(\frac{\dot{x}_0+\beta x_0}{\Tilde{\omega}}\right)}^2}\text{ and }\phi=-\arctan\left(\frac{\dot{x}_0+\beta x_0}{\Tilde{\omega}x_0}\right)$$
  \end{proposition}
  \begin{proposition}
    Relating the energy of an underdamped harmonic oscillator, we have:
    $$E(t)\simeq\frac{\mu{\omega_0}^2A^2}{2}e^{-2\beta t}=E(0)e^{-2\beta t}$$ where $\mu$ is a constant. The rate at which the energy is dissipated is: $$\left|\frac{\dd{E}}{\dd{t}}(t)\right|=2\beta E(t)\implies\frac{E}{\left|dE/\dd{t}\right|}=\frac{1}{2\beta}$$
    If $\beta\ll\omega_0$, then: $$Q\approx 2\pi\frac{E}{\Delta E}$$ where $\Delta E$ is the energy dissipated in a pseudo-period $\Tilde{T}=2\pi/\Tilde{\omega}\approx2\pi/\omega_0$.
  \end{proposition}
  \begin{proposition}[Critically damped harmonic oscillator: $\beta=\omega_0$]
    Coefficients $c_1$, $c_2$ of the general solution of \mcref{CM_b=w} are: $$c_1=x_0,\quad c_2=x_0\beta+\dot{x}_0$$ This harmonic oscillator is the one that returns to the equilibrium state more quickly.
  \end{proposition}
  \begin{proposition}[Overdamped harmonic oscillator: $\beta<\omega_0$]
    Coefficients $c_1$, $c_2$ of the general solution of \mcref{CM_b>w} are: $$c_1=\frac{x_0(\Tilde{\omega}-\beta)-\dot{x}_0}{2\Tilde{\omega}}\quad c_2=\frac{x_0(\Tilde{\omega}+\beta)+\dot{x}_0}{2\Tilde{\omega}}$$
  \end{proposition}
  \begin{proposition}
    Consider the RLC circuit as shown in \mcref{CM_RLC}. We have that:
    $$L\ddot{q}+R\dot{q}+\frac{1}{C}q=0$$ Let $\beta=\frac{R}{2L}$ and $\omega_0=\frac{q}{LC}$. If $\frac{\beta}{\omega_0}=\frac{R}{2}\sqrt{\frac{C}{L}}<1$, the harmonic system is underdamped and: $$q(t)=Ae^{-\beta t}\cos(\Tilde{\omega}t+\phi)$$ where $\tilde{\omega}=\sqrt{{\omega_0}^2-\beta^2}$. Differentiating $q(t)$, we get: $$I(t)=\dv{q}{t}(t)=-A\left[\beta\cos(\Tilde{\omega}t+\phi)+\Tilde{\omega}\sin(\Tilde{\omega}t+\phi)\right]$$ The quality factor is: $$Q=\frac{\omega_0}{2\beta}=\frac{1}{R}\frac{L}{C}$$
    \begin{center}
      \begin{minipage}{\linewidth}
        \centering
        \includestandalone[mode=image|tex,width=0.5\linewidth]{Physics/2nd/Classical_mechanics/Images/RLC}
        \captionof{figure}{RLC circuit}
        \label{CM_RLC}
      \end{minipage}
    \end{center}
  \end{proposition}
  \subsubsection{Driven harmonic oscillators}
  \begin{proposition}
    Consider the following differential equation $$\ddot{x}+2\beta\dot{x}+{\omega_0}^2 x=f(t)$$ with initials values of $x(0)=x_0$ and $\dot{x}(0)=\dot{x}_0$. We will only consider th case when $f(t)$ is of the form $f(t)=f_0\cos(\omega t+\psi)$. A particular solution in that case is:
    $$x_\text{p}(t)=A\cos(\omega t+\psi-\delta)$$
    where $$A=\frac{f_0}{\sqrt{{({\omega_0}^2-\omega^2)}^2+4\beta^2\omega^2}}\text{ and }\delta=\arctan\left({\frac{2\beta\omega}{{\omega_0}^2-\omega^2}}\right)$$ Therefore for the general solution we have three cases to consider:
    \begin{itemize}
      \item If $\beta<\omega_0$:
            \begin{equation}
              x(t)=e^{-\beta t}(c_1\cos\Tilde{\omega}t+c_2\sin\Tilde{\omega}t)+A\cos(\omega t+\psi-\delta)
              \label{CM_d-b<w}
            \end{equation}
      \item If $\beta=\omega_0$:
            \begin{equation}
              x(t)=e^{-\beta t}\left(c_1+c_2t\right)+A\cos(\omega t+\psi-\delta)
              \label{CM_d-b=w}
            \end{equation}
      \item If $\beta>\omega_0$:
            \begin{equation}
              x(t)=c_1e^{-(\beta+\Tilde{\omega})t}+c_2e^{-(\beta-\Tilde{\omega})t}+A\cos(\omega t+\psi-\delta)
              \label{CM_d-b>w}
            \end{equation}
    \end{itemize}
    Here $c_1$, $c_2$ are constants depending on the initial values.
  \end{proposition}
  \begin{proposition}[Underdamped driven oscillator]
    Coefficients $c_1$, $c_2$ of the general solution of \mcref{CM_d-b<w} are:
    \begin{gather*}
      c_1=x_0-A\cos\left(\psi-\delta\right)\\
      c_2=\frac{\dot{x}_0-\omega A\sin\left(\psi-\delta\right)+\beta\left[x_0-A\cos\left(\psi-\delta\right)\right]}{\tilde{\omega}}
    \end{gather*}
  \end{proposition}
  \begin{proposition}[Critically damped driven oscillator]
    Coefficients $c_1$, $c_2$ of the general solution of \mcref{CM_d-b=w} are:
    \begin{gather*}
      c_1=x_0-A\cos\left(\psi-\delta\right)\\
      c_2=\dot{x}_0-\omega A\sin\left(\psi-\delta\right)+\beta\left[x_0-A\cos\left(\psi-\delta\right)\right]
    \end{gather*}
  \end{proposition}
  \begin{proposition}[Overdamped driven oscillator]
    Coefficients $c_1$, $c_2$ of the general solution of \mcref{CM_d-b>w} are:
    \begin{equation*}
      \begin{split}
        \begin{multlined}[t]
          c_1=A\frac{(\beta-\Tilde{\omega})\cos\left(\psi-\delta\right)-\omega\sin\left(\psi-\delta\right)}{2\Tilde{\omega}}+\\+\frac{-(\beta-\Tilde{\omega})x_0-\dot{x}_0}{2\Tilde{\omega}}
        \end{multlined}\\
        \begin{multlined}[t]
          c_2=A\frac{-(\beta+\Tilde{\omega})\cos\left(\psi-\delta\right)+\omega\sin\left(\psi-\delta\right)}{2\Tilde{\omega}}+\\+\frac{(\beta+\Tilde{\omega})x_0+\dot{x}_0}{2\Tilde{\omega}}
        \end{multlined}
      \end{split}
    \end{equation*}
  \end{proposition}
  \begin{definition}
    Given a driven oscillator, we say it is in the \emph{steady-state part} if $t\gg 1/\beta$. In that case $x(t)$ become: $$x(t)\simeq A\cos(\omega t+\psi-\delta)$$ While the dependency on $c_1$, $c_2$ is non-negligible, we say the driven oscillator is in the \emph{transient part}.
  \end{definition}
  \begin{definition}
    For a driven damped oscillator we define the \emph{resonant frequency} as: $$\omega_\text{r}:=\sqrt{{\omega_0}^2-2\beta^2}$$ We say that the oscillator is in \emph{resonance in amplitude} if $\omega=\omega_\text{r}$.
  \end{definition}
  \begin{proposition}[Resonance in amplitude]
    The amplitude $A$ of a damped driven oscillator can be expressed as:
    $$\frac{A}{f_0/{\omega_0}^2}=\frac{1}{\sqrt{{\left[1-{\left(\omega/\omega_0\right)}^2\right]}^2+{\left(\omega/\omega_0\right)}^2Q^{-2}}}$$
    Thus, if $\beta\ll \omega_0$, then $\omega_\text{r}\approx\omega_0$ and: $$\frac{A}{f_0/{\omega_0}^2}\bigg|_{\omega=\omega_\text{r}}=\frac{Q\omega_0}{\sqrt{{\omega_0}^2-\beta^2}}\approx Q$$
    \begin{center}
      \begin{minipage}{\linewidth}
        \centering
        \includestandalone[mode=image|tex,width=0.75\linewidth]{Physics/2nd/Classical_mechanics/Images/ress_amplitude}
        \captionof{figure}{Resonance in amplitude}
      \end{minipage}
    \end{center}
  \end{proposition}
  \begin{proposition}
    The energy in the steady-state part is:
    $$E=\frac{\mu A^2}{2}\left[\omega^2{\sin(\omega t+\psi-\delta)}^2+{\omega_0}^2\cos^2(\omega t+\psi-\delta)\right]$$ where $\mu$ is a constant. If $\omega\approx\omega_0$ and $\beta\ll\omega_0$, then: $$E\approx\frac{\mu {f_0}^2}{8}\frac{1}{(\omega-\omega_0)^2+\beta^2}$$
  \end{proposition}
  \begin{definition}
    We define the \emph{cutoff frequencies} as the frequencies at which energy (or power) is reduced to half of the maximum value. In the previous case, we have: $$\omega_1=\omega_0-\beta,\quad\omega_2=\omega_0+\beta$$ The value  $\Delta\omega=\omega_2-\omega_1=2\beta$ is called the \emph{bandwidth}. Therefore, we can redefined the quality factor in the following way: $$Q=\frac{\omega_0}{2\beta}=\frac{\omega_0}{\Delta\omega}$$
  \end{definition}
  \begin{center}
    \begin{minipage}{\linewidth}
      \centering
      \includestandalone[mode=image|tex,width=0.75\linewidth]{Physics/2nd/Classical_mechanics/Images/ress_energy}
      \captionof{figure}{Resonance in energy}
    \end{minipage}
  \end{center}
  \begin{proposition}
    $$x(t)=\frac{A_0}{{\omega_0}^2}+\sum_{k=1}^\infty\frac{A_k}{\sqrt{{({\omega_0}^2-\omega_k}^2}}$$
  \end{proposition}
  \begin{proposition}
    Consider the circuit shown in \mcref{CM_RLC-genS}. Suppose $V_\text{in}=\epsilon_0\cos\omega t$, the capacitor is charging and the system is underdamped. We want to determine the voltage $V_\text{out}$.
    \begin{center}
      \begin{minipage}{\linewidth}
        \centering
        \includestandalone[mode=image|tex,width=0.5\linewidth]{Physics/2nd/Classical_mechanics/Images/RLC-genS}
        \captionof{figure}{RLC circuit in series with a voltage source}
        \label{CM_RLC-genS}
      \end{minipage}
    \end{center}
    The ODE of the system for the charge $q(t)$ is: $$\ddot{q}+\frac{R}{L}\dot{q}+\frac{q}{LC}=\frac{V_\text{in}}{L}$$ Thus, in steady-state part we have: $$q(t)=\frac{\epsilon_0/\omega}{\sqrt{{\left(L\omega-\frac{1}{C\omega}\right)}^2+R^2}}\cos(\omega t-\delta)$$ where $\delta=-\arctan\left(\frac{R}{L\omega-\frac{1}{C\omega}}\right)$.
    And finally:
    \begin{equation}\label{CM_Vout}
      V_\text{out}=RI=-\frac{R}{\sqrt{{\left(L\omega-\frac{1}{C\omega}\right)}^2+R^2}}\epsilon_0\sin(\omega t-\delta)
    \end{equation}
    The cutoff frequencies are: $$\sqrt{\frac{1}{LC}-{\left(\frac{R}{2L}\right)}^2}\pm\frac{R}{2L}$$
    Therefore, the bandwidth is $\Delta\omega=\frac{R}{L}$ and the quality factor $Q=\frac{1}{R}\sqrt{\frac{L}{C}}$.
  \end{proposition}
  \begin{definition}
    Using the notation of the previous proposition, we define the \emph{inductive reactance}, \emph{capacitive reactance} and the \emph{reactance} as $X_L:=L\omega$, $X_C:=\frac{1}{C\omega}$, $X=X_L-X_C$, respectively. We define the \emph{impedance} as $Z=\sqrt{X^2+R^2}$. Hence, \mcref{CM_Vout} can be written as: $$V_\text{out}=-\frac{R}{Z}\epsilon_0\sin(\omega t-\delta),\quad\delta=-\arctan\left(\frac{R}{X}\right)$$
  \end{definition}
  % \begin{proposition}
  %   Consider the circuit shown in \mcref{CM_RLC-genP}. Suppose $V_\text{in}=\epsilon_0\cos\omega t$ and that the capacitor is charging. We want to determine the voltage $V_\text{out}$.
  %   \begin{center}
  %     \begin{minipage}{\linewidth}
  %       \centering
  %       \includestandalone[mode=image|tex,width=0.75\linewidth]{Physics/2nd/Classical_mechanics/Images/RLC-genP}
  %       \captionof{figure}{RLC circuit in parallel with a voltage source}
  %       \label{CM_RLC-genP}
  %     \end{minipage}
  %   \end{center}
  %   The ODE of the system for the charge $q(t)$ is: $$\ddot{q}+\frac{R}{L}\dot{q}+\frac{q}{LC}=\frac{V_\text{in}}{L}$$ Thus, in steady-state part we have: $$q(t)=\frac{\epsilon_0/\omega}{\sqrt{{\left(L\omega-\frac{1}{C\omega}\right)}^2+\frac{1}{R^2}}}\cos(\omega t-\delta)$$ where $\delta=-\arctan\left(\frac{1/R}{L\omega-\frac{1}{C\omega}}\right)$.
  %   And finally:
  %   \begin{equation}\label{CM_Vout}
  %     V_\text{out}=RI=-\frac{R}{\sqrt{{\left(L\omega-\frac{1}{C\omega}\right)}^2+R^2}}\epsilon_0\sin(\omega t-\delta)
  %   \end{equation}
  %   The cutoff frequencies are: $$\sqrt{\frac{1}{LC}+{\left(\frac{1}{2RC}\right)}^2}\pm\frac{1}{2RC}$$
  %   Therefore, the bandwidth is $\Delta\omega=\frac{1}{RC}$ and the quality factor $Q=\frac{1}{R}\sqrt{\frac{C}{L}}$.
  % \end{proposition}
  \subsubsection{Impulsive forces}
  \begin{proposition}[Impulsive forces]
    Consider a driven oscillator whose equation is:
    \begin{equation}\label{CM_impulse}
      \ddot{x}+2\beta\dot{x}+{\omega_0}^2x=f
    \end{equation}
    with $\beta<\omega_0$ and $$f(t)=
      \begin{cases}
        0   & \text{if }t<0                  \\
        f_0 & \text{if }0\leq t\leq \Delta t \\
        0   & \text{if }t>\Delta t
      \end{cases}$$
    That is, $f$ is a piecewise function\footnote{We shall suppose that the system was at equilibrium for $t<0$.}. Moreover, assuming that $x(0)=\dot{x}(0)=0$, the general solution to this ODE when $0<t<\Delta t$ is:
    $$x(t)=\frac{f_0}{{\omega_0}^2}+\exp{-\beta t}\left(c_1\cos(\tilde{\omega}t)+c_2\sin(\tilde{\omega}t)\right)$$
    where $\tilde{\omega}=\sqrt{{\omega_0}^2-\beta^2}$ and $c_1=-\frac{f_0}{{\omega_0}^2}$, $c_2=-\frac{\beta f_0}{\omega_1{\omega_0}^2}$.
    Therefore: $$x(\Delta t)=\text{O}({\Delta t}^2)\quad\text{and}\quad\dot{x}(\Delta t)=f_0\Delta t+\text{O}({\Delta t}^2)$$
    If $t>\Delta t$, then the general solution to the ODE is:
    $$x(t)=\exp{-\beta t}\left(k_1\cos(\tilde{\omega}t)+k_2\sin(\tilde{\omega}t)\right)$$ where $k_1=\text{O}({\Delta t}^2)$, $k_2=\frac{f_0\Delta t}{\tilde{\omega}}+\text{O}({\Delta t}^2)$
    Therefore, $\forall t>\Delta t$: $$x(t)=f_0\Delta t\frac{\exp{-\beta t}}{\tilde{\omega}}\sin(\tilde{\omega}t)+\text{O}({\Delta t}^2)$$
  \end{proposition}
  \begin{definition}
    Let $\Delta t\geq 0$ and $$\delta_{\Delta t}(t)=
      \begin{cases}
        \frac{1}{\Delta t} & \text{if }0\leq t\leq \Delta t \\
        0                  & \text{if }t<0,t>\Delta t
      \end{cases}
    $$
    We define the \emph{Dirac delta distribution} as: $$\delta(t)=\lim_{\Delta t\to 0}\delta_{\Delta t}(t)$$
  \end{definition}
  \begin{proposition}
    Let $f:\RR\rightarrow\RR$ be a continuous function. Then: $$\int_{-\infty}^{+\infty}f(t)\delta(t)\dd{t}=f(0)$$
  \end{proposition}
  \begin{definition}
    We define the \emph{Heaviside step function} as: $$H(t)=\indi{t\geq 0}=
      \begin{cases}
        1 & \text{if }t\geq 0 \\
        0 & \text{if }t<0
      \end{cases}
    $$
  \end{definition}
  \begin{proposition}
    We have the following properties:
    \begin{itemize}
      \item $H(t)+(-t)=1$ $\forall t\in\RR\setminus\{0\}$
      \item $H'(t)=\delta(t)$
    \end{itemize}
  \end{proposition}
  \begin{proposition}[Green's function]
    Consider \mcref{CM_impulse} with $f(t)=\delta_{\Delta t}(t)$ and let $\displaystyle G(t):=\lim_{\Delta t\to 0}x(t)$. Then: $$G(t)=\frac{H(t)}{\tilde{\omega}}\exp{-\beta t}\sin(\tilde{\omega}t)$$
  \end{proposition}
  \begin{theorem}
    Consider a driven oscillator whose equation is:
    \begin{equation*}
      \ddot{x}+2\beta\dot{x}+{\omega_0}^2x=f
    \end{equation*}
    Then, the general solution is: $$x(t)=\int_{-\infty}^{+\infty}f(s)G(t-s)\dd{s}$$
    If $f(t)=0$ $\forall t<0$, then:
    $$x(t)=\int_0^t\frac{f(s)}{\tilde{\omega}}\exp{-\beta (t-s)}\sin(\tilde{\omega}(t-s))\dd{s}$$
    Furthermore, if $\supp (f)=[0,T]$, then: $$|x(t)|\leq K\exp{-\beta t}\quad\forall t> T$$
  \end{theorem}
  \subsubsection{Non linear oscillations}
  \begin{definition}
    Consider a pendulum whose rod (of length $L$) is in a non-small angle $\theta_0$ at initial time. The ODE that satisfies $\theta(t)$ is:
    $$\ddot{\theta}+\frac{g}{L}\sin\theta=0$$
    Then, the period of the pendulum does depend on $\theta_0$. Indeed:
    \begin{multline*}
      T=2\pi\sqrt{\frac{L}{g}}\sum_{n=0}^\infty{\left(\frac{(2n)!}{2^{2n}{(n!)}^2}\right)}^2{\sin(\theta_0/2)}^{2n}=\\=2\pi\sqrt{\frac{L}{g}}\left(1+\frac{{\theta_0}^2}{16}+\frac{11}{3072}{\theta_0}^4+\cdots\right)
    \end{multline*}
  \end{definition}
  \subsection{Central forces}
  \subsubsection{Definition and properties}
  \begin{definition}[Central force]
    A \emph{central force} is a force of the form
    \begin{equation}\label{CM_central-force}
      \vf{F}(\vf{r})=f(r)\vf{e}_r
    \end{equation} where $r=\|\vf{r}\|$ and $\vf{e}_r=\vf{r}/r$ is the unit radial vector. The origin $\vf{r}=0$ is called \emph{center of forces}.
  \end{definition}
  \begin{definition}
    A central force $\vf{F}(\vf{r})=f(r)\vf{e}_r$ is \emph{attractive} if $f(r)<0$ and is \emph{repulsive} if $f(r)>0$.
  \end{definition}
  \begin{proposition}
    All central forces $\vf{F}(\vf{r})$ are conservative and $$f(r)=-U'(r)$$ where $U(r)$ is the potential energy of the central force.
  \end{proposition}
  \subsubsection{Conservation of angular momentum and areal velocity}
  \begin{proposition}
    The angular momentum with respect to the center of forces is conserved, that is, $\dot{\vf{L}}=0$. Hence the movement is in a plane (perpendicular to $\vf{L}$).
  \end{proposition}
  \begin{proposition}[Kepler's 2nd law]
    The areal velocity $\dv{A}{t}$ is constant. Indeed: $$\dv{A}{t}=\frac{L}{2m}=\const$$ where $L=\|\vf{L}\|$.
  \end{proposition}
  \subsubsection{Trajectory equation}
  \begin{proposition}
    Remembering the relations $$\vf{e}_r=\cos\theta\vf{i}+\sin\theta\vf{j}\quad\vf{e}_\theta=-\sin\theta\vf{i}+\cos\theta\vf{j}$$ we obtain:
    \begin{equation}
      \vf{r}=r\vf{e}_r\quad\dot{\vf{r}}=\dot{r}\vf{e}_r+r\dot{\theta}\vf{e}_\theta\quad\ddot{\vf{r}}=(\ddot{r}-r\dot{\theta}^2)\vf{e}_r+(2\dot{r}\dot{\theta}+r\ddot{\theta})\vf{e}_\theta
      \label{CM_unit}
    \end{equation}
  \end{proposition}
  \begin{proposition}
    From \mcref{CM_unit} if a particle of mass $m$ is moving under the action of a central force $\vf{F}(\vf{r})=f(r)\vf{e}_r$, Newton's second law can be written as: $$\ddot{r}-r\dot{\theta}^2=\frac{f(r)}{m},\quad 2\dot{r}\dot{\theta}+r\ddot{\theta}=0$$ And we can obtain the following differential equations: $$\dot{\theta}=\frac{L}{m r^2}=\frac{\ell}{r^2},\quad\ddot{r}-\frac{\ell^2}{r^3}=\frac{f(r)}{m}$$ where we have defined the magnitude $\ell:=L/m$. Finally, we get the \emph{trajectory equation}: $$\dv[2]{}{\theta}\left(\frac{1}{r}\right)+\frac{1}{r}=-\frac{1}{m\ell^2}r^2f(r)\footnote{Note that if we know $r(\theta)$, we can deduce $r(t)$ and $\theta(t)$. Indeed, since $\dot{\theta}=\frac{\ell}{r(\theta)^2}$ we have that $g(\theta):=\frac{1}{\ell}\int r(\theta)^2\dd{\theta}=t$ which implies $\theta(t)=g^{-1}(t)$ and $r(t)=r(g^{-1}(t))$.}$$
  \end{proposition}
  \subsubsection{Conservation of energy and orbits}
  \begin{proposition}
    The kinetic energy of particle of mass $m$ undergoing a central force of the form of \mcref{CM_central-force} is:
    $$K=\frac{1}{2}m{\|\dot{\vf{r}}\|}^2=\frac{1}{2}m\dot{r}^2+\frac{m\ell^2}{2r^2}$$
    The total energy is thus: $$E=K+U(r)=\frac{1}{2}m\dot{r}^2+U(r)+\frac{m\ell^2}{2r^2}$$
  \end{proposition}
  \begin{definition}
    Consider a particle of mass $m$ undergoing into a potential $U(r)$. We define the \emph{effective potential} as:
    $$U_\text{eff}=U(r)+\frac{m\ell^2}{2r^2}$$
    The term $m\ell^2/(2r^2)$ gives the \emph{centrifugal force}: $$f_\text{c}=-\dv{}{r}\left(\frac{m\ell^2}{2r^2}\right)\implies\vf{f}_\text{c}=mr\dot{\theta}^2\vf{e}_r$$ Therefore:
    \begin{equation}
      E=\frac{1}{2}m\dot{r}^2+U_\text{eff}
      \label{CM_energy}
    \end{equation}
  \end{definition}
  \begin{definition}
    An orbit is \emph{bounded} if $r(t)\in(r_\text{min},r_\text{max})$ for all $t\in\RR$. In that case we'll have $E<0$. Moreover, note that at $r=r_\text{min}$ or $r=r_\text{max}$ (\emph{turning points}) we have $\dot{r}=0$.

    An orbit is \emph{unbounded} if $\displaystyle\lim_{t\to\infty} r(t)=\infty$. In that case we'll have $E>0$.
  \end{definition}
  \begin{proposition}
    The angle $\dd{\theta}$ that results from a displacement $\dd{r}$ of $r$ is:
    $$\dd\theta=\pm\frac{\ell/r^2}{\sqrt{(2/m)(E-U_\text{eff})}}\dd{r}$$
    where the sign depends on the orientation of the orbit.
    Therefore, the angle $\Delta \theta$ that results from one complete transit of $r$ from $r_\text{min}$ to $r_\text{max}$ and back to $r_\text{min}$ is: $$\Delta\theta=2\int_{r_\text{min}}^{r_\text{max}}\frac{\ell/r^2}{\sqrt{(2/m)(E-U_\text{eff})}}\dd{r}.\footnote{Here we have taken the positive orientation, that is, $\vf{L}$ pointing to the positive $z$-axis.}$$ The orbits are \emph{closed} if $$\Delta\theta=2\pi\frac{p}{q},\quad p,q\in\NN$$
  \end{proposition}
  \begin{theorem}[Bertrand's theorem]
    Consider a particle undergoing a central force. Then, the two unique potentials for which every bounded orbit is closed are: $$U(r)=-\frac{k}{r}\quad\text{and}\quad U(r)=\frac{k}{2}r^2\quad k>0$$
  \end{theorem}
  \subsubsection{Conics}
  \begin{definition}
    A \emph{conic} is a curve obtained as the intersection of the surface of a cone with a plane.
  \end{definition}
  \begin{proposition}
    A set of points $\mathfrak{P}$ form a conic if and only if $\forall P\in\mathfrak{P}$ there exist a fixed point $F$ and a fixed line $\ell$ such that: $$\varepsilon:=\frac{d(P,F)}{d(P,\ell)}=\const$$ If we center the origin at $F$, we obtain In that case, $F$ is called \emph{focus}; $\ell$, \emph{directrix}, and $\varepsilon$, \emph{eccentricity}. Using the notation of \mcref{CM_conics}, we have:
    \begin{equation}\label{CM_eq-conics}
      \varepsilon=\frac{\sqrt{x^2+y^2}}{|x-d|}=\frac{r}{|r\cos\theta -d|}
    \end{equation}
    The line $QQ'$ is called \emph{latus rectum}. We define the \emph{semi-latus rectum} as: $$\alpha:=\frac{\overline{QQ'}}{2}=r(\pi/2)=\varepsilon d$$
    \begin{center}
      \begin{minipage}{\linewidth}
        \centering
        \includestandalone[mode=image|tex,width=0.5\linewidth]{Physics/2nd/Classical_mechanics/Images/conics}
        \captionof{figure}{}
        \label{CM_conics}
      \end{minipage}
    \end{center}
  \end{proposition}
  \begin{definition}[Ellipse: $0<\varepsilon<1$]
    Consider the previous case with $0<\varepsilon<1$. In that case, \mcref{CM_eq-conics} becomes:
    \begin{equation}\label{CM_eq-ellipse}
      r(\theta)=\frac{\alpha}{\varepsilon\cos\theta+1}
    \end{equation}
    Hence, $r(\theta)$ exists and it is finite for any value of $\theta$. Therefore, the curve is closed. This kind of conic is called \emph{ellipse}\footnote{Looking at \mcref{CM_ellipse} one may note that because of the symmetry of the ellipse it should be another focus $F'$ symmetric to $F$ with respect to $C$ and another directrix $\ell'$ symmetric to $\ell$ with respect to $C$. Indeed, this is true.}.
    Following the notation of \mcref{CM_ellipse}, we have the following properties for the ellipse:
    \begin{itemize}
      \item $\displaystyle A=r_\text{min}=\frac{\alpha}{1+\varepsilon}$
      \item $\displaystyle A'=r_\text{max}=\frac{\alpha}{1-\varepsilon}$
      \item \emph{Semi-major axis}: $\displaystyle a=\frac{\alpha}{1-\varepsilon^2}$
      \item \emph{Semi-minor axis}: $\displaystyle b=\frac{\alpha}{\sqrt{1-\varepsilon^2}}$
      \item \emph{Linear eccentricity}: $\displaystyle c=\frac{\alpha\varepsilon}{1-\varepsilon^2}$
      \item $a^2-b^2=c^2$
    \end{itemize}
    In Cartesian coordinates, the points of an ellipse satisfy: $${\left(\frac{x}{a}\right)}^2+{\left(\frac{y}{b}\right)}^2=1$$
    \begin{center}
      \begin{minipage}{\linewidth}
        \centering
        \includestandalone[mode=image|tex,width=0.75\linewidth]{Physics/2nd/Classical_mechanics/Images/ellipse}
        \captionof{figure}{Ellipse}
        \label{CM_ellipse}
      \end{minipage}
    \end{center}
  \end{definition}
  \begin{proposition}
    Taking $d=R/\varepsilon$, from \mcref{CM_eq-ellipse} we get $\displaystyle r(\theta)=R$, which is the equation of a circle of radius $R$.
  \end{proposition}
  \begin{definition}[Parabolla: $\varepsilon=1$]
    Consider now the case with $\varepsilon=1$. In that case, \mcref{CM_eq-conics} becomes: $$r(\theta)=\frac{\alpha}{\cos\theta+ 1}$$ Note that $r(\theta)$ is defined $\forall\theta\in[0,2\pi)\setminus\{\pi\}$. Therefore, the curve is open and unbounded. This kind of conic is called \emph{parabolla}.

    In Cartesian coordinates, the points of a parabolla satisfy: $$x=-\frac{y^2-\alpha^2}{2\alpha}$$
  \end{definition}
  \begin{definition}[Hyperbola: $\varepsilon>1$]
    Consider now the case with $\varepsilon>1$. In that case, \mcref{CM_eq-conics} becomes: $$r(\theta)=\frac{\alpha}{\varepsilon\cos\theta\pm 1}$$ Note that $r(\theta)$ is defined only for $\left|\frac{1}{\varepsilon}\right|<\cos\theta$. Therefore, the curve is open and unbounded. This kind of conic is called \emph{hyperbola}.
    Following the notation of \mcref{CM_hyperbola}, we have the following properties for the ellipse:
    \begin{itemize}
      \item $\displaystyle A=r_\text{min}=\frac{\alpha}{1+\varepsilon}$
      \item \emph{Semi-major axis}: $\displaystyle a=\frac{\alpha}{\varepsilon^2-1}$
      \item \emph{Semi-minor axis}: $\displaystyle b=\frac{\alpha}{\sqrt{\varepsilon^2-1}}$
      \item \emph{Linear eccentricity}: $\displaystyle c=\frac{\alpha\varepsilon}{\varepsilon^2-1}$
      \item $a^2+b^2=c^2$
    \end{itemize}
    \begin{center}
      \begin{minipage}{\linewidth}
        \centering
        \includestandalone[mode=image|tex,width=0.75\linewidth]{Physics/2nd/Classical_mechanics/Images/hyperbola}
        \captionof{figure}{Hyperbola}
        \label{CM_hyperbola}
      \end{minipage}
    \end{center}
    In Cartesian coordinates, the points of a hyperbola satisfy: $${\left(\frac{x}{a}\right)}^2-{\left(\frac{y}{b}\right)}^2=1$$
  \end{definition}
  \subsubsection{Potential \texorpdfstring{$-k/r$}{-k/r}}
  \begin{proposition}
    For that kind of potentials, we have: $$f(r)=-\frac{k}{r^2}$$
    The force is thus attractive if $k>0$ and repulsive if $k<0$. The effective potential is: $$U_\text{eff}=-\frac{k}{r}+\frac{m\ell^2}{2r^2}$$
  \end{proposition}
  \begin{theorem}
    The trajectory of a particle of mass $m$ moving in this central potential is: $$r(\theta)=\frac{\alpha}{\varepsilon\cos\theta+\sign k}$$ where $\displaystyle\alpha=\frac{L^2}{m|k|}$ and $\displaystyle\varepsilon=\sqrt{1+\frac{2EL^2}{mk^2}}$. This is the equation of a conic.
    \begin{itemize}
      \item If $E<0$, then $0<\varepsilon<1$ and the orbit is an ellipse with one of their foci\footnote{Plural of focus.} being the center of forces (\emph{Kepler first law}). Their axes are:
            \begin{equation}\label{CM_axesKepler}
              a=\frac{|k|}{2|E|}\quad b=\frac{L}{\sqrt{2m|E|}}
            \end{equation}
      \item If $E=0$, then $\varepsilon=0$ and the orbit is a parabola.
      \item If $E>0$, then $\varepsilon>1$ and the orbit is a hyperbola with one of their foci being the center of forces and their axes being the same as in \mcref{CM_axesKepler}.
    \end{itemize}
  \end{theorem}
  \begin{definition}
    The \emph{apsis} of an orbit is the farthest (\emph{apocenter}, corresponding to $r=r_\text{max}$) or nearest (\emph{pericenter}, corresponding to $r=r_\text{min}$) point in the orbit with respect to the center of forces\footnote{For the special cases of the motion around the Sun or the Earth, the respective terms \emph{aphelion}-\emph{perihelion} and \emph{apogee}-\emph{perigee} are used instead.}.
  \end{definition}
  \begin{law}[Kepler third law]
    The ratio of the square of an object's orbital period with the cube of the semi-major axis of its orbit is the same for all objects orbiting the same primary. More precisely: $$T^2=\frac{4\pi^2a^3}{GM}$$ where $T$ is the period of the orbit and $M$ is the primary object\footnote{This law is only valid for the case $M\to\infty$, because otherwise we would have to use the reduced mass $\mu$ (which depends on the mass of the planet).}.
  \end{law}
  \begin{theorem}[Bertrand's theorem]
    There exist circular orbits with arbitrary radius.
  \end{theorem}
  \textcolor{blue}{FALTA COSA}
  \subsection{Coupled oscillations}

  \subsection{Rotations}
\end{multicols}
\end{document}