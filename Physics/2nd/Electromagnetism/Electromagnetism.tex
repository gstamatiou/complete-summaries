\documentclass[../../../main.tex]{subfiles}

\begin{document}
\begin{multicols}{2}[\section{Electromagnetism}]
  \subsection{Vector calculus}
  \subsubsection{Vector algebra}
  FALTA COSA
  \begin{prop}
    Let $\vectorfunction{u},\vectorfunction{v}\in\RR^n$ be vectors such that $\vectorfunction{u}=(u_1,\ldots,u_n)$ and $\vectorfunction{v}=(v_1,\ldots,v_n)$ and $\alpha\in\RR$. Then:
    \begin{itemize}
      \item $\vectorfunction{u}+\vectorfunction{v}=(u_1+v_1,\ldots,u_n+v_n)$
      \item $\alpha\vectorfunction{v}=(\alpha v_1,\ldots,\alpha v_n)$
    \end{itemize}
  \end{prop}
  \begin{prop}
    The function $$\function{\cdot}{\RR^n\times\RR^n}{\RR}{((u_1,\ldots,u_n),(v_1,\ldots,v_n))}{\displaystyle\sum_{i=1}^nu_iv_i}$$ is an inner product\footnote{Recall definition \ref{LA_inner}.}, which is called \textit{dot product}, and it has the following associated norm: $$\|\vectorfunction{v}\|=\sqrt{\vectorfunction{v}\cdot\vectorfunction{v}}=\sqrt{\sum_{i=1}^n{v_i}^2}$$ for all $\vectorfunction{v}=(v_1,\ldots,v_n)\in\RR^n$.
  \end{prop}
  \textcolor{green}{FALTA DEFINIR BÉ EL CROSS PRODUCT}
  \begin{definition}
    Let $V$ be a vector space of dimension 3\footnote{It can also be defined for vector spaces of dimension 0, 1 and 7.}. A \textit{cross product} is a function: (POSAR EN FUNCIÓ function)
    such that satisfies the following properties $\forall \vectorfunction{u},\vectorfunction{v},\vectorfunction{w}\in V$ and $\forall \lambda,\mu\in V$
    \begin{enumerate}
      \item \hfill
            \begin{gather*}
              (\lambda\vectorfunction{u}+\mu\vectorfunction{v})\crossprod \vectorfunction{w}=\lambda(\vectorfunction{u}\crossprod\vectorfunction{w})+\mu(\vectorfunction{v}\crossprod\vectorfunction{w})\\
              \vectorfunction{u}\crossprod(\lambda\vectorfunction{v}+\mu\vectorfunction{w})=\lambda(\vectorfunction{u}\crossprod\vectorfunction{v})+\mu(\vectorfunction{u}\crossprod\vectorfunction{w})
            \end{gather*}
      \item $\vectorfunction{u}\cdot(\vectorfunction{u}\crossprod\vectorfunction{v})=\vectorfunction{v}\cdot(\vectorfunction{u}\crossprod\vectorfunction{v})=0$
      \item $\vectorfunction{u}\crossprod\vectorfunction{u}=0$
    \end{enumerate}
  \end{definition}
  \begin{prop}
    Let $\vectorfunction{u},\vectorfunction{v}\in\RR^3$ be vectors such that $\vectorfunction{u}=(u_1,u_2,u_3)$ and $\vectorfunction{v}=(v_1,v_2,v_3)$. Then, the function $\crossprod:\RR^3\times\RR^3\rightarrow\RR^3$ defined as $$\vectorfunction{u}\crossprod\vectorfunction{v}=
      u_1v_1+u_2v_2+u_3v_3$$ for all $\vectorfunction{u}=(u_1,u_2,u_3),\vectorfunction{v}=(v_1,v_2,v_3)\in\RR^3$ is a cross product. Moreover, $$\|\vectorfunction{u}\crossprod\vectorfunction{v}\|=\vectorfunction{u}\vectorfunction{v}\sin\theta$$ where $\theta$ is the angle between $\vectorfunction{u}$ and $\vectorfunction{v}$. FALTA COPIAR LA FORMULA DE EiM.
  \end{prop}
  QUATERNIONS
  \begin{prop}
    Let $\vectorfunction{u},\vectorfunction{v}\in\RR^3$ be vectors. Then:
    \begin{itemize}
      \item $\vectorfunction{u}\perp\vectorfunction{v}\iff\vectorfunction{u}\cdot\vectorfunction{v}=0$
      \item $\vectorfunction{u}\parallel\vectorfunction{v}\iff\vectorfunction{u}\crossprod\vectorfunction{v}=\vectorfunction{0}$
    \end{itemize}
  \end{prop}
  \begin{definition}
    Let $\vectorfunction{u},\vectorfunction{v}\in\RR^3$ be vectors such that $\vectorfunction{u}=(u_1,u_2,u_3)$ and $\vectorfunction{v}=(v_1,v_2,v_3)$.. We define the \textit{dyadic product $\vectorfunction{u}\otimes\vectorfunction{v}$} (or $\vectorfunction{u}\vectorfunction{v}$) of $\vectorfunction{u}$ and $\vectorfunction{v}$ as: $$\vectorfunction{u}\otimes\vectorfunction{v}:=\vectorfunction{u}\vectorfunction{v}:=
      \begin{pmatrix}
        u_1v_1 & u_1v_2 & u_1v_3 \\
        u_2v_1 & u_2v_2 & u_2v_3 \\
        u_3v_1 & u_3v_2 & u_3v_3 \\
      \end{pmatrix}$$
  \end{definition}
  \begin{definition}
    A \textit{scalar field} associates a scalar value to each point in a space.
    A \textit{vector field} associates a vector to each point in a space.
  \end{definition}
  \subsection{Electrostatics}
  \begin{prop}[Columb's law]
    Let $q_1,q_2$ be two point charges at positions $\vectorfunction{r}_1,\vectorfunction{r}_2$, respectively. Then the force $\vectorfunction{F}_2$ experienced by $q_2$ in the vicinity of $q_1$ is given by $$\vectorfunction{F}_{12}=kq_1q_2\frac{\vectorfunction{r}_1-\vectorfunction{r}_2}{\|\vectorfunction{r}_1-\vectorfunction{r}_2\|^3},$$ where $k=\frac{1}{4\pi\varepsilon_0}$ and $\varepsilon_0=8,854\;F/m$ is the vacuum permittivity.
  \end{prop}
  \begin{prop}[Electric field]
    We define the \textit{electric field} $\vectorfunction{E}$ as the force per unit of charge. For a point charge, we have that the electric field created by $q_1$ i the position of $r_2$ is $$\vectorfunction{F}_2=q_2\vectorfunction{E}_1(\vectorfunction{r}_2),\quad\vectorfunction{E}_1(\vectorfunction{r}_2)=kq_1\frac{\vectorfunction{r}_1-\vectorfunction{r}_2}{\|\vectorfunction{r}_1-\vectorfunction{r}_2\|^3}.$$
  \end{prop}
  \begin{prop}[Superposition principle] \textcolor{green}{ARREGLAR}
    Let $\rho(\vectorfunction{r})=\frac{dq}{d\mathcal{V}}$ be the volume charge density of an object. Then we have that $$\vectorfunction{F}=\int_\mathcal{V}\rho(\vectorfunction{r})\vectorfunction{E}(\vectorfunction{r})d\mathcal{V},\quad\vectorfunction{E}(\vectorfunction{r})=k\int_\mathcal{V}\rho(\vectorfunction{r}')\frac{\vectorfunction{r}-\vectorfunction{r}'}{\|\vectorfunction{r}-\vectorfunction{r}'\|^3}d\mathcal{V}.\footnote{Analogously we can define $\sigma(\vectorfunction{r})=\frac{dq}{d\mathcal{S}}$ to be the surface charge density and $\lambda(\vectorfunction{r})=\frac{dq}{d\ell}$ to be the linear charge density, and the integrals become as expected.}$$
  \end{prop}
  \begin{prop}[Electric field superposition principle]
    Let $\rho(\vectorfunction{r})=dq/d\mathcal{V}$, $\sigma(\vectorfunction{r})=dq/d\mathcal{A}$ and $\lambda(\vectorfunction{r})=dq/d\ell$ be the volume, surface and linear charge densities of an object, respectively. Then the resulting electric field at a point $\vectorfunction{r}$ is
    \begin{gather*}
      \vectorfunction{E}(\vectorfunction{r})=k\int_\mathcal{V}\rho(\vectorfunction{r}')\frac{\vectorfunction{r}-\vectorfunction{r}'}{\|\vectorfunction{r}-\vectorfunction{r}'\|^3}d\mathcal{V}'.\\
      \vectorfunction{E}(\vectorfunction{r})=k\int_\mathcal{A}\sigma(\vectorfunction{r}')\frac{\vectorfunction{r}-\vectorfunction{r}'}{\|\vectorfunction{r}-\vectorfunction{r}'\|^3}d\mathcal{A}'.\\
      \vectorfunction{E}(\vectorfunction{r})=k\int_L\lambda(\vectorfunction{r}')\frac{\vectorfunction{r}-\vectorfunction{r}'}{\|\vectorfunction{r}-\vectorfunction{r}'\|^3}d\ell.
    \end{gather*}
  \end{prop}
  \begin{prop}[Gau\ss's\space theorem]
    \begin{equation}
      \grad\cdot\vectorfunction{E}=\frac{\rho}{\varepsilon_0}\iff\oint_\mathcal{S}\vectorfunction{E}\cdot\vectorfunction{n}d\mathcal{S}=\frac{1}{\varepsilon_0}\int_\mathcal{V}\rho(r)d\mathcal{V}=\frac{Q_T}{\varepsilon_0},
      \label{gauss}
    \end{equation} where $Q_T$ is the total charge enclosed within $\mathcal{V}$.
  \end{prop}
  \begin{prop}[Work]
    The work required to move a point charge $q$ from $a$ to $b$ is $$W_{b\to a}=-q\int_a^b\vectorfunction{E}\cdot d\ell,$$ where the negative sign indicates that the work is done against the field.
  \end{prop}
  \begin{prop}[Electric potential]
    The electric potential $\phi$ in a point $\vectorfunction{r}$ is defined as:
    \begin{equation}
      \vectorfunction{E}=-\grad\phi,\quad\phi(\vectorfunction{r})=k\int\frac{\rho(\vectorfunction{r}')}{\|r-r'\|}d\mathcal{V}.
      \label{potential}
    \end{equation} And then, $$\phi_a-\phi_b=-\int_b^a\vectorfunction{E}\cdot d\ell=\int_a^b\vectorfunction{E}\cdot d\ell.$$ Alternatively we can define the potential from the electric energy.  And also if we consider the point $P$ as a reference point we can define the \textit{electric energy} $U_a$ as follows $$U_a-U_b=-q\int_P^a\vectorfunction{E}\cdot d\ell+q\int_P^b\vectorfunction{E}\cdot d\ell=-q\int_b^a\vectorfunction{E}\cdot d\ell=W_{b\to a}.$$ And finally we get $$\Delta U(\vectorfunction{r})=q\Delta\phi(\vectorfunction{r}).$$
  \end{prop}
  \begin{prop}[Poisson and Laplace equations]
    Taking into account formulas \ref{gauss}, \ref{potential}, we get Poisson's equation $$\laplacian\phi=-\frac{\rho}{\varepsilon_0}.$$ If $\rho=0$, we obtain Laplace's equation: $$\laplacian\phi=0.$$
  \end{prop}
  \begin{definition}
    A conductor is a material in which charges can move freely.
  \end{definition}
  \begin{prop}[Faraday's cage]
    Inside a cavity with no charge of a conductor we have $\vectorfunction{E}=0$ no matter how many charges and the potential are outside. This cavity is known as \textit{Faraday's cage}.
  \end{prop}
  \subsubsection{System of $N$ conductors}
  \textcolor{green}{FALTA COSA.}
  \begin{definition}[Capacitor]
    A capacitor is formed by two conductors of charges $\pm q$ and a potential difference $\Delta\phi$ not depending on the charge of other conductors. As a result, we have the following equality: $$\Delta\phi=\frac{q}{C},$$ where $C$ is the \textit{capacitance} of the capacitor an it unit is the \textit{farad} ($[C]=F$).
  \end{definition}
  \begin{prop}[Capacitors in series and parallel]
    The total capacitance of $n$ capacitors in series is $$\frac{1}{C_\text{total}}=\sum_{j=1}^n\frac{1}{C_i}.$$
    The total capacitance of $n$ capacitors in parallel is $$C_\text{total}=\sum_{j=1}^nC_i.$$
  \end{prop}
  \subsubsection{Potential energy of a charge distribution}
  \begin{prop}[Discrete charge distribution]
    Consider a distribution of $n$ charges $q_i$. If $\phi_{ij}$ is the potential caused by the charge $j$ on the point where the charge $i$ is located, we have that the energy of the distribution $W$ is $$W=\sum_{i>j}q_i\phi_{ij}=\frac{1}{2}\sum_{i=1}^nq_i\phi_i,$$ where $\displaystyle\phi_i=\sum_{i\ne j}\phi_{ij}$.
  \end{prop}
  \begin{prop}[Continuous charge distribution]
    Consider a continuous charge distribution of density $\rho$. Then $$W=\frac{1}{2}\int_\mathcal{V}\rho\phi d\mathcal{V}=\frac{\varepsilon_0}{2}\int_{\mathbb{R}^3}E^2d\mathcal{V}.$$
  \end{prop}
  \begin{definition}
    The radius in which the electrostatic energy equals the rest energy of an electron is called \textit{classical electron radius} and it's equal to: $$r_0=\frac{1}{4\pi\varepsilon_0}\frac{e^2}{m_ec^2}.$$
  \end{definition}
\end{multicols}
\end{document}