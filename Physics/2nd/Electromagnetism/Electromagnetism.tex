\documentclass[../../../main.tex]{subfiles}

\begin{document}
\begin{multicols}{2}[\section{Electromagnetism}]
  \subsection{Electrostatics}
  \begin{concept}[Columb's law]
    Let $q_1,q_2$ be two point charges at positions $\boldsymbol{r}_1,\boldsymbol{r}_2$, respectively. Then the force $\boldsymbol{F}_2$ experienced by $q_2$ in the vicinity of $q_1$ is given by $$\boldsymbol{F}_{12}=kq_1q_2\frac{\boldsymbol{r}_1-\boldsymbol{r}_2}{\|\boldsymbol{r}_1-\boldsymbol{r}_2\|^3},$$ where $k=\frac{1}{4\pi\varepsilon_0}$ and $\varepsilon_0=8,854\;F/m$ is the vacuum permittivity.
  \end{concept}
  \begin{concept}[Electric field]
    We define the \textit{electric field} $\boldsymbol{E}$ as the force per unit of charge. For a point charge, we have that the electric field created by $q_1$ i the position of $r_2$ is $$\boldsymbol{F}_2=q_2\boldsymbol{E}_1(\boldsymbol{r}_2),\quad\boldsymbol{E}_1(\boldsymbol{r}_2)=kq_1\frac{\boldsymbol{r}_1-\boldsymbol{r}_2}{\|\boldsymbol{r}_1-\boldsymbol{r}_2\|^3}.$$
  \end{concept}
  \begin{concept}[Superposition principle] \textcolor{green}{ARREGLAR}
    Let $\rho(\boldsymbol{r})=\frac{dq}{d\mathcal{V}}$ be the volume charge density of an object. Then we have that $$\boldsymbol{F}=\int_\mathcal{V}\rho(\boldsymbol{r})\boldsymbol{E}(\boldsymbol{r})d\mathcal{V},\quad\boldsymbol{E}(\boldsymbol{r})=k\int_\mathcal{V}\rho(\boldsymbol{r}')\frac{\boldsymbol{r}-\boldsymbol{r}'}{\|\boldsymbol{r}-\boldsymbol{r}'\|^3}d\mathcal{V}.\footnote{Analogously we can define $\sigma(\boldsymbol{r})=\frac{dq}{d\mathcal{S}}$ to be the surface charge density and $\lambda(\boldsymbol{r})=\frac{dq}{d\ell}$ to be the linear charge density, and the integrals become as expected.}$$
  \end{concept}
  \begin{concept}[Electric field superposition principle]
    Let $\rho(\boldsymbol{r})=dq/d\mathcal{V}$, $\sigma(\boldsymbol{r})=dq/d\mathcal{A}$ and $\lambda(\boldsymbol{r})=dq/d\ell$ be the volume, surface and linear charge densities of an object, respectively. Then the resulting electric field at a point $\boldsymbol{r}$ is
    \begin{gather*}
      \boldsymbol{E}(\boldsymbol{r})=k\int_\mathcal{V}\rho(\boldsymbol{r}')\frac{\boldsymbol{r}-\boldsymbol{r}'}{\|\boldsymbol{r}-\boldsymbol{r}'\|^3}d\mathcal{V}'.\\
      \boldsymbol{E}(\boldsymbol{r})=k\int_\mathcal{A}\sigma(\boldsymbol{r}')\frac{\boldsymbol{r}-\boldsymbol{r}'}{\|\boldsymbol{r}-\boldsymbol{r}'\|^3}d\mathcal{A}'.\\
      \boldsymbol{E}(\boldsymbol{r})=k\int_L\lambda(\boldsymbol{r}')\frac{\boldsymbol{r}-\boldsymbol{r}'}{\|\boldsymbol{r}-\boldsymbol{r}'\|^3}d\ell.
    \end{gather*}
  \end{concept}
  \begin{concept}[Gau\ss's\space theorem]
    \begin{equation}
      \grad\cdot\boldsymbol{E}=\frac{\rho}{\varepsilon_0}\iff\oint_\mathcal{S}\boldsymbol{E}\cdot\boldsymbol{n}d\mathcal{S}=\frac{1}{\varepsilon_0}\int_\mathcal{V}\rho(r)d\mathcal{V}=\frac{Q_T}{\varepsilon_0},
      \label{gauss}
    \end{equation} where $Q_T$ is the total charge enclosed within $\mathcal{V}$.
  \end{concept}
  \begin{concept}[Work]
    The work required to move a point charge $q$ from $a$ to $b$ is $$W_{b\to a}=-q\int_a^b\boldsymbol{E}\cdot d\ell,$$ where the negative sign indicates that the work is done against the field.
  \end{concept}
  \begin{concept}[Electric potential]
    The electric potential $\phi$ in a point $\boldsymbol{r}$ is defined as:
    \begin{equation}
      \boldsymbol{E}=-\grad\phi,\quad\phi(\boldsymbol{r})=k\int\frac{\rho(\boldsymbol{r}')}{\|r-r'\|}d\mathcal{V}.
      \label{potential}
    \end{equation} And then, $$\phi_a-\phi_b=-\int_b^a\boldsymbol{E}\cdot d\ell=\int_a^b\boldsymbol{E}\cdot d\ell.$$ Alternatively we can define the potential from the electric energy.  And also if we consider the point $P$ as a reference point we can define the \textit{electric energy} $U_a$ as follows $$U_a-U_b=-q\int_P^a\boldsymbol{E}\cdot d\ell+q\int_P^b\boldsymbol{E}\cdot d\ell=-q\int_b^a\boldsymbol{E}\cdot d\ell=W_{b\to a}.$$ And finally we get $$\Delta U(\boldsymbol{r})=q\Delta\phi(\boldsymbol{r}).$$
  \end{concept}
  \begin{concept}[Poisson and Laplace equations]
    Taking into account formulas \ref{gauss}, \ref{potential}, we get Poisson's equation $$\laplacian\phi=-\frac{\rho}{\varepsilon_0}.$$ If $\rho=0$, we obtain Laplace's equation: $$\laplacian\phi=0.$$
  \end{concept}
  \begin{definition}
    A conductor is a material in which charges can move freely.
  \end{definition}
  \begin{concept}[Faraday's cage]
    Inside a cavity with no charge of a conductor we have $\boldsymbol{E}=0$ no matter how many charges and the potential are outside. This cavity is known as \textit{Faraday's cage}.
  \end{concept}
  \subsubsection*{System of $N$ conductors}
  \textcolor{green}{FALTA COSA.}
  \begin{definition}[Capacitor]
    A capacitor is formed by two conductors of charges $\pm q$ and a potential difference $\Delta\phi$ not depending on the charge of other conductors. As a result, we have the following equality: $$\Delta\phi=\frac{q}{C},$$ where $C$ is the \textit{capacitance} of the capacitor an it unit is the \textit{farad} ($[C]=F$).
  \end{definition}
  \begin{concept}[Capacitors in series and parallel]
    The total capacitance of $n$ capacitors in series is $$\frac{1}{C_\text{total}}=\sum_{j=1}^n\frac{1}{C_i}.$$
    The total capacitance of $n$ capacitors in parallel is $$C_\text{total}=\sum_{j=1}^nC_i.$$
  \end{concept}
  \subsubsection*{Potential energy of a charge distribution}
  \begin{concept}[Discrete charge distribution]
    Consider a distribution of $n$ charges $q_i$. If $\phi_{ij}$ is the potential caused by the charge $j$ on the point where the charge $i$ is located, we have that the energy of the distribution $W$ is $$W=\sum_{i>j}q_i\phi_{ij}=\frac{1}{2}\sum_{i=1}^nq_i\phi_i,$$ where $\displaystyle\phi_i=\sum_{i\ne j}\phi_{ij}$.
  \end{concept}
  \begin{concept}[Continuous charge distribution]
    Consider a continuous charge distribution of density $\rho$. Then $$W=\frac{1}{2}\int_\mathcal{V}\rho\phi d\mathcal{V}=\frac{\varepsilon_0}{2}\int_{\mathbb{R}^3}E^2d\mathcal{V}.$$
  \end{concept}
  \begin{definition}
    The radius in which the electrostatic energy equals the rest energy of an electron is called \textit{classical electron radius} and it's equal to: $$r_0=\frac{1}{4\pi\varepsilon_0}\frac{e^2}{m_ec^2}.$$
  \end{definition}
\end{multicols}
\end{document}