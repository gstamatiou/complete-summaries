\documentclass[../../../main.tex]{subfiles}

\begin{document}
\begin{multicols}{2}[\section{Electromagnetism}]
  \subsection{Vector calculus}
  \subsubsection{Vector algebra}
  \begin{definition}
    In Cartesian coordinates a \emph{vector} $\vf{v}$ is an expression of the form: $$\vf{v}=v_x\vf{e}_x+v_y\vf{e}_y+v_z\vf{e}_z$$
    where $v_x,v_y,v_z\in\RR$ and $\vf{e}_x$, $\vf{e}_y$ and $\vf{e}_z$ are the \emph{unit vectors} in the directions of the axes $x$, $y$ and $z$, respectively. In this case, we will write $\vf{v}=(v_x,v_y,v_z)$.
  \end{definition}
  \begin{prop}
    Let $\vf{u},\vf{v}\in\RR^3$ be vectors such that $\vf{u}=(u_x,u_y,u_z)$ and $\vf{v}=(v_x,v_y,v_z)$ and $\alpha\in\RR$. Then:
    \begin{itemize}
      \item $\vf{u}+\vf{v}=(u_x+v_x,u_y+v_y,u_z+v_z)$
      \item $\alpha\vf{v}=(\alpha v_x,\alpha v_y,\alpha v_z)$
    \end{itemize}
  \end{prop}
  \begin{prop}
    The function $$\function{\cdot}{\RR^3\times\RR^3}{\RR}{((u_x,u_y,u_z),(v_x,v_y,v_z))}{u_xv_x+u_yv_y+u_zv_z}$$ is an inner product\footnote{Recall \cref{LA_inner}.} in $\RR^3$, which is called \emph{dot product}, and it has the following associated norm: $$\|\vf{v}\|=\sqrt{\vf{v}\cdot\vf{v}}=\sqrt{{v_x}^2+{v_y}^2+{v_z}^2}$$ for all $\vf{v}=(v_x,v_y,v_z)\in\RR^n$.
  \end{prop}
  \begin{prop}
    Let $\vf{u},\vf{v},\vf{w}\in\RR^3$ be vectors and $\alpha,\beta \in\RR$. Then:
    \begin{enumerate}
      \item $\vf{u}\cdot\vf{v}=\vf{v}\cdot\vf{u}$
      \item $\vf{u}\cdot(\alpha\vf{v}+\beta\vf{w})=\alpha(\vf{u}\cdot\vf{v})+\beta(\vf{u}\cdot\vf{w})$
      \item $(\alpha\vf{u})\cdot(\beta\vf{v})=\alpha\beta(\vf{u}\cdot\vf{v})$
      \item $\vf{u}\cdot\vf{v}=\norm{\vf{u}}\norm{\vf{v}}\cos\theta$, where $\theta$ is the angle between $\vf{u}$ and $\vf{v}$.
    \end{enumerate}
  \end{prop}
  \begin{definition}
    Let $\vf{u},\vf{v}\in\RR^3$ be vectors such that $\vf{u}=(u_x,u_y,u_z)$ and $\vf{v}=(v_x,v_y,v_z)$. We define the \emph{cross product} between $\vf{u}$ and $\vf{v}$ as: $$\vf{u}\times\vf{v}=
      \begin{vmatrix}
        \vf{e}_x & \vf{e}_y & \vf{e}_z \\
        u_x      & u_y      & u_z      \\
        v_x      & v_y      & v_z      \\
      \end{vmatrix}$$
  \end{definition}
  \begin{prop}
    Let $\vf{u},\vf{v},\vf{w}\in\RR^3$ be vectors and $\alpha,\beta \in\RR$. Then:
    \begin{enumerate}
      \item $\vf{u}\times\vf{v}=-(\vf{v}\times\vf{u})$
      \item $\vf{u}\times(\alpha\vf{v}+\beta\vf{w})=\alpha(\vf{u}\times\vf{v})+\beta(\vf{u}\times\vf{w})$
      \item $(\alpha\vf{u})\times(\beta\vf{v})=\alpha\beta(\vf{u}\times\vf{v})$
      \item $\vf{u}\times\vf{v}=\norm{\vf{u}}\norm{\vf{v}}\sin\theta$, where $\theta$ is the angle between $\vf{u}$ and $\vf{v}$.
    \end{enumerate}
  \end{prop}
  \begin{prop}
    Let $\vf{u},\vf{v},\vf{w}\in\RR^3$ be vectors. Then:
    \begin{enumerate}
      \item $\vf{u}\cdot(\vf{v}\times\vf{w})=\det(\transpose{u},\transpose{v},\transpose{w})$
      \item $\vf{u}\times(\vf{v}\times\vf{w})=\vf{v}(\vf{u}\cdot\vf{w})-\vf{w}(\vf{u}\cdot\vf{v})$
      \item $(\alpha\vf{u})\times(\beta\vf{v})=\alpha\beta(\vf{u}\times\vf{v})$
      \item $\vf{u}\times\vf{v}=\norm{\vf{u}}\norm{\vf{v}}\sin\theta$, where $\theta$ is the angle between $\vf{u}$ and $\vf{v}$.
    \end{enumerate}
  \end{prop}
  \begin{prop}
    Let $\vf{u},\vf{v}\in\RR^3$ be vectors. Then:
    \begin{itemize}
      \item $\vf{u}\perp\vf{v}\iff\vf{u}\cdot\vf{v}=0$
      \item $\vf{u}\parallel\vf{v}\iff\vf{u}\crossprod\vf{v}=\vf{0}$
    \end{itemize}
  \end{prop}
  \begin{definition}
    Let $\vf{u},\vf{v}\in\RR^3$ be vectors such that $\vf{u}=(u_x,u_y,u_z)$ and $\vf{v}=(v_x,v_y,v_z)$. We define the \emph{dyadic product} $\vf{u}\otimes\vf{v}$ (or $\vf{u}\vf{v}$) of $\vf{u}$ and $\vf{v}$ as: $$\vf{u}\otimes\vf{v}:=
      \begin{pmatrix}
        u_xv_x & u_xv_y & u_xv_z \\
        u_yv_x & u_yv_y & u_yv_z \\
        u_zv_x & u_zv_y & u_zv_z \\
      \end{pmatrix}$$
  \end{definition}
  \begin{definition}
    A \emph{scalar field} is a function $f:\RR^3\rightarrow\RR$ that associates a scalar value to each point in a space.
  \end{definition}
  \begin{definition}
    A \emph{vector field} is a function $\vf{F}:\RR^3\rightarrow\RR^3$ associates a vector to each point in a space.
  \end{definition}
  \begin{definition}
    We define \emph{position vector} of a point $(x,y,z)\in\RR^3$ as: $$\vf{r}:=x\vf{e}_x+y\vf{e}_y+z\vf{e}_z$$
    whose norm is: $$r:=\norm{\vf{r}}=\sqrt{x^2+y^2+z^2}$$
  \end{definition}
  \subsubsection{Vectorial operators}
  \begin{definition}
    We define the \emph{nabla operator} $\grad$ as: $$\grad:=\pdv{}{x}\vf{e}_x+\pdv{}{y}\vf{e}_y+\pdv{}{z}\vf{e}_z$$
  \end{definition}
  \begin{definition}[Gradient]
    Let $f:\RR^3\rightarrow\RR$ be a function of class $\mathcal{C}^1$. We define the \emph{gradient} of $f$ as: $$\grad f:=\pdv{f}{x}\vf{e}_x+\pdv{f}{y}\vf{e}_y+\pdv{f}{z}\vf{e}_z$$
  \end{definition}
  \begin{definition}[Flux]
    Let $\vf{F}:\RR^3\rightarrow\RR^3$ be a vector field and $S\subset\RR^3$ be a surface. We define the \emph{flux} $\Phi$ of $\vf{F}$ across $S$ as:
    $$\Phi:=\iint_S\vf{F}\cdot \dd \vf{S}=\iint_S\vf{F}\cdot\vf{n} \dd S$$
    where $\vf{n}$ is the unit normal vector to $\dd S$.
  \end{definition}
  \begin{definition}[Divergence]
    Let $\vf{F}:\RR^3\rightarrow\RR^3$ be a vector field of class $\mathcal{C}^1$. We define the \emph{divergence} of $\vf{F}$ at a point $x\in\RR^3$ as:
    $$\divp \vf{F}(x):=\lim_{r\to 0}\frac{1}{\text{vol}(B(x,r))}\oiint_{S}\vf{F}\cdot\vf{n} \dd S$$
  \end{definition}
  \begin{prop}
    Let $\vf{F}:\RR^3\rightarrow\RR^3$ be a vector field such that $\vf{F}=(F_x,F_y,F_z)$. Then, in Cartesian coordinates we have: $$\divp \vf{F}=\pdv{F_x}{x}+\pdv{F_y}{y}+\pdv{F_z}{z}$$
  \end{prop}
  \begin{definition}[Laplacian]
    Let $f:\RR^3\rightarrow\RR$ be a function of class $\mathcal{C}^2$. We define the \emph{laplacian} of $f$ as: $$\laplacian f:=\divp\grad f$$
    which in Cartesian coordinates can be written as:
    $$\laplacian f=\pdv[2]{f}{x}+\pdv[2]{f}{y}+\pdv[2]{f}{z}$$
  \end{definition}
  \begin{definition}
    Let $\vf{F}:\RR^3\rightarrow\RR^3$ be a vector field of class $\mathcal{C}^2$ such that $\vf{F}=(F_x,F_y,F_z)$. We define the \emph{laplacian} of $\vf{F}$ as: $$\laplacian\vf{F}:=\laplacian F_x\vf{e}_x+\laplacian F_y\vf{e}_y+\laplacian F_z\vf{e}_z$$
  \end{definition}
  \begin{definition}
    Let  $\vf{F}:\RR^3\rightarrow\RR^3$ be a vector field of class $\mathcal{C}^2$. Then: $$\laplacian\vf{F}=\grad(\divp\vf{F})-\rotp\rotp\vf{F}$$
  \end{definition}
  \begin{definition}[Rotational]
    Let $\vf{F}:\RR^3\rightarrow\RR^3$ be a vector field of class $\mathcal{C}^1$. We define the \emph{rotational} of $\vf{F}$ at a point $x\in\RR^3$ as:
    $$\rotp\vf{F}(a)\cdot\vf{n}=\lim_{r\to 0}\frac{1}{\text{area}(D(x,r))}\int_{\partial D_r}\vf{F}\cdot \dd \vf{s}$$
    where $D(x,r)$ is the disk of center $x$ and radius $r$.
  \end{definition}
  \begin{prop}
    Let $\vf{F}:\RR^3\rightarrow\RR^3$ be a vector field such that $\vf{F}=(F_x,F_y,F_z)$. Then, in Cartesian coordinates we have:
    \begin{multline*}
      \rotp \vf{F}=
      \begin{vmatrix}
        \vf{e}_x  & \vf{e}_y  & \vf{e}_z  \\
        \pdv{}{x} & \pdv{}{y} & \pdv{}{z} \\
        F_x       & F_y       & F_z       \\
      \end{vmatrix} = \left(\frac{\partial F_z}{\partial y}-\frac{\partial F_y}{\partial z}\right)\vf{e}_x+\\+\left(\frac{\partial F_x}{\partial z}-\frac{\partial F_z}{\partial x}\right)\vf{e}_y+\left(\frac{\partial F_y}{\partial x}-\frac{\partial F_x}{\partial y}\right)\vf{e}_z
    \end{multline*}
  \end{prop}
  \subsubsection{Theorems of vector calculus}
  \begin{theorem}[Stokes' theorem]
    Let $S$ be a parametrized surface of class $\mathcal{C}^1$ and $\partial S$ be its boundary. Let $\vf{F}:\RR^3\rightarrow\RR^3$ be a vector field of class $\mathcal{C}^1$ in a domain containing $S\cup\partial S$. Then: $$\int_{\partial S}\vf{F}\cdot \dd \vf{s}=\iint_S\rotp\vf{F}\cdot\vf{n} \dd S$$
  \end{theorem}
  \begin{theorem}[Gau\ss' or divergence theorem]
    Let $\vf{F}:\RR^3\rightarrow\RR^3$ be a vector field of class $\mathcal{C}^1$ on a symmetric region $V\subset\RR^3$ with boundary $\partial V$. Then: $$\iint_{\partial V}\vf{F}\cdot\vf{n} \dd S=\iiint_V\divp\vf{F}\dd V$$
  \end{theorem}
  \begin{theorem}[Helmholtz theorem]
    Let $\vf{F}:V\rightarrow\RR^3$ be a vector field of class $\mathcal{C}^2$ defined on a bounded domain $V\subset\RR^3$. Let $\rho=\divp\vf{F}$ and $\vf{J}=\rotp\vf{F}$. Then, $\vf{F}$ can be written as: $$\vf{F}=-\grad\phi+\rotp\vf{A}$$
    where:
    \begin{gather*}
      \Phi(\vf{r}) =\frac 1 {4\pi} \int_V \frac{\rho(\vf{r}_1)}{\norm{\vf{r}-\vf{r}_1}}\dd^3r_1 -\frac 1 {4\pi} \oint_S\frac{\vf{F} (\vf{r}_1)\cdot\vf{n}}{\norm{\vf{r}-\vf{r}_1}}\dd^2r_1\\
      \vf{A}(\vf{r}) =\frac 1 {4\pi} \int_V \frac{\vf{J}(\vf{r}_1)}{\norm{\vf{r}-\vf{r}_1}}\dd^3r_1 +\frac 1 {4\pi} \oint_S \frac{\vf{F} (\vf{r}_1)\times\vf{n}}{\norm{\vf{r}-\vf{r}_1}}\dd^2r_1
    \end{gather*}
    Moreover if $V=\RR^3$ (and is therefore unbounded) and $\vf{F}$ vanishes faster than $1/r$, when $r\to\infty$, then:
    \begin{gather*}
      \Phi(\vf{r}) =\frac 1 {4\pi} \int_{\RR^3} \frac{\rho(\vf{r}_1)}{\norm{\vf{r}-\vf{r}_1}}\dd^3r_1 \\
      \vf{A}(\vf{r}) =\frac 1 {4\pi} \int_{\RR^3} \frac{\vf{J}(\vf{r}_1)}{\norm{\vf{r}-\vf{r}_1}}\dd^3r_1
    \end{gather*}
  \end{theorem}
  \begin{theorem}
    Let $\vf{F}:\RR^3\rightarrow\RR^3$ be a vector field of class $\mathcal{C}^1$. Then:
    \begin{enumerate}
      \item $\rotp\vf{F}=0\iff\vf{F}=\grad\phi$, for some function $\phi:\RR^3\rightarrow\RR$ of class $\mathcal{C}^2$.
      \item $\divp\vf{F}=0\iff\vf{F}=\rot\vf{G}$, for some vector field $\vf{G}:\RR^3\rightarrow\RR^3$ of class $\mathcal{C}^2$.
    \end{enumerate}
    In the first case, we say that $\vf{F}$ is \emph{conservative} or \emph{irrotational}, and in the second, we say that $\vf{F}$ is \emph{solenoidal}.
  \end{theorem}
  \subsubsection{Cylindrical coordinates}
  \begin{definition}
    The \emph{cylindrical coordinates} are the coordinates obtained making the change of variable $\varphi:[0,\infty)\times[0,2\pi)\times\RR \rightarrow\RR^3$ defined as:
    \begin{align*}
      x & =r\cos\varphi \\
      y & =r\sin\varphi \\
      z & =z
    \end{align*}
  \end{definition}
  \begin{definition}
    The \emph{unit vectors in cylindrical coordinates} are:
    \begin{align*}
      \vf{e}_r       & =\frac{\pdv{\vf{r}}{r}}{\norm{\pdv{\vf{r}}{r}}}=\cos\varphi\vf{e}_x+\sin\varphi\vf{e}_y              \\
      \vf{e}_\varphi & =\frac{\pdv{\vf{r}}{\varphi}}{\norm{\pdv{\vf{r}}{\varphi}}}=-\sin\varphi\vf{e}_x+\cos\varphi\vf{e}_y \\
      \vf{e}_z       & =\frac{\pdv{\vf{r}}{z}}{\norm{\pdv{\vf{r}}{z}}}=\vf{e}_z
    \end{align*}
  \end{definition}
  \begin{theorem}[Integral in cylindrical coordinates]
    Let $f:\RR^3\rightarrow\RR$ be a function and $\varphi:[0,\infty)\times[0,2\pi)\times\RR \rightarrow\RR^3 $ be such that: $$\varphi(r,\varphi,z)\longmapsto(r\cos\varphi,r\sin\varphi,z)$$ Then, we have: $$\int_{\varphi(U)}f(x,y,z)\dd x\dd y\dd z=\int_Uf(r\cos\varphi,r\sin\varphi,z)r\dd r \dd\varphi \dd z$$
  \end{theorem}
  \begin{prop}
    Let $f:\RR^3\rightarrow\RR$ be a function. Then the gradient of $f$ in cylindrical coordinates is:
    $$\grad f=\pdv{f}{r}\vf{e}_r+\frac{1}{r}\pdv{f}{\varphi}\vf{e}_\varphi+\pdv{f}{z}\vf{e}_z\footnote{In fact, this other expression of the laplacian is used for the definition of the laplacian in non-Cartesian coordiantes.}$$
  \end{prop}
  \begin{prop}
    Let $f:\RR^3\rightarrow\RR$ be a function and $\vf{F}:\RR^3\rightarrow\RR^3$ be a vector field such that $\vf{F}=F_r\vf{e}_r+F_\theta\vf{e}_\theta+F_z\vf{e}_z$. Then, we have:
    \begin{align*}
      \grad f      & =\pdv{f}{r}\vf{e}_r+\frac{1}{r}\pdv{f}{z}\vf{e}_z+\pdv{f}{z}\vf{e}_z          \\
      \divp \vf{F} & =\frac{1}{r}\pdv{(r F_r)}{r}+\frac{1}{r}\pdv{F_\varphi}{\varphi}+\pdv{F_z}{z} \\
      \rotp \vf{F} & =\frac{1}{r}
      \begin{vmatrix}
        \vf{e}_r  & r\vf{e}_\varphi & \vf{e}_z  \\
        \pdv{}{r} & \pdv{}{\varphi} & \pdv{}{z} \\
        F_r       & rF_\varphi      & F_z       \\
      \end{vmatrix}                                                                   \\
      \laplacian f & =\frac{1}{r}\pdv{f}{r}+\frac{1}{r^2}\pdv[2]{f}{\varphi}+\pdv[2]{f}{z}
    \end{align*}
  \end{prop}
  \subsubsection{Spherical coordinates}
  \begin{definition}
    The \emph{spherical coordinates} are the coordinates obtained making the the change of variable $\varphi:[0,\infty)\times[0,2\pi)\times\RR \rightarrow\RR^3$ defined as:
    \begin{align*}
      x & =r\sin\theta\cos\varphi \\
      y & =r\sin\theta\sin\varphi \\
      z & =r\cos\theta
    \end{align*}
  \end{definition}
  \begin{definition}
    The \emph{unit vectors in spherical coordinates} are:
    \begin{align*}
      \vf{e}_r       & = \frac{\pdv{\vf{r}}{r}}{\norm{\pdv{\vf{r}}{r}}}=\sin\theta\cos\varphi\vf{e}_x+\sin\theta\sin\varphi\vf{e}_y+\cos\theta\vf{e}_x           \\
      \vf{e}_\theta  & = \frac{\pdv{\vf{r}}{\theta}}{\norm{\pdv{\vf{r}}{\theta}}}=\cos\theta\cos\varphi\vf{e}_x+\cos\theta\sin\varphi\vf{e}_y-\sin\theta\vf{e}_z \\
      \vf{e}_\varphi & = \frac{\pdv{\vf{r}}{\varphi}}{\norm{\pdv{\vf{r}}{\varphi}}}=-\sin\varphi\vf{e}_x+\cos\varphi\vf{e}_y
    \end{align*}
  \end{definition}
  \begin{theorem}[Integral in spherical coordinates]
    Let $f:\RR^3\rightarrow\RR$ be a function and $\varphi:[0,\infty)\times[0,2\pi)\times[0,\pi]\rightarrow\RR^3 $ be such that: $$\varphi(r,\varphi,\theta)\longmapsto(r\sin\theta\cos\varphi,r\sin\theta\sin\varphi,r\cos\theta)$$ Then, we have:
    \begin{multline*}
      \int f(x,y,z)\dd x\dd y\dd z=\\=\int f(r\sin\theta\cos\varphi,r\sin\theta\sin\varphi,r\cos\theta)r^2\sin\theta \dd r \dd \varphi \dd \theta
    \end{multline*}
  \end{theorem}
  \begin{prop}
    Let $f:\RR^3\rightarrow\RR$ be a function and $\vf{F}:\RR^3\rightarrow\RR^3$ be a vector field such that $\vf{F}=F_r\vf{e}_r+F_\theta\vf{e}_\theta+F_\varphi\vf{e}_\varphi$. Then, we have:
    \begin{gather*}
      \grad f       =\pdv{f}{r}\vf{e}_r+\frac{1}{r}\pdv{f}{\theta}\vf{e}_\theta+\frac{1}{r\sin\theta}\pdv{f}{\varphi}\vf{e}_\varphi                                                                       \\
      \divp \vf{F}  =\frac{1}{r^2}\pdv{(r^2 F_r)}{r}+\frac{1}{r\sin\theta}\pdv{(F_\theta\sin\theta)}{\theta}+\frac{1}{r\sin\theta}\pdv{F_\varphi}{\varphi}                                                \\
      \rotp \vf{F}  =\frac{1}{r^2\sin\theta}
      \begin{vmatrix}
        \vf{e}_r  & r\vf{e}_\theta & r\sin\theta\vf{e}_\varphi \\
        \pdv{}{r} & \pdv{}{\theta} & \pdv{}{\varphi}           \\
        F_r       & rF_\theta      & r\sin\theta F_\varphi     \\
      \end{vmatrix}                                                                                                                                                                           \\
      \laplacian f  =\frac{1}{r^2}\left[\pdv{\left(r^2\pdv{f}{r}\right)}{r}+\frac{1}{\sin\theta}\pdv{\left(\sin\theta\pdv{f}{\theta}\right)}{\theta}+\frac{1}{{(\sin\theta)}^2}\pdv[2]{f}{\varphi}\right]
    \end{gather*}
  \end{prop}
  \subsection{Electrostatics}
  \begin{prop}[Columb's law]
    Let $q_1,q_2$ be two point charges at positions $\vf{r}_1,\vf{r}_2$, respectively. Then the force $\vf{F}_2$ experienced by $q_2$ in the vicinity of $q_1$ is given by $$\vf{F}_{12}=kq_1q_2\frac{\vf{r}_1-\vf{r}_2}{\|\vf{r}_1-\vf{r}_2\|^3},$$ where $k=\frac{1}{4\pi\varepsilon_0}$ and $\varepsilon_0=8,854\;F/m$ is the vacuum permittivity.
  \end{prop}
  \begin{prop}[Electric field]
    We define the \emph{electric field} $\vf{E}$ as the force per unit of charge. For a point charge, we have that the electric field created by $q_1$ at the position of $r_2$ is $$\vf{F}_2=q_2\vf{E}_1(\vf{r}_2),\quad\vf{E}_1(\vf{r}_2)=kq_1\frac{\vf{r}_1-\vf{r}_2}{\|\vf{r}_1-\vf{r}_2\|^3}.$$
  \end{prop}
  \begin{prop}[Superposition principle] \textcolor{green}{ARREGLAR}
    Let $\rho(\vf{r})=\frac{dq}{d\mathcal{V}}$ be the volume charge density of an object. Then we have that $$\vf{F}=\int_\mathcal{V}\rho(\vf{r})\vf{E}(\vf{r})d\mathcal{V},\quad\vf{E}(\vf{r})=k\int_\mathcal{V}\rho(\vf{r}')\frac{\vf{r}-\vf{r}'}{\|\vf{r}-\vf{r}'\|^3}d\mathcal{V}.\footnote{Analogously we can define $\sigma(\vf{r})=\frac{dq}{d\mathcal{S}}$ to be the surface charge density and $\lambda(\vf{r})=\frac{dq}{d\ell}$ to be the linear charge density, and the integrals become as expected.}$$
  \end{prop}
  \begin{prop}[Electric field superposition principle]
    Let $\rho(\vf{r})=dq/d\mathcal{V}$, $\sigma(\vf{r})=dq/d\mathcal{A}$ and $\lambda(\vf{r})=dq/d\ell$ be the volume, surface and linear charge densities of an object, respectively. Then the resulting electric field at a point $\vf{r}$ is
    \begin{gather*}
      \vf{E}(\vf{r})=k\int_\mathcal{V}\rho(\vf{r}')\frac{\vf{r}-\vf{r}'}{\|\vf{r}-\vf{r}'\|^3}d\mathcal{V}'.\\
      \vf{E}(\vf{r})=k\int_\mathcal{A}\sigma(\vf{r}')\frac{\vf{r}-\vf{r}'}{\|\vf{r}-\vf{r}'\|^3}d\mathcal{A}'.\\
      \vf{E}(\vf{r})=k\int_L\lambda(\vf{r}')\frac{\vf{r}-\vf{r}'}{\|\vf{r}-\vf{r}'\|^3}d\ell.
    \end{gather*}
  \end{prop}
  \begin{prop}[Gau\ss's\space theorem]
    \begin{equation}
      \grad\cdot\vf{E}=\frac{\rho}{\varepsilon_0}\iff\oint_\mathcal{S}\vf{E}\cdot\vf{n}d\mathcal{S}=\frac{1}{\varepsilon_0}\int_\mathcal{V}\rho(r)d\mathcal{V}=\frac{Q_T}{\varepsilon_0},
      \label{E_gauss}
    \end{equation} where $Q_T$ is the total charge enclosed within $\mathcal{V}$.
  \end{prop}
  \begin{prop}[Work]
    The work required to move a point charge $q$ from $a$ to $b$ is $$W_{b\to a}=-q\int_a^b\vf{E}\cdot d\ell,$$ where the negative sign indicates that the work is done against the field.
  \end{prop}
  \begin{prop}[Electric potential]
    The electric potential $\phi$ in a point $\vf{r}$ is defined as:
    \begin{equation}
      \vf{E}=-\grad\phi,\quad\phi(\vf{r})=k\int\frac{\rho(\vf{r}')}{\|r-r'\|}d\mathcal{V}.
      \label{E_potential}
    \end{equation} And then, $$\phi_a-\phi_b=-\int_b^a\vf{E}\cdot d\ell=\int_a^b\vf{E}\cdot d\ell.$$ Alternatively we can define the potential from the electric energy.  And also if we consider the point $P$ as a reference point we can define the \emph{electric energy} $U_a$ as follows $$U_a-U_b=-q\int_P^a\vf{E}\cdot d\ell+q\int_P^b\vf{E}\cdot d\ell=-q\int_b^a\vf{E}\cdot d\ell=W_{b\to a}.$$ And finally we get $$\Delta U(\vf{r})=q\Delta\phi(\vf{r}).$$
  \end{prop}
  \begin{prop}[Poisson and Laplace equations]
    Taking into account \cref{E_gauss,E_potential}, we get Poisson's equation $$\laplacian\phi=-\frac{\rho}{\varepsilon_0}.$$ If $\rho=0$, we obtain Laplace's equation: $$\laplacian\phi=0.$$
  \end{prop}
  \begin{definition}
    A conductor is a material in which charges can move freely.
  \end{definition}
  \begin{prop}[Faraday's cage]
    Inside a cavity with no charge of a conductor we have $\vf{E}=0$ no matter how many charges and the potential are outside. This cavity is known as \emph{Faraday's cage}.
  \end{prop}
  \subsubsection{System of \texorpdfstring{$N$}{N} conductors}
  \textcolor{green}{FALTA COSA.}
  \begin{definition}[Capacitor]
    A capacitor is formed by two conductors of charges $\pm q$ and a potential difference $\Delta\phi$ not depending on the charge of other conductors. As a result, we have the following equality: $$\Delta\phi=\frac{q}{C},$$ where $C$ is the \emph{capacitance} of the capacitor an it unit is the \emph{farad} ($[C]=F$).
  \end{definition}
  \begin{prop}[Capacitors in series and parallel]
    The total capacitance of $n$ capacitors in series is $$\frac{1}{C_\text{total}}=\sum_{j=1}^n\frac{1}{C_i}.$$
    The total capacitance of $n$ capacitors in parallel is $$C_\text{total}=\sum_{j=1}^nC_i.$$
  \end{prop}
  \subsubsection{Potential energy of a charge distribution}
  \begin{prop}[Discrete charge distribution]
    Consider a distribution of $n$ charges $q_i$. If $\phi_{ij}$ is the potential caused by the charge $j$ on the point where the charge $i$ is located, we have that the energy of the distribution $W$ is $$W=\sum_{i>j}q_i\phi_{ij}=\frac{1}{2}\sum_{i=1}^nq_i\phi_i,$$ where $\displaystyle\phi_i=\sum_{i\ne j}\phi_{ij}$.
  \end{prop}
  \begin{prop}[Continuous charge distribution]
    Consider a continuous charge distribution of density $\rho$. Then $$W=\frac{1}{2}\int_\mathcal{V}\rho\phi d\mathcal{V}=\frac{\varepsilon_0}{2}\int_{\mathbb{R}^3}E^2d\mathcal{V}.$$
  \end{prop}
  \begin{definition}
    The radius in which the electrostatic energy equals the rest energy of an electron is called \emph{classical electron radius} and it's equal to: $$r_0=\frac{1}{4\pi\varepsilon_0}\frac{e^2}{m_ec^2}.$$
  \end{definition}
\end{multicols}
\end{document}