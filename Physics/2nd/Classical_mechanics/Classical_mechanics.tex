\documentclass[class=article,10pt,crop=false]{standalone}
\usepackage{standalone}
\usepackage{preamble}

\begin{document}
\begin{multicols}{2}[\section{Classical mechanics}]
\subsection{Motion in one dimension}
\subsubsection{Integration of Newton's 2nd law}
\begin{concept}[Newton's 2nd law]
If we consider the case of one particle with constant mass $m$ that moves in one dimension, then satisfies $$\ddot{x}(t)=\frac{1}{m}F(x(t),\dot{x}(t),t),$$
where we have supposed the force function $F$ is known. We also suppose initial position and velocity, denoted by $x(t_0)=x_0$ and $\dot{x}(t_0)=\dot{x}_0$ respectively, are known.
\end{concept}
\begin{concept}[Integration of Newton's 2nd law]
We consider a force that only depend on time, position and velocity.
\begin{itemize}
    \item Time dependence:
    \begin{gather*}
       \ddot{x}(t)=\frac{F(t)}{m}\implies\dot{x}(t)=\dot{t_0}+\int_{t_0}^t\frac{F(t')}{m}dt',\\
       \frac{dx}{dt}=\dot{x}\implies x(t)=x(t_0)+\int_{t_0}^t\dot{x}(t')dt'.
    \end{gather*}
    \item Position dependence:
    \begin{gather*}
         \frac{d\dot{x}}{dt}=\frac{F(x)}{m}\implies\dot{x}(x)^2=\dot{x}(x_0)^2+2\int_{x_0}^x\frac{F(x')}{m}dx',\\
         \frac{dx}{dt}=\dot{x}\implies x(t)=g^{-1}(t),
    \end{gather*}
    where $\displaystyle g(x)=\int_{x_0}^x\frac{1}{\dot{x}(x')}dx'=t$.
    \item Velocity dependence:
    \begin{gather*}
        \ddot{x}(t)=\frac{F(\dot{x})}{m}\implies\dot{x}(t)=h^{-1}(t),\\
        x(t)=x(t_0)+\int_{t_0}^th^{-1}(t')dt',
    \end{gather*}
    where $\displaystyle h(\dot{x})=\int_{\dot{x}_0}^{\dot{x}}\frac{m}{F(\dot{x}')}d\dot{x}'=t$.
\end{itemize}
\end{concept}
\subsubsection{Variable mass}
\begin{concept}[Mass accretion]
Consider two objects of masses $m(t)$ and $dm$ and velocities $\boldsymbol{v}(t)$ and $\boldsymbol{u}(t)$ respectively, which in an interval of time $dt$ the second one collide with the first one and become a unique object. If $\boldsymbol{F}^\text{ext}$ is the external force acting to the system, we have 
\begin{equation}
    \boldsymbol{F}^\text{ext}=m\dot{\boldsymbol{v}}+(\boldsymbol{v}-\boldsymbol{u})\dot{m}=\dot{\boldsymbol{p}}-\dot{m}\boldsymbol{u},
    \label{mass}
\end{equation} where $\dot{\boldsymbol{p}}$ is the momentum of the object that gains mass.\footnote{The formula is also valid for the case when the object is losing mass, i.e. $\dot{m}<0$.}
\end{concept}
\subsubsection{Rocket motion}
Consider a rocket that expels gas at a velocity of $\boldsymbol{c}$ with respect to the rocket to propel itself. Suppose the mass of the rocket is $m(t)$ and $m_0:=m(t_0)$. Our proposal is to find an expression of $\boldsymbol{v}$ to describe the motion of the rocket. For doing that, first we need to express the magnitudes in an external frame of reference. In particular, the velocity of the fuel will be $\boldsymbol{u}=\boldsymbol{v}+\boldsymbol{c}$. By equation \ref{mass} we have $$m\dot{\boldsymbol{v}}=\boldsymbol{F}^\text{ext}+\dot{m}\boldsymbol{c}.$$
\begin{concept}[Rocket without gravity]
In this situation we have $\boldsymbol{F}^\text{ext}=0$ and we will suppose $\boldsymbol{v}=v\boldsymbol{j}$, $\boldsymbol{c}=-c\boldsymbol{j}$. Finally we get \begin{equation}
    m\frac{dv}{dt}=-c\frac{dm}{dt}\implies v=c\log\frac{m_0}{m}.
    \label{rocket}
\end{equation} Consider now the discrete case, i.e. when the function $\dot{m}$ is not differentiable. For that we can consider instantaneous ejections of $\Delta m=(m_0-m_f)/n$ amount of mass where $m_f$ is the mass of the rocket after $n$ ejections of mass. For this case, we have $$v=c\sum_{k=1}^n\frac{(m_0-m_f)/n}{m_f+k(m_0-m_f)/n}=c\sum_{k=1}^n\frac{\Delta m}{m_f+k\Delta m}.\footnote{Obviously if we tend $n$ to infinity we get the equation \ref{rocket}.}$$
\end{concept}
\begin{concept}[Rocket with gravity]
Now $\boldsymbol{F}^\text{ext}=-mg\boldsymbol{j}$, and for simplicity we will consider only the case where $\dot{m}=-\beta$, $\beta>0$. Therefore, we obtain \begin{equation}
    m\frac{dv}{dt}=-mg+c\beta\implies v=c\log\frac{m_0}{m}-\frac{g}{\beta}(m_0-m).
    \label{rockg1}
\end{equation}
We observe that if $m_0g>\beta c$ then $dv/dt$ will be negative, which is not possible. Therefore in this case the formula is not correct if we are considering the rocket launch. In this case the formula becomes 
\begin{equation}
    v=c\log\frac{\beta c}{mg}-\frac{g}{\beta}\left(\frac{\beta c}{g}-m\right).
    \label{rockg2}
\end{equation}
Because of $$\dot{m}=-\beta\implies m(t)=m_0-\beta t,$$ we can express formulas \ref{rockg1}, \ref{rockg2} as
\begin{gather*}
    v(t)=c\log\frac{m_0}{m_0-\beta t}-gt,\\
    v(t)=c\log\frac{\beta c}{m_0g-g\beta t}-gt-\frac{g}{\beta}\left(\frac{\beta c}{g}-m_0\right),
\end{gather*}
respectively.
\end{concept}
\subsection{Oscillations}
\subsubsection{Simple harmonic oscillator}
\begin{concept}[Movement equation]
Consider the following differential equation: $$\ddot{x}+\omega_0^2 x=0,$$ with initials values of $x(0)=x_0$ and $\dot{x}(0)=\dot{x}_0$. Then the general solution is \begin{equation}
    x(t)=x_0\cos\omega_0t+\frac{\dot{x}_0}{\omega_0}\sin\omega_0t=A\cos(\omega_0t+\phi),
    \label{mhs}
\end{equation} where $\displaystyle A=\sqrt{x_0^2+\left(\frac{\dot{x}_0}{\omega_0}\right)^2}$ and $\displaystyle \phi=-\arctan\frac{\dot{x}_0}{\omega_0x_0}$. Such constants $\omega_0$ $[\omega_0]=\text{rad}\cdot \text{s}^{-1}$, $A$ $[A]=\text{m}$ and $\phi$ $[\phi]=\text{rad}$ are called \textit{angular frequency}, \textit{amplitude} and \textit{initial phase}, respectively. Observe that the function in equation \ref{mhs} is periodic with period $T=\frac{2\pi}{\omega_0}$ and frequency $\nu=T^{-1}=\frac{\omega_0}{2\pi}$.\footnote{From this last two equations we immediately deduce that $[T]=\text{s}$ and $[\nu]=\text{s}^{-1}=\text{Hz}$.}
\end{concept}
\begin{concept}[Phase space]
The phase space of the simple harmonic oscillator is $$\boldsymbol{u}=\begin{pmatrix}
x(t)\\
\dot{x}(t)
\end{pmatrix}=\begin{pmatrix}
A\cos(\omega_0t+\phi)\\
-A\omega_0\sin(\omega_0t+\phi)
\end{pmatrix}.$$
\end{concept}
\begin{definition}
Let $U(x)$ be a potential function of class $\mathcal{C}^2(\mathbb{R})$. We say $x_0$ is a \textit{point of stable equilibrium} if $U$ attains a maximum in $x_0$. Analogously, we say $x_0$ is a \textit{point of unstable equilibrium} if $U$ attains a minimum in $x_0$.
\end{definition}
\begin{concept}[Behaviour near a minimum]
Suppose $x_0$ is a point of stable equilibrium an let $U(x)$ be the potential function associated with a particle of mass $m$. Then if we disturb slightly the particle, it will start to oscillate at a frequency $$\omega_0=\sqrt{\frac{U''(x_0)}{m}}.$$
\end{concept}
\begin{concept}[Examples]
\hfill
\begin{itemize}
    \item Mass hanging from a spring: Let $y(t)$ be the position of the mass measured from initial string's length (without the mass) to the position of the mass at time $t$. If we disturb the system with an external force so that the mass starts to oscillate, we have $$y(t)=\frac{mg}{k}+A\cos(\omega_0t+\phi),\quad\omega_0=\sqrt{\frac{k}{m}}.$$
    \begin{figure}[ht]
        \centering
        \captionbox{Mass hanging from a spring.}{    \resizebox{\linewidth}{!}{\subimport{Images/}{springs.tex}}}
    \end{figure} 
    \item Simple pendulum: $$\theta(t)=A\cos(\omega_0t+\phi),\quad\omega_0=\sqrt{\frac{g}{l}}.$$
    \begin{figure}[ht]
        \centering
        \captionbox{Simple pendulum.}{    \resizebox{.5\linewidth}{!}{\subimport{Images/}{simple_pendulum.tex}}}
    \end{figure} 
    \item Physical pendulum: 
    $$\theta(t)=A\cos(\omega_0t+\phi),\quad\omega_0=\sqrt{\frac{mgD}{I_e}}.$$
    \begin{figure}[ht] 
        \centering 
        \captionbox{Physical pendulum.}{    \resizebox{.5\linewidth}{!}{\subimport{Images/}{physical_pendulum.tex}}}
    \end{figure} 
    \item LC circuit: $$q(t)=A\cos(\omega_0t+\phi),\quad\omega_0=\frac{1}{\sqrt{LC}}.$$
\end{itemize}
\end{concept}
\subsubsection{Damped harmonic oscillator}
\begin{concept}[Movement equation]
Consider the following differential equation: $$\ddot{x}+2\beta\dot{x}+\omega_0^2 x=0,$$ with initials values of $x(0)=x_0$ and $\dot{x}(0)=\dot{x}_0$. Then we have three cases for the general solution:
\begin{itemize}
    \item If $\beta<\omega_0$,
    \begin{equation}
        x(t)=e^{-\beta t}\left(c_1\cos\Tilde{\omega}t+c_2\sin\Tilde{\omega}t\right).
        \label{b<w}
    \end{equation}
    \item If $\beta=\omega_0$, 
    \begin{equation}
        x(t)=e^{-\beta t}\left(c_1+c_2t\right).
        \label{b=w}
    \end{equation}
    \item If $\beta>\omega_0$,
    \begin{equation}
        x(t)=c_1e^{-(\beta+\Tilde{\omega})t}+c_2e^{-(\beta-\Tilde{\omega})t}.
        \label{b>w}
    \end{equation}
\end{itemize}
Here $c_1,c_2$ are constants depending on the initial values and we have defined $\displaystyle \Tilde{\omega}=\sqrt{\left|\omega_0^2-\beta^2\right|}$.
\end{concept}
\begin{concept}[Energy of damped harmonic oscillator]
$$E=\frac{\mu}{2}\left(\dot{x}^2+\omega_0^2x^2\right),$$ where $\mu$ is a constant.
\end{concept}
\begin{concept}[Underdamped harmonic oscillator: $\beta<\omega_0$]
Coefficients $c_1,c_2$ of the general solution \ref{b<w} are: $$c_1=x_0,\quad c_2=\frac{\dot{x}_0+\beta x_0}{\Tilde{\omega}}.$$ The equation, can be simplified to $$x(t)=Ae^{-\beta t}\cos(\Tilde{\omega}t+\phi),$$ where $\displaystyle A=\sqrt{x_0^2+\left(\frac{\dot{x}_0+\beta x_0}{\Tilde{\omega}}\right)^2}$ and\\ $\displaystyle\phi=-\arctan\frac{\dot{x}_0+\beta x_0}{\Tilde{\omega}x_0}$. 
\end{concept}
\begin{definition}[Quality factor]
The \textit{quality factor} is defined as follows: $$Q:=\frac{\omega_0}{2\beta}.$$
\end{definition}
\noindent From that, we can rewrite the expression of $\Tilde{\omega}$ to get: $$\Tilde{\omega}=\omega_0\sqrt{1-\frac{1}{4Q^2}}.$$
\begin{concept}[Energy of underdamped harmonic oscillator]
For 
$$E(t)=\frac{\mu\omega_0^2A^2}{2}e^{-2\beta t}=E_0e^{-2\beta t}.$$ The rate at which the energy is dissipated is $$\left|\frac{dE}{dt}(t)\right|=2\beta E(t)\implies\frac{E}{\left|dE/dt\right|}=\frac{1}{2\beta}.$$
If $\beta\ll\omega_0$, then $$Q=2\pi\frac{E}{\Delta E},$$ where $\Delta E$ is the energy dissipated in a pseudo-period $\Tilde{T}=2\pi/\Tilde{\omega}\approx2\pi/\omega_0$.
\end{concept}
\begin{concept}[Critically damped harmonic oscillator: $\beta=\omega_0$]
Coefficients $c_1,c_2$ of the general solution \ref{b=w} are: $$c_1=x_0,\quad c_2=x_0\omega_0+\dot{x}_0.$$ This harmonic oscillator is the one that returns to balance more quickly.
\end{concept}
\begin{concept}[Overdamped harmonic oscillator: $\beta<\omega_0$]
Coefficients $c_1,c_2$ of the general solution \ref{b>w} are: $$c_1=\frac{x_0(\Tilde{\omega}-\beta)-\dot{x}_0}{2\Tilde{\omega}},\quad c_2=\frac{x_0(\Tilde{\omega}+\beta)+\dot{x}_0}{2\Tilde{\omega}}.$$
\end{concept}
\subsubsection{Driven harmonic oscillators}
\begin{concept}[Movement equation]
Consider the following differential equation: $$\ddot{x}+2\beta\dot{x}+\omega_0^2 x=f(t)=f_0\cos(\omega t+\psi),$$ with initials values of $x(0)=x_0$ and $\dot{x}(0)=\dot{x}_0$. Then the particular solution is:
$$x_p(t)=A\cos(\omega t+\psi-\phi),$$
where $\displaystyle A=\frac{f_0}{\sqrt{{(\omega_0^2-\omega^2)}^2+4\beta^2\omega^2}}$ and\\ $\displaystyle\phi=\arctan{\frac{2\beta\omega}{\omega_0^2-\omega^2}}$. Therefore for the general solution we have three cases to consider:
\begin{itemize}
    \item If $\beta<\omega_0$,
    \begin{equation}
        x(t)=e^{-\beta t}(c_1\cos\Tilde{\omega}t+c_2\sin\Tilde{\omega}t)+A\cos(\omega t+\psi-\phi).
        \label{d-b<w}
    \end{equation}
    \item If $\beta=\omega_0$, 
    \begin{equation}
        x(t)=e^{-\beta t}\left(c_1+c_2t\right)+A\cos(\omega t+\psi-\phi).
        \label{d-b=w}
    \end{equation}
    \item If $\beta>\omega_0$,
    \begin{equation}
        x(t)=c_1e^{-(\beta+\Tilde{\omega})t}+c_2e^{-(\beta-\Tilde{\omega})t}+A\cos(\omega t+\psi-\phi).
        \label{d-b>w}
    \end{equation}
\end{itemize}
Here $c_1,c_2$ are constants depending on the initial values.
\end{concept}
\begin{concept}[Underdamped driven oscillator]
Coefficients $c_1,c_2$ of the general solution \ref{d-b<w} are:
\begin{gather*}
    c_1=x_0-A\cos\left(\psi-\phi\right),\\
	c_2=\frac{\dot{x}_0-\omega A\sin\left(\psi-\phi\right)+\beta\left[x_0-A\cos\left(\psi-\phi\right)\right]}{\tilde{\omega}}.
\end{gather*}
\end{concept}
\begin{concept}[Critically damped driven oscillator]
Coefficients $c_1,c_2$ of the general solution \ref{d-b=w} are:
\begin{gather*}
    c_1=x_0-A\cos\left(\psi-\phi\right),\\
	c_2=\dot{x}_0A+\omega_0x_0+A\left(\omega\sin\left(\phi-\psi\right)-\omega_0\cos\left(\psi-\phi\right)\right).
\end{gather*}
\end{concept}
\begin{concept}[Overdamped driven oscillator]
Coefficients $c_1,c_2$ of the general solution \ref{d-b>w} are:
\begin{equation*}
    \begin{split}
        \begin{multlined}[t]
        c_1=A\frac{(\beta-\Tilde{\omega})\cos\left(\psi-\phi\right)-\omega\sin\left(\psi-\phi\right)}{2\Tilde{\omega}}+\\+\frac{-(\beta-\Tilde{\omega})x_0-\dot{x}_0}{2\Tilde{\omega}},
        \end{multlined}\\
        \begin{multlined}[t]
        c_2=A\frac{-(\beta+\Tilde{\omega})\cos\left(\psi-\phi\right)+\omega\sin\left(\psi-\phi\right)}{2\Tilde{\omega}}+\\+\frac{(\beta+\Tilde{\omega})x_0+\dot{x}_0}{2\Tilde{\omega}}.
        \end{multlined}
    \end{split}
\end{equation*}
\end{concept}
\begin{definition}
Given a driven oscillator, we say it is in the \textit{steady-state part} if $t\gg 1/\beta$. In that case $x(t)$ become: $$x(t)=A\cos(\omega t+\psi-\phi).$$ While the dependency on $c_1,c_2$ is non-negligible, we say the driven oscillator is in the \textit{transient part}.
\end{definition}
\begin{concept}[Resonance in amplitude]
If $\omega=\omega_r:=\sqrt{\omega_0^2-2\beta^2}$ we say the oscillator is in \textit{resonance in amplitude}. For $\omega=\omega_r$ we have $$A_r=\frac{f_0}{2\beta\sqrt{\omega_0^2-\beta^2}}.$$
\end{concept}
\begin{concept}[Energy in steady-state part]
$$E=\frac{\mu A^2}{2}\left[\omega^2\sin^2(\omega t+\psi-\phi)+\omega_0^2\cos^2(\omega t+\psi-\phi)\right].$$ If $\omega\approx\omega_0$ and $\beta\ll\omega_0$ then $$E=\frac{\mu f_0^2}{8}\frac{1}{(\omega-\omega_0)^2+\beta^2}.$$ Observe $E$ has a maximum at $\omega=\omega_0$ with the value of $E^\text{max}=\frac{\mu f_0^2}{8\beta^2}$.
\end{concept}
\begin{definition}
We define the \textit{cutoff frequencies} as this two frequencies: $$\omega_1=\omega_0-\beta,\quad\omega_2=\omega_0+\beta.$$ The value $\Delta\omega=\omega_2-\omega_1=2\beta$ is called the \textit{bandwidth}. Therefore, we can redefined the quality factor as: $$Q=\frac{\omega_0}{2\beta}=\frac{\omega_0}{\Delta\omega}=\frac{\nu_0}{\Delta\nu}$$
\end{definition}
\textcolor{green}{FALTA COSA.}
\begin{concept}[Impulsive forces]
Consider a driven oscillator of equation $\ddot{x}+2\beta\dot{x}+\omega_0^2x=f$, where $$f(t)=\left\{\begin{array}{cc}
    0 & \text{if } t<t'\\
    f_0 & \text{if }t'\leq t\leq t'+\Delta t \\
    0 & \text{if } t>t'+\Delta t
\end{array}\right.$$
\end{concept}
\textcolor{green}{FALTA COSA.}
\subsection{Central forces}
\subsubsection{Definition and properties}
\begin{definition}[Central force]
A \textit{central force} is a force of the form $$\boldsymbol{F}(\boldsymbol{r})=f(r)\boldsymbol{e}_r,$$ where $r=\|\boldsymbol{r}\|$ and $\boldsymbol{e}_r=\boldsymbol{r}/r$ is the unit radial vector.
\end{definition}
\begin{definition}
The origin $\boldsymbol{r}=0$ is called \textit{center of forces}.
\end{definition}
\begin{prop}
All central forces are conservative and $$f(r)=-U'(r),$$ where $U(r)$ is the potential energy of the central force.
\end{prop}
\subsubsection{Conservation of angular momentum and areal velocity}
\begin{prop}
The angular momentum with respect to the center of forces is conserved, that is, $\dot{\boldsymbol{L}}=0$.
\end{prop}
\begin{prop}[Kepler's 2nd law]
The areal velocity $dA/dt$ is constant. In fact, $$\frac{dA}{dt}=\frac{L}{2m}.$$
\end{prop}
\begin{concept}[Unit vectors]
Remember we have $$\boldsymbol{e}_r=\boldsymbol{i}\cos\theta+\boldsymbol{j}\sin\theta,\quad\boldsymbol{e}_\theta=-\boldsymbol{i}\sin\theta+\boldsymbol{j}\cos\theta.$$ Therefore we obtain, \begin{equation}
    \boldsymbol{r}=r\boldsymbol{e}_r,\quad\boldsymbol{\dot{r}}=\dot{r}\boldsymbol{e}_r+r\dot{\theta}\boldsymbol{e}_\theta,\quad\boldsymbol{\ddot{r}}=(\ddot{r}-r\dot{\theta}^2)\boldsymbol{e}_r+(2\dot{r}\dot{\theta}+r\ddot{\theta})\boldsymbol{e}_\theta.
    \label{unit}
\end{equation}
\end{concept}
\begin{concept}[Trajectory equation]
From \ref{unit}, Newton's second law can be written as: $$\ddot{r}-r\dot{\theta}^2=\frac{f(r)}{m},\quad 2\dot{r}\dot{\theta}+r\ddot{\theta}=0.$$ And we can obtain the following differential equations: $$\dot{\theta}=\frac{L}{m r^2}:=\frac{l}{r^2},\quad\ddot{r}-\frac{l^2}{r^3}=\frac{f(r)}{m},$$ where we have defined the magnitude $l:=L/m$. Finally, we get the \textit{trajectory equation}: $$\frac{d^2}{d\theta^2}\left(\frac{1}{r}\right)+\frac{1}{r}=-\frac{1}{ml^2}r^2f(r).$$
\end{concept}
\subsubsection{Conservation of energy}
\begin{concept}[Kinetic energy]
$$K=\frac{1}{2}m|\boldsymbol{\dot{r}}^2=\frac{1}{2}m\dot{r}^2+\frac{ml^2}{2r^2}.$$ 
\end{concept}
\begin{definition}
We define the effective potential as $$U_\text{eff}=U(r)+\frac{ml^2}{2r^2}.$$
\end{definition}
\begin{concept}[Energy]
$$E=\frac{1}{2}m\dot{r}^2+U_\text{eff}=\frac{1}{2}m\dot{r}^2+\frac{ml^2}{2r^2}+U(r).$$
\end{concept}
\end{multicols}
\end{document}