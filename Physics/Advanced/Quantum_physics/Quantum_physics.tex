\documentclass[../../../main_physics.tex]{subfiles}

\begin{document}
\renewcommand{\col}{\phy}
\begin{multicols}{2}[\section{Quantum physics}]
  \subsection{Mathematical formulation of quantum mechanics}
  \subsubsection{Bras and kets}
  \begin{definition}[Ket]
    We define a \emph{Hilbert space} $\mathcal{H}$ as a complete vector space over $\CC$ (whose elements are called \emph{kets} and they are reperesented by $\ket{\vf{\psi}}\in\mathcal{H}$ as a column vector\footnote{These will be the \emph{quantum states} of the system so usually this terms is also used to refer to them.}) together with an inner product $\langle\cdot,\cdot\rangle$ such that satisfy the following properties $\forall a,b\in\CC$ and $\forall \ket{\vf{\phi}_1},\ket{\vf{\phi}_2},\ket{\vf\psi},\ket{\vf\phi}\in \CC$:
    \begin{enumerate}
      \item Linearity: $\braket{\vf\psi}{a\ket{\vf{\phi}_1}+b\ket{\vf{\phi}_2}}=a\braket{\vf\psi}{\vf{\phi}_1}+b\braket{\vf\psi}{\vf{\phi}_2}$
      \item Positive defined: $\braket{\psi}{\psi}>0$ if $\ket{\psi}\ne 0$
      \item Hermitian: $\braket{\vf{\phi}}{\vf\psi}={\braket{\vf\psi}{\vf{\phi}}}^*$
    \end{enumerate}
    It is complete with the norm $\norm{\ket{\psi}}:=\sqrt{\braket{\psi}{\psi}}$. Along the document we will suppose that the dimension of the Hilbert space is $n$.
  \end{definition}
  \begin{definition}[Bra]
    Let $\ket{\vf{e}_i}$, $i=1,\cdots,n$ be a basis of the Hilbert space $\mathcal{H}$ of dimension $n$ and $\ket{\vf\psi}=\sum_{i=1}^n\alpha_i\ket{\vf{e}_i}\in\mathcal{H}$. We define the \emph{bra}, $\bra{\vf\psi}$, of $\ket{\vf\psi}$ as the adjoint, ${\ket{\vf\psi}}^\dag\in\mathcal{H}^*$, of $\ket{\vf\psi}$. In coordinates it is represented as a row vector:
    $$\bra{\vf\psi}=\sum_{i=1}^n{\alpha_i}^*\bra{\vf{e}_i}({\alpha_1}^*,\cdots,{\alpha_n}^*)$$ where $\bra{\vf{e}_i}$, $i=1,\cdots,n$, is the dual basis of the dual space $\mathcal{H}^*$.
  \end{definition}
  \subsubsection{Inner and outer products}
  \begin{definition}
    Let $\mathcal{H}$ be Hilbert space of dimension $n$ and $\ket{\vf\psi}=\sum_{i=1}^n\alpha_i\ket{\vf{e}_i},\ket{\vf{\phi}}=\sum_{i=1}^n\beta_i\ket{\vf{e}_i}\in\mathcal{H}$. We will write the inner product between them as the following \emph{bracket}: $$\braket{\vf\psi}{\vf{\phi}}=
      \begin{pmatrix}
        {\alpha_1}^* & \cdots & {\alpha_n}^*
      \end{pmatrix}
      \begin{pmatrix}
        \beta_1 \\
        \vdots  \\
        \beta_n
      \end{pmatrix}=\sum_{i=1}^n{\alpha_i}^*\beta_i$$
    We will say that $\ket{\vf\psi}$ is \emph{normalized} if $\braket{\vf\psi}=1$. We will say that $\ket{\vf\psi}$ and $\ket{\vf{\phi}}$ are \emph{orthogonal} if $\braket{\vf\psi}{\vf{\phi}}=0$.
  \end{definition}
  \begin{definition}
    Let $\mathcal{H}$ be Hilbert space of dimension $n$ and $\ket{\vf\psi}=\sum_{i=1}^n\alpha_i\ket{\vf{e}_i},\ket{\vf{\phi}}=\sum_{i=1}^n\beta_i\ket{\vf{e}_i}\in\mathcal{H}$. We will define the \emph{outer product} between them as:
    \begin{align*}
      \dyad{\vf\psi}{\vf{\phi}} & =
      \begin{pmatrix}
        \alpha_1 \\
        \vdots   \\
        \alpha_n
      \end{pmatrix}\begin{pmatrix}
                     {\beta_1}^* & \cdots & {\beta_n}^*
                   \end{pmatrix}= \\
                                & =
      \begin{pmatrix}
        \alpha_1{\beta_1}^* & \cdots & \alpha_1{\beta_n}^* \\
        \vdots              & \ddots & \vdots              \\
        \alpha_n{\beta_1}^* & \cdots & \alpha_n{\beta_n}^* \\
      \end{pmatrix}
    \end{align*}
  \end{definition}
  \subsubsection{Linear operators}
  \begin{definition}
    A \emph{linear operator} $\vf{A}$ is an operator that carries vectors to vectors in a way that $\forall a,b\in\CC$ and $\forall \ket{\vf\psi_1},\ket{\vf\psi_2}\in \CC$ we have: $$\vf{A}\left(a\ket{\vf\psi_1}+b\ket{\vf\psi_2}\right)=a\vf{A}\ket{\vf\psi_1}+b\vf{A}\ket{\vf\psi_2}$$
    Moreover is the standard basis $\ket{\vf{e}_i}$ is mapped to the basis $\ket{\vf{v}_j}=\sum_{i=1}^nv_{ij}\ket{\vf{e}_i}$, $j=1,\ldots,n$, we can write the operator $\vf{A}$ in the form: $$\vf{A}=\sum_{j=1}^n\dyad{\vf{v}_j}{\vf{e}_j}=\sum_{i,j=1}^nv_{ij}\dyad{\vf{e}_i}{\vf{e}_j}=:\sum_{i,j=1}^nA_{ij}\dyad{\vf{e}_i}{\vf{e}_j}$$ where we have defined $A_{ij}:=v_{ij}=\mel{\vf{e}_i}{\vf{A}}{\vf{e}_j}$. This way, the matrix $\vf{A}$ can also be written as $\vf{A}=(A_{ij})$.
  \end{definition}
  \begin{definition}
    Let $\vf{A}$ be a linear operator. We define its \emph{adjoint}, $\vf{A}^\dag$, as the operator defined as follows. If $\ket{\vf\psi}$, $\ket{\vf\phi}$ are two vectors such that $\ket{\vf\phi}=\vf{A}\ket{\vf\psi}$, then $\bra{\vf\phi}=\bra{\vf\psi}\vf{A}^\dag$. Moreover, if $\vf{A}=\sum_{i,j=1}^nA_{ij}\dyad{\vf{e}_i}{\vf{e}_j}$, then: $$\vf{A}^\dag=\sum_{i,j=1}^n{A_{ij}}^*\dyad{\vf{e}_j}{\vf{e}_i}$$ That is, $\vf{A}^\dag=(\transpose{({A_{ij}}^*)})$.
  \end{definition}
  \begin{proposition}
    The \emph{daga} $\dag$ operator satisfies the following properties $\forall a\in\CC$, for all vectors $\ket{\vf\psi}$, $\ker{\vf\phi}$ and for all linear operators $\vf{A}$, $\vf{B}$.
    \begin{enumerate}
      \item ${\ket{\vf\psi}}^\dag=\bra{\vf\psi}$
      \item ${(\dyad{\vf\phi}{\vf\psi})}^\dag=\dyad{\vf\psi}{\vf\phi}$
      \item ${(a\vf{A})}^\dag=a^*\vf{A}^\dag$
      \item ${(\vf{A}\ket{\vf\psi})}^\dag=\bra{\vf\psi}\vf{A}^\dag$
      \item ${(\vf{AB})}^\dag=\vf{B}^\dag\vf{A}^\dag$
    \end{enumerate}
  \end{proposition}
  \begin{definition}
    Let $\vf{A}=(A_{ij})$ be a linear operator. The \emph{trace} of $\vf{A}$ is: $$\trace\vf{A}=\sum_{i=1}^n\mel{\vf{e}_i}{\vf{A}}{\vf{e}_i}$$
  \end{definition}
  \begin{proposition}
    The trace of an operator has the following properties for all vectors $\ket{\vf\psi}$, $\ker{\vf\phi}$ and for all linear operators $\vf{A}$, $\vf{B}$ and $\vf{C}$.
    \begin{enumerate}
      \item $\trace(\vf{ABC})=\trace(\vf{BCA})=\trace(\vf{CAB})$
      \item $\trace(\vf{A}\ket{\vf\psi}\bra{\vf\phi})=\mel{\vf\phi}{\vf{A}}{\vf\psi}$
    \end{enumerate}
  \end{proposition}
  \subsubsection{Hermitian operators and diagonalization}
  \begin{definition}
    Let $\vf{A}$ be a linear operator. We say that $\vf{A}$ is \emph{Hermitian} (or \emph{self-adjoint}) if $\vf{A}=\vf{A}^\dag$.
  \end{definition}
  \begin{definition}
    Let $\vf{A}$ be a linear operator. We say that $\vf{A}$ is \emph{normal} if $\vf{A}\vf{A}^\dag=\vf{A}^\dag\vf{A}$.
  \end{definition}
  \begin{proposition}
    Let $\vf{A}$ be a linear normal operator. Then, $\vf{A}$ is diagonalizable.
  \end{proposition}
  \begin{corollary}
    Let $\vf{A}$ be a Hermitian operator. Then, $\vf{A}$ is diagonalizable.
  \end{corollary}
  \begin{proposition}
    Let $\vf{A}$ be a Hermitian operator. Then:
    \begin{itemize}
      \item $\lambda\in\sigma(\vf{A})\implies\lambda\in\RR$
      \item Eigenvectors of different eigenvalues are orthogonal.
    \end{itemize}
  \end{proposition}
  \begin{definition}
    Let $\vf{A}$ be a Hermitian operator. We say that two or more eigenvectors of $\vf{A}$ are \emph{degenerate} if they share the same eigenvalue.
  \end{definition}
  \begin{proposition}
    Let $\vf{A}$ be a Hermitian operator. We can express $\vf{A}$ in its \emph{spectral decomposition} as follows: $$\vf{A}=\sum_{i=1}^m\lambda_i\vf{P}_i$$
    where $\vf{P}_i$ is the orthogonal projection to the invariant subspaces generated by the eigenvectors of $\lambda_i$: $$\vf{P}_i=\sum_{r=1}^k\dyad{\vf\phi_{i,r}}{\vf\phi_{i,r}}$$
    where $\ket{\vf\phi_{i,r}}$ are the different eigenvectors of eigenvalue $\lambda_i$.
  \end{proposition}
  \begin{definition}[Commutator]
    Let $\vf{A}$, $\vf{B}$ be two linear operators. We define the \emph{commutator} between them as: $$\comm{\vf{A}}{\vf{B}}:=\vf{AB}-\vf{BA}$$
  \end{definition}
  \begin{definition}[Anticommutator]
    Let $\vf{A}$, $\vf{B}$ be two linear operators. We define the \emph{commutator} between them as: $$\acomm{\vf{A}}{\vf{B}}:=\vf{AB}+\vf{BA}$$
  \end{definition}
  \begin{definition}
    Two operators $\vf{A}$, $\vf{B}$ are \emph{compatible} if there exists a basis $(\ket{\vf{e}_1},\ldots,\ket{\vf{e}_n})$ such that $\vf{A}\ket{\vf{e}_i}=a_i\ket{\vf{e}_i}$ and $\vf{B}\ket{\vf{e}_i}=b_i\ket{\vf{e}_i}$ $\forall i$.
  \end{definition}
  \begin{theorem}
    Two Hermitian operators $\vf{A}$, $\vf{B}$ commute (i.e. $\comm{\vf{A}}{\vf{B}}=0$) if and only if they are compatible.
  \end{theorem}
  \begin{definition}
    Given a $f\in\mathcal{C}^\infty(\RR)$ and a linear operator $\vf{A}$, we define $f(\vf{A})$ as: $$f(\vf{A}):=\sum_{n=0}^\infty \frac{f^{(n)}(0)}{n!}\vf{A}^n$$
  \end{definition}
  \begin{proposition}
    Let $f\in\mathcal{C}^\infty(\RR)$ and $\vf{A}$ be a Hermitian operator. Suppose $\vf{A}=\sum_{i=1}^m\lambda_i\vf{P}_i$ is the spectral decomposition of $\vf{A}$. Then: $$f(\vf{A})=\sum_{i=1}^mf(\lambda_i)\vf{P}_i$$
  \end{proposition}
  \subsubsection{Unitary operators}
  \begin{definition}
    Let $\vf{A}$ be a linear operator. We say that $\vf{A}$ is \emph{unitary} if $\vf{A}\vf{A}^\dag=\vf{A}^\dag\vf{A}=\vf{I}_n$ (i.e. $\vf{A}^\dag=\vf{A}^{-1}$).
  \end{definition}
  \begin{proposition}
    Let $\vf{A}$ be an unitary operator. Then, $\vf{A}$ preserves the inner product.
  \end{proposition}
  \begin{proposition}
    The change of basis matrices are unitary.
  \end{proposition}
  \subsection{Postulates of quantum mechanics}
  \subsubsection{Postulates}
  \begin{definition}[Postulate I]
    The state of an isolated physical system is represented, at a fixed time $t$, by a normalized \emph{state vector} $\ket{\vf\psi}$ (i.e. $\braket{\vf\psi}{\vf\psi}=1$) belonging to a Hilbert space $\mathcal{H}$ called the \emph{state space}.
  \end{definition}
  \begin{definition}
    Let $\ket{\vf\psi}$, $\ket{\vf\phi}$ be two states of a system. A \emph{superposition} of them is a state $$\ket{\vf\Psi}=\alpha\ket{\vf\psi}+\beta\ket{\vf\phi}$$ where $\alpha,\beta\in\CC$.
  \end{definition}
  \begin{definition}
    Physically the kets $\ket{\vf\psi}$ and $\alpha\ket{\vf\psi}$, $\alpha\in\CC$, are exactly the same. We define a \emph{ray} of $\ket{\vf\psi}$ as the equivalence class of all multiples of $\ket{\vf\psi}$. Thus we will have a bijective correspondence between the set of unit rays and each physical state of the system.
  \end{definition}
  \begin{definition}[Postulate II]
    Every measurable physical quantity $\mathcal{A}$ is described by a Hermitian operator $\vf{A}$ acting in the state space $\mathcal{H}$. This operator is an \emph{observable}, meaning that its eigenvectors form a basis for $\mathcal{H}$.
  \end{definition}
  \begin{definition}[Postulate III (Non-degenerated)]
    The possible values that a physical quantity $\mathcal{A}$ can take (or the possible values of the measurement) are the eigenvalues $a_i$ of theobservable $\vf{A}$. After the measurement of the physical quantity $\mathcal{A}$, the state of the system will be the eigenvector of the associated eigenvalue of $\vf{A}$.
  \end{definition}
  \begin{definition}[Postulate III]
    If the measurement of the physical quantity $\mathcal{A}$ on the system in the state $\ket{\vf\psi}$ gives the result $a_i$, then the state of the system immediately after the measurement is the normalized projection of $\ket{\vf\psi}$ onto the eigenspaces associated with $a_i$:
    $$\ket{\vf\psi}\overset{a_i}{\implies}\frac{\vf{P}_i\ket{\vf\psi}}{\sqrt{\mel{\vf\psi}{\vf{P}_i}{\vf\psi}}}$$
  \end{definition}
  \begin{definition}[Postulate IV]
    When the physical quantity $\mathcal{A}$ is measured on a system in a normalized state $\ket{\vf\psi}$, the probability of obtaining an eigenvalue $a_i$ of the corresponding observable $\vf{A}=\sum_{i=1}^ma_i\vf{P}_i$ is given by:
    $$p_i=\trace(\vf{P}_i\dyad{\vf\psi}{\vf\psi})={\norm{\vf{P}_i\ket{\vf\psi}}}^2=\mel{\vf\psi}{\vf{P}_i}{\vf\psi}$$
    If $\vf{A}$ is non-degenerated, it simplifies to: $$p_i={\abs{\braket{a_i}{\vf\psi}}}^2$$
  \end{definition}
  \begin{definition}[Postulate V]
    The time evolution of the state vector $\ket{\vf\psi}$ is governed by the \emph{Schrödinger equation}, where $\vf{H}$ is the observable associated with the total energy of the system (called the \emph{Hamiltonian}): $$\ii \hbar\dv{}{t}\ket{\vf\psi}=\vf{H}\ket{\vf{\psi}}$$
  \end{definition}
  \begin{proposition}
    Let $\ket{\vf{E}_i(t)}$ be an eigenvector of the Hamiltonian such that $\vf{H}\ket{\vf{E}_i(t)}=E_i\ket{\vf{E}_i(t)}$. Then, the evolution of the state is given by: $$\ket{\vf{E}_i(t)}=\exp{-\ii\frac{E_i t}{\hbar}}\ket{\vf{E}_i(t_0)}$$ These kind of states are called \emph{stationary states} because they only have a global phase varying and so the physical properties remain the same.
  \end{proposition}
  \begin{proposition}
    The time evolution maps the orthonormal basis of eigenvectors $(\ket{\vf{E}_1},\ldots,\ket{\vf{E}_n})$ of $\vf{H}$ into another orthogonal basis $(\exp{-\ii\frac{E_it}{\hbar}}\ket{\vf{E}_1},\ldots,\exp{-\ii\frac{E_it}{\hbar}}\ket{\vf{E}_n})$. Hence, this is equivalent to say that the linear operator that describes the evolution is unitary: $$\vf{U}_t=\sum_{i=1}^n\exp{-\ii\frac{E_it}{\hbar}}\dyad{\vf{E}_i}=\exp{-\ii\frac{\vf{H}t}{\hbar}}$$
  \end{proposition}
  \subsubsection{Statistic analysis}
  \begin{proposition}
    The expected value of the observable $\vf{A}$ for the system in state represented by the unit vector $\ket{\vf\psi}\in\mathcal{H}$ is: $$\ev{\vf{A}}_{\vf\psi}=\ev{\vf{A}}{\vf\psi}$$
    Its variance is: $${{(\Delta\vf{A})}_{\vf\psi}}^2=\ev{\vf{A}^2}_{\vf\psi}-{\ev{\vf{A}}_{\vf\psi}}^2=\ev{\vf{A}^2}{\vf\psi}-\ev{\vf{A}}{\vf\psi}^2$$
  \end{proposition}
  \subsubsection{Indetermination relation}
  \begin{theorem}[Robertson inequality]
    For all state and for any pair of observables $(\vf{A},\vf{B})$ we have: $$\Delta\vf{A}\Delta\vf{B}\geq \frac{1}{2}\abs{\langle\comm{\vf{A}}{\vf{B}}\rangle}$$
  \end{theorem}
  \begin{corollary}[Heisenberg uncertainty principle]
    In the case of the position $\vf{X}$ and moment $\vf{P}$ we have $\comm{\vf{X}}{\vf{Y}}=\ii \hbar\vf{I}$ and so Robertson inequality becomes: $$\Delta\vf{X}\Delta\vf{P}\geq \frac{\hbar}{2}$$
  \end{corollary}
  \subsubsection{Evolution of expected values}
  \begin{proposition}[Ehrenfest theorem]
    The time evolution of the expected value in the state $\ket{\vf\psi}\in\mathcal{H}$ is given by:
    $$\dv{}{t}\ev{\vf{A}}_{\vf\psi}=\frac{1}{\ii\hbar}\ev{\comm{\vf{A}}{\vf{H}}}{\vf\psi(t)}=\frac{1}{\ii\hbar}\ev{\comm{\vf{A}}{\vf{H}}}_{\vf\psi}$$
  \end{proposition}
  \subsubsection{Compativility and commutation between observables}
  \begin{definition}
    A \emph{Complete set of commuting observables} is a set $\{\vf{A}_i:i\in I\}$ of observables such that $\comm{\vf{A}_i}{\vf{A}_j}=0$ $\forall i,j\in\ I$ and the basis in common in unique (i.e. they share always the base basis of eigenvectors).
  \end{definition}
  \subsubsection{Stern-Gerlach experiment, Larmor precession and the qubit}
  \begin{proposition}
    The \emph{Stern-Gerlach} experiment let us measure the component of the magnetic moment that points towards the direction of the magnetic field. The magnetic moment in each component $i$ is proportional to the spin $S_i$ with the following relation:
    $$\vf\mu_i=-g\frac{q}{2 m} \vf{S}_i=\gamma\vf{S}_i\qquad i\in\{x,y,z\}$$
    where $\gamma$ is the \emph{gyromagnetic ratio}. So measure the observable $\vf\mu_i$ is equivalent to measure the observable $\vf{S}_i=\pm\frac{\hbar}{2}$. The postulates give the following matrix for each $\vf{S}_i$: $$\vf{S}_x=\frac{\hbar}{2}
      \begin{pmatrix}
        0 & 1 \\
        1 & 0
      \end{pmatrix}\quad\vf{S}_y=\frac{\hbar}{2}
      \begin{pmatrix}
        0   & -\ii \\
        \ii & 0
      \end{pmatrix}\quad\vf{S}_z=\frac{\hbar}{2}
      \begin{pmatrix}
        1 & 0  \\
        0 & -1
      \end{pmatrix}$$
    Moreover we have that $$\comm{\hat{S}_i}{\hat{S}_j}=\ii\hbar\varepsilon_{ijk}\vf{S}_k$$ where $\varepsilon_{ijk}$ is the \emph{Levi-Civita symbol}: $$\varepsilon_{ijk}=
      \begin{cases}
        +1 & \text{if } (i,j,k) =(1,2,3), (2,3,1), (3,1,2), \\
        -1 & \text{if } (i,j,k) =(3,2,1), (1,3,2), (2,1,3), \\
        0  & \text{if } i = j, j = k, k = i
      \end{cases}$$
  \end{proposition}
  \begin{definition}
    We define the \emph{Pauli matrices} as: $$\vf{\sigma}_x=
      \begin{pmatrix}
        0 & 1 \\
        1 & 0
      \end{pmatrix}\quad\vf{\sigma}_y=
      \begin{pmatrix}
        0   & -\ii \\
        \ii & 0
      \end{pmatrix}\quad\vf{\sigma}_z=
      \begin{pmatrix}
        1 & 0  \\
        0 & -1
      \end{pmatrix}$$
    Hence, $\vf{S}_i=\frac{\hbar}{2}\vf\sigma_i$.
  \end{definition}
  \begin{definition}
    Given $\vf{n}\in\RR^3$ with $\norm{n}=1$ and $\vf{n}=(\sin\theta\cos\varphi,\sin\theta\sin\varphi,\cos\theta)$ in spherical coordinates. We define the operator $\vf{S}_{\vf{n}}$ as:
    $$\vf{S}_{\vf{n}}=\vf{n}\cdot\vf\sigma=\frac{\hbar}{2}\begin{pmatrix}
        \cos\theta                 & \sin\theta\exp{-\ii\varphi} \\
        \sin\theta\exp{\ii\varphi} & -\cos\theta
      \end{pmatrix}$$
    where $\vf\sigma:=\transpose{\begin{pmatrix}
          \vf{\sigma}_x & \vf{\sigma}_y & \vf{\sigma}_z
        \end{pmatrix}}$.
    Thus, given a state $\ker{\vf{\psi}}\in\mathcal{H}$ we can find a representant of the ray it creates centered at the unit sphere. This sphere is called \emph{Bloch sphere}.
  \end{definition}
  \begin{definition}
    A \emph{qubit} is a quantum system of dimension 2.
  \end{definition}
  \begin{definition}
    Suppose we apply a magnetic field $\vf{B}$ in the $z$-axis. The Hamiltonian of the system will be $$\vf{H}=-\vf\mu\cdot\vf{B}=-\vf\mu_z B=g\frac{e}{2m} \vf{S}_z B = \frac{e}{m}\vf{S}_z B$$ beacause $g\approx 2$ for the electron.
    We define the \emph{Bohr magneton} as: $$\mu_{\mathrm{B}}:=\frac{e\hbar}{2 m}$$
    And so we have that: $$\vf{H}=\mu_{\mathrm{B}} B\vf\sigma_z=\mu_{\mathrm{B}} B\begin{pmatrix}
        1 & 0  \\
        0 & -1
      \end{pmatrix}=\frac{1}{2}\hbar \omega\vf\sigma_z$$
    where $\omega=\frac{2\mu_{\mathrm{B}} B}{\hbar}$. In this case the unitary operator that describes the evolution is: $$\vf{U}_t=\exp{-\ii\frac{\vf{H}t}{\hbar}}=\exp{-\ii\frac{\omega t}{2}\vf\sigma_z}$$
  \end{definition}
  \begin{proposition}
    Pauli matrices $\vf\sigma_x$, $\vf\sigma_y$ and $\vf\sigma_z$ satisfy the following properties $\forall i,j,k\in\{x,y,z\}$:
    \begin{enumerate}
      \item $\trace \vf\sigma_i=0$
      \item $\comm{\vf\sigma_i}{\vf\sigma_j}=2\ii \varepsilon_{ijk}\vf\sigma_k$
      \item ${\vf\sigma_x}^2={\vf\sigma_y}^2={\vf\sigma_z}^2=\vf{I}_2$
      \item $\acomm{\vf\sigma_i}{\vf\sigma_j}=2\delta_{jk}\vf{I}_2$
      \item $\vf\sigma_i\vf\sigma_j=\ii \varepsilon_{ijk}\vf\sigma_k+\delta_{jk}\vf{I}_2$
    \end{enumerate}
  \end{proposition}
  \begin{theorem}
    The set $(\vf{I}_2,\vf\sigma_x,\vf\sigma_y,\vf\sigma_z)$ is a basis on the vector space of real Hermitian matrices of dimension 2. Hence,any Hermitian operator $\vf{A}$ can be written as $\vf{A}=\frac{1}{2}\sum_{i=0}^3d_i\vf\sigma_i$, where we have denoted $\vf\sigma_0=:\vf{I}_2$.
  \end{theorem}
  \begin{lemma}
    Let $\vf{A}=\frac{1}{2}\sum_{i=0}^3d_i\vf\sigma_i$ be a Hermitian operator. Then: $$\trace(\vf\sigma_j\vf{A})=d_j\in\RR$$
  \end{lemma}
  \begin{proposition}
    Let $\vf{A}=\frac{1}{2}\sum_{i=0}^3d_i\vf\sigma_i$ be a Hermitian operator and denote $\vf{d}:=\transpose{(d_1,d_2,d_3)}=:\norm{d}\vf{n}$. Then we can write $\vf{A}$ as:
    \begin{align*}
      \vf{A}=\frac{1}{2}\left(d_0+\norm{d}\vf{n}\cdot\vf\sigma\right)=: & \frac{1}{2}\left(d_0\vf{I}_2+\norm{d}\vf{n}\cdot\vf\sigma\right) \\=:&\frac{1}{2}\left(d_0\vf{I}_2+\norm{d}\vf\sigma_{\vf{n}}\right)
    \end{align*}
    Hence, its eigenvalues will be $\frac{1}{2}(d_0\pm\norm{d})$.
  \end{proposition}
  \begin{theorem}
    The equation of movement of a state $\ket{\vf{m}}$ in a hamiltonian $\vf{H}=\frac{\hbar}{2}\left(d_0\vf{I}_2+\norm{d}\vf\sigma_{\vf{n}}\right)$ is:
    $$\dv{\vf{m}}{t}=-\norm{\vf{d}}({\vf{m}}\crossprod{\vf{n}})$$
  \end{theorem}
  \subsection{Wave mechanics}
\end{multicols}
\end{document}